\section{A Review of Ten Stateful Unlearning Methods: Methodologies and Key Insights}

Unlearning problem is generally specified with respect to a subset of data instances that must be erased, which we denote as the \textit{forget set} ($\Df$). 
To maintain the usefulness of the model, one often introduces an additional collection of examples that should remain unaffected, referred to as the \textit{retain set} ($\Dr$). 
Together, these two sets define the dual objectives of unlearning: enforcing removal of knowledge from $\Df$ while preserving performance on $\Dr$. 
This trade-off can naturally be expressed as a regularized optimization framework \citep{liu2024rethinking,yao2023large,zhang2024negative}:
\begin{align}
\hspace*{-3mm}
\begin{array}{l}
 \displaystyle \minimize_{\boldsymbol{\theta}}   
 \,\, \mathbb E_{(x, y) \in \Df} [ \ellf(y | x; \boldsymbol{\theta}) ]
 + \lambda \, \mathbb E_{(x, y) \in \Dr} [ \ellr(y | x; \boldsymbol{\theta}) ] ,
\end{array}
\hspace*{-3mm}
\label{eq:prob_LLM_MU}
\end{align}
Here, $\btheta$ denotes the parameters to be optimized during unlearning, $\lambda \geq 0$ acts as a regularization weight controlling the penalty for divergence caused by forgetting, and $\ellf$ and $\ellr$ correspond to the loss terms associated with forgetting and retaining, respectively, when generating $y$ conditioned on $x$.

A considerable body of work has focused on designing and analyzing suitable forget and retain loss functions to address problem \eqref{eq:prob_LLM_MU} \citep{yao2023large, zhang2024negative, fan2024simplicity, fan2025towards, wang2025invariance, li2024wmdp, zou2024improving, gandikota2024erasing, sheshadri2024latent, tamirisa2024tamper, yuan2024closer, singh2025unlearning}. 
According to the underlying principles of LLM unlearning, existing approaches can be broadly categorized into divergence-driven, representation-mismatch, and rejection-based methods.

\paragraph{Divergence-driven methods.} 
In this class of methods, the forget loss $\ellf$ is designed to progressively widen the gap between the prediction logits of the unlearned model and those of the original model on the forget set $\Df$. 
For instance, gradient difference (\textbf{GradDiff}) \citep{yao2023large} achieves unlearning by applying gradient ascent on $\Df$. 
Negative Preference Optimization (\textbf{NPO}) \citep{zhang2024negative} instead interprets forget samples as negative preferences in Direct Preference Optimization (DPO) \citep{rafailov2024direct}, thereby converting the unbounded gradient-ascent loss into a \textit{bounded loss}, as shown below:
\begin{align}
  & \hspace*{-2mm} \ell_{\mathrm{NPO}}(\boldsymbol{\theta}) =  \mathbb E_{(x,y) \in \Df} {\left [    - \frac{2}{\beta} \log \sigma  \left ( - \beta \log \left ( \frac{\pi_{\btheta} (y | x) }{\pi_{\mathrm{ref}} (y | x)}\right ) \right ) 
    \right ]
}.  \hspace*{-3mm} \label{eq:NPO_loss}
\end{align}

\textbf{SimNPO} \citep{fan2024simplicity} was the first to highlight the issue of reference-model bias in NPO and modified $\ellf$ to 
$- \frac{2}{\beta} \log \sigma \left ( - \frac{\beta}{|y|}\log  \pi_{\btheta} (y | x) \right )$. 
Building upon this, \citet{fan2025towards} and \citet{wang2025invariance} further improved the robustness of NPO by incorporating Sharpness-Aware Minimization (SAM) \citep{foret2021sharpnessaware} and Invariant Risk Minimization (IRM) \citep{arjovsky2019invariant}.

\paragraph{Representation-mismatch methods.} 
This family of approaches enforces forgetting by directly altering the internal representations of the unlearned model on $\Df$. 
A representative example is Representation Mismatch Unlearning (\textbf{RMU}), where the hidden states of the unlearned model $M_{\boldsymbol{\theta}_\text{u}}$ are mapped to a uniformly random vector $\mathbf{u}$, as defined in \eqref{eq:RMU_loss}:

\begin{align}
  & \hspace*{-2mm} 
  \ell_{\mathrm{RMU}}(\boldsymbol{\theta}) = 
  \mathbb E_{(x,y) \in \Df} \left[
    \frac{1}{\ell_{\mathrm{RMU}}}
    \sum_{\text{token } t \in x} 
    || {M_{\boldsymbol{\theta}_\text{u}}(t) - c \cdot \mathbf{u}||}_2^2 
  \right].
  \label{eq:RMU_loss}
\end{align}

Representation Rerouting (\textbf{RR}) \citep{zou2024improving} further reduces the similarity between the unlearned model $M_{\boldsymbol{\theta}_\text{u}}$ and the original model $M_{\boldsymbol{\theta}_\text{o}}$. 
Other works, such as \citet{sheshadri2024latent} and \citet{tamirisa2024tamper}, combine RMU with adversarial training or meta-learning strategies to improve robustness.

\paragraph{Rejection-based methods.}
Rejection-based techniques aim to minimize the probability that the unlearned model produces outputs for inputs in $\Df$, encouraging the model to respond with a rejection instead. 
For instance, the I Don't Know (\textbf{IDK}) strategy \citep{maini2024tofu} implements this objective using the following loss:
\begin{align}
  & \hspace*{-2mm} \ell_{\mathrm{IDK}}(\boldsymbol{\theta}) =  \mathbb E_{(x \in \Df, y \in \mathcal{D}_\text{IDK})} \left [ - \log \pi_{\btheta}(y | x) \right ] .
  \hspace*{-3mm} \label{eq:IDK_loss}
\end{align}

% \begin{wraptable}{r}{0.6\textwidth} % {r} 代表靠右, 第二个参数是
 \begin{table}[htb]
% \vspace{-8mm}
\centering
\caption{Taxonomy of 12 unlearning methods grouped into {\MDiv}, {\MRep}, and {\MRej} families. A checkmark (\cmark) marks incorporation of \textit{robust unlearning}. Methods are evaluated on benchmarks WMDP (W), MUSE (M), TOFU (T), WHP (H), PKU-SafeRLHF (P) and Circuit Breaker (C), using unlearning effectiveness (UE), utility retention (UT), and robustness (Rob). UE and UT are measured by multiple-choice (MCQ) or open question-answering (Open-QA), while Rob is tested at model-level or input-level. Benchmark abbreviations denote both methods and evaluation settings.
}
\label{tab:taxonomy}
\renewcommand{\arraystretch}{1.25}
\vspace{4mm}
\resizebox{0.7\linewidth}{!}{%
% \resizebox{1.0\linewidth}{!}{%
\begin{tabular}{c|c|c|cc|cc|cc}
\toprule[1pt]
\toprule
\multirow{2}{*}{\textbf{Method}} & \multirow{2}{*}{\textbf{Reference}} & \multirow{2}{*}{\textbf{\begin{tabular}[c]{@{}c@{}}Robust\\Design\end{tabular}}} & \multicolumn{2}{c|}{\textbf{UE}} & \multicolumn{2}{c|}{\textbf{UT}} & \multicolumn{2}{c}{\textbf{Rob}} \\
\cline{4-9}
& & & \textbf{MCQ} & \textbf{Open-QA} & \textbf{MCQ} & \textbf{Open-QA} & \textbf{Model} & \textbf{Input} \\
\midrule
\rowcolor{LightCyan!50}
% \multicolumn{9}{c}{\textbf{\textit{Divergence-driven optimization}}} \\
% \multicolumn{9}{c}{\textbf{\MDiv}} \\
\multicolumn{9}{c}{\textbf{\textcolor{CDiv}{\textit{Divergence-driven optimization}}}} \\
\midrule
\rowcolor{gray!20}
\textbf{GradDiff} & \citep{yao2023large} & \xmark & & P & & P & & \\
\textbf{NPO} & \citep{zhang2024negative} & \xmark & & T & & T & & \\
\rowcolor{gray!20}
\textbf{SimNPO} & \citep{fan2024simplicity} & \xmark & W & T/M & W & T/M & T & \\
\textbf{NPO+SAM} & \citep{fan2025towards} & \cmark & W & M & W & M & W/M & W \\
\rowcolor{gray!20}
\textbf{NPO+IRM} & \citep{wang2025invariance} & \cmark & W & M & W & M & W/M & \\
\midrule
\rowcolor{LightCyan!50}
% \multicolumn{9}{c}{\textbf{\textit{Representation misalignment}}} \\
% \multicolumn{9}{c}{\textbf{\MRep}} \\
\multicolumn{9}{c}{\textbf{\textcolor{CRep}{\textit{Representation misalignment}}}} \\
\midrule
\textbf{RMU} & \citep{li2024wmdp} & \xmark & W & & W & W & & W \\
\rowcolor{gray!20}
\textbf{RR} & \citep{zou2024improving} & \xmark & C & & C & C & & C \\
\textbf{RMU+LAT} & \citep{sheshadri2024latent} & \cmark & W/H & & W/H & W & W & H \\
\rowcolor{gray!20}
\textbf{TAR} & \citep{tamirisa2024tamper} & \cmark & W & & W & & W & \\
\midrule
\rowcolor{LightCyan!50}
% \multicolumn{9}{c}{\textbf{\textit{Rejection-based targeted unlearning}}} \\
% \multicolumn{9}{c}{\textbf{\MRej}} \\
\multicolumn{9}{c}{\textbf{\textcolor{CRej}{\textit{Rejection-based targeted unlearning}}}} \\
\midrule
\textbf{ELM} & \citep{gandikota2024erasing} & \xmark & W/H & & W/H & W/H & & W \\
\rowcolor{gray!20}
\textbf{DPO} & \citep{zhang2024negative} & \xmark & & T & T & T & & \\
\textbf{IDK+AP} & \citep{yuan2024closer} & \xmark & & T & T & T & & \\

% \textbf{JensUn} & \citep{singh2025unlearning} & \xmark & L/R & L/R & L/R & R & L & \\

\bottomrule
\bottomrule[1pt]
\end{tabular}
% \vspace{-6mm}
}
% \end{wraptable}
\end{table}



% \begin{table}[h!]
% \centering
% \caption{Overview of 12 unlearning methods. The methods are categorized by principle: divergence-driven, representation-mismatch, and rejection-based. A checkmark (\cmark) indicates robust design, while a cross (\xmark) indicates otherwise. Evaluation metrics reported in the original papers include unlearning effectiveness (UE), utility (UT), and robustness evaluation (RE). Task formats are multiple-choice questions (MCQ) or Open-QA questions (Gen), and attack types are denoted as model/input. Letters denote the unlearning benchmarks used: WMDP (W), MUSE (M), TOFU (T), WHP (H), PKU-SafeRLHF (P), LKF (L), RWKU (R), and Circuit Breaker (C).}
% \renewcommand{\arraystretch}{1.25}
% \vspace{2mm}
% \resizebox{0.5\textwidth}{!}{%
% \begin{tabular}{c|c|c|cc|cc|cc}
% \toprule[1pt]
% \toprule
% \multirow{2}{*}{\textbf{Method}} & \multirow{2}{*}{\textbf{Reference}} & \multirow{2}{*}{\textbf{\begin{tabular}[c]{@{}c@{}}Robust\\Design\end{tabular}}} & \multicolumn{2}{c|}{\textbf{UE}} & \multicolumn{2}{c|}{\textbf{UT}} & \multicolumn{2}{c}{\textbf{RE}} \\
% \cline{4-9}
% & & & \textbf{MCQ} & \textbf{Gen} & \textbf{MCQ} & \textbf{Gen} & \textbf{Model} & \textbf{Input} \\
% \midrule
% \rowcolor{LightCyan!50}
% \multicolumn{9}{c}{\textbf{\textit{Divergence-driven}}} \\
% \midrule
% \rowcolor{gray!20}
% \textbf{GradDiff} & \citet{yao2023large} & \xmark & & P & & P & & \\
% \textbf{NPO} & \citet{zhang2024negative} & \xmark & & T & & T & & \\
% \rowcolor{gray!20}
% \textbf{SimNPO} & \citet{fan2024simplicity} & \xmark & W & T/M & W & T/M & T & \\
% \textbf{NPO+SAM} & \citet{fan2025towards} & \cmark & W & M & W & M & W/M & W \\
% \rowcolor{gray!20}
% \textbf{NPO+IRM} & \citet{wang2025invariance} & \cmark & W & M & W & M & W/M & \\
% \midrule
% \rowcolor{LightCyan!50}
% \multicolumn{9}{c}{\textbf{\textit{Representation-mismatch}}} \\
% \midrule
% \textbf{RMU} & \citet{li2024wmdp} & \xmark & W & & W & & & W \\
% \rowcolor{gray!20}
% \textbf{RR} & \citet{zou2024improving} & \xmark & C & & C & C & & C \\
% \textbf{RMU+LAT} & \citet{sheshadri2024latent} & \cmark & W/H & & W/H & W & W & H \\
% \rowcolor{gray!20}
% \textbf{TAR} & \citet{tamirisa2024tamper} & \cmark & W & & W & & W & \\
% \midrule
% \rowcolor{LightCyan!50}
% \multicolumn{9}{c}{\textbf{\textit{Rejection-based}}} \\
% \midrule
% \textbf{IDK+AP} & \citet{yuan2024closer} & \xmark & & T & T & T & & \\
% \rowcolor{gray!20}
% \textbf{JensUn} & \citet{singh2025unlearning} & \xmark & L/R & L/R & L/R & R & L & \\
% \textbf{ELM} & \citet{gandikota2024erasing} & \xmark & W/H & & W/H & W/H & & W \\
% \bottomrule
% \bottomrule[1pt]
% \end{tabular}
% }
% \label{tab:unlearn_method}
% \end{table}

Extensions such as \textbf{IDK+AP }\citep{yuan2024closer} replace $\ellr$ with a preference-based loss on $\Dr$ similar to NPO, while \citet{singh2025unlearning} propose minimizing the Jensen–Shannon divergence between the output distribution on $\Df$ and the rejection distribution. 
Erasing via Language Modeling (\textbf{ELM}) \citep{gandikota2024erasing} minimizes the discrepancy between $\pi_{\btheta_\text{u}}(y|x)$ and $\pi_{\btheta_\text{0}}(y|c, x)$, where $c$ is a predefined prefix such as \textit{``As a novice in bioweapons''}.

In \textbf{Table\,\ref{tab:unlearn_method}}, we provide a summary of twelve representative unlearning methods and indicate whether they incorporate a robust design. 
The evaluation is organized along three key dimensions: unlearning effectiveness (UE), utility (UT), and robustness evaluation (RE). 
For UE and UT, the reported metrics are further divided according to the evaluation protocol into multiple-choice question (MCQ) and open question-answer (Open-QA) settings. 
For RE, we distinguish between model-level attacks (e.g., in-domain relearning \citep{hu2024jogging,fan2025towards} and out-of-domain fine-tuning \citep{wang2025invariance}) and input-level attacks (e.g., jailbreaking \citep{lucki2024adversarial}). 
The letters denote the benchmarks used in the corresponding studies, such as W for WMDP \citep{li2024wmdp}. 
As shown in the table, WMDP emerges as one of the most frequently adopted benchmarks, and therefore our subsequent experiments will primarily focus on WMDP.



% \begin{table}[h!]
% \centering
% \caption{}
% \renewcommand{\arraystretch}{1.25}
% \vspace{2mm}
% \resizebox{0.55\textwidth}{!}{% % Increased resizebox width slightly to accommodate the new column
% \begin{tabular}{c|c|c|cc|cc|cc|c}
% \toprule[1pt]
% \toprule
% \multirow{2}{*}{\textbf{Method}} & \multirow{2}{*}{\textbf{Reference}} & \multirow{2}{*}{\textbf{\begin{tabular}[c]{@{}c@{}}Robust\\Design\end{tabular}}} & \multicolumn{2}{c|}{\textbf{UE}} & \multicolumn{2}{c|}{\textbf{UT}} & \multicolumn{2}{c|}{\textbf{RE}} & \multirow{2}{*}{\textbf{Benchmark}} \\
% \cline{4-9}
% & & & \textbf{MCQ} & \textbf{Gen} & \textbf{MCQ} & \textbf{Gen} & \textbf{Model} & \textbf{Input} & \\
% \midrule
% \rowcolor{LightCyan!50}
% \multicolumn{10}{c}{\textbf{\textit{Divergence-driven}}} \\
% \midrule
% \rowcolor{gray!20}
% \textbf{GradDiff} & \citet{yao2023large} & \xmark & & P & & P & & & PKU-SafeRLHF \\
% \textbf{NPO} & \citet{zhang2024negative} & \xmark & & T & & T & & & TOFU \\
% \rowcolor{gray!20}
% \textbf{SimNPO} & \citet{fan2024simplicity} & \xmark & W & T/M & W & T/M & T & & WMDP/TOFU/MUSE \\
% \textbf{NPO+SAM} & \citet{fan2025towards} & \cmark & W & M & W & M & W/M & W & WMDP/MUSE \\
% \rowcolor{gray!20}
% \textbf{NPO+IRM} & \citet{wang2025invariance} & \cmark & W & M & W & M & W/M & & WMDP/MUSE \\
% \midrule
% \rowcolor{LightCyan!50}
% \multicolumn{10}{c}{\textbf{\textit{Representation-mismatch}}} \\
% \midrule
% \textbf{RMU} & \citet{li2024wmdp} & \xmark & W & & W & & & W & WMDP \\
% \rowcolor{gray!20}
% \textbf{RR} & \citet{zou2024improving} & \xmark & C & & C & C & & C & Circuit Breaker \\
% \textbf{ELM} & \citet{gandikota2024erasing} & \xmark & W/H & & W/H & W/H & & W & WMDP/WHP \\
% \rowcolor{gray!20}
% \textbf{RMU+LAT} & \citet{sheshadri2024latent} & \cmark & W/H & & W/H & W & W & H & WMDP/WHP \\
% \textbf{TAR} & \citet{tamirisa2024tamper} & \cmark & W & & W & & W & & WMDP \\
% \midrule
% \rowcolor{LightCyan!50}
% \multicolumn{10}{c}{\textbf{\textit{Rejection-based}}} \\
% \midrule
% \rowcolor{gray!20}
% \textbf{IDK+AP} & \citet{yuan2024closer} & \xmark & & T & T & T & & & TOFU \\
% \textbf{JensUn} & \citet{singh2025unlearning} & \xmark & L/R & L/R & L/R & R & L & & LKF/RWKU \\
% \bottomrule
% \bottomrule[1pt]
% \end{tabular}
% }
% \end{table}