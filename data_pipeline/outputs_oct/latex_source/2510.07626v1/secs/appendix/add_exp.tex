\section{Additional Experiment Results}
\label{appx:add_exp}

% \begin{table}[h!]
%   \centering
%   \caption{Comparison of MCQ (WMDP) and Open-QA (GSM8K) tasks with ground-truth answers and model predictions. \SL{[talk to me.]}}
%   \label{tab:selection_generation}
%   \renewcommand{\arraystretch}{1.25} % 行距稍微增加
%   \resizebox{0.95\linewidth}{!}{%
%     \small
%     \begin{tabular}{>{\centering\arraybackslash}m{0.15\linewidth} 
%                     m{0.35\linewidth} 
%                     >{\centering\arraybackslash}m{0.1\linewidth} 
%                     m{0.35\linewidth}}
%       \toprule[1pt]
%       \toprule
%        & \textbf{Question} & \textbf{Label} & \textbf{Model Output} \\
%       \midrule
%       \centering \textbf{MCQ (WMDP)} &
%       In RNA virus genome synthesis, what potential risk is associated with obtaining virions? A... B.Re-creating dangerous pathogens; C...  
%       D.Accelerated biological research &
%       \textbf{B} &
%       \textbf{\textcolor{red}{Original Model}:} B \newline
%       \textbf{\textcolor{blue}{NPO-unlearned Model}:} D \\
%       \midrule
%       \centering \textbf{Open-QA (GSM8K)} &
%       James decides to run 3 sprints 3 times a week. He runs 60 meters each sprint.  
%       How many meters does he run a week? &
%       \textbf{540} &
%       \textbf{\textcolor{red}{Original Model}:} James runs 3 sprints 3 times a week. So he runs $3 \times 3 = 9$ sprints in total.  
%       Each sprint is 60 meters, yielding $9 \times 60 = 540$ meters. \newline
%       \textbf{\textcolor{blue}{NPO-unlearned Model}:} @nate@nate 3 sprints @-@ 3 times a week = 9 sprints a week the the the @@@@@@ \\
%       \bottomrule
%       \bottomrule[1pt]
%     \end{tabular}
%   }
% \end{table}


\begin{table}[h!]
  \centering
  \caption{Example of answers from the \textcolor{original}{original} (pre-unlearned) model (Llama-3 8B Instruct) and the \textcolor{CDiv}{NPO} / \textcolor{CRep}{RMU} unlearned model on  WMDP, evaluated on both MCQ and Open-QA tasks.}
  \vspace{2mm}
  \label{tab:selection_generation}
  \renewcommand{\arraystretch}{1.2} % 行距稍微增加
  \resizebox{1.0\linewidth}{!}{%
    \small
    \begin{tabular}{>{\arraybackslash}m{0.40\linewidth} 
                    >{\centering\arraybackslash}m{0.06\linewidth} 
                    m{0.13\linewidth} 
                    m{0.45\linewidth}}
      \toprule[1pt]
      \toprule
      \centering\textbf{Question} & \textbf{Label} & ~~~\textbf{MCQ} & ~~~~~~~~~~~~~~~~~~~~~~~~~~~\textbf{Open-QA} \\
      \midrule
      % \begin{tabular}{l}
      %       In RNA virus genome synthesis, what potential risk is associated with obtaining virions? \\ 
      % A... B.Re-creating dangerous pathogens; C... D.Accelerated biological research 
      % \end{tabular}
      In RNA virus genome synthesis, what potential risk is associated with obtaining virions? \newline
      A.Limiting access to genome information \newline
      B.Re-creating dangerous pathogens \newline C.Decentralized access to synthetic RNA \newline
      D.Accelerated biological research. Answer:
      & B & 
      \textbf{\textcolor{original}{Original}:} B \newline \newline
      \textbf{\textcolor{CDiv}{NPO}:} D \newline \newline
      \textbf{\textcolor{CRep}{RMU}:} D &
      \textbf{\textcolor{original}{Original}:} B. Re-creating dangerous pathogens \newline \newline
      \textbf{\textcolor{CDiv}{NPO}:} @nate@nate@nate@nate@nate@nate... \newline \newline
      \textbf{\textcolor{CRep}{RMU}:} it is not the way it is added by someone...
      \\
      \bottomrule
      \bottomrule[1pt]
    \end{tabular}
  }
\end{table}




\paragraph{Answer selection vs. answer generation.} 
\textbf{Table\,\ref{tab:selection_generation}} highlights the contrast between answer selection and answer generation for an NPO-unlearned Llama-3 8B Instruct on WMDP. Under the MCQ setting, both NPO and RMU successfully alter the model’s prediction to an incorrect choice, suggesting effective unlearning of forget-relevant knowledge. However, in the Open-QA setting, the same models produce incoherent or nonsensical text rather than valid answers, indicating that their generative ability on forget queries is internally disrupted. This mismatch reveals a critical limitation of relying solely on MCQ-based evaluations, as they can obscure issues of over-forgetting that may also degrade performance on non-forget inputs.


\begin{figure}[htbp]
    \centering
    \hspace{8mm}
    \includegraphics[width=0.25\linewidth]{figs/bar_logits_legend.pdf} \\
    \includegraphics[width=0.4\linewidth]{figs/bar_logits_freq.pdf}
    \caption{ABCD and top-4 token logits of the original (Llama-3 8B Instruct), NPO unlearned and RMU unlearned model on the WMDP evaluation set.}
    \label{fig:bar_logits}
    %\vspace{-3mm}
\end{figure}


\paragraph{From logits to behavior: over-forgetting in divergence-driven unlearning.}
As indicated by Fig.\,\ref{fig:evaluatio_wmdp}-(c), {\MDivB} is prone to over-forgetting on Open-QA tasks. To further investigate this limitation, we compare the prediction logit distributions of the unlearned models (NPO and RMU) with the original pre-unlearned model over the answer options (A/B/C/D) and their top-4 predictions.  
\textbf{Fig.\,\ref{fig:bar_logits}} illustrates how prediction logit distributions differ across unlearning methods on WMDP, comparing the options with each model’s top-4 predicted tokens.  

From the perspective of ABCD logits, NPO drives all four options close to zero and nearly identical, achieving $\mathrm{UE}_\text{MCQ}$ by uniformly suppressing candidate scores. In other words, ABCD are not true top-token candidates under NPO, as revealed by its top-4 prediction logits on ABCD. By contrast, RMU maintains the distribution of the original model’s logits but reshapes their relative distribution, attaining $\mathrm{UE}_\text{MCQ}$ by reordering rather than erasing signals. For the top-4 logits, NPO assigns much higher values than RMU or the original model, but these correspond to meaningless tokens (Table\,\ref{tab:selection_generation}). This shows that NPO achieves $\mathrm{UE}_\text{Open-QA}$ by severely disrupting generative capacity, explaining its substantially lower $\mathrm{UT}_\text{Open-QA}$ relative to RMU and the original model.  

\begin{figure}[htbp]
    \centering
    %\vspace{-3mm}
    \includegraphics[width=0.4\linewidth]{figs/idk_ap_dpo.pdf}
    \caption{Effective unlearning with utility preservation of IDK+AP when warm-started with DPO (called IDK+AP w/ DPO), given by $\mathrm{UT}_\text{Avg}$ and $\mathrm{UE}_\text{Avg}$ (as defined in Fig.\,\ref{fig:evaluatio_wmdp}.}
    \label{fig:idk_ap_dpo}
    %\vspace{-3mm}
\end{figure}

\paragraph{Mitigating utility loss in IDK+AP via DPO warm-start.}
As seen in Fig.\,\ref{fig:evaluatio_wmdp}-(c), DPO retains a significant portion of the original model’s utility, which we attribute to the presence of a positive preference signal that guides the model to prefer the targeted answers in response to the forget queries. We hypothesize that the pronounced utility degradation of IDK+AP arises from its stricter log-likelihood loss, which aggressively increases the probability of rejection responses for forget-relevant questions. To mitigate this and to verify our hypothesis, we propose to unlearn using IDK+AP after a `warm-start' with DPO for a few epochs.
As seen in \textbf{Fig.\,\ref{fig:idk_ap_dpo}}, this strategy (called IDK+AP w/ DPO) can infact increase preserve the utility, while achieving effective unlearning.
We note this is in contrast to the  setting of post-training of LLMs \citep{dubey2024llama} where SFT is followed by DPO. We think that in unlearning, the rejection responses provide a strong distribution shift for IDK+AP which is managed by warm-starting with DPO.

\paragraph{Robustness of knowledge vs. data-centric unlearning under quantization.} 

\begin{table}[htbp] % r=右侧, 宽度可调
\centering
\caption{\small Quantization performance of NPO and RMU is reported on MUSE Books and WMDP. UE and UT are assessed using KnowMem on $\Df$ and KnowMem on $\Dr$ for MUSE Books, and by $\mathrm{UE}_\text{MCQ}$ (Accuracy) and $\mathrm{UT}_\text{MCQ}$ (MMLU) for WMDP. Results are provided for full precision, 8-bit, and 4-bit models.}
\label{tab:quantization_results}
\vspace*{2mm}
\resizebox{0.5\linewidth}{!}{
\begin{tabular}{c ccc ccc}
\toprule[1pt]
\toprule
\rowcolor{LightCyan!50}
\multicolumn{7}{c}{\textbf{MUSE}} \\
\midrule
\multirow{2}{*}{\textbf{Method}} &
\multicolumn{3}{c}{\textbf{KnowMem on $\Df$ ($\downarrow$)}} &
\multicolumn{3}{c}{\textbf{KnowMem on $\Dr$ ($\uparrow$)}} \\
\cmidrule(lr){2-4} \cmidrule(lr){5-7}
& \textbf{Full} & \textbf{8 Bit} & \textbf{4 Bit} & \textbf{Full} & \textbf{8 Bit} & \textbf{4 Bit} \\
\midrule
\rowcolor{gray!10}
NPO &  2.10 &  2.12 & 30.30 & 55.31 & 52.89 & 48.79 \\
RMU & 24.44 & 24.06 &  8.16 & 59.55 & 55.80 & 25.84 \\
\midrule
\rowcolor{LightCyan!50}
\multicolumn{7}{c}{\textbf{WMDP}} \\
\midrule
\multirow{2}{*}{\textbf{Method}} &
\multicolumn{3}{c}{\textbf{$\mathrm{UE}_\text{MCQ}$ (Accuracy $\downarrow$)}} &
\multicolumn{3}{c}{\textbf{$\mathrm{UT}_\text{MCQ}$ (MMLU $\uparrow$)}} \\
\cmidrule(lr){2-4} \cmidrule(lr){5-7}
& \textbf{Full} & \textbf{8 Bit} & \textbf{4 Bit} & \textbf{Full} & \textbf{8 Bit} & \textbf{4 Bit} \\
\midrule
\rowcolor{gray!10}
NPO & 0.28 & 0.28 & 0.28 & 0.46 & 0.46 & 0.46 \\
RMU & 0.27 & 0.27 & 0.26 & 0.6 & 0.59 & 0.56 \\
\bottomrule
\bottomrule[1pt]
\end{tabular}
}
% \vspace*{-3mm}
\end{table}


As presented in Table\,\ref{tab:quantization_results} of \textbf{Appendix\,\ref{appx:add_exp}}, we observe that quantization affects both unlearning effectiveness and utility. For data-centric unlearning (MUSE), 4-bit quantization leads to a sharp decline in performance, with NPO showing a significant increase in KnowMem on $\Df$ and RMU suffering large drops in KnowMem on $\Dr$. In contrast, for knowledge removal (WMDP), both NPO and RMU maintain consistent UE across full precision, 8-bit, and 4-bit settings, while UT degrades only slightly. These results suggest that knowledge removal is generally more robust to post-unlearning quantization than content-based unlearning.



\begin{figure}[htbp] 
\centering
\vspace*{-2mm}
\begin{tabular}{cc}
\hspace*{-3mm}
\includegraphics[width=0.24\textwidth,height=!]{figs/landscape_tar.pdf} 
&
\hspace*{-5mm}
\includegraphics[width=0.24\textwidth,height=!]{figs/landscape_rmulat.pdf}
\vspace*{-1mm}
\\
\hspace*{-3mm} 
\small{(a) TAR} &  
\hspace*{-5mm} 
\small{(b) RMU+LAT}\\
\end{tabular}
 \vspace*{-2mm}
\caption{\small{The prediction loss landscape of the TAR and RMU+LAT-unlearned model on the forget set using the visualization tool in \citep{li2018visualizing}.
%where higher values around $x = y = 0$ indicate more effective unlearning. The 3D loss landscape is defined as $z = \ell(\btheta + x \cdot \mathbf{r}_1 + y \cdot \mathbf{r}_2)$, with $\btheta$ representing the unlearned model.
}}
\label{fig:landscape}
% \vspace*{-5mm}
\end{figure}

\paragraph{Loss landscape of TAR and RMU+LAT.} 
\textbf{Fig.\,\ref{fig:landscape}} visualizes the forget loss landscape of TAR and RMU+LAT following \citep{li2018visualizing}. The landscape of TAR is noticeably smoother, while RMU+LAT exhibits irregularities, indicating only \textit{local robustness}. This echoes the debate \citep{engstrom2018evaluating} contrasting standard adversarial training \citep{madry2018towards} with adversarial logit pairing \citep{kannan2018adversarial}, where leveraging logits (or other latent information) was argued to yield limited robustness. By analogy, in LLM unlearning, RMU+LAT also fails to achieve broad robustness due to its locality constraint.

