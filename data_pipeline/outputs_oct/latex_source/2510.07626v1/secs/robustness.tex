\section{Multi-Faceted Robustness Assessments for Unlearning: Needed and Complementary}
\label{sec:eval_Rob}

\begin{tcolorbox}[enhanced,attach boxed title to top center={yshift=-3mm,yshifttext=-1mm},
  colback=white,colframe=green!35!black,colbacktitle=green!5!white,
  title={\parbox{0.75\linewidth}{\centering Summary insights into robustness evaluations of unlearning methods}}, 
  coltitle=black, boxrule=0.8pt, 
  boxed title style={size=small,colframe=blue!45!black} ]

\textbf{(1)} In-domain relearning serves as a \textit{worst-case robustness evaluation}, as it most severely reduces unlearning effectiveness.  

\textbf{(2)} Robustness of in-domain relearning and out-of-domain fine-tuning are opposite: {\MDiv} methods are generally more robust under in-domain relearning, whereas {\MRep} methods exhibit stronger robustness under out-of-domain fine-tuning. 

\textbf{(3)} Quantization disproportionately impacts {\MRep}, and jailbreak attack robustness closely resembles that of in-domain relearning.  

\textbf{(4)} Robust design substantially improves robustness: NPO+SAM, NPO+IRM, and TAR significantly strengthen the resilience of both NPO and RMU.  

\end{tcolorbox}


Robustness evaluation of unlearning methods can be organized by the level at which adversarial attacks are applied \citep{che2025model}. 
We distinguish between \textit{model-level attacks}, including in-domain relearning \citep{hu2024jogging,fan2025towards}, out-of-domain fine-tuning \citep{wang2025invariance}, and quantization, and \textit{input-level attacks}, such as jailbreaking \citep{lucki2024adversarial}. 
In what follows, we analyze the robustness of existing unlearning methods across these attack categories.

\paragraph{In-domain relearning vs. out-of-domain fine-tuning.}

% \begin{wrapfigure}{r}{0.4\textwidth}
%     \centering
%     \vspace{-4mm}
%     \includegraphics[width=0.38\textwidth]{figs/heatmap_ft.pdf}
%     \caption{WMDP evaluation accuracy of twelve unlearning methods on WMDP Bio with Llama-3 8B Instruct before and after in-domain relearning and out-of-domain fine-tuning (GSM8K, SST2, and MNLI). * denotes the presence of a robust design in the corresponding method.}
%     \label{fig:heatmap_ft}
% \end{wrapfigure}

In-domain relearning refers to fine-tuning the unlearned model directly on a subset of the forget set $\Df$, which can substantially reduce unlearning effectiveness. 
By contrast, out-of-domain fine-tuning adapts the unlearned model to downstream datasets unrelated to the unlearning benchmark, such as GSM8K, SST2, or MNLI. 
In \textbf{Fig.~\ref{fig:robust_wmdp}(a)}, we present the WMDP evaluation accuracy of twelve unlearning methods before and after exposure to both in-domain and out-of-domain fine-tuning. 
Lower accuracy (lighter color) indicates stronger robustness, and methods incorporating explicit robust design are marked with an asterisk for clarity.

For methods without robust design, {\MRep} methods tend to exhibit much stronger robustness against out-of-domain fine-tuning compared to {\MDiv} methods. 
In contrast, for in-domain relearning attacks, {\MDiv} methods generally outperform {\MRep} methods. 
Adding robustness-oriented designs, such as NPO+SAM, NPO+IRM, and TAR, significantly improves robustness against both in-domain relearning and out-of-domain fine-tuning. However, RMU+LAT does not yield consistent robustness gains compared to RMU. 
It is also worth noting that in-domain relearning often proves to be the stronger attack, causing larger drops in unlearning effectiveness. From this perspective, in-domain relearning can be regarded as a worst-case fine-tuning scenario, but it does not fully capture the robustness properties of models when subjected to out-of-domain fine-tuning.


% \begin{wrapfigure}{r}{0.5\linewidth}  % r=右侧放置,宽度 0.5 行宽
%   \vspace*{-2mm}
%   \centering
%   \includegraphics[width=0.4\textwidth]{figs/2d_ue_ut_legend.pdf}\\[-1mm]

%   \begin{tabular}{cc}
%     \includegraphics[width=0.23\textwidth]{figs/2d_mmlu_gsm8k_quan.pdf} &
%     \includegraphics[width=0.23\textwidth]{figs/2d_acc_es_quan.pdf} \\
%     \small{(a) UE vs. relearning epoch \#} &  
%     \small{(b) UE vs. relearning data \#}\\
%   \end{tabular}

%   \vspace{-3mm}
%   \caption{\small{Comparison of unlearning effectiveness (UE) under different relearning settings.}}
%   \label{fig:relearn_wmdp_method}
%   \vspace*{-5mm}
% \end{wrapfigure}

\begin{figure}[htb]
%\vspace*{-2mm}
\center
\hspace*{50mm}
\includegraphics[width=0.6\linewidth]{figs/2d_ue_ut_legend.pdf}\\
% \vspace{-1mm}
\begin{tabular}{ccc}
% \hspace*{-3mm}
% \hspace*{-6mm}
\includegraphics[width=0.32\textwidth,height=!]{figs/heatmap_ft.pdf}
&
\includegraphics[width=0.30\textwidth,height=!]{figs/2d_acc_es_quan.pdf} 
&
% \hspace*{-6mm}
\includegraphics[width=0.29\textwidth,height=!]{figs/2d_mmlu_gsm8k_quan.pdf}
\vspace*{-1mm}
\\
\hspace*{3mm}{\small (a) $\mathrm{UT}_\text{MCQ}$ and $\mathrm{UT}_\text{Open-QA}$} &
% \hspace*{-3mm}
{\small (b) $\mathrm{UE}_\text{MCQ}$ vs. $\mathrm{UE}_\text{Open-QA}$} & 
% \hspace*{-3mm}
{\small (c) $\mathrm{UE}_\text{Avg}$ vs. $\mathrm{UT}_\text{Avg}$} \\
\end{tabular}
\vspace{-4mm}
\caption{\small{    
(a) WMDP evaluation accuracy of twelve unlearning methods on WMDP Bio with Llama-3 8B Instruct before and after in-domain relearning and out-of-domain fine-tuning (GSM8K, SST2, and MNLI). * denotes the presence of a robust design in the corresponding method. (b) The results of $\mathrm{UE}_\text{MCQ}$ (Accuracy) and $\mathrm{UE}_\text{Open-QA}$ (ES) for 12 unlearning methods, reported both before quantization and after 4-bit quantization. Solid lines connect results of the same method, while the black dashed line indicates performance after quantization. (c) The results of $\mathrm{UT}_\text{MCQ}$ (MMLU) and $\mathrm{UT}_\text{Open-QA}$ (GSM8K) for 12 unlearning methods, reported both before quantization and after 4-bit quantization.
}
}
\label{fig:robust_wmdp}
% \vspace*{-5mm}
\end{figure}



\paragraph{Quantization.} 
Quantization reduces the storage and computational costs of large language models by mapping high-precision parameters into a discrete range without modifying the model architecture \citep{hong2025stablequant,zheng2025empirical}. 
\citet{zhang2024catastrophic} were the first to investigate the effect of quantization on unlearned models in the context of the MUSE benchmark, showing that quantization can be regarded as a form of model-level attack. By lowering numerical precision, quantization may potentially cause an unlearned model to recover parts of the knowledge that was intended to be forgotten.

We further evaluate the robustness of unlearned models under 4-bit quantization, using four metrics: $\mathrm{UE}_\text{MCQ}$ (accuracy), $\mathrm{UE}_\text{Open-QA}$ (entailment score), $\mathrm{UT}_\text{MCQ}$ (MMLU), and $\mathrm{UT}_\text{Open-QA}$ (GSM8K). As shown in \textbf{Fig.~\ref{fig:robust_wmdp}(b) and (c)}, quantization generally improves unlearning effectiveness (lower UE values) but degrades utility. The impact is more pronounced for {\MRep} methods than for {\MDiv} methods. Notably, methods such as RR are not robust to quantization, as their UE values actually increase after quantization.


% 1. On the WMDP benchmark, quantization has little impact on evaluation outcomes, as all unlearning methods remain robust regardless of whether performance is measured using MCQ or Open-QA tasks. 2. In contrast, on the MUSE benchmark, quantization is effective against only a subset of methods, such as NPO and GradDiff. 3. To further explain this phenomenon, we analyze the overlap in the distribution of the top-50 tokens, which highlights how quantization shifts prediction distributions in a way that selectively affects certain approaches.


\paragraph{Input-level attacks and interactions between model-level and input-level robustness.} 
In addition to model-level perturbations, unlearned models are also vulnerable to input-level attacks that manipulate prompts in order to bypass unlearning. Among these, \textit{jailbreaking attacks} \citep{lucki2024adversarial, lynch2024eight, patil2023can} have received particular attention, since carefully crafted adversarial prompts can recover sensitive information that was intended to be forgotten at inference time.

\begin{wrapfigure}{r}{0.4\textwidth}
    \centering
    \vspace{-4mm}
    \includegraphics[width=0.38\textwidth]{figs/heatmap_robustness.pdf}
    \caption{WMDP evaluation accuracy of twelve unlearning methods on WMDP Bio with Llama-3 8B Instruct before and after in-domain relearning (IDR), out-of-domain fine-tuning (ODF on GSM8K), quantization (Quan), and input-level jailbreaking attacks (Jail). An asterisk (*) denotes the presence of a robust design.}
    \label{fig:heatmap_ft}
\end{wrapfigure}

As illustrated in \textbf{Fig.~\ref{fig:heatmap_ft}}, {\MDiv} methods without robust design (GradDiff, NPO, SimNPO) generally demonstrate stronger robustness against jailbreaking attacks compared with {\MRep} methods without robust design (RMU, RR). This may be attributed to the degraded generative capacity of {\MDiv} methods, which inadvertently hinders their ability to reveal sensitive knowledge. In contrast, {\MRej} methods such as ELM show the weakest robustness, indicating that rejection can often be bypassed by adversarial prompts. When robust design is incorporated, most methods (except RMU+LAT) exhibit significantly enhanced resilience against jailbreaking attacks.

What's more, we jointly present robustness across multiple attack types: model-level in-domain relearning (IDR), out-of-domain fine-tuning (ODF), quantization (Quan), and input-level jailbreaking attacks (Jail) in \textbf{Fig.~\ref{fig:heatmap_ft}}. The results suggest that the robustness patterns of jailbreaking attacks are more closely aligned with those of in-domain relearning, rather than out-of-domain fine-tuning. Specifically, {\MDiv} methods without robust design outperform {\MRep} methods in both jailbreaking and in-domain relearning settings, whereas the trend is reversed for out-of-domain fine-tuning. This highlights an important interaction: robustness to input-level attacks may share underlying characteristics with robustness to model-level in-domain relearning, but differs substantially from robustness to out-of-domain fine-tuning.




% \paragraph{Interactions between model-level and input-level attacks.} 
% Model-level and input-level attacks are complementary in how they probe vulnerabilities of unlearned models. 
% To visualize their relationship, use heatmaps comparing the outcomes of model-level attacks with those of input-level attacks.

% Quantization
% 1. Quantization 量化通过将高精度参数映射到一个离散范围来减少 LLM 的存储和计算需求,而不改变模型结构
% 2. Quantization 在 WMDP 数据集上,无论使用 MCQ 还是 Open-QA 的 evaluation 都不能起作用,所有 unlearning method 对于 quantization 都是鲁棒的.
% 3. Quantization 对于 MUSE 而言,只对部分 method 有效(如 NPO,Grad Diff)
% 4. 使用 top50 token 分布的 distribution 的 overlap 进行解释

% Input-level attack:
% 1. Jailbreaking attack 
% 2. In-context learning

% Tadeoff of robustnedd and UE

% The relationship between model-level and inpu-level attack: heatmap
