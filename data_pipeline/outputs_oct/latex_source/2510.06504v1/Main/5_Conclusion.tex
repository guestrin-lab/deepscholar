\vspace{-0.5mm}
\section{Conclusion}
\vspace{-0.5mm}
In this paper, we presented \name, comprising 1) \namesyn, a novel and effective framework that composes single-person motions into two-person interaction from LLM-generated text descriptions; and 2) \namegen, a high-quality and fine-grained two-person interaction generation framework equipped with word-level conditioning. The effectiveness of \namesyn has been confirmed by an ablation study, user study, qualitative results, and latent visualizations. Utilizing data generated by \namesyn, \namegen achieves a significant FID boost, achieving SoTA-level R-precision and FID, setting a new state-of-the-art for the two-person motion generation task.

\noindent
\textbf{Limitations and Future Work.} While \namegen demonstrates strong motion fidelity and faithfulness, it does not account for physical plausibility during generation, which can result in artifacts such as floating motions and ground penetration. Incorporating physics priors~\citep{peng2018deepmimic, peng2022ase, dou2023c, jiang2023drop, luo2023universal, zhang2024physpt, huang2025modskill} offers a promising avenue for future work. Additionally, although \namesyn provides an effective synthesis framework, extending it to learn motions directly from video~\citep{li2023coordinate, huang2024closely, qiu2023psvt, yuan2022glamr} remains an interesting and relatively unexplored direction.




