\section{Conclusion, Limitations and Future Work}

In this work, we 
present the
% \AKdeletes{present}\AKadds{study} the \AKdeletes{CSTOP problem as the} 
problem of assigning scores for temporal properties \AKadds{(STOPs)} given potentially noisy score predictors for local properties. We represent these properties using LTL and propose a scoring function LogSTOP for assigning STOPs. We then introduce the QMTP and TP2VR benchmarks for evaluating query matching and ranked retrieval 
% with 15 diverse temporal property templates
with temporal properties
over objects / actions in videos and emotions in speech. LogSTOP with simple neural predictors outperforms LVLMs / LALMs, Temporal Logic-based baselines, and text-to-retrieval methods on the benchmarks.
% CSTOP\AKadds{-RealTLV-video} benchmark for evaluating LogCSTOP on the downstream application of query matching with temporal properties (TP-matching) and find that it outperforms LVLMs \AKdeletes{and LALMs} on TP-matching with videos\AKdeletes{ and speech respectively}.

\noindent{\textbf{Limitations. }} 
% \textcolor{red}{maybe mention that the assumptions are simplistic.}
% There are certain limitations of the proposed language and LogCSTOP.
There are properties such as "there are always 2 cars" that cannot directly be expressed in LTL. 
% For example, the property "there are always 2 cars" can be easily expressed in text prompts to LVLMs but not in LTL since counting is not supported. 
Future work should hence explore more expressive logics~\citep{strem,huang2023laser} or construct local predictors for complex properties.
% Secondly, we only evaluate LogSTOP on one downstream application but these scores can also be useful for other applications such as ranking and fine-tuning of local predictors with high level feedback. Finally, we only consider one input modality for local properties (video\AKdeletes{ / audio}). 
While we only focus on sequences with single modalities, it will be interesting to see LogSTOP being for multi-modal applications where the local properties are over different modalities with scores from different local predictors. 
% For example, the property "there exists a red car that is present in all frames of the video" cannot be expressed in the language since quantifiers are not supported (i.e., "there exists" cannot be specified), local properties cannot be tagged (car detected in frame 1 might be different from the car detected in frame 2) and attributes of local properties (e.g., "red") cannot be specified. The last issue here can be mitigated by predictors that can detect more flexible local properties such as CLIP and future work should explore this along with the extension of the language to support the first two features. Moreover, the current framework assumes strong independence of confidence scores across local properties and time steps -- it will be interesting to see solutions that can mitigate any of the assumptions. 
% Finally, this work looks at two applications of CSTOP. There are other interesting downstream applications of CSTOP that can lead to exciting future work: updating the local predictors using high-level feedback from CSTOPs, etc. 

% Limitations:

% \begin{enumerate}
%     \item Language is restrictive ("car" is not ID'd, depends on the classes supported by the models). No quantifiers.
% \end{enumerate}

% Future work:

% \begin{enumerate}
%     \item Models are frozen (not updated, similar to inference with pre-trained models)
%     \item Downstream applications: updating models using high-level feedback, uncertainty quantification
% \end{enumerate}