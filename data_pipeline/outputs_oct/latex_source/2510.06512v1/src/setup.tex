\section{Representing Temporal Properties over Prediction Sequences}
\label{sec:def}

Let \(X = [x_1, \ldots, x_T]\) 
denote a sequence of data items of length \(T\), where \(x_t\) denotes the data item at timestep \(1 \leq t \leq T\). 
Let \(\mathcal{X}\) denote the set of all such sequences, and \(\mathcal{T}\) denote the set of all timesteps \(\{1, \ldots, T\}\).
Further, let \(C\) denote a finite set of local properties of interest. 
In general, \(X\) and \(C\) could correspond to sequences of arbitrary modalities and properties respectively,
including but not limited to objects or actions in videos, and speakers or emotions in audio clips.
While these sequences could be over continuous time, we assume that they are discretized into \(T\) timesteps for simplicity.

We assume that there exists a true labeling function \(y : \mathcal{X} \times C \times \mathcal{T} \mapsto \{0, 1\}\) 
such that \(y(X, c, t) = 1\)
if the property \(c \in C\) is expressed by \(x_t\), and
\(y(X, c, t) = 0\)
otherwise. 
For example,
\(X\) could correspond to a video with frames \(x_t\) at timestep \(t\) and \(C\) could be objects of interest such as "car" and "pedestrian". In this case, \(y(X, c, t) = 1\) would indicate that object \(c\) is present in frame \(x_t\).

In this work, we are interested in temporal compositions of these local properties. 
For example, given the true labels for the object "car" in individual frames, 
how can we define the label for a car being present in all frames or alternatively any frame? 
Or, given the true labels for "car" and "pedestrian", 
how can we define the label for a car being present in all frames until a pedestrian is detected? 

We find that Linear Temporal Logic (LTL), 
widely used for formal specification and verification of reactive systems ~\citep{ltl}, 
provides a suitable \textit{language} for expressing such temporal properties. 
Formally, a temporal property \(\varphi\) over local properties \(C\) can be expressed in LTL as follows:  
\[\varphi \coloneq \top \mid \bot \mid c \mid \neg \varphi \mid \varphi_1 \land \varphi_2 \mid \varphi_1 \lor \varphi_2 \mid \bigcirc \varphi \mid \BoxOp{I} \varphi \mid \varphi_1 \UntilOp{I} \varphi_2 \] 
where, \(c \in C\) is a local property and \(\varphi_1, \varphi_2\) are temporal properties. \(\neg, \land, \lor\) are the logical \textit{not}, \textit{and}, and \textit{or} operators respectively. \(\bigcirc\), \(\BoxOp{I}\) and \(\UntilOp{I}\)  are temporal operators \textit{Next}, \textit{Always}, and \textit{Until} respectively. Other temporal operators such as "Eventually \(\varphi\)" (\(\DiaOp{I} \varphi\)) can then be derived as \(\neg \BoxOp{I} \neg \varphi\). 

This language can now be used to represent properties from the previous examples. For instance,  the temporal properties for a car being present in all frames and any frame in a video can be represented as \(\varphi = \BoxOp{I}\, \mathsf{car}\) and \(\varphi = \DiaOp{I}\, \mathsf{car}\) respectively. Furthermore, the property that a car is present in all frames until a pedestrian is present can be represented as \(\varphi = \mathsf{car} \UntilOp{I} \mathsf{pedestrian}\).


The ground truth labeling function \(y(X, c, t)\) for local properties can be lifted to \(y(X, \varphi, t)\), meaning that the sequence \(X\) expresses temporal property \(\varphi\) starting at timestep \(t\), using the standard semantics for LTL over finite sequences~\citep{ltlf} as follows: for \(1 \leq t \leq T\),

\begin{itemize}
    \item \(y(X, \top, t) = 1\) and \(y(X, \bot, t) = 0\)
    \item \(y(X, \neg \varphi, t) = 1\) iff \(y(X, \varphi, t) = 0\)
    \item \(y(X, \varphi_1 \land \varphi_2, t) = 1\) iff \(y(X, \varphi_1, t) = 1\) and \(y(X, \varphi_2, t) = 1\)
    \item \(y(X, \varphi_1 \lor \varphi_2, t) = 1\) iff \(y(X, \varphi_1, t) = 1\) or \(y(X, \varphi_2, t) = 1\)
    \item \(y(X, \bigcirc\varphi, t) = 1\) iff \(t < T\) and \(y(X, \varphi, t+1) = 1\)
    \item \(y(X, \BoxOp{I}\varphi, t) = 1\) iff \(y(X, \varphi, t') = 1\) for all \(t \leq t' \leq T\)
    \item \(y(X, \varphi_1 \UntilOp{I} \varphi_2, t) = 1\) iff there exists a \(t \leq t' \leq T\) such that \(y(X, \varphi_2, t') = 1\) and \(y(X, \varphi_1, t'') = 1\) for all \(t \leq t'' < t'\)
\end{itemize}


Given the true labeling functions for local properties \(y(X, c, \cdot)\), 
this semantics can perfectly determine if a sequence \(X\) expresses a temporal property \(\varphi\) 
(if and only if \(y(X, \varphi, 1) = 1\)). 
In practice, however, we do not have access to these true labeling functions. 
We assume that noisy predictors can be used instead to provide 
\textit{scores} \(\hat{y} : \mathcal{X} \times C \times \mathcal{T} \mapsto [a, b]\) for the label being 1,
where \(a\) and \(b\) are the lower and upper bounds of the score range respectively. 
The estimate for true label \(y(X, c, t)\) can then be computed as \(\tilde{y}(X, c, t) = \hat{y}(X, c, t) > \tau\) 
for some threshold \(\tau\). 
Most neural models, including object detection models such as YOLO, 
provide scores in \([0, 1]\) and an object \(c\) is said to be detected at \(t\) 
if \(\hat{y}(X, c, t) > \tau\), where \(\tau\) usually is \(0.5\). 
The accuracy of these predictors then just measures how well \(\tilde{y}(X, c, \cdot)\) estimates \(y(X, c, \cdot)\). 

We formally introduce the problem of assigning Scores for TempOral Properties (STOPs) as follows: 
\textit{
Given predictors for local properties \(\hat{y} : \mathcal{X} \times C \times \mathcal{T} \mapsto [a, b]\) 
and a temporal property \(\varphi\) defined over local properties in \(C\), 
how can a score for \(\varphi\) and sequence \(X\) at time step \(1 \leq t \leq T\), \(\hat{y}(X, \varphi, t)\), be assigned?
}