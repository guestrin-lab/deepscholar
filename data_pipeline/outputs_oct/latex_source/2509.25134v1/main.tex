
\documentclass[10pt, twocolumn, letterpaper]{article}

\usepackage[pagenumbers]{iccv} %

\usepackage[dvipsnames]{xcolor}
\definecolor{thedarkblue}{RGB}{0,0,120} %
\definecolor{mydarkblue}{rgb}{0,0.08,0.45} %
\definecolor{darkblue}{rgb}{0,0.08,180}
\colorlet{TufteRed}{red!80!black}
\definecolor{theblue}{RGB}{0,0,180}
\colorlet{thered}{TufteRed}
      
\usepackage{microtype}
\usepackage{balance}


\usepackage{booktabs}
\usepackage{tabularx}



\renewcommand{\qedsymbol}{$\blacksquare$}
\newcommand{\eat}[1]{\ignorespaces}
\usepackage{comment}

\usepackage{tikz}
\usepackage{verbatim}
\usetikzlibrary{arrows}
\usetikzlibrary{shapes,snakes}
\usetikzlibrary{decorations.pathmorphing} %
\usetikzlibrary{fit}					%
\usetikzlibrary{backgrounds}	%

\usepackage{ragged2e}
\usepackage{makecell}
\usepackage{multirow}
\usepackage{microtype}
\usepackage{balance}
\usepackage{setspace}

\graphicspath{{./}{./graphics/}}
\newcolumntype{H}{>{\setbox0=\hbox\bgroup}c<{\egroup}@{}}

\newcolumntype{R}[1]{>{\RaggedLeft\arraybackslash}} %
\newcolumntype{L}[1]{>{\RaggedRight\arraybackslash}} %

\newcommand\TTT{\rule{0pt}{3.2ex}}
\newcommand\BBB{\rule[-1.4ex]{0pt}{0pt}}


\newcommand{\rbr}[1]{\left(#1\right)}
\newcommand{\cbr}[1]{\left\{#1\right\}}
\newcommand{\nbr}[1]{\left\|#1\right\|}
\newcommand{\abr}[1]{\left|#1\right|}
\newcommand{\abs}[1]{\left|#1\right|}
\newcommand{\floor}[1]{\left\lfloor #1 \right\rfloor}
\newcommand{\ceil}[1]{\left\lceil #1 \right\rceil}
\newcommand{\inner}[2]{\left\langle #1,#2 \right\rangle}

\newcommand{\etal}{\emph{et al.}}
\newcommand{\ea}{\emph{et al.}}
\newcommand{\eg}{\emph{e.g.}}
\newcommand{\ie}{\emph{i.e.}}
\newcommand{\iid}{\emph{iid}}
\newcommand{\cf}{\emph{cf.}\ }
\newcommand{\wrt}{\emph{w.r.t.}\ }
\newcommand{\st}{\emph{s.t.}\ }

\newtheorem{corollary}{\bfseries{Corollary}}
\newtheorem{proof2}{PROOF}
\newtheorem*{fact}{Fact}
\newtheorem*{note}{\hspace{-1em}\textsc{Note}}
\newtheorem{corol}{Corollary}%
\newtheorem{axiom}{Axiom}%
\newtheorem{cond}{Condition}%
\newtheorem{property2}{Property}%
\newtheorem{property}{Property}%
\newtheorem{lemma}{\hspace{-1em}\bfseries{Lemma}}
\newtheorem{Definition}{\hspace{-1em}\bfseries{Definition}}
\newtheorem{Claim}{Claim}%

\AtBeginEnvironment{pmatrix}{\setlength{\arraycolsep}{2pt}}



\providecommand{\tensor}[1]{\boldsymbol{\mathcal{#1}}}%
\providecommand{\mat}[1]{\boldsymbol{\mathrm{#1}}}%
\renewcommand{\vec}[1]{\boldsymbol{\mathrm{#1}}}
\providecommand{\sca}[1]{{\mathrm{#1}}}%
\DeclareMathOperator{\rank}{rank}%
\DeclareMathOperator{\diag}{diag}%
\DeclareMathOperator{\Diag}{Diag}%
\providecommand{\itr}[2]{#1^{(#2)}}
\providecommand{\itn}[1]{^{(#1)}}%
\providecommand{\eps}{\varepsilon}%
\providecommand{\kron}{\otimes}
\DeclareMathOperator{\tvec}{vec}
\providecommand{\pmat}[1]{\begin{pmatrix} #1 \end{pmatrix}}
\providecommand{\bmat}[1]{\begin{bmatrix} #1 \end{bmatrix}}
\providecommand{\spmat}[1]{\left(\begin{smallmatrix} #1 \end{smallmatrix}\right)}
\providecommand{\sbmat}[1]{\left[\begin{smallmatrix} #1 \end{smallmatrix}\right]}

\DeclareMathOperator*{\minimize}{minimize}
\DeclareMathOperator*{\maximize}{maximize}
\DeclareMathOperator*{\argmax}{argmax}
\DeclareMathOperator*{\argmin}{argmin}
\DeclareMathOperator*{\argsort}{arg\,sort}
\providecommand{\subjectto}{\ensuremath{\text{subject to}}}
\providecommand{\MINof}[1][]{{\displaystyle \minimize_{#1}}}
\providecommand{\MIN}[2]{\begin{array}{ll} \MINof[#1] & #2 \end{array}}
\providecommand{\MINone}[3]{\begin{array}{ll} \MINof[#1] & #2 \\ \subjectto  & #3 \end{array}}
\providecommand{\MINtwo}[4]{\begin{array}{ll} \MINof[#1] & #2 \\ \subjectto  & #3 \\ & #4 \end{array}}
\providecommand{\MINthree}[5]{\begin{array}{ll} \MINof[#1] & #2 \\ \subjectto  & #3 \\ & #4 \\ & #5 \end{array}}
\providecommand{\MINfour}[6]{\begin{array}{ll} \MINof[#1] & #2 \\ \subjectto  & #3 \\ & #4 \\ & #5 \\ & #6 \end{array}}
\providecommand{\MAXof}[1][]{{\displaystyle \maximize_{#1}}}
\providecommand{\MAX}[2]{\begin{array}{ll} \MAXof[#1] & #2 \end{array}}
\providecommand{\MAXone}[3]{\begin{array}{ll} \MAXof[#1] & #2 \\ \subjectto & #3 \end{array}}
\providecommand{\MAXtwo}[4]{\begin{array}{ll} \MAXof[#1] & #2 \\ \subjectto  & #3 \\ & #4 \end{array}}
\providecommand{\MAXthree}[5]{\begin{array}{ll} \MAXof[#1] & #2 \\ \subjectto  & #3 \\ & #4 \\ & #5 \end{array}}
\providecommand{\MAXfour}[6]{\begin{array}{ll} \MAXof[#1] & #2 \\ \subjectto  & #3 \\ & #4 \\ & #5 \\ & #6 \end{array}}

\DeclareMathOperator{\E}{E}
\DeclareMathOperator{\hugeE}{\mbox{\huge\raise-0.3ex\hbox{E}}}
\DeclareMathOperator{\p}{\mathbb{P}}
\DeclareMathOperator{\hugep}{\mbox{\huge\raise-0.3ex\hbox{$\p$}}}
\DeclareMathOperator{\Var}{Var}
\DeclareMathOperator{\Cov}{Cov}
\DeclareMathOperator{\Bias}{Bias}
\DeclareMathOperator{\sign}{sign}
\DeclareMathOperator{\Std}{Std}
\providecommand{\Eof}{\E\BracketOf}
\providecommand{\hugeEof}{\hugeE\BracketOf}
\providecommand{\Stdof}{\Std\BracketOf}
\providecommand{\varof}{\Std\BracketOf}
\providecommand{\Covof}[2]{\Cov\BrackeOf#1,#2\right]}
\providecommand{\prob}[1][]{\p_{#1}\BraceOf}
\providecommand{\hugeprob}[1][]{\hugep_{#1}\BraceOf}

\DeclareMathOperator{\degree}{degree}
\DeclareMathOperator{\trace}{trace}
\providecommand{\set}{\mathcal}
\providecommand{\graph}{\mathcal}
\providecommand{\mathdef}{\equiv}
\providecommand{\card}{\absof}
\providecommand{\cardof}{\absof}
\providecommand{\eqdef}{\equiv}
\providecommand{\eps}{\varepsilon}
\DeclareMathOperator{\bigO}{O}
\providecommand{\bigOof}{\bigO\ParensOf}
\providecommand{\absof}[1]{$\left| #1 \right|$}

\newcommand{\RR}{\mathbb{R}}
\newcommand{\CC}{\mathbb{C}}



\providecommand{\eye}{\mat{I}}
\providecommand{\mA}{\ensuremath{\mat{A}}}
\providecommand{\mB}{\ensuremath{\mat{B}}}
\providecommand{\mC}{\ensuremath{\mat{C}}}
\providecommand{\mD}{\ensuremath{\mat{D}}}
\providecommand{\mE}{\ensuremath{\mat{E}}}
\providecommand{\mF}{\ensuremath{\mat{F}}}
\providecommand{\mG}{\ensuremath{\mat{G}}}
\providecommand{\mH}{\ensuremath{\mat{H}}}
\providecommand{\mI}{\ensuremath{\mat{I}}}
\providecommand{\mJ}{\ensuremath{\mat{J}}}
\providecommand{\mK}{\ensuremath{\mat{K}}}
\providecommand{\mL}{\ensuremath{\mat{L}}}
\providecommand{\mM}{\ensuremath{\mat{M}}}
\providecommand{\mN}{\ensuremath{\mat{N}}}
\providecommand{\mO}{\ensuremath{\mat{O}}}
\providecommand{\mP}{\ensuremath{\mat{P}}}
\providecommand{\mQ}{\ensuremath{\mat{Q}}}
\providecommand{\mR}{\ensuremath{\mat{R}}}
\providecommand{\mS}{\ensuremath{\mat{S}}}
\providecommand{\mT}{\ensuremath{\mat{T}}}
\providecommand{\mU}{\ensuremath{\mat{U}}}
\providecommand{\mV}{\ensuremath{\mat{V}}}
\providecommand{\mW}{\ensuremath{\mat{W}}}
\providecommand{\mX}{\ensuremath{\mat{X}}}
\providecommand{\mY}{\ensuremath{\mat{Y}}}
\providecommand{\mZ}{\ensuremath{\mat{Z}}}
\providecommand{\mLambda}{\ensuremath{\mat{\Lambda}}}
\providecommand{\mzero}{\ensuremath{\mat{0}}}

\providecommand{\tA}{\ensuremath{\tensor{A}}}
\providecommand{\tB}{\ensuremath{\tensor{B}}}
\providecommand{\tC}{\ensuremath{\tensor{C}}}
\providecommand{\tD}{\ensuremath{\tensor{D}}}
\providecommand{\tE}{\ensuremath{\tensor{E}}}
\providecommand{\tF}{\ensuremath{\tensor{F}}}
\providecommand{\tG}{\ensuremath{\tensor{G}}}
\providecommand{\tH}{\ensuremath{\tensor{H}}}
\providecommand{\tI}{\ensuremath{\tensor{I}}}
\providecommand{\tJ}{\ensuremath{\tensor{J}}}
\providecommand{\tK}{\ensuremath{\tensor{K}}}
\providecommand{\tL}{\ensuremath{\tensor{L}}}
\providecommand{\tM}{\ensuremath{\tensor{M}}}
\providecommand{\tN}{\ensuremath{\tensor{N}}}
\providecommand{\tO}{\ensuremath{\tensor{O}}}
\providecommand{\tP}{\ensuremath{\tensor{P}}}
\providecommand{\tQ}{\ensuremath{\tensor{Q}}}
\providecommand{\tR}{\ensuremath{\tensor{R}}}
\providecommand{\tS}{\ensuremath{\tensor{S}}}
\providecommand{\tT}{\ensuremath{\tensor{T}}}
\providecommand{\tU}{\ensuremath{\tensor{U}}}
\providecommand{\tV}{\ensuremath{\tensor{V}}}
\providecommand{\tW}{\ensuremath{\tensor{W}}}
\providecommand{\tX}{\ensuremath{\tensor{X}}}
\providecommand{\tY}{\ensuremath{\tensor{Y}}}
\providecommand{\tZ}{\ensuremath{\tensor{Z}}}

\providecommand{\ones}{\vec{e}}
\providecommand{\va}{\ensuremath{\vec{a}}}
\providecommand{\vb}{\ensuremath{\vec{b}}}
\providecommand{\vc}{\ensuremath{\vec{c}}}
\providecommand{\vd}{\ensuremath{\vec{d}}}
\providecommand{\ve}{\ensuremath{\vec{e}}}
\providecommand{\vf}{\ensuremath{\vec{f}}}
\providecommand{\vg}{\ensuremath{\vec{g}}}
\providecommand{\vh}{\ensuremath{\vec{h}}}
\providecommand{\vi}{\ensuremath{\vec{i}}}
\providecommand{\vj}{\ensuremath{\vec{j}}}
\providecommand{\vk}{\ensuremath{\vec{k}}}
\providecommand{\vl}{\ensuremath{\vec{l}}}
\providecommand{\vm}{\ensuremath{\vec{l}}}
\providecommand{\vn}{\ensuremath{\vec{n}}}
\providecommand{\vo}{\ensuremath{\vec{o}}}
\providecommand{\vp}{\ensuremath{\vec{p}}}
\providecommand{\vq}{\ensuremath{\vec{q}}}
\providecommand{\vr}{\ensuremath{\vec{r}}}
\providecommand{\vs}{\ensuremath{\vec{s}}}
\providecommand{\vt}{\ensuremath{\vec{t}}}
\providecommand{\vu}{\ensuremath{\vec{u}}}
\providecommand{\vv}{\ensuremath{\vec{v}}}
\renewcommand{\vv}{\ensuremath{\vec{v}}}
\providecommand{\vw}{\ensuremath{\vec{w}}}
\providecommand{\vx}{\ensuremath{\vec{x}}}
\providecommand{\vy}{\ensuremath{\vec{y}}}
\providecommand{\vz}{\ensuremath{\vec{z}}}
\providecommand{\vpi}{\ensuremath{\vecalt{\pi}}} 

\providecommand{\ssa}{\ensuremath{\sca{a}}}
\providecommand{\ssb}{\ensuremath{\sca{b}}}
\providecommand{\ssc}{\ensuremath{\sca{c}}}
\providecommand{\ssd}{\ensuremath{\sca{d}}}
\providecommand{\sse}{\ensuremath{\sca{e}}}
\providecommand{\ssf}{\ensuremath{\sca{f}}}
\providecommand{\ssg}{\ensuremath{\sca{g}}}
\providecommand{\ssh}{\ensuremath{\sca{h}}}
\providecommand{\ssi}{\ensuremath{\sca{i}}}
\providecommand{\ssj}{\ensuremath{\sca{j}}}
\providecommand{\ssk}{\ensuremath{\sca{k}}}
\providecommand{\ssl}{\ensuremath{\sca{l}}}
\providecommand{\ssm}{\ensuremath{\sca{l}}}
\providecommand{\ssn}{\ensuremath{\sca{n}}}
\providecommand{\sso}{\ensuremath{\sca{o}}}
\providecommand{\ssp}{\ensuremath{\sca{p}}}
\providecommand{\ssq}{\ensuremath{\sca{q}}}
\providecommand{\ssr}{\ensuremath{\sca{r}}}
\providecommand{\sss}{\ensuremath{\sca{s}}}
\providecommand{\sst}{\ensuremath{\sca{t}}}
\providecommand{\ssu}{\ensuremath{\sca{u}}}
\providecommand{\ssv}{\ensuremath{\sca{v}}}
\providecommand{\ssw}{\ensuremath{\sca{w}}}
\providecommand{\ssx}{\ensuremath{\sca{x}}}
\providecommand{\ssy}{\ensuremath{\sca{y}}}
\providecommand{\ssz}{\ensuremath{\sca{z}}}
\providecommand{\sspi}{\ensuremath{\scaalt{\pi}}} 


\DeclareMathOperator{\cut}{cut}
\DeclareMathOperator{\vol}{vol}



\definecolor{googleblue}{HTML}{4285F4}
\definecolor{googlered}{HTML}{DB4437}
\definecolor{googlepurple}{HTML}{A142F4} %
\definecolor{googlegreen}{HTML}{0F9D58}


\definecolor{iccvblue}{rgb}{0.21,0.49,0.74}
\usepackage[pagebackref, breaklinks, colorlinks, allcolors=iccvblue]{hyperref}


\usepackage{bm}
\usepackage{amsmath,amssymb}
\usepackage{url}
\usepackage{multirow}
\usepackage{tikz}
\usepackage{comment}
\usepackage[accsupp]{axessibility}  %
\usepackage{etoolbox}


\usepackage[capitalize]{cleveref}
\crefname{appendix}{Sec.}{Secs.}
\Crefname{appendix}{Sec.}{Secs.}

\newlength\squareheight
  \setlength\squareheight{6.75pt}
  \newcommand\squareslash{\tikz{\draw (0,0) rectangle (\squareheight,\squareheight);\draw(0,0) -- (\squareheight,\squareheight)}}
\DeclareMathOperator\squarediv{\squareslash}

\usepackage{algorithm}
\usepackage{listings}
\usepackage{algorithmic}
\newcommand{\1}{\mbox{1}\hspace{-0.25em}\mbox{l}}



\newcommand{\suppref}[1]{Supp.~\cref{#1}}
\newcommand{\ours}[0]{LayerD}
\newcommand{\Ours}[0]{LayerD}



\def\paperID{9379} %
\def\confName{ICCV}
\def\confYear{2025}

\title{\ours{}: Decomposing Raster Graphic Designs into Layers}


\author{
    Tomoyuki Suzuki\textsuperscript{1} \quad Kang-Jun Liu\textsuperscript{2} \quad Naoto Inoue\textsuperscript{1} \quad Kota Yamaguchi\textsuperscript{1} \\
    $^{1}$CyberAgent \quad $^{2}$Tohoku University 
}



\begin{document}

    \twocolumn[
    \begin{@twocolumnfalse}
    {%
    \maketitle
    \vspace{-3mm}
    \centering
    \includegraphics[keepaspectratio, width=1\linewidth]{figures/teaser.pdf}
    \captionof{figure}{
      \ours{} effectively decomposes raster graphic design images into layers, where the input design contains various elements such as typographic entities, embellishments, vector shapes, or even image materials.
      Once decomposed, one can apply image editing operations such as color conversion or translation at the layer level, or further apply other post-processing such as OCR to vectorize each raster layer.
    }
    \label{fig:teaser}
    \vspace{3mm}
    }
    \end{@twocolumnfalse}]
    Generating realistic videos with diffusion transformers demands significant computation, with attention layers becoming the central bottleneck.
Even producing a short clip requires running a transformer over a very long sequence of embeddings, e.g., more than 30K embeddings for a 5-second video. These long sequence lengths thus incur significant compute latencies. Prior work aims to mitigate this bottleneck by exploiting sparsity in the attention layers to reduce the computation required. However, these works typically rely on block-sparse attention, which skips score computation only when all entries in a \emph{block} of attention scores (corresponding to $M$ queries and $M$ keys, with $M=64$ typically) are zero. This coarse-granular skipping of attention scores does not fully exploit sparsity in the attention map and leaves significant room for improvement.


In this work, we propose \X, a sparse attention mechanism for long-context diffusion transformers that leverages sparsity at a fine granularity. 
Unlike block-sparse attention, which skips entire $M \times M$ blocks, our approach skips computations at the granularity of $M \times 1$ \emph{slices} of the attention map. Each slice is produced as a result of query-key dot products between a block of query vectors and a \emph{single key}. To implement our proposed sparse attention mechanism, we construct a new highly efficient bulk-load operation called asynchronous-gather load. This load operation gathers a sparse set of relevant key-value vectors from memory and arranges them into packed tiles in the GPU's shared memory. In this manner, only a sparse set of keys relevant to those queries are loaded into shared memory when computing attention for a block of queries, in contrast to loading full blocks of key tokens in block-sparse attention. Our fine-grained sparse attention, applied to video diffusion models, achieves an average 1.55x (up to 1.65x) speedup for 5 second, 480p videos, and an average 1.41x (up to 1.49x) for 5 second, 720p videos on a single H100 GPU.

\textbf{\textcolor{magenta}{Code:}} \url{https://github.com/sankeerth95/FG-Attn}

% \begin{center}
% \begingroup
% \endgroup
% \end{center}


    \section{Introduction\label{sec:intro}}

In several different contexts, hardware designers have developed
systems in which the contents of memory can survive system crashes.
Compute Express Link (CXL) is a new open standard that enables cache
coherent shared memory across a network.  CXL also supports 
memory interface to persistent storage~\cite{samsungcxlssd}.
Finally, the contents of battery-backed NVDIMM's can survive power
outages.

In all of these contexts, the in-memory data manipulated by a
computation can survive machine crashes, and it can be desirable to
ensure that key data structures are crash consistent so that they can
be safely accessed after crashes.  Achieving crash consistency is
complicated by the volatility of CPU caches, necessitating explicit
flush and fence instructions to persist data.~\footnote{CXL can 
optionally use energy storage to implement flush on failure, though
this requires support by both the device and
system components. This is not a panacea; many failure modes
would still result in lost cache lines.}  Software developers must use special flush and fence instructions to force data to be
written back to the underlying memory.

There is a body of work~\cite{psan,yashme,pmtest} on finding
bugs in persistent memory programs that utilize flushes and fences.  These tools range from model checkers~\cite{yat,jaaru}
to various dynamic bug finding
tools~\cite{xfdetector,pmemcheck,agamotto}.  The prevalence of such bugs,
with 183 new bugs reported by various tools~\cite{jaaru, psan,
  witcher-sosp21, pmdebugger-asplos21, xfdetector, pmtest, agamotto}
in a small expert-written program set, underscores the difficulty of
manual management and motivates automating flush/fence insertion via
compilers.

We developed a new tool that automatically generates the necessary
flush and fence operations.  Strict
persistency~\cite{memorypelley2014} ensures that the ``persistency
memory order is identical to volatile memory order''.  The observation
here is that bugs caused by missing flush and fence operations can be
eliminated by making stores become persistent in the same order that
they become visible to other threads.

Enforcing strict
persistency is expensive as it requires threads to stall frequently to wait for previous stores to be persisted. Fortunately, we can further loosen the requirements for strict persistency.  If it
is impossible for the program to observe that stores were persisted in
a different order, we can permit those stores to be reordered, and partly recover from the performance penalties of strict persistency.
Robustness leverages this additional degree of freedom and ensures
that any execution of a program under a weak persistency model is
equivalent to some execution of the program under the strict
persistency model~\cite{psan}.  This suffices to ensure the correct
usage of flush and fence operations as additional flush and fence
operations will not alter the set of possible post-crash program
executions. As missing flushes and fences are a major source of bugs in persistent memory program, a tool that eliminates these concerns greatly simplifies persistent memory programming. 

Robustness to persistency models has been explored before in the context of bug finding.  PSan uses a dynamic analysis combined with random execution or model checking to check robustness.  PSan
suffers from the same limitations as all dynamic analysis---it
requires test cases and may miss bugs that are not revealed by the
test cases.  Thus, the analysis used by PSan cannot be the foundation
of a compiler that automatically inserts flush and fence operations.

\textbf{\tool is
  applicable to CXL shared memory and all non-volatile memory 
  types.  We collectively refer to all of these memory types as
  persistent memory (PM) throughout the paper for convenience.}

This paper makes the following contributions:
\begin{itemize}
\item {\bf Static Analysis:}  It presents the first static analysis that can efficiently analyze full programs for missing flushes or fences using robustness as a correctness criteria.

\item {\bf PM Analysis:} 
It presents the first static analysis that can determine which operations will only modify volatile or local memory, reducing analysis and annotation overhead.   

\item {\bf Automated Flush Insertion:} It presents the first static tool that can both repair missing flush and fence bugs as well as insert any missing fences, freeing the developer from this task.

\item {\bf Eliminates Missing Flush Bugs:} It presents the first tool that ensures the absence of missing flush bugs.

\item {\bf Evaluation:} It evaluates \tool on a wide range benchmarks, incurring minimal runtime overhead.
\end{itemize}

We have made \tool's source code, benchmarks, and scripts for reproducing evaluation results available at: \url{https://github.com/uci-plrg/PMRobust-docker}
    \section{Related works}
\label{sec_related_works}
\subsection{Autoregressive image generation}
Early work~\cite{van2016pixelcnn} generates images directly at pixel level. Later approaches adopt a two-stage pipeline: images are first quantized into discrete tokens~\cite{esser2021vqgan,yu2021vit_vqgan}, then generated with Transformers in raster order~\cite{ding2021cogview,ge2023seed,ramesh2021dalle,yu2022parti,he2024mars,wang2024emu3}.
Recent efforts scale this paradign with larger models and stronger conditioning. LlamaGen~\cite{sun2024llamagen} provides class and text-conditioned baselines; Lumina-mGPT~\cite{liu2024lumina_mgpt} and Anole~\cite{chern2024anole} fine-tune Chameleon~\cite{chameleonteam2025chameleon} for improved text-conditioned generation. Unified frameworks further bridge understanding and generation~\cite{wu2024janus,chen2025janus_pro,jiao2025unitoken,unitok} in a single Transformer.
Meanwhile, image tokenizers have evolved for better reconstruction~\cite{yu2024titok,lee2022rqvae,yu2023lfq,zhao2024bsq} or multimodal integration~\cite{zhang2025v2flow,qu2024tokenflow}.

While proven effective, the vanilla autoregressive paradigm suffers from slow and rigid next-token prediction. To improve efficiency, recent studies explore more strategies, including multi-token prediction via random masking~\cite{chang2022maskgit,bai2024meissonic,xie2024show_o}, coarse-to-fine modeling~\cite{tian2024var,ma2024star,tang2024hart,han2024infinity}or hybrid approaches~\cite{he2025nar,yu2024rar}. Nonetheless, vector-quantized models still rely on sampling from token distributions, making generation quality sensitive to the sampling strategy.

\subsection{Sampling strategies in autoregressive models}
Transformers model the probability distribution over tokens, requiring specific sampling strategies to obtain concrete outputs. Common approaches in language modeling include top-\emph{k}~\cite{radford2019gpt2} and top-\emph{p}~\cite{holtzman2019top_p} sampling, which truncate the candidate space by rank or cumulative probability. EDT~\cite{zhang2024edt} dynamically adjusts temperature based on entropy to balance diversity and precision. Other approaches explore repetition penalties~\cite{keskar2019repetition_penalty}, contrastive decoding~\cite{chuang2023dola}, speculative decoding~\cite{leviathan2023speculative_decoding,chen2023accelerating}, and search-based techniques~\cite{meister2020beam_search,snell2024lookahead,guan2025rstar,lightman2023letsverifystepstep,snell2024scalingllmtts} to reduce hallucination or speed up inference.

In visual generation, a higher degree of randomness is often needed to produce more realistic and detailed content. LlamaGen~\cite{sun2024llamagen} and Lumina-mGPT~\cite{liu2024lumina_mgpt} demonstrate that much larger top-\emph{k} values than those used in language models help avoid over-smoothed and low-detail outputs. Recent methods~\cite{teng2024sjd,jang2025lantern} apply speculative~\cite{leviathan2023spec_decode} or parallel decoding~\cite{he2024zipar,wang2024par} to accelerate image synthesis. 
However, they overlook the highly uneven spatial information distribution in images compared to text, and do not tailor decoding for autoregressive image generation.
    \section{Problem Formulation}


Graphic layer decomposition is the task of decomposing a raster graphic design image $\bm{x} \in [0, 1]^{H \times W \times 3}$ into a sequence of layers $Y=(\bm{l}_k \in [0,1]^{H \times W \times 4})_{k=0}^{K}$.
Here, $H$ and $W$ represent the height and width of the image, respectively. $\bm{x}$ is an RGB image, and $\bm{l}_k$ is an RGBA image, with 3 and 4 channels, respectively.
$k$ represents the blending order of the layer, \ie, the z-index.
$\bm{l}_{k>0}$ is the foreground layer, and $\bm{l}_0$ is the background layer.

The layer sequence $Y$ is composited by the following recursive process from $k=1$ to $k=K$ ($\bm{x} = \bm{x}_{K}$):
\begin{align}
    \label{eq:rasterize}
    \bm{x}^{\text{C}}_{k} &= {\rm B}(\bm{l}_{k}, \bm{x}^{\text{C}}_{k-1}),
    \\ &= \bm{l}^{\text{C}}_{k} \odot \bm{l}^{\text{A}}_{k} + \bm{x}^{\text{C}}_{k-1} \odot (1 - \bm{l}^{\text{A}}_{k}).
\end{align}
Here, the superscript $\text{A}$ represents the alpha channel, and $\text{C}$ represents one of the RGB channels. ${\rm B}(\cdot)$ is the alpha blending function, $\odot$ is element-wise multiplication, and $\bm{x}_{k}$ is the $k$-th blended image.

In this study, we solve the inverse problem of the above, \ie, layer decomposition that estimates the layer sequence $Y$ from the raster image $\bm{x}$.
The granularity of the layer depends on the dataset, and in this study, we treat the human-made graphic designs in the dataset as ground truth.

    \section{LLM-based Multi-Agent Blackboard System}

This section introduces an alternative communication paradigm for LLM-based multi-agent systems inspired by blackboard systems \citep{10.1145/356810.356816}, distinct from the widely used master–slave architecture. As outlined in \textsection \ref{sec:introduction}, blackboard-based multi-agent systems provide several advantages over the master-slave approach. Here, rather than directly assigning tasks to sub-agents, the main agent posts its requests (i.e., sub-tasks for which it requires assistance) on a shared blackboard, which functions as a broadcast channel accessible to all other agents. Each helper agent independently evaluates whether it can respond to a request, considering its own capabilities, availability, cost, and other factors. If an agent decides to contribute, it writes its response to the corresponding request, and the main agent then decides whether to use or ignore the provided information. \textit{This way, all agents in the system retain full autonomy over their actions, and no centralized controller forces them to execute a specific task.} While the blackboard paradigm is applicable to a wide range of multi-agent systems, we focus on data science tasks that require data discovery, where its characteristics are particularly advantageous, as discussed in \textsection \ref{sec:introduction}. The remainder of this section details our method and its design for data science problems that require information discovery.



\paragraph{Overview:} 

An overview of our proposed method is presented in Figure~\ref{fig:overview}. The system $\pi_{s}$ operates over the data lake $\sD$ by first partitioning $\sD$ into $C$ clusters of related files. Each cluster $\sD_i$ is assigned to a file agent $\pi_{f_i}$, which is responsible for handling, loading, processing, and retrieving information from the files within its cluster. In addition, a search agent $\pi_{s}$ is included to retrieve external information from the web that may be required to solve the problem. The overall system $\pi_{s}$ is composed of a main agent $\pi_{m}$, which is responsible for solving the query $q$, and a set of $C+1$ helper agents $\Pi_{\text{helper}} = \{\pi_{f_i}\}_{i=1}^{M} \cup \{\pi_{s}\}$ that provide specialized assistance. The query $q$ is presented to $\pi_{m}$, which iteratively selects an action $a \in \sA$ from the action space $\sA$, executes the chosen action, and observes the resulting outcome from the environment. Among its actions, the main agent may interact with a blackboard $\beta$, a shared communication medium where it can post a request $r$ without addressing a specific sub-agent. The helper agents $\Pi_{\text{helper}}$ continuously monitor the blackboard, determine whether they can address a posted request, and, if so, provide their outputs on the corresponding response board $\beta_{r}$. These responses are then collected and made available to $\pi_{m}$, which incorporates them into its decision-making process.\footnote{Responses are not written back to the blackboard $\beta$ to avoid dependencies where one sub-agent's output could influence the behavior of others negatively. Instead, all responses are directed exclusively to the response board $\beta_{r}$, ensuring independent operation of sub-agents and exclusive access by the main agent $\pi_{m}$.} The main agent is limited to at most $T$ sequential actions (including actions that interact with the blackboard) to solve the query $q$, ultimately producing a program $p$ in python programming language that computes the final answer to $q$.

\paragraph{Clustering Data Lake:} 

There are multiple approaches for partitioning the data lake into clusters; applying clustering algorithms over file representations, random partitioning, or other heuristic methods. For simplicity, we do not utilize file content and instead rely solely on file names during clustering. Specifically, the file names are provided to an LLM---Gemini-2.5-Pro\footnote{Available at: \url{https://cloud.google.com/vertex-ai/generative-ai/docs/models/gemini/2-5-pro}}---which using the prompt shown in Figure~\ref{fig:clustering-prompt}, clusters the files into categories based only on their names.\footnote{This method represents just one simple possible approach to clustering, chosen for simplicity; more scalable and accurate alternatives could equally be employed in real world scenarios.} An example of this clustering is provided in Figure~\ref{fig:clustering-example} in Appendix~\ref{app:case-study}, where the model successfully groups related files together. For instance, it clusters all files originating from the National Interagency Fire Center into a category labeled ``NIFC Wildfire Statistics.'' The number of automatically derived clusters for each dataset is reported in Table~\ref{tab:stats} in Appendix~\ref{app:dataset}.



% The remainder of this section details the design of the main agent and the helper agents, emphasizing how their coordination supports effective information discovery in data science tasks.

\subsection{Main Agent}
\label{sec:main-agent}

The primary role of the main agent is to solve the problem in collaboration with the helper agents. The main agent follows the ReAct framework \citep{yao2023react}, where at each step $t$, given the query $q$ and the history of actions and observations $\sH_{t-1}$, it first reasons about what is the best next action and selects an action from a predefined action space, executes the action, observes the outcome, and appends the resulting observation to update the history $\sH_{t}$.\footnote{In this work, the inputs, outputs of the model, and observations are appended directly to the prompt of the LLM, formatted according to its chat-based input template.} The prompt used by the main agent is shown in Figure~\ref{fig:main-agent-blackboard-prompt} in Appendix~\ref{app:prompts}. The agent selects one of the following predefined actions in each step, executes them, and observe their outcomes:

\begin{itemize}[leftmargin=*]
    \item \textit{\textbf{Planning:}} In this action, the LLM decomposes the problem into smaller sub-problems and outlines a plan for addressing each of them. This action has no external effect on the environment but serves as an internal reasoning step to guide the LLM's problem-solving process. In response, the system simply acknowledges the proposed plan and instructs the LLM to proceed.
    
    \item \textit{\textbf{Reasoning:}} In this action, the LLM focuses on a specific aspect of the problem and explains its reasoning, analysis, or interpretation of the available observations and steps taken so far in this process. Similar to the planning step, this action has no external effect on the environment but functions as an internal reasoning mechanism to guide the LLM's problem-solving process. In response, the system simply acknowledges the reasoning and prompts the LLM to continue.
    
    \item \textit{\textbf{Executing Code:}} In this action, the agent generates python code, which is executed using a python interpreter. If the code runs successfully, the resulting outputs are returned to the agent for observation; otherwise, the agent receives the corresponding error messages. This action enables the agent to explore the problem interactively, inspect data files, and experiment with them to gain a deeper understanding of their content and structure and how to process them.
    
    \item \textit{\textbf{Requesting Help:}} In this action, the agent formulates a request for assistance from the sub-agents, specifying, for example, the types of data files or information needed, or the resources required to apply a tool or solve a sub-problem. This request is posted on the blackboard $\beta$ for visibility by the helper agents. Once the sub-agents respond, if they respond, their responses on the response board $\beta_r$ are collected and provided back to the main agent as the outcome of this action for observation and further use in its decision-making process.
    
    \item \textit{\textbf{Answering:}} In this action, the agent concludes the problem-solving process by generating a final program that produces the answer to the query. This action terminates the process, and the output of this step constitutes the final program $p$ generated by the system to address the problem.
\end{itemize}

\subsection{Helper Agents}
\label{sec:sub-agents}

In a data science, information discovery can typically be categorized into two tasks: (1) identifying the specific files that contain the data necessary to the problem, and (2) retrieving general knowledge about concepts relevant to the problem, such as domain-specific terms or details of particular algorithms and methods. To support these, our framework employs two types of helper agents:

\paragraph{File Agent:} 

Handling all the files in a data lake with a single agent is not feasible for several reasons: it typically involve a large number of files, many of which are lengthy and may exceed the agents context window; the files span diverse topics, which can confuse the agent and hinder effective reasoning; and accessing and processing all files simultaneously can be computationally expensive and inefficient, leading to unnecessary overhead and slower problem-solving. For these reasons, in our framework each file agent is assigned responsibility for a subset of data files determined to be relevant, as described earlier in the clustering procedure. In an offline phase, the file agent $\pi_{f_i}$ takes as input a subset of the data lake $\sD_{i}$ and operates through a two-step procedure. In the first step, the agent selects a subset\footnote{When filenames indicate multiple files containing the same type of data over different time periods, the agent does not need to inspect all of them to infer the structure; a small representative sample is sufficient.} (or all) of the files to examine their content. The contents of them are presented to the agent for inspection (details of presentation are in Appendix~\ref{app:implementation}). In the second step, after observing the selected files, the agent reasons about and analyzes them, learning how they are structured, what pre-processing or transformations may be required, and how they should be processed in general. An example of such an analysis is provided in Figure~\ref{fig:file-agent-analyze-example} in Appendix~\ref{app:case-study}. Then, in the online phase, the agent listens for requests from the main agent. Upon receiving a request, based on the analysis it did earlier, it determines whether it can contribute to answering it. If so, the agent generates a detailed plan specifying which files in $\sD_i$ are relevant, how they should be loaded in Python code, what libraries to use, the steps required for data processing, and samples from the data. The prompt used to guide the file agent is shown in Figure~\ref{fig:file-agent-prompt} in Appendix~\ref{app:prompts}. 

\paragraph{Search Agent:}

Certain data science problems require task-specific knowledge about algorithms or domain expertise that the LLM may not possess. To address this, we design a web-search agent that retrieves relevant information from a search engine. This agent operates according to the prompt shown in Figure~\ref{fig:search-agent-prompt} in Appendix~\ref{app:prompts}. Given a request $r$ posted on the blackboard $\beta$, the agent first determines whether it is capable of addressing the request. It is specifically restricted to general web-based information retrieval and does not respond to requests involving access to local files or datasets. If the agent determines that the request can be answered, it enters an iterative search process with a maximum of $T_{\text{search}} = 3$ steps. At each step $t$, the agent generates a set of queries $\sQ_{t}$, which are submitted to a search engine---in this work, Google Custom Search Engine\footnote{We use Google Custom Search Engine, configured to exclude all websites associated with the datasets used in this paper to prevent data leakage: \url{https://developers.google.com/custom-search}}---to retrieve $k=3$ webpage per query. The content of the webpages are then extracted using \textit{beautifulsoup} library\footnote{Available at: \url{https://pypi.org/project/beautifulsoup4/}} to be presented to the search agent. The extracted documents are then evaluated by the agent to determine whether they provide sufficient information to answer the request. If so, the agent generates a response to the request, which is posted to the response board $\beta_r$. If the information is insufficient, a new set of queries is generated to continue gathering relevant data from the web.

    \section{Decomposition Metrics}
\label{sec:metrics}
There are two problems in evaluating the quality of predicted layers $\hat{Y}$ against the ground truth $Y$.
First, the number of layers in the ground truth and the predicted layers can differ, making it non-trivial to compare directly.
To address this, we apply order-aware layer alignment using DTW~\citep{Mueller07_InformationRetrieval_SPRINGER}.
Second, the quality of the predicted layers can be evaluated from two perspectives: visual quality and granularity.
If we do not consider these aspects separately, we may underestimate the quality of the predicted layers due to differences in granularity, even if they are practically useful. 
We measure granularity by the number of edits required to align the two; we allow merging adjacent layers in the z-index and report both the number of edits and the visual quality after the editing operations.

\paragraph{Layer alignment}
\label{sec:layer_alignment}
As pre-processing, we first group the ground truth and predicted layers based on visibility.
Specifically, we extract layers whose visible regions (\ie, alpha values greater than zero) are not occluded by any other higher layers in z-index, blend them into a single layer, and repeat the same operation with the remaining layers.
This operation never affects the appearance of the composite image and forms what we refer to as a top-layer.

Next, we find alignment between the two layer sequences with different lengths using DTW, which considers the sequence order even if the lengths differ.
We obtain a set of pairs $P=\{(k_s,q_s)|s=0,1,\ldots,S\}$, where $k_s$ and $q_s$ represent the layer indices, and $S$ is the number of pairs.
Note that resolved pairs satisfy the monotonicity condition, \ie., $k_s$ and $q_s$ are increasing sequences; in other words, layers cannot be shuffled during alignment (see \suppref{sup:dtw}).
We define the distance metric for the layer pair as the sum of the negative value of the alpha's soft IoU and 
L1 distance of the RGB channels weighted by the ground-truth alpha, as introduced in \cite{suzuki2024fast}.

Finally, we compute the quality metric between the two layer sequences as follows:
\begin{align}
    \label{eq:metric}
    \mathcal{E}(\hat{Y}, Y) = \frac{1}{S}\sum_{s=0}^{S} e(\hat{\bm{l}}_{k_s}, \bm{l}_{q_s}),
\end{align}
where $e(\cdot)$ is an arbitrary function that measures the similarity or distance between layers. 
We use the weighted L1 distance of the RGB channels and the soft IoU of the alpha channel as $e(\cdot)$, similar to DTW's distance metric.



\paragraph{Layer merge}

Due to the ill-posed nature of layer decomposition, decomposition results sometimes do not align well with the ground truth. \TODO{Provide examples?}
In this work, we relax the alignment constraints by allowing \emph{edits}.
The idea is inspired by minimum edit distance~\citep{wagner1974string}, which is commonly used for string alignment. We define a specific edit operation set for layers, and report both the maximum number of allowed edits and the distance metric used in DTW after edits.
This gives a straightforward insight into how many layer-level edits are required for good alignment.

For simplicity, we define a single edit operation; \texttt{Merge}, which merges two consecutive layers in z-index, when the edit yields the highest positive distance improvement.
We apply edits iteratively until no further improvements are possible or the number of layers is reduced to 2.
The ground truth is also mergeable.
Visual examples of the edit process can be found in \suppref{fig:sup-edit-process-1,fig:sup-edit-process-2}.
















    \section{Experiments}

\subsection{Experimental Setup}
\label{sec:exp-setup}

\paragraph{Benchmarks:}


To the best of our knowledge, KramaBench is the only public benchmark for data science problems that explicitly incorporates a data discovery phase, which we adopt in our evaluation. In addition, we repurpose two existing datasets, DS-Bench \citep{jing2025dsbench} and DA-Code \citep{huang-etal-2024-da}, to include in this phase. Specifically, we manually filtered out all questions that do not require any data file for answering, as well as those that lack sufficient hints for data discovery.\footnote{For example, questions that request the computation of a data science metric on a column without specifying the structure or content of the relevant file.} After filtering, we aggregated all remaining files across questions into a unified data lake, such that the model must perform discovery to identify relevant files at inference time. In this setup, only the question and the data lake are provided to the model, requiring it to identify the relevant files to answer the question, following the same protocol as KramaBench.
% \footnote{The constructed datasets will be released publicly upon acceptance of the paper.} 
Further details on this filtering process, along with dataset statistics in Table~\ref{tab:stats}, are provided in Appendix~\ref{app:dataset}.


\paragraph{Evaluation:}

To evaluate the generated programs, we execute each and compare its output against the ground-truth reference for the corresponding question. For each dataset, we adopt its standard evaluation protocol. For KramaBench, we use the official evaluation script provided in its repository.\footnote{Available at: \url{https://github.com/mitdbg/KramaBench}} For DA-Code, we likewise rely on the official evaluation script released by its authors.\footnote{Available at: \url{https://github.com/yiyihum/da-code}} For DS-Bench, we use the original evaluation method, in which an LLM serves as the judge. The generated programs output is compared against the reference answer using Gemini-2.5-Pro as the judge LLM, with the evaluation prompt shown in Figure \ref{fig:eval-ds-bench} in Appendix \ref{app:dataset}, producing a binary score.

\paragraph{Inference Setup:} We set the maximum actions of the main agent to $T = 10$. We use nucleus sampling \citep{Holtzman2020The} with a temperature of $0.1$ for more deterministic inference and default value for other hyperparameters. Proprietary models are accessed via Vertex AI,\footnote{Available at: \url{https://cloud.google.com/vertex-ai?hl=en}} while open-source models are served by vLLM.\footnote{Available at: \url{https://docs.vllm.ai/en/latest/}} At each step, we cap the number of generated tokens at 8,192. We use Gemini-2.5-Pro and -Flash \citep{comanici2025gemini25pushingfrontier}, and Claude-4-Opus \citep{anthropic2025claude4} as the proprietary and Qwen3-Coder\footnote{Available at: \url{https://huggingface.co/Qwen/Qwen3-Coder-30B-A3B-Instruct}} with 30 billion parameters \citep{qwen3technicalreport} as the open-source LLMs. Experiments are conducted on 2 NVIDIA A100 (each with 80GB VRAM) GPUs.

% \begin{table}[]
%     \centering
%     \caption{Results of our method and the baselines on the KramaBench, DS-Bench, and DA-Code benchmarks. The best results are highlighted in \textbf{bold}.}
%     \label{tab:main-result}
%     \adjustbox{max width=\textwidth}{
%     \begin{tabular}{ll|l|cccccc|c|c|c}
%         \toprule
%         \multirow{2}{*}{\textbf{Method}} & & \multirow{2}{*}{\textbf{LLM}} & \multicolumn{6}{|c|}{\textbf{KramaBench}} & \multirow{2}{*}{\textbf{DS-Bench}} & \multirow{2}{*}{\textbf{DA-Code}} & {\textbf{Average}} \\
%         \cmidrule{4-9}
%         & & & Archaeology & Astronomy & Biomedical & Environment & Legal & Wildfire & & & (macro) \\
%         \midrule
        
%         \multirow{4}{*}{DS-GRU} & (1) & Qwen3-Coder & 0.00\% & 1.80\% & 2.11\% & 1.15\% & 3.27\% & 13.54\% & 0.00\% & 0.00\% & 2.73\% \\
        
%         & (2) & Gemini 2.5 Flash & 0.00\% & 7.83\% & 0.09\% & 10.93\% & 12.46\% & 13.34\% & 5.53\% & 0.00\% & 6.38\% \\
        
%         & (3) & Gemini 2.5 Pro & 25.00\% & 6.69\% & 10.64\% & 27.47\% & 5.94\% & 39.36\% & 3.95\% & 0.00\% & 14.88\% \\
%         & (4) & Claude 4 Opus & 8.33\% & 1.38\% & 1.90\% & 8.14\% & 9.80\% & 23.14\% & 3.55\% & 0.00\% & 7.03\% \\
%         \midrule
        
%         \multirow{4}{*}{RAG} & (5) & Qwen3-Coder & 0.00\% & 3.16\% & 4.99\% & 0.54\% & 6.19\% & 16.93\% & 6.32\% & 0.00\% & 4.76\% \\
        
%         & (6) & Gemini 2.5 Flash & 16.66\% & 3.57\% & 13.98\% & 28.57\% & 10.97\% & 33.67\% & 22.92\% & 2.75\% & 17.44\% \\
        
%         & (7) & Gemini 2.5 Pro & \textbf{33.33\%} & 8.47\% & 32.53\% & 31.36\% & 25.55\% & 38.32\% & 27.27\% & 0.00\% & 25.32\% \\
        
%         & (8) & Claude 4 Opus & \textbf{33.33\%} & 11.52\% & 23.42\% & 31.61\% & 31.80\% & 45.80\% & 35.57\% & 3.85\% & 27.11\% \\
%         \midrule
        
%         \multirow{4}{*}{Master-Slave} & (9) & Qwen3-Coder & 0.00\% & 3.55\% & 3.39\% & 7.77\% & 8.90\% & 21.79\% & 7.55\% & 0.00\% & 6.61\% \\
        
%         & (10) & Gemini 2.5 Flash & 16.66\% & 3.16\% & 13.98\% & 17.46\% & 21.75\% & 25.80\% & 26.48\% & 0.55\% & 14.87\% \\
        
        
%         & (11) & Gemini 2.5 Pro & \textbf{33.33\%} & 8.47\% & 24.74\% & 32.81\% & 34.64\% & 58.98\% & 34.38\% & 5.49\% & 29.10\% \\
        
%         & (12) & Claude 4 Opus & \textbf{33.33\%} & 8.69\% & 32.28\% & 39.16\% & \textbf{44.08\%} & 48.35\% & 45.84\% & 2.75\% & 31.81\% \\
%         \midrule
        
%         \multirow{4}{*}{Blackboard} & (13) & Qwen3-Coder & 0.00\% & 7.69\% & 7.85\% & 4.47\% & 6.36\% & 23.97\% & 14.22\% & 1.11\% & 8.47\% \\
        
%         & (14) & Gemini 2.5 Flash & 16.66\% & 3.57\% & 14.78\% & 22.92\% & 27.09\% & 41.04\% & 28.06\% & 0.55\% & 18.22\% \\
        
%         & (15) & Gemini 2.5 Pro & \textbf{33.33\%} & 17.95\% & 36.83\% & \textbf{39.31\%} & 34.92\% & \textbf{62.88\%} & 38.73\% & \textbf{9.34\%} & 34.16\% \\
        
%         & (16) & Claude 4 Opus & \textbf{\textbf{33.33\%}} & \textbf{18.69\%} & \textbf{45.31\%} & 34.35\% & 42.48\% & 50.06\% & \textbf{49.80\%} & 7.14\% & \textbf{35.14\%} \\
%         \bottomrule
%     \end{tabular}}
% \end{table}



\paragraph{Baselines:} To evaluate our method against alternative approaches for solving data science problems involving data discovery, we compare it with the following baselines:
\begin{itemize}[leftmargin=*]
    \item \textit{\textbf{DS-GRU:}} We adopt the only existing baseline (to the best of our knowledge) for data discovery in data science problems, which appends all available files directly into the LLM prompt and attempts to solve the problem \citep{lai2025kramabenchbenchmarkaisystems}. This baseline uses a self-correction loop that retries when errors occur in generated codes. For details, we refer the reader to \citet{lai2025kramabenchbenchmarkaisystems}.
    
    \item \textit{\textbf{Retrieval-Augmented Generation (RAG):}} This retrieves the top 5 files\footnote{This number is chosen based on the average number of files required to solve the problems (1.6) and the length of the context window of the backbone LLMs used in this paper.} based on the file names and contents (the method for presenting a file content to the LLM is explained in Appendix~\ref{app:implementation}) from the data lake using E5-large\footnote{Available at: \url{https://huggingface.co/intfloat/e5-large-v2}} \citep{wang2022text}, a 330M-parameter embedding model and use it to solve the problem. It then follows the same procedure as the main agent described in Section~\ref{sec:main-agent}, with two key modification: 1) the retrieved files contents and addresses are presented directly to the LLM in the prompt and 2) the general help-request action is replaced with a restricted action that only allows direct requests to the search agent. This design isolates the effect of substituting the file discovery mechanism with RAG, enabling a controlled study of its impact on performance. The prompt used for this baseline is shown in Figure~\ref{fig:main-agent-rag-prompt} in Appendix~\ref{app:prompts}.
    
    \item \textit{\textbf{Master-Slave:}} This baseline follows the same procedure as the main agent described in Section~\ref{sec:main-agent}. The key difference is that, instead of posting requests on the blackboard, the agent directly invokes sub-agents (consisting of the search agent and the file agents as explained in Section~\ref{sec:sub-agents}) based on their description by referencing their names and assign task to them. The prompt used for this baseline is shown in Figure~\ref{fig:main-agent-master-slave-prompt} in Appendix~\ref{app:prompts}.
\end{itemize} 


% \begin{table}[]
%     \centering
%     \caption{Results of our method and the baselines on the KramaBench, DS-Bench, and DA-Code benchmarks. The best results for each backbone LLMs are highlighted in \textbf{bold}.}
%     \label{tab:main-result}
%     \adjustbox{max width=\textwidth}{
%     \begin{tabular}{ll|c|cccccc|c|c|c|c}
%         \toprule
%         & \multirow{2}{*}{\textbf{Method}}& \multirow{2}{*}{\textbf{LLM}} & \multicolumn{7}{|c|}{\textbf{KramaBench}} & \multirow{2}{*}{\textbf{DS-Bench}} & \multirow{2}{*}{\textbf{DA-Code}} & {\textbf{Average}} \\
%         \cmidrule{4-10}
%         & & & Archaeology & Astronomy & Biomedical & Environment & Legal & Wildfire & Average & & & (macro) \\
%         \midrule
        
%         (1) & {DS-GRU} & \multirow{4}{*}{Qwen3-Coder} & \textbf{0.00\%} & 1.80\% & 2.11\% & 1.15\% & 3.27\% & 13.54\% & 3.64\% & 0.00\% & 0.00\% & 1.21\% \\
        
%         (2) & {RAG} &  & \textbf{0.00\%} & 3.16\% & 4.99\% & 0.54\% & 6.19\% & 16.93\% & 5.30\% & 6.32\% & 0.00\% & 3.87\% \\
        
%         (3) & {Master-Slave} &  & \textbf{0.00\%} & 3.55\% & 3.39\% & \textbf{7.77\%} & \textbf{8.90\%} & 21.79\% & 7.56\% & 7.55\% & 0.00\% & 5.03\% \\
        
%         \cmidrule{1-2} \cmidrule{4-13}
        
%         (4) & {Blackboard} & & \textbf{0.00\%} & \textbf{7.69\%} & \textbf{7.85\%} & 4.47\% & 6.36\% & \textbf{23.97\%} & \textbf{8.39\%} & \textbf{14.22\%} & \textbf{1.11\%} & \textbf{7.90\%} \\
        
%         \midrule
%         \midrule
        
%         (5) & {DS-GRU} & \multirow{4}{*}{Gemini 2.5 Flash} & 0.00\% & \textbf{7.83\%} & 0.09\% & 10.93\% & 12.46\% & 13.34\% & 7.44\% & 5.53\% & 0.00\% & 4.32\% \\
        
%         (6) & {RAG} & & \textbf{16.66\%} & 3.57\% & 13.98\% & \textbf{28.57\%} & 10.97\% & 33.67\% & 17.90\% & 22.92\% & \textbf{2.75\%} & 14.52\% \\
        
%         (7) & {Master-Slave} & & \textbf{16.66\%} & 3.16\% & 13.98\% & 17.46\% & 21.75\% & 25.80\% & 16.46\% & 26.48\% & 0.55\% & 14.49\% \\
        
%         \cmidrule{1-2} \cmidrule{4-13}
        
%         (8) & {Blackboard} & & \textbf{16.66\%} & 3.57\% & \textbf{14.78\%} & 22.92\% & \textbf{27.09\%} & \textbf{41.04\%} & \textbf{21.01\%} & \textbf{28.06\%} & 0.55\% & \textbf{16.54\%} \\
        
%         \midrule
%         \midrule
        
%         (9) & {DS-GRU} & \multirow{4}{*}{Gemini 2.5 Pro} & 25.00\% & 6.69\% & 10.64\% & 27.47\% & 5.94\% & 39.36\% & 19.18\% & 3.95\% & 0.00\% & 7.71\% \\
        
%         (10) & {RAG} & & \textbf{33.33\%} & 8.47\% & 32.53\% & 31.36\% & 25.55\% & 38.32\% & 28.26\% & 27.27\% & 0.00\% & 18.51\% \\
        
%         (11) & {Master-Slave} & & \textbf{33.33\%} & 8.47\% & 24.74\% & 32.81\% & 34.64\% & 58.98\% & 32.16\% & 34.38\% & 5.49\% & 24.01\% \\
        
%         \cmidrule{1-2} \cmidrule{4-13}
        
%         (12) & {Blackboard} & & \textbf{33.33\%} & \textbf{17.95\%} & \textbf{36.83\%} & \textbf{39.31\%} & \textbf{34.92\%} & \textbf{62.88\%} & \textbf{37.53\%} & \textbf{38.73\%} & \textbf{9.34\%} & \textbf{28.53\%} \\
        
%         \midrule
%         \midrule
        
%         (13) & {DS-GRU} & \multirow{4}{*}{Claude 4 Opus} & 8.33\% & 1.38\% & 1.90\% & 8.14\% & 9.80\% & 23.14\% & 8.78\% & 3.55\% & 0.00\% & 4.11\% \\
        
        
%         (14) & {RAG} &  & \textbf{33.33\%} & 11.52\% & 23.42\% & 31.61\% & 31.80\% & 45.80\% & 29.58\% & 35.57\% & 3.85\% & 23.00\% \\


%         (15) & {Master-Slave} & & \textbf{33.33\%} & 8.69\% & 32.28\% & \textbf{39.16\%} & \textbf{44.08\%} & 48.35\% & 34.31\% & 45.84\% & 2.75\% & 27.63\% \\

%         \cmidrule{1-2} \cmidrule{4-13}

%         (16) & {Blackboard} &  & \textbf{\textbf{33.33\%}} & \textbf{18.69\%} & \textbf{45.31\%} & 34.35\% & 42.48\% & \textbf{50.06\%} & \textbf{37.37\%} & \textbf{49.80\%} & \textbf{7.14\%} & \textbf{31.43\%} \\
%         \bottomrule
%     \end{tabular}}
% \end{table}

\begin{table*}
    \centering
    \caption{Results on the KramaBench, DS-Bench, and DA-Code benchmarks. The best results for each LLM are highlighted in \textbf{bold}. The KramaBench categories are abbreviated: Arc. (Archaeology), Ast. (Astronomy), Bio. (Biomedical), Env. (Environment), Leg. (Legal), and Wild. (Wildfire).}
    \label{tab:main-result}
    \adjustbox{max width=\textwidth}{
    \begin{tabular}{ll c cccccc c c c c}
        \toprule
        & \multirow{2}{*}{\textbf{Method}}& \multirow{2}{*}{\textbf{LLM}} & \multicolumn{7}{c}{\multirow{1}{*}{\textbf{KramaBench}}} & \multirow{2}{*}{\textbf{\makecell{DS-\\Bench}}} & \multirow{2}{*}{\textbf{\makecell{DA-\\Code}}} & \multirow{2}{*}{\textbf{\makecell{Average \\ (macro)}}} \\
        \cmidrule{4-10}
        & & & Arc. & Ast. & Bio. & Env. & Leg. & Wild. & Average & & & \\
        \midrule
        
        (1) & {DS-GRU} & \multirow{4}{*}{\makecell{Qwen3-\\Coder}} & \textbf{0.00\%} & 1.80\% & 2.11\% & 1.15\% & 3.27\% & 13.54\% & 3.64\% & 0.00\% & 0.00\% & 1.21\% \\
        
        (2) & {RAG} &  & \textbf{0.00\%} & 3.16\% & 4.99\% & 0.54\% & 6.19\% & 16.93\% & 5.30\% & 6.32\% & 0.00\% & 3.87\% \\
        
        (3) & {Master-Slave} &  & \textbf{0.00\%} & 3.55\% & 3.39\% & \textbf{7.77\%} & \textbf{8.90\%} & 21.79\% & 7.56\% & 7.55\% & 0.00\% & 5.03\% \\
        
        \cmidrule{1-2} \cmidrule{4-13}
        
        (4) & {Blackboard} & & \textbf{0.00\%} & \textbf{7.69\%} & \textbf{7.85\%} & 4.47\% & 6.36\% & \textbf{23.97\%} & \textbf{8.39\%} & \textbf{14.22\%} & \textbf{1.11\%} & \textbf{7.90\%} \\
        
        \midrule
        \midrule
        
        (5) & {DS-GRU} & \multirow{4}{*}{\makecell{Gemini 2.5\\Flash}} & 0.00\% & \textbf{7.83\%} & 0.09\% & 10.93\% & 12.46\% & 13.34\% & 7.44\% & 5.53\% & 0.00\% & 4.32\% \\
        
        (6) & {RAG} & & \textbf{16.66\%} & 3.57\% & 13.98\% & \textbf{28.57\%} & 10.97\% & 33.67\% & 17.90\% & 22.92\% & \textbf{2.75\%} & 14.52\% \\
        
        (7) & {Master-Slave} & & \textbf{16.66\%} & 3.16\% & 13.98\% & 17.46\% & 21.75\% & 25.80\% & 16.46\% & 26.48\% & 0.55\% & 14.49\% \\
        
        \cmidrule{1-2} \cmidrule{4-13}
        
        (8) & {Blackboard} & & \textbf{16.66\%} & 3.57\% & \textbf{14.78\%} & 22.92\% & \textbf{27.09\%} & \textbf{41.04\%} & \textbf{21.01\%} & \textbf{28.06\%} & 0.55\% & \textbf{16.54\%} \\
        
        \midrule
        \midrule
        
        (9) & {DS-GRU} & \multirow{4}{*}{\makecell{Gemini 2.5\\Pro}} & 25.00\% & 6.69\% & 10.64\% & 27.47\% & 5.94\% & 39.36\% & 19.18\% & 3.95\% & 0.00\% & 7.71\% \\
        
        (10) & {RAG} & & \textbf{33.33\%} & 8.47\% & 32.53\% & 31.36\% & 25.55\% & 38.32\% & 28.26\% & 27.27\% & 0.00\% & 18.51\% \\
        
        (11) & {Master-Slave} & & \textbf{33.33\%} & 8.47\% & 24.74\% & 32.81\% & 34.64\% & 58.98\% & 32.16\% & 34.38\% & 5.49\% & 24.01\% \\
        
        \cmidrule{1-2} \cmidrule{4-13}
        
        (12) & {Blackboard} & & \textbf{33.33\%} & \textbf{17.95\%} & \textbf{36.83\%} & \textbf{39.31\%} & \textbf{34.92\%} & \textbf{62.88\%} & \textbf{37.53\%} & \textbf{38.73\%} & \textbf{9.34\%} & \textbf{28.53\%} \\
        
        \midrule
        \midrule
        
        (13) & {DS-GRU} & \multirow{4}{*}{\makecell{Claude 4\\Opus}} & 8.33\% & 1.38\% & 1.90\% & 8.14\% & 9.80\% & 23.14\% & 8.78\% & 3.55\% & 0.00\% & 4.11\% \\
        
        
        (14) & {RAG} &  & \textbf{33.33\%} & 11.52\% & 23.42\% & 31.61\% & 31.80\% & 45.80\% & 29.58\% & 35.57\% & 3.85\% & 23.00\% \\


        (15) & {Master-Slave} & & \textbf{33.33\%} & 8.69\% & 32.28\% & \textbf{39.16\%} & \textbf{44.08\%} & 48.35\% & 34.31\% & 45.84\% & 2.75\% & 27.63\% \\

        \cmidrule{1-2} \cmidrule{4-13}

        (16) & {Blackboard} &  & \textbf{33.33\%} & \textbf{18.69\%} & \textbf{45.31\%} & 34.35\% & 42.48\% & \textbf{50.06\%} & \textbf{37.37\%} & \textbf{49.80\%} & \textbf{7.14\%} & \textbf{31.43\%} \\
        \bottomrule
    \end{tabular}}
    % \vspace{-0.4cm}
\end{table*}


\subsection{Empirical Findings}

\paragraph{Main Results:} 


We conduct our experiments on the datasets described in Section~\ref{sec:exp-setup} using our method and the baselines. The results are presented in Table~\ref{tab:main-result}. These results demonstrate that our method, the Blackboard System, outperforms all baselines on average across all the datasets. Specifically, the Blackboard System surpasses the DS-GRU, RAG and Master-Slave approaches on all three datasets and achieves similar or higher performance in 4 out of 6 categories on KramaBench. Furthermore, we observe that the Blackboard System consistently outperforms the baselines regardless of the backbone LLM, highlighting its robustness and generalizability. We attribute this improvement to the design of the Blackboard System, where tasks are not explicitly assigned to helper agents; instead, each agent autonomously decides whether to participate based on its capabilities. This self-selection enhances both problem-solving efficiency and data discovery performance.




\paragraph{File Discovery Performance:}

To analyze the effectiveness of different methods in data discovery, we report recall, precision, and F1-score for the file discovery task, i.e., identifying the correct files required to answer each question. The results of this experiment, using Gemini 2.5 Pro as the backbone LLM, are presented in Table~\ref{tab:result-file-discovery}. The results in this table indicate that the blackboard system achieves the highest recall, precision, and F1-score compared to all baselines, both on average and across the three datasets. In particular, for KramaBench, the blackboard system attains the highest F1-score in 4 out of 6 domains. We attribute this improvement to the design of the blackboard system, where the main agent does not directly assign requests to specific file agents, as in the master–slave setup. Instead, each file agent independently decides whether it can contribute based on its capabilities and data holdings, leading to more accurate and comprehensive file discovery.




\paragraph{Effect of Web Search (Search Agent) on the Performance:}

We observed that in some cases the backbone LLM lacks the necessary domain-specific knowledge or familiarity with specialized algorithms to fully understand and solve the problem. To address this limitation, the inclusion of a search agent that can retrieve relevant external information may be beneficial. To evaluate this, we compare the blackboard system with and without the search agent. The results on KramaBench, shown in Figure~\ref{fig:search-wo-search} using Gemini 2.5 Pro as the backbone LLM, demonstrate that incorporating the search agent improves the average performance of the blackboard system. Further analysis reveals that when the main agent encounters unfamiliar concepts, it issues requests to obtain such information from the web. In these cases, the search agent typically responds by retrieving the required knowledge, thereby enabling the main agent to continue solving the problem effectively. Illustrative examples of this behavior are provided in Figures~\ref{fig:search-example-1} and \ref{fig:search-example-2} in Appendix~\ref{app:case-study}, highlighting the importance of the search agent in scenarios where external domain knowledge is essential.

\begin{table}[]
    \centering
    \caption{File discovery performance reported using recall, precision, and f1-score. The results are obtained using Gemini 2.5 Pro as the backbone LLM. The best results are highlighted in \textbf{bold}.}
    \label{tab:result-file-discovery}
    \adjustbox{max width=\textwidth}{
    \begin{tabular}{ll l cccccc c c c c}
        \toprule
        &\multirow{2}{*}{\textbf{Method}} & \multirow{2}{*}{\textbf{Metric}} & \multicolumn{7}{c}{\textbf{KramaBench}} & \multirow{2}{*}{\textbf{DS-Bench}} & \multirow{2}{*}{\textbf{DA-Code}} & {\textbf{Average}} \\
        \cmidrule{4-10}
        & & & Archaeology & Astronomy & Biomedical & Environment & Legal & Wildfire & Average & & & (macro) \\
        \midrule
        \multirow{3}{*}{(1)} & \multirow{3}{*}{RAG } & recall & 0.875 & 0.125 & \textbf{0.666} & 0.3506 & 0.127 & 0.238 & 0.396 & 0.035 & 0.257 & 0.229 \\
        & & precision & \textbf{1.000} & 0.125 & 0.666 & 0.450 & 0.133 & 0.452 & 0.471 & 0.047 & 0.456 & 0.324  \\
        & & F1 & 0.916 & 0.125 & 0.629 & 0.332 & 0.105 & 0.301 & 0.401 & 0.034 & 0.307 & 0.247 \\
        \midrule
        \multirow{3}{*}{(2)} & \multirow{3}{*}{Master-Slave} & recall & \textbf{0.916} & 0.5138 & 0.648 & {0.382} & \textbf{0.444} & \textbf{0.567} & 0.578 & 0.323 & 0.546 & 0.482 \\
        & & precision & 0.930 & \textbf{0.750} & \textbf{0.722} & {0.500} & \textbf{0.494} & \textbf{0.642} & 0.673 & 0.503 & 0.767 & 0.647 \\
        & & F1 & 0.913 & 0.577 & \textbf{0.674} & 0.389 & \textbf{0.450} & \textbf{0.576} & 0.596 & 0.358 & 0.584 & 0.513 \\
        \midrule
        \multirow{3}{*}{(3)} & \multirow{3}{*}{Blackboard} & recall & \textbf{0.916} & \textbf{0.576} & 0.648 & \textbf{0.604} & 0.383 & 0.464 & \textbf{0.598} & \textbf{0.402} & \textbf{0.600} & \textbf{0.533} \\
        & & precision & \textbf{1.000} & 0.733 & \textbf{0.722} & \textbf{0.703} & 0.302 & 0.603 & \textbf{0.677} & \textbf{0.584} & \textbf{0.837} & \textbf{0.699} \\
        & & F1 & \textbf{0.944} & \textbf{0.618} & \textbf{0.674} & \textbf{0.588} & 0.304 & 0.495 & \textbf{0.603} & \textbf{0.438} & \textbf{0.643} & \textbf{0.561} \\
        \bottomrule
    \end{tabular}}
\end{table}


\begin{figure}
    \centering
    \includegraphics[width=\textwidth]{figs/search_without_search_compare.png}
    % \vspace{-0.4cm}
    % \centering
    \caption{Performance of Blackboard System w/ and w/o search agent (Gemini 2.5 Pro).}
    % \vspace{-0.4cm}
    \label{fig:search-wo-search}
\end{figure}


\paragraph{Effect of Number of Main Agent's Actions on the Performance:}

To examine the impact of the maximum number of actions available to the main agent, we vary this parameter across ${2, 4, 6, 8, 10}$ and evaluate the blackboard system on KramaBench using Gemini 2.5 Pro as the backbone LLM. The results, presented in Figure~\ref{fig:num-actions}, indicate that increasing the action budget consistently improves the average performance of the system. This trend aligns with intuition: a larger exploration budget allows the agent to more thoroughly analyze the problem, consider alternative strategies, better investigate the solution space, and generate a better program that answers the question.

\begin{figure}
    \centering
    \includegraphics[width=\textwidth]{figs/num_actions_performance.png}
    % \vspace{-0.4cm}
    \caption{Performance of Blackboard System with various maximum actions by the main agent.}
    \label{fig:num-actions}
    % \vspace{-0.5cm}
\end{figure}

\paragraph{Case Studies:}

To qualitatively analyze the blackboard system---specifically how it formulates requests and how this process improves the generated program---we present several case studies:
\begin{itemize}[leftmargin=*]

\item \textbf{Writing Request on the blackboard:} An example of a request posted by on the blackboard is shown in Figure~\ref{fig:request-example} in Appendix~\ref{app:case-study}. In this case, the main agent, given the data science question, formulates a request that specifies the likely column names and data formats needed to solve the problem, along with some guidance for interpretation. In response, several helper agents (3 out of 8 in this example) chose to contribute. Although the relevant files were distributed across different clusters managed by different file agents, each responding agent independently provided the file addresses, code snippets for loading the data, and explanations of the data structure along with suggested preprocessing steps. Collectively, these responses covered all the ground-truth files required to answer the question. This case study demonstrates how the main agent can effectively leverage the blackboard mechanism to discover and integrate necessary information.

\item \textbf{Comparing Generated Program by Blackboard System with Master-Slave System:} To study this further, we present an example of programs generated by the Blackboard system and the Master–Slave system in Figure~\ref{fig:program-example} in Appendix~\ref{app:case-study}. In this case, the Blackboard agent achieved a better solution because it accurately interpreted the prompt and selected the correct data files. Specifically, it identified that the patients \texttt{Age} was located in the \texttt{mmc1.xlsx} file and, more importantly, that the requested \texttt{APP-Z score} was in the \texttt{mmc7.xlsx} file. In contrast, the Master–Slave agent misinterpreted the request and instead used a general protein abundance score (\texttt{APP\_log2\_abundance}) from the wrong file, \texttt{mmc2.xlsx}. This critical error in data selection led the Master–Slave agent to produce an incorrect result of \texttt{74}, while the Blackboard agents precise data discovery and reasoning yielded the correct answer of \texttt{60}.

\end{itemize}
    \section{Conclusion}

In this work, we presented a full-stack investigation of LLM unlearning, encompassing methodology, evaluation, and robustness. We established a principled taxonomy that organizes twelve representative unlearning methods into three families: {\MDiv}, {\MRep}, and {\MRej}, providing a systematic lens to understand their underlying mechanisms. Our analysis revealed that conventional multiple-choice questioning (MCQ) evaluations of unlearning effectiveness (UE) and utility retention (UT) offer an incomplete picture, and we introduced open question answering (Open-QA) as a complementary paradigm to better capture generative behaviors and expose the strengths and limitations of different methods. Furthermore, we provide a comprehensive robustness assessment across model-level and input-level attacks, revealing nuanced relationships among in-domain relearning, out-of-domain fine-tuning, quantization, and jailbreak attacks. These findings clarify the trade-offs of current unlearning algorithms and guide the design of future methods that are both effective and robust. The use of LLM, limitation and broader impact are further discussed in \textbf{Appendix\,\ref{appx:llm_usage}}, \textbf{Appendix\,\ref{appx:limit}} and \textbf{Appendix\,\ref{appx:impact}}.

    { \small \bibliographystyle{ieeenat_fullname} \bibliography{main} }

    \clearpage
\setcounter{figure}{0}
\renewcommand{\thefigure}{\Alph{figure}}
\renewcommand{\thetable}{\Alph{table}}
\renewcommand{\thealgorithm}{\Alph{algorithm}}
\maketitlesupplementary
\appendix

\section{Editing Examples}
\label{sup:application}
We show editing examples in \cref{fig:application-examples}.
Here, we use \ours{} to decompose the input image into layers, divide each layer into connected components, and group text components using CRAFT~\cite{baek2019character} to facilitate editing.
We import the layers into PowerPoint\footnote{\url{https://www.microsoft.com/powerpoint}} and perform various edits, from simple layout manipulation to applying built-in image effects, \emph{at the layer level}.
As the examples show, once the images are decomposed, users can intuitively edit them with precise control over each graphic element.

\section{Additional Results}
\label{sup:results}
We present additional examples of decomposed graphic design images using our method in \cref{fig:sup-qualitative-1,fig:sup-qualitative-2}.
These examples are selected from the Crello~\cite{yamaguchi2021canvasvae} test set and demonstrate the effectiveness of our method across diverse design styles.


\section{Failure Cases}
\label{sup:failure}
In \cref{fig:sup-failure-1,fig:sup-failure-2}, we show typical failure cases of our method.
The first set of failure cases (\cref{fig:sup-failure-1}) involves objects that are too small, such as detailed text descriptions, which are challenging to decompose due to their limited spatial extent.
We believe that these can be mitigated by increasing the resolution of the input images.
The second set of failure cases (\cref{fig:sup-failure-2}) is due to the ambiguity of the layer granularity.
For these samples, it is difficult even for humans to decompose them into the same layers consistently.
Although our evaluation metrics account for such ambiguity, we may need to improve training objectives or the post-refinement process to address these cases.

\section{User Study}
\label{sup:userstudy}
We conduct a user study in which 21 cloudworkers experienced in layer-based image editing rate the practical utility of 50 decomposition results---randomly ordered and anonymized---from \ours{} and our two baselines on the same images using a five-point scale.
\cref{tab:user-study} summarizes the results of the user study.
\ours{} achieves the highest average score, and a significant majority of the users (71.4\%) rate \ours{} the highest average score.
This result further emphasizes the practical superiority of our method.

\begin{table}[h]
    \centering
    \caption{
        Results of the user study. We report the average score, the number of users who rate each method as the best on average across all samples (\#Pref. users), and the number of samples for which each method is rated the best on average across all users (\#Win samples).
    }
    \begin{tabular}{lccc}
        \toprule
        & Score & \#Pref. users & \#Win samples\\
        \midrule
        LayerD & \bf{3.74} & \bf{15 (71.4\%)} & \bf{27 (54.0\%)} \\
        YOLO base & 3.52 & 2 (9.5\%) & 15 (30.0\%) \\
        VLM base & 3.31 & 4 (19.0\%) & 8 (16.0\%) \\
        \bottomrule
    \end{tabular}
    \label{tab:user-study}
\end{table}

\begin{figure*}[h]
    \centering
    \includegraphics[keepaspectratio, width=0.89\linewidth]{figures/application-examples-vis.pdf}
    \caption{Editing examples on Crello~\cite{yamaguchi2021canvasvae} test set.
    The leftmost images are the original images, and the remaining images are edited ones based on the decomposed layers. We use \ours{} to decompose the original images into layers, divide them into connected components, and group text components using CRAFT~\cite{baek2019character}.
    Then, we perform various \emph{layer-level} edits, from simple layout changes to applying built-in image effects, on PowerPoint.}
    \label{fig:application-examples}
\end{figure*}



\begin{figure*}[h]
    \centering
    \includegraphics[keepaspectratio, width=0.85\linewidth]{figures/additional-samples-1.pdf}
    \caption{Additional qualitative results of our method on Crello~\cite{yamaguchi2021canvasvae} test set.
    The leftmost column shows the input image, and the remaining columns show the decomposed layers from back to front.
    }
    \label{fig:sup-qualitative-1}
\end{figure*}

\begin{figure*}[h]
    \centering
    \includegraphics[keepaspectratio, width=0.85\linewidth]{figures/additional-samples-2.pdf}
    \caption{Additional qualitative results of our method on Crello~\cite{yamaguchi2021canvasvae} test set.
    The leftmost column shows the input image, and the remaining columns show the decomposed layers from back to front.
    }
    \label{fig:sup-qualitative-2}
\end{figure*}


\begin{figure*}[h]
    \centering
    \includegraphics[keepaspectratio, width=0.82\linewidth]{figures/failure-samples-small-object.pdf}
    \caption{Failure samples for too small objects on Crello~\cite{yamaguchi2021canvasvae} test set.
    The leftmost column shows the input image, and the remaining columns show the decomposed layers from back to front.
    }
    \label{fig:sup-failure-1}
\end{figure*}


\begin{figure*}[h]
    \centering
    \includegraphics[keepaspectratio, width=0.82\linewidth]{figures/failure-samples-granularity.pdf}
    \caption{Failure samples due to the ambiguity of the layer granularity on Crello~\cite{yamaguchi2021canvasvae} test set.
    The leftmost column shows the input image, and the remaining columns show the decomposed layers from back to front.
    }
    \label{fig:sup-failure-2}
\end{figure*}


\section{Influence of Matting and Inpainting Model Choices}
We vary the matting backbones (Swin-L/T~\cite{liu2021swin}, PVT-M/S~\cite{wang2021pyramid}) and replace the inpainting model with FLUX.1 Fill [dev]~\cite{blackforest2024fluxfill} and evaluate their influence.
The larger matting models improve performance while using FLUX.1 Fill [dev] shows significant degradation.
Generative inpainting often introduces unwanted objects, which interfere with subsequent decomposition steps.
This highlights the need for graphic design-specific inpainting as well as refinement.

\begin{figure}[H]
  \centering
  \begin{subfigure}[b]{\linewidth}
    \centering
    \includegraphics[width=\linewidth]{figures/matting-model-ablation.pdf}
    \caption{Results with different matting backbones, SwinTransformer~\cite{liu2021swin} and PVT~\cite{wang2021pyramid} variants. The inpainting model is fixed to LaMa~\cite{lama}.}
    \label{fig:matting_ablation}
  \end{subfigure}
  \hfill
  \begin{subfigure}[b]{\linewidth}
    \centering
    \includegraphics[width=\linewidth]{figures/impaint-model-ablation.pdf}
    \caption{Results with different inpainting models, LaMa~\cite{lama} and FLUX~\cite{flux}.}
    \label{fig:inpaint_ablation}
  \end{subfigure}

  \caption{Evaluation results of \ours{} with different matting (a) and inpainting model (b) choices.}
  \label{fig:ablation}
\end{figure}





\section{Detail of Decomposition Metrics}
\subsection{Dynamic Time Warping}
\label{sup:dtw}
We implement the Dynamic Time Warping (DTW) as shown in \cref{alg:dtw}. 
Given decomposition results $\hat{Y}=(\hat{\bm{l}}_k)_{k=0}^{K}$ and ground truth $Y=(\bm{l}_q)_{q=0}^{Q}$, the output pairs must include $(0,0)$ and $(K,Q)$ as the start point and end point with a step size of 1, and every layer must be included in at least one pair. An average distance is then computed over all pairs as the final output.
\begin{algorithm}[]
\caption{Dynamic Time Warping (DTW)}
\label{alg:dtw}
\definecolor{codeblue}{rgb}{0.25,0.5,0.5}
\definecolor{codekw}{rgb}{0.85, 0.18, 0.50}
\lstset{
  breaklines=true,
  columns=fullflexible,
  basicstyle=\fontsize{7.2pt}{7.2pt}\ttfamily\selectfont,
  commentstyle=\fontsize{7.2pt}{7.2pt}\color{codeblue},
  keywordstyle=\fontsize{7.2pt}{7.2pt}\color{codekw},
}
\begin{lstlisting}[language=python]
# Inputs:
#  - ls: decomposition results of length K (from bottom to top)
#  - gts: ground truth of length Q (from bottom to top)
#  - dist: distance func bounded in [0, 1]
#
# Outputs:
#  - pairs: a list of (l_idx, gt_idx)
#  - D: distance

# Step 1: Compute Cost Matrix
C = np.zeros((len(ls), len(gts))) 
for i in range(len(ls)):
    for j in range(len(gts)):
        C[i,j] = dist(ls[i], ls[j])

# Step 2: Compute Accumulated Cost Matrix
D = np.zeros((len(ls), len(gts))) 
for i in range(1, len(ls)):
    D[i, 0] = D[i-1,0] + C[i,0]
for j in range(1, len(gts)):
    D[0, j] = D[0,j-1] + C[0,j]
for i in range(1, len(ls)):
    for j in range(1, len(gts)):
        D[i,j] = C[i, j] + min(D[i-1,j], D[i,j-1], D[i-1,j-1])

# Step 3: Backtrace to Find Optimal Alignment
i, j = len(ls)-1, len(gts)-1
pairs = [(i,j)]
while True:
    if i==0 and j==0:
        break
    elif i==0:
        pairs.append((i,j-1))
        j-=1
    elif j==0:
        pairs.append((i-1,j))
        i -= 1
    elif D[i-1,j-1]<=D[i-1,j] and D[i-1,j-1]<=D[i,j-1]:
        pairs.append((i-1,j-1))
        i -= 1
        j -= 1
    elif D[i-1,j]<=D[i-1,j-1] and D[i-1,j]<=D[i,j-1]:
        pairs.append((i-1,j))
        i -= 1
    else:
        pairs.append((i,j-1))
        j -= 1

D = sum([Dist(ls[i], gts[j]) for i,j in pairs])/len(pairs)

return pairs, D
\end{lstlisting}
\end{algorithm}



\subsection{Edits algorithm}
\label{sup:edits}
We employ an iterative refinement process with DTW to quantify the number of edits required to align the decomposition results with the given ground truth. At each iteration, we apply the edit (\texttt{Merge}) that yields the highest gain until either the maximum number of edits is reached or the number of layers is reduced to two, as shown in \cref{alg:merge_edit,alg:find_merge_gains}. 
To efficiently approximate the optimal edit, we adopt a greedy search strategy: at iteration $i$, we focus on changes in distances between consecutive layers---specifically, layers $i$, $i+1$, and $i+2$ (if present)---rather than evaluating all layers globally. The optimal edit is then selected from among all candidates at each iteration, ensuring a balance between computational efficiency and alignment accuracy. 
Although \cref{alg:merge_edit,alg:find_merge_gains} describe only the merging of predicted layers for simplicity, we apply the same merging procedure to both the predicted and ground truth layers to address both under- and over-decomposition.
See \cref{fig:sup-edit-process-1,fig:sup-edit-process-2} for visualization of the alignment and merging process.

\begin{algorithm}[h]
\caption{MergeEdit}
\label{alg:merge_edit}
\definecolor{codeblue}{rgb}{0.25,0.5,0.5}
\definecolor{codekw}{rgb}{0.85, 0.18, 0.50}
\lstset{
  breaklines=true,
  columns=fullflexible,
  basicstyle=\fontsize{7.2pt}{7.2pt}\ttfamily\selectfont,
  commentstyle=\fontsize{7.2pt}{7.2pt}\color{codeblue},
  keywordstyle=\fontsize{7.2pt}{7.2pt}\color{codekw},
}
\begin{lstlisting}[language=python]
# Inputs:
#  - ls: decomposition results of length K (bottom to top)
#  - gts: ground truth of length Q (bottom to top)
#  - emax: maximum number of edits 
#  - dist: distance function bounded in [0, 1]
#
# Outputs:
#  - pairs: a list of (l_idx, gt_idx)
#  - D: distance
#  - e: number of edits

e = 0
while e < emax and len(ls) > 2: 
    pairs, _ = dtw(ls, gts)
    merged_ids, gains = find_gains(ls, gts, 
                                pairs, dist)
    if len(gains) > 0:
        best_id = merged_ids[argmin(gains)]
        merged = merge(ls[best_id], ls[best_id+1])
        ls[best_id] = merged
        ls.pop(best_id+1)
    else:
        break
    e += 1
return dtw(ls, gts), e


def merge(x, y):  # Merge func by OpenCV 
    return Image.alpha_composite(x, y)
\end{lstlisting}
\end{algorithm}

\clearpage

\begin{algorithm}[h]
\caption{FindGains}
\label{alg:find_merge_gains}
\definecolor{codeblue}{rgb}{0.25,0.5,0.5}
\definecolor{codekw}{rgb}{0.85, 0.18, 0.50}
\lstset{
  breaklines=true,
  columns=fullflexible,
  basicstyle=\fontsize{7.2pt}{7.2pt}\ttfamily\selectfont,
  commentstyle=\fontsize{7.2pt}{7.2pt}\color{codeblue},
  keywordstyle=\fontsize{7.2pt}{7.2pt}\color{codekw},
}
\begin{lstlisting}[language=python]
# Inputs:
#  - ls: decomposition results of length K (bottom to top)
#  - gts: ground truth of length Q (bottom to top)
#  - pairs: list of (l_idx, gt_idx) obtained from DTW
#  - dist: distance function bounded in [0, 1]
#
# Outputs:
#  - merged_ids: list of indices where merging occurs
#  - gains: list of corresponding distance reductions

merged_ids, gains = [], []
for i in range(len(ls)-1):
    # Step 1: Compute merged layer candidates
    subls = [merge(ls[i], ls[i+1])] + ([ls[i+2]] if i+2 < len(ls) else [])
    
    # Step 2: Gather corresponding ground truth layers
    subgts = [
        [gts[p[1]] for p in pairs if p[0] == i],
        [gts[p[1]] for p in pairs if p[0] == i+1]
    ]

    # Step 3: Compute current distance sum
    curD = sum([dist(ls[i], subgt) for subgt in subgts[0]]) + \
            sum([dist(ls[i+1], subgt) for subgt in subgts[1]])

    # Step 4: Compute distance sum after merging
    Ds = []
    for j in range(len(subls)):
        for k in range(len(subgts)):
            Ds.append(sum([dist(subls[j], subgt) for subgt in subgts[k]]))      
    Ds = [d + Ds[0] for d in Ds[1:]]
    minD = min(Ds)        
    
    # Step 5: Check if merging reduces distance
    if minD < curD:
        merged_ids.append(i)
        gains.append(minD - curDs)
return merged_ids, gains


def merge(x, y):  # Merge func by OpenCV 
    return Image.alpha_composite(x, y)
\end{lstlisting}
\end{algorithm}

\newpage

\section{Loss functions}
\label{sup:loss}

We use binary cross-entropy loss $\mathcal{L}_{\text{BCE}}$, IoU loss $\mathcal{L}_{\text{IoU}}$, and SSIM loss $\mathcal{L}_{\text{SSIM}}$ in our training as BiRefNet~\citep{birefnet}. Definitions of each loss function are as follows.
\begin{align}
    \mathcal{L}_{\text{BCE}}(\hat{\bm{l}}^{\text{A}}, \bm{l}^{\text{A}}) &= \frac{1}{|\Omega|} \sum_{i,j\in\Omega} -\bm{l}^{\text{A}}_{i,j} \log \hat{\bm{l}}^{\text{A}}_{i,j} \notag \\
    &\quad - (1 - \bm{l}^{\text{A}}_{i,j}) \log (1 - \hat{\bm{l}}^{\text{A}}_{i,j}),
\end{align}
\begin{equation}
    \mathcal{L}_{\text{IoU}}(\hat{\bm{l}}^{\text{A}}, \bm{l}^{\text{A}}) = 1 -  \frac{\sum\limits_{i,j\in\Omega}\bm{l}^{\text{A}}_{i,j} \hat{\bm{l}}^{\text{A}}_{i,j}}{\sum\limits_{m,n\in\Omega}\bm{l}^{\text{A}}_{m,n} + \hat{\bm{l}}^{\text{A}}_{m,n} - \bm{l}^{\text{A}}_{m,n} \hat{\bm{l}}^{\text{A}}_{m,n}},
\end{equation}
{\scriptsize
\begin{equation}
    \mathcal{L}_{\text{SSIM}}(\hat{\bm{l}}^{\text{A}}, \bm{l}^{\text{A}}) = 1- \frac{1}{|\mathcal{P}|} \sum_{p\in \mathcal{P}} \frac{(2\mu_{\bm{l}^{\text{A}}_p} \mu_{\hat{\bm{l}}^{\text{A}}_p} + C_1)(2\sigma_{\bm{l}^{\text{A}}_p\hat{\bm{l}}^{\text{A}}_p} + C_2)}{(\mu_{\bm{l}^{\text{A}}_p}^2 + \mu_{\hat{\bm{l}}^{\text{A}}_p}^2 + C_1)(\sigma_{\bm{l}^{\text{A}}_p}^2 + \sigma_{\hat{\bm{l}}^{\text{A}}_p}^2 + C_2)},
\end{equation}
}
where $\Omega$ denotes the set of spatial indices, and $\mathcal{P}$ represents the set of overlapping patches. The local mean $\mu_{\hat{\bm{l}}^{\text{A}}_p}$ and variance $\sigma^2_{\hat{\bm{l}}^{\text{A}}_p}$, as well as the local mean $\mu_{\bm{l}^{\text{A}}_p}$ and variance $\sigma^2_{\bm{l}^{\text{A}}_p}$ of ground truth, are computed within corresponding patches indexed by $p \in \mathcal{P}$. The covariance $\sigma_{\bm{l}^{\text{A}}_p\hat{\bm{l}}^{\text{A}}_p}$ quantifies structural similarity between the prediction and ground truth patches. $C_1$ and $C_2$ are constants and the setting details follow \cite{birefnet}, except that both the predicted and ground-truth alpha maps $\hat{\bm{l}}^{\text{A}}$ and $ \bm{l}^{\text{A}}$,  are not binarized due to shading and smooth transitions commonly used in graphic design.



\begin{figure*}[p]
    \centering
    \includegraphics[keepaspectratio, width=0.9\linewidth]{figures/merge-edit-1.pdf}
    \caption{
        Visual example of the DTW-based layer alignment and editing process. Red lines connect matched layers between \ours{}'s prediction and the ground truth; their thickness represents the matching score (the inverse of the distance), \ie, the thicker the line, the higher the score. Green boxes indicate the layers that are merged during the editing process. All layers are sorted from back to front, with the backmost layer on the left and the frontmost on the right.
        Although the decomposition result appears useful for editing the input image, its quality is underestimated due to a mismatch in granularity with the ground truth. Layer merging resolves this mismatch, enabling a more faithful evaluation of the decomposition quality.
    }
    \label{fig:sup-edit-process-1}
\end{figure*}

\begin{figure*}[p]
    \centering
    \includegraphics[keepaspectratio, width=0.93\linewidth]{figures/merge-edit-2.pdf}
    \caption{
        Visual example of the DTW-based layer alignment and editing process. Red lines connect matched layers between \ours{}'s prediction and the ground truth; their thickness represents the matching score (the inverse of the distance), \ie, the thicker the line, the higher the score. Green boxes indicate the layers that are merged during the editing process. All layers are sorted from back to front, with the backmost layer on the left and the frontmost on the right.
        \ours{} overdecomposes the white background, but in practical scenarios, it is easy to merge these into a single layer.
        Our evaluation treats such cases as requiring a single edit operation, reflecting the actual editing workload for users.
    }
    \label{fig:sup-edit-process-2}
\end{figure*}

    
\end{document}
