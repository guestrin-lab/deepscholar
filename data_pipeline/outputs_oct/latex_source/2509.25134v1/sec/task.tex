\section{Problem Formulation}


Graphic layer decomposition is the task of decomposing a raster graphic design image $\bm{x} \in [0, 1]^{H \times W \times 3}$ into a sequence of layers $Y=(\bm{l}_k \in [0,1]^{H \times W \times 4})_{k=0}^{K}$.
Here, $H$ and $W$ represent the height and width of the image, respectively. $\bm{x}$ is an RGB image, and $\bm{l}_k$ is an RGBA image, with 3 and 4 channels, respectively.
$k$ represents the blending order of the layer, \ie, the z-index.
$\bm{l}_{k>0}$ is the foreground layer, and $\bm{l}_0$ is the background layer.

The layer sequence $Y$ is composited by the following recursive process from $k=1$ to $k=K$ ($\bm{x} = \bm{x}_{K}$):
\begin{align}
    \label{eq:rasterize}
    \bm{x}^{\text{C}}_{k} &= {\rm B}(\bm{l}_{k}, \bm{x}^{\text{C}}_{k-1}),
    \\ &= \bm{l}^{\text{C}}_{k} \odot \bm{l}^{\text{A}}_{k} + \bm{x}^{\text{C}}_{k-1} \odot (1 - \bm{l}^{\text{A}}_{k}).
\end{align}
Here, the superscript $\text{A}$ represents the alpha channel, and $\text{C}$ represents one of the RGB channels. ${\rm B}(\cdot)$ is the alpha blending function, $\odot$ is element-wise multiplication, and $\bm{x}_{k}$ is the $k$-th blended image.

In this study, we solve the inverse problem of the above, \ie, layer decomposition that estimates the layer sequence $Y$ from the raster image $\bm{x}$.
The granularity of the layer depends on the dataset, and in this study, we treat the human-made graphic designs in the dataset as ground truth.
