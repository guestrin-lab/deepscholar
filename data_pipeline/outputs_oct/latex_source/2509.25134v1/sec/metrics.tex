\section{Decomposition Metrics}
\label{sec:metrics}
There are two problems in evaluating the quality of predicted layers $\hat{Y}$ against the ground truth $Y$.
First, the number of layers in the ground truth and the predicted layers can differ, making it non-trivial to compare directly.
To address this, we apply order-aware layer alignment using DTW~\citep{Mueller07_InformationRetrieval_SPRINGER}.
Second, the quality of the predicted layers can be evaluated from two perspectives: visual quality and granularity.
If we do not consider these aspects separately, we may underestimate the quality of the predicted layers due to differences in granularity, even if they are practically useful. 
We measure granularity by the number of edits required to align the two; we allow merging adjacent layers in the z-index and report both the number of edits and the visual quality after the editing operations.

\paragraph{Layer alignment}
\label{sec:layer_alignment}
As pre-processing, we first group the ground truth and predicted layers based on visibility.
Specifically, we extract layers whose visible regions (\ie, alpha values greater than zero) are not occluded by any other higher layers in z-index, blend them into a single layer, and repeat the same operation with the remaining layers.
This operation never affects the appearance of the composite image and forms what we refer to as a top-layer.

Next, we find alignment between the two layer sequences with different lengths using DTW, which considers the sequence order even if the lengths differ.
We obtain a set of pairs $P=\{(k_s,q_s)|s=0,1,\ldots,S\}$, where $k_s$ and $q_s$ represent the layer indices, and $S$ is the number of pairs.
Note that resolved pairs satisfy the monotonicity condition, \ie., $k_s$ and $q_s$ are increasing sequences; in other words, layers cannot be shuffled during alignment (see \suppref{sup:dtw}).
We define the distance metric for the layer pair as the sum of the negative value of the alpha's soft IoU and 
L1 distance of the RGB channels weighted by the ground-truth alpha, as introduced in \cite{suzuki2024fast}.

Finally, we compute the quality metric between the two layer sequences as follows:
\begin{align}
    \label{eq:metric}
    \mathcal{E}(\hat{Y}, Y) = \frac{1}{S}\sum_{s=0}^{S} e(\hat{\bm{l}}_{k_s}, \bm{l}_{q_s}),
\end{align}
where $e(\cdot)$ is an arbitrary function that measures the similarity or distance between layers. 
We use the weighted L1 distance of the RGB channels and the soft IoU of the alpha channel as $e(\cdot)$, similar to DTW's distance metric.



\paragraph{Layer merge}

Due to the ill-posed nature of layer decomposition, decomposition results sometimes do not align well with the ground truth. \TODO{Provide examples?}
In this work, we relax the alignment constraints by allowing \emph{edits}.
The idea is inspired by minimum edit distance~\citep{wagner1974string}, which is commonly used for string alignment. We define a specific edit operation set for layers, and report both the maximum number of allowed edits and the distance metric used in DTW after edits.
This gives a straightforward insight into how many layer-level edits are required for good alignment.

For simplicity, we define a single edit operation; \texttt{Merge}, which merges two consecutive layers in z-index, when the edit yields the highest positive distance improvement.
We apply edits iteratively until no further improvements are possible or the number of layers is reduced to 2.
The ground truth is also mergeable.
Visual examples of the edit process can be found in \suppref{fig:sup-edit-process-1,fig:sup-edit-process-2}.















