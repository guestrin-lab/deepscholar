\section{Related Work}








We review control strategies for multirotor \glspl{uav} transporting cable-suspended payloads, covering classical model-based methods as well as learning-based approaches for both single- and multi-agent coordination. 
For a comprehensive survey of aerial cable transport, see~\cite{estevez_review_2024}.

\subsection{Traditional Model-based Approaches}


Model-based control approaches for aerial payload transport include centralized cascaded geometric controllers that provide stability guarantees for payload and manipulation control~\cite{sreenath_dynamics_2013}, as well as decentralized controllers that exploit internal cable tension for quasi-static attitude stabilization~\cite{tognon_aerial_2018}. 
However, these methods rely on noisy payload acceleration as a feedback signal and model cables as rigid rods, which limit the applicability in agile scenarios. 

Centralized and decentralized nonlinear model predictive control (NMPC) have advanced multi-\gls{uav} payload manipulation without acceleration feedback and can incorporate high-level objectives (e.g., obstacle collision avoidance)~\cite{ sun_nonlinear_2023,de2025distributed}. 
However, centralized NMPC is computationally expensive and suffers when scaled to large teams, while decentralized NMPC depends on iterative inter-robot communication and can suffer from deadlocks. Both approaches still adopt the rigid-rod cable assumption inherited from reactive controllers.  

The controllers devised by the previous methods adopt simplified models which restrict control performance to quasi-static regimes. Thus, other works have employed the full system dynamics in offline motion planning through optimization-based planners to account for more accurate models and enable agile maneuver planning~\cite{wahba2025pc, wahba_kinodynamic_2024,Wang2025SafeAA,sun2025agile}. Nevertheless, despite considering the full dynamics, these approaches still rely on the rigid-rod cable assumption.

To overcome the limitations of the rigid-rod assumption, more accurate cable models have been incorporated by switching the dynamics that explicitly capture the taut–slack transitions~\cite{wang2024impact, recalde2025hpc}. NMPC formulations with such hybrid models enable motions unattainable under rigid-rod assumptions but have so far been limited to single-\gls{uav} systems. Extending them to cooperative multi-robot manipulation remains challenging due to the added complexity of modeling, optimization, and coordination.

In contrast, our work explores reinforcement learning (RL) as a promising direction for achieving agile maneuvers and high robustness in multi-\gls{uav} payload transport while accounting for realistic cable dynamics.
\subsection{Reinforcement Learning-based Approaches}

Early application of \gls{rl} to \gls{uav} control focused on single-agent scenarios, and showed that RL agents trained with model-free \gls{rl} algorithms perform on par with or better than classical controllers~\cite{Koch2018ReinforcementLF}. 
Later works have demonstrated the effectiveness of \gls{rl}-trained \glspl{uav} in handling harsh initial conditions, executing aggressive maneuvers, and operating near the limits of their dynamic capabilities~\cite{Song2023ReachingTL, xing_multi-task_2024}. 
Other approaches have extended single-\gls{uav} control to payload transport and aerial manipulation, where RL-based controllers have also proven effective in adapting to unknown payload dynamics, maintaining stability and rejecting payload disturbances~\cite{hua_new_2022}.
Notably, \cite{cao2025flare} is capable of mode-switching and handling flexible cables for single \glspl{uav}.
However, compared to the single-\gls{uav}-payload-transportation problem, our current work on multi-\gls{uav} collaborative payload transport is considerably more challenging due to the coupling between vehicles, the need for precise coordination to regulate cable tensions, and the heightened risk of collisions.

\glsreset{marl}\gls{marl} has emerged as a powerful framework for cooperative tasks.
Some \Gls{marl} approaches employ \gls{ctde} by using a shared critic, such as \cite{Lowe2017MultiAgentAF, yu_surprising_2022}. Others, such as \gls{ippo}~\cite{witt_is_2020}, follow a completely decentralized regime. A decentralized scheme such as \gls{ippo} avoids scalability and communication overheads and is thus utilized in our work.
Many works have successfully used \gls{marl}, particularly methods such as \gls{mappo} and \gls{ippo}, in real-world, multi-robot collaborative tasks~\cite{Pandit2024LearningDM,  Chen2025DecentralizedNO}.
\gls{marl} strategies have also achieved success in swarm coordination, collaborative pursuit and evasion, and obstacle avoidance~\cite{huang_collision_2024, zhao_deep_2024}.

In the area of payload transportation, the adaptability and disturbance rejection capabilities of single-\gls{uav} methods have naturally led to multi-\gls{uav} extensions such as \cite{Lin2024PayloadTW, Estevez2024Reinforcement, xu_omnidrones_2024}. However, unlike the centralized approach presented in \cite{Lin2024PayloadTW}, we adopt a decentralized approach, thereby avoiding communication and scalability overheads.
While \cite{xu_omnidrones_2024} considers only rigid-link payloads and \cite{Estevez2024Reinforcement} assumes the payload cables are always taut, we make no such assumptions and model payload links with realistic flexible cables. Neither of these works reports transfer to the real world, whereas we achieve successful zero-shot sim2real transfer and demonstrate the robustness and agility of our robots under harsh real-world conditions. 
The closest to our current work is the approach from~\cite{zeng2025decentralized}, which also uses a decentralized training scheme and showcases robustness under harsh real-world settings. 
However, unlike our fully decentralized \gls{ippo}-based approach, this work uses the \gls{ctde}-based \gls{mappo} algorithm. It relies on a low-level controller, whereas we do not, and thanks to our highly parallelized training setup, our training speed is about ten times faster. In addition, \cite{zeng2025decentralized} still retains the rigid-rod assumption, where our work focuses on exploiting the hybrid cable model to achieve agile maneuvers and recovery from harsh configurations.  












