\section{Introduction}
\glsresetall
\glspl{uav} have moved from research prototypes to widely deployed tools in civil and industrial settings, including disaster response, agriculture, logistics, and inspection \cite{Idrissi2022AROA,Lyu2023UnmannedAVA}. Among emerging applications, collaborative cable-suspended payload transport is compelling: cables are lightweight, adaptable to irregular loads, and introduce compliance that attenuates vibration. However, multi-\gls{uav} transport of a shared suspended payload is challenging due to pendulum dynamics, tension coupling, contact events, and disturbances. 

There are two well-established paradigms to tackle control in robotics: model-based and learning-based methods. Classical model-based controllers offer stability guarantees and well-founded design principles, but struggle with modeling errors and scalability. Centralized coordination can provide optimal solutions but incurs bottlenecks and single points of failure, while decentralized schemes can tackle scalability but may lack performance guarantees~\cite{batra_decentralized_2022,estevez_review_2024}. \gls{rl}, in contrast, has emerged as a complementary paradigm that can provide adaptivity in scenarios with complex dynamics or incomplete models.
It can directly learn control from interaction and improve robustness to unmodeled effects and disturbances. 
Recent results demonstrate strong performance for single and multi \gls{uav} tasks such as agile flights under challenging conditions \cite{kaufmann_champion-level_2023,Eschmann2024}. 

Building on this progress, we leverage \gls{rl} to address the decentralized control problem of multi-\gls{uav} payload transport with hybrid cable dynamics that capture taut–slack transitions. 
To the best of our knowledge, this is the first use of \gls{rl} for controlling multiple robots with constrained onboard microcontrollers operating near their actuation limits. 
Our objectives are to stabilize the suspended payload under external disturbances, manage cable mode switching, and safely distribute forces among the robots without collisions.
Fig.~\ref{fig:overview} illustrates the training pipeline, deployed system, and shows hardware demonstrations with two robots transporting a payload, including robust recovery from harsh conditions and strong disturbance rejection.

We implement our method on the Crazyflie 2.1 research platform, which is widely used for cooperative transport and \gls{uav} control studies \cite{wahba_kinodynamic_2024,huang_collision_2024}.
The target platform has a low thrust-to-weight ratio of 1.4, which makes operation close to the motor limits much more likely than on other platforms such as high-end racing multirotors.
Our method allows each robot to execute its own policy onboard at 250~Hz and relies on local state estimation and relative pose information. 
The policy output maps directly to motor \gls{pwm} at high frequency, without relying on cascaded low-level controllers. This enables us to operate close to the actuation limits which is particularly important for agile flights and disturbance rejection. 

On the learning side, we leverage high throughput training with large scale parallelized simulation and extensive domain randomization to improve robustness and shorten iteration time. 
The resulting controller remains fully decentralized with a small computational footprint and no need for inter-robot communication, which simplifies deployment and improves real world performance.

 






In summary, our main contributions are:
\begin{itemize}
  \item \textbf{End-to-end decentralized \gls{marl} for multiple quadrotors carrying a cable-suspended payload.} We train a fully decentralized policy with direct motor PWM commands (no low-level cascades) for cable-suspended payload transport.
  \item \textbf{Empirical evaluation in simulation and hardware.} We validate on Crazyflie 2.1 platform. In wind trials with a mean wind speed of 3.5~m/s the policy maintains stable formations and recovers from large external disturbances.
  
  \item \textbf{High-throughput simulation.} A GPU-parallelized JAX/MJX pipeline captures cable–payload–robot interactions in contact-rich scenarios, providing a foundation for future aerial manipulation and related tasks.
\end{itemize}
