\usepackage[dvipsnames]{xcolor}
\definecolor{thedarkblue}{RGB}{0,0,120} %
\definecolor{mydarkblue}{rgb}{0,0.08,0.45} %
\definecolor{darkblue}{rgb}{0,0.08,180}
\colorlet{TufteRed}{red!80!black}
\definecolor{theblue}{RGB}{0,0,180}
\colorlet{thered}{TufteRed}
      
\usepackage{microtype}
\usepackage{balance}


\usepackage{booktabs}
\usepackage{tabularx}



\renewcommand{\qedsymbol}{$\blacksquare$}
\newcommand{\eat}[1]{\ignorespaces}
\usepackage{comment}

\usepackage{tikz}
\usepackage{verbatim}
\usetikzlibrary{arrows}
\usetikzlibrary{shapes,snakes}
\usetikzlibrary{decorations.pathmorphing} %
\usetikzlibrary{fit}					%
\usetikzlibrary{backgrounds}	%

\usepackage{ragged2e}
\usepackage{makecell}
\usepackage{multirow}
\usepackage{microtype}
\usepackage{balance}
\usepackage{setspace}

\graphicspath{{./}{./graphics/}}
\newcolumntype{H}{>{\setbox0=\hbox\bgroup}c<{\egroup}@{}}

\newcolumntype{R}[1]{>{\RaggedLeft\arraybackslash}} %
\newcolumntype{L}[1]{>{\RaggedRight\arraybackslash}} %

\newcommand\TTT{\rule{0pt}{3.2ex}}
\newcommand\BBB{\rule[-1.4ex]{0pt}{0pt}}


\newcommand{\rbr}[1]{\left(#1\right)}
\newcommand{\cbr}[1]{\left\{#1\right\}}
\newcommand{\nbr}[1]{\left\|#1\right\|}
\newcommand{\abr}[1]{\left|#1\right|}
\newcommand{\abs}[1]{\left|#1\right|}
\newcommand{\floor}[1]{\left\lfloor #1 \right\rfloor}
\newcommand{\ceil}[1]{\left\lceil #1 \right\rceil}
\newcommand{\inner}[2]{\left\langle #1,#2 \right\rangle}

\newcommand{\etal}{\emph{et al.}}
\newcommand{\ea}{\emph{et al.}}
\newcommand{\eg}{\emph{e.g.}}
\newcommand{\ie}{\emph{i.e.}}
\newcommand{\iid}{\emph{iid}}
\newcommand{\cf}{\emph{cf.}\ }
\newcommand{\wrt}{\emph{w.r.t.}\ }
\newcommand{\st}{\emph{s.t.}\ }

\newtheorem{corollary}{\bfseries{Corollary}}
\newtheorem{proof2}{PROOF}
\newtheorem*{fact}{Fact}
\newtheorem*{note}{\hspace{-1em}\textsc{Note}}
\newtheorem{corol}{Corollary}%
\newtheorem{axiom}{Axiom}%
\newtheorem{cond}{Condition}%
\newtheorem{property2}{Property}%
\newtheorem{property}{Property}%
\newtheorem{lemma}{\hspace{-1em}\bfseries{Lemma}}
\newtheorem{Definition}{\hspace{-1em}\bfseries{Definition}}
\newtheorem{Claim}{Claim}%

\AtBeginEnvironment{pmatrix}{\setlength{\arraycolsep}{2pt}}



\providecommand{\tensor}[1]{\boldsymbol{\mathcal{#1}}}%
\providecommand{\mat}[1]{\boldsymbol{\mathrm{#1}}}%
\renewcommand{\vec}[1]{\boldsymbol{\mathrm{#1}}}
\providecommand{\sca}[1]{{\mathrm{#1}}}%
\DeclareMathOperator{\rank}{rank}%
\DeclareMathOperator{\diag}{diag}%
\DeclareMathOperator{\Diag}{Diag}%
\providecommand{\itr}[2]{#1^{(#2)}}
\providecommand{\itn}[1]{^{(#1)}}%
\providecommand{\eps}{\varepsilon}%
\providecommand{\kron}{\otimes}
\DeclareMathOperator{\tvec}{vec}
\providecommand{\pmat}[1]{\begin{pmatrix} #1 \end{pmatrix}}
\providecommand{\bmat}[1]{\begin{bmatrix} #1 \end{bmatrix}}
\providecommand{\spmat}[1]{\left(\begin{smallmatrix} #1 \end{smallmatrix}\right)}
\providecommand{\sbmat}[1]{\left[\begin{smallmatrix} #1 \end{smallmatrix}\right]}

\DeclareMathOperator*{\minimize}{minimize}
\DeclareMathOperator*{\maximize}{maximize}
\DeclareMathOperator*{\argmax}{argmax}
\DeclareMathOperator*{\argmin}{argmin}
\DeclareMathOperator*{\argsort}{arg\,sort}
\providecommand{\subjectto}{\ensuremath{\text{subject to}}}
\providecommand{\MINof}[1][]{{\displaystyle \minimize_{#1}}}
\providecommand{\MIN}[2]{\begin{array}{ll} \MINof[#1] & #2 \end{array}}
\providecommand{\MINone}[3]{\begin{array}{ll} \MINof[#1] & #2 \\ \subjectto  & #3 \end{array}}
\providecommand{\MINtwo}[4]{\begin{array}{ll} \MINof[#1] & #2 \\ \subjectto  & #3 \\ & #4 \end{array}}
\providecommand{\MINthree}[5]{\begin{array}{ll} \MINof[#1] & #2 \\ \subjectto  & #3 \\ & #4 \\ & #5 \end{array}}
\providecommand{\MINfour}[6]{\begin{array}{ll} \MINof[#1] & #2 \\ \subjectto  & #3 \\ & #4 \\ & #5 \\ & #6 \end{array}}
\providecommand{\MAXof}[1][]{{\displaystyle \maximize_{#1}}}
\providecommand{\MAX}[2]{\begin{array}{ll} \MAXof[#1] & #2 \end{array}}
\providecommand{\MAXone}[3]{\begin{array}{ll} \MAXof[#1] & #2 \\ \subjectto & #3 \end{array}}
\providecommand{\MAXtwo}[4]{\begin{array}{ll} \MAXof[#1] & #2 \\ \subjectto  & #3 \\ & #4 \end{array}}
\providecommand{\MAXthree}[5]{\begin{array}{ll} \MAXof[#1] & #2 \\ \subjectto  & #3 \\ & #4 \\ & #5 \end{array}}
\providecommand{\MAXfour}[6]{\begin{array}{ll} \MAXof[#1] & #2 \\ \subjectto  & #3 \\ & #4 \\ & #5 \\ & #6 \end{array}}

\DeclareMathOperator{\E}{E}
\DeclareMathOperator{\hugeE}{\mbox{\huge\raise-0.3ex\hbox{E}}}
\DeclareMathOperator{\p}{\mathbb{P}}
\DeclareMathOperator{\hugep}{\mbox{\huge\raise-0.3ex\hbox{$\p$}}}
\DeclareMathOperator{\Var}{Var}
\DeclareMathOperator{\Cov}{Cov}
\DeclareMathOperator{\Bias}{Bias}
\DeclareMathOperator{\sign}{sign}
\DeclareMathOperator{\Std}{Std}
\providecommand{\Eof}{\E\BracketOf}
\providecommand{\hugeEof}{\hugeE\BracketOf}
\providecommand{\Stdof}{\Std\BracketOf}
\providecommand{\varof}{\Std\BracketOf}
\providecommand{\Covof}[2]{\Cov\BrackeOf#1,#2\right]}
\providecommand{\prob}[1][]{\p_{#1}\BraceOf}
\providecommand{\hugeprob}[1][]{\hugep_{#1}\BraceOf}

\DeclareMathOperator{\degree}{degree}
\DeclareMathOperator{\trace}{trace}
\providecommand{\set}{\mathcal}
\providecommand{\graph}{\mathcal}
\providecommand{\mathdef}{\equiv}
\providecommand{\card}{\absof}
\providecommand{\cardof}{\absof}
\providecommand{\eqdef}{\equiv}
\providecommand{\eps}{\varepsilon}
\DeclareMathOperator{\bigO}{O}
\providecommand{\bigOof}{\bigO\ParensOf}
\providecommand{\absof}[1]{$\left| #1 \right|$}

\newcommand{\RR}{\mathbb{R}}
\newcommand{\CC}{\mathbb{C}}



\providecommand{\eye}{\mat{I}}
\providecommand{\mA}{\ensuremath{\mat{A}}}
\providecommand{\mB}{\ensuremath{\mat{B}}}
\providecommand{\mC}{\ensuremath{\mat{C}}}
\providecommand{\mD}{\ensuremath{\mat{D}}}
\providecommand{\mE}{\ensuremath{\mat{E}}}
\providecommand{\mF}{\ensuremath{\mat{F}}}
\providecommand{\mG}{\ensuremath{\mat{G}}}
\providecommand{\mH}{\ensuremath{\mat{H}}}
\providecommand{\mI}{\ensuremath{\mat{I}}}
\providecommand{\mJ}{\ensuremath{\mat{J}}}
\providecommand{\mK}{\ensuremath{\mat{K}}}
\providecommand{\mL}{\ensuremath{\mat{L}}}
\providecommand{\mM}{\ensuremath{\mat{M}}}
\providecommand{\mN}{\ensuremath{\mat{N}}}
\providecommand{\mO}{\ensuremath{\mat{O}}}
\providecommand{\mP}{\ensuremath{\mat{P}}}
\providecommand{\mQ}{\ensuremath{\mat{Q}}}
\providecommand{\mR}{\ensuremath{\mat{R}}}
\providecommand{\mS}{\ensuremath{\mat{S}}}
\providecommand{\mT}{\ensuremath{\mat{T}}}
\providecommand{\mU}{\ensuremath{\mat{U}}}
\providecommand{\mV}{\ensuremath{\mat{V}}}
\providecommand{\mW}{\ensuremath{\mat{W}}}
\providecommand{\mX}{\ensuremath{\mat{X}}}
\providecommand{\mY}{\ensuremath{\mat{Y}}}
\providecommand{\mZ}{\ensuremath{\mat{Z}}}
\providecommand{\mLambda}{\ensuremath{\mat{\Lambda}}}
\providecommand{\mzero}{\ensuremath{\mat{0}}}

\providecommand{\tA}{\ensuremath{\tensor{A}}}
\providecommand{\tB}{\ensuremath{\tensor{B}}}
\providecommand{\tC}{\ensuremath{\tensor{C}}}
\providecommand{\tD}{\ensuremath{\tensor{D}}}
\providecommand{\tE}{\ensuremath{\tensor{E}}}
\providecommand{\tF}{\ensuremath{\tensor{F}}}
\providecommand{\tG}{\ensuremath{\tensor{G}}}
\providecommand{\tH}{\ensuremath{\tensor{H}}}
\providecommand{\tI}{\ensuremath{\tensor{I}}}
\providecommand{\tJ}{\ensuremath{\tensor{J}}}
\providecommand{\tK}{\ensuremath{\tensor{K}}}
\providecommand{\tL}{\ensuremath{\tensor{L}}}
\providecommand{\tM}{\ensuremath{\tensor{M}}}
\providecommand{\tN}{\ensuremath{\tensor{N}}}
\providecommand{\tO}{\ensuremath{\tensor{O}}}
\providecommand{\tP}{\ensuremath{\tensor{P}}}
\providecommand{\tQ}{\ensuremath{\tensor{Q}}}
\providecommand{\tR}{\ensuremath{\tensor{R}}}
\providecommand{\tS}{\ensuremath{\tensor{S}}}
\providecommand{\tT}{\ensuremath{\tensor{T}}}
\providecommand{\tU}{\ensuremath{\tensor{U}}}
\providecommand{\tV}{\ensuremath{\tensor{V}}}
\providecommand{\tW}{\ensuremath{\tensor{W}}}
\providecommand{\tX}{\ensuremath{\tensor{X}}}
\providecommand{\tY}{\ensuremath{\tensor{Y}}}
\providecommand{\tZ}{\ensuremath{\tensor{Z}}}

\providecommand{\ones}{\vec{e}}
\providecommand{\va}{\ensuremath{\vec{a}}}
\providecommand{\vb}{\ensuremath{\vec{b}}}
\providecommand{\vc}{\ensuremath{\vec{c}}}
\providecommand{\vd}{\ensuremath{\vec{d}}}
\providecommand{\ve}{\ensuremath{\vec{e}}}
\providecommand{\vf}{\ensuremath{\vec{f}}}
\providecommand{\vg}{\ensuremath{\vec{g}}}
\providecommand{\vh}{\ensuremath{\vec{h}}}
\providecommand{\vi}{\ensuremath{\vec{i}}}
\providecommand{\vj}{\ensuremath{\vec{j}}}
\providecommand{\vk}{\ensuremath{\vec{k}}}
\providecommand{\vl}{\ensuremath{\vec{l}}}
\providecommand{\vm}{\ensuremath{\vec{l}}}
\providecommand{\vn}{\ensuremath{\vec{n}}}
\providecommand{\vo}{\ensuremath{\vec{o}}}
\providecommand{\vp}{\ensuremath{\vec{p}}}
\providecommand{\vq}{\ensuremath{\vec{q}}}
\providecommand{\vr}{\ensuremath{\vec{r}}}
\providecommand{\vs}{\ensuremath{\vec{s}}}
\providecommand{\vt}{\ensuremath{\vec{t}}}
\providecommand{\vu}{\ensuremath{\vec{u}}}
\providecommand{\vv}{\ensuremath{\vec{v}}}
\renewcommand{\vv}{\ensuremath{\vec{v}}}
\providecommand{\vw}{\ensuremath{\vec{w}}}
\providecommand{\vx}{\ensuremath{\vec{x}}}
\providecommand{\vy}{\ensuremath{\vec{y}}}
\providecommand{\vz}{\ensuremath{\vec{z}}}
\providecommand{\vpi}{\ensuremath{\vecalt{\pi}}} 

\providecommand{\ssa}{\ensuremath{\sca{a}}}
\providecommand{\ssb}{\ensuremath{\sca{b}}}
\providecommand{\ssc}{\ensuremath{\sca{c}}}
\providecommand{\ssd}{\ensuremath{\sca{d}}}
\providecommand{\sse}{\ensuremath{\sca{e}}}
\providecommand{\ssf}{\ensuremath{\sca{f}}}
\providecommand{\ssg}{\ensuremath{\sca{g}}}
\providecommand{\ssh}{\ensuremath{\sca{h}}}
\providecommand{\ssi}{\ensuremath{\sca{i}}}
\providecommand{\ssj}{\ensuremath{\sca{j}}}
\providecommand{\ssk}{\ensuremath{\sca{k}}}
\providecommand{\ssl}{\ensuremath{\sca{l}}}
\providecommand{\ssm}{\ensuremath{\sca{l}}}
\providecommand{\ssn}{\ensuremath{\sca{n}}}
\providecommand{\sso}{\ensuremath{\sca{o}}}
\providecommand{\ssp}{\ensuremath{\sca{p}}}
\providecommand{\ssq}{\ensuremath{\sca{q}}}
\providecommand{\ssr}{\ensuremath{\sca{r}}}
\providecommand{\sss}{\ensuremath{\sca{s}}}
\providecommand{\sst}{\ensuremath{\sca{t}}}
\providecommand{\ssu}{\ensuremath{\sca{u}}}
\providecommand{\ssv}{\ensuremath{\sca{v}}}
\providecommand{\ssw}{\ensuremath{\sca{w}}}
\providecommand{\ssx}{\ensuremath{\sca{x}}}
\providecommand{\ssy}{\ensuremath{\sca{y}}}
\providecommand{\ssz}{\ensuremath{\sca{z}}}
\providecommand{\sspi}{\ensuremath{\scaalt{\pi}}} 


\DeclareMathOperator{\cut}{cut}
\DeclareMathOperator{\vol}{vol}



\definecolor{googleblue}{HTML}{4285F4}
\definecolor{googlered}{HTML}{DB4437}
\definecolor{googlepurple}{HTML}{A142F4} %
\definecolor{googlegreen}{HTML}{0F9D58}
