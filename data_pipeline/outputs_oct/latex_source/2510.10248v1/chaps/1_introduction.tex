\section{Introduction}

Molecular property prediction serves as a cornerstone in modern drug discovery and materials science, enabling researchers to computationally estimate critical molecular characteristics such as bioavailability, toxicity, and therapeutic efficacy before costly experimental validation~\citep{drugdiscoveryreview1,drugdiscoveryreview4,drugdiscoveryreview3,drugdiscoveryreview2}. Traditional experimental approaches for determining molecular properties are prohibitively expensive and time-consuming, often requiring weeks to months and costing thousands of dollars per compound~\citep{cost2,cost3,cost1}. For instance, a single ADMET (Absorption, Distribution, Metabolism, Excretion, Toxicity) screening can cost upwards of \$10,000 per molecule, making it impractical to evaluate the millions of compounds in chemical space~\citep{admetlab,admet}. This bottleneck has driven the urgent need for accurate and efficient computational models that can predict molecular properties at scale~\citep{recenttrends}
% neuralmp,analyzingpp,recenttrends}.

Despite decades of research, current molecular property prediction approaches face fundamental limitations that hinder their practical adoption. Early computational methods relied on hand-crafted molecular descriptors and traditional machine learning algorithms, which struggle with task transferability and require extensive feature engineering for each new application~\citep{quantumdescriptors,extendedfingerprints,moleculenet}. More recent advances have introduced specialized molecular models, including graph neural networks (GNNs)~\citep{crystalgnn,unimol} and molecular language models~\citep{chemberta,biot5-plus,moleculargpt}, which have achieved impressive performance by learning molecular representations directly from graph structures or SMILES strings. However, these approaches suffer from a critical limitation: they lack interpretability and fail to provide chemically meaningful explanations for their predictions. When a model predicts that a molecule is toxic, chemists cannot understand which structural features or chemical principles led to this conclusion, limiting the model's utility in real-world drug development pipelines where understanding the rationale behind predictions is crucial for decision-making.

The fundamental limitation shared by all existing molecular property prediction methods is the absence of \textbf{effective chemical reasoning}—the ability to analyze molecular structures, identify relevant functional groups, apply chemical principles, and provide coherent explanations for property predictions. When experienced chemists evaluate molecular properties, they follow a structured reasoning process: examine the molecular structure to identify key functional groups (e.g., hydroxyl groups affecting solubility~\citep{organic}), consider relevant chemical principles (e.g., Lipinski's Rule for drug-likeness~\citep{fiverule}), analyze structure-activity relationships~\citep{sartoxic}, and synthesize these insights to make informed predictions~\citep{augmenting,xai}. This reasoning capability is crucial not only for accuracy but also for trust and adoption in practical applications, as chemists need to understand the rationale behind predictions to make informed decisions about lead compound optimization and safety assessments.

To address these fundamental limitations, we propose MPPReasoner, a novel multimodal framework that successfully introduces chemical reasoning capabilities into molecular property prediction. MPPReasoner represents the first systematic attempt to cultivate domain-specific reasoning abilities for molecular property prediction, enabling models to analyze molecular structures, apply established chemical principles, and provide human-interpretable explanations during the prediction process. Built upon Qwen2.5-VL-7B-Instruct~\citep{qwen25vl}, MPPReasoner integrates multimodal molecular representations by combining 2D molecular images with SMILES strings, enabling comprehensive structural understanding from both visual and textual modalities. 
Our training methodology employs a two-stage strategy to progressively develop chemical reasoning capabilities: 1) Supervised Fine-Tuning (SFT) with carefully curated reasoning trajectories generated through expert knowledge and teacher models, establishing foundational reasoning patterns; 2) Reinforcement Learning from Principle-Guided Rewards (RLPGR), a novel reward framework that leverages verifiable, rule-based feedback to enhance chemical reasoning quality. Unlike traditional reinforcement learning (RL) approaches, RLPGR decomposes chemical reasoning into hierarchical reward components that evaluate logical consistency, chemical principle application accuracy, and molecular structure analysis precision through computational verification.

Extensive experiments on 8 diverse molecular property prediction datasets demonstrate the effectiveness of our approach, achieving substantial performance improvements with average ROC-AUC scores~\citep{roc,auc} of 0.8068 on in-distribution (ID) tasks and 0.7801 on out-of-distribution (OOD) tasks, outperforming the best existing baselines by 7.91 and 4.53 percentage points respectively. Notably, our model exhibits exceptional OOD generalization capabilities, with particularly significant improvements on OOD datasets where many specialist models lack evaluation capability. Through expert evaluation and detailed case studies, we demonstrate that our approach produces chemically sound explanations that provide valuable insights into molecular property relationships.
The main contributions of this work are as follows:

\begin{itemize}[leftmargin=*]
    \item We successfully introduce chemical reasoning capabilities into molecular property prediction tasks through MPPReasoner, representing a systematic approach to enable structured analysis, chemical principles application, and mechanistic explanations during the prediction process.
    
    \item We propose a comprehensive training strategy that combines high-quality reasoning trajectories SFT and RLPGR, a novel hierarchical reward framework targeting chemical reasoning quality through verifiable, rule-based feedback on logical consistency, comparative analysis, chemical principle usage and molecular structure analysis .
    
    \item We construct a carefully curated dataset of chemical reasoning trajectories generated through expert knowledge and few-shot prompting with multiple teacher models, providing a valuable resource for training reasoning-capable molecular property prediction models.
    
    \item We demonstrate significant performance improvements across 8 datasets with superior OOD generalization, while providing enhanced interpretability through expert-validated reasoning paths that offer insights for chemists in real-world applications.
\end{itemize}
