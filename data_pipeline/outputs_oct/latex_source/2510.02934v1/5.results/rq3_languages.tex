\subsubsection{Impact of Programming Language Variability}


\begin{table}[!htp]\centering
\caption{Correctness assessment performance in F1-score of the approaches across different programming languages}
\label{tab:sensitivity_languages}
\resizebox{\columnwidth}{!}{ % use this if the table is too large
\begin{tabular}{l|rrrrr}\toprule
Targeted language &CodeBERT &CodeT5+ &\openia &\tool \\\midrule
CPP &0.65 &0.67 &0.82 &0.76 \\
C Sharp &0.74 &0.75 &0.80 &0.85 \\
Java &0.59 &0.64 &0.68 &0.76 \\
JavaScript &0.66 &0.80 &0.86 &0.89 \\
PHP &0.54 &0.65 &\cellcolor[HTML]{f8f9fa}0.71 &0.83 \\
Python &0.37 &0.50 &0.67 &0.78 \\
Shell Script &0.58 &0.58 &0.64 &0.67 \\
TypeScript &0.72 &0.82 &0.88 &0.86 \\
\midrule
Average &0.61 &0.68 &0.76 &0.80 \\
\bottomrule
\end{tabular}
}
\end{table}

Table~\ref{tab:sensitivity_languages} shows the generalization performance of the approaches in assessing the code correctness across different programming languages. This experiment is conducted on the multilingual version of \textit{HumanEval} benchmark.
For each task in a target language, \deepseek-6.7B is employed to generate 10 candidate solutions. 
To evaluate cross-language generalization, one programming language is held out for testing, while the classifiers are trained on either the generated code (for post-hoc classifications) or internal representations (for \openia and \tool) from the remaining languages. 

Overall, \textit{\tool exhibits strong generalization capability across diverse programming languages}. \tool achieves the highest average F1-Score of 0.80, outperforming \openia by 5\% and the black-box methods by up to 32\%. 
Notably, for Python programs, \tool obtains an F1-score of 0.78, 
improving over the baselines by margins ranging from 17\% to 111\%.

Moreover, \textit{white-box approaches, such as \tool and \openia, demonstrate stability across languages}, maintaining their performance within a relatively narrow F1-score range. In contrast, black-box models, like CodeBERT and CodeT5+, show greater variability, with performance dropping significantly for certain languages. 
For example, while CodeT5+ performs competitively on TypeScript (F1-score of 0.82), its performance declines sharply to just 0.50 on Python. 
The reason is that black-box approaches operate on final code outputs; thus, they could be sensitive to language-specific characteristics such as keywords, syntax, and conventions. Meanwhile, white-box approaches leverage LLMs' internal representations, allowing them to capture deeper semantic understanding rather than relying on shallow surface-level features such as token sequences.
As a result, the white-box approaches, like \tool or \openia, can exhibit greater robustness across languages and programming paradigms.