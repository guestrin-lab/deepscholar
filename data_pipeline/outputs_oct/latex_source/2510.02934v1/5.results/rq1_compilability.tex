\subsubsection{Compilability Assessment}


% \input{tables/tab_performance_compilability_standalone}

% \input{tables/tab_performance_compilability_repo}


\begin{table*}\centering
\caption{{Correctness assessment performance in \textit{\textbf{compilability}} criterion }}
\label{tab:rq1_performance_compilability}
%\resizebox{ extwidth}{!}{ % use this if the table is too large
\begin{tabular}{l|l|rr|rrr}\toprule
Code LLM& &\multicolumn{2}{c}{Independent-unit code} &\multicolumn{2}{|c}{Repo-level code} \\\cmidrule{3-6}
 & &Accuracy &F1-Score &Accuracy &F1-Score \\
 \midrule
\multirow{7}{*}{\deepseek-1.3B} 

&\cellcolor[HTML]{ffe599}\oracle &\cellcolor[HTML]{ffe599}0.90 &\cellcolor[HTML]{ffe599}0.86 &\cellcolor[HTML]{ffe599}0.67 &\cellcolor[HTML]{ffe599}0.67  \\
&\inhouse &0.53 &0.37 &0.48 &0.32 \\
&\external &\textbf{0.91} &\textbf{0.89} &0.55 &0.50 \\
&CodeBERT &0.84 &0.83 &0.61 &0.60 \\
&CodeT5+ &0.88 &0.84 &0.61 &0.59 \\
&\openia &0.88 &0.85 &0.60 &0.59 \\
% &\oracle &\underline{0.90} &\underline{0.86} &\textbf{0.67} &\textbf{0.67} \\
&\tool &\text{0.89} &\text{0.85} &\textbf{0.65} &\textbf{0.65} \\


\midrule
\multirow{7}{*}{\deepseek-6.7B} 
&\cellcolor[HTML]{ffe599}\oracle &\cellcolor[HTML]{ffe599}0.96 &\cellcolor[HTML]{ffe599}0.94 &\cellcolor[HTML]{ffe599}0.68 &\cellcolor[HTML]{ffe599}0.68  \\

&\inhouse &0.76 &0.65 &0.61 &0.46 \\
&\external &0.94 &0.94 &0.61 &0.55 \\
&CodeBERT &0.94 &0.93 &0.57 &0.58 \\
&CodeT5+ &\textbf{0.96} &0.94 &0.58 &0.59 \\
&\openia &\textbf{0.96} &0.94 &0.61 &0.61 \\
% &\oracle &\textbf{0.96} &0.94 &\textbf{0.68} &\textbf{0.68} \\
&\tool &\textbf{0.96} &\textbf{0.95} &\textbf{0.64} &\textbf{0.64} \\

\midrule
\multirow{7}{*}{\codellama-7B} 
&\cellcolor[HTML]{ffe599}\oracle &\cellcolor[HTML]{ffe599}0.90 &\cellcolor[HTML]{ffe599}0.86 &\cellcolor[HTML]{ffe599}0.66 &\cellcolor[HTML]{ffe599}0.66  \\

&\inhouse &0.39 &0.22 &0.58 &0.42 \\
&\external &\textbf{0.91} &\textbf{0.88} &0.60 &0.55 \\
&CodeBERT &0.87 &0.84 &0.55 &0.55 \\
&CodeT5+ &0.56 &0.65 &0.60 &0.59 \\
&\openia &0.88 &0.86 &0.60 &0.61 \\
% &\oracle &0.90 &0.86 &\textbf{0.66} &\textbf{0.66} \\
&\tool &0.90 &0.86 &\textbf{0.66} &\textbf{0.66} \\

\midrule
\multirow{7}{*}{\codellama-13B} 
&\cellcolor[HTML]{ffe599}\oracle &\cellcolor[HTML]{ffe599}0.90 &\cellcolor[HTML]{ffe599}0.88 &\cellcolor[HTML]{ffe599}0.70 &\cellcolor[HTML]{ffe599}0.71  \\

&\inhouse &0.41 &0.23 &0.57 &0.42 \\
&\external &\textbf{0.91} &\textbf{0.91} &0.60 &0.54 \\
&CodeBERT &0.81 &0.79 &0.60 &0.59 \\
&CodeT5+ &0.90 &0.88 &0.62 &0.62 \\
&\openia &0.87 &0.87 &0.66 &0.66 \\
% &\oracle &0.90 &0.88 &\textbf{0.70} &\textbf{0.71} \\
&\tool &0.90 &0.87 &\textbf{0.69} &\textbf{0.69} \\

\midrule
\multirow{7}{*}{\magiccoder-7B} 
&\cellcolor[HTML]{ffe599}\oracle &\cellcolor[HTML]{ffe599}0.96 &\cellcolor[HTML]{ffe599}0.94 &\cellcolor[HTML]{ffe599}0.68 &\cellcolor[HTML]{ffe599}0.69  \\

&\inhouse &0.73 &0.62 &0.60 &0.45 \\
&\external &0.95 &\textbf{0.95} &\textbf{0.66} &0.63 \\
&CodeBERT &0.94 &0.93 &0.58 &0.58 \\
&CodeT5+ &0.88 &0.90 &0.57 &0.57 \\
&\openia &\textbf{0.96} &0.94 &0.61 &0.61 \\
% &\oracle &\textbf{0.96} &0.94 &\textbf{0.68} &\textbf{0.69} \\
&\tool &\textbf{0.96} &0.94 &\textbf{0.66} &\textbf{0.67} \\

\midrule
\multirow{7}{*}{\codegemma-7B} 
&\cellcolor[HTML]{ffe599}\oracle &\cellcolor[HTML]{ffe599}0.88 &\cellcolor[HTML]{ffe599}0.86 &\cellcolor[HTML]{ffe599}0.68 &\cellcolor[HTML]{ffe599}0.68  \\

&\inhouse &0.44 &0.27 &0.52 &0.36 \\
&\external &0.88 &0.86 &0.55 &0.49 \\
&CodeBERT &0.81 &0.82 &0.58 &0.56 \\
&CodeT5+ &0.78 &0.80 &0.61 &0.60 \\
&\openia &0.84 &0.79 &0.67 &0.67 \\
% &\oracle &0.88 &0.86 &\underline{0.68} &\underline{0.68} \\
&\tool &\textbf{0.90} &\textbf{0.89} &\textbf{0.69} &\textbf{0.69} \\
\bottomrule
\end{tabular}
\end{table*}

Table~\ref{tab:rq1_performance_compilability} shows the results of the compilability assessment under two levels of code generation granularity, \textit{independent-unit} code and \textit{repo-level} code. As seen, \textit{most correctness assessment approaches achieve strong performance in \textit{independent-unit} code, yet show considerable decline in the more complex \textit{repo-level} setting}. Among them, \textit{\tool demonstrates the highest robustness, maintaining the smallest performance drop between the two settings.}



Specifically, in the \textit{independent-unit} setting, \external achieves the highest performance, with an average accuracy of 0.92, which is 69\% higher than \inhouse and 11\% higher than black-box classification with CodeT5+. However, its performance drops by 1.5 times, reaching an average accuracy of 0.59 in the \textit{repo-level} setting. 
%
Indeed, determining whether a piece of code is compilable is mainly dependent on checking syntax and structural correctness, without the need for deep reasoning. 
In \textit{independent-unit} snippets, all relevant information, such as imports, declared variables, and function definitions, is self-contained, allowing LLMs like \gpt to simply scan the code to verify its syntactic validity. 
Meanwhile, \textit{repo-level} code often references variables, functions, or modules defined across multiple files. Without explicit access to the entire project context, the LLM struggles to accurately determine compilability, leading to a significant performance drop in this setting.

In the more complex setting of \textit{repo-level} code assessment, \tool achieves the highest performance, outperforming \external by 12\% on average. 
For example, when evaluating the compilability of code generated by \codegemma-7B, both \tool and \external reach the accuracy of 0.90 on \textit{independent-unit} code. 
However, for \textit{repo-level} code, \tool obtains an accuracy of 0.69, which is 27\% higher than that of \external. Interestingly, in this case, \tool even slightly surpasses the performance of \oracle.
%
These results demonstrate that \tool is less sensitive to the increased code complexity, making it a more reliable choice for practical correctness assessment scenarios.