\section{Outlook and Conclusion}
\label{sec:conclusion}

We presented a large-scale dataset for visual defect and anomaly detection in retail logistics. Comprising 238,421 images including 29,316 defective samples, it captures challenges of retail logistics processes and represents one of the largest and most diverse datasets of its kind.  The dataset overcomes critical limitations in existing benchmarks and enables the research community to address the remaining challenges in visual defect and anomaly detection.
It allows for benchmarking methods in various scenarios, with and without training and reference images. We demonstrate the complexity of the proposed task by evaluating a number of state-of-the-art approaches and highlight the need for more robust solutions, particularly in anomaly-detection settings.

This dataset marks a significant step towards developing defect detection systems capable of handling real-world scenarios, setting a new standard for research in retail logistics applications of visual inspection. We encourage future research to explore the dataset by developing novel approaches. Key questions for future work include but are not limited to:
(1) How can anomaly detection methods be generalized to deal with significant item and pose variability?
(2) How can methods effectively leverage both training data and reference images?
(3) How can we create methods that not only detect defects but also explain their reasoning?
