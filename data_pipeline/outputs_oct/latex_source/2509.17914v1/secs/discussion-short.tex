
We demonstrate XaaS containers with representative HPC applications.
%
However, modern HPC workloads are often not limited to a single application~\cite{9309042}. 
%
Large HPC workflows like MOFA~\cite{yan2025mofadiscoveringmaterialscarbon} are built from several different tasks, each with its own requirements for CPU and GPU computation.
%
Performance-portable containers could create a seamless deployment of a heterogeneous workflow across different HPC hardware.
%
To transition the IR format to large and complex applications, we need to support dependency management (Section~\ref{sec:xaas-dependency}) and software installation (Section~\ref{sec:xaas-installation}).

\vspace{-1em}

\subsection{Dependency Management}
\label{sec:xaas-dependency}
%
When building different versions of an application, we should not repeat the entire build step for all dependencies.
%
Instead, dependencies should be composed into a final package, as is already the case for package managers such as Spack~\cite{10.1145/2807591.2807623}.
%
Standard containers distribute dependencies as binary images assembled for the selected specialization.
%
In IR containers, each dependency should be distributed in the IR form.
%
However, they must be located during the configuration phase of IR containers, and build parameters are affected by compilation and linking flags of dependencies.
%
This leads to a conflict: we want to deploy partially compiled applications, but still provide installation configuration to compose IR containers. %the generation of build configurations.

To support composability, future work can include creating \emph{fat} binaries with embedded IR, similarly to existing approaches for deploying GPU device code in CUDA and SYCL~\cite{cudacompilationtrajectory}.
%
This approach will generate a full installation target, allowing for seamless operation of build systems, while providing the necessary metadata and IRs to optimize and regenerate the dependency for the target.
%
Furthermore, extended dependency management could support version constraints, similarly to existing solutions in package managers.
%
This feature will restrict the matching process of specialization points, and prevent build failures caused by incompatibilities between the containerized application and its dependencies.
\vspace{-1em}

\subsection{Installation}
\label{sec:xaas-installation}
%
IR containers include the application's source code, even if the entire application is compiled to an intermediate representation.
%
This is not a strict requirement of our method but a limitation of existing build systems.
%
To finalize the application build, we need to perform linking and installation.
%
However, these steps can include many user-defined and customized instructions, such as generating custom headers, and they cannot be easily extracted from the build configuration.
%
Consequently, the IR image embeds all build configurations. % or recreates the selected one during deployment.
%
Automating installation would reduce the complexity of the container and solve the problem of IR container composability.
%
Furthermore, we can deduplicate installation targets as currently done with IR files.
%
Then, the IR container will only need to contain a shared installation core and a delta for each build configuration.

