
To fundamentally change the way we distribute HPC software, we first need to understand how \emph{specialization points} affect the build and installation process (Section~\ref{sec:containers-build-specialization}).
%
Since discovering specialization points is complex due to the lack of standardization in build systems, we apply semi-automatic detection with the help of artificial intelligence (Section~\ref{sec:containers-specialization-discovery}).
%
By detecting specialization points, we can design performance-portable containers that delay the impact of specialization until we know the final specification (Section~\ref{sec:ir-container}).

\begin{figure*}[t!]
    \centering
    \begin{minipage}[t]{0.4\textwidth}
        \centering
        \begin{minted}[fontsize=\footnotesize]{json}
{
  "gpu_build": {"value": true, "build_flag": "-DGMX_GPU"},
  "gpu_backends": {
    "CUDA": {"min_version": "12.1", "flag": "-DGMX_GPU=CUDA"},
    "HIP": {"min_version": "5.4.3", "flag": "-DGMX_GPU=HIP"}
  },
  "vectorization": {
    "None": {"flag": "-DGMX_SIMD=None"},
    "SSE4.1": {"flag": "-DGMX_SIMD=SSE4.1"},
    "AVX2_256": {"flag": "-DGMX_SIMD=AVX2_256"},
    "AVX_512": {"flag": "-DGMX_SIMD=AVX_512"},
    "ARM_NEON_ASIMD": {"flag": "-DGMX_SIMD=ARM_NEON_ASIMD"}
}}
        \end{minted}
        \textbf{(a)} Specialization Points
    \end{minipage}
    \hspace{0.55cm}
    \begin{minipage}[t]{0.23\textwidth}
        \centering
        \begin{minted}[fontsize=\footnotesize]{json}
 {
  "CPU Info": {
    "Architecture": "x86_64",
    "Vectorization": [
     "avx512f", "avx", "avx2", "sse4_1"
    ]
  },
  "GPU Backends": {
    "CUDA": { "version": "12.1",
      "lib": ["/lib/libcuda.so.1"], 
    }
  }
 }
        \end{minted}
        \textbf{(b)} System Features
    \end{minipage}
    \hspace{0.05cm}
    %\hfill
    \begin{minipage}[t]{0.25\textwidth}
        \centering
        \begin{minted}[fontsize=\footnotesize ]{json}
{
  "common_specialization": {
    "vectorization_flags": {
      "SSE4.1": "-DGMX_SIMD=SSE4.1",
      "AVX_512": "-DGMX_SIMD=AVX_512",
      "AVX2_256": "-DGMX_SIMD=AVX2_256"
    },
    "gpu_backends": {
      "CUDA": { "version": "12.1",
        "flag": "-DGMX_GPU=CUDA",
      },
    }}
}
        \end{minted}
        %\vspace{8.3mm}
        \textbf{(c)} Common Specialization Points
    \end{minipage}
    \caption{Simplified example of GROMACS' specialization points and how the checker determines the intersection of specialization points.}
    \vspace{-1em}
    \label{fig:specialization-points}
\end{figure*}


\subsection{HPC Specialization}
\label{sec:containers-build-specialization}

The build process of an application can be split into three major parts:
\textbf{configuration} that resolves dependencies and decides what should be built, and how;
\textbf{compilation and linking}, responsible for turning source files into libraries and executables;
and \textbf{installation}, which places headers, binaries, and project resources in a selected destination.
%
To create a transparent and seamless experience for HPC users, any solution must support all three steps.

During \textbf{configuration}, source modules and files are enabled or disabled depending on the selected specialization.
%
Compiler flags are adjusted, and the build system adds compile-time definitions embedded into the application.
%
Paths to dependencies are resolved, and additional packages can be fetched into the build directory.

Once source files are \textbf{compiled}, headers of chosen libraries will be introduced, preprocessing directives are applied, and compile-time definitions like C++ templates are resolved.
%
After that stage, we can no longer switch between libraries that are not ABI-compatible since the application has been introduced to types with different representations and functions with incompatible signatures.
%
Furthermore, preprocessing directives can potentially exclude certain code paths and already decide which kernels will be generated, as shown in the example of matrix transpose in BLAS libraries (Figure~\ref{fig:transpose-implementations}).
%
Since this operation is not standardized, different implementations are needed, but they can only be enabled if the selected library is present in the system.

Once the source files are translated into the intermediate representation and optimized, the ISA is chosen, processor-specific decisions are made, and the final code is emitted.
%
At this point, the code is no longer portable between different systems.
%
Furthermore, it is no longer feasible to change vectorization settings or apply optimizations
valid only on specific types of CPUs.

At the \textbf{linking} stage, applications are relinked to a specific implementation of a dependency.
%
This decision can be changed later, as long as the library is linked dynamically and its implementations are ABI compatible.
%
For example, an application compiled against MPICH can be relinked to use Cray's specialized MPICH implementation.
%
While future MPI implementations will be ABI compatible~\cite{10.1145/3615318.3615319}, this method is currently limited since MPI types can have different implementations.
%
After that point, the only possible performance adjustments are runtime options, such as switching network providers in applications built on top of libfabric.

Finally, the application is \textbf{installed}, which includes copying the contents of the package.
%
Specialization affects the generation of project-specific headers and the installation of libraries, since the inclusion of specific dependencies is affected by user decisions.


\subsection{Specialization Discovery}
\label{sec:containers-specialization-discovery}
%
To generate the list of specialization points an application supports, we need to parse build scripts and understand what dependencies and optimizations can be selected during configuration.
%
Unfortunately, this process is not standardized in common HPC programming languages, like C++ and Fortran.
%
In addition to supporting different build systems such as autotools, handwritten Makefiles, CMake, Bazel, or custom scripts, there is often no single way of determining dependencies within one ecosystem.
%
For example, third-party libraries can be located in CMake using standard CMake calls such as \texttt{find\_package}, with custom find modules for libraries not supported by CMake, by using \texttt{pkg-config}, or with a manual search for specific headers and libraries.
%
Moreover, large projects often define custom routines for locating packages.

Analyzing configuration files to identify specialization points is difficult to automate due to the many diverse and unique patterns.
%
At the same time, it is a task that humans can handle easily. 
%
Thus, we employ a Large Language Model (LLM) to help users identify specialization points by processing the project configuration files with a structured prompt.
%
We apply \emph{in-context learning} by including in the prompt examples of specialization options, build flags, and CMake commands, helping the LLM to extract specialization options accurately and capture all relevant choices in the build file.
%
The model outputs a JSON file containing the detected specialization points. 
%
To enforce consistency and facilitate automated processing, we supply a predefined JSON schema, guiding the model to adhere to a structured format and minimizing anomalies. % in its responses.
%
As the accuracy and correctness of LLM systems vary heavily, the results of LLM extraction still serve mainly as a guideline for the developer to prepare the final specification (Section~\ref{sec:eval-llms}).

On the target system, we collect information on system features and available specialization points.
%
Then, we intersect these results with the specialization discovery of the application.
%
At this point, we exclude the non-supported configuration options and present the user with a list of options for each specialization point.
%
Figure~\ref{fig:specialization-points} illustrates a subset of GROMACS’s specialization points alongside the system features of our test environment.
%
GROMACS supports OpenCL, SYCL, HIP, and CUDA as GPU backends, whereas the system is limited to CUDA and OpenCL.
%
The automatic checker identifies the intersection of supported GPU backends, and allows the user to manually select the final specialization points.

