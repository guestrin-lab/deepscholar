
\begin{table*}[t!]
  \caption{Performance and cost of LLMs parsing GROMACS configuration. Token counts, latency, and estimated cost are averaged from 10 runs executed from Z{\"u}rich, Switzerland. F1, precision, and recall metrics are aggregated across all runs and reported as Min/Median/Max per model.}
\centering
\label{tab:LLMs-compact}
\renewcommand{\arraystretch}{0.95}
\setlength{\tabcolsep}{4pt}
\begin{adjustbox}{max width=\textwidth}
\begin{tabular}{@{}l l l l l 
                r r r   % F1
                r r r   % Precision
                r r r@{} % Recall
              }
\toprule
%\textbf{Model Version} & \textbf{Avg Token(In)} & \textbf{Avg Token(Out)} & \textbf{Avg Time (s)} & \textbf{Cost (\$)} & 
\textbf{Model Version} & \textbf{Tokens} & \textbf{Tokens Out} & \textbf{Time (s)} & \textbf{Cost (\$)} & 
F1$_\text{min}$ & F1$_\text{med}$ & F1$_\text{max}$ & 
P$_\text{min}$ & P$_\text{med}$ & P$_\text{max}$ & 
R$_\text{min}$ & R$_\text{med}$ & R$_\text{max}$ \\
\midrule

gemini-flash-1.5-exp & 15803 ± 0 & 2333.5 ± 147.6 & 16.40 ± 1.00 & 0.002 & 
0.863 & 0.902 & 0.942 & 
0.838 & 0.898 & 0.948 & 
0.868 & 0.909 & 0.936 \\

gemini-flash-2-exp & 15803 ± 0 & 2610.8 ± 189.4 & 11.96 ± 0.86 & 0.003 & 
0.873 & 0.978 & 0.994 & 
0.873 & 0.981 & 1.000 & 
0.873 & 0.981 & 0.988 \\

claude-3-5-haiku-20241022 & 17841 ± 0 & 1568.9 ± 174.2 & 20.09 ± 1.96 & 0.021 & 
0.628 & 0.672 & 0.702 & 
0.807 & 0.863 & 0.894 & 
0.5 & 0.539 & 0.622 \\

claude-3-5-sonnet-20241022 & 17841 ± 0 & 1528.7 ± 39.2 & 126.18 ± 335.31 & 0.077 & 
0.661 & 0.672 & 0.692 & 
0.875 & 0.878 & 0.882 & 
0.539 & 0.544 & 0.57 \\

claude-3-7-sonnet-20250219 & 17841 ± 0 & 3122.7 ± 155.1 & 50.29 ± 21.67 & 0.100 & 
0.878 & 0.883 & 0.911 & 
0.857 & 0.868 & 0.923 & 
0.9 & 0.9 & 0.9 \\

o3-mini-2025-01-31 & 13538 ± 0 & 8003.9 ± 1160.8 & 108.40 ± 40.02 & 0.050 & 
0.559 & 0.924 & 0.968 & 
0.549 & 0.909 & 0.987 & 
0.57 & 0.923 & 1.000 \\

gpt-4o-2024-08-06 & 13539 ± 0 & 1540.0 ± 146.1 & 26.06 ± 6.96 & 0.049 & 
0.547 & 0.774 & 0.887 & 
0.508 & 0.892 & 0.979 & 
0.53 & 0.675 & 0.859 \\

\bottomrule
\end{tabular}
\end{adjustbox}
\end{table*}


In the evaluation, we focus on the following research questions:
\begin{itemize}[topsep=0pt]
	\item Can LLM systems process large C++ project configurations?
	\item Can source containers provide performance portability?
	\item Can IR containers perform better than portable containers?
	\item Can IR containers optimize deployments?
\end{itemize}

\vspace{-1em}

\subsection{Benchmarking Setup}

We demonstrate the portability of our containers on three systems:
\begin{itemize}[topsep=0pt]
	\item \textbf{CSCS Ault} Heterogenous system with Sarus~\cite{10.1007/978-3-030-34356-9_5} containers. We use Ault23 (Intel 6130 CPU, V100 GPU), Ault25 (AMD EPYC 7742, A100 GPU), and Ault01-04 (Intel 6154 CPU).
	      %with a RoCE network and
	\item \textbf{CSCS Alps.Clariden} Cray system with the GH200 superchip, Cray Slingshot network, and Podman~\cite{10030014} containers.
	\item \textbf{Aurora} Cray system with Intel Xeon CPU Max CPUs, Intel Data Center GPU Max, and Apptainer~\cite{10.1371/journal.pone.0177459} containers.
\end{itemize}

The deployment images for Clariden are built on the compute nodes, and we use a local development machine with Docker for Ault23 and Aurora, as neither system supports container building.

\indent We consider two high-performance applications as case studies.

\noindent\textbf{GROMACS 2025.0}~\cite{https://doi.org/10.1002/jcc.20291}
%
supports many vectorization options, with automatic detection of the best option on the system.
%
In addition to MPI and OpenMP, it contains mutually exclusive GPU backends. % for CUDA, SYCL, and OpenCL.

\noindent\textbf{llama.cpp}~\cite{llamacpp}
%
The C++ LLM inference engine achieves good portability by separating the implementation into multiple backends, which can be loaded dynamically at runtime.


\subsection{Specialization Discovery}
\label{sec:eval-llms}
%
We propose automating the process of discovering specialization points with the help of LLMs.
%
Given LLM's tendency to hallucinate and produce incorrect output, this raises the question: can these results be trusted?
%
Thus, we evaluate the analysis of GROMACS with models from OpenAI, Anthropic, and Google.
%
For each model, we apply \emph{in-context learning} by providing examples from CMake configuration options, internal commands, and flags for GROMACS, Quantum Espresso, and Kokkos.
%
We normalize the structure of specialization points, and compare specializations found in the ground truth and LLM result, counting true/false positives and negatives.
%
We repeat the prompt 10 times and evaluate the characterization of the build system, vectorization, FFT and linear algebra libraries, parallel computing solutions, and GPU backends.
%
Results in Table~\ref{tab:LLMs-compact} show that the correctness and performance vary widely between models.
%
Both GPT models produce inconsistent results across repetitions (F-score 0.55--0.97).
%
Gemini models perfom best, which can be explained by the large context window.
%
Examples of failures include returning only a subset of options (Claude 3.5, GPT-4o) and mixing FFT and linear algebra libraries (GPT-4o, Gemini 1.5).

% measured with gemini tokenizer
The basic CMake configuration contains 13299 tokens, while all CMake scripts add 154,946 more tokens.
%
Recommended configurations are described in the documentation that adds almost 1M tokens (without plots), which might be out of reach for many current models~\cite{liu-etal-2024-lost}.
%
Documentation can be processed iteratively, but this would significantly increase processing time and further increase the uncertainty of the final result.
%
Thus, while in-context learning can enable LLM models to perform well in the automatic analysis of specialization points,
the processing pipeline requires tight human supervision and corrections, and performance preferences might be
provided by system operators or application developers.

\emph{Generalization}
%
We evaluated llama.cpp for which we provide no prompt examples.
%
We pass CMake configurations of llama.cpp and its main subproject \emph{ggml} (2544 and 4574 tokens, respectively).
%
Best performing models are Sonnet 3.7 (F1 0.55--0.62) and GPT-o3 (F1 0.62--0.7).
%
Models often underperform due to minor discrepancies (inconsistent hyphen/underscore, missing \emph{-D} prefix).
%
Normalization improves performance: Sonnet 3.7 (F1 0.63--0.74), Gemini Flash 2 (F1 0.53--0.79), and GPT-o3 (F1 0.73--0.79).
%
Additionally, ggml contains over 20 optimization flags; including them in evaluation decreases overall performance while improving Sonnet's best result to 0.78.


\begin{figure}[t]
	\centering
	\includegraphics[width=\linewidth]{figures/gromacs_portability.pdf}
	\vspace{-1em}
	\caption{Performance portability of GROMACS between systems.}
	\vspace{-1em}
	\label{fig:gromacs-execution time}
\end{figure}

\begin{figure}[t]
	\centering
	\includegraphics[width=\linewidth]{figures/llama_portability.pdf}
	\vspace{-1em}
	\caption{Performance portability of llama.cpp between systems.}
	\vspace{-1em}
	\label{fig:llamacpp-execution time}
\end{figure}

\subsection{Performance Portability}
\label{eval:sec-source-containers}

We evaluate the performance portability of \toolname{} source containers by comparing them against local builds, specialized containers, and Spack packages or modules - when available.
%
We deploy different source images based on system discovery and user input results.

\subsubsection{GROMACS}
%
We execute test cases A and B from UEABS~\cite{ueabs} 30 times, and subtract the I/O overhead reported by GROMACS from timings (Figure~\ref{fig:gromacs-execution time}).
%
We use default GCC 11.4 in XaaS source images, and Spack-installed GCC 11.5 for other test cases.
%
Naive build uses the default CMake command from the documentation, which results with lack of GPU acceleration even when the CUDA module is loaded.
%
Both naive and native builds pick up MKL from the HPC modules environment.
%
On Ault23, we compare Spack installation with MPI and CUDA, and a second configuration with explicit selection of MKL, which achieves performance close to the XaaS source container.
%
According to GROMACS logs, the default Spack installation performs worse in the CPU part of the application, indicating possible issues with multithreading or the automatically selected OpenBLAS.
%
% Processing ault23 - B - gromacs_testcase4_testcaseB, samples: 30, 30
% Processing ault23 - testcase4, time: 138.48863333333335 +- 0.04346232419715079
% Processing ault23 - B - gromacs2025_testcase4_testcaseB, samples: 30, 30
% Processing ault23 - testcase4, time: 136.72633333333334 +- 0.057986244274133014
%
% Processing ault23 - B - gromacs2025_testcase5_testcaseB, samples: 30, 30
% Processing ault23 - testcase5, time: 28.391166666666667 +- 0.04138302141366523
% Processing ault23 - B - gromacs_testcase5_testcaseB, samples: 30, 30
% Processing ault23 - testcase5, time: 29.283299999999997 +- 0.04099056083102681
Spack cases uses the latest available GROMACS 2024.4, and a subsequent reevaluation of 2025.0 with test B on Ault23 demonstrated an average improvement of 1-2 seconds.

We use two baselines on Aurora: a hand-written specialized container and a module version of GROMACS 2024.5 since it cannot be installed by Spack.
%
For XaaS and specialized containers, we use the oneAPI image recommended by system operators.
%
The module version uses MPI, while other benchmarks use the internal Threads-MPI due to MPI compatibility issues in containers on Aurora (Section~\ref{sec:network_performance}).
%
However, the default source container uses only CPUs because the build is incompatible with Intel Max GPUs.
%
There, GROMACS uses a compile-time definition specialized only for this device, found in the documentation but not the build configuration.
%
For the \emph{manual} fix, source containers need an additional source of knowledge, such as documentation parsing (Section~\ref{sec:eval-llms}) or specialization parameters provided by developers (Section~\ref{sec:xaas-source-containers}).

\textbf{Portable Container.}
%
A GPU-capable container is possible only with the SYCL backend.
%
This \emph{exotic} configuration~\cite{gromacsexotic} uses the vkFFT library~\cite{10036080}, which can be compiled for one hardware backend only.
%
Instead, we used the recently added oneMath library~\cite{onemath}, which supports MKL and cuFFT simultaneously.
%
We built GROMACS with the CUDA plugin for SYCL~\cite{codeplaycudaplugin} in the oneAPI image, and compared it against our source containers on V100 (Ault23) and A100 (Ault25).
%
The SYCL container is 11\%-20\% slower on test A, and fails to run test B.
%
Portability is further limited by two factors: GPU compatibility still requires compile-time definitions, and the CUDA plugin generates code for one GPU architecture at a time.

\subsubsection{llama.cpp}
%
We use the internal llama.cpp benchmark, running 40 repetitions of prompt processing (512) and text generation (128), with a 4-bit quantized LLama-2-13B-chat model.
%
We configure source containers to compiler and library versions close to those available on the system.
%
On Aurora, we have to manually patch the source code to compile with the Intel icpx compiler, as the SYCL backend cannot be compiled with Clang 21.
%
For XaaS source containers, we switch to the official oneAPI image at deployment.
%
In this benchmark, the specialized build performs comparably to \toolname{}, while the naive default build does not enable GPU (Figure~\ref{fig:llamacpp-execution time}).

\begin{figure}[t]
	\centering
	\includegraphics[width=\linewidth]{figures/ir_comparison.pdf}
	\includegraphics[width=\linewidth]{figures/ir_gpu_comparison.pdf}
	\caption{IR containers on CPU (top, Ault01-04) and GPU (bottom).}
	\vspace{-1.5em}
	\label{fig:gromacs-ir-container}
\end{figure}

\subsection{IR Containers}
%
We evaluate the performance portability of IR containers with GROMACS on CPU, with OpenMP, and tuned against five different CPU architectures, from SSE4.1 to AVX-512.
%
This requires discovering CPU tuning flags and delaying optimizations until final deployment (Section~\ref{sec:xaas-ir-containers-pipeline}).
%
For each variant, we build a separate container layer hosting \texttt{fftw3} library tuned for that architecture.
%
Additionally, we build IR containers with CUDA 12.8 by generating IR files with embedded device code, which we later lower to the target platform.
%
We compare against portable and specialized containers built with Clang 19 for SSE4.1 and AVX\_512 (CPU), and CUDA containers (GPU).
%
Figure~\ref{fig:gromacs-ir-container} shows the runtime of GROMACS with I/O time excluded, demonstrating that a specialization of the IR container can improve performance by up to 2x when compared to a performance-oblivious container.
%
We also evaluate a separate deployment of IR containers configured against CUDA and two vectorization levels for Ault23 and Ault25 nodes.
%
On the GPU, we provide performance comparable to a specialized container, with a slight increase in I/O time for test case B.

\emph{Configurability and System Dependency}
%
This experiment validates positively Hypotheses 1 and 2 on GROMACS (Section~\ref{sec:xaas-ir-containers}).
%
To deploy five GROMACS containers tuned to different ISAs, developers must build 8710 translation units (TU).
%
In our container, we build only 2695 IRs (69\% reduction).
%
The reduction would not be possible without the \toolname{} pipeline, as 96\% of compiled source files have incompatible build flags across projects; the primary reason is the inclusion of header files in the build directory.
%
Preprocessing determines that only 14.3\% of additional TUs require separate IR compilation.
%
However, 95\% of identical targets have different CPU tuning, which is resolved by the vectorization pass of our pipeline.

Four build configurations with two vectorization settings and CUDA require 7052 TUs, which we reduce to 2694 IRs (76\% reduction).
%
The build combinations obtained with enabling OpenMP and/or MPI require compiling and lowering 6976 TUs.
%
Thanks to preprocessing and determining when the OpenMP flag has no effect, we build only 2333 IRs (66.4\% reduction).

\vspace{-0.75em}

\subsection{Network Performance}
\label{sec:network_performance}
%
We focused on single-node deployments to demonstrate specialization to the available hardware and software,
and did not evaluate distributed execution due to technical limitations on our systems.
%
On Aurora, Apptainer containers did not function with MPI, and we had to use Thread-MPI instead.
%
On Clariden, co-location of MPI ranks is needed to utilize the four GH200 chips on each node.
%
However, the intra-node MPI communication is implemented separately from \emph{cxi}, the Slingshot provider in Libfabric~\cite{https://doi.org/10.1002/cpe.8203}.
%
Thus, containerized MPIs can access the high-speed network through Libfabric replacement, but cannot use shared memory.
%
While bare-metal Cray-MPICH achieves up to 64 GB/s on the same socket, co-located containers reach only up to 23.5 GB/s (OpenMPI).
%
LinkX~\cite{pritchard2023open} is a Libfabric provider that combines remote and local communication, and provides up to 64 (MPICH) and 70 (OpenMPI) GB/s of intra-node bandwidth.
%
However, the provider is still experimental, e.g., it does not function well on all benchmarks and hardware.

