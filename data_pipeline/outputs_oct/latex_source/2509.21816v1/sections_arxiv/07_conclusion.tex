\section{Conclusion}

In this paper, we formalize the problem of task-to-tutorial generation, develop an evaluation protocol covering eighteen key aspects, and present a dataset comprising 1,559 diverse Excel tasks. To address this problem, we propose the first framework capable of automatically generating Excel tutorials (including documents and videos) solely from task descriptions. Our framework adopts a three-stage strategy consisting of (1) task instantiation, (2) automatic trajectory collection, and (3) tutorial generation. Experimental results demonstrate that our ExeAgent significantly outperforms baseline methods in terms of the success rate of Excel task completion, achieving an improvement of 8.5\%. Both human evaluation and LLM evaluation confirm the high quality of the generated tutorials. Furthermore, a case study indicates that our tutorials match or occasionally exceed the quality of expert-authored materials. Finally, cost analysis reveals that our approach substantially reduces both time and monetary expenses compared with manual tutorial creation.


% In this work, we present the Excel Guide Agent, an autonomous system for automatically generating Excel tutorials. Our approach takes user queries as input, generates and executes solutions within matched template environments, and records reasoning traces and key screenshots. These are subsequently processed into structured help documents and instructional videos. For documentation, the system automatically produces concise and readable titles and descriptions for each step, aligned with corresponding screenshots, and outputs them in multiple formats. For video tutorials, it generates step titles, narration scripts, and subtitles, ensuring clarity and accessibility through adaptive layout, non-destructive image processing, and precise audio–subtitle synchronization. A user study with ten participants demonstrates that our tutorials significantly improve task efficiency and learning experience, outperforming expert-written tutorials. Cost analysis further confirms the advantages of our approach in terms of both time and monetary efficiency.









% \begin{figure}
%     \centering
%     \begin{tikzpicture}
%         \begin{axis}[
%             ybar stacked,
%             bar width=0.3cm,
%             enlarge x limits=0.15,
%             ymin=0, ymax=100,
%             ylabel={Percentage (\%)},
%             xtick=data,
%             xticklabels={Understandability, Task Completion, Efficiency, Sequential Order, Satisfaction, Preference, Text-Image Mapping, Correctness, Cohesiveness, Completeness, Clarity},
%             xticklabel style={align=center, rotate=45, anchor=east},
%             legend style={at={(0.5,1.03)}, anchor=south, legend columns=-1},
%         ]
%         \addplot+[ybar, fill=color1,draw=black] coordinates {(0,65) (1,59) (2,55) (3,75) (4,55) (5,77) (6,57) (7,74) (8,65) (9,57) (10,68)};
%         \addplot+[ybar, fill=color2,draw=black] coordinates {(0,10) (1,11) (2,7) (3,10) (4,12) (5,5) (6,15) (7,10) (8,7) (9,14) (10,7)};
%         \addplot+[ybar, fill=color3,draw=black] coordinates {(0,8) (1,8) (2,10) (3,5) (4,15) (5,8) (6,10) (7,6) (8,8) (9,9) (10,10)};
%         \addplot+[ybar, fill=color4,draw=black] coordinates {(0,8) (1,8) (2,11) (3,7) (4,10) (5,7) (6,10) (7,6) (8,12) (9,4) (10,8)};
%         \addplot+[ybar, fill=color5,draw=black] coordinates {(0,9) (1,14) (2,17) (3,3) (4,8) (5,3) (6,8) (7,4) (8,8) (9,16) (10,7)};
%         \legend{1 (bad), 2, 3, 4, 5 (good)}
%         \end{axis}
%     \end{tikzpicture}
%     \caption{Stacked Bar Chart Example}
% \end{figure}




% \begin{figure}
%     \centering
%     \begin{tikzpicture}
%         \begin{axis}[
%             ybar stacked,
%             bar width=0.25cm, % 减小柱子宽度
%             enlarge x limits=0.08, % 调整x轴边距以适应分组
%             ymin=0, ymax=100,
%             ylabel={Percentage (\%)},
%             xtick=data,
%             xticklabels={Understandability, Task Completion, Efficiency, Sequential Order, Satisfaction, Preference, Text-Image Mapping, Correctness, Cohesiveness, Completeness, Clarity},
%             xticklabel style={align=center, rotate=45, anchor=east},
%             legend style={at={(0.5,1.05)}, anchor=south, legend columns=-1},
%         ]

%         % --- 第一组柱状图 (向左偏移) ---
%         \addplot+[ybar, fill=color1, draw=black, bar shift=-0.15cm] coordinates {(0,65) (1,59) (2,55) (3,75) (4,55) (5,77) (6,57) (7,74) (8,65) (9,57) (10,68)};
%         \addplot+[ybar, fill=color2, draw=black, bar shift=-0.15cm] coordinates {(0,10) (1,11) (2,7) (3,10) (4,12) (5,5) (6,15) (7,10) (8,7) (9,14) (10,7)};
%         \addplot+[ybar, fill=color3, draw=black, bar shift=-0.15cm] coordinates {(0,8) (1,8) (2,10) (3,5) (4,15) (5,8) (6,10) (7,6) (8,8) (9,9) (10,10)};
%         \addplot+[ybar, fill=color4, draw=black, bar shift=-0.15cm] coordinates {(0,8) (1,8) (2,11) (3,7) (4,10) (5,7) (6,10) (7,6) (8,12) (9,4) (10,8)};
%         \addplot+[ybar, fill=color5, draw=black, bar shift=-0.15cm] coordinates {(0,9) (1,14) (2,17) (3,3) (4,8) (5,3) (6,8) (7,4) (8,8) (9,16) (10,7)};
        
%         % --- 第二组柱状图 (向右偏移) ---
%         % 注意:这里的数据是作为示例虚构的
%         % 'forget plot' 选项可以防止在图例中重复出现
%         \addplot+[ybar, fill=color1, draw=black, bar shift=0.15cm, forget plot] coordinates {(0,55) (1,62) (2,60) (3,70) (4,65) (5,80) (6,60) (7,70) (8,70) (9,65) (10,72)};
%         \addplot+[ybar, fill=color2, draw=black, bar shift=0.15cm, forget plot] coordinates {(0,15) (1,10) (2,10) (3,12) (4,10) (5,8) (6,12) (7,12) (8,10) (9,10) (10,10)};
%         \addplot+[ybar, fill=color3, draw=black, bar shift=0.15cm, forget plot] coordinates {(0,10) (1,9) (2,8) (3,8) (4,8) (5,5) (6,10) (7,8) (8,5) (9,8) (10,8)};
%         \addplot+[ybar, fill=color4, draw=black, bar shift=0.15cm, forget plot] coordinates {(0,10) (1,9) (2,10) (3,5) (4,9) (5,4) (6,8) (7,5) (8,8) (9,7) (10,5)};
%         \addplot+[ybar, fill=color5, draw=black, bar shift=0.15cm, forget plot] coordinates {(0,10) (1,10) (2,12) (3,5) (4,8) (5,3) (6,10) (7,5) (8,7) (9,10) (10,5)};
        
%         \legend{1 (bad), 2, 3, 4, 5 (good)}
%         \end{axis}
%     \end{tikzpicture}
%     \caption{Grouped Stacked Bar Chart Example}
% \end{figure}


% \begin{figure}[htbp] % 使用 [htbp] 提供更好的浮动定位
%     \centering
%     % 为了让标签显示完全,可以缩小字体
%     \small 
%     \begin{tikzpicture}
%         \begin{axis}[
%             ybar stacked,
%             bar width=3pt, % 使用 pt (磅) 作为单位更常见
%             enlarge x limits=0.08,
%             ymin=0, ymax=100,
%             ylabel={Percentage (\%)},
%             % --- 使用 symbolic x coords ---
%             symbolic x coords={Understandability, Task Completion, Efficiency, Sequential Order, Satisfaction, Preference, Text-Image Mapping, Correctness, Cohesiveness, Completeness, Clarity},
%             xtick=data, % 告诉 pgfplots 在每个 symbolic coord 位置画刻度
%             xticklabel style={align=center, rotate=45, anchor=east},
%             legend style={at={(0.5,1.05)}, anchor=south, legend columns=-1},
%         ]

%         % --- 第一组柱状图 (向左偏移) ---
%         \addplot+[fill=color1, draw=black, bar shift=-6pt] coordinates {
%             (Understandability, 65) (Task Completion, 59) (Efficiency, 55) (Sequential Order, 75) (Satisfaction, 55) (Preference, 77) (Text-Image Mapping, 57) (Correctness, 74) (Cohesiveness, 65) (Completeness, 57) (Clarity, 68)
%         };
%         \addplot+[fill=color2, draw=black, bar shift=-6pt] coordinates {
%             (Understandability, 10) (Task Completion, 11) (Efficiency, 7) (Sequential Order, 10) (Satisfaction, 12) (Preference, 5) (Text-Image Mapping, 15) (Correctness, 10) (Cohesiveness, 7) (Completeness, 14) (Clarity, 7)
%         };
%         \addplot+[fill=color3, draw=black, bar shift=-6pt] coordinates {
%             (Understandability, 8) (Task Completion, 8) (Efficiency, 10) (Sequential Order, 5) (Satisfaction, 15) (Preference, 8) (Text-Image Mapping, 10) (Correctness, 6) (Cohesiveness, 8) (Completeness, 9) (Clarity, 10)
%         };
%         \addplot+[fill=color4, draw=black, bar shift=-6pt] coordinates {
%             (Understandability, 8) (Task Completion, 8) (Efficiency, 11) (Sequential Order, 7) (Satisfaction, 10) (Preference, 7) (Text-Image Mapping, 10) (Correctness, 6) (Cohesiveness, 12) (Completeness, 4) (Clarity, 8)
%         };
%         \addplot+[fill=color5, draw=black, bar shift=-6pt] coordinates {
%             (Understandability, 9) (Task Completion, 14) (Efficiency, 17) (Sequential Order, 3) (Satisfaction, 8) (Preference, 3) (Text-Image Mapping, 8) (Correctness, 4) (Cohesiveness, 8) (Completeness, 16) (Clarity, 7)
%         };
        
%         % --- 第二组柱状图 (向右偏移) ---
%         \addplot+[fill=color1, draw=black, bar shift=6pt, forget plot] coordinates {
%             (Understandability, 55) (Task Completion, 62) (Efficiency, 60) (Sequential Order, 70) (Satisfaction, 65) (Preference, 80) (Text-Image Mapping, 60) (Correctness, 70) (Cohesiveness, 70) (Completeness, 65) (Clarity, 72)
%         };
%         \addplot+[fill=color2, draw=black, bar shift=6pt, forget plot] coordinates {
%             (Understandability, 15) (Task Completion, 10) (Efficiency, 10) (Sequential Order, 12) (Satisfaction, 10) (Preference, 8) (Text-Image Mapping, 12) (Correctness, 12) (Cohesiveness, 10) (Completeness, 10) (Clarity, 10)
%         };
%         \addplot+[fill=color3, draw=black, bar shift=6pt, forget plot] coordinates {
%             (Understandability, 10) (Task Completion, 9) (Efficiency, 8) (Sequential Order, 8) (Satisfaction, 8) (Preference, 5) (Text-Image Mapping, 10) (Correctness, 8) (Cohesiveness, 5) (Completeness, 8) (Clarity, 8)
%         };
%         \addplot+[fill=color4, draw=black, bar shift=6pt, forget plot] coordinates {
%             (Understandability, 10) (Task Completion, 9) (Efficiency, 10) (Sequential Order, 5) (Satisfaction, 9) (Preference, 4) (Text-Image Mapping, 8) (Correctness, 5) (Cohesiveness, 8) (Completeness, 7) (Clarity, 5)
%         };
%         \addplot+[fill=color5, draw=black, bar shift=6pt, forget plot] coordinates {
%             (Understandability, 10) (Task Completion, 10) (Efficiency, 12) (Sequential Order, 5) (Satisfaction, 8) (Preference, 3) (Text-Image Mapping, 10) (Correctness, 5) (Cohesiveness, 7) (Completeness, 10) (Clarity, 5)
%         };
        
%         \legend{1 (bad), 2, 3, 4, 5 (good)}
%         \end{axis}
%     \end{tikzpicture}
%     \caption{Grouped Stacked Bar Chart using Symbolic Coordinates}
% \end{figure}




