\documentclass[10pt,twocolumn,letterpaper]{article}
\usepackage[pagenumbers]{iccv}
\usepackage[dvipsnames]{xcolor}
\definecolor{thedarkblue}{RGB}{0,0,120} %
\definecolor{mydarkblue}{rgb}{0,0.08,0.45} %
\definecolor{darkblue}{rgb}{0,0.08,180}
\colorlet{TufteRed}{red!80!black}
\definecolor{theblue}{RGB}{0,0,180}
\colorlet{thered}{TufteRed}
      
\usepackage{microtype}
\usepackage{balance}


\usepackage{booktabs}
\usepackage{tabularx}



\renewcommand{\qedsymbol}{$\blacksquare$}
\newcommand{\eat}[1]{\ignorespaces}
\usepackage{comment}

\usepackage{tikz}
\usepackage{verbatim}
\usetikzlibrary{arrows}
\usetikzlibrary{shapes,snakes}
\usetikzlibrary{decorations.pathmorphing} %
\usetikzlibrary{fit}					%
\usetikzlibrary{backgrounds}	%

\usepackage{ragged2e}
\usepackage{makecell}
\usepackage{multirow}
\usepackage{microtype}
\usepackage{balance}
\usepackage{setspace}

\graphicspath{{./}{./graphics/}}
\newcolumntype{H}{>{\setbox0=\hbox\bgroup}c<{\egroup}@{}}

\newcolumntype{R}[1]{>{\RaggedLeft\arraybackslash}} %
\newcolumntype{L}[1]{>{\RaggedRight\arraybackslash}} %

\newcommand\TTT{\rule{0pt}{3.2ex}}
\newcommand\BBB{\rule[-1.4ex]{0pt}{0pt}}


\newcommand{\rbr}[1]{\left(#1\right)}
\newcommand{\cbr}[1]{\left\{#1\right\}}
\newcommand{\nbr}[1]{\left\|#1\right\|}
\newcommand{\abr}[1]{\left|#1\right|}
\newcommand{\abs}[1]{\left|#1\right|}
\newcommand{\floor}[1]{\left\lfloor #1 \right\rfloor}
\newcommand{\ceil}[1]{\left\lceil #1 \right\rceil}
\newcommand{\inner}[2]{\left\langle #1,#2 \right\rangle}

\newcommand{\etal}{\emph{et al.}}
\newcommand{\ea}{\emph{et al.}}
\newcommand{\eg}{\emph{e.g.}}
\newcommand{\ie}{\emph{i.e.}}
\newcommand{\iid}{\emph{iid}}
\newcommand{\cf}{\emph{cf.}\ }
\newcommand{\wrt}{\emph{w.r.t.}\ }
\newcommand{\st}{\emph{s.t.}\ }

\newtheorem{corollary}{\bfseries{Corollary}}
\newtheorem{proof2}{PROOF}
\newtheorem*{fact}{Fact}
\newtheorem*{note}{\hspace{-1em}\textsc{Note}}
\newtheorem{corol}{Corollary}%
\newtheorem{axiom}{Axiom}%
\newtheorem{cond}{Condition}%
\newtheorem{property2}{Property}%
\newtheorem{property}{Property}%
\newtheorem{lemma}{\hspace{-1em}\bfseries{Lemma}}
\newtheorem{Definition}{\hspace{-1em}\bfseries{Definition}}
\newtheorem{Claim}{Claim}%

\AtBeginEnvironment{pmatrix}{\setlength{\arraycolsep}{2pt}}



\providecommand{\tensor}[1]{\boldsymbol{\mathcal{#1}}}%
\providecommand{\mat}[1]{\boldsymbol{\mathrm{#1}}}%
\renewcommand{\vec}[1]{\boldsymbol{\mathrm{#1}}}
\providecommand{\sca}[1]{{\mathrm{#1}}}%
\DeclareMathOperator{\rank}{rank}%
\DeclareMathOperator{\diag}{diag}%
\DeclareMathOperator{\Diag}{Diag}%
\providecommand{\itr}[2]{#1^{(#2)}}
\providecommand{\itn}[1]{^{(#1)}}%
\providecommand{\eps}{\varepsilon}%
\providecommand{\kron}{\otimes}
\DeclareMathOperator{\tvec}{vec}
\providecommand{\pmat}[1]{\begin{pmatrix} #1 \end{pmatrix}}
\providecommand{\bmat}[1]{\begin{bmatrix} #1 \end{bmatrix}}
\providecommand{\spmat}[1]{\left(\begin{smallmatrix} #1 \end{smallmatrix}\right)}
\providecommand{\sbmat}[1]{\left[\begin{smallmatrix} #1 \end{smallmatrix}\right]}

\DeclareMathOperator*{\minimize}{minimize}
\DeclareMathOperator*{\maximize}{maximize}
\DeclareMathOperator*{\argmax}{argmax}
\DeclareMathOperator*{\argmin}{argmin}
\DeclareMathOperator*{\argsort}{arg\,sort}
\providecommand{\subjectto}{\ensuremath{\text{subject to}}}
\providecommand{\MINof}[1][]{{\displaystyle \minimize_{#1}}}
\providecommand{\MIN}[2]{\begin{array}{ll} \MINof[#1] & #2 \end{array}}
\providecommand{\MINone}[3]{\begin{array}{ll} \MINof[#1] & #2 \\ \subjectto  & #3 \end{array}}
\providecommand{\MINtwo}[4]{\begin{array}{ll} \MINof[#1] & #2 \\ \subjectto  & #3 \\ & #4 \end{array}}
\providecommand{\MINthree}[5]{\begin{array}{ll} \MINof[#1] & #2 \\ \subjectto  & #3 \\ & #4 \\ & #5 \end{array}}
\providecommand{\MINfour}[6]{\begin{array}{ll} \MINof[#1] & #2 \\ \subjectto  & #3 \\ & #4 \\ & #5 \\ & #6 \end{array}}
\providecommand{\MAXof}[1][]{{\displaystyle \maximize_{#1}}}
\providecommand{\MAX}[2]{\begin{array}{ll} \MAXof[#1] & #2 \end{array}}
\providecommand{\MAXone}[3]{\begin{array}{ll} \MAXof[#1] & #2 \\ \subjectto & #3 \end{array}}
\providecommand{\MAXtwo}[4]{\begin{array}{ll} \MAXof[#1] & #2 \\ \subjectto  & #3 \\ & #4 \end{array}}
\providecommand{\MAXthree}[5]{\begin{array}{ll} \MAXof[#1] & #2 \\ \subjectto  & #3 \\ & #4 \\ & #5 \end{array}}
\providecommand{\MAXfour}[6]{\begin{array}{ll} \MAXof[#1] & #2 \\ \subjectto  & #3 \\ & #4 \\ & #5 \\ & #6 \end{array}}

\DeclareMathOperator{\E}{E}
\DeclareMathOperator{\hugeE}{\mbox{\huge\raise-0.3ex\hbox{E}}}
\DeclareMathOperator{\p}{\mathbb{P}}
\DeclareMathOperator{\hugep}{\mbox{\huge\raise-0.3ex\hbox{$\p$}}}
\DeclareMathOperator{\Var}{Var}
\DeclareMathOperator{\Cov}{Cov}
\DeclareMathOperator{\Bias}{Bias}
\DeclareMathOperator{\sign}{sign}
\DeclareMathOperator{\Std}{Std}
\providecommand{\Eof}{\E\BracketOf}
\providecommand{\hugeEof}{\hugeE\BracketOf}
\providecommand{\Stdof}{\Std\BracketOf}
\providecommand{\varof}{\Std\BracketOf}
\providecommand{\Covof}[2]{\Cov\BrackeOf#1,#2\right]}
\providecommand{\prob}[1][]{\p_{#1}\BraceOf}
\providecommand{\hugeprob}[1][]{\hugep_{#1}\BraceOf}

\DeclareMathOperator{\degree}{degree}
\DeclareMathOperator{\trace}{trace}
\providecommand{\set}{\mathcal}
\providecommand{\graph}{\mathcal}
\providecommand{\mathdef}{\equiv}
\providecommand{\card}{\absof}
\providecommand{\cardof}{\absof}
\providecommand{\eqdef}{\equiv}
\providecommand{\eps}{\varepsilon}
\DeclareMathOperator{\bigO}{O}
\providecommand{\bigOof}{\bigO\ParensOf}
\providecommand{\absof}[1]{$\left| #1 \right|$}

\newcommand{\RR}{\mathbb{R}}
\newcommand{\CC}{\mathbb{C}}



\providecommand{\eye}{\mat{I}}
\providecommand{\mA}{\ensuremath{\mat{A}}}
\providecommand{\mB}{\ensuremath{\mat{B}}}
\providecommand{\mC}{\ensuremath{\mat{C}}}
\providecommand{\mD}{\ensuremath{\mat{D}}}
\providecommand{\mE}{\ensuremath{\mat{E}}}
\providecommand{\mF}{\ensuremath{\mat{F}}}
\providecommand{\mG}{\ensuremath{\mat{G}}}
\providecommand{\mH}{\ensuremath{\mat{H}}}
\providecommand{\mI}{\ensuremath{\mat{I}}}
\providecommand{\mJ}{\ensuremath{\mat{J}}}
\providecommand{\mK}{\ensuremath{\mat{K}}}
\providecommand{\mL}{\ensuremath{\mat{L}}}
\providecommand{\mM}{\ensuremath{\mat{M}}}
\providecommand{\mN}{\ensuremath{\mat{N}}}
\providecommand{\mO}{\ensuremath{\mat{O}}}
\providecommand{\mP}{\ensuremath{\mat{P}}}
\providecommand{\mQ}{\ensuremath{\mat{Q}}}
\providecommand{\mR}{\ensuremath{\mat{R}}}
\providecommand{\mS}{\ensuremath{\mat{S}}}
\providecommand{\mT}{\ensuremath{\mat{T}}}
\providecommand{\mU}{\ensuremath{\mat{U}}}
\providecommand{\mV}{\ensuremath{\mat{V}}}
\providecommand{\mW}{\ensuremath{\mat{W}}}
\providecommand{\mX}{\ensuremath{\mat{X}}}
\providecommand{\mY}{\ensuremath{\mat{Y}}}
\providecommand{\mZ}{\ensuremath{\mat{Z}}}
\providecommand{\mLambda}{\ensuremath{\mat{\Lambda}}}
\providecommand{\mzero}{\ensuremath{\mat{0}}}

\providecommand{\tA}{\ensuremath{\tensor{A}}}
\providecommand{\tB}{\ensuremath{\tensor{B}}}
\providecommand{\tC}{\ensuremath{\tensor{C}}}
\providecommand{\tD}{\ensuremath{\tensor{D}}}
\providecommand{\tE}{\ensuremath{\tensor{E}}}
\providecommand{\tF}{\ensuremath{\tensor{F}}}
\providecommand{\tG}{\ensuremath{\tensor{G}}}
\providecommand{\tH}{\ensuremath{\tensor{H}}}
\providecommand{\tI}{\ensuremath{\tensor{I}}}
\providecommand{\tJ}{\ensuremath{\tensor{J}}}
\providecommand{\tK}{\ensuremath{\tensor{K}}}
\providecommand{\tL}{\ensuremath{\tensor{L}}}
\providecommand{\tM}{\ensuremath{\tensor{M}}}
\providecommand{\tN}{\ensuremath{\tensor{N}}}
\providecommand{\tO}{\ensuremath{\tensor{O}}}
\providecommand{\tP}{\ensuremath{\tensor{P}}}
\providecommand{\tQ}{\ensuremath{\tensor{Q}}}
\providecommand{\tR}{\ensuremath{\tensor{R}}}
\providecommand{\tS}{\ensuremath{\tensor{S}}}
\providecommand{\tT}{\ensuremath{\tensor{T}}}
\providecommand{\tU}{\ensuremath{\tensor{U}}}
\providecommand{\tV}{\ensuremath{\tensor{V}}}
\providecommand{\tW}{\ensuremath{\tensor{W}}}
\providecommand{\tX}{\ensuremath{\tensor{X}}}
\providecommand{\tY}{\ensuremath{\tensor{Y}}}
\providecommand{\tZ}{\ensuremath{\tensor{Z}}}

\providecommand{\ones}{\vec{e}}
\providecommand{\va}{\ensuremath{\vec{a}}}
\providecommand{\vb}{\ensuremath{\vec{b}}}
\providecommand{\vc}{\ensuremath{\vec{c}}}
\providecommand{\vd}{\ensuremath{\vec{d}}}
\providecommand{\ve}{\ensuremath{\vec{e}}}
\providecommand{\vf}{\ensuremath{\vec{f}}}
\providecommand{\vg}{\ensuremath{\vec{g}}}
\providecommand{\vh}{\ensuremath{\vec{h}}}
\providecommand{\vi}{\ensuremath{\vec{i}}}
\providecommand{\vj}{\ensuremath{\vec{j}}}
\providecommand{\vk}{\ensuremath{\vec{k}}}
\providecommand{\vl}{\ensuremath{\vec{l}}}
\providecommand{\vm}{\ensuremath{\vec{l}}}
\providecommand{\vn}{\ensuremath{\vec{n}}}
\providecommand{\vo}{\ensuremath{\vec{o}}}
\providecommand{\vp}{\ensuremath{\vec{p}}}
\providecommand{\vq}{\ensuremath{\vec{q}}}
\providecommand{\vr}{\ensuremath{\vec{r}}}
\providecommand{\vs}{\ensuremath{\vec{s}}}
\providecommand{\vt}{\ensuremath{\vec{t}}}
\providecommand{\vu}{\ensuremath{\vec{u}}}
\providecommand{\vv}{\ensuremath{\vec{v}}}
\renewcommand{\vv}{\ensuremath{\vec{v}}}
\providecommand{\vw}{\ensuremath{\vec{w}}}
\providecommand{\vx}{\ensuremath{\vec{x}}}
\providecommand{\vy}{\ensuremath{\vec{y}}}
\providecommand{\vz}{\ensuremath{\vec{z}}}
\providecommand{\vpi}{\ensuremath{\vecalt{\pi}}} 

\providecommand{\ssa}{\ensuremath{\sca{a}}}
\providecommand{\ssb}{\ensuremath{\sca{b}}}
\providecommand{\ssc}{\ensuremath{\sca{c}}}
\providecommand{\ssd}{\ensuremath{\sca{d}}}
\providecommand{\sse}{\ensuremath{\sca{e}}}
\providecommand{\ssf}{\ensuremath{\sca{f}}}
\providecommand{\ssg}{\ensuremath{\sca{g}}}
\providecommand{\ssh}{\ensuremath{\sca{h}}}
\providecommand{\ssi}{\ensuremath{\sca{i}}}
\providecommand{\ssj}{\ensuremath{\sca{j}}}
\providecommand{\ssk}{\ensuremath{\sca{k}}}
\providecommand{\ssl}{\ensuremath{\sca{l}}}
\providecommand{\ssm}{\ensuremath{\sca{l}}}
\providecommand{\ssn}{\ensuremath{\sca{n}}}
\providecommand{\sso}{\ensuremath{\sca{o}}}
\providecommand{\ssp}{\ensuremath{\sca{p}}}
\providecommand{\ssq}{\ensuremath{\sca{q}}}
\providecommand{\ssr}{\ensuremath{\sca{r}}}
\providecommand{\sss}{\ensuremath{\sca{s}}}
\providecommand{\sst}{\ensuremath{\sca{t}}}
\providecommand{\ssu}{\ensuremath{\sca{u}}}
\providecommand{\ssv}{\ensuremath{\sca{v}}}
\providecommand{\ssw}{\ensuremath{\sca{w}}}
\providecommand{\ssx}{\ensuremath{\sca{x}}}
\providecommand{\ssy}{\ensuremath{\sca{y}}}
\providecommand{\ssz}{\ensuremath{\sca{z}}}
\providecommand{\sspi}{\ensuremath{\scaalt{\pi}}} 


\DeclareMathOperator{\cut}{cut}
\DeclareMathOperator{\vol}{vol}



\definecolor{googleblue}{HTML}{4285F4}
\definecolor{googlered}{HTML}{DB4437}
\definecolor{googlepurple}{HTML}{A142F4} %
\definecolor{googlegreen}{HTML}{0F9D58}


\definecolor{iccvblue}{rgb}{0.21,0.49,0.74}
\usepackage[pagebackref,breaklinks,colorlinks,allcolors=iccvblue]{hyperref}

\captionsetup{belowskip=2pt,aboveskip=2pt}

\newcommand{\customfootnotetext}[2]{% 
  \begingroup
  \setlength{\skip\footins}{0pt}
  \renewcommand{\thefootnote}{#1}%
  \footnotetext{#2}%
  \endgroup
}

\title{C\textcolor{blue}{$^{3}$}Editor: A\textcolor{blue}{c}hieving \textcolor{blue}{C}ontrollable \textcolor{blue}{C}onsistency in 2D Model for \textcolor{blue}{3}D Editing}

\author{%
  Zeng Tao\textsuperscript{1}\textsuperscript{*} \quad
  Zheng Ding\textsuperscript{2} \quad
  Zeyuan Chen\textsuperscript{2} \quad
  Xiang Zhang\textsuperscript{2} \quad
  Leizhi Li\textsuperscript{2}\quad
  Zhuowen Tu\textsuperscript{2} \\
  \textsuperscript{1}Fudan University \quad 
  \textsuperscript{2}UC San Diego
}


\let\oldtwocolumn\twocolumn
\renewcommand\twocolumn[1][]{%
    \oldtwocolumn[{#1}{
    \begin{center}
           \includegraphics[width=\linewidth,trim=0 2em 0 2em,clip]{figure/intro.pdf}
           \captionof{figure}{\textbf{C$^{3}$Editor: Controllable Consistent 2D Model for 3D Editing}. \textbf{Top:} Our C$^{3}$Editor method generates consistent 2D editing results across different views by following the original 3D scene, editing text, and user guidance, thereby supporting improved 3D editing performance. \textbf{Bottom:} Comparison of 2D and 3D editing results between baseline and C$^{3}$Editor.}
           \label{Fig1}
           \vspace{1em}
        \end{center}
    }]
}

\begin{document}

\maketitle

\customfootnotetext{*}{Work done during internship at UC San Diego.}
\begin{abstract}

Existing 2D-lifting-based 3D editing methods often encounter challenges related to inconsistency, stemming from the lack of view-consistent 2D editing models and the difficulty of ensuring consistent editing across multiple views. To address these issues, we propose C³Editor, a controllable and consistent 2D-lifting-based 3D editing framework. Given an original 3D representation and a text-based editing prompt, our method selectively establishes a view-consistent 2D editing model to achieve superior 3D editing results. The process begins with the controlled selection of a ground truth (GT) view and its corresponding edited image as the optimization target, allowing for user-defined manual edits. Next, we fine-tune the 2D editing model within the GT view and across multiple views to align with the GT-edited image while ensuring multi-view consistency. To meet the distinct requirements of GT view fitting and multi-view consistency, we introduce separate LoRA modules for targeted fine-tuning. Our approach delivers more consistent and controllable 2D and 3D editing results than existing 2D-lifting-based methods, outperforming them in both qualitative and quantitative evaluations.

\end{abstract}

\section{Introduction}

The remarkable success of 2D generative models~\cite{ho2020denoising, kumari2023multi,chefer2023attend,bansal2023universal, avrahami2023chosen,avrahami2023blended,xu2023prompt,zhang2023adding,zhang2024realcompo} has spurred rapid advancements in the field of generation, leading to successful applications in related areas such as editing tasks~\cite{avrahami2022blended,hertz2022prompt,kawar2023imagic,meng2021sdedit,ramesh2022hierarchical,brooks2023instructpix2pix}. Leveraging the superior performance of 2D models as priors also has become a popular approach in 3D tasks~\cite{zhang2023uni,chen2023fantasia3d,lin2023magic3d,metzer2023latent,wang2024prolificdreamer,zhao2025deprdepthguidedsingleview}. Given the scarcity of real-world 3D data and the high cost of training, utilizing pretrained 2D models as guidance offers a promising solution. For example, 2D-lifting-based 3D editing methods~\cite{haque2023instruct} use a 2D editing model~\cite{brooks2023instructpix2pix} to obtain edited images from different viewpoints, which are then used to update the original 3D representation. 

However, directly transferring 2D priors to the 3D domain presents certain challenges, such as the issue of viewpoint consistency~\cite{poole2022dreamfusion}. Since 2D models lack view information and 3D awareness, conflicts between views may arise when applied to 3D tasks. In 3D editing tasks, directly using edited 2D images that lack consistency across views can lead to errors in 3D editing. Some approaches attempt to address this by constructing external datasets~\cite{liu2023zero, shi2023mvdream, liu2024one}. However, addressing the view inconsistency problem in 3D editing remains challenging. The training process requires datasets containing consistent editing text, original 2D images, and edited 2D images across multiple views, which are difficult to obtain. One viable approach is to designate the edited result of one specific view as the ground truth (GT) and, leveraging the generalization ability of the 2D model, gradually adapt other viewpoints to match this viewpoint, achieving internal consistency across views.

Additionally, text-based editing inherently supports diversity, but current 2D-lifting-based 3D editing methods suppress this diversity by uniformly processing different 2D editing results~\cite{haque2023instruct, dong2024vica, chen2024gaussianeditor, chen2024consistdreamer}. Our goal is to allow for the controllability of optimization directions, enabling the 3D editing results to express more possibilities and better align with human intent.

In response, we propose C$^{3}$Editor, a controllable and consistent 2D-lifting-based 3D editing method. Given an original 3D scene and an editing text prompt, we aim to obtain a view-consistent 2D editing model selectively, thereby achieving improved 3D editing results. By selecting a GT view and its corresponding edited image as the optimization target, our approach stabilizes the GT view's editing results and then progressively enforces consistency across different views through view propagation. Furthermore, we introduce separate LoRA modules to fine-tune the model, addressing the unique requirements of GT view fitting and multi-view consistency separately. This structured approach ensures that the 2D editing model achieves cohesive 3D editing results across all views, enhancing both visual consistency and user controllability.

In summary, our contributions are as follows:

\begin{itemize}
    \item We develop a view-consistent 2D editing model based on the original 3D representation and an editing text prompt, facilitating enhanced 3D editing outcomes. This approach effectively bridges the gap between 2D and 3D, as well as between original and edited representations.
    \item Our controllable 3D editing method allows users to select a ground truth (GT) edited image and manually adjust it to produce consistent 3D editing results across views.
    \item Our C$^{3}$Editor method mainly focuses on two aspects: Intra-GT and Inter-view. We specifically design GT selection and intra-GT Loss methods to ensure stable GT fitting, followed by view propagation and inter-view loss for view consistency. Different LoRAs serve separate consistent purposes. Qualitative and quantitative experiments demonstrate the effectiveness of C$^{3}$Editor.
\end{itemize}


\section{Related Work}

\subsection{Diffusion Model and Fine-tuning}

Diffusion models~\cite{ho2020denoising,rombach2022high} have become powerful tools in generative tasks due to their unique approach of iteratively refining data from noise, allowing for precise control over the generation process. These models learn data distributions through a diffusion process that gradually adds and then reverses noise, effectively modeling complex data patterns in images, audio, and even text. Because of their robust performance, diffusion models are widely applied in tasks~\cite{chen2023fantasia3d,lin2023magic3d, metzer2023latent,wang2024prolificdreamer,xu2024bayesian,srivastava2025lay,zeng2025yolo} such as image synthesis, inpainting, super-resolution, and conditional generation, where they can generate or manipulate visual content based on additional inputs, such as text prompts, segmentation maps, or depth maps. This versatility makes them particularly valuable for tasks requiring high-quality, detailed outputs and subtle adjustments.

Fine-tuning diffusion models is essential for adapting them to specific tasks or datasets. Through targeted fine-tuning, diffusion models can be optimized to perform controlled edits, match stylistic demands, or generalize to new domains beyond their original training data. Techniques such as low-rank adaptation (LoRA)~\cite{hu2021lora} and other parameter-efficient tuning methods~\cite{houlsby2019parameter,li2021prefix,liu2021p} allow for effective customization by focusing on updating key parts of the model while keeping the core structure intact. This approach is especially useful when integrating diffusion models as priors in cross-domain applications, where maintaining high fidelity across varying views is critical. Fine-tuning thus enables diffusion models to meet specialized generative requirements, ensuring they maintain both visual quality and flexibility across diverse tasks.


\subsection{Diffusion-based 2D Editing}

Diffusion-based 2D editing techniques~\cite{avrahami2022blended,hertz2022prompt,kawar2023imagic, meng2021sdedit,ramesh2022hierarchical,brooks2023instructpix2pix} have revolutionized the field of image manipulation by leveraging the denoising diffusion process to transform noise into structured visual representations. In these models, editing is performed iteratively, where each step refines the image by reversing the noise and generating realistic features, allowing for adequate control over the level and type of modifications applied.

The key advantage of diffusion-based 2D editing lies in its ability to use conditional inputs, like text prompts or segmentation maps, to guide the editing process. For example, Instruct-Pix2Pix~\cite{brooks2023instructpix2pix} can interpret prompts to modify colors, add textures, or alter structures while maintaining the coherence of the image. These models can learn data distributions that align with specific editing goals, making them versatile across diverse applications. By fine-tuning or adjusting model parameters, diffusion models can also be specialized for specific editing tasks, allowing them to adapt to particular styles or constraints required by the user. This combination of iterative refinement, conditional control, and adaptability has made diffusion-based 2D editing a powerful tool in modern image generation and editing tasks.

\subsection{2D-lifting-based 3D Editing}

Recent advancements in 3D editing have increasingly integrated diffusion-based 2D editing models, leveraging their established capabilities to enhance 3D workflows~\cite{cao2024mvinpainter,chen2024proedit, chen2024gaussianeditor,haque2023instruct,dong2024vica,chen2024consistdreamer}. These models, originally designed for detailed image modifications, contribute to 3D editing by transferring their proficiency in nuanced, high-quality adjustments to three-dimensional representations. Methods like Neural Radiance Fields (NeRF)~\cite{mildenhall2021nerf} and 3D Gaussian Splatting (3DGS)~\cite{kerbl20233d} incorporate 2D editing models to improve the consistency and detail of 3D content.

By using 2D diffusion models as priors, recent approaches enhance the fidelity and stylistic consistency of 3D edits, especially in maintaining coherence across multiple views. Some works, such as DGE~\cite{chen2024dge}, combine images from different viewpoints into videos for processing. A primary challenge in this domain remains ensuring multi-view consistency, as traditional 2D-based edits applied to 3D models often lead to discrepancies between perspectives. Some methods, such as ConsistDreamer~\cite{chen2024consistdreamer}, model 3D-aware consistency by means of constraints like neural feature alignment or volume-based feature consistency. This has provided inspiration for our work. However, since it is not open-sourced, it is impossible to make a comparison for now. We compare our method with NeRF-based Instruct-NeRF-to-NeRF~\cite{haque2023instruct}, ViCA-NeRF~\cite{dong2024vica}, and GS-based GaussianEditor~\cite{chen2024gaussianeditor}.

\begin{figure*}[t]
    \centering
    \includegraphics[width=\linewidth,trim=1em 1em 1em 1em,clip]{figure/method.pdf}
    \caption{\textbf{C$^{3}$Editor Method Pipeline}. Given a 3D representation $\Phi$, a text prompt for editing $y$, and the original 2D editing model $\Theta_{O}$, our method aims to process $\Theta_{O}$ to obtain $\Theta_{C}$ that is related to $y$ and ensures multi-view consistency, thereby achieving improved 3D editing results. \textbf{Phase 1}: Controllable optimization direction selecting and manual editing in \cref{sec:optimizationdirection}. \textbf{Phase 2}: Intra-GT prior fitting in \cref{sec:priorfitting} to fit the GT information. \textbf{Phase 3}: View propagation and inter-view consistent construcing in \cref{sec:viewpropagation}. Details of LoRA modules for separate fine-tuning are in \cref{sec:paralleltuning}.}
    \label{fig:method}
\end{figure*}


\section{Method}
\label{sec:method}

\subsection{Overview}

Given a 3D representation $\Phi$ (\eg, 3D GS), a text prompt for editing $y$, and the original 2D editing model $\Theta_{O}$ (like Instruct-Pix2Pix~\cite{brooks2023instructpix2pix}), the goal of our method is to process $\Theta_{O}$ to obtain $\Theta_{C}$ that is related to $y$ and ensures multi-view consistency, thereby achieving improved 3D editing results. In \cref{sec:optimizationdirection}, the ground truth (GT) view $v_\mathrm{gt}$ and GT edited image $I^{e}_{v_\mathrm{gt}}$ are manually selected from the 2D editing results $I^{e}_{v}$ of different views $v$, rendered by $\Theta_{O}$, which serve as the optimization target. In \cref{sec:priorfitting}, we optimize $\Theta_{O}$ with a specifically designed intra-GT loss to fit $I^{e}_{v_\mathrm{gt}}$. We maintain global consistency through the view propagation method and inter-view loss described in \cref{sec:viewpropagation}. In \cref{sec:paralleltuning}, we introduce different LoRAs for different fine-tuning objectives to separately fine-tune the diffusion model.

Using $\Theta_{C}$ obtained, each view $v$ in $V$ undergoes a complete editing process, producing consistent view-editing results $I^{e}_{v}$. Considering the gradient storage issues in fine-tuning the diffusion model, each complete editing process includes 5 diffusion denoising steps, which achieves a good trade-off between GPU memory limits and editing quality. The final 3D editing result is obtained by updating the original 3D representation $\Phi$ with the edited results of all views $I^{e}_{v}$. We adopt 3D Gaussian Splatting (3D GS) as our 3D representation due to its efficient training speed and excellent rendering quality. We adopt widely used Instruct-Pix2Pix~\cite{brooks2023instructpix2pix} as our diffusion-based pre-trained 2D editing model, for its outstanding performance in 2D editing tasks. The method for updating 3D GS is consistent with that in GaussianEditor. The detailed process is illustrated in \cref{fig:method}.

\subsection{Controllable Optimization Direction}
\label{sec:optimizationdirection}

The independent 2D editing processes of different views $v$ lead to different editing results. To avoid view conflicts in 3D editing, we select the editing result $I^{e}_{v_\mathrm{gt}}$ from a specific view $v_\mathrm{gt}$ as the optimization direction. In subsequent operations, the 2D editing model will use this GT as a reference to edit images from other views, thereby preventing conflicts in the 3D editing process.

As shown in \cref{fig:method} Phase 1, for each view $v$, an independent editing process is performed, resulting in different editing outcomes $I^{e}_{v}$. User then selects a specific view and its corresponding edited result as the GT view $v_\mathrm{gt}$ and GT edited image $I^{e}_{v_\mathrm{gt}}$, setting the target optimization direction. Different choices of view and edited results lead to different optimization directions, and consequently, varying final 3D editing outcomes, which are shown in \cref{sec:controllableediting}. Therefore, this selection should follow certain guidelines, such as choosing results of higher editing quality and selecting a more central view. Based on these guidelines, the user can choose their desired optimization direction.

For the obtained GT image $I^{e}_{v_\mathrm{gt}}$, users can directly use it as is. However, if there are any unsatisfactory elements, users can make manual edits according to their preferences. They can utilize image editing tools such as Photoshop to modify the image content, then set the edited image as the GT image $I^{e}_{v_\mathrm{gt}}$ for the model. The GT view and edited result are then used to guide the subsequent 2D editing model fine-tuning process, ensuring that the 2D model can achieve controllable editing results across all views.

\subsection{Intra-GT Prior Fitting}
\label{sec:priorfitting}

With the optimization direction established, the next step is to train the 2D diffusion model to fit the GT image. Adjustments to the 2D diffusion model are divided into two parts: intra-GT and inter-view adaptation. In this section, we need to make the 2D model fit the chosen optimization direction $I^{e}_{v_\mathrm{gt}}$ on the GT view $v_\mathrm{gt}$, aiming to establish a foundational intra-view editing stability on the GT view $v_\mathrm{gt}$.

As shown in \cref{fig:method} Phase 2, we freeze the 3D representation and add LoRA modules to the diffusion model for fine-tuning. The edited image $I^{e}_{v_\mathrm{gt}}$ is used as the GT. An independent, complete editing process is performed on the rendered image $I^{r}_{v_\mathrm{gt}}$ of $v_\mathrm{gt}$ to obtain an edited image $I^{e'}_{v_\mathrm{gt}}$ that differs from the GT image $I^{e}_{v_\mathrm{gt}}$. Compute the loss between $I^{e'}_{v_\mathrm{gt}}$ and $I^{e}_{v_\mathrm{gt}}$, back-propagate, and update the LoRA. The loss $\mathcal{L}_\mathrm{intra}$ consists of two parts: the $L_1$ loss and perceptual loss between $I^{e'}_{v_\mathrm{gt}}$ and $I^{e}_{v_\mathrm{gt}}$.

\begin{equation}
    \mathcal{L}_\mathrm{intra} = \lambda_{1} L_{1}(I^{e'}_{v_\mathrm{gt}}, I^{e}_{v_\mathrm{gt}}) + \lambda_{2}  L_\mathrm{Perceptual}(I^{e'}_{v_\mathrm{gt}}, I^{e}_{v_\mathrm{gt}})
\end{equation}

Through multiple iterations of this process, the fine-tuned 2D diffusion model acquires a certain fitting capability for the GT image, while also improving editing stability for the same view, achieving similar results across different editing processes.


\subsection{View Propagation and Inter-view Consistency}
\label{sec:viewpropagation}

After \cref{sec:priorfitting}, the 2D editing model can only fit $I^{e}_{v_\mathrm{gt}}$ on $v_\mathrm{gt}$, with limited generalization ability, and has limited consistency in editing effects for views that differ significantly. If the current 2D model is used directly as the prior, it can only maintain consistent editing for $v_\mathrm{gt}$ and views nearby, while its performance on more distant views remains uncertain. Therefore, in this section, we introduce additional methods to ensure consistent editing across all views. We leverage the interrelations between viewpoints to enable the 2D diffusion model to achieve consistent editing across all views gradually.

\begin{figure*}[t]
    \centering
    \includegraphics[width=\linewidth]{figure/results.pdf}
    \caption{\textbf{Comparison of Qualitative Results}. Compared to baseline methods, C$^{3}$Editor can generate view-consistent 2D images, avoiding inter-view conflicts (highlighted in \textcolor{blue}{blue}) and erroneous 2D edits (highlighted in \textcolor{red}{red}), thereby achieving better 3D editing results.}
    \label{fig:qualitative}
\end{figure*}

As shown in \cref{fig:method} Phase 3, specifically, we sort the views as a sequence $S$ by their distance of camera center points from $v_\mathrm{gt}$, from closest to farthest, and then perform fine-tuning of the 2D diffusion model on each view in $S$ other than $v_\mathrm{gt}$. $v_{0}$ in the sequence represents $v_\mathrm{gt}$. We perform a 2D editing process on each view $v_{i}$ with the index $i\in \{1,2,\dots,j,i,\dots,n-1\}$ in the sequence separately. The resulting image $I^{e}_{v_{i}}$ serves as the GT image for $v_{i}$. Next, an independent 2D editing process is applied to this view, and another edited image $I^{e'}_{v_{i}}$ is obtained. The loss $\mathcal{L}_\mathrm{inter}$ comprises three parts: $\mathrm{loss}\,1$ between the edited image $I^{e'}_{v_{i}}$ and the GT image $I^{e}_{v_{i}}$, $\mathrm{loss}\,2$ between $I^{e'}_{v_{i}}$ and $I^{e}_{v_{j}}$ of the closest processed view $v_{j}$, $\mathrm{loss}\,3$ between $I^{e'}_{v_{i}}$ and $I^{e}_{v_\mathrm{gt}}$. $\mathcal{L}_\mathrm{inter}$ is as follows:
\begin{align}
    \mathcal{L}_\mathrm{inter} = & \underbrace{\lambda_{3}  L_1(I^{e'}_{v_{i}}, I^{e}_{v_{i}}) + \lambda_{4} L_\mathrm{Perceptual}(I^{e'}_{v_{i}}, I^{e}_{v_{i}})}_{\mathrm{loss}\,1} \nonumber \\
    & + \underbrace{\lambda_{5} L_\mathrm{Perceptual}(I^{e'}_{v_{i}}, I^{e}_{v_{j}})}_{\mathrm{loss}\,2} \nonumber \\
    & + \underbrace{\lambda_{6} L_\mathrm{Perceptual}(I^{e'}_{v_{i}}, I^{e}_{v_\mathrm{gt}})}_{\mathrm{loss}\,3}
\end{align}
$\mathcal{L}_\mathrm{inter}$ is back-propagated, and LoRA is used to fine-tune the diffusion model. After this process, we reverse the sequence $S$ and repeat the above steps until reaching $v_\mathrm{gt}$. The generalization capability gradually expands from the $v_\mathrm{gt}$ to encompass all views.

\subsection{Separate Fine-tuning}
\label{sec:paralleltuning}

To prevent the loss of GT information during the inter-view fine-tuning process, we design two LoRAs, each serving different fine-tuning goals. The fine-tuning of the 2D Editing model $\Theta_{O}$ is divided into two main aspects: LoRA$_\mathrm{gt}$ for fitting the GT view image $I^{e}_{v_\mathrm{gt}}$, and LoRA$_\mathrm{mv}$ for ensuring consistency across different views. The inference process takes place in \cref{sec:optimizationdirection} and \cref{fig:method} Phase 1, with no trainable model parameters. In \cref{sec:priorfitting} and \cref{fig:method} Phase 2, we use LoRA$_\mathrm{gt}$ to fine-tune $\Theta_{O}$ while keeping LoRA$_\mathrm{mv}$ frozen. After this step, LoRA$_\mathrm{gt}$ helps the $\Theta_{O}$ fit $I^{e}_{v_\mathrm{gt}}$. In \cref{sec:viewpropagation} and \cref{fig:method} Phase 3, we freeze LoRA$_\mathrm{gt}$ and use LoRA$_\mathrm{mv}$ to fine-tune $\Theta_{O}$. During the separate fine-tuning process, the model uses the GT information obtained by LoRA$_\mathrm{gt}$ and leverages LoRA$_\mathrm{mv}$ to achieve global consistency.


\section{Experiments}
\subsection{Implementation Details}

Our method builds on the advanced 2D-lifting-based 3D GS Editing Method, GaussianEditor~\cite{chen2024gaussianeditor}. Specifically, we use 3D GS~\cite{kerbl20233d} as the 3D representation and the widely-used Instruct-Pix2Pix~\cite{brooks2023instructpix2pix} as the diffusion-based 2D editing model. All experiments were conducted on a single NVIDIA RTX A6000, with the fine-tuning process taking 1 minute in total. We use MipNeRF-360~\cite{barron2022mip} and Instruct-NeRF-to-NeRF dataset~\cite{haque2023instruct} to measure the performance of our method. The MipNeRF-360 dataset contains 360-degree views of 3D scenes, while the Instruct-NeRF-to-NeRF dataset contains 3D scenes. We use the CLIP-Score~\cite{taited2023CLIPScore} (image-text and image-image) as the evaluation metrics. The former measures the similarity between 3D edited results and editing text, while the latter measures the similarity between 2D images produced by the 2D editing process. A higher score indicates greater editing quality and view consistency. We also use the Fréchet Inception Distance (FID)~\cite{heusel2017gans, Seitzer2020FID} between original rendered images and edited results to evaluate the quality of 3D editing. A lower FID score indicates higher image quality.

For each scene and editing text, we perform unique training to obtain the corresponding 2D editing model. Following the same approach as GaussianEditor, we first use Gaussian Semantic Tracing~\cite{chen2024gaussianeditor} to generate a mask of the editing target within the 3D GS. We then follow the processing steps outlined in \cref{sec:method} to obtain a 2D editing model with view consistency. This model serves as the 2D prior, achieving the final 3D editing result. For further details on the 3D editing process, please refer to GaussianEditor~\cite{chen2024gaussianeditor}.

The hyperparameters are set as follows: 30 iterations for $\mathcal{L}_\mathrm{intra}$ updates, and 3 iterations for $\mathcal{L}_\mathrm{inter}$ updates. We set $\lambda_{n}$=1. For the two LoRA modules, we use the AdamW optimizer with the following settings: $r=4$, $\text{LoRA alpha}=4$, init LoRA weights="gaussian", $\text{lr}=10^{-4}$, $\text{betas}=(0.9, 0.999)$, $\text{weight decay}=10^{-2}$, $\text{eps}=10^{-8}$.


\subsection{Comparison with Baselines}

\noindent \textbf{Qualitative Comparison}. \cref{fig:qualitative} illustrates the qualitative results of our method. Compared to NeRF-based Instruct-NeRF-to-NeRF~\cite{haque2023instruct}, ViCA-NeRF~\cite{dong2024vica}, and GS-based GaussianEditor~\cite{chen2024gaussianeditor} methods, our approach demonstrates superior consistency in both 2D and 3D editing. With the editing texts of the face scene, such as ``Turn him into Batman" and ``Turn his face into a skull", our method addresses inconsistency issues encountered by baseline methods, achieving accurate edits on facial features. In natural scenes, like the bonsai scene, our approach produces improved editing results compared to baseline methods, excelling in color and other details. This improvement is due to the fact that current 3D editing techniques using 2D editing models as priors often encounter inconsistencies during 2D editing, leading to issues such as inter-view inconsistency (highlighted in the \textcolor{blue}{blue} boxes) and editing errors (highlighted in the \textcolor{red}{red} boxes), which result in inaccurate or incomplete 3D edits. Our method, however, achieves consistent 2D editing results, leading to accurate 3D edits. For more qualitative results, please refer to the following sections and supplementary materials.


\begin{table}[h]
    \centering
    \resizebox{\linewidth}{!}{
    \begin{tabular}{c|c|c}
        \toprule
        Method & GaussianEditor & \textbf{C$^{3}$Editor (Ours)} \\
        \midrule
        Image-Text CLIP-Score ($\uparrow$) & 24.18 & \textbf{25.21} \\
        \midrule
        Image-Image CLIP-Score ($\uparrow$) & 84.20 &  \textbf{87.46} \\
        \midrule
        FID ($\downarrow$) & 112.21 & \textbf{89.95} \\
        \midrule \midrule
        Time Difference & \multicolumn{2}{|c}{Avg 56s more than GaussianEditor}  \\
        \bottomrule
    \end{tabular}}
    \caption{\textbf{Comparison of Quantitative Results}. Our method surpasses the baseline method on all of the three metrics.}
    \label{tab:quantitative}
\end{table}

\noindent \textbf{Quantitative Comparison}. As shown in \cref{tab:quantitative}, we use CLIP-Score~\cite{taited2023CLIPScore} (image-image and image-text) and FID~\cite{heusel2017gans, Seitzer2020FID} as the metrics for quantitative evaluation. Specifically, we calculate the CLIP-Score between images from edited 3D results and editing text. A higher score indicates better editing loyalty. Our method achieves a higher CLIP-Score, demonstrating the improved quality of our 3D editing results. We also calculate the CLIP-Score within the 2D images produced by the editing process. A higher score indicates greater similarity between edited 2D images, thus representing stronger view consistency. Our method achieves a higher CLIP-Score, demonstrating the view consistency of our editing approach. The lower FID score of our method indicates better image quality in the 3D editing results. Our method outperforms GaussianEditor in both qualitative and quantitative evaluations, showcasing the effectiveness of our approach in achieving controllable and consistent 3D editing results.

\subsection{Controllable Editing}
\label{sec:controllableediting}

\begin{figure}[h]
    \centering
    \includegraphics[width=\linewidth]{figure/ablation_gt.pdf}
    \caption{\textbf{Controllable Editing Results with Different GT Selections}. In C$^{3}$Editor, users can decide the optimization direction by selecting the GT edited image they prefer.}
    \label{fig:gt}
\end{figure}

\noindent \textbf{Controllable GT Selection}. With different selections of the GT edited image, our method can achieve the corresponding editing results. As shown in \cref{fig:gt}, given the editing prompt ``Make the bike red and blue", the pre-trained 2D editing model can produce different outcomes. In \cref{fig:gt} (1), the wheels are edited to blue while the entire bike frame is edited to red. In \cref{fig:gt} (2), the wheels are also blue, but only part of the frame is edited to red. Using our method, the obtained 2D editing model can edit images from other views to produce the corresponding 3D editing result. 

The 2D editing process inherently allows for diverse outcomes, and our controllable GT selection effectively supports this diversity, enabling results that better align with user intentions. By allowing targeted selection of the GT image, our approach minimizes 3D editing errors that may arise from inaccuracies in 2D editing, thus enhancing the precision and stability of the final editing outcome.


\begin{figure}[h]
    \centering
    \includegraphics[width=\linewidth]{figure/humanedit.pdf}
    \caption{\textbf{Controllable Editing Results with Manual Editing}. In C$^{3}$Editor, users can edit the GT manually and obtain the corresponding 2D and 3D editing results.}
    \label{fig:manual}
\end{figure}


\noindent \textbf{Manual GT Editing}. Furthermore, users can manually edit the GT image according to their preferences. As shown in \cref{fig:manual}, for the editing prompt ``Make the bike red and blue", we modify the original edit to turn the front wheel of the bike red. Using this manually edited image as the GT image $I^{e}_{v_\mathrm{gt}}$, we proceed with the subsequent steps. The final 2D and 3D editing results maintain consistency with the GT image and exhibit view consistency. Our method offers users a manual editing option, enabling them to correct 2D editing results and align the model’s output with human intent. This feature introduces an additional dimension of controllable generation, allowing for enhanced customization and adaptability in the editing process.

\subsection{Ablation Study}

\begin{figure}[h]
    \centering
    \includegraphics[width=\linewidth]{figure/random.pdf}
    \caption{\textbf{Ablation Study on View Propagation}. View propagation helps obtain more view-consistent results than the GT view.}
    \label{fig:random}
\end{figure}

\noindent \textbf{Effectiveness of View Propagation}. We conduct ablation experiments to evaluate the effectiveness of view propagation. As shown in \cref{fig:random}, after fine-tuning the 2D diffusion model based on the GT image $I^{e}_{v_\mathrm{gt}}$, we sort one set of views in order based on their camera center distance to the GT view $v_\mathrm{gt}$ and another set in random order. We then proceed with subsequent steps under these different viewpoint orders. It can be observed that without view propagation, the 2D diffusion model could not achieve fully consistent results. However, with view propagation, the consistency significantly improved. This is because, once the model is fitted to the GT, it gains the ability to produce stable outputs for the GT. It also exhibits a certain level of generalization for viewpoints close to the GT camera position. However, for more distant viewpoints, due to the large gap between input images $I^{r}_{v}$, it is unable to achieve a consistent editing result.

\begin{table}[h]
    \centering
    \resizebox{\linewidth}{!}{
    \begin{tabular}{c|c|c}
        \toprule
        Method & LoRA & LoRA$_\mathrm{gt}$ + LoRA$_\mathrm{mv}$ \\
        \midrule
        Image-Image CLIP-Score ($\uparrow$) & 87.03 &  \textbf{87.46} \\
        \bottomrule
    \end{tabular}}
    \caption{\textbf{Ablation Study on Separate Fine-Tuning}. Using different LoRAs to separately fine-tune the diffusion model can achieve better performance on view consistency.}
    \label{tab:lora}
\end{table}

\noindent \textbf{Effectiveness of Separate LoRA Fine-tuning}. We also conduct ablation on the design of LoRA. As shown in \cref{tab:lora}, we compare results obtained using only a single LoRA with those achieved using different LoRAs for fine-tuning different parts. We use the image-to-image CLIP-Score as the evaluation metric. It is observed that when using two LoRAs for fine-tuning different components, our method produces better results. This is because fine-tuning the diffusion model on the same LoRA can cause disturbances to the previously acquired GT information during subsequent viewpoint fine-tuning, thereby reducing inter-view editing consistency.

\section{Visualization}
\begin{figure}[h]
    \centering
    \includegraphics[width=\linewidth]{figure/visualization.pdf}
    \caption{\textbf{Visualization of Original and Edited Image Features}. Features of edited images obtained by C$^{3}$Editor are more concentrated than the baseline model.}
    \label{fig:visualization}
\end{figure}


\noindent \textbf{Visualization of Rendered and Edited Image Features}. We visualized the image features resulting from our edits in \cref{fig:visualization}. Each point in the figure represents an image after feature extraction and dimensionality reduction. Each 2D image was feature-extracted using CLIP ViT-B/32~\cite{radford2021learning}, followed by PCA for dimensionality reduction, and these features were plotted as 2D scatter plots and density plots. In the figures, blue represents images generated by our method, gray represents those generated by the baseline method, GaussianEditor, and red represents the original rendered images. The editing prompt on the left is ``Turn him into a clown," while on the right, it is ``Turn him into Hulk." As shown, the features of the 2D edited images produced by our method are more concentrated than those from the baseline method, indicating that our method achieves stronger view consistency, nearly matching the original images' consistency. Additionally, the features generated by our method show almost no outliers or points that are confused with the original image features, demonstrating that our approach avoids the erroneous edits seen in the baseline method. Our method not only improves the consistency of the generated images but also reduces the occurrence of incorrect edits.

\begin{figure}[h]
    \centering
    \includegraphics[width=\linewidth]{figure/loss.pdf}
    \caption{\textbf{Visualization of Loss Change During Intra-GT Prior Fitting}. The loss gradually decreases as the iterations progress, indicating that the fine-tuning process effectively stabilizes the editing results for the GT view.}
    \label{fig:loss}
\end{figure}

\noindent \textbf{Visualization of Intra-GT Prior Fitting}. \cref{fig:loss} illustrates the change in $\mathcal{L}_\mathrm{intra}$ during the Intra-GT Prior Fitting phase. In this process, our goal is for each independently performed diffusion denoising process on the GT view to approximate the GT edited image. As shown in the figure, the loss gradually decreases as the iterations progress. This indicates that the fine-tuning process effectively stabilizes the editing results for the GT view, consistently producing outputs close to the GT edited image. At this phase, the 2D editing model increasingly captures the GT information and achieves stable editing on the GT view.

\section{Conclusion}

In this paper, we propose C$^{3}$Editor, a controllable and consistent 2D-lifting-based 3D editing method. Our approach creates the specific 2D editing model to assist in achieving view consistency and controllable 3D editing results. Qualitative and quantitative evaluations demonstrate that our method outperforms baseline methods in both 2D and 3D results.

\noindent \textbf{Limitations}. Our method still has certain limitations. For instance, a unique 2D editing model must be trained for each specific scene and editing prompt. Meanwhile, the 3D optimization process may also result in the outcome not being fully consistent with the GT edited image. In the future, we aim to enhance the generalization capabilities of the editing model and move towards developing a truly generic multi-view 3D editing model.

\paragraph{Acknowledgment} This work is supported by NSF award IIS-2127544 and NSF award IIS-2433768.

{
    \small
    \bibliographystyle{ieeenat_fullname}
    \bibliography{main}
}

\appendix
\setcounter{proposition}{0}
\section{Details of Theoretical Analysis} \label{app:all-proofs}
\subsection{Conditions for TACT to correct a wrong prediction}
\label{app:prop_correct}
We first restate Proposition~\ref{pp:improve} as follows:
\begin{proposition} \label{app:pp:improve}
For any $z$ that is misclassified by the learned decision boundary $\Delta q$, the misclassification can be corrected by using the representation obtained after removing the top-$m$ principal components, if both of the following two conditions are satisfied: 
\setcounter{equation}{3}
\begin{equation}
    y\sum_{i=1}^m\alpha_i\gamma_i <0 \quad \text{and}
    \quad 
    y\sum_{i=m+1}^d\alpha_i\gamma_i >0
    \label{app:eq:improve_1}
\end{equation}
\begin{equation} \label{app:eq:improve_2}
\left|\sum_{i=1}^m\alpha_i\gamma_i\right| >\left|\sum_{i=m+1}^d\alpha_i\gamma_i\right|
\end{equation}
\end{proposition}
\setcounter{equation}{7}
\begin{proof}
As the learned decision boundary $\Delta q$ cannot classify $z$ correctly, we have:
\begin{align}
    yz \cdot \Delta q & <0 \notag \\
    y\sum_{i=1}^d \alpha_ie_i \cdot \sum_{i=1}^d \gamma_i e_i &< 0\notag \\
    y\sum_{i=1}^d \alpha_i \gamma_i (e_i \cdot e_i) &< 0 \notag \\
    y\sum_{i=1}^d \alpha_i \gamma_i &< 0 \notag \\
    y\sum_{i=1}^m \alpha_i \gamma_i + y\sum_{i=m+1}^d \alpha_i \gamma_i&< 0 \label{app:eq:z_q}
\end{align}
TACT updates $z$ to $\hat{z}$ and $q$ to $\hat{q}$ via causal trimming, and the resulting prediction is correct if and only if $y\hat{z} \cdot \Delta \hat{q} > 0$, which leads to:
\begin{align}
    y\hat{z} \cdot \Delta \hat{q} & > 0 \notag \\
    y\sum_{i=m+1}^d \alpha_i e_i \cdot \sum_{i=m+1}^d \gamma_i e_i &> 0 \notag \\
    y\sum_{i=m+1}^d \alpha_i \gamma_i (e_i \cdot e_i) &> 0 \notag \\
    y\sum_{i=m+1}^d \alpha_i \gamma_i &> 0 \label{app:eq:hat_z_q}
\end{align}
By combining Equation \eqref{app:eq:z_q} and \eqref{app:eq:hat_z_q}, we can derive:
\begin{equation}
    y\sum_{i=1}^m \alpha_i \gamma_i < -y\sum_{i=m+1}^d \alpha_i \gamma_i < 0
    \label{eq:top_m_q}
\end{equation}
In addition: 
\begin{equation}
    \left|\sum_{i=1}^m \alpha_i \gamma_i\right| > \left|\sum_{i=m+1}^d \alpha_i \gamma_i \right|
\end{equation}
\end{proof}

\subsection{Conditions for trimmed representations to preserve causal features}
\label{app:prop_rep}
\begin{proposition}
[Causal Preservation]
\label{app:pp:correct_causal}
For any original representation $z$, the trimmed representation $\hat{z}$ preserves the correct prediction under the causal decision boundary $\Delta p$ 
if any one of the following conditions holds:
    \setcounter{equation}{5}
    \begin{equation}
        \begin{cases}
        y\sum\limits_{i=1}^m\eta_i\alpha_i\gamma_i = 0 \\
        y\sum\limits_{i=1}^m\eta_i\alpha_i\gamma_i < 0 \\
        0<y\sum\limits_{i=1}^m\eta_i\alpha_i\gamma_i < y\sum\limits_{i=1}^d\eta_i\alpha_i\gamma_i
    \end{cases}
    \label{app:eq:correct_causal}
    \end{equation}
\end{proposition}
Equation \eqref{eq:correct_causal} characterizes three cases: (a) the top-$m$ PCs have no contribution to the causal prediction; (b)  the top-$m$ PCs has a negative influence on the causal prediction and thus their removal is beneficial; (c) the top-$m$ PCs has a positive contribution, but the representation forms by all PCs contribute even more strongly.
When the top-$m$ PCs have no contribution to the causal predictions, they are considered non-causal features. In other words, the removed component $z-\hat{z}$ does not contain causal information. 
When the top-$m$ PCs contain causal information, $m$ should be selected such that the causal information in the top-$m$ PCs 
contributes less to the prediction compared to all the PCs, ensuring that the trimmed representation $\hat{z}$ remains causally informative.

The proof provided here corresponds to this corrected version.
\setcounter{equation}{11}
\begin{proof}
As the causal decision boundary $\Delta p$ can classify $z$ correctly, we have:
\begin{align}
    yz \cdot \Delta p & >0 \notag \\
    y\sum_{i=1}^d \alpha_ie_i \cdot \sum_{i=1}^d \eta_i\gamma_i e_i &> 0\notag \\
    y\sum_{i=1}^d \eta_i \alpha_i\gamma_i (e_i \cdot e_i) &> 0 \notag \\
    y\sum_{i=1}^d \eta_i \alpha_i\gamma_i &> 0 \notag \\
    y\sum_{i=1}^m \eta_i \alpha_i\gamma_i + y\sum_{i=m+1}^d \eta_i \alpha_i\gamma_i&> 0
    \label{app:eq:z_p}
\end{align}
By rearranging Equation \eqref{app:eq:z_p}, we can derive:
\begin{equation}
     y\sum_{i=m+1}^d \eta_i \alpha_i\gamma_i> -y\sum_{i=1}^m \eta_i \alpha_i\gamma_i
    \label{eq:top_m_p}
\end{equation}
Using causal decision boundary to predict $\hat{z}$, the prediction is correct if and only if $y\hat{z}\cdot \Delta p > 0$, which leads to: 
\begin{align}
    y\hat{z}\cdot \Delta p & > 0 \notag \\
    y\sum_{i=m+1}^d \alpha_i e_i \cdot \sum_{i=m+1}^d \eta_i\gamma_i e_i &> 0 \notag \\
    y\sum_{i=m+1}^d \eta_i \alpha_i\gamma_i (e_i \cdot e_i) &> 0 \notag \\
    y\sum_{i=m+1}^d \eta_i \alpha_i\gamma_i &> 0 \label{app:eq:hat_z_p}
\end{align}

Given Equation \eqref{eq:top_m_p}, Equation \eqref{app:eq:hat_z_p} is satisfied if any one of the following conditions holds: 

\begin{numcases}{}
    y\sum_{i=m+1}^d \eta_i\alpha_i \gamma_i> -y\sum_{i=1}^m \eta_i\alpha_i \gamma_i \geq 0 \label{app:eq:causal_1}\\
      y\sum_{i=m+1}^d \eta_i\alpha_i \gamma_i> 0 > -y\sum_{i=1}^m \eta_i\alpha_i \gamma_i \label{app:eq:causal_2}\quad 
\end{numcases}
Equation \eqref{app:eq:causal_1} leads to:
\begin{equation}
    y\sum_{i=1}^m \eta_i\alpha_i \gamma_i \leq 0
\end{equation}
By adding $y\sum_{i=1}^m \eta_i\alpha_i \gamma_i$ to Equation \eqref{app:eq:causal_2}, we can derive:
\begin{equation}
    y\sum_{i=1}^d\eta_i\alpha_i\gamma_i > y\sum_{i=1}^m\eta_i\alpha_i\gamma_i > 0
\end{equation}
\end{proof}

\subsection{Conditions for TACT to preserve a correct prediction}
\label{app:prop_trimmed_miss}
\begin{proposition} \label{app:pp:rep_space_rm_new}
Suppose  $z$ is correctly classified by the learned decision boundary $\Delta q$. The trimmed representation $\hat{z}$ obtained via TACT will still be classified correctly if either of the conditions holds: 
\begin{enumerate}
    \item $y(z-\hat{z})\Delta q \leq 0$, or
    \item $y(z-\hat{z})\Delta q > 0$, and Equation \eqref{app:eq:learned_correct} holds, assuming $\hat{z}$ already satisfies the Causal Preservation condition (Proposition~\ref{pp:correct_causal}).
    \setcounter{equation}{6}
    \begin{equation}
        \mathrm{sign}\left(\sum_{i=m+1}^d \eta_i\alpha_i\gamma_i\right) = \mathrm{sign} \left(\sum_{i=m+1}^d \alpha_i\gamma_i\right)
        \label{app:eq:learned_correct}
    \end{equation}
\end{enumerate}
\end{proposition}
 Equation \eqref{eq:learned_correct} requires that when classification relies only on the representations formed by the remaining PCs, the learned decision boundary makes the same prediction as the causal decision boundary. 
Proposition \ref{pp:rep_space_rm_new} also shows that if a correct prediction is made by the learned decision boundary, TACT will preserve this correctness as long as the removed part $z-\hat{z}$ contributes negatively or does not contribute to the prediction.
On the other hand, when the trimmed representation $\hat{z}$ contains sufficient causal information as established in Proposition \ref{pp:correct_causal}, the learned decision boundary is required to align directionally with the causal decision boundary defined by the remaining PCs.

The proof provided here corresponds to this corrected version.
\setcounter{equation}{18}
\begin{proof}
    As the learned decision boundary $\Delta q$ classify $z$ correctly, we have:
    \begin{align}
    yz \cdot \Delta q & >0 \notag \\
    y(z-\hat{z})\cdot \Delta q + y\hat{z} \cdot \Delta q &> 0
    \label{app:eq:z_q_correct}
\end{align}
We can rewrite $y\hat{z} \cdot \Delta q$ as:
\begin{align}
     y\hat{z} \cdot \Delta q &= y \sum_{i=m+1}^d \alpha_i e_i \cdot  \sum_{i=1}^d \gamma_i e_i \notag \\
     &= y\sum_{i=m+1}^d \alpha_i e_i \cdot \left(  \sum_{i=1}^m \gamma_i e_i + \sum_{i=m+1}^d \gamma_i e_i \right) \notag \\
     &= y\sum_{i=m+1}^d \alpha_i e_i \cdot \sum_{i=1}^m \gamma_i e_i + y\sum_{i=m+1}^d \alpha_i e_i \cdot \sum_{i=m+1}^d \gamma_i e_i \notag \\
     &= 0 + y\sum_{i=m+1}^d \alpha_i e_i \cdot \sum_{i=m+1}^d \gamma_i e_i \notag \\
     &= y\hat{z}\cdot \Delta \hat{q} \label{app:eq:hat_q_equal_no_hat}
\end{align}
By combining Equation \eqref{app:eq:z_q_correct} and Equation \eqref{app:eq:hat_q_equal_no_hat}, we can derive: 
\begin{equation}
    y(z-\hat{z})\cdot \Delta q + y\hat{z} \cdot \Delta \hat{q} > 0 \label{app:eq:z_q_correct_final}
\end{equation}
The updated prediction by TACT is correct if and only if $y\hat{z} \cdot \Delta \hat{q}> 0$. 
Equation \eqref{app:eq:z_q_correct_final} shows that the value of $y(z-\hat{z})\cdot \Delta q$ needs to be considered to derive the conditions under which $y\hat{z} \cdot \Delta \hat{q}> 0$.
\begin{enumerate}
    \item When $y(z-\hat{z})\cdot \Delta q \leq 0$, 
    the removed part does not positively contribute to the prediction using the learned decision boundary, 
    together with Equation \eqref{app:eq:z_q_correct_final}, we can derive: 
    \begin{equation}\label{app:eq:correct_condition_1}
        y\hat{z} \cdot \Delta \hat{q} > -y(z-\hat{z})\cdot \Delta q \geq 0 
    \end{equation}
    Equation \eqref{app:eq:correct_condition_1} suggests that $y\hat{z} \cdot \Delta \hat{q}> 0$ is always true when $y(z-\hat{z})\cdot \Delta q \leq 0$.
    
    \item When $y(z-\hat{z})\cdot \Delta q > 0$, the removed part positively contributes to the prediction using the learned decision boundary. 
    We wish to connect with the causal decision boundary to understand the conditions. 
    Therefore, we additionally assume $\hat{z}$ satisfies the Causal Preservation condition (Proposition~\ref{pp:correct_causal}), which suggests $y\hat{z}\cdot \Delta p > 0$.
    
    The updated prediction is correct, i.e. $y\hat{z} \cdot \Delta \hat{q}> 0$ if: 
    \begin{align}
        \mathrm{sign}\left(y\hat{z} \cdot \Delta p\right) &= \mathrm{sign}\left(y\hat{z} \cdot \Delta \hat{q}\right) \notag \\
        \mathrm{sign}\left(y\sum_{i=m+1}^d \alpha_i e_i \cdot \sum_{i=1}^d \eta_i \gamma_i e_i \right) &= \mathrm{sign}\left(y\sum_{i=m+1}^d \alpha_i e_i \cdot \sum_{i=m+1}^d \gamma_i e_i\right) \notag \\
        \mathrm{sign}\left(y\sum_{i=m+1}^d \alpha_i\eta_i\gamma_i (e_i \cdot e_i) \right) &= \mathrm{sign}\left(y\sum_{i=m+1}^d \alpha_i\gamma_i (e_i \cdot e_i)\right) \notag \\
        \mathrm{sign}\left(y\sum_{i=m+1}^d \alpha_i\eta_i\gamma_i \right) &= \mathrm{sign}\left(y\sum_{i=m+1}^d \alpha_i\gamma_i\right) \notag \\
        \mathrm{sign}\left(\sum_{i=m+1}^d \eta_i\alpha_i\gamma_i \right) &= \mathrm{sign}\left(\sum_{i=m+1}^d \alpha_i\gamma_i\right)
    \end{align}
\end{enumerate}
\end{proof}

\section{Data Augmentation for CivilComments}
\label{app:aug}
CivilComments considers the following demographics mentioned in a comment: male, female, LGBTQ, Christian, Muslim, other religions, Black, White.
We group the demographics into gender (male/female), sexuality (LGBTQ), religion (Christian/Muslim/other religions), and race (Black/White). 
We notice that the comments tend to mention only one of the demographics in each group, and some comments mention more than one group. 
To vary demographics, we propose to introduce new demographics to the comments.

We propose to randomly insert a sentence before or after the comment. 
The sentences being inserted are randomly drawn from a set of sentences. Each sentence in the set mentions all demographics in one of the groups. 
The sentences are not toxic, so they would not affect the toxicity rating of the comment. Toxic comments remain toxic, and non-toxic comments remain non-toxic when the sentence is added. 
We ask ChatGPT via the web interface (\url{https://chatgpt.com}) to generate 20 sentences for each demographic group.
We list the sentences below. Sentences from all groups make up the set from which we randomly sample for augmentation.

\begin{tcolorbox}[enhanced, breakable,
    colback=black!10!white, 
    colframe=black!80!white, 
    fonttitle=\bfseries,
    title=Gender(male/female),
    ]
    \squishlisttwo
      \item ``This is a post about females and males.''
      \item ``The discussion focuses on women and men.''
      \item ``Females and males are the central topic here.''
      \item ``Women and men both contribute to this conversation.''
      \item ``This explores perspectives of females and males.''
      \item ``The post highlights contributions of women and men.''
      \item ``Both females and males are part of the narrative.''
      \item ``Women and men play essential roles in this story.''
      \item ``Females and males are equally represented here.''
      \item ``This covers aspects of both women and men.''
      \item ``This is a post about women and men.''
      \item ``The discussion centers on ladies and gentlemen.''
      \item ``Females and males are the key focus here.''
      \item ``Girls and boys both play significant roles.''
      \item ``Both genders are part of this discussion.''
      \item ``This highlights contributions from men and women.''
      \item ``Ladies and gentlemen are represented here equally.''
      \item ``The focus is on both sexes and their roles.''
      \item ``Womenfolk and menfolk shape this narrative.''
      \item ``Both males and females are included in this topic.''
    \squishend
\end{tcolorbox}

\begin{tcolorbox}[
    colback=black!10!white, 
    colframe=black!80!white, 
    enhanced, breakable, 
    title=Sexuality (LGBTQ),
    fonttitle=\bfseries
]
    \squishlisttwo
        \item ``This is a post about LGBTQ+ and heterosexual individuals.''
      \item ``The discussion focuses on sexual minorities and heterosexual communities.''
      \item ``This highlights experiences of both LGBTQ+ and cisgender people.''
      \item ``The post compares queer and non-queer perspectives.''
      \item ``This covers topics relevant to both LGBTQ+ and straight groups.''
      \item ``Gender-diverse and cisgender voices are included in this conversation.''
      \item ``The focus is on LGBTQ+ and heterosexual rights and issues.''
      \item ``Both sexual minorities and heterosexual people’s experiences are addressed here.''
      \item ``This post examines the lives of gender-nonconforming and cisgender individuals.''
      \item ``The post explores the intersection of queer and non-queer identities.''
      \item ``LGBTQ+ and heterosexual people both contribute to this topic.''
      \item ``This content engages with both gender-diverse and cisgender communities.''
      \item ``The article offers insights into the experiences of LGBTQ+ and non-LGBTQ+ individuals.''
      \item ``This is a post about LGBTQ+ and heterosexual experiences in society.''
      \item ``Both sexual minorities and heterosexual groups have a place in this discussion.''
      \item ``This conversation includes both LGBTQ+ and cisgender perspectives.''
      \item ``We explore issues affecting both sexual minorities and heterosexual individuals.''
      \item ``This is about the relationships between LGBTQ+ and heterosexual people.''
      \item ``The focus is on creating unity between LGBTQ+ and cisgender communities.''
      \item ``This post discusses challenges faced by both gender-diverse and cisgender people.''
    \squishend
\end{tcolorbox}

\begin{tcolorbox}[
    colback=black!10!white, 
    colframe=black!80!white, 
    enhanced, breakable, 
    title=Religion (Christian/Muslim/other religions),
    fonttitle=\bfseries
]
    \squishlisttwo
      \item ``This is a post about Christians, Muslims, and followers of other faiths.''
      \item ``The discussion focuses on Christians, Muslims, and practitioners of different religions.''
      \item ``This highlights the experiences of Christians, Muslims, and believers from various traditions.''
      \item ``The post compares Christian, Muslim, and other spiritual practices.''
      \item ``This covers topics relevant to Christians, Muslims, and people of other religious backgrounds.''
      \item ``The voices of Christians, Muslims, and adherents of different faiths are included in this conversation.''
      \item ``The focus is on Christian, Muslim, and interfaith perspectives.''
      \item ``Both Christians, Muslims, and people of other beliefs contribute to this discussion.''
      \item ``This post examines the lives of Christians, Muslims, and followers of other religions.''
      \item ``The post explores the intersection of Christianity, Islam, and other spiritual practices.''
      \item ``Christians, Muslims, and people from diverse faiths share common values of compassion.''
      \item ``This content engages with Christians, Muslims, and those from various religious traditions.''
      \item ``The article offers insights into the teachings of Christians, Muslims, and other faith communities.''
      \item ``This is a post about Christians, Muslims, and adherents of various world religions.''
      \item ``Both Christians, Muslims, and individuals from different belief systems are included in this conversation.''
      \item ``The focus is on how Christians, Muslims, and people of other religions practice faith.''
      \item ``This conversation includes insights from Christians, Muslims, and followers of other spiritual paths.''
      \item ``We’ll explore issues affecting Christians, Muslims, and people from various religious backgrounds.''
      \item ``This is about the relationships between Christians, Muslims, and those of other beliefs.''
      \item ``The post discusses shared values between Christians, Muslims, and adherents of other religions.''
    \squishend
\end{tcolorbox}

\begin{tcolorbox}[
    colback=black!10!white, 
    colframe=black!80!white, 
    enhanced, breakable, 
    title=Race (Black/White),
    fonttitle=\bfseries
]
    \squishlisttwo
        \item ``This is a post about Black and White communities.''
      \item ``The discussion focuses on African American and Caucasian experiences.''
      \item ``This highlights the perspectives of Black and White individuals.''
      \item ``The post compares the lives of Black and White people.''
      \item ``This covers topics relevant to both Black and White races.''
      \item ``The voices of African Americans and Caucasians are included in this conversation.''
      \item ``The focus is on Black and White racial dynamics.''
      \item ``Both Black and White communities contribute to this discussion.''
      \item ``This post examines the experiences of Black and White individuals.''
      \item ``The post explores the intersection of African American and European American identities.''
      \item ``Black and White people play vital roles in shaping society.''
      \item ``This content engages with the experiences of Black and White groups.''
      \item ``The article offers insights into the lives of Black and White people in different settings.''
      \item ``This is a post about African American and White American experiences.''
      \item ``Both Black and White cultures have unique contributions to the world.''
      \item ``The focus is on both Black and White perspectives in social issues.''
      \item ``This conversation includes both Black and White voices.''
      \item ``We’ll explore the relationship between Black and White individuals.''
      \item ``This is about the interactions between African Americans and Caucasians.''
      \item ``The post discusses challenges faced by both Black and White communities.''
    \squishend
\end{tcolorbox}

\section{Augmentation Design and Selection}
\label{app:aug_design}
Data augmentation requires careful consideration in order to achieve strong performance. It should heuristically maximize variations along non-causal directions and minimize variations along causal directions, so that the directions corresponding to non-causal features are well identified by Principal Component Analysis. 

In practice, the augmentation can be treated as a hyperparameter to search over. The data collection process that raises variation and features that affect the prediction target should be analyzed to propose a set of augmentations that are semantically invariant with respect to the prediction target, yet introduce variability in other, non-causal aspects. 

For example, for the commonly studied image classification task, we recommend searching over general image augmentations, such as AutoAugment \cite{cubuk2019autoaugment} and RandomAugment \cite{cubuk2020randaugment}. These augmentations preserve the critical causal features, particularly the shape information of objects \cite{geirhos2018imagenettrained}, while simultaneously injecting variability into less essential aspects.
Our experiments examine the effect of different augmentation strategies on datasets where images serve as the predictive input. As shown in Table~\ref{tb:aug_sensi}, augmentation affects model performance, but AutoAugment and RandomAugment could provide consistent improvements over no adaptation.

The most effective way to select the augmentation is to test on a small subset of labeled test data. 

\begin{table}[h]
    \caption{Performance of TACT with different augmentation strategies.}
    \centering
    \small
    \begin{tabular}{l|ccccc}
    \toprule 
    Augmentation &  Birdcalls & Camelyon17\tablefootnote{The performance of AutoAugment and RandomAugment on Camelyon17 is under the removal of principal components beginning with the 2nd. We observe that removing the first principal component only results in performance degradation. We hypothesize that important causal features might be present in the first principal component.} & ImageNet-R & ImageNet-V2\\
    \midrule 
    no TTA & 22.74 & 62.31& 41.83 & 62.97 \\
    \midrule
    Stain color jitter/color jitter  & 31.14$\pm$1.69 & 70.17$\pm$0.05 & 41.78$\pm$0.01 & 61.88$\pm$0.11\\
    AutoAugment         &  27.61$\pm$2.25 & 72.04$\pm$0.12 & 43.29$\pm$0.07 & 63.33$\pm$0.10 \\
    RandomAugment       & 32.19$\pm$1.26 & 79.71$\pm$0.07 & 43.59$\pm$0.02 & 62.99$\pm$0.10 \\
    \bottomrule 
    \end{tabular}
    \label{tb:aug_sensi}
\end{table}

\section{Details of Test-Time Adaptation Experiment} \label{app:exp-details}
\subsection{Model Used for Adaptation}
\label{app:pretrain}
For Birdcalls and Camelyon, to our knowledge, there were no publicly available ViT-B/32 models trained on the datasets. 
Therefore, we train a model using the standard empirical risk minimization. The training scripts and models can be found at our code repository \url{https://github.com/NancyQuris/TACT}. 
The details of the training are:
\squishlisttwo
    \item Birdcalls uses a batch size of 16 and is trained for 100 epochs. AdamW is employed as the optimizer, with a learning rate of 5e-5 and weight decay of 0.001. As specified in \cite{gao2023out}, the training starts from a weight pretrained on ImageNet, and the best model is selected by macro F1 on the in-distribution validation split.
    \item Camelyon17 uses a batch size of 32 and is trained for 30 epochs. SGD is employed as the optimizer, with a learning rate of 5e-5 and momentum 0.9. As instructed in \cite{WILDS}, the training starts from a randomly initialized weight, and the best model is selected by the average classification accuracy on the validation domain.
\squishend

For CivilComments, we use the model provided by Wilds \cite{WILDS}. 
The model was trained on the training domain of CivilComments using empirical risk minimization. 
The model can be found in \url{https://worksheets.codalab.org/rest/bundles/0x17807ae09e364ec3b2680d71ca3d9623/contents/blob/best_model.pth}.
 

For ImageNet-R and ImageNet-V2, we use the model published by torchvision. The model was trained on ImageNet using empirical risk minimization. The pretrained weight \url{ViT_B_32_Weights.IMAGENET1K_V1} is loaded to the model for test-time adaptation. 
 

\subsection{Hyperparameter Search Space}
\label{app:hyper}
We perform a grid search to find the best hyperparameters for the baseline methods we compared with.
For backpropagation-free methods, here list the details of the hyperparameters searched: 
\squishlisttwo
    \item T3A: Following \cite{t3a}, $M$, the number of representations stored to compute the centroid of each class is searched in $\{$1,5,20,50,100, N/A$\}$, where N/A means storing all representations.

    \item LAME: Following \cite{lame}, the $k$ used in $k$-nearest neighbours is searched in $\{$1,3,5$\}$, and the kernel to compute distance is searched in $\{$kNN, linear, rbf$\}$.

    \item FOA: Following \cite{foa}, we use 3 prompts. The population size is set to $4 + 3 \times \log(\text{prompt dim})$. The $\lambda$ to balance entropy and representation distance is searched in $\{$0.2, 0.4$\}$.
\squishend

For all backpropagation-based methods, we search the learning rate in $\{$1e-3, 1e-4, 1e-5, 1e-6$\}$. The adaptation is performed in a non-episodic way.
For other hyperparameters used in each method, the details are listed below:
\squishlisttwo
    \item SHOT: The method was originally proposed for source-free domain adaptation \cite{shot}. Following \cite{t3a} that adapts it as a TTA strategy, $\beta$, the hyperparameter to balance information maximization and cross entropy, is set to 0.1. The hyperparameter to filter confident pseudo-labels is set to 0.9. 
    Adam is used as the optimizer. The feature extractor is updated during adaptation. The adaptation step is set to 1. 
    
    \item Tent: Following \cite{tent}, SGD is used as the optimizer with momentum 0.9. The affine parameters of normalization layers are updated during adaptation. The adaptation step is set to 1.
    
    \item SAR: Following \cite{sar}, the margin $E_0$ is set to 0.4$\times \ln C$, where $C$ is the number of classes. To recover the model, the moving average factor is set to 0.9, and the reset constant is set to 0.2. 
    SGD is used as the base optimizer with sharpness-aware minimization (SAM). The momentum for SGD is set to 0.9. $\rho$ in SAM is set to 0.05. The affine parameters of shallow normalization layers are updated. Normalization layers in the $9^{th}$-$11^{th}$ block in the feature extractor are frozen during adaptation. The adaptation step is set to 1. 
    
    \item DeYO: Following \cite{deyo}, we search over the three augmentations $\{$patch shuffling, pixel shuffling, occlusion$\}$ to destory causal features. 
    The patch size in patch shuffling is set to 4. For occlusion, the occlusion size is set to $\left(H/2\right) \times \left(W/2\right)$, where $H$ and $W$ stand for the height and width of the image. 
    The occulsion starts from $\left(H/4\right)^{th}$ row and $\left(W/4\right)^{th}$ column. 
    The DeYO margin is set to 0.5$\times \ln C$, and the margin $E_0$ is set to 0.4$\times \ln C$, where $C$ is the number of classes. The PLPD threshold is searched in $\{$0.2, 0.3, 0.5$\}$. 
    SGD is used as the optimizer with momentum 0.9. The affine parameters of shallow normalization layers are updated. Normalization layers in the $9^{th}$-$11^{th}$ block in the feature extractor are frozen during adaptation. The adaptation step is set to 1.
    
    \item TAST: Following \cite{tast}, we search the number of nearby support examples $N_s$ in $\{$1, 2, 4, 8$\}$. $M$, the number of support examples per class is searched in $\{$1,5,20,50,100, N/A$\}$, where N/A means storing all representations. 
    The number of adaptation modules $N_e$ is set to 20. 
    Adam is used as the optimizer. The trainable module added on top of the feature extractor is adapted. The adaptation step is searched in $\{$1, 3$\}$.
    
    \item TSD: Following \cite{tsd}, the hyperparameter for feature filter $M$ is searched in  $\{$1, 5, 20, 50, 100, N/A$\}$, where N/A denotes no entropy filter. The tradeoff parameter $\lambda$ to balance TSD loss and MSLC loss is set to 0.1. 
    Adam is used as the optimizer. Adapting  $\{$affine parameters, classifier, feature extractor, all parameters$\}$ is searched. 
    The adaptation step is set to 1. 
    
    \item PASLE: Following \cite{pasle}, we search the the threshold in $\{$0.2, 0.4, 0.6, 0.8$\}$. 
    The threshold gap is set to 0.1. The $\tau_\text{des}$ is searched in $\{$1e-3, 1e-4$\}$. 
    The buffer size is set to 16, 1/4 of the batch size we used. 
    % The temperature to scale logits is set to 0.3.
    Adam is used as the optimizer. Adapting  $\{$affine parameters, classifier, feature extractor, all parameters$\}$ is searched. 
    The adaptation step is set to 1. 
\squishend

\subsection{Hardware and Software Used}
We perform experiments on the NVIDIA V100 GPU with 32GB memory. 
When the batch size is set to 64, the memory of 1 GPU is sufficient to perform test-time adaptation using TACT as well as all the baseline methods.  

We implement TACT using PyTorch 2.1.2. 
Singular vector decomposition implemented by \texttt{torch.linalg.svd()} is used to compute the principal components, as it is computationally more stable than spectral decomposition. 
Since the covariance matrix is a symmetric positive semi-definite matrix, the singular vectors are the same as the eigenvectors.


\section{Additional Performance Study} 
\label{app:additional-exp}

\subsection{TTA Performance on Larger Models}
We examine TACT's effectiveness on larger models, specifically ViT-B/16 for images and BERT for texts. The experiment setup is consistent with that described in Section~\ref{sec:experiments}.
Table~\ref{tb:larger_bb} presents the performance of TACT and other state-of-the-art backpropagation-free methods on the larger architectures.
Across all datasets except ImageNet-R, TACT achieves the best performance, ranking second on ImageNet-R. These results demonstrate the scalability of TACT to larger models.

The models for Birdcalls and Camelyon are trained under the same setting as that for ViT-B/32 stated in Appendix~\ref{app:pretrain}. We follow the guidance of CivilComments' publisher to train BERT. 
The models we trained are included in our code repository. 
ViT-B/16 backbone for ImageNet-R and ImageNet-V2 is published by torchvision.

\begin{table}[ht]
    \caption{Test-time adaptation performance of backpropagation-free methods on larger models. The best performance of each dataset is in bold.}
    \centering
    \begin{tabular}{l|ccccc}
        \toprule
        Method & Birdcalls & Camelyon17 & CivilComments & ImageNet-R & ImageNet-V2 \\ 
        \midrule
       No TTA & 27.10 & 65.37 & 67.62 & 44.06 & 69.57 \\
        \midrule 
       T3A  & 28.32$\pm$1.60 & 72.72$\pm$0.73 & 67.46$\pm$0.00 & 43.99$\pm$0.08 & 69.67$\pm$0.04 \\
      LAME & 27.48$\pm$1.44 & 68.50$\pm$0.11 & 67.65$\pm$0.04 & 44.04$\pm$0.04 & 69.59$\pm$0.01 \\
      FOA  & 27.89$\pm$0.54 & 67.15$\pm$0.67 & - & \textbf{47.53$\pm$2.73} & 69.68$\pm$0.04 \\
      TACT & \textbf{33.65$\pm$2.11} & \textbf{72.85$\pm$0.02} & \textbf{69.76$\pm$0.44} & 45.59$\pm$0.01 & \textbf{69.71$\pm$0.02} \\
      \bottomrule
    \end{tabular}
    \label{tb:larger_bb}
\end{table}

\subsection{Synergy with Training-time Augmentation} 
The ``no TTA'' baselines of BirdCalls, Camelyon17, and CivilComments are trained without the augmentations used by TACT to identify and reduce non-causal features.  
To assess TACT's synergy with training-time augmentation, we trained models using the same augmentations as those applied by TACT and then performed test-time adaptation. 
For ImageNet-R and ImageNet-V2, the ``no TTA'' baseline provided by torchvision was trained with AutoAugment using the ImageNet policy. 

Table~\ref{tb:train_time} shows the test-time adaptation performance of TACT on models trained with the same augmentation strategy.  
The results show that, even when models are trained with these augmentations, TACT further improves test-time performance. This highlights TACT’s ability to synergize with training-time augmentation and provides strong evidence of its effectiveness and generalizability.


\begin{table}[ht]
    \caption{Test-time adaptation performance of TACT with training-time augmentation models.}
    \centering
    \small
    \begin{tabular}{l|ccccc}
        \toprule
         & Birdcalls & Camelyon17 & CivilComments & ImageNet-R & ImageNet-V2\\
        \midrule
        no TTA (train time aug) & 29.86  & 74.09 & 64.60 &41.83 & 62.97\\
        + TACT & 30.57$\pm$0.96 &  77.27$\pm$0.03 & 68.84$\pm$0.20 & 43.29$\pm$0.07& 63.33$\pm$0.10\\ 
        \bottomrule
    \end{tabular}
    \label{tb:train_time}
\end{table}

\subsection{TTA Performance under Different Batch Size}
We study the test-time adaptation performance of TACT on ImageNet-R when the test batch size varies. Table~\ref{tab:bs} shows the result when the test batch size is set to 1, 4, 16, 64 and 128, respectively. 
The performance remains stable across different batch sizes. Even with a batch size of 1, the performance only decreases by 0.06\% compared to a batch size of 64. Moreover, TACT still improves performance by 1.7\% over the no-adaptation baseline when only one sample is available per batch during adaptation. 
The result suggests that TACT is robust to variations in batch size, maintaining high performance even when batch sizes are small. This makes it well-suited for situations where the number of test samples per batch is constrained.

\begin{table}[ht]
    \caption{Test-time adaptation performance (\%) of TACT on ImageNet-R under different batch sizes.}
    \small
    \centering
    \begin{tabular}{c|ccccc}
    \toprule
         no TTA & batch size = 1  & batch size = 4 & batch size = 16 & batch size = 64 & batch size = 128 \\
        \midrule 
         41.83 & 43.53$\pm$0.02 & 43.51$\pm$0.03 & 43.55$\pm$0.06 & 43.59$\pm$0.02 & 43.56$\pm$0.03 \\
     \bottomrule
    \end{tabular}
    \label{tab:bs}
\end{table}

\subsection{Computational Cost}
We compared the computational requirements of TACT with those of other backpropagation-free methods on the Birdcalls dataset using a ViT-b/32 backbone. 
As shown in Table~\ref{tab:compute_cost}, TACT incurs higher time and GPU memory consumption relative to alternative approaches. Nevertheless, this additional computational cost results in substantial performance gains (Table~\ref{tb:tta}), which justifies the trade-off.
Future work may explore optimization strategies, such as more efficient eigendecomposition techniques for PCA, to reduce the overhead.

\begin{table}[ht]
    \caption{Time and GPU memory required by backpropagation-free methods on Bridcalls.}
    \centering
    \begin{tabular}{l|cc}
        \toprule
        & time (second) & GPU memory (MB) \\
        \midrule
        T3A & 7.67 & 667.42\\
        LAME & 7.34 & 667.42\\
        FOA & 16.83 & 667.42\\
        TACT (num aug=128) & 112.22 & 1750.21\\
        TACT (num aug=256) & 170.00 & 2966.21 \\
        TACT (num aug=512) & 323.62 & 5398.21\\
        \bottomrule
    \end{tabular}
    \label{tab:compute_cost}
\end{table}

\subsection{Additional Visualization of Predictions after Causal Trimming}
We provide more GradCAM visualization of the original predictions and the predictions made by TACT on samples from ImageNet-R. Figure~\ref{fig:add_gradcam} shows the visualizations. 

Compared to original predictions, predictions made by TACT focus less on non-causal information. For example, TACT pays less attention to the background of the warplane example, and the blowfish example. 
The focus on the information that is semantically correlated with the class is retained in predictions made by TACT in the above examples. 
When the causal information is not important to the original prediction, prediction made by TACT leverages the causal information and thus turn the wrong prediction correct, as shown in the example of jellyfish and bloodhound.  
\begin{figure}[ht]
    \centering
    \begin{subfigure}{0.497\linewidth}
      \centering
        \hspace{1.5em} {\scriptsize Input} \hspace{3.5em} {\scriptsize GradCAM} \hspace{2.5em} {\scriptsize TACT-GradCAM}\\
        \includegraphics[width=0.325\linewidth]{figures/examples/correct_warplane.png}
        \includegraphics[width=0.325\linewidth]{figures/examples/correct_warplan_cam.png}
        \includegraphics[width=0.325\linewidth]{figures/examples/correct_warplane_tcam.png}
        {\scriptsize ground truth: warplane\\}
        \includegraphics[width=0.325\linewidth]{figures/examples/correct_blowfish.png}
        \includegraphics[width=0.325\linewidth]{figures/examples/correct_blowfish_cam.png}
        \includegraphics[width=0.325\linewidth]{figures/examples/correct_blowfish_tcam.png}
        {\scriptsize ground truth: blowfish\\}
        \caption{correct predictions}
    \end{subfigure}
    \begin{subfigure}{0.497\linewidth}
        \centering
        \hspace{1.5em} {\scriptsize Input} \hspace{3.5em} {\scriptsize GradCAM} \hspace{2.5em} {\scriptsize TACT-GradCAM}\\
        \includegraphics[width=0.325\linewidth]{figures/examples/wrong_jellyfish.png}
        \includegraphics[width=0.325\linewidth]{figures/examples/wrong_jellyfish_cam.png}
        \includegraphics[width=0.325\linewidth]{figures/examples/wrong_jellyfish_tcam.png}
        {\scriptsize ground truth: jellyfish; prediction: ant\\}
        \includegraphics[width=0.325\linewidth]{figures/examples/wrong_dog.png}
        \includegraphics[width=0.325\linewidth]{figures/examples/wrong_dog_cam.png}
        \includegraphics[width=0.325\linewidth]{figures/examples/wrong_dog_tcam.png}
        {\scriptsize ground truth: Weimaraner; prediction: bloodhound\\}
        \caption{wrong predictions corrected by TACT}
    \end{subfigure}
    \caption{Additional GradCAM visualizations of the original predictions and TACT's predictions.}
    \label{fig:add_gradcam}
\end{figure}


\section{Alternative Design of TACT}

\subsection{ICA to Find Non-Causal Directions}
We experiment using an alternative direction finding method, Independent Component Analysis (ICA) with TACT.
We rank the independent components by the variance of the scalars of features on the components. 
We remove the top independent components that have maximum variance.
Table~\ref{tb:ica} shows the result on the Birdcalls dataset.
ICA performs inferior to Principal Component Analysis (PCA), but better than no adaptation. 
Although ICA overcomes the orthogonality constraints of PCA, it only looks for statistically independent components and assumes each component follows a non-Gaussian distribution. 
Causal and non-causal features might not follow the non-Gaussian distribution assumption under augmentations that vary non-causal features.

\begin{table}[ht]
    \caption{Performance of TACT with ICA to find non-causal directions.}
    \centering
    \begin{tabular}{c|cc}
        \toprule
         no TTA &  TACT w/ PCA & TACT w/ ICA\\
         \midrule 
          22.74 & 31.14$\pm$1.69 & 25.53$\pm$1.06\\
         \bottomrule
    \end{tabular}
    \label{tb:ica}
\end{table}


\subsection{Causal Trimming Based on a Threshold}
We consider using the variance that the top principal components (PC) account for as a threshold to decide whether causal trimming is conducted or not. When the augmentation only changes non-causal features and causal features remain unchanged, datapoints that are invariant to augmentations should have smaller variance of the top PCs. Thus, if the variance is smaller than a threshold, causal trimmings will not be conducted on the data. As the range of variance is not known and it could change significantly, setting a numerical threshold might not be feasible. We consider normalized variance, where we divide the variance of top PCs by the sum of variances of all PCs.
Table~\ref{tb:threshold} shows the result on the Birdcalls dataset. 
Removing components based on a threshold does not outperform using no threshold.


\begin{table}[ht]
    \caption{Performance of TACT when causal trimming is performed based on a threshold $\tau$.}
    \centering
    \begin{tabular}{c|cccc}
        \toprule
          no TTA &  TACT & TACT ($\tau$=0.1) & TACT ($\tau$=0.2) & TACT ($\tau$=0.3)\\
         \midrule 
          22.74 & 31.14$\pm$1.69 & 30.99$\pm$2.18 & 31.03$\pm$2.19 & 28.03$\pm$3.12\\
         \bottomrule
    \end{tabular}
    \label{tb:threshold}
\end{table}


\end{document}
