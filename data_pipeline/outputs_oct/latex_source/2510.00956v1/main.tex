% \newcommand{\CLASSINPUTtoptextmargin}{0.7in}
\documentclass[10pt,conference]{IEEEtran}
\IEEEoverridecommandlockouts
% The preceding line is only needed to identify funding in the first footnote. If that is unneeded, please comment it out.
\usepackage{cite}
\usepackage{amsmath,amssymb,amsfonts}
\usepackage{algorithmic}
\usepackage{graphicx}
\usepackage{textcomp}
\usepackage{xcolor}
\usepackage{multicol}
\usepackage{multirow}
\usepackage{titlesec}
% \usepackage{showframe}

\usepackage[caption=false,font=footnotesize]{subfig}
% \def\BibTeX{{\rm B\kern-.05em{\sc i\kern-.025em b}\kern-.08em
%     T\kern-.1667em\lower.7ex\hbox{E}\kern-.125emX}}
\setlength{\columnsep}{0.2in}
\begin{document}

\title{Bridging the Gap Between Simulated and Real Network Data Using Transfer Learning
\thanks{This paper was submitted to IEEE ICC 2026.}
}

\author{Anonymous Authors}
\author{\IEEEauthorblockN{Carlos Güemes-Palau,
Miquel Ferriol-Galmés,
Jordi Paillisse Vilanova, 
Albert López-Brescó,}
\IEEEauthorblockN{Pere Barlet-Ros and
Albert Cabellos-Aparicio}
\IEEEauthorblockA{
% \textit{Barcelona Neural Networking Center} \\
\textit{Barcelona Neural Networking Center, Universitat Politècnica de Catalunya}\\
Barcelona, Spain\\
\{carlos.guemes, miquel.ferriol, jordi.paillisse, albert.lopez, pere.barlet, alberto.cabellos\}@upc.edu }
}
\maketitle

\begin{abstract}
Machine Learning (ML)-based network models provide fast and accurate predictions for complex network behaviors but require substantial training data. Collecting such data from real networks is often costly and limited, especially for critical scenarios like failures. As a result, researchers commonly rely on simulated data, which reduces accuracy when models are deployed in real environments.
% To address this, 
We propose a hybrid approach leveraging transfer learning to combine simulated and real-world data. Using RouteNet-Fermi, we show that fine-tuning a pre-trained model with a small real dataset significantly improves performance. Our experiments with OMNeT++ and a custom testbed reduce the Mean Absolute Percentage Error (MAPE) in packet delay prediction by up to 88\%. With just 10 real scenarios, MAPE drops by 37\%, and with 50 scenarios, by 48\%.
\end{abstract}

\begin{IEEEkeywords}
Network Modeling, Network Performance Modeling, Network Simulation, Transfer Learning
\end{IEEEkeywords}

% ---------- MAIN BODY ----------
\setlength{\textfloatsep}{7pt}
\titlespacing*{\section}{0pt}{*1}{*0.95}  % {left}{before-sep}{after-sep}
\titlespacing*{\subsection}{0pt}{*1}{*0.95}


\section{Introduction}
\label{sec:introduction}
\section{Introduction}\label{sec:introduction}
Mixture-of-Experts (MoE) is an architectural paradigm that adaptively combines predictions from multiple neural modules, known as "experts," via a learned gating mechanism. This concept has evolved from ensemble-based MoEs, where experts, jointly trained with a gating function, are often full, independent models whose outputs are combined to improve overall performance and robustness \citep{jacobs1991adaptive}. More recently, MoE layers have been integrated within larger neural architectures, with experts operating in a latent domain. These "latent MoEs" offer significant scalability benefits, especially in large language models (LLMs) \citep{shazeer2017outrageously,fedus2022switch}.
MoE makes it possible to train massive but efficient LLMs, where each token activates only a fraction of the model’s parameters, enabling specialization, better performance, and lower computational cost compared to equally sized dense models.

Regardless of their specific implementation, conventional MoE systems typically produce point estimates, lacking a direct quantification of their uncertainty. In critical applications, this absence of uncertainty information hinders interpretability, making it difficult for users to gauge the reliability of a prediction and limits informed decision-making, as the system cannot express its confidence or identify ambiguous cases. Importantly, the learned gating mechanism, which dictates the relative contribution of each expert, does not take into account expert confidence, potentially leading to suboptimal routing decisions.

In this work, we propose Mixture-of-Gaussians with Uncertainty-based Gating (MoGU), a framework for uncertainty-aware MoE architectures, which provides explicit uncertainty quantification for both individual experts and the overall MoE model. Our approach fundamentally reimagines the expert's output: instead of a point estimate, we model each expert's prediction as a random variable drawn from a normal distribution. In this setup, each expert simultaneously predicts both the mean (the label estimate) and variance of the distribution, representing its predictive uncertainty. This shift enables a more nuanced understanding of expert behavior and the derivation of the overall model's uncertainty. Furthermore, we introduce a novel gating mechanism where the estimated uncertainty of each expert directly informs its relative contribution to the overall MoE prediction, bypassing the need for a separate gating function typically found in traditional MoE setups. This creates a self-aware MoE where more confident experts naturally exert greater influence.

We evaluate MoGU on time series forecasting as our primary regression task. This choice is motivated by the inherent uncertainty in real-world time series data and the wide variety of expert architectures applicable to forecasting tasks across numerous domains \citep{time_series_survey, wang2024deep}. Our evaluation spans various expert types, forecasting benchmarks and forecasting horizon sizes, allowing for a comprehensive assessment of our method's efficacy. MoGU is shown to consistently yield more accurate forecasts compared to input-based gating MoE architectures, while simultaneously, providing uncertainty estimates that are positively correlated with prediction error. These estimates are available at both the individual expert and overall model levels. By further distinguishing between aleatoric (data-related) and epistemic (model-related) uncertainty, MoGU offers valuable insights into the source of a model's uncertainty. We also conducted a detailed ablation study to validate our key design choices.

In summary, our contributions are as follows: 
\begin{itemize}
\item \textbf{MoGU: A Novel Framework for Uncertainty-Aware MoE Architectures}: We introduce a novel framework that directly quantifies uncertainty for both individual experts and the overall model, moving beyond conventional point estimates. A key innovation is a routing mechanism that uses each expert’s estimated predictive uncertainty to dynamically determine its contribution to the final MoE output, replacing traditional input-based gating mechanisms.
\item \textbf{MoGU Improves Time Series Forecasting}: Our method effectively reduces forecasting error across various benchmarks, horizon lengths, and expert architectures.
\item \textbf{MoGU Provides Meaningful Uncertainty Estimates for Time Series Forecasting}: MoGU generates uncertainty estimates at the expert-level and overall. These estimates are positively correlated with prediction error, providing valuable insight into the model's confidence and the sources of its uncertainty.
\end{itemize}

By embedding uncertainty estimation into prediction and gating, MoGU moves beyond input-based gating  MoEs toward architectures that are more accurate, transparent, and reliable.



\section{Background}
\label{sec:background}
\input{sections/background_v2}

\section{Problem statement}
\label{sec:problem_statement}
\begin{figure}[t]
    \centering
    \includegraphics[width=0.85\linewidth]{images/pipeline_v2.pdf}
    \caption{Summary of the proposed hybrid approach. Simulated network scenarios are used to first train a network model. Then, this model is fine-tuned using a smaller dataset from real-world network data.}
    \label{fig:transfer_learning_pipeline}
\end{figure}

Building accurate ML models for network behavior prediction faces a fundamental challenge: the scarcity and diversity of real-world network data. Network data often requires extensive testbed deployments or prolonged production network monitoring. Privacy concerns and hardware limitations further restrict access to critical metrics, such as detailed packet captures or router-specific configurations.
In contrast, simulated data is abundant and diverse but fails to capture the nuances of real-world network dynamics, including hardware-specific behaviors and unexpected edge cases. This mismatch results in models that perform well on simulated scenarios but struggle when applied to real-world networks.

To address this issue, we propose a hybrid approach that combines the strengths of simulated and real-world data using transfer learning, illustrated in Figure~\ref{fig:transfer_learning_pipeline}.
The core idea is to train an ML network model on simulated data and later refine it with real-world network data. We begin by leveraging the diversity and abundance of synthetic scenarios to learn generalized patterns. Using a network simulator, for example, we can generate a comprehensive dataset with diverse network configurations and scenarios. Then, we fine-tune the model by transferring weights from the simulated network model while using a small dataset of real-world network data to adapt it to the specific environment. By doing so, we aim to bridge the gap between simulation and reality, enabling the model to make accurate predictions in real-world scenarios with minimal reliance on extensive real-world datasets.

%\section{Testbed}
%\label{sec:testbed}
%\input{sections/testbed}

\section{Methodology and experimental design}
\label{sec:methodology_and_experimental_design}
% Explain environment/problem
% To address the problem of bridging the gap between simulated and real-world network data, we propose leveraging transfer learning with the RouteNet-Fermi model~\cite{ferriolgalmés2022routenetfermi}. Our approach involves two main components: (1) training the RouteNet-Fermi model on simulated data and (2) fine-tuning it with real-world network data. This methodology enables us to combine the broad generalization capabilities gained from simulation with real-world specificity.

Our approach involves two main components: (1) training the RouteNet-Fermi~\cite{ferriolgalmés2022routenetfermi} model on simulated data and (2) fine-tuning it with real-world network data. This methodology enables us to combine the broad generalization capabilities gained from simulation with real-world specificity.

\subsection{Model architecture}

We use a modified RouteNet-Fermi architecture to predict network performance metrics. The architecture can be decomposed into three main blocks:
\begin{enumerate}
    \item Encoding: Multi-layer perceptrons (MLPs) generate initial embeddings for network elements.
    \item Message Passing Algorithm (MPA): The embeddings are refined using the relationships between network elements by employing Gated Recurrent Units (GRU)~\cite{cho2014learningphraserepresentationsusing}.
    \item Readout: The final flow embeddings are used to predict performance metrics via an MLP.
\end{enumerate}

It should be noted that RouteNet-Fermi assumes stationary traffic, a condition that does not always apply to real-world network data. In turn, we adapt RouteNet-Fermi to non-stationary traffic by splitting network scenarios into temporal windows and predicting performance metrics for each window individually. This ensures the stationarity assumption applies only within shorter intervals, allowing the model to adapt to changing traffic conditions as in real-world scenarios.

Adapting the architecture requires two key modifications. First, the input features are adjusted to include window-specific attributes rather than global flow-level parameters. This includes features such as flow bandwidth and packet rate per window. Second, we introduce a GRU neural network during the MPA phase to capture inter-window dependencies. This mechanism updates queue embeddings in each window using those from the previous window, enabling the model to propagate temporal information. Overall, these measures aim to improve accuracy under non-stationary traffic conditions.

% \mpp{Adapting the architecture requires two key modifications. First, the input features are adjusted to include window-specific attributes, such as average bandwidth and packet rate within each temporal window. These attributes are crucial for capturing transient traffic dynamics in non-stationary network scenarios, where traffic conditions evolve over time. By focusing on shorter temporal intervals, the model can better track and respond to fluctuations in network performance, which static, aggregated metrics would miss.}

% \mpp{Second, to capture inter-window dependencies, we introduce a Gated Recurrent Unit (GRU) neural network during the Message Passing Algorithm (MPA) phase. The GRU is designed to retain information from previous windows and propagate it across the network, enabling the model to account for temporal patterns and trends. This enhancement allows the architecture to adapt to traffic changes over time, improving its predictive accuracy in real-world scenarios.}

% Types of TL used
\subsection{Manual transfer learning}
\label{sec:proposed_ft}
To summarize, we currently have an ML-model architecture, RouteNet-Fermi, a large dataset of simulated network scenarios, and a small dataset of real-world network scenarios. We propose using transfer learning to leverage the strengths of the simulated dataset while adapting the model to real-world conditions.
% \textbf{FIXME: This paragraph is redundant, I suggest to summarize it to 1 or 2 sentences} To summarize, we currently have an ML-model architecture, RouteNet-Fermi, and two datasets to work with. On one hand, we have a dataset made from real-world network scenarios, which is limited in size and variability, making it challenging to train a robust model directly. On the other, we have a dataset composed of simulated scenarios, that while abundant and diverse, result in sub-optimal models due to domain mismatch. To address this, we propose using transfer learning to leverage the strengths of the simulated dataset while adapting the model to real-world conditions, all without introducing excessive bias from the simulated samples.
% Reviewing Figure~\ref{fig:transfer_learning_pipeline}, our approach consists in transferring the weights from a pre-trained network model with the simulated data to act as a foundation for the network model for real-network data. By reusing the learned knowledge encoded in the pre-trained weights, the receiver model gains a significant advantage in adapting to the real-world environment with minimal data.
We start by training the network model with the simulated data to act as a foundation for the network model for real-network data. When fine-tuning, a critical decision is determining how to handle the weights of each block in the network.
Following the principles outlined in \cite{zeiler2013visualizingunderstandingconvolutionalnetworks}, we evaluate those configurations that adhere to the following guidelines:
\begin{itemize}
    \item Layer dependencies: We avoid configurations where a block is frozen or fine-tuned if preceded by a re-trained block. Otherwise, it would disrupt the natural flow of learned representations. We also avoid configurations where a block is frozen if preceded by a fine-tuned block.
    \item Trainable weights: We never freeze all blocks, as this would leave no trainable parameters for adaptation.
    \item Always transfer something: We never re-train all blocks, as it is equivalent to training the model from scratch.
\end{itemize}
% Following the principles outlined in \cite{zeiler2013visualizingunderstandingconvolutionalnetworks}, we evaluate those configurations that respect layer dependencies. That is, avoid configurations where a block is frozen or fine-tuned if preceded by a re-trained block as otherwise, it would disrupt the natural flow of learned representations. For the same reason we also avoid configurations where a block is frozen if preceded by a fine-tuned block.

The resulting testbed configurations are listed in Table~\ref{tab:results} in the evaluation. We split the network into blocks rather than individual layers to align with RouteNet-Fermi’s architecture. Unlike traditional NNs like MLPs, which are sequential, RouteNet-Fermi operates more like an ensemble of smaller NNs that work in parallel. For instance, the MLPs in the encoding block process individual network elements independently. Grouping these layers into blocks provides a structured approach to fine-tuning while ensuring that dependencies between blocks are respected. Furthermore, the shallow depth of the RouteNet-Fermi's internal NNs, with the deepest component being a 3-layered MLP, limits the benefit of fine-tuning individual layers.

\begin{figure}[t]
    \centering
    \includegraphics[width=0.8\linewidth]{images/fine_tuning_example.pdf}
    \caption{Visual example of fine-tuning a RouteNet-Fermi~\cite{ferriolgalmés2022routenetfermi} model, where the Encoding is frozen, the MPA is fine-tuned, and the Readout is re-trained.}
    \label{fig:fine_tuning_example}
\end{figure}

In Figure~\ref{fig:fine_tuning_example}, we show an example of how the model can be fine-tuned. In this example, the chosen fine-tune configuration was to freeze the Encoding block, fine-tune the MPA block, and re-train the Readout block. As a result, we only transfer the Encoding and MPA weights from the donor model, while the Readout block's weights are randomly initialized as in traditional training. Then, during the fine-tuning training, the Encoding block is excluded so as not to modify its weights. Note that the fine-tuning training is otherwise similar to the original training process, but using the real-network samples and with a diminished learning rate ($\approx10\times$  smaller).

\subsection{Automated transfer learning}
\label{sec:automated_ft}

In addition to the previous manual configurations, we also test our approach using automated fine-tuning approaches from the state-of-the-art. These do not require manually deciding which blocks to freeze, fine-tune, or retrain, reducing trial-and-error:

\begin{itemize}
    \item Autofreeze~\cite{liu2021autofreeze}: This method consists of loading all donor weights and setting them as trainable. During training, blocks whose weight gradients fall under a threshold are frozen, allowing weights to be adjusted while minimizing computational costs.
    \item L2-SP~\cite{pmlr-v80-li18a}: This method consists of adding a regularization term in the loss function involving the L2-distance between the receiver and donor weights. This guides the learning of the receiver model and is more effective at avoiding overfitting than the standard L2 regularization.
    \item GTOT-Tuning~\cite{zhang2022fine}: A more advanced version of L2-SP meant for Graph Neural Networks (GNNs). Instead of comparing weights, it measures the differences between node embeddings after the MPA using the Masked Wasserstein Distance (MWD). The mask in the MWD allows it to incorporate relational information.
\end{itemize}

\subsection{Testbed}
\label{subsec:testbed}
\begin{figure}[t]
    \centering
    \includegraphics[width=0.90\linewidth]{images/testbed.pdf}
    \caption{Diagram summarizing the testbed's structure.}
    \label{fig:testbed}
\end{figure}

To collect real network samples, we use a custom testbed with up to 8 real routers and traffic generators connected via switches. VLAN-based configurations allow emulation of diverse network topologies. Links range from 1 to 40 Gbps to simulate modern network conditions. Traffic is generated using 2–8 servers running MGEN and Tcpreplay. An optical splitter enables passive traffic capture for analysis, ensuring low interference. Figure~\ref{fig:testbed} shows the testbed's structure.


% \begin{itemize}
%     \item The testbed comprises up to 8 Huawei NetEngine 8000 M1A routers, interconnected via two Huawei S5732-H48UM 2CC 5G Bundle switches. A third Cisco WS-C4506-E switch links the traffic generators to one of the router switches, enabling flexible traffic configurations.
%     \item Traffic is generated using 2 to 8 servers running the MGEN~\cite{MGEN} and Tcpreplay~\cite{Tcpreplay} software. These servers create traffic for each source/destination pair, allowing the testbed to replicate a variety of network topologies through VLAN-based routing paths.
%     \item An optical splitter duplicates both ingoing and outgoing traffic from the traffic generators to a packet capture server. This server processes the captured data for subsequent analysis. By using optical splitters, we minimize any potential impact on traffic transmission or recording accuracy, ensuring high-fidelity measurements.
%     \item The routers and traffic generators are connected to the switches using 1 Gbps links, representative of modern network conditions. To prevent bottlenecks in the testbed's control plane, higher-capacity links are employed: 10 Gbps links connect the traffic generator switch to one of the router switches, and two 40 Gbps links, configured in trunk mode, connect the two router switches.
% \end{itemize}

\section{Evaluation}
\label{sec:evaluation}

\section{Results}
\label{sec:results}


\subsection{Methodology}
\label{sec:methodology}

\textbf{Video Diffusion Model.} We evaluate \X using the following open source, widely available video models to generate the videos:
\begin{itemize}
    \item Wan 2.1~\cite{wan} 1.3B, 14B, 480p and 720p models, at 81 frames.
    \item HunyuanVideo 720p~\cite{hunyuanvideo} at 720p. 81 frames.
\end{itemize}

All experiments are conducted using bfloat16 precision. We implement CUDA kernels for \X with the aid of device primitives from ThunderKittens~\cite{thunderkittens} for a H100 GPU. To evaluate the quality of the videos generated, we use the VBench~\cite{vbench} VLM benchmarking scores, alongside visual comparisons of frames from the generated videos. We test two configurations of \X: one using the caching strategy to determine the mask (\X-cached), and the other using the pooling strategy (\X-pooling). For the \X-cached strategy, the threshold is set to $0.5/N$, where $N$ is the number of embedding vectors in the latent space representation of the video. The attention mask is cached once every $15$ DiT iterations. We compare \X with two prior works that use block sparse attention to leverage sparsity in attention scores in DiTs: Radial Attention~\cite{radialattn} and SparseVideoGen~\cite{sparsevideogen}. SparseVideoGen~\cite{sparsevideogen} uses a local-global attention computation strategy (windowed attention) across spatio templaral tokens. Radial attention uses a static attention mask that leads to an exponentially decaying compute density along the antidiagonal of the attention map.

\subsection{End-to-end Speedup}
\label{sec:e2espeedup}

Fig.~\ref{fig:e2e_normalized} shows the end-to-end time required to generate the video, normalized to baseline. We observe that \X is able to achieve an average speedup of $1.48\times$ and up to $1.65\times$. 
\X achieves a speedup as a result of accelerating the attention computation time during training. Fig.~\ref{fig:attn_normalized} shows the average runtime needed to compute the attention of every layer, normalized to the PyTorch implementation baseline. For the attention computation, \X achieves a speedup of $1.93\times$ on average, up to $2.38\times$. \X achieves a higher speedup when generating videos at 720p.
Our approach achieves a higher speedup of $1.2\times$ compared to SparseVideoGen~\cite{sparsevideogen} and $1.22\times$ compared to RadialAttention~\cite{radialattn}. The observed speedup comes from skipping a larger fraction of attention scores. However, this advantage diminishes at higher video resolutions (720p compared to 480p). This is because, in self-attention, interactions between blocks of embeddings that correspond to distant regions of the video are typically zero. As the resolution increases, each embedding vector covers a smaller region of the input, leading to a greater number of embeddings. This increases the proportion of zero-valued attention scores, which block-sparse attention can skip. Consequently, while more scores are skipped, the relative speedup achieved by \X decreases.



\begin{figure}[!htb]
    \centering
    \includegraphics[trim=0 90 0 80, clip, width=\linewidth]{figs2/e2enorm_speedup.pdf}
    \caption{Normalized end-to-end speedup in seconds for video generation.}
    \label{fig:e2e_normalized}
\end{figure}

\begin{figure}[!htb]
    \includegraphics[trim=0 90 0 90, clip, width=\linewidth]{figs2/attnnorm_speedup.pdf}
    \caption{Normalized attention computation speedup compared to baseline.}
    \label{fig:attn_normalized}
\end{figure}

  

% \begin{figure}[!htb]
%     \centering
%     \begin{subfigure}{0.5\textwidth}
%         \includegraphics[width=\linewidth]{figs2/e2enorm_speedup.pdf}
%         \caption{Normalized end-to-end speedup in seconds for video generation.}
%         \label{fig:e2e_normalized}
%     \end{subfigure}
%     \hfill
%     \begin{subfigure}{0.5\textwidth}
%         \includegraphics[width=\linewidth]{figs2/attnnorm_speedup.pdf}
%         \caption{Normalized attention computation speedup compared to baseline.}
%         \label{fig:attn_normalized}
%     \end{subfigure}
%     \caption{Normalized performance comparison for video generation.}
%     \label{fig:normalized_comparison}
% \end{figure}


\subsection{Qualitative Analysis}
\label{sec:qualitative_analysis}


Table~\ref{tab:vbench} shows the VBench~\cite{vbench} video benchmarking results when compared to the baseline. We observe that \X achieves negligible degradation in quality when compared to the baseline.

% Please add the following required packages to your document preamble:
% \usepackage[table,xcdraw]{xcolor}
% Beamer presentation requires \usepackage{colortbl} instead of \usepackage[table,xcdraw]{xcolor}
\begin{table*}
\centering
\caption{VBench quality metrics}
\label{tab:vbench}
\begin{tabular}{|l|r|r|r|r|}
\hline
                          & \multicolumn{1}{l|}{\textit{\textbf{\begin{tabular}[c]{@{}l@{}}Aesthetic\\ Quality\end{tabular}}}} & \multicolumn{1}{l|}{\textit{\textbf{\begin{tabular}[c]{@{}l@{}}Subject\\ Consistency\end{tabular}}}} & \multicolumn{1}{l|}{\textit{\textbf{\begin{tabular}[c]{@{}l@{}}Background\\ Consistency\end{tabular}}}} & \multicolumn{1}{l|}{\textit{\textbf{\begin{tabular}[c]{@{}l@{}}Overall\\ Consistency\end{tabular}}}} \\ \hline
Wan-1.3B 480p baseline    & 0.601                                                                                              & 0.936                                                                                                & 0.958                                                                                                   & 0.23                                                                                                 \\
Wan-1.3B 480p FGAttn      & 0.605                                                                                              & 0.939                                                                                                & 0.96                                                                                                    & 0.23                                                                                                 \\ \hline
Wan-1.3B 720p Baseline    & 0.61                                                                                               & 0.944                                                                           & 0.962                                                                                                   & 0.233                                                                                                \\
Wan-1.3B 720p FGAttn      & 0.61                                                                                               & 0.944                                                                                                & 0.964                                                                                                   & 0.232                                                                                                \\ \hline
Wan-14B 480p baseline     & 0.623                                                                                              & 0.953                                                                                                & 0.97                                                                                                    & 0.25                                                                                                 \\
Wan-14B 480p FGAttn       & 0.616                                                                                              & 0.952                                                                                                & 0.975                                                                                                   & 0.247                                                                                                \\ \hline
Wan-14B 720p baseline     & 0.621                                                                                              & 0.945                                                                                                & 0.969                                                                                                   & 0.248                                                                                                \\
Wan-14B 720p FGAttn       & {\color[HTML]{000000} 0.619}                                                                       & 0.942                                                                                                & {\color[HTML]{000000} 0.961}                                                                            & 0.245                                                                                                \\ \hline
Hunyuan-13B 720p baseline & 0.62                                                                                               & 0.944                                                                                                & 0.962                                                                                                   & 0.239                                                                                                \\
Hunyuan-13B 720p FGAttn   & 0.62                                                                                               & 0.94                                                                                                 & 0.962                                                                                                   & 0.239                                                                                                \\ \hline
\end{tabular}
\end{table*}

Figs.~\ref{fig:hunyuanvideo}, \ref{fig:wan1_3b} and \ref{fig:wan14b} show the visual representation of the produced video compared to the original (the top row of each set of videos represents the baseline video) for the HunyuanVideo model, Wan 1.3B model, and the Wan 14B model, respectively. We find that across all the prompts tested here, \X can recover the original video with no quality degradation. \X also retains the generated video style and does not significantly shift the distribution captured by the underlying model. 





\subsection{Ablation Study}
\label{sec:ablation}

Fig.~\ref{fig:ablation} depicts the average attention computation time for video generation as the threshold parameter is varied. We sweep the threshold parameter from $0.1/N$ to $1/N$, where $N$ is the number of embedding vectors in the latent space representation of the video. A higher threshold enables skipping a larger amount of computation, thereby leading to a speedup.  

\begin{figure}[!htb]
    \centering
    \includegraphics[width=\linewidth]{figs2/ablation.pdf}
    \caption{Normalized video generation time at different thresholds applied to \X-cached.}
    \label{fig:ablation}
\end{figure}



\begin{figure*}[!htb]
    \centering
    \includegraphics[width=\linewidth]{figs2/hunyuan.pdf}
    \caption{Samples of videos generated using baseline HunyuanVideo model, and \X-HunyuanVideo (The baseline generates first row, second row generated using \X)}
    \label{fig:hunyuanvideo}
\end{figure*}




\begin{figure*}[!htb]
    \centering
    \includegraphics[width=\linewidth]{figs2/wan1_3b.pdf}
    \caption{Samples of videos generated using baseline Wan-1.3B model, and \X-Wan1.3B. (First row is generated by the baseline, second row is generated using \X)}
    \label{fig:wan1_3b}
\end{figure*}





\begin{figure*}[!htb]
    \centering
    \includegraphics[width=\linewidth]{figs2/wan14b.pdf}
    \caption{Samples of videos generated using baseline Wan-14B model, and \X-Wan14B (First row is generated by the baseline, second row is generated by \X)}
    \label{fig:wan14b}
\end{figure*}




\section{Related work}
\label{sec:related_work}
\section{Related Work}
\label{sec:related_work}


\textbf{Block sparse attention.} 
Several implementations of block sparse attention~\cite{bsa, flashattn, flexattn, flashinfer, flashmask} propose a coarse-grained sparse attention mechanism that skips entire blocks of attention score computations at granularity of $64\times 64$ or $128\times 128$ at half-precision. Current block-sparse attention mechanisms either prevent further reduction of block size (do not compile) or cause significant hardware underutilization and performance overhead, since they are constrained by the tensor core matrix multiplication width (\cref{sec:motivation_skip_attn_fine_granularity}). Several works in the large language model literature~\cite{minference, xattn, flashdecode, nsa, seerattn} utilize block sparse attention to accelerate attention computation. 
% Several works in large language model literature ~\cite{minference, xattn, flashdecode, nsa, seerattn} use block sparse attention to speedup attention computation.

\textbf{Block sparse attention for videoDiTs.} 
Recent works, such as Radial Attention~\cite{radialattn}, X-attention~\cite{xattn}, SparseVideoGen~\cite{sparsevideogen}, and SparseVideoGen2~\cite{sparsevideogen2}, have applied block sparse attention implementations to video diffusion models. These approaches consider a fixed sparsity pattern in the attention map based on empirical observations of significant patterns. Other works, such as Video Sparse Attention~\cite{vsa}, incorporate learned sparse attention patterns by using a parameterized model to derive the attention map mask. Both approaches utilize coarse-grained sparse attention mechanisms. In contrast, our method enables fine-grained skipping of attention blocks, providing more opportunities for skipping computation. We compare \X with SparseVideoGen and Radial Attention in~\cref{sec:results}. 
Moreover, trainable sparse attention methods such as Video Sparse Attention (VSA)~\cite{vsa} can be reformulated to generate sparse masks compatible with \X \textquotesingle s attention kernel. These methods are orthogonal to \X \textquotesingle s kernel implementation and can be used in conjunction as mask-determination strategies for \X.

% While trainable sparse attention methods can outperform training-free approaches, these methods could benefit from \X's fine-grained approach.
% Recent works such as Radial attention~\cite{radialattn}, X-attention~\cite{xattn}, SparseVideoGen~\cite{sparsevideogen}, SparseVideoGen2~\cite{sparsevideogen2} apply block sparse attention implementations for video diffusion models. They consider a fixed sparsity pattern in the attention map based on empirical observation of significant. Other works such as video sparse attention~\cite{vsa} incorporate learnt sparse attention patterns. These works use a parameterized model to derive the mask of the attention maps. Both of these works make use of coarse grain sparse attention mechanisms. Our approach allows fine-grain skipping of attention blocks enabling more opportunity to skip computation. We compare \X with SVG, radial attention in~\cref{sec:results}. Trainable sparse attention methods can make  of coarse grain sparse attention methods, and could potentially benefit from \X. 

\textbf{Other techniques to accelerate video diffusion.}
SpargeAttention~\cite{spargeattn}, SageAttention~\cite{sageattn}, and SageAttention2~\cite{sageattn2} propose general attention approximation techniques, such as quantization and token compression mechanisms, that can be applied during inference for both LLM and DiT models. Token compression-based approaches may skip essential tokens relevant to the video, which could lead to inconsistent video generation (pointed out by~\cite{radialattn}). These approaches are orthogonal to our \X.% SpargeAttention~\cite{spargeattn}, SageAttention~\cite{sageattn}, SageAttention2~\cite{sageattn2}, propose general attention approximation techniques such as quantization and token compression mechanisms that can be applied at inference time to LLM and DiT models. Approaches based on token compression skip essential tokens relevant to the video and may produce inconsistent video (\tofix{add reference here}). Quantization-based techniques are orthogonal to our approach (a lower-precision version of \X can be implemented). 






\section{Conclusion}
\label{sec:conclusion}
\section{Conclusion}

In this work, we presented a full-stack investigation of LLM unlearning, encompassing methodology, evaluation, and robustness. We established a principled taxonomy that organizes twelve representative unlearning methods into three families: {\MDiv}, {\MRep}, and {\MRej}, providing a systematic lens to understand their underlying mechanisms. Our analysis revealed that conventional multiple-choice questioning (MCQ) evaluations of unlearning effectiveness (UE) and utility retention (UT) offer an incomplete picture, and we introduced open question answering (Open-QA) as a complementary paradigm to better capture generative behaviors and expose the strengths and limitations of different methods. Furthermore, we provide a comprehensive robustness assessment across model-level and input-level attacks, revealing nuanced relationships among in-domain relearning, out-of-domain fine-tuning, quantization, and jailbreak attacks. These findings clarify the trade-offs of current unlearning algorithms and guide the design of future methods that are both effective and robust. The use of LLM, limitation and broader impact are further discussed in \textbf{Appendix\,\ref{appx:llm_usage}}, \textbf{Appendix\,\ref{appx:limit}} and \textbf{Appendix\,\ref{appx:impact}}.


% ---------- MAIN BODY END ----------

\section*{Acknowledgments}
This publication is part of the I+D+i project titled BLOSSOMS, grant PID2024-158530OB-I00, funded by MICIU/AEI/10.13039/501100011033/ and by ERDF/EU. This work is also partially funded by the Catalan Institution for Research and Advanced Studies (ICREA).
Carlos Güemes is funded by the AGAUR-FI ajuts (Grant Ref. 2023 F-1 00083) Joan Oró of the Secretariat of Universities and Research of the Department of Research and Universities of the Generalitat of Catalonia and the European Social Plus Fund.

\bibliographystyle{IEEEtran}
\bibliography{main}

\end{document}
