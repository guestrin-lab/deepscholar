\subsection{Network modeling}
State-of-the-art network models can be broadly categorized into two fields: DES and ML-based models. DES simulators such as OMNeT++\cite{Varga2019} and NS3\cite{Riley2010} have dominated the field, providing highly detailed simulations of network behavior. 
In contrast, ML-based models offer a scalable alternative to DES by leveraging data-driven approaches. GNNs have emerged as the dominant architecture in this space \cite{ wang2022xnet, li2024glancegraphbasedlearnabledigital}, with the RouteNet family of models \cite{ferriolgalmés2022routenetfermi} leading the way. DES-ML hybrids aim to accelerate DES by replacing specific simulation components with ML-based approximations. Notable examples include MimicNet~\cite{10.1145/3452296.3472926}, DeepQueueNet~\cite{10.1145/3544216.3544248} and m3~\cite{10.1145/3651890.3672243}.
% In contrast, ML-based models offer a scalable alternative to DES by leveraging data-driven approaches. These can be further divided into pure ML models and DES-ML hybrids. DES-ML hybrids aim to accelerate DES by replacing specific simulation components with ML-based approximations. Notable examples include MimicNet~\cite{10.1145/3452296.3472926}, which introduced the concept, and DeepQueueNet~\cite{10.1145/3544216.3544248} and m3~\cite{10.1145/3651890.3672243}, which refined the approach. Conversely, pure ML models rely entirely on ML algorithms. GNNs have emerged as the dominant architecture in this space \cite{ wang2022xnet, li2024glancegraphbasedlearnabledigital}, with the RouteNet family of models \cite{ferriolgalmés2022routenetfermi} leading the way. %The latest iteration, RouteNet-Fermi~\cite{ferriolgalmés2022routenetfermi}, achieves state-of-the-art performance in network modeling tasks. 
However, these models are typically trained on simulated data, which limits their accuracy and reliability in real-world networks without proper fine-tuning.

% Transfer learning
\subsection{Transfer learning in network modeling}
The application of transfer learning in network modeling remains relatively limited.
% For instance, in~\cite{10635943}, the authors identify the discrepancies between simulated and real network scenarios and address them by developing a transfer learning-based network model. Their approach employs a Neural Processes (NPs) architecture, which focuses on learning and leveraging latent variables in input data. While effective for their use case, NPs, as dense neural networks, are inherently limited in capturing the relational information present in network scenarios—a capability that Graph Neural Networks (GNNs) like RouteNet excel at.
For instance, in~\cite{10635943}, the authors employ a Neural Processes (NPs) architecture trained through a transfer learning-based network model using simulated and real network data. While effective, NPs are inherently limited in capturing the relational information present in network scenarios, unlike GNNs such as RouteNet.
In GLANCE~\cite{li2024glancegraphbasedlearnabledigital}, authors employ fine-tuning, but for transferring knowledge between performance metrics rather than addressing the challenges of adapting models to real-world network scenarios.
% In another study~\cite{li2024glancegraphbasedlearnabledigital}, the authors propose GLANCE, a GNN similar in architecture to RouteNet. However, their use of fine-tuning focuses on transferring knowledge between performance metrics rather than addressing the challenges of adapting models to real-world network scenarios.
% Transfer learning is more prevalent in fields adjacent to network modeling, such as traffic prediction~\cite{8667446, 9413270}, intrusion detection~\cite{MAHDAVI2022109542, 9065240}, and energy consumption reduction~\cite{7041046, 8879693}.
Transfer learning is more prevalent in fields adjacent to network modeling, such as traffic prediction~\cite{8667446}, intrusion detection~\cite{MAHDAVI2022109542}, and energy consumption reduction~\cite{8879693}.

% Transfer learning is more prevalent in fields adjacent to network modeling. It is widely applied in traffic prediction~\cite{9685204, 8667446, 9858136, 9662277, 9413270}, intrusion detection~\cite{MAHDAVI2022109542, BIERBRAUER2023118641, 9065240}, network function virtualization management~\cite{10.1145/3278532.3278547}, and energy consumption reduction in wireless networks~\cite{7041046, 8879693}. % Notably, \cite{9065240} employs a reinforcement learning model to optimize the transfer learning process, demonstrating its effectiveness in dynamic network environments. Similarly, \cite{9413270} introduces TL-DLCRNN, a GNN-based approach for traffic prediction, showcasing the applicability of GNNs in leveraging transfer learning for network-related tasks.