\section{Setting the Context}
\label{sec:settingcontext}

In this section, we introduce background concepts that form the basis of our approach and frame this work inside the umbrella project Exosoul~\cite{AutiliACCESS19} which is about protecting citizens' ethics and privacy in the digital world.

Understanding how individuals make ethical decisions in real-life scenarios is a crucial step in designing systems that can adapt to their users' ethical preferences~\cite{autili2025engineering}.
While various approaches have proposed the use of LLMs for moral reasoning, they are primarily designed to fine-tune the models and test their reasoning capabilities rather than assessing their inherent capacity for moral reasoning~\cite{han2022aligning}~\cite{ICSE2025PAPER}. Our end goal, however, is to automatically generate user ethical profiles utilizing LLMs, which reflect how individual users would act in an ethically charged situation. Hence, this involves evaluating LLMs' ethical reasoning capabilities.


\textbf{The choice of questionnaire.} To develop our approach, we build on top of the questionnaire introduced in~\cite{alfieri2022exosoul} and revised in~\cite{alfieri2023ethical}. The questionnaire is introduced by a multidisciplinary group, including ethicists, philosophers, and researchers engaged in applied ethics and cognitive science, to collect ethical preferences from users in real-life situations. The questionnaire is composed of questions that reflect everyday moral dilemmas, designed to manually generate users' ethical profiles from their responses. To adapt this questionnaire for our approach, we generated declarative statements from these questions (further discussed in Section~\ref{sec:approach}). This translation allows us to leverage the questionnaire in a more structured format, making it suitable for evaluating whether LLMs can identify the presence of ethically charged actions within these statements. The original questionnaire and all the used statements are presented in the replication package.\footnote{\label{replication}{https://github.com/ASE25authors/ase-aep}}

%\footnote{\label{replication}\todo{The replication package can be found at this link: https://bit.ly/3HgiquG}}



\textbf{The choice of theories.} Ethical considerations in the fields of computer science and software engineering have become increasingly important as technology advances~\cite{singh2023approaches}. The extensive use of artificial intelligence and machine learning has further made it essential to evaluate the ethical implications of such technologies~\cite{UNESCOGuidelines,ryan2020artificial,anderson2018artificial}. Various ethical theories can be employed to ensure their ethical development and application~\cite{singh2023approaches,tolmeijer2020implementations}. In this work, we evaluate whether LLMs possess ethical reasoning capabilities and identify the ethical significance of actions by evaluating their responses against three foundational moral theories: utilitarianism, virtue ethics, and deontology. The selection of these theories is based not only on their significant influence but also on their widespread adoption across both applied ethics and the broader AI and machine ethics literature~\cite{tolmeijer2020implementations,guarini2013introduction,anderson2020machine,o2012review,jedlivckova2024ethical}. Among the selected theories, Utilitarianism evaluates actions based on their outcomes, aiming to maximize overall well-being or happiness~\cite{mill2016utilitarianism}. Virtue ethics focuses on the moral character and intentions~\cite{hursthouse2017virtue}. Deontology judges actions based on adherence to moral duties or principles, regardless of the outcome~\cite{sep-ethics-deontological}. The statements we used in our approach are grounded in real-life situations representing an ethical dilemma, %each of which can be interpreted through one or more of these ethical theories, making these theories well-suited for the scope of our work.
in which each decision taken may correspond to one or more of these ethical theories, highlighting the relevance of these ethical theories to our work. %answers Rev C on why only one “best apply” and prevents ambiguity already in Sec. II
\major{In our setting, multiple theories may be applicable to the same scenario. We therefore treat the selected theory as a normative lens, an interpretive perspective used to frame the subsequent yes/no acceptability judgment. The task is not to assert the “true” or exclusive theory, but to elicit structured moral reasoning under an explicitly plural and interpretive design.}





%\textbf{Ethical profiles.} An ethical profile is a set of a user's ethical preferences. Ethical profiles are dynamic and context-sensitive, enabling the generation of context-specific profiles representing how a user would act in a specific context. Several studies have proposed methods for generating users’ ethical profiles using tools such as questionnaires, surveys, product reviews, and social media interactions~\cite{boja2019user,gilbert2023rise,dong2021profiling,alfieri2022exosoul,alfieri2023ethical}. This is precisely where project [anonymized] comes into play with one of its main objectives: automating the generation of ethical profile. Figure 1 briefly overviews the overall approach of the project for what concerns the ethical profile creation... 
%\pat{describe deeply}
%pipeline showing the main components to translate multimedia input from the user to 

\textbf{Ethical profiling.} An ethical profile is a structured representation of a user's ethical preferences, designed to reflect how the user would respond to ethically significant decisions in context-specific scenarios. These profiles are inherently dynamic and situated, evolving with the user’s behavior and values. Prior studies have proposed their construction through explicit elicitation, such as surveys, questionnaires, or review analysis~\cite{boja2019user,gilbert2023rise,dong2021profiling,alfieri2022exosoul,alfieri2023ethical}. %yet these methods are not designed to scale or adapt to real-time decision-making.
However, the continuous and manual input required from the user limits their scope and adaptability, highlighting the need to automate the generation of ethical profiles.

\begin{figure}[ht]
%\hspace*{-1.2em}
\includegraphics[width=0.49\textwidth]{figures/Final_approach1.pdf}
\caption{Automated Ethical Profile Generation}
\label{fig:approach1}
\end{figure}

The broader vision of project Exosoul~\cite{AutiliACCESS19} is to propose a modular approach to protect and empower individuals in asymmetric interactions with complex digital environments. The primary goal of the project is to mediate ethical, privacy-related, and social frictions through adaptive components that infer, negotiate, and operationalize the user's ethical stance in context. Figure~\ref{fig:approach1} shows one technical subcomponent of the approach: a pipeline for the automated interpretation of ethically relevant content and generation of the user's ethical profile. The pipeline comprises seven modules. \textit{Multimedia Input} collects inputs from user activity or environment, such as video, audio, and text. \textit{Input Translator} performs
\textit{Input Conversion} to convert the inputs into symbolic representations and \textit{Input Summarization} to condense relevant content into structured prompts. \textit{Ethical Interpreter} classifies the summarized input by theory, acceptability, and explanation using \textit{LLM(s) Prompting} to provide general-purpose interpretive ethical outputs, and \textit{Agreement Measurement} compares multiple reasoning traces to assess consistency. \textit{Ethical Profile Generator} aggregates these outputs to synthesize the user’s ethical profile. 

This paper concerns specifically the \textit{Ethical Interpreter} component and aims to evaluate whether current LLMs exhibit the reasoning capability, internal coherence, and interpretive stability required to support this component. Thus, the focus of this paper is not on ethical profile generation, rather the focus is on empirically %validate
\major{validating} the core interpretive layer necessary for supporting the automation of the ethical profile generation.

%However, these approaches often produce either generic profiles that lack contextual depth or context-specific profiles that are impractical, as they require users to provide input for each situation. Therefore, the end goal of our approach is to explore the ethical reasoning capabilities of LLMs and use them to automatically generate ethical profiles by analyzing their behavior.

%\vspace{0.2cm}
%\subsubsection{Related Work} 





%In this paper, we focus on the ethics interpreter component and explore the ethical reasoning capabilities of LLMs and use them to automatically generate ethical profiles by analyzing their behavior.