\section{Conclusion}
We presented PhysHMR, a unified framework for reconstructing physically plausible human motion from monocular videos by directly mapping visual inputs to humanoid control actions. Unlike prior methods, PhysHMR learns a visual-to-action policy that integrates physical dynamics during inference. To improve efficiency and robustness, we introduce motion distillation from a mocap-trained expert and a novel pixel-as-ray strategy that provides soft global guidance without relying on noisy 3D root predictions. 

\noindent \textbf{Limitation and Future Work.}
While PhysHMR generates high-fidelity motion, a real-to-sim gap persists due to differences in body mechanics and contact properties, which can sometimes lead to visible artifacts.  Future work will incorporate personalized physical parameters to better reflect real-world dynamics. Additionally, motion reconstruction from a single monocular video is underconstrained due to ambiguity and occlusion; using a conditional generative model instead of a deterministic policy may better capture diverse and physically plausible motions. Our current framework does not explicitly support human–scene interactions (e.g., sitting or leaning against surfaces), which we plan to address through environment reconstruction and interaction-aware control in our future works.
