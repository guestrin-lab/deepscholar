\begin{abstract}
% Large Language Models have shown notable potential in program comprehension and vulnerability detection, yet their role in precise vulnerability localization for large-scale open-source projects remains underexplored. Existing approaches often analyze code in isolation, encounter difficulties with long contexts, and generate predictions at the file or function level. These constraints lead to elevated false positives and limited diagnostic guidance, leaving the precise identification of vulnerable code an open and persistent challenge. We propose a project-level, multi-round T2L-Agent (Trace-to-Line Agent) framework that plans tasks automatically and incrementally narrows scope from modules to exact lines. The T2L-Agent decomposes complex project snapshots into manageable chunks and designs step-wise diagnostic strategies, and integrates multi-round feedback and automatic debugging diagnostics—including crash points, stack traces, and coverage deltas—to progressively refine localization from modules down to specific code blocks or lines. This iterative process enables adaptive reasoning beyond single-pass predictions. We evaluate our framework on the ARVO dataset, which provides reproducible builds and proof-of-concept inputs, allowing rigorous end-to-end experiments with fine-grained difficulty stratification. Results on 100 projects from ARVO demonstrate its precise identification ability of faulty code blocks. Collectively, these findings advance vulnerability detection toward deployable, precision diagnostic tooling for open-source software workflows.
Large language models show promise for vulnerability discovery, yet prevailing methods inspect code in isolation, struggle with long contexts, and focus on coarse function or file level detections—offering limited actionable guidance to engineers who need precise line-level localization and targeted patches in real-world software development. We present T2L-Agent (Trace-to-Line Agent), a project-level, end-to-end framework that plans its own analysis and progressively narrows scope from modules to exact vulnerable lines. T2L-Agent couples multi-round feedback with an Agentic Trace Analyzer (ATA) that fuses runtime evidence—crash points, stack traces, and coverage deltas with AST-based code chunking, enabling iterative refinement beyond single pass predictions and translating symptoms into actionable, line-level diagnoses. To benchmark line-level vulnerability discovery, we introduce T2L-ARVO, a diverse, expert-verified 50-case benchmark spanning five crash families and real-world projects. \texttt{T2L-ARVO} is specifically designed to support both coarse-grained detection and fine-grained localization, enabling rigorous evaluation of systems that aim to move beyond file-level predictions. On \texttt{T2L-ARVO}, T2L-Agent achieves up to 58.0\% detection and 54.8\% line-level localization, substantially outperforming baselines. Together, the framework and benchmark push LLM-based vulnerability detection from coarse identification toward deployable, robust, precision diagnostics that reduce noise and accelerate patching in open-source software workflows. Our framework and benchmark are publicly available as open source at \url{https://github.com/haoranxi/T2LAgent}.
\end{abstract}