\vspace{-2mm}
\section{Limitation and Future Work}
\vspace{-2mm}

Our work has three key limitations. First, the \texttt{T2L-ARVO} benchmark includes only 50 manually verified cases. While it offers broad vulnerability coverage and balanced categories, the limited sample size constrains evaluation due to the human verification efforts. Also, ARVO's dataset structure is designed for human developers, which needs more fine-grained metadata that could benefit LLM-based localization. Second, although \texttt{T2L-Agent} improves localization accuracy from 0\% to 54.8\% through three innovations, cost efficiency remains a concern. The agent operates effectively under a \$1.0 budget via task-aware planning and early stopping, but large-scale deployment across thousands of vulnerabilities would demand significant optimization. Third, higher model thinking budgets fail to boost localization performance, indicating that increased compute alone is insufficient. This points to a need for smarter ways to exploit model reasoning. Future work should explore more efficient architectures. Such as model cascading to coordinate cheaper and stronger models, and specialized multi-agent systems where roles are tailored to tools like our Agentic Trace Analyzer. These strategies may retain quality while scaling to production workloads.

\vspace{-2mm}
\section{Conclusion}
\vspace{-2mm}
T2L addresses a key gap between LLM-based vulnerability localization and real-world practice. We contribute three advances that move from coarse identification to precise diagnostics. First, we propose a new formulation for LLM-based vulnerability detection: chunk-wise detection and line-level localization, enabling structured and fine-grained evaluation. Second, \texttt{T2L-ARVO} introduces a benchmark for agentic line-level localization, with 50 expert-verified cases across diverse vulnerabilities. Third, \texttt{T2L-Agent} improves performance via our Agentic Trace Analyzer, which fuses runtime and static signals, as well as Divergence Tracing and Detection Refinement in a feedback-driven workflow. \texttt{T2L-Agent} achieves 44–58\% detection and 38–54.8\% localization, marking a step toward deployable systems for real-world code security.