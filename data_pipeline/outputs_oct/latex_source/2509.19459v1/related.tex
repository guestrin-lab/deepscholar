\section{Related Work}

There is work on checking/testing PM programs to find bugs. In
particular, XFDetector~\cite{xfdetector} uses a finite state machine
to track the consistency and persistency of persistent
data. PMTest~\cite{pmtest} lets developers annotate a program with
checking rules to infer the persistency status of writes and ordering
constraints between writes. Pmemcheck~\cite{pmemcheck} checks how many
stores were not made persistent and detects memory overwrites using
binary rewriting. Yat~\cite{yat} model checks
PM programs. 
Agamotto~\cite{agamotto} finds bugs in PM programs by
using symbolic execution. An algorithm by Huang et al.~\cite{pminvariant} infers invariants from PM programs that are then used to check for bugs.
Although these tools are able to find many bugs, none of these tools
can assure the absence of flush/fence bugs like \tool can.  POG~\cite{pog} and
Pierogi~\cite{pierogi} provide logics that can be used to manually
reason about program behaviors.

A line of work~\cite{nv-htm,crafty-pldi20,persistent-htm-giles-2017,dudetm} uses (software or hardware) transactions to provide
(failure and thread) atomicity. NVL-C provides language and compiler
support for the use of transactions to access non-volatile
memory~\cite{nvlc}.  While NVL-C can provide crash consistency, it
does so by incurring the overheads of using transactions to provide
crash consistency.  Another line of
work~\cite{atlas-follow-up,atlas,nvthreads,justdo-logging,ido}
advocates use of locks or synchronization-free
regions~\cite{persistency-sfr}.  Memento~\cite{memento} provides
detectable checkpointing---it extends standard checkpoint with support
to allow the system to be able to detect the status of in flight
operations when the crash occurred.  These approaches typically incur
large overheads to support the necessary logging.

StaticPersist~\cite{staticpersist} is a static analysis to
determine which objects must be allocated in PM.  The
idea is to use annotations to declare a set of durable roots, and the
analysis determines which objects are reachable from the 
roots. AutoPersist~\cite{autopersist} is a Java extension to support
NVM.  Developers specify durable roots and when an
object becomes reachable from a durable root, AutoPersist moves it to
PM.  Both
StaticPersist and AutoPersist provide a higher-level API for
programming PM, while \tool targets a
lower-level model by automatically inserting flush and
fence operations.

Hippocrates~\cite{hippocrates-asplos21} and
PMBugAssist~\cite{pmbugassist} insert flushes
and fences to repair PM bugs.  Hippocrates chooses the placement of
flushes and fences using a reduction procedure, whereas PMBugAssist
uses a SMT solver. They both focus on
repairing specific PM bugs given as program input and rely on other PM
bug detection tools to produce bug traces, which is different from our
task of exhaustively detecting potential PM bugs and fixing them at
the same time.

Our work is related to a line of work on static fence
insertion for concurrent programs to ensure sequential consistency~\cite{musketeer2017, lee2000relaxed, fang2003automatic}. The static approach to fence insertion has so far not been applied to PM programs, a
gap which we bridge. In terms of techniques, previous work such as musketeer~\cite{musketeer2017} and pensieve~\cite{fang2003automatic} focused on variants of delay set analysis~\cite{delayset} combined with flow-insensitive escape analysis to minimise fence insertions, whereas we make use of a flow-sensitive escape analysis to avoid redundant fence insertions. 

FliT~\cite{flit} presents a technique that uses counters to
eliminate flush operations on atomic loads, which we applied in \tool
to further reduce the overhead it introduces.
