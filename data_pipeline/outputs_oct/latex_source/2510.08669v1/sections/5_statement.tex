\section*{Ethics Statement}
\vspace{-3mm}
%本工作提出FreqCa,一种通过频率感知缓存加速扩散模型推理的技术方法。我们的研究纯粹专注于计算效率改进,不会引入超出底层扩散模型固有风险之外的新伦理风险。我们在实验中仅使用公开可用的模型和数据集,我们的加速技术是模型无关且内容中性的。虽然我们的方法减少了推理时间和计算成本,可能使生成式AI更易获得,但我们承认这种易获得性既适用于有益的使用场景,也适用于潜在有害的使用场景。我们鼓励根据AI生成内容的现有伦理准则负责任地部署加速扩散模型,包括适当披露合成媒体和考虑潜在的社会影响。
This work presents FreqCa, a technical method for accelerating diffusion model inference through frequency-aware caching. Our research focuses purely on computational efficiency improvements and does not introduce new ethical risks beyond those inherent to the underlying diffusion models. We use only publicly available models and datasets in our experiments, and our acceleration technique is model-agnostic and content-neutral. While our method reduces inference time and computational costs, which could potentially make generative AI more accessible, we acknowledge that this accessibility applies to both beneficial and potentially harmful use cases. We encourage responsible deployment of accelerated diffusion models in accordance with existing ethical guidelines for AI-generated content, including proper disclosure of synthetic media and consideration of potential societal impacts.

\section*{Reproducibility Statement}
\vspace{-3mm}
%我们致力于确保FreqCa框架的完全可重现性。为此,第3节提供了我们核心算法组件的完整数学公式:频率分解(FFT/DCT)、累积残差特征(CRF)缓存和二阶Hermite预测。所有实验配置在第4.1节中详细说明,包括评估的模型(如FLUX.1-dev、Qwen-Image)、使用的数据集(DrawBench和GEdit)以及完整的评价指标集合(如ImageReward、CLIP Score、PSNR)。第4节和附录展示了我们详细的消融研究、分解方法和预测策略的超参数选择,以及计算复杂度分析。补充材料中提供了匿名源代码仓库,包含完整的推理和训练脚本、带随机种子的配置文件以及数据预处理流水线。该仓库将在论文接收后公开发布。
We are committed to ensuring the full reproducibility of our FreqCa framework. To this end, Section 3 provides the complete mathematical formulations for our core algorithmic components: frequency decomposition (FFT/DCT), Cumulative Residual Feature (CRF) caching, and second-order Hermite prediction. All experimental configurations are detailed in Section 4.1, specifying the models evaluated (e.g., FLUX.1-dev, Qwen-Image), the datasets used (DrawBench and GEdit), and the full set of evaluation metrics (e.g., ImageReward, CLIP Score, PSNR). Section 4 and the Appendix present our detailed ablation studies, hyperparameter choices for decomposition methods and prediction strategies, and the computational complexity analysis. An anonymous source code repository is provided in the supplementary materials, containing complete inference and training scripts, configuration files with random seeds, and data preprocessing pipelines. The repository will be made publicly available upon acceptance.

