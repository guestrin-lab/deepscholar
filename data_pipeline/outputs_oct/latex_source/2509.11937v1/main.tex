\documentclass{article}

% Recommended, but optional, packages for figures and better typesetting:
\usepackage{microtype}
\usepackage{graphicx} % Required for inserting images
\usepackage{subfigure}
\usepackage{booktabs} % for professional tables
\usepackage{amsmath}

\usepackage{tablefootnote}

\usepackage{float}

% hyperref makes hyperlinks in the resulting PDF.
% If your build breaks (sometimes temporarily if a hyperlink spans a page)
% please comment out the following usepackage line and replace
% \usepackage{icml2025} with \usepackage[nohyperref]{icml2025} above.
\usepackage{hyperref}
\usepackage{adjustbox}
\usepackage[most]{tcolorbox} % nice colored box
\usepackage{pgfplots}
\pgfplotsset{compat=1.18}
\usepackage{caption}
\usepackage{listings}
\usepackage[table]{xcolor}

\usepackage{pifont}
\usepackage[table]{xcolor}
\definecolor{TickGreen}{HTML}{DFF0D8}  % light green
\definecolor{LightGrey}{HTML}{F2F2F2}  % light grey
\newcommand{\cmark}{\ding{51}}       % tick mark
\newcommand{\greencell}{\cellcolor{TickGreen}{\cmark}}
\newcommand{\greycell}{\cellcolor{LightGrey}{\phantom{\cmark}}}
% Use the following line for the initial blind version submitted for review:
%\usepackage{icml2025}

% If accepted, instead use the following line for the camera-ready submission:
\usepackage[accepted]{icml2025}

% if you use cleveref..
\usepackage[capitalize,noabbrev]{cleveref}

% Todonotes is useful during development; simply uncomment the next line
%    and comment out the line below the next line to turn off comments
%\usepackage[disable,textsize=tiny]{todonotes}
\usepackage[textsize=tiny]{todonotes}
\newcommand{\mmore}{\textls[100]{\texttt{MMORE}}}
\definecolor{green1}{HTML}{5ACEA0}

% The \icmltitle you define below is probably too long as a header.
% Therefore, a short form for the running title is supplied here:
\icmltitlerunning{MMORE: Massive Multimodal Open RAG \& Extraction
}

\begin{document}

\twocolumn[
\icmltitle{MMORE: Massive Multimodal Open RAG \& Extraction
}

% It is OKAY to include author information, even for blind
% submissions: the style file will automatically remove it for you
% unless you've provided the [accepted] option to the icml2025
% package.

% List of affiliations: The first argument should be a (short)
% identifier you will use later to specify author affiliations
% Academic affiliations should list Department, University, City, Region, Country
% Industry affiliations should list Company, City, Region, Country

% You can specify symbols, otherwise they are numbered in order.
% Ideally, you should not use this facility. Affiliations will be numbered
% in order of appearance and this is the preferred way.
\icmlsetsymbol{equal}{*}

\begin{icmlauthorlist}
\icmlauthor{Alexandre Sallinen}{epfl}
\icmlauthor{Stefan Krsteski}{epfl}
\icmlauthor{Paul Teiletche}{epfl}
\icmlauthor{Marc-Antoine Allard}{epfl}
\icmlauthor{Baptiste Lecoeur}{epfl}
\icmlauthor{Michael Zhang}{epfl}
\icmlauthor{David Kalajdzic}{epfl}
\icmlauthor{Matthias Meyer}{eth}
\icmlauthor{Fabrice Nemo}{epfl}
% \icmlauthor{Charlotte Meyer}{epfl}
% \icmlauthor{Matthew Meyer}{epfl}
% \icmlauthor{Lea Grieder}{efpl}
% \icmlauthor{Laetitia Wilhelm}{eth}
\icmlauthor{Mary-Anne Hartley}{epfl,harv}
\end{icmlauthorlist}

\icmlaffiliation{epfl}{École Polytechnique Fédérale de Lausanne (EPFL), Switzerland}
\icmlaffiliation{eth}{ETH Zürich, Switzerland}
\icmlaffiliation{harv}{T.H. Chan School of Public Health, Harvard University, USA}

% You need this line to close the twocolumn block and make affiliations appear:

\centering
\begin{tabular}{@{}c@{\hspace{0.5cm}}c@{}}
$^1$ EPFL, Switzerland & $^2$ ETHZ, Switzerland \\
\multicolumn{2}{c}{$^3$ Harvard University, USA} \\
\end{tabular}
\vskip 0.3in
]
\begin{figure*}[t]
    \centering    \includegraphics[width=0.75\linewidth]{pipeline.png}
    \vspace{-3mm}
    \caption{The end-to-end pipeline from file-type–specific processing to retrieval-augmented generation (RAG).}
    \label{fig:mmore_system_overview}
\end{figure*}

\begin{abstract}
We introduce \textbf{\mmore{}}, an open-source pipeline for \textbf{\texttt{M}}assive \textbf{\texttt{M}}ultimodal \textbf{\texttt{O}}pen \textbf{\texttt{R}}etrieval-Augmented Generation and \textbf{\texttt{E}}xtraction, designed to ingest, transform, and retrieve knowledge from heterogeneous document formats at scale. \mmore{} supports more than fifteen file types, including text, tables, images, emails, audio, and video, and processes them into a unified format to enable downstream applications for LLMs. The architecture offers modular, distributed processing, enabling scalable parallelization across CPUs and GPUs. On processing benchmarks, \mmore{} demonstrates a 3.8-fold speedup over single-node baselines and 40\% higher accuracy than Docling on scanned PDFs. The pipeline integrates hybrid dense-sparse retrieval and supports both interactive APIs and batch RAG endpoints. Evaluated on PubMedQA, \mmore{}-augmented medical LLMs improve biomedical QA accuracy with increasing retrieval depth. \mmore{} provides a robust, extensible foundation for deploying task-agnostic RAG systems on diverse, real-world multimodal data. The codebase is available at \href{https://github.com/swiss-ai/mmore}{https://github.com/swiss-ai/mmore}.
    
\end{abstract}

\section{Introduction}\label{sec:introduction}
Mixture-of-Experts (MoE) is an architectural paradigm that adaptively combines predictions from multiple neural modules, known as "experts," via a learned gating mechanism. This concept has evolved from ensemble-based MoEs, where experts, jointly trained with a gating function, are often full, independent models whose outputs are combined to improve overall performance and robustness \citep{jacobs1991adaptive}. More recently, MoE layers have been integrated within larger neural architectures, with experts operating in a latent domain. These "latent MoEs" offer significant scalability benefits, especially in large language models (LLMs) \citep{shazeer2017outrageously,fedus2022switch}.
MoE makes it possible to train massive but efficient LLMs, where each token activates only a fraction of the model’s parameters, enabling specialization, better performance, and lower computational cost compared to equally sized dense models.

Regardless of their specific implementation, conventional MoE systems typically produce point estimates, lacking a direct quantification of their uncertainty. In critical applications, this absence of uncertainty information hinders interpretability, making it difficult for users to gauge the reliability of a prediction and limits informed decision-making, as the system cannot express its confidence or identify ambiguous cases. Importantly, the learned gating mechanism, which dictates the relative contribution of each expert, does not take into account expert confidence, potentially leading to suboptimal routing decisions.

In this work, we propose Mixture-of-Gaussians with Uncertainty-based Gating (MoGU), a framework for uncertainty-aware MoE architectures, which provides explicit uncertainty quantification for both individual experts and the overall MoE model. Our approach fundamentally reimagines the expert's output: instead of a point estimate, we model each expert's prediction as a random variable drawn from a normal distribution. In this setup, each expert simultaneously predicts both the mean (the label estimate) and variance of the distribution, representing its predictive uncertainty. This shift enables a more nuanced understanding of expert behavior and the derivation of the overall model's uncertainty. Furthermore, we introduce a novel gating mechanism where the estimated uncertainty of each expert directly informs its relative contribution to the overall MoE prediction, bypassing the need for a separate gating function typically found in traditional MoE setups. This creates a self-aware MoE where more confident experts naturally exert greater influence.

We evaluate MoGU on time series forecasting as our primary regression task. This choice is motivated by the inherent uncertainty in real-world time series data and the wide variety of expert architectures applicable to forecasting tasks across numerous domains \citep{time_series_survey, wang2024deep}. Our evaluation spans various expert types, forecasting benchmarks and forecasting horizon sizes, allowing for a comprehensive assessment of our method's efficacy. MoGU is shown to consistently yield more accurate forecasts compared to input-based gating MoE architectures, while simultaneously, providing uncertainty estimates that are positively correlated with prediction error. These estimates are available at both the individual expert and overall model levels. By further distinguishing between aleatoric (data-related) and epistemic (model-related) uncertainty, MoGU offers valuable insights into the source of a model's uncertainty. We also conducted a detailed ablation study to validate our key design choices.

In summary, our contributions are as follows: 
\begin{itemize}
\item \textbf{MoGU: A Novel Framework for Uncertainty-Aware MoE Architectures}: We introduce a novel framework that directly quantifies uncertainty for both individual experts and the overall model, moving beyond conventional point estimates. A key innovation is a routing mechanism that uses each expert’s estimated predictive uncertainty to dynamically determine its contribution to the final MoE output, replacing traditional input-based gating mechanisms.
\item \textbf{MoGU Improves Time Series Forecasting}: Our method effectively reduces forecasting error across various benchmarks, horizon lengths, and expert architectures.
\item \textbf{MoGU Provides Meaningful Uncertainty Estimates for Time Series Forecasting}: MoGU generates uncertainty estimates at the expert-level and overall. These estimates are positively correlated with prediction error, providing valuable insight into the model's confidence and the sources of its uncertainty.
\end{itemize}

By embedding uncertainty estimation into prediction and gating, MoGU moves beyond input-based gating  MoEs toward architectures that are more accurate, transparent, and reliable.


\section{Related Work}
% % mention LLMWhisperer (this is not open-source and its paid, so our contribution here is easy)
% % doctr - they only do document parsing 
% % Surya - this one is the most relevant because we use it in our pipeline for PDFs intesively. The difference is we only reuse surya for PDFs and we offer native parallelization on multi-node multi-gpu systems in contrast

Large-scale transformation of unstructured documents into structured, machine‑readable format has attracted substantial attention. We group prior work into two strands: \textbf{(i)} document ingestion and parsing pipelines, and \textbf{(ii)} RAG frameworks. To our knowledge, neither line of work simultaneously offers the modality coverage and end‑to‑end throughput required for industrial‑ and small‑scale multimodal assistants that we target with \mmore{}.\\
\textbf{Document Ingestion Pipelines.} GPU‑accelerated microservice suites such as \textit{NV‑Ingest}~\cite{nvingest} convert PDFs and office documents into page‑level JSON enriched with text blocks, tables, and graphics, and can optionally export embeddings for downstream indexing. \textit{Docling}~\cite{auer2024docling} extends the modality set to spreadsheets, and other common formats, but executes primarily on a single node and therefore exhibits limited throughput in production settings. Classical OCR tools like \textit{doctr}~\cite{doctr2021} handle text detection and recognition but rely on external systems for layout, embeddings, and indexing. \textit{Surya}~\cite{paruchuri2025surya} adds multilingual OCR and layout analysis but lacks built-in multi-GPU or cluster parallelism. Commercial services such as \textit{LLMWhisperer}~\cite{llmwhisperer2025} offer similar functionality behind a paywall, which restricts reproducibility and hinders open experimentation. In contrast, \mmore{} combines extraction, transformation, embedding, and indexing into a single open‑source pipeline that natively parallelizes across multi‑node, multi‑GPU deployments. Moreover, \mmore{} uniquely handles audiovisual assets, enabling unified RAG over text, images, and time‑based media. \\
\textbf{RAG Frameworks.} Open‑source libraries such as \textit{LangChain}~\cite{Chase_LangChain_2022} and \textit{LlamaIndex}~\cite{Liu_LlamaIndex_2022} provide high-level abstractions for chunking, embedding, retrieval, and prompting. However, they rely on external loaders for modality‑specific parsing and give no guidance on efficient high-throughput ingestion. 
Several recent pipelines, such as \textit{Unstructured.io}~\cite{unstructured2025} and Haystack~\cite{haystack2019} for document parsing, or \textit{M3IT}~\cite{li2023m3it} and \textit{OpenFlamingo}~\cite{awadalla2023openflamingo} for multimodal model alignment, address specific components of this pipeline. Yet none provide an integrated, open-source framework that supports ingestion, transformation, and retrieval across heterogeneous, real-world file types at scale. 

\mmore{} combines a scalable ingestion layer with a task-agnostic retrieval API, unifying document processing and RAG tools to enable multimodal assistants from raw enterprise data in one library.
%%% vérifier les sourches encore


\section{Architecture}

% MMORE provides an end-to-end RAG platform: you can use the same tool for processing your database of documents, making an index out of it, and query the LLM of your choice with the relevant documents.
\mmore{} provides an end-to-end platform, enabling users to process large document collections, build retrieval indices, and query LLMs with relevant multimodal content, all within a unified framework, as illustrated in Figure~\ref{fig:mmore_system_overview}.

\subsection{Processing}

At the core of \mmore{} lies a modular, scalable processing pipeline, designed for efficient, multimodal data extraction. Importantly, \mmore{} reuses open-source extraction tools such as \textit{Surya}~\cite{paruchuri2025surya} for PDF parsing, \textit{Whisper}~\cite{radford2023robust} for audio transcription, and standard Python libraries for office file formats, allowing us to focus on scalable orchestration and integration. A complete list of supported extractors is provided in Appendix~\ref{app:ingestion}. The design prioritizes three main strengths: \textbf{(i)} multimodal document processing, \textbf{(ii)} extensibility to new file types, and \textbf{(iii)} high-throughput distributed execution.\\
\textbf{Multimodal Data Extraction.}
\label{processing:multimodal}
The processor module extracts heterogeneous content from documents and standardizes it into a unified JSON-based format, referred to as the \textit{MultimodalSample} (see Appendix \ref{app:data_format}). Each sample consists of plain text interleaved with modality placeholders (e.g. images) and a list of the extracted modalities, preserving their type and location. Embedded media are extracted and saved to disk, with placeholder tokens (e.g., \texttt{<attachment>}) inserted at the corresponding positions within the text. This design supports downstream tasks that require text with tightly linked visual elements, such as multimodal pre-training or RAG.\\
\textbf{Extensibility.}
\label{processing:extensibility}
To facilitate extensibility, we designed a common processor interface that abstracts file-specific handling into modular components. Adding support for a new file type requires only implementing a lightweight subclass, promoting long-term maintainability and community-driven contributions. Each processor needs to define a class that takes a file path as input and outputs a \textit{MultimodalSample}, leveraging the standardized output format across the system. To date, \mmore{} supports more than 15 file types, including, but not limited to, PDFs, DOCX, PPTX, spreadsheets, media files, emails, and HTML pages.\\
\textbf{Distributed Processing.}
\label{processing:distribution}\mmore{} natively supports both intra-node and inter-node parallelization, exploiting all available CPU and GPU resources without requiring manual configuration from the user. The system is built on top of \textit{Dask}~\cite{dask}, enabling automatic workload balancing, fault tolerance, and seamless scaling across deployment settings, from standalone machines to large multi-node clusters. This design scales across use cases, from individual researchers to large organizations. To further support both ends of the spectrum, \mmore{} offers two processing modes: a fast mode for speed and a default mode for accuracy, allowing users to balance performance and fidelity as needed.

% at the end 
% To accommodate different computational needs, we provide both \textit{fast} and \textit{slow} processing modes, offering a trade-off between higher accuracy with slower processing and lower accuracy with faster processing. 

\subsection{RAG}

The RAG pipeline is composed of three independent components: \textbf{(i)} post-processing, \textbf{(ii)} indexing and retrieval, and \textbf{(iii)} an integrated RAG service. Each part is modular and can be run independently.

% old version Those parts can be run independently, allowing users to perform only the operation they want. They also each have a configuration file associated with them, allowing the user to easily fine-grain the settings depending on what the user wants.

\textbf{Post-processing.}
\label{RAG:postproc}
% old version Post-processing is meant to perform data processing on the extracted texts from documents. The infrastructure for post-processing is customizable; one can easily implement a new post-processor, and then the configuration for post-processing specifies which post-processors should be used and with which parameters. The post-processor is meant to produce a new JSONL file with the post-processed content.
This stage filters the extracted text to improve quality for downstream tasks. \mmore{} exploits the existing \textit{datatrove} \cite{penedo2024datatrove}, a high-throughput filtering library, and includes native support for several post-processing components, including Named Entity Recognition, Chunking, and Tagging. \\
% Additional post-processors can be integrated with minimal effort, and the entire pipeline is configured using lightweight YAML files. This stage produces a cleaned and optionally enriched JSONL dataset, ready for downstream indexing or training.
% old versio no.2 The infrastructure is modular and extensible: MMORE natively supports the following post-processors: Named Entity Recognition, Chunker, Tagger\footnote{The Tagger enhances the metadata of indexed documents based on their content}, and Filter\footnote{The Filter filters out documents that are deemed useless, based on predefined criteria}. New post-processors can easily be implemented, and pipelines can be configured through lightweight YAML files. The post-processing stage produces a new JSONL file containing cleaned and optionally enhanced document samples.
\textbf{Indexing and Retrieval.}
\label{RAG:indexing_retrieval}
% old version Indexing is a crucial component of the RAG pipeline, as the retrieval system has to deal with the data representation made by the indexer to retrieve the most relevant documents. We also made the indexing modular: MMORE currently supports two indexers, a regular RAG indexer based on dense and sparse embeddings computed for each document, and a GraphRAG. Users may define a new indexer.
Indexing is crucial to RAG performance, as retrieval relies on how documents are represented. \mmore{} uses a hybrid indexing strategy, storing both sparse and dense embeddings for each document. Sparse representations support lexical matching and improve interpretability, while dense embeddings enable semantic search using neural similarity. This duality allows users to choose or combine retrieval embeddings depending on their downstream task. The retriever is accessible via our integrated RAG system or as a standalone API.\\
\textbf{Integrated RAG system.}
\label{RAG:ragsystem}
The RAG system supports both API-based querying and offline batch processing. In batch mode, users provide a JSONL file containing retrieval queries; the system processes each entry and saves the results to a new JSONL file. Both modes allow customization of the model, prompt template, index source, and other parameters via configuration files or API options.

% The RAG system supports both API-based queries and as a batch-mode service. When using the latter, the user provides a JSONL file with requests, and the RAG pipeline processes each input, saving the responses in a new JSONL file. Users can customize the LLM, prompt template, index source, and other parameters in the configuration file (or through the API).
% \input{sections/pipeline}
\section{Evaluation Setup}
We evaluate \mmore{}'s processing and RAG modules independently. Below, we detail our methodology for assessing efficiency, accuracy, and scalability.

\subsection{Processing}
The processing module is evaluated along two axes: efficiency and accuracy, versus \textit{Docling}~\cite{auer2024docling} as a baseline due to its popularity and ease of use.

\textbf{Efficiency.} We benchmark processing speed using a single A100 80GB. For scalability analysis, we use an 18-page paper and synthetically generate longer documents by duplicating its content to reach 36, 54, 90, to 720 pages. This setup allows us to test throughput for both single-device and distributed processing. The distributed experiments are conducted on a Kubernetes cluster with 1 vs 4 nodes (1 A100 per node) to evaluate parallelization efficiency. To highlight \mmore{}'s strength in handling heterogeneous data, we also evaluate its performance across a diverse set of 19 files, spanning 9 unique file types.

\textbf{Accuracy.} To assess text extraction quality, we create a benchmark using public-domain books from Project Gutenberg~\cite{projectgutenberg} by pairing PDF inputs with their corresponding plain-text ground truths. We select two contrasting cases: "The Blue Castle" (a clean, digital-friendly PDF) and "The Great Gatsby" (a scanned, image-based file). Each document is truncated to 50k characters to ensure computational feasibility, 
particularly for metrics like Levenshtein distance. We report standard metrics: BLEU~\cite{bleuscore} for n-gram overlap,
ROUGE-L~\cite{lin2004rouge}, and character error rate (CER)~\cite{levenshtein}. Metric formulations are provided in the Appendix~\ref{app:metrics}

\subsection{RAG}
To evaluate our RAG pipeline, we focus on the PubMedQA benchmark~\cite{jin2019pubmedqa}, a biomedical question-answering task. We construct a retrieval corpus by indexing all PubMed abstracts and conclusions into a dense vector database using \mmore{}. At inference time, the top-\(k\) most relevant documents are retrieved using a similarity search and prepended to the original question as context for the language model. We experiment with both Meditron3-8B and Meditron3-70B~\cite{sallinen2025llama}, evaluating how different values of \(k\) affect downstream accuracy. This setup isolates the effect of retrieval depth on performance within a consistent biomedical knowledge source.
% Please add the following required packages to your document preamble:
% \usepackage{booktabs}
% \usepackage{multirow}
\begin{table}[]
\centering
\setlength{\abovecaptionskip}{0.1cm}
\setlength{\belowcaptionskip}{0.1cm}
\caption{Performance of sampling strategies on various models.
``Baseline'' is original sampling; ``+Prob./Ours'' refers to Sec.~\ref{sec_entropy_temperature}; ``+Masking/+Scale-wise'' are paradigm-specific from Sec~\ref{sec_adaption_more_ar}.
}
\setlength{\tabcolsep}{4.5mm}{
\resizebox{0.9\columnwidth}{!}{%
\begin{tabular}{@{}ll|cccc@{}}
\toprule
Method                       & Config. & FID$\downarrow$   & CLIP-Score$\uparrow$ & DPG$\uparrow$ & HPSv2.1$\uparrow$ \\ \midrule
SDv2.1~\cite{rombach2022stablediffusion} & - & 22.87 & 26.31 & 68.09 & 26.38 \\
PixArt-$\alpha$~\cite{chen2023pixart_alpha} & - & 33.23 & 25.70 & 71.52 & 30.04 \\
SDXL~\cite{podell2023sdxl} & - & 23.20 & 26.46 & 74.21 & 28.54 \\
SDv3-medium~\cite{esser2024sdv3} & - & 29.82 & 26.24 & 85.85 & 30.22 \\ \midrule
\multirow{2}{*}{LlamaGen~\cite{sun2024llamagen}}    & Baseline    & 21.94 & 25.95 & 43.51 & 21.24  \\
                             & Ours        & \textbf{20.36} &  \textbf{25.96}  & \textbf{48.63} & \textbf{21.39}            \\ \midrule
\multirow{2}{*}{Lumina-mGPT~\cite{liu2024lumina_mgpt}} & Baseline    & 29.15 & 26.04     & 79.68 & \textbf{28.92}
\\
                             & Ours        & \textbf{27.44} & \textbf{26.25}      & \textbf{79.77} & 28.87  \\ \midrule
\multirow{3}{*}{Meissonic~\cite{bai2024meissonic}}   & Baseline    & 53.61 & 25.27      & 63.83 & 29.33   \\
                             & +Prob.      & \textbf{48.37} & 25.49      & 66.19 & 29.94   \\
                             & +Masking.   & 48.43 & \textbf{25.54}      & \textbf{67.08} & \textbf{30.04}   \\ \midrule
\multirow{3}{*}{STAR~\cite{ma2024star}}        & Baseline    & 35.05 & 25.43      & 70.25 & 28.79   \\
                             & +Prob.      & 32.75 & 25.56      & 70.83 & 28.93   \\
                             & +Scale-wise & \textbf{32.37} & \textbf{25.61}      & \textbf{70.86} & \textbf{29.06}   \\ \bottomrule
\end{tabular}
}}
\label{tab_results}
\vspace{-3mm}
\end{table}
\section{Conclusion}

In this work, we presented a full-stack investigation of LLM unlearning, encompassing methodology, evaluation, and robustness. We established a principled taxonomy that organizes twelve representative unlearning methods into three families: {\MDiv}, {\MRep}, and {\MRej}, providing a systematic lens to understand their underlying mechanisms. Our analysis revealed that conventional multiple-choice questioning (MCQ) evaluations of unlearning effectiveness (UE) and utility retention (UT) offer an incomplete picture, and we introduced open question answering (Open-QA) as a complementary paradigm to better capture generative behaviors and expose the strengths and limitations of different methods. Furthermore, we provide a comprehensive robustness assessment across model-level and input-level attacks, revealing nuanced relationships among in-domain relearning, out-of-domain fine-tuning, quantization, and jailbreak attacks. These findings clarify the trade-offs of current unlearning algorithms and guide the design of future methods that are both effective and robust. The use of LLM, limitation and broader impact are further discussed in \textbf{Appendix\,\ref{appx:llm_usage}}, \textbf{Appendix\,\ref{appx:limit}} and \textbf{Appendix\,\ref{appx:impact}}.



\bibliography{mmore}
\bibliographystyle{icml2025}

\onecolumn
\appendix
\clearpage
\section{Appendix}
\subsection{Document Ingestion}
\label{app:ingestion}
To better situate \mmore{} within the ecosystem of document ingestion systems, Table~\ref{tab:ingestion_comparison_expanded} presents a fine-grained comparison with two representative alternatives: \textit{Docling} and \textit{NV-Ingest} (part of NeMo Retriever). We evaluate them across modality support, indexing capabilities, and RAG integration. Green cells indicate native support, while grey cells denote the absence of the corresponding capability.

\begin{table}[h]
    \centering
    \small
    \resizebox{0.5\textwidth}{!}{%
    \begin{tabular}{lccc}
        \toprule
        \textbf{Feature} & \textbf{Docling} & \textbf{NV-Ingest\tablefootnote{\href{https://docs.nvidia.com/nemo/retriever/extraction/overview/}{NeMo Retriever Documentation}}} & \textbf{MMORE} \\
        \midrule
        % ---------------------- SUPPORTED MODALITIES ----------------------
        \rowcolor{gray!20}
        \multicolumn{4}{l}{\textit{Supported Modalities}} \\
        \midrule
        PDF & \greencell & \greencell & \greencell \\
        DOCX & \greencell & \greencell & \greencell \\
        PPTX & \greencell & \greencell & \greencell \\
        XLSX / spreadsheets & \greencell & \greycell & \greencell \\
        TXT & \greencell & \greencell & \greencell \\
        HTML & \greencell & \greycell & \greencell \\
        Markdown & \greencell & \greycell & \greencell \\
        CSV & \greencell & \greycell & \greencell \\
        Images (PNG/JPEG/SVG/TIFF/BMP) & \greencell & \greencell & \greencell \\
        Audio & \greycell & \greycell & \greencell \\
        Video & \greycell & \greycell & \greencell \\
        EML & \greycell & \greycell & \greencell \\
        \midrule
        % ---------------------- INDEXING & EMBEDDING ----------------------
        \rowcolor{gray!20}
        \multicolumn{4}{l}{\textit{Indexing \& Embedding}} \\
        \midrule
        Native engine included & \greycell & \greencell & \greencell \\
        LangChain / LlamaIndex connector & \greencell & \greencell & \greencell \\
        \midrule
        % ---------------------- RAG INTEGRATION ----------------------
        \rowcolor{gray!20}
        \multicolumn{4}{l}{\textit{RAG}} \\
        \midrule
        Built--in RAG pipeline & \greycell & \greycell & \greencell \\
        Plugin--based RAG & \greencell & \greencell & \greycell \\
        \midrule
        % ---------------------- LICENSE ----------------------
        Open--Source license & MIT & Apache~2.0 & Apache~2.0 \\
        \bottomrule
    \end{tabular}%
    }
    \caption{Fine-grained comparison of Docling, NV-Ingest, and MMORE document-ingestion pipelines. Green cells indicate native support; grey cells indicate absence of the capability.}
    \label{tab:ingestion_comparison_expanded}
\end{table}


\mmore{} supports a wide range of file formats through modular extractors. For each supported type, we define a \textit{default mode} prioritizing accuracy and a \textit{fast mode} optimized for speed. When no alternative tool is available, the fast mode is left unspecified (--). A complete list of tools used per file type is shown in Table \ref{tab:ingestion_tools}.

\begin{table*}[h]
\centering
\small
\resizebox{0.99\textwidth}{!}{%
\begin{tabular}{l l l}
\toprule
\textbf{File Type} & \textbf{Default Mode Tool(s)} & \textbf{Fast Mode Tool(s)} \\
\midrule
\rowcolor{gray!5}
DOCX & \texttt{python-docx} for text and image extraction & -- \\
MD & \texttt{markdown} for text, \texttt{markdownify} for HTML conversion & -- \\
\rowcolor{gray!5}
PPTX & \texttt{python-pptx} for text and image extraction & -- \\
XLSX & \texttt{openpyxl} for table and text extraction & -- \\
\rowcolor{gray!5}
TXT & Python built-in \texttt{open()} & -- \\
EML & Python built-in \texttt{email} module & -- \\
\rowcolor{gray!5}
Audio/Video (MP4, MP3, etc.) & \texttt{moviepy} for frames, \texttt{whisper-large-v3-turbo} for transcription & \texttt{whisper-tiny} \\
PDF & \texttt{marker-pdf} for OCR/structured data & \texttt{PyMuPDF} \\
\rowcolor{gray!5}
HTML & \texttt{BeautifulSoup} & -- \\
\bottomrule
\end{tabular}%
}
\caption{Overview of supported file types and extraction tools in \mmore{}. Full URLs are included in the project documentation.}
\label{tab:ingestion_tools}
\end{table*}


% \begin{table*}[t!]
% \centering
% \small
% \resizebox{\textwidth}{!}{%
% \begin{tabular}{lccc}
% \toprule
% \textbf{Feature} & \textbf{Docling} & \textbf{NV-Ingest\tablefootnote{\href{https://docs.nvidia.com/nemo/retriever/extraction/overview/}{NeMo Retriever Documentation}}} & \textbf{MMORE} \\
% \midrule
% Model Stack & Pydantic v2 + native PDF backend + ONNX/PyTorch AI models\tablefootnote{\label{fn:docling-github}See GitHub repo} 
%             & NIM microservices (NeMo Retriever modules)
%             & Modular Python processors + Milvus indexing + LangChain RAG \\
% \rowcolor{gray!10}
% Supported Modalities & PDF, DOCX, PPTX, XLSX, TXT, HTML, Markdown, CSV, images (PNG, JPEG, TIFF, BMP, SVG)\tablefootnote{\label{docling-github}GitHub: full format list}
%                      & PDF, DOCX, PPTX, TXT, images (JPEG, PNG, SVG, TIFF)
%                      & PDF, media (video/audio), spreadsheets, EML \\
% Throughput (best) & GPU: 2.08 pages/sec (L4); CPU: 1.35 pages/sec (M3 Max)\tablefootnote{\label{docling-technical-report}See technical report}
%                   & 12 pages/sec, 900 embeddings/sec (1× H100)
%                   & 3.89 pages/sec (4× A100) \\
% \rowcolor{gray!10}
% Indexing \& Embedding & Exports JSON/Markdown; connectors for LangChain/LlamaIndex; no native engine
%                       & Optional Milvus indexing via NeMo Retriever
%                       & Integrated Milvus + LangChain \\
% RAG Integration & Via LangChain/LlamaIndex plugins
%                 & LangChain via JSON output
%                 & Built-in LangChain interface \\
% \rowcolor{gray!10}
% Open-Source License & MIT
%                    & Apache 2.0
%                    & Apache 2.0 \\
% \bottomrule
% \end{tabular}%
% }
% \caption{Comparison of Docling, NV-Ingest, and MMORE Document Ingestion Pipelines.}
% \label{tab:ingestion_comparison}
% \end{table*}


\subsection{Multimodal Sample}

The format provides a standardized representation for processed documents, combining extracted text with references to non-text elements. As shown in the example, the "text" field contains the document's content with \texttt{<attachment>} placeholders (which are configurable) marking modality locations, while the modalities array contains all embedded objects with their types and storage paths.


\label{app:data_format}
\begin{tcolorbox}[
  enhanced,
  colback=black!3,          % Light gray background inside
  colframe=black!50,        % Medium gray border
  coltitle=white,           % White title text
  fonttitle=\bfseries,      % Bold title font
  title=Format Example:, % Box title
  colbacktitle=black!50,    % Dark gray title background
  boxrule=0.4mm,
  toptitle=1mm,
  bottomtitle=1mm
]

\begin{lstlisting}[
    basicstyle=\small\ttfamily,
    numbers=none,
    backgroundcolor=\color{black!3}
]
{
  "text": "A report containing a cool image <attachment> and a chart <attachment>...",
  "modalities": [
    {
      "type": "image",
      "value": "chart_url_2.png"
    },
    {
      "type": "image",
      "value": "chart_url_1.png"
    }
  ]
}
\end{lstlisting}

\noindent\rule{\linewidth}{0.4pt}

\small{\textit{The standardized format for document processing.}}


\end{tcolorbox}

\subsection{Processing Accuracy - Metrics}
\label{app:metrics}

To quantify extraction accuracy, we used a combination of machine translation, summarization and string similarity metrics. Their definitions are given below.

\textbf{BLEU Score (bilingual evaluation understudy)} \cite{bleuscore}:  
The BLEU score evaluates the overlap between the n-grams (sequences of words of length \(n\)) between the extracted text and the ground truth. It is defined as:

\begin{equation}  
\text{BLEU} = \text{BP} \cdot \exp \left( \sum_{n=1}^{N} w_n \log p_n \right)
\end{equation}


where \(p_n\) is the precision for n-grams of length \(n\), ranging from [1 to 4], \(w_n\) are the weights (uniform), and brevity penalty (\(\text{BP}\)), given by:

\begin{equation}  
\text{BP} = 
\begin{cases} 
1 & \text{if } c > r \\ 
\exp\left(1 - \frac{r}{c}\right) & \text{if } c \leq r
\end{cases}
\end{equation}

Here, \(c\) is the length of the candidate (extracted) text, and \(r\) is the length of the reference (ground truth). BLEU considers how much of the extracted text matches the reference text in terms of word sequences, while also penalizing outputs that are too short.

\textbf{ROUGE-L (recall-oriented understudy for gisting evaluation)} \cite{lin2004rouge}:  
ROUGE-L measures the quality of the extracted text using the longest common subsequence (LCS) between the extracted text and the ground truth. The LCS is the longest sequence of words appearing in the same order in both texts (though not necessarily consecutively). ROUGE-L is calculated as:

\begin{equation}  
\text{ROUGE-L} = F_\text{measure} = \frac{(1 + \beta^2) \cdot \text{Precision} \cdot \text{Recall}}{\beta^2 \cdot \text{Precision} + \text{Recall}}
\end{equation}

where \(\beta\) is a weighting factor (set to 1 for equal weighting), and:

\begin{equation}  
\begin{aligned}
\text{Precision} &= \frac{\text{LCS}}{\text{Length of Extracted Text}}, \\
\text{Recall} &= \frac{\text{LCS}}{\text{Length of Ground Truth}}.
\end{aligned}
\end{equation}

\textbf{Levenshtein distance - character error rate (CER)} \cite{levenshtein}:  
Given two strings, \( s_1 \) (extracted text) and \( s_2 \) (ground truth), the Levenshtein distance \( d(s_1, s_2) \) measures the minimum number of insertions, deletions, or substitutions required to transform \( s_1 \) into \( s_2 \). We normalize this distance over the length of the ground truth and is defined as:

\begin{equation}
\text{CER} = \frac{d(s_1, s_2)}{|s_1|}
\end{equation}
\end{document}
