\documentclass{article}

\usepackage{amsmath}
\usepackage{array}
\usepackage{multirow}
\usepackage{caption}


\usepackage{PRIMEarxiv}

\usepackage[utf8]{inputenc} % allow utf-8 input
\usepackage[T1]{fontenc}    % use 8-bit T1 fonts
\usepackage{hyperref}       % hyperlinks
\usepackage{url}            % simple URL typesetting
\usepackage{booktabs}       % professional-quality tables
\usepackage{amsfonts}       % blackboard math symbols
\usepackage{nicefrac}       % compact symbols for 1/2, etc.
\usepackage{microtype}      % microtypography
\usepackage{lipsum}
\usepackage{fancyhdr}       % header
\usepackage{graphicx}       % graphics
\graphicspath{{media/}}     % organize your images and other figures under media/ folder

%Header
\pagestyle{fancy}
\thispagestyle{empty}
\rhead{ \textit{ }} 

% % Update your Headers here
% \fancyhead[LO]{Running Title for Header}
% % \fancyhead[RE]{Firstauthor and Secondauthor} % Firstauthor et al. if more than 2 - must use \documentclass[twoside]{article}



  
%% Title
\title{Stock Prediction via a Dual Relation Fusion Network incorporating Static and Dynamic Relations
% %%%% Cite as
% %%%% Update your official citation here when published 
% \thanks{\textit{\underline{Citation}}: 
% \textbf{Authors. Title. Pages.... DOI:000000/11111.}} 
}

% \author{
%   Long Chen \\
%   Shaanxi Key Laboratory of Information Communication Network and Security \\
%   Xi’an University of Posts and Telecommunications \\
%   Xi’an, China\\
%   \texttt{chenlong@xupt.edu.cn} \\
%   %% examples of more authors
%    \And
%   Huixin Bai \\
%   Shaanxi Key Laboratory of Information Communication Network and Security \\
%   Xi’an University of Posts and Telecommunications \\
%   Xi’an, China\\
%   \texttt{05221170@stu.xupt.edu.cn} \\
%    \And
%    Mingxin Wang \\
%   Shaanxi Key Laboratory of Information Communication Network and Security \\
%   Xi’an University of Posts and Telecommunications \\
%   Xi’an, China\\
%   \texttt{mingxinwang@stu.xupt.edu.cn} \\
%    \And
%   Xiaohua Huang \\
%   School of Information Science and Technology \\
%   Northwest University\\
%   Xi’an, China \\
%   \texttt{huangxiaohua@stumail.nwu.edu.cn} \\
%    \And
%   Ying Liu \\
%   School of Information Science and Technology \\
%   Northwest University \\
%   Xi’an, China\\
%   \texttt{liuying@stumail.nwu.edu.cn} \\
%    \And
%   Jie Zhao \\
%   School of Computer Science and Technology \\
%   Xidian University \\
%   Xi’an, China\\
%   \texttt{jzhao1992@stu.xidian.edu.cn} \\
%    \And
%   Ziyu Guan \\
%   School of Computer Science and Technology \\
%   Xidian University \\
%   Xi’an, China\\
%   \texttt{zyguan@xidian.edu.cn} \\
%   %% \AND
%   %% Coauthor \\
%   %% Affiliation \\
%   %% Address \\
%   %% \texttt{email} \\
%   %% \And
%   %% Coauthor \\
%   %% Affiliation \\
%   %% Address \\
%   %% \texttt{email} \\
%   %% \And
%   %% Coauthor \\
%   %% Affiliation \\
%   %% Address \\
%   %% \texttt{email} \\
% }

% \usepackage{titling}
\usepackage{orcidlink}
\usepackage{footmisc}
% \setlength{\droptitle}{-3em}

% 简化版命令,只保留姓名和上标
\newcommand{\Author}[2]{% Name, Institution-number
  \textbf{#1}\textsuperscript{#2}%
}

\author{
  \Author{Long Chen}{1} \and
  \Author{Huixin Bai\textsuperscript{*}}{1} \and
  \Author{Mingxin Wang}{1} \and
  \Author{Xiaohua Huang}{2} \and
  \Author{Ying Liu}{2} \and
  \Author{Jie Zhao}{3} \and
  \Author{Ziyu Guan}{4} 
  \\[1em]
  \begin{minipage}{\textwidth}
  \centering
  \footnotesize
  \textsuperscript{1}Shaanxi Key Laboratory of Information Communication Network and Security, Xi'an University of Posts and Telecommunications, Xi'an, China\\
  \textsuperscript{2}School of Information Science and Technology, Northwest University, Xi'an, China\\
  \textsuperscript{3}School of Data Science and Artificial Intelligence, Chang'an University, Xi'an, China\\[0.5em]
  \textsuperscript{4}School of Computer Science and Technology, Xidian University, Xi'an, China\\
  \textsuperscript{*}Corresponding author\\
  [0.5em]
  \textbf{Correspondence:} \\
  \{\texttt{chenlong@}, \texttt{05221170@stu.}, \texttt{Mingxin Wang@stu.}\}\texttt{xupt.edu.cn}, \\
  \{\texttt{huangxiaohua}, \texttt{liuying}\}\texttt{@stumail.nwu.edu.cn}, \\
  \texttt{jiezhao@chd.edu.cn},\\
  \texttt{zyguan@xidian.edu.cn}
  \end{minipage}
}

\date{}




\begin{document}
\maketitle


\begin{abstract}
Accurate modeling of inter-stock relationships is critical for stock price forecasting. However, existing methods predominantly focus on single-state relationships, neglecting the essential complementarity between dynamic and static inter-stock relations. To solve this problem, we propose a Dual Relation Fusion Network (DRFN) to capture the long-term relative stability of stock relation structures while retaining the flexibility to respond to sudden market shifts. Our approach features a novel relative-static relation component that models time-varying long-term patterns and incorporates overnight informational influences. We capture dynamic inter-stock relationships through distance-aware mechanisms, while evolving long-term structures via recurrent fusion of dynamic relations from the prior day with the pre-defined static relations. Experiments demonstrate that our method significantly outperforms the baselines across different markets, with high sensitivity to the co-movement of relational strength and stock price. Code: \textit{https://github.com/bhx-E/DRFN}.
\end{abstract}


% keywords can be removed
% \keywords{First keyword \and Second keyword \and More}


\section{Introduction}
Reliable stock price predictions facilitate investors in making profitable investment decisions and aid financial regulators in identifying potential financial risks. Early research on stock prediction focused on enriching the diversity of input information by introducing more market indicators \cite{mehtab2021stock} or combining texts and indicators \cite{li2020multimodal,gao2022integrating}. The subsequent studies focus on learning the temporal patterns of time-series historical data \cite{gulmez2023stock,li2024master}. However, these studies often overlook inter-stock relationships, which are critical drivers of stock price fluctuations. Accurately modeling these relationships in real-world markets remains challenging due to their inherent complexity and diversity. Specifically, they span: industry-sector correlations, vertical supply chain linkages, competitive dynamics, and cross-asset contagion effects. Existing studies learning inter-stock correlations typically adopt single-regime models, focusing exclusively on short-term fluctuations \cite{zhu2024lsr} or long-term dependencies \cite{chen2023predicting}. This leads to dual limitations: purely static models fail to capture evolving relations, while solely dynamic approaches may undermine model stability. During significant events (e.g., global crises or policy shifts), previously trained models often fail abruptly as relationships shift beyond their adaptive capacity. To this end, dynamic modeling frameworks incorporate static structural anchors to mitigate volatility. He et al. \cite{he2022static} proposed a static-dynamic graph neural network for stock recommendations. They used a data-driven graph learner to capture static/dynamic stock relations and a graph interactor to fuse multi-graph information, followed by graph convolution and ranking modules for stock recommendations. However, this study omits three critical issues: (1) News impact neglect: The effect of news on both short- and long-term stock relationships is unaddressed; (2) Time-varying structure: Stock relations evolve dynamically with new information (e.g., Apple-OpenAI's competitive-to-collaborative structural shift through strategic AI integration); and (3) Nonlinear interaction: Dynamic and static relationships exhibit complex, uncertain couplings, and simple concatenation cannot capture market complexity.

To tackle these critical shortcomings, we propose a Dual-Relation Fusion Network for stock price prediction. The key idea is to elucidate the complementarity between static and dynamic relationships, thereby enabling the model to maintain stable relationship patterns acquired over time and to rapidly adapt to abrupt changes. Specifically, to address market inefficiencies, behavioral biases, and lagged event responses — all phenomena requiring complementary multi-source information — we align news and market indicator embeddings via contrastive learning, followed by bilinear fusion. This captures the nuanced interactions between qualitative sentiment and quantitative signals, yielding robust stock representations with an enhanced ability to anticipate price movements driven by market inefficiencies. Next, we design a novel relation fusion module to capture static-dynamic complementarity through three core components: 1) Dynamic Relation Learning: Model synergistic (positive) and adversarial (negative) correlations via cross-attention with distance modulation, capturing short-term mutual influences between stocks. 2) Relative Static Relation modeling: Enhance predefined static relationships by fusing prior-day dynamic features with weighted static features, capturing time-varying long-term structures and overnight information impacts. 3) Dynamic-Relative Static Fusion: Fuse dynamic and relative-static features using bidirectional cross-attention. After relation modeling, we integrate the learned relations with the original news and market embeddings via residual connections to predict the next-day stock returns.

The main contributions are as follows: 1) Dual Relation Fusion framework: Integrates short-term volatility dynamics with relative long-term dependencies, simultaneously capturing structural stability while maintaining market responsiveness. 2) Distance-Modulated Correlation Modeling: Quantifies synergistic-adversarial stock dynamics through distance-aware mechanisms to capture time-varying inter-stock patterns. 3) Adaptive Static Relation Construction: Models evolve long-term structures via recurrent fusion of prior-day dynamic relations with the pre-defined static relations.

\section{Related Work}
Earlier studies have used historical data and technical indicators of individual stocks for predictions \cite{patel2015predicting,nelson2017stock}. Li et al. \cite{li2020multimodal} used online news text and stock price time-series market indicators as input, and they proposed a multimodal event-driven LSTM model to perform the interaction and integration of the multimodal inputs to enhance the performance of stock price prediction. However, these studies ignored correlations between stocks and deemed them independent entities, making it difficult to capture their interactive influence on stock predictions. Subsequent studies have recognized this issue and developed stock prediction methods based on relational modeling. Graph models are widely used in relationship modeling. Ying et al. \cite{ying2020time} designed a time-aware relational attention mechanism to capture the time-varying correlation strength between stocks. However, the use of predefined relations lacks an effective mechanism for handling attribute-related influences. The method in \cite{cheng2021modeling} infers latent firm relations from market signals and, with its attribute-driven graph attention network, captures attribute-sensitive momentum spillovers and complex market signal interactions. Huynh et al. \cite{huynh2023efficient} attempted to capture more sophisticated dynamic relations using industry hypergraphs, wavelet hypergraph convolutions, and temporal generative filters to capture multi-order and internal dynamic relationships. Relation modeling is further improved by capturing momentary and cross-time stock correlations through intra- and inter-stock information aggregation, using a gating mechanism for automatic feature selection based on market information \cite{li2024master}. However, this may not fully capture the long-term invariant relationships between stocks. LSR-IGRU \cite{zhu2024lsr} constructed long-short-term relationship matrices using secondary industry and overnight price information, enhancing GRU input to better integrate temporal and relationship information. Extending LSR-IGRU, we propose DRFN to capture persistent relational structures while maintaining flexibility in responding to sudden market shifts.




\begin{figure*}[t]
    \centering
        \includegraphics[scale=0.11]{System_Framework.jpg}
    \caption{Overview of our framework.}
    \label{fig:System Framework Overview}
\end{figure*}

\section{Method}
We predict the next-day returns for a set of \(Z\) input stocks, formulated as a regression problem. The daily return is computed as the ratio of the interday closing price differential to the prior trading day's closing price: \(r_{t+1} = \frac{c_{t+1} - c_{t}}{c_{t}}\). For stock $i$, the input data $D = \{D_{i}^{\theta}\}^t_{\theta=t-T+1}$ contains news texts and market indicators from the past $T$ days; Specifically, on day $t$, $D_{i}^{t}$ comprises $Q$ news texts $\{ n_{i,q}^t \}_{q=1}^Q$ and six market indicators: open ($p^t_o$), close ($p^t_c$), adjusted close ($p^t_{ac}$), high ($p^t_h$), low ($p^t_l$), and volume ($p^t_v$), which form the vector $\mathbf{p}^{t}_{i} = [ p^t_{o}, p^t_{c}, p^t_{ac}, p^t_{h}, p^t_{l}, p^t_{v}]$.

The DRFN framework is illustrated in {Figure~\ref{fig:System Framework Overview}. First, we encode the input textual news and market indicator data using BERT\&GRU and GRU, respectively. The obtained temporal hidden states from both modalities are then aligned and fused to generate multimodal fusion vectors. Next, the fusion vectors are fed into the GRU module for extracting temporal patterns. Subsequently, the relation fusion module models long-term static and short-term dynamic inter-stock relations using the temporal multimodal features and the predefined static relations. Finally, the relation features are combined with the residual-propagated temporal hidden states derived from both textual news and market indicators to form unified representations. These representations are then transformed by a nonlinear mapping function for stock price predictions.


\subsection{News Text Encoding}
News sentiment drives investor behavior, triggering stock price changes. More importantly, news shapes stock correlations. For instance, supply chain news affects upstream and downstream companies; major positive news for one firm can negatively impact its competitors. We encode each news text using a pre-trained BERT model as follows:
\begin{equation}
    \mathbf{n}_{i,q}^{\theta} = BERT(n_{i,q}^{\theta}) \in \mathbb{R}^{L}
\end{equation}
The $Q$ encoded textual news embeddings in one day are represented by the matrix $\mathbf{N}_i^\theta = \left[ \mathbf{n}_{i,1}^\theta, \mathbf{n}_{i,2}^\theta, \dots, \mathbf{n}_{i,Q}^\theta \right]$. We then use a GRU to learn temporal patterns from all news over the $T$-day period:
\begin{equation}
    \mathbf{h}_{i,n}^{\theta} = GRU(\mathbf{N}_i^{\theta}, \mathbf{h}_{i,n}^{{\theta}-1}) \in \mathbb{R}^{L}
\end{equation}
where $\mathbf{h}_{i,n}^{\theta}$ represents the temporal hidden state of news embeddings at time step $\theta \in [t-T+1, t]$.

\subsection{Market Indicator Encoding}
Market indicators capture market dynamics, including trends and volatility. We model the market indicator dynamics using a separate GRU:
\begin{equation}
    \mathbf{h}_{i,m}^{\theta} = GRU(\mathbf{P}_i^{\theta}, \mathbf{h}_{i,m}^{\theta-1}) \in \mathbb{R}^{L'}
\end{equation}
where $\mathbf{h}_{i,m}^{\theta}$ denotes the temporal hidden states for the market indicators over the $T$-day time window.

\subsection{Multimodal Alignment and Fusion}
Market indicators and textual news exhibit strong complementarity, primarily grounded in Behavioral Finance. Investor decisions are systematically influenced by sentiment propagated through news narratives. Consequently, while quantitative market indicators reflect collective trading actions and outcomes, textual news reveals the qualitative narratives, sentiments, and behavioral drivers underpinning these actions. This provides a critical qualitative context that is absent from the quantitative data alone. For $Z$ stocks, we apply contrastive learning with InfoNCE loss \cite{oord2018representation} to align the temporal hidden states of the market indicators and textual news:
\begin{align}
    \begin{split}
        \mathcal{L}_{\text{Align}} = {} & -\frac{1}{T} \cdot \frac{1}{Z}\sum_{\theta=t-T+1}^{t}\sum_{i=1}^{Z} \\
        & \log \frac{\exp(\text{sim}(\mathbf{h}_{i,n}^{\theta}, \mathbf{h}_{i,m}^{\theta}) / \tau)}{\sum_{j=1}^{Z} \exp(\text{sim}(\mathbf{h}_{i,n}^{\theta}, \mathbf{h}_{j,m}^{\theta}) / \tau)}
    \end{split}
    \label{eq:modal_alignment}
\end{align}
where \( sim(\cdot) \) is the cosine similarity function; \( \tau \) denotes the temperature hyperparameter. Cross-modal alignment reduces the distance between these two modalities within the same stock, while increasing the distance across stocks between one stock's market indicator features and another stock's textual news features.

Following alignment, we use concatenation and weighted bilinear transformation to fuse the aligned cross-modal features and exploit their information complementarity:
\begin{equation}
    \mathbf{x}^{\theta}_i = tanh\left( \mathbf{W}_i [\mathbf{h}^{\theta}_{i,m} \Vert \mathbf{h}^{\theta}_{i,n}] + 
    \mathbf{h}^{\theta}_{i,m} \mathbf{T}_i^{[1:K]} \mathbf{h}^{\theta}_{i,n} + \mathbf{b}_i \right)
    \label{eq:fusion_tensor}
\end{equation}
where $\mathbf{T}_i^{[1:K]} \in \mathbb{R}^{L \times L' \times K}$ and $\mathbf{W}_i \in \mathbb{R}^{K \times (L + L')}$ are weight matrices, $\mathbf{b}_i$ is the bias term, and these parameters are shared across all time steps. $[\cdot \| \cdot]$ represents the concatenation operation. Note that we perform this fusion across all $T$ days. In fusion, concatenation preserves the original feature integrity, whereas the $K$-dimensional bilinear tensor product enables a structured cross-modal interaction. The $k$-th entry is computed using one tensor slice. Bilinear operations surpass linear combinations in modeling complex market phenomena. For instance, the impact of news sentiment on stock prices is often nonlinear, exerting greater influence beyond certain intensity thresholds, a relationship effectively captured by bilinear fusion.

\subsection{Temporal Pattern Learning}
After multimodal fusion, the sequence of fused vectors \(\{\mathbf{x}^{\theta}_i\}_{\theta=t-T+1}^{t}\) is fed into a GRU, which outputs all hidden states:
\begin{equation}
   \mathbf{v}^\theta_i = \text{GRU}\left( \{\mathbf{x}^{\theta}_i\}_{\theta=t-T+1}^{t} \right) \in \mathbb{R}^F
    \label{eq:sequential_tensor}
\end{equation}
where \(F\) denotes the hidden size of the GRU. This module captures cross-modal temporal collaborative patterns by dynamically adjusting memory retention through its update gate, focusing on the short-term impacts of news events while retaining long-term periodic patterns in market dynamics.
\begin{figure*}[t]
    \centering
        \includegraphics[scale=0.153]{dynamic_and_static_fusion.jpg}
    \caption{Relation Fusion Module}
    \label{fig:Dynamic and Relative Static Fusion}
\end{figure*}

\subsection{Relation Fusion Module}
Inter-stock relationships are not fixed and are subject to dynamic changes due to macroeconomic factors, industry, market sentiment, and other factors. To capture this complementarity, we propose a novel relation fusion module that combines short-term dynamic relationships with long-term relatively static relationships. The module structure is shown in Figure~\ref{fig:Dynamic and Relative Static Fusion}. It comprises five components: Predefined Static Relations, Dynamic Relation Learning (the zoomed-in callout in Figure~\ref{fig:Dynamic and Relative Static Fusion}), Relative-Static Relation Modeling, Dynamic-Relative Static Fusion, and Relational-Temporal Interactive Fusion.

\textbf{Predefined Static Relations.} Predefined static relationships are long-term, stable business relationships that exist among firms, derived from known market structures and enterprise connections. The predefined static relation matrix $\mathbf{S}_{pre}$ is constructed from company entities and relationships provided by Wikidata \cite{vrandevcic2014wikidata}. It is a binary matrix where 1 denotes related stocks and 0 denotes unrelated stocks, serving as the static relation structure over the past $T$-day time window. We apply a diagonal mask to $\mathbf{S}_{pre}$ to exclude self-correlation.
 
\textbf{Dynamic Relation Learning.} Stock correlations evolve dynamically under external influences. For instance, the new energy and traditional energy sectors may shift from a positive correlation during policy stimuli to a negative correlation during capital reconfiguration. To capture these time-varying inter-stock dynamics, we integrate dual-correlation modeling with distance modulation:
\begin{equation}
    \mathbf{A}^{{\theta},(a)} = \frac{\mathbf{V}^{{\theta}} \left(a \cdot \mathbf{V}^{\theta}\right)^\top}{\sqrt{F}} \in \mathbb{R}^{Z \times Z}
\end{equation}
\begin{equation}
    D^{{\theta},(a)}_{ij} = \left\| \mathbf{v}^{{\theta}}_i - a \cdot \mathbf{v}^{{\theta}}_j \right\|_1 \in \mathbb{R}^{Z \times Z}
\end{equation}
\begin{equation}
    \tilde{\mathbf{A}}^{{\theta},(a)} = \mathbf{A}^{{\theta},(a)} \odot \exp\left(-\mathbf{D}^{{\theta},(a)}\right)
\end{equation}
where $\mathbf{V}^\theta = [\mathbf{v}^\theta_1, \mathbf{v}^\theta_2, \cdots, \mathbf{v}^\theta_Z] \in \mathbb{R}^{Z \times F}$ denotes the hidden state matrix from GRU. The attention matrix $\mathbf{A}^{{\theta},(a)}$ models the relational structures among stocks at every time, with the scalar $a \in \{+1, -1\}$ switching between positive and negative correlation modes. The term $\sqrt{F}$ serves as a scaling factor to normalize the attention scores. $\mathbf{D}^{{\theta},(a)}$ quantifies pairwise behavioral discrepancies, while $\tilde{\mathbf{A}}^{{\theta},(a)}$ fuses directional alignment and distance-based modulation via Hadamard product $\odot$.
\begin{equation}
    \mathbf{F}^{{\theta},(a)} = softmax \left( \frac{ \tilde{\mathbf{A}}^{{\theta},(a)} \cdot (\tilde{\mathbf{A}}^{{\theta},(-a)})^\top }{ \sqrt{Z} } \right) \cdot \tilde{\mathbf{A}}^{{\theta},(-a)}
    \label{eq:dualatt}
\end{equation}
\begin{equation}
    \mathbf{D}_{dynamic}^{\theta} = GELU \left( \left[ \mathbf{F}^{{\theta},(+1)} \, \| \, \mathbf{F}^{\theta,(-1)} \right] \cdot \mathbf{W}_d \right)
    \label{eq:gelu}
\end{equation}
Next, cross-attention in Eq.~\ref{eq:dualatt} performs cross-perspective fusion by: 1) Identifying stock pairs exhibiting strong positive but weak negative correlations (or vice versa) through interactions between $\tilde{\mathbf{A}}^{\theta,(a)}$ and $\tilde{\mathbf{A}}^{\theta,(-a)}$;  2) Assigning cross-perspective importance weights via softmax to highlight dominant association patterns. Then a nonlinear projection with a learnable parameter matrix $\mathbf{W}_d \in \mathbb{R}^{2Z \times Z}$ integrates the positive and negative correlation fusion matrices $\mathbf{F}^{\theta,(+1)}$ and $\mathbf{F}^{\theta,(-1)}$ to produce the dynamic relation matrix $\mathbf{D}_{\text{dynamic}}^{\theta} \in \mathbb{R}^{Z \times Z}$, which dynamically evolves with market conditions.

\textbf{Relative-Static Relation Modeling.} Overnight information is released intensively after the opening of the next trading day and has a significant impact on stock price movements \cite{lou2019tug}. Nevertheless, predefined static relations solely capture long-term relation patterns while neglecting this ultra-short-term information. To address this limitation, we integrate the predefined static relations with the dynamic relation from the preceding day, resulting in a final static relation termed the relative-static relation. Specifically, we employ linear weighting to combine the dynamic relation of the previous time step \(\mathbf{D}_{dynamic}^{\theta-1} \in \mathbb{R}^{Z \times Z}\) and the predefined static relation \(\mathbf{S}_{pre} \in \mathbb{R}^{Z \times Z}\). The computational formula is as follows:
\begin{equation}
    \mathbf{S}_{relative}^{\theta} = \alpha\cdot\mathbf{D}_{dynamic}^{\theta-1}+(1-\alpha) \cdot [\mathbf{W}_s\odot\mathbf{S}_{pre}]
    \label{eq:Relative Static}
\end{equation}
where \(\mathbf{S}_{relative}^{\theta} \in \mathbb{R}^{Z \times Z}\) is the output at time step $\theta$. We add a learnable parameter matrix $\mathbf{W}_s$ that is reused across all time steps to dynamically adjust \(\mathbf{S}_{pre}\). This adjustment enables the predefined static relation to capture the time-varying characteristics of the relationship strength among stocks while maintaining a stable long-term relation structure. Besides, we introduce a learnable scalar parameter $\alpha$ that determines the relative contributions of dynamic and static relationships. Note that at the initial time step $t-T+1$, \(\mathbf{S}_{pre}\) serves as the static relation component since prior dynamic relations are unavailable.

\textbf{Dynamic-Relative Static Fusion.} To capture the complementarity between dynamic and static relationships, we design a dual-relation fusion mechanism that generates comprehensive representations reflecting both short-term fluctuations and long-term stability. Real market conditions reveal no fixed proportionality between these relationships; instead, they present complex and uncertain interactions. Therefore, we implement bidirectional cross-attention for adaptive fusion, enabling dynamic-static integration through bidirectional information flow:

\begin{align}
\mathbf{R}^{\theta} &= GELU\Big( \Big[ \text{Attention}\left( \mathbf{S}^{\theta}_{\text{relative}} ,\mathbf{D}^{\theta}_{\text{dynamic}}, \mathbf{D}^{\theta}_{\text{dynamic}} \right) \notag \\
&\quad \parallel\ \text{Attention}\left( \mathbf{D}^{\theta}_{\text{dynamic}}, \mathbf{S}^{\theta}_{\text{relative}}, \mathbf{S}^{\theta}_{\text{relative}} \right) \Big] \cdot \mathbf{W}_{DRS}  \Big) 
\label{eq:Dynamic and Static}
\end{align}
\begin{equation}
\text{Attention}(Q, K, V) = Softmax\left( \frac{QK^\top}{\sqrt{d_k}} \right) V
\end{equation}
where $\mathbf{R}^{\theta}$ is the fused representation at step $\theta$, and $\mathbf{W}_{DRS} \in \mathbb{R}^{2Z \times Z}$ is a learnable weight matrix. Unlike traditional fusion methods, our bidirectional cross-attention mechanism adaptively controls the degree of fusion between relationships through its inherent adaptability, thereby effectively exploiting their complementary synergy.

\textbf{Relational-Temporal Interactive Fusion.} Building on the stock relations, we further fuse their temporal multimodal features by applying a multi-head bilinear attention mechanism to the relation matrix \(\mathbf{R}^{t}\) and the temporal multimodal feature matrix \(\mathbf{V}^t\):
\begin{equation}
\mathbf{S}^t = \underset{b=1}{\overset{B}{\|}}  tanh\left( \left[ \mathbf{R}^{t,[b]} \cdot (\mathbf{V}^t \mathbf{W}_{TR}^{[b]}) \right] \right) \in R^{Z \times BF'}
\label{eq:Comprehensive Relation}
\end{equation}
where \(\mathbf{S}^t = [\mathbf{s}^t_1, \mathbf{s}^t_2, \cdots, \mathbf{s}^t_Z]\) denotes the final fused representation at time step \(t\), the terminal point of the \(T\)-day time window. $\mathbf{W}_{TR}^{[b]} \in R^{F \times F'}$ denotes the interaction weight matrix of the $b$-th attention head. This attention mechanism guides the model in learning the interaction between relationships and time by exploring multiple feature subspaces. For instance, a stock may be competitive or complementary to other stocks, and these relationships may be reflected in time series as synergies or differences in characteristics such as stock price volatility and trading volume changes.

\subsection{Adaptive Residual Connection and Output}
To capture the immediate impact of news and indicators, we employ residual connections and adaptively fuse temporal embeddings with relation features to generate a unified representation. The fused representation is subsequently transformed via a nonlinear projection to predict stock returns. The final output is computed as:
\begin{align}
\hat{r}^{t+1}_i = \tanh\Big( \mathbf{W_o} \cdot [&\, (\mathbf{s}^t_i)^\top 
+ \alpha_i^{(1)} \cdot (\mathbf{W}_n \cdot(\mathbf{h}_{i,n}^t)^\top + \mathbf{b}_n) \notag \\
&+ \alpha_i^{(2)} \cdot (\mathbf{W}_m \cdot (\mathbf{h}_{i,m}^t)^\top + \mathbf{b}_m)] + b_o \Big)
\label{eq:output_fusion}
\end{align}
where $\mathbf{W}_o \in \mathbb{R}^{1 \times BF'}$, $\mathbf{W}_n \in \mathbb{R}^{BF' \times L}$, and $\mathbf{W}_m \in \mathbb{R}^{BF' \times L'}$ denote weight matrices, with corresponding biases $b_o \in \mathbb{R}$, $\mathbf{b}_n \in \mathbb{R}^{BF'}$, and $\mathbf{b}_m \in \mathbb{R}^{BF'}$. The scalars $\alpha_i^{(1)}$ and $\alpha_i^{(2)}$ are learnable parameters. This adaptive fusion addresses two key challenges: 1) Feature Heterogeneity: Input features are diverse, making simple concatenation noisy. 2) Dynamic Sensitivity: Stocks exhibit varying sensitivities (e.g., policy-sensitive stocks rely more on news, while energy stocks focus more on crude oil prices). Learnable weights enable dynamic parameter adjustment, thus balancing the contributions of different information sources.
\subsection{Objective Function}
We combine Mean Squared Error (MSE) loss and cross-modal alignment loss as the overall objective function:
\begin{equation}
\mathcal{L} = \mathcal{L}_{\text{MSE}} + \lambda \mathcal{L}_{\text{Align}}
\label{eq:total_loss}
\end{equation}
\begin{equation}
\mathcal{L}_{\text{MSE}} = \frac{1}{Z} \sum_{i=1}^{Z} \left( r_i^{t+1} - \hat{r}_i^{t+1} \right)^2
\label{eq:mse_loss}
\end{equation}
where $r_i^{t+1}$ and $\hat{r}_i^{t+1}$ denote the ground-truth and predicted returns of stock $i$ at time $t+1$, respectively, and $\lambda$ is a balancing hyperparameter.


\section{Experiments}

\begin{table*}[t]
\centering
\begin{tabular}{c|c|c|c|
>{\centering\arraybackslash}p{2cm}|
>{\centering\arraybackslash}p{2cm}|
>{\centering\arraybackslash}p{2cm}|
>{\centering\arraybackslash}p{2cm}}

\toprule
\multirow{2}{*}{Model} & \multirow{2}{*}{Relation} & \multirow{2}{*}{Text} & \multirow{2}{*}{Price} & \multicolumn{2}{c|}{CMIN-US} & \multicolumn{2}{c}{CMIN-CN}  \\
&   &   &    & RMSE (Avg.)    & MAE (Avg.)     & RMSE (Avg.)   & MAE (Avg.)      \\
\midrule
LSTM  & \texttimes & \checkmark & \checkmark  &  0.0232   &0.0180     &0.0359    &0.0285   \\
GRU   & \texttimes  & \checkmark & \checkmark & 0.0230    &0.0177     &0.0356    & 0.0277   \\
GCN   & \checkmark & \checkmark & \checkmark & 0.0221     & 0.0168    &0.0347    & 0.0264        \\
GAT   & \checkmark & \checkmark & \checkmark & 0.0217   &0.0164   & 0.0343   &0.0257  \\
MASTER & \checkmark & \texttimes & \checkmark   &0.0205    &  0.0156   & 0.0331   & 0.0245  \\
SAMBA & \texttimes & \texttimes & \checkmark   & 0.0201   &0.0153    & 0.0325   & 0.0237     \\
LSR-IGRU & \checkmark  & \texttimes & \checkmark &0.0192     & 0.0141    &0.0317     &0.0228      \\
\textbf{DRFN} & \checkmark  &\checkmark &  \checkmark & \textbf{0.0166} & \textbf{0.0117} & \textbf{0.0270} & \textbf{0.0191} \\  
\bottomrule
\end{tabular}
\caption{Main comparison results}
\label{tab:a}
\end{table*}

\subsection{Dataset and Experiment Settings}
We use datasets from China (CMIN-CN) and the US (CMIN-US) markets~\cite{luo2023causality} for evaluation. CMIN-US contains 105 stocks over 1,006 trading days; CMIN-CN comprises 227 stocks across 968 days\footnote{Samples from missing trading days are removed.}. Both datasets are chronologically split 8:1:1 into training, validation, and test sets. We use the Root Mean Square Error (RMSE) and Mean Absolute Error (MAE) as evaluation metrics. \emph{Due to market volatility, each baseline is trained 10 times with random initialization, and the average performance is reported.}

Cross-market hyperparameter tuning produced identical configurations on both validation sets, demonstrating strong cross-market generalization. Data specifications: maximum daily news count $Q=30$ (40 tokens/text); BERT-encoded text vectors $L=768$ (GRU preserves dimension); GRU-encoded temporal vectors of market indicators $L'=64$. Optimal model parameters: time window $T=5$ ($T \in [2,10]$), attention heads $B=6$ ($B \in \{2,4,6,8,10\}$) using even counts for symmetric modeling. Dimensions: GRU $F=128$ and attention $F'=256$ (both $\in \{32,64,128,256,512\}$), balance parameter $\lambda=1$ ($\lambda \in \{0.1,0.5,1,1.5,2\}$). The model was trained using the Adam optimizer with a learning rate of \(2 \times 10^{-3}\), implemented on a single NVIDIA Tesla P100 GPU (16G), with a total runtime of approximately 11 hours for convergence.

\subsection{Baselines and Main Comparisons}
\textbf{LSTM and GRU.} Long Short-Term Memory (LSTM) \cite{hochreiter1997long} and Gated Recurrent Unit (GRU) \cite{chung2014empirical} are classic sequence models widely used for time-series prediction tasks. We employ a 2-layer LSTM and 2-layer GRU as sequence-model-based baseline methods.

\noindent\textbf{GAT.} Graph Attention neTwork \cite{velickovic2017graph}  assigns varying weights to different nodes within a neighborhood. We use GAT with two masked self-attention layers.

\noindent\textbf{GCN.} The Graph Convolutional Network \cite{kipf2017semi} is commonly used to capture stock relations. We employ a GCN with two convolutional layers as a baseline. The attributes of adjacent nodes are linearly aggregated to the central node using normalized stock relationships. 

\noindent\textbf{SAMBA.} \cite{mehrabian2025mamba} proposed SAMBA, a framework that combines the Mamba architecture and graph neural networks (GNNs). SAMBA employs a bidirectional Mamba module to capture long-term dependencies in historical price data and to learn inter-stock-feature dependencies via adaptive graph convolution.

\noindent\textbf{MASTER.} {\cite{li2024master} proposed a Market-Guided Stock TransformER model to capture momentary and cross-time correlations among stocks using market indicators.

\noindent\textbf{LSR-IGRU.} \cite{zhu2024lsr} proposed an Improved GRU to capture the Long Short-term Relations of stocks. It uses secondary industry information and overnight price changes to construct long- and short-term stock relation representations, and then the improved GRU fuses the original input with these representations via concatenating and weighting unit hidden states.

\noindent\textbf{DRFN.} This is our method.

\noindent Note that LSTM and GRU use concatenation of technical indicators and text features as inputs. GCN and GAT employ indicator and text feature combinations as graph nodes and connect them with predefined static relations.

The comparison results for the two markets are similar, shown in Table \ref{tab:a}. LSTM and GRU achieve the lowest performance because they capture only fundamental temporal patterns, which are insufficient for modeling underlying relationships in multimodal data. GAT and GCN surpass the two basic sequence models because graph-based architectures effectively model the stock relationships. In comparison, MASTER achieves a better performance by capturing fine-grained stock relations, both intra- and inter-stock, along with asymmetric and dynamic inter-stock correlations. SAMBA outperforms MASTER, due to its enhanced capture of long-term dependencies via bidirectional time-series modeling and its effective modeling of intra-stock feature dependencies through adaptive structural modeling. LSR-IGRU outperforms SAMBA through its incorporation of primary/secondary industry data and overnight prices in relation modeling. This approach effectively captures cross-industry dependencies and short-term price fluctuation patterns, yielding superior results through richer market-signal extraction. By leveraging complementary learning of static and dynamic stock relations, our method achieves optimal performance across both markets.

\begin{table*}[t]
\centering
\begin{tabular}{c|
>{\centering\arraybackslash}p{2cm}|
>{\centering\arraybackslash}p{2cm}|
>{\centering\arraybackslash}p{2cm}|
>{\centering\arraybackslash}p{2cm}} 
\toprule
\multirow{2}{*}{Model} & \multicolumn{2}{c|}{CMIN-US} & \multicolumn{2}{c}{CMIN-CN}  \\
& RMSE (Avg.)    & MAE (Avg.)      & RMSE (Avg.)   & MAE (Avg.)    \\
\midrule
Static & 0.0221   &0.0162    & 0.0337  &0.0250     \\
Relative Static & 0.0208  & 0.0151    & 0.0326   & 0.0240   \\
Dynamic  & 0.0196   &0.0145    &0.0315    &0.0233     \\
% Weight Fusion &0.0183    &0.0130   &0.0294    &0.0219    \\
w/o Modality Alignment  & 0.0191   &0.0139     &0.0310    &0.0225    \\
w/o adaptive weighted residual mechanism  & 0.0174   & 0.0123    & 0.0280   &0.0204    \\
\textbf{DRFN} & \textbf{0.0166} & \textbf{0.0117} & \textbf{0.0270} & \textbf{0.0191} \\  
\bottomrule
\end{tabular}
\caption{Ablation study}
\label{tab:b}
\end{table*}

\subsection{Relation Modeling Variants and Ablation Study}
Considering the centrality of relation modeling to our framework, we evaluate the relation fusion module against its variants to demonstrate its effectiveness, with complementary ablation studies of other components. The results are shown in Table \ref{tab:b}. Replacing the relation fusion module with Predefined Static Relations (\emph{Static}) causes a significant performance deterioration. This demonstrates that stock relations are inherently dynamic. \emph{Relative Static} employs the Relative Static Relating Modeling, outperforming \emph{Static} by capturing time-varying patterns of the stocks' relations in long-term trading ranges and incorporating overnight information. \emph{Dynamic} (using only Dynamic Relation Learning) performs best among the three variants, proving that short-term dynamic stock relations correlate more strongly with price fluctuations than long-term static relations. Nevertheless, it consistently underperforms our static-dynamic fusion method because static structural anchors are essential in dynamic modeling frameworks to mitigate market volatility. Further ablation studies highlight modality alignment's necessity and the direct influence of input embeddings on stock dynamics.

\subsection{Relation Attention Sensitivity to Price Fluctuations}

% 图3 - 统一使用 figure 环境
\begin{figure}[htbp]
    \centering
    \includegraphics[width=0.6\textwidth]{sensitivity_comparison.jpg}  % 调整宽度适应单栏
    \caption{Sensitivity analysis}
    \label{fig:sensitivity_analysis}
\end{figure}

% 图4 - 同样使用 figure 环境
\begin{figure}[htbp]
    \centering
    \includegraphics[width=0.9\textwidth]{Example_Analysis.jpg}  % 调整宽度适应单栏
    \caption{Comparative sensitivity of different relation modeling methods to AAPL and AMZN price movements}
    \label{fig:AAPL_AMZN_analysis}
\end{figure}
Modeling stock relationships is the main contribution of our work. We evaluate whether attention intensity between stocks in the relational module responds to price fluctuations. Specifically, we assess the sensitivity of three relational modeling approaches to market turbulence on the test set\footnote{Given consistent cross-market comparison results, we analyze the full 90-trading-day test set of US market data in CMIN-US.}: 1) the Static Relation method employed in GAT \cite{velickovic2017graph}, 2) the Dynamic Relation Learning proposed in LSR-IGRU \cite{zhu2024lsr}, and 3) our Static-Dynamic Fusion. To measure stock relationship sensitivity to price fluctuations, we compute the Pearson correlation between relational strength changes ($\Delta S$) and price changes ($\Delta C$):
\begin{equation}
\rho = \frac{ \sum_{\theta=1}^{T'} \left( \Delta S^{(\theta)} - \overline{\Delta S} \right) \left( \Delta C^{(\theta)} - \overline{\Delta C} \right) }
{ \sqrt{ \sum_{\theta=1}^{T'} \left( \Delta S^{(\theta)} - \overline{\Delta S} \right)^2 } \cdot \sqrt{ \sum_{\theta=1}^{T'} \left( \Delta C^{(\theta)} - \overline{\Delta C} \right)^2 } }
\end{equation}
where $\Delta S^{(\theta)}$ and $\Delta C^{(\theta)}$ are the predicted changes in relational strength and stock price from $\theta-1$ to $\theta$, with $\overline{\Delta S}$ and $\overline{\Delta C}$ their respective means over 90-day windows ($T' = 90$). The results are shown in {Figure~\ref{fig:sensitivity_comparison Analysis}}. Static relation modeling with predefined stock relations inherently lacks price sensitivity, owing to its fixed connectivity. While modeling dynamic stock relations adapts inter-stock connectivity to market fluctuations, its relational sensitivity to price volatility remains minimal. This is why LSR-IGRU underperforms our method because it models stock relationships through overnight price similarity and historical patterns but fails to capture real-time relationship dynamics driven by price fluctuations, a fundamental limitation that causes performance deterioration. Our Static-Dynamic Fusion module achieves superior sensitivity to relation-price co-movement. By recurrently integrating historical and current dynamics, we accumulate cross-period relations, thereby enhancing co-movement sensitivity.

We further compare the sensitivity of different relation modeling methods to market shifts by visualizing the relational strength and stock prices for AAPL and AMZN (January 10--23, 2018) in Figure~\ref{fig:AAPL and AMZN Example Analysis}. Long-term analysis reveals a weak correlation between the dynamic module and actual price fluctuations (coefficients: 0.053 and -0.401), particularly in directional consistency. In stark contrast, our Static-Dynamic Fusion module achieves significantly stronger correlations (0.439 and 0.473). A Short-term analysis of January 10-11 is a representative case: During synchronous AAPL-AMZN price surges, our Static-Dynamic Fusion module closely tracked price movements in both direction and amplitude, confirming strong short-term synergy; the dynamic module, however, failed to capture this co-movement. Building on our quantitative analysis, the case study visually illustrates enhanced relation-to-price sensitivity in our static-dynamic fusion module, achieved through temporal superposition effects of dynamic relations.

\section{Conclusions}
In this study, we propose a dual-relation fusion network to achieve complementary fusion between static and dynamic stock relations, resulting in the learning of the long-term relative stability of the stock relation structure while maintaining the model's flexibility in responding to sudden market fluctuations. Our method demonstrates consistent superiority in comparative experiments, with the relation fusion module exhibiting enhanced sensitivity to temporal stock price fluctuations.

\section{Acknowledgments}
This paper is supported by the National Natural Science Foundation of China (62203354, 62472340, 6230336, 62133012, 62425605, 62306235, 62502051), the Key Research and Development Program of Shaanxi (Program Nos. 2024GX-YBXM-122, 22025CY-YBXM-041, 2024CY2-GJHX-15, 2022ZDLGY01-10), the Natural Science Basic Research Program of Shaanxi (Program Nos. 2025JC-QYXQ-040, 2025JC-YBQN-900, 2025CY-YBXM-067), Guangdong Basic and Applied Basic Research Foundation (Program No.2025A1515010267), the Postdoctoral Research Project Sustentation Fund of Shaanxi (No.2024BSHSDZZ112), the Postdoctoral Fellowship Program of CPSF (Grant No. GZC20251101) and the Key Scientific Research Projects of the Shaanxi Provincial Education Department (No. 24JR152).

\bibliographystyle{unsrt}  

\bibliography{main} 
% \documentclass[sigconf]{acmart}
%\documentclass[manuscript]{acmart}
% \documentclass[manuscript,review,anonymous]{acmart}
\usepackage{tabularx}
\usepackage{array}
\usepackage{float}
\usepackage{hyphenat}

\usepackage[linesnumbered,ruled,vlined]{algorithm2e}
\usepackage{comment}
\setlength{\tabcolsep}{10pt}
\renewcommand{\arraystretch}{1.5}

\AtBeginDocument{%
  \providecommand\BibTeX{{%
    Bib\TeX}}}

\setcopyright{none}
\copyrightyear{2024}
\acmYear{2024}
\acmDOI{XXXXXXX.XXXXXXX}

\begin{document}

\title{When Should Users Check? A Decision-Theoretic Model of Confirmation Frequency in Multi-Step AI Agent Tasks}
% other options: 
% \title{Modeling Confirmation Frequency in Agentic AI: Balancing User Burden and Rollback Cost}

\author{Jieyu Zhou}
\affiliation{%
  \institution{Georgia Institute of Technology}
  \city{Atlanta}
  \state{Georgia}
  \country{USA}
}
\email{jzhou625@gatech.edu}

\author{Aryan Roy}
\affiliation{%
  \institution{Georgia Institute of Technology}
  \city{Atlanta}
  \state{Georgia}
  \country{USA}
}
\email{aroy389@gatech.edu}

\author{Sneh Gupta}
\affiliation{%
  \institution{Georgia Institute of Technology}
  \city{Atlanta}
  \state{Georgia}
  \country{USA}
}
\email{sgupta852@gatech.edu}

\author{Daniel Weitekamp}
\affiliation{%
  \institution{Georgia Institute of Technology}
  \city{Atlanta}
  \state{Georgia}
  \country{USA}
}
\email{dweitekamp3@gatech.edu}

\author{Christopher J. MacLellan}
\affiliation{%
  \institution{Georgia Institute of Technology}
  \city{Atlanta}
  \state{Georgia}
  \country{USA}
}
\email{cmaclell@gatech.edu}

\begin{teaserfigure}
  \centering
  \includegraphics[width=\textwidth]{diagram.png}
  \caption{Confirmation–Diagnosis–Correction–Redo (CDCR) Pattern. Based on the expected time of user interaction under the CDCR pattern, our model decides whether the agent should proceed autonomously or prompt the user to step in and supervise.}
  \Description{Confirmation–Diagnosis–Correction–Redo (CDCR)}
  \label{fig:CDCR}
\end{teaserfigure}

\begin{abstract}
Existing AI agents typically execute multi-step tasks autonomously and only allow user confirmation at the end. During execution, users have little control, making the confirm-at-end approach brittle: a single error can cascade and force a complete restart. Confirming every step avoids such failures, but imposes tedious overhead. Balancing excessive interruptions against costly rollbacks remains an open challenge. We address this problem by modeling confirmation as a minimum time scheduling problem. We conducted a formative study with eight participants, which revealed a recurring \textit{Confirmation–Diagnosis–Correction–Redo (CDCR)} pattern in how users monitor errors. Based on this pattern, we developed a decision-theoretic model to determine time-efficient confirmation point placement. We then evaluated our approach using a within-subjects study where 48 participants monitored AI agents and repaired their mistakes while executing tasks. Results show that 81\% of participants preferred our intermediate confirmation approach over the confirm-at-end approach used by existing systems, and task completion time was reduced by 13.54\%. 
\end{abstract}
\maketitle


\keywords{AI agents, mathematical model, error handling}






\section{Introduction}
With the advancement of Artificial Intelligence (AI), Agentic AI has emerged as a concept describing agents that assist humans in carrying out complex tasks \cite{durante2024agent, pallagani2024prospects, huang2024understanding, wang2024survey, yao2023react, sumers2023cognitive, schick2023toolformer, wei2022chain}. These AI agents have been deployed across diverse domains, including multi-modal devices \cite{rabbit2023, chang2024partnr}, web-based systems \cite{huq2023s,li2023using, li2017sugilite}, and game controls \cite{wang2023voyager,wang2023describe,lawley2023val}. Their rapid proliferation has reignited a long-standing HCI debate \cite{horvitz1999principles, shneiderman1997direct, lieberman1997autonomous, maes1995agents}: should agents operate under direct human supervision and control or should they operate in a mostly autonomous fashion?

Most commercial AI agents today prioritize autonomy, executing tasks end-to-end with little to no user oversight \cite{openai2025agent, anthropic2024claude, manus2024}. Yet these agents remain highly error-prone, achieving only ~30\% accuracy on multi-step benchmarks \cite{ma2024spreadsheetbench, xie2024osworld, xu2024theagentcompanybenchmarkingllmagents}, largely due to their limited capacity for generalizable reasoning and planning \cite{stechly2024chain, valmeekam2024planning}. Errors are especially costly in long-horizon agentic tasks, which often span 30 steps and 10–30 minutes of execution \cite{hu2024dawn, bonatti2024windows}. A single mistake can cascade into complete task failure, forcing users to redo the entire process—doubling execution time and, through repeated API calls, adding both monetary costs \cite{kapoor2024ai, feng2024cocoa, zhang2025leveraging} and increasing the agent's carbon footprint \cite{patterson2022carbon, luccioni2023estimating}. This situation highlights the need to engage user inspections over agent execution.

The timing of such inspections is essentially a tradeoff between autonomy and control. At one extreme, too much user control (e.g., the user confirming after every step) prevents error propagation but imposes substantial interaction overhead, reducing efficiency to the level of manual execution or even worse. At the other extreme, too much automation (e.g., the user only confirming at the end) reduces the burden on the user but increases the risks of errors and costs associated with re-executing the tasks. The challenge is deciding when user checks should occur to balance the efficiency of automation with the safeguards of user control.

Prior HCI research has largely focused on \textbf{how} to provide user control, through visualizing the reasoning process, editing execution results, and providing error feedback \cite{arawjo2024chainforge, epperson2025interactive, kazemitabaar2024improving, feng2024cocoa, lam2022more, ashktorab2019resilient, li2020multi, levy2021assessing}. Few studies examine \textbf{when} such control should occur. A notable exception is Horvitz’s seminal work on expected utility models for single-event decisions \cite{horvitz1999principles}, but these approaches do not extend to long-horizon tasks, where sequential dependencies and cumulative costs shape overall outcomes.  

In this paper, we address the open question of \textbf{when user checks should occur in long-horizon agentic tasks}. We first review representative agent benchmarks and deployed systems and present the results of a formative study we conducted with eight participants. The study revealed dissatisfaction with the confirm-at-end strategy used by existing agentic platforms. It also surfaced a recurring \textbf{Confirmation–Diagnosis–Correction–Redo (CDCR)} pattern employed by users to supervise agents. Building on these insights, we developed a decision-theoretic model that identifies periodic confirmation points that minimize total expected task completion time while ensuring task correctness. Finally, we directly validated our approach by comparing it to the confirm-at-end strategy using a within-subjects study with 48 participants. Results show that 81\% of participants (39 of 48) preferred our intermediate confirmation strategy over the confirm-at-end baseline. We also found that our strategy reduced the average task completion time by 13.54\% compared to the baseline. 

This paper makes four key contributions: 

1) A formative study identifying user expectations for alternative confirmation strategies compared to the current confirm-at-end approach, and revealing a recurring CDCR pattern;

2) A decision-theoretic model for checkpoint placement that minimizes user interaction while ensuring correct task execution;

3) An empirical evaluation with 48 participants demonstrating both strong user preference (81\%) for intermediate confirmation and a 13.54\% reduction in task completion time relative to the confirm-at-end baseline;

4) A discussion framing our model not only as a confirmation scheduling tool but also as a design probe to surface potential directions for user-supervised systems.



\section{Related Work}
HCI research has primarily examined how to give users control, while reliability engineering has modeled when to intervene. We unify these perspectives by using mathematical models grounded in user behavior to determine when human control should occur in long-horizon agentic tasks.
\subsection{Human Control in AI Agents}
The tension between system autonomy and direct user control has been a recurring theme in HCI for decades \cite{horvitz1999principles, shneiderman1997direct, lieberman1997autonomous, maes1995agents}. Current AI agents are highly automated, integrating context, tools, and memory to solve complex real-world tasks \cite{thrun2002probabilistic}. Yet these reasoning and tool use systems are highly error-prone, yielding only ~30\% accuracy on 10-step tasks \cite{wei2022chain, dziri2023faith}. Errors propagate like a snowball—growing exponentially and causing downstream failures \cite{zhao2024retrieval, ji2023survey}. Agent self-correction is inconsistent and does not scale with task complexity \cite{kadavath2022language}. These challenges highlight the necessity of human control.

To support such control, HCI researchers have proposed a wide range of tools for human involvement in agent planning and execution \cite{shao2024collaborative, das2024vime, zhao2024lightva, suh2024luminate, feng2024cocoa, lawley2023val, epperson2025interactive, kazemitabaar2024improving}. For instance, Cocoa builds on computational notebook interfaces to support co-planning and co-execution of research tasks \cite{feng2024cocoa}, AGDebugger enables pausing and editing agent behaviors in programming \cite{epperson2025interactive}, VAL lets users instruct agents to execute tasks in user-specified ways \cite{lawley2023val}, and Step-Phasewise decomposes complex analysis workflows for verification \cite{kazemitabaar2024improving}. These systems focus on how to give users control, but not when that control should be invoked in multi-step tasks.

If we want to study "when", we inevitably need mathematical models. Beyond the agentic AI context, prior work has examined the timing of user engagement through mathematical models \cite{horvitz1999principles, cockburn2022probability, todi2021adapting, zhao2022bayesian}. For instance, \citet{horvitz1999principles} proposed expected-utility models for deciding whether to invoke services, and \citet{cockburn2022probability} showed that users’ probability-weighting biases shape when they engage or ignore assistive features. However, these works focus on single-event decisions and are not applicable to agentic AI, where tasks involve many interdependent steps. To capture the long-horizon nature of agent execution, we need to observe users’ sequential behaviors around errors and build mathematical models that identify the optimal moments for user confirmation from a global, system-level perspective.

\subsection{Error Prevention and Recovery Models}
Outside HCI, reliability engineering offers inspiration. Researchers have long developed mathematical models to optimize inspection intervals for complex systems such as railway tracks, drainage infrastructure, and manufacturing plants \cite{ten2013optimizing, iren2014cost, zhou2015preventive}. These studies show how inspection strategies affect whole-system performance, considering dependencies such as workforce allocation, spare parts, and interdependent subsystems \cite{dekker1997review, wang2002survey, de2020review}. The central challenge is to balance preventive maintenance (early inspections to prevent failures) with corrective maintenance (recovering after failures occur), minimizing long-run costs by trading off inspection labor and detection probabilities against repair and downtime losses \cite{assis2021dynamic, wang2012overview, barlow1960optimum, taghipour2012optimal}.
For AI agents, the analogy is clear: deciding when to confirm execution steps likewise balances preventive checks with corrective recovery. But unlike engineering, the relevant cost is not purely economic—it is shaped by user experience, including confirmation burden and error recovery effort.

HCI work on errors has largely focused on how to prevent or repair mistakes. Examples include encouraging users to issue longer or repeated commands for accuracy \cite{lam2022more}, designing conversations that make error recovery more resilient \cite{ashktorab2019resilient}, referring to third-party apps for additional information during repair \cite{li2020multi}, and studying whether error detection should be system- or user-initiated \cite{levy2021assessing}. These studies enrich our understanding of error handling, but they do not provide a mathematical account of when intervention should occur across multi-step tasks.

In summary, while prior HCI work has richly explored how to give users control and how to prevent or repair errors, it has rarely asked when such intervention should occur across long-horizon tasks. Similarly, reliability engineering provides rigorous models of inspection timing, but they optimize purely for economic cost in physical systems. Our contribution is to bridge these two traditions: we introduce a mathematical perspective on the \textit{when} question in agentic AI, grounded not in labor or repair costs, but in the dynamics of user experience—balancing confirmation burden against recovery effort. This shift reframes human–agent interaction as a problem of scheduling when to switch initiative between agents acting and users verifying and correcting, opening new space for both theoretical modeling and practical design of adaptive confirmation strategies.



\begin{figure*}
  \centering
  \includegraphics[width=1\linewidth]{task_type.png}
  \caption{Representative Task Types and AI Agent Interface Example}
  \Description{Left: Task taxonomy generated from 12 representative benchmarks. Right: Example of an AI agent interface.}
  \label{fig:task type and interface}
\end{figure*}

\section{Grounding Confirmation Modeling in Agentic AI Contexts}
Before building the model, we conducted a formative study to address two research questions:
\begin{itemize}
    \item RQ1: How do users monitor AI agents during task execution? Is there any consistency in user behavior patterns across different task types and interface designs?
    \item RQ2: When should confirmation happen to support users in effectively monitoring AI agents?
\end{itemize}
To ensure that the observed user behaviors and the resulting modeling are grounded in realistic usage contexts, we first reviewed representative task benchmarks and existing AI agents (Section 3.1). Based on this review, we then designed a test environment that combines real-world tasks with deployed agentic systems to examine RQ1 and RQ2 (Section 3.2).

\subsection{Agentic AI Task Types and Existing Systems}
\subsubsection{Task Types}
To understand the landscape of agentic AI tasks and guide the design of our user study, we reviewed 12 benchmarks that cover a wide spectrum of real-world scenarios \cite{xie2024osworld, valmeekam2023planbench, xie2024travelplanner, zhou2023webarena, xu2024theagentcompanybenchmarkingllmagents, ma2024spreadsheetbench, jing2024dsbench, yao2022webshop, wei2025browsecomp, wang2024zsc, chang2024partnr}. These benchmarks are grounded in user-generated platforms such as StackOverflow, Reddit, and other public-facing systems, where users frequently ask how to complete complex or ambiguous tasks. Each benchmark contains around 300 pieces of task data. From these benchmarks, we identified four broad application domains—Office, Daily Life, Virtual Environment, and Mixed Workflow—each encompassing several representative task types (Figure~\ref{fig:task type and interface}, left). These domains provided the basis for selecting tasks in our formative study. Despite advances in LLMs, current models such as GPT-4 or Claude 3.5 still struggle to reliably complete long-horizon tasks. Reported success rates for benchmarks like OSWorld \cite{xie2024osworld}, SpreadsheetBench \cite{ma2024spreadsheetbench}, and TheAgentCompany \cite{xu2024theagentcompanybenchmarkingllmagents} often fall between 30\% and 60\%, with failure modes including incorrect tool use, misinterpreted instructions, or skipped steps. This relatively high error rate underscores the need for human-in-the-loop oversight, where users monitor and intervene during task execution.

\subsubsection{Existing Systems}
We tested tasks from the benchmarks mentioned above in nine agentic systems \cite{openai2025agent, anthropic2024claude, manus2024, flowith2024, crewai2024, layla2024, pokeeai2024, genspark2024, browseruse2024}. We also reviewed recent surveys on agent interface design \cite{luera2024survey, zheng2022ux, weisz2024design}. Typically, these agents operate within remote desktop environments, where the agent perceives its environment through periodic screenshots. The vision-language model (VLM) reasons about the current state and user instructions, and then performs primitive actions, such as mouse clicks and keyboard inputs. For instance, when the agent perceives a hotel booking website, it will enter the destination city on that website automatically, as shown in the middle of Figure~\ref{fig:task type and interface}. Over multiple steps, the agent decomposes a high-level goal into a sequence of fine-grained interface operations.

From the interaction perspective, we observed three common UI patterns across existing agents: (1) in some agents, each step is a clickable node linked to its corresponding screenshot \cite{manus2024, flowith2024, crewai2024, layla2024}; (2) in others, the agent’s activity or reasoning process is visually embedded within the screenshot, with a sliding progress bar to navigate through the log \cite{openai2025agent, anthropic2024claude, genspark2024}; and (3) some agents provide only screenshot playback without an activity log \cite{pokeeai2024, browseruse2024}. These UI patterns focus only on passively displaying the agent’s execution, but they do not proactively engage users to verify correctness. Given the low accuracy rates of current agents and the high cost of errors, confirmation mechanisms are exceedingly rare: most systems execute entire task sequences autonomously and pause only when they require missing information (e.g., payment credentials) \cite{openai2025agent}. Most errors (e.g., failed login, page not loading) are not detected in real time, and agents simply stop without notifying the user or recovering \cite{kadavath2022language}. These gaps motivated us to explore how to schedule intermediate confirmations in ways that ensure timely user intervention without introducing unnecessary overhead.


\subsection{Designing the Formative Study Environment}
Drawing from prior benchmarks and real-world agent deployments, we selected representative tasks to cover a broad spectrum of domains. By using real-world tasks and existing systems, we aim to replicate users’ authentic experiences, ensuring that subsequent model development would be grounded in realistic usage contexts.

\subsubsection{Environment Setup}
We selected five tasks to represent a broad spectrum of domains, including everyday activities, office workflows, and virtual environments. To cover the three UI patterns described above, we selected one representative agent for each pattern and randomly assigned tasks to these agents:
\begin{itemize}
    \item Agent 1 Manus \cite{manus2024} (node-based navigation) — (1) Travel planning: booking hotels, flights, taxis, and restaurants; (2) Shopping cart: adding ingredients for making pizza to an online shopping cart.
    \item Agent 2 ChatGPT Agent \cite{openai2025agent} (embedded sequential navigation) — (3) Document processing: converting scanned invoices into a spreadsheet and emailing it to the reimbursement office; (4) Image editing: removing unrelated objects, changing backgrounds, and replacing objects in an image.
    \item Agent 3 Browser Use \cite{browser_use2024} (screenshot-only playback) — (5) Video game play: navigating through different terrains, boiling ingredients, and plating meals in the game Overcooked.
\end{itemize}
This diversity of agents and tasks allows us to examine whether consistent patterns of user supervision behavior emerge across different UI designs and task types.

\subsubsection{Procedure}
We recruited eight participants for a 45-minute user study, structured in two phases. Each participant completed all five tasks, allowing us to compare how both task type and confirmation frequency influenced user behavior within the same individuals.

Phase 1 — RQ1 (User Behavior): In Task 1, participants were instructed to check every step to familiarize themselves with the agent. In Tasks 2–5, they confirmed results only at the end of task execution. Participants were asked to think aloud, verbalizing their judgments about correctness and describing how they located errors in the sequence of steps. Phase 2 — RQ2 (Confirmation Preferences): We conducted a semi-structured interview to explore participants’ desired confirmation frequency, comparing their experiences with end-of-task versus step-by-step confirmation, and eliciting their ideal frequency. 

All sessions were recorded and transcribed. We conducted a thematic analysis \cite{inbook, beyer1999contextual}, in which two researchers collaboratively coded the transcripts. We refined emerging themes through the lens of our research questions (RQ1 \& 2).



\section{Formative Study Findings}

\begin{comment}
    change the order of rq1 and rq2
\end{comment}
Our thematic analysis of the formative study data identified a dominant Confirmation–Diagnosis–Correction–Redo (CDCR) pattern where participants dealt with AI agent errors by reviewing the execution history from beginning to end. This strategy was evident across diverse task types and UI designs. We also observed some exceptions to this pattern (RQ1). For RQ2, participants expressed a desire for confirmation strategies beyond the prevalent ``confirm-at-the-end'' norm, suggesting opportunities for intermediate checkpoints. These insights informed the framing of our confirmation frequency model (section 4.3).

\subsection{Confirm-at-end is not enough}
Seven out of eight participants expressed their dissatisfaction with the current confirm-at-end strategy. As P4 noted:
\begin{quote}
``I did not like confirming at the end. Because the results were so far off from what I originally intended, that was a waste of time to do. Looking through the trace to try and find the error, I see that it was having a lot of other problems. If it had stopped there, maybe I could have clarified what my intentions were when I said the thing. Instead of wasting its time and getting something that was so far off from what I actually wanted to get.  And if we could maybe confirm at least a little bit more frequently, I think that would really help my confidence in the agent, [and improve] my communication [with the agent].''
\end{quote}
These findings suggest the dominant ``confirm-at-the-end'' approach should be reconsidered. Introducing intermediate confirmations can improve efficiency by allowing users to ensure the agent stays on the right track. Confirming every step is impractical: ``If I have to check every step my agent is making, I'd rather do it myself. (P6)'' The challenge, therefore, is to strike a balance between the fixed time cost of reviewing every agent action and the expected time cost of needing to confirm task completion at the end and then possibly undertake a long post-hoc diagnostic and redo process if the agent failed. We discuss our model for balancing between these two extremes with checkpoint scheduling in the next section. 


Scheduling confirmations requires consideration of two components: (1) the time costs for each part of the Confirmation\hyp{}Diagnosis\hyp{}Correction\hyp{}Redo (CDCR) cycle, including the time associated with waiting for the agent to generate actions, and (2) the agent's accuracy. We found that when accuracy was high, users were fine with less frequent confirmation. 


\subsection{Common Usage Patterns}
\subsubsection{Common Agent Errors} 
The most frequent agent error in our formative study was a mismatch between the activity described by the agent (i.e., the action listed) and what was actually executed (i.e., the screenshot). For example, in the travel planning task, the activity list indicated ``click July 4th''as the start date, but the screenshot showed July 9th selected. Another frequent issue was that the agent did not always execute actions in full alignment with the user’s intended goal.

\subsubsection{The CDCR Pattern}
We identified a recurring Confirmation\hyp{}Diagnosis\hyp{}Correction\hyp{}Redo (CDCR) sequence (Figure \ref{fig:CDCR}). After the agent completes a task, participants typically \textit{confirm} that the final screenshot matches their expected result. If not, they first \textit{diagnose} the error by scanning the entire execution history, often skimming the activity list or sliding through the screenshot recording. Once they get the whole picture of the execution process, they scan the execution sequence step-by-step from the beginning, reviewing the activity descriptions, reasoning traces, and associated screenshots along the way. After locating the erroneous step, participants usually \textit{correct} the agent by telling it how to fix the mistake. In the final \textit{redo} stage, participants wait for the agent to replan from the erroneous step to the end based on their correction.

\subsubsection{Exceptions to the CDCR pattern}
We found that interface differences did not fundamentally change user behavior. However, variations in task type and user familiarity revealed several edge cases where this pattern did not hold. (1) Ambiguous confirmability: In some tasks, there was no clear ground truth for what counted as an error. For example, in hotel booking, the agent might choose a location that was slightly more expensive but offered better transportation access. While arguably suboptimal, this choice was not objectively wrong. (2) Non-linear diagnosis: Some users skipped intermediate steps during diagnosis. For example, P2, who was highly experienced with Overcooked, noticed the player placing soup on the wrong pad and bypassed the cooking phase, jumping directly to the serving phase. (3) Partial correction: When errors were minor and unlikely to affect subsequent planning, users preferred to download the result file and fix the error themselves rather than having the agent re-execute the task.

\subsection{Implications for Model Scope and Applicability}


Building on the recurring confirmation–diagnosis–correction–redo pattern, our model is grounded in three key assumptions:
\begin{enumerate}
    \item The task has a clear distinction between correct and incorrect outputs.
    \item The user is not deeply familiar with the task, and thus needs to review from the beginning to understand the agent’s actions. 
    \item Errors can propagate, and without timely correction, they affect subsequent similar actions.
\end{enumerate}
While factors such as task structure and user preference are important for real-world deployment, our current modeling abstracts away these user-specific and task-structural influences to focus on the core trade-off between confirmation cost and error diagnosis cost. This simplification positions our model within a broader landscape of supervision patterns, while leaving these additional dimensions for further investigation.

Our intended application scenarios involve long, multi-step tasks with repeated sub-actions where each sub-action carries a non-trivial probability of error. For example, in image editing workflows involving repeated operations such as removing objects, changing backgrounds, and cropping images, each step is similar in structure but may still fail unpredictably. Our model aims to identify confirmation strategies that maximize efficiency in such contexts.

\section{Modeling Confirmation as a Minimum Time Scheduling Problem}

\begin{figure*}
    \centering
    \includegraphics[width=1\linewidth]{example_interval.png}
    \caption{An example of the interval between the last verified state and the next confirmation point. The colors denote the correctness of the plan execution at that point, but this correctness is only observed when the user inspects them (denoted by solid circles). When the user finds an error in state $s_j$, they must go back to $s_i$ and check each state one at a time until they find the first error. This requires them to inspect $m-1$ states. Once the first error is identified, the agent then has to redo $j-m+1$ actions. Note, the dashed circles signify that the correctness of a state is not known until diagnosis.}
    \label{fig:example-interval}
\end{figure*}

This section introduces a formal model of confirmation scheduling. Our goal is to determine \emph{when} the system should prompt the user to confirm correctness during a multi-step agentic task so as to minimize the user time needed to execute a whole task correctly. In other words, at each step, our system decides whether the agent should proceed autonomously or whether the user should be asked to step in and supervise. We proceed in three stages: (i) we define the environment (states, actions, transitions); (ii) we derive the expected time of a single confirmation interval using the \textsc{CDCR} decomposition; and (iii) we present a dynamic programming algorithm to identify the minimum time checkpoints. Finally, we describe how our model's parameters can be personalized.

\subsection{Problem Formulation}
We can formulate the scheduling of confirmations as a stochastic decision process:
\[
\mathcal{P} = (\mathcal{S}, \mathcal{A}, \mathcal{T}),
\]
where $\mathcal{S}$ is the state space, $\mathcal{A}$ is the action space, and $\mathcal{T}$ is the transition function. 

\subsubsection{State space $\mathcal{S}$} Given an agent action sequence of length N, there will be $2 \times (N+1)$ states in the space. Among these, $2N$ states correspond to the points after each action has been executed, which we denote as $s_i = \texttt{correct}$ or $s_i = \texttt{incorrect}$ for $i\in {1,\dots,N}$. The correctness value of $s_i$ is determined by whether all actions up to step $i$ have been executed correctly by the agent.
For example, if the system is in state $s_3=\texttt{correct}$, this means that the agent's first three actions have been executed correctly. In addition, there are two special states that are always correct: (1) an initial state $s_0$, representing the point before any actions have been executed, and (2) a terminal state $s_{\texttt{done}}$, indicating successful completion of the agent's entire action sequence.

At any given point, it knows that all agent actions up to the last user-verified state are correct, but it does not know whether the states beyond this point are correct (see Figure~\ref{fig:example-interval}).


\subsubsection{Action space $\mathcal{A}$.} In any given state $s_i$, the system can choose between one of two actions. It can either execute action $i+1$ from the agent's plan (\texttt{execute}) or request confirmation from the user (\texttt{verify}). Note that the system does not choose the agent's action (e.g., button clicks, scrolling, tool use), but only whether to proceed autonomously with execution or seek user verification. In this way, the system determines whether the agent or the user takes the lead. When the system is in state $s_0$ it can only choose to \texttt{execute} because this state is always correct by default, so confirmation is unnecessary. Similarly, when the system is in the final state $s_{N}$, it can only \texttt{verify} to transition to the terminal $s_{\texttt{done}}$ state because there are no more agent actions to execute.


\subsubsection{Transition function $\mathcal{T}$.}
The transition function depends on the two action options chosen by the system. We distinguish two cases:

\textit{(1) Execute.} When the system chooses to $\texttt{execute}$ it will transition from one state to another according to the following transition probabilities:
\begin{itemize}
    \item $p(s_{i}=\texttt{correct}|s_{i-1}=\texttt{correct}, \texttt{execute}) = p_{a_i}$
    \item $p(s_{i}=\texttt{incorrect}|s_{i-1}=\texttt{correct}, \texttt{execute}) = 1- p_{a_i}$
    \item $p(s_{i}=\texttt{correct}|s_{i-1}=\texttt{incorrect}, \texttt{execute}) = 0$
    \item $p(s_{i}=\texttt{incorrect}|s_{i-1}=\texttt{incorrect}, \texttt{execute}) = 1$
\end{itemize}
\noindent where $1 < i \leq N$ and $p_{a_i}$ corresponds to the probability that the agent will execute action $i$ from its plan correctly. These transitions also capture the cumulative probability of errors; i.e., once an agent makes a mistake, all subsequent states will be incorrect. Note that there is no \texttt{execute} transition out of the $s_N$ state. An important point here is that the correctness of all states after the last user verified state is not directly observable to the system; it becomes known only when the user inspects them. To compute the probability that any given state $s_i$ is correct, the system must weigh the transition from the previous state by its beliefs about the correctness of the prior state: 
$p(s_i=\texttt{correct}) = p(s_{i-1}=\texttt{correct})p(s_i=\texttt{correct}|s_{i-1}=\texttt{correct}, \texttt{execute})$. By chaining over transitions, we can compute the probability that any state $j$ is correct given the last user verified state $i$, where $i < j$: $p(s_j=\texttt{correct}|s_i=\texttt{correct}) = \prod_{m=i+1}^j p_{a_m}$. Note, we do not need to consider actions before the last user verified step because we already know they are all correct.


\textit{(2) Verify.} When the system chooses $\texttt{verify}$, the user inspects the current state $s_j$ and determines if it is \texttt{correct}.  
\begin{itemize}
    \item If $s_j=\texttt{correct}$ and $j<N$, then $p(s_m=\texttt{correct})$ is set to 1 for all $m < j$ and system remains in $s_j$.
    % The system proceeds to decide on action $a_{i+1}$.  
    \item If $s_j=\texttt{correct}$ and $j=N$, the system transitions to the terminal state $s_{\texttt{done}}$.
    \item If $s_j=\texttt{incorrect}$, then the user goes back to the last verified state and starts inspecting each of the following states to diagnose where the first error state occurs. If the last state verified by the user was $s_i=\texttt{correct}$ and during diagnosis the user determines that the first incorrect state is $s_m=\texttt{incorrect}$, then the system sets $p(s_k=\texttt{correct}) = 1$ for all $i<k<m$ and transitions to state $s_{m-1}$ (e.g., see Figure \ref{fig:example-interval}). The probability that the first incorrect state will occur at state $s_m$ follows a geometric distribution:     $p(\text{first \texttt{incorrect} at }s_m|s_i=\texttt{correct}) = 
    p(s_{i+1}=\texttt{correct}, \cdots, s_{m-1}=\texttt{correct}, s_m=\texttt{incorrect})
    = \left(\prod_{k=i+1}^{m-1} p_{a_k}\right)\,(1-p_{a_m})$.
\end{itemize}



\begin{figure*}[h]
    \centering
    \includegraphics[width=1\linewidth]{dp_diagram.png}
    \caption{Dynamic programming algorithm and results for an agent action sequence of length $N=5$ with action success probabilities $p=[0.7,0.7,0.9,0.85,0.85]$. 
    %Costs are parameterized as $t_{\text{confirm}}[0:5]=1$, $t_{\text{diagnose}}[0:5]=1$, and $t_{\text{redo}}[0:5]=1$. 
    Costs are parameterized as $t_{\text{confirm}}[k] = 1$, 
$t_{\text{diagnose}}[k] = 1$, and $t_{\text{redo}}[k] = 1$, 
for all $k \in \{0,\dots,N\}$.
    Each cell $[\textit{start}, \textit{end}]$ specifies the expected user time to correctly execute the rest of the action sequence from $start$ to $N$, if the system has a verification checkpoint at state \textit{end}. 
    White checkmarks indicate the $next\_ckpt$ locations for each $start$. The right-hand annotations summarize the resulting checkpoint policy; }
    \label{fig:dp}
\end{figure*}

\subsection{Time Cost}
The system measures cost in terms of \textit{user interaction time} (seconds), 
following the Confirmation--Diagnosis--Correction--Redo (CDCR) pattern. 
We first derive the expected time cost from step $i$ to its next confirmation point $j$. The interval begins at a state $s_i=\texttt{correct}$, which corresponds to the last state that the user has inspected and determined to be correct. The system then repeatedly chooses to
\texttt{execute} actions until it reaches state $s_j$. At this point the system chooses to ask the user to \texttt{verify} the correctness of $s_j$.

First, we denote the per-step time components in the CDCR pattern as follows:
\begin{itemize}
\item $t_{\text{confirm}}^i$: time for the user to confirm whether state $s_i$ is correct. 
  Concretely, the user compares the screenshot of state with the agent’s activity list 
  (i.e., the sequence of executed actions) to verify correctness.

\item $t_{\text{diagnose}}^m$: the time required to inspect each intermediate state $s_m$ by looking at its screenshot and the description of the single agent action executed immediately before.\footnote{This is different from $t^i_{\text{confirm}}$, which requires evaluating a sequence of agent actions rather than a single agent action.}
%    \item $t_{\text{diagnose}}$: the time required to inspect an intermediate state, via its screenshot and corresponding agent action description.

  \item $t_{\text{correct}}^m$: time for the user to specify what went wrong with action $a_m$ and 
  provide updated instructions to the agent to correct the mistake.

  \item $t_{\text{redo}}^m$: time for the agent to re-execute the action $a_m$ after the correction.
\end{itemize}

If we know that the user has verified all the states up to $s_i$, then we can compute the expected time to go from $i$ to $N$ if the next checkpoint is at $j$, where $i<j \leq N$, as:


\begin{align}
T[i,j] &= t_{\text{confirm}}^{j}
+ p(s_j=\texttt{correct}\mid s_i=\texttt{correct}) \min_{k} T[j,k] \notag\\
&\quad + \sum_{m=i+1}^{j} p(\text{first \texttt{incorrect} at } s_m\mid s_i=\texttt{correct})\notag\\
&\quad
\Bigl(
  \sum_{k=i+1}^{m} t_{\text{diagnose}}^{k} + t_{\text{correct}}^{m}
  + \sum_{k=m}^{j} t_{\text{redo}}^{k} + \min_{k} T[m-1,k]
\Bigr)
\end{align}





The first term corresponds to the time it takes the user to confirm state $s_j$. The second term corresponds to the case where the state $s_j$ is correct, which occurs with probability $p(s_j=\texttt{correct}|s_i=\texttt{correct})$. If this happens, then the expected remaining time is recursively defined as the expected time needed to traverse from $j$ to $N$ when the system chooses the next checkpoint $k$ such that this time is minimized (denoted as $\min_k T[j,k]$). The third term corresponds to the cases where $s_j=\texttt{incorrect}$. If this happens, we sum over all the possible locations of the first error weighted by their probability. For each first incorrect state $s_m$, the  expected remaining time is composed of four parts: the time needed to diagnose the error, the time needed to correct the mistake, the time needed to redo the incorrect steps, and the expected time needed to go from $m$ to $N$ if the system chooses the next checkpoint $k$ such that this time is minimized.




\subsection{Dynamic Programming Checkpoint Solver}


Given the recurrent format of T[i, j], we developed a dynamic programming algorithm to efficiently identify the best verification locations that minimize the total user time, see Figure \ref{fig:dp}. 

The algorithm starts at the end of the sequence, since there is always a verification at the end. It then works backwards populating the values of $T[:,:]$ according to the previously defined formula.\footnote{We drop $t_{\text{correct}}^m$ (the time needed for the user to instruct the agent on how to fix $a_m$) from the dynamic programming update because all errors need to be fixed no matter when verifications happen, so it turns out that dropping this term does not change the checkpoint locations.}  At each iteration, the algorithm computes the best next checkpoint for each starting location $i$, denoted as $next\_ckpt[i]$.

The populated $next\_ckpt$ vector can be used to determine the best checkpoint locations. For a given starting location $i$ (which assumes $s_i=\texttt{correct}$), $next\_ckpt[i]$ specifies the location of the best next checkpoint. For the example in Figure \ref{fig:dp}, when starting at state 1, the best next checkpoint is at state 2, if starting at state 2, the best next checkpoint is at 3, and finally when starting at state 3, the next best checkpoint is at state 5.

If the user determines that a state is incorrect during one of the checkpoint verifications, then they follow the CDCR procedure. This might result in them ending up in an earlier state. If the agent action sequence does not change, then the precomputed $T$ and $next\_ckpt$ can be reused without computation. If, however, the action sequence changes, then these need to be recomputed.


\subsection{Modeling Personalization Capability}
In our system, we have both agent-side and human-side quantities. On the agent side, evaluating agent performance is an active research area, with benchmarks reporting both task accuracy \cite{xie2024osworld, valmeekam2023planbench, xie2024travelplanner, zhou2023webarena, xu2024theagentcompanybenchmarkingllmagents, ma2024spreadsheetbench} and execution time \cite{zhang2025optimizing, zhang2025leveraging, liu2023llm, waytowich2024atari}. Accordingly, in our model, the step-level accuracy $p_{a_i}$ and the execution time $t_{\text{redo}}^k$ can be grounded in these benchmark results. For example, in image editing tasks, benchmarks report the success rate of invoking a given tool as well as the average time required to execute a single operation.

On the human side, our framework allows both population-level and per-user parameterization of timing variables such as $t_{\text{confirm}}^k$ and $t_{\text{diagnose}}^k$. These quantities can capture hierarchical effects, with individual reaction times treated as adjustments to the population average and refined as more user data become available. In this paper, however, we focus on the population-level model and leave personalization to future work.


\section{Model Verification Study Design}
We compared two confirmation formats: intermediate confirmation, guided by our minimum time scheduling model, and confirm-at-end, the baseline in current AI agents. The study tested whether intermediate confirmation improves efficiency and user experience.

\subsection{Simulated Environment}
Real AI agents vary widely in execution time and outcome quality. We built a simulated environment that fixed execution time and accuracy for each task, allowing us to isolate confirmation frequency as the primary factor. The user interface used a clickable-node tracking design (shown in Figure~\ref{fig:env}), which participants in our formative study reported as helpful for navigation and understanding task progress. We selected three representative task domains: shopping cart management, image editing, and the Overcooked game. These cover three broad categories: daily life, work-related, and virtual tasks, while also differing substantially in execution time, tool usage, and workflow complexity. Error cases were drawn from real agent failures observed in prior experiments: in shopping, the agent failed to replace an out-of-stock item; in image editing, the agent applied an incorrect mask that prevented object removal; and in Overcooked, the agent grounded to the wrong ingredient dispenser (e.g., selecting the tomato instead of the onion).

\begin{figure*}
    \centering
    \includegraphics[width=1\linewidth]{env.png}
    \caption{Example of the simulated shopping task interface with intermediate confirmation, the activity list on the left and the current action screenshot on the right, enabling users to compare and identify the first error.}
    \label{fig:env}
\end{figure*}

\subsection{Within-Subjects Design}
We employed a within-subjects design, where each participant completed two trials (intermediate and end) in all three domains. To control for potential biases, we introduced three layers of counterbalancing: task order, confirmation comparison order, and error location. 1) Task order. Each domain was treated as a block. We counterbalanced block order using a $3 \times 6$ Williams design, ensuring that each task domain appeared equally often in each position and that domain sequences were balanced across participants. 2) Confirmation strategy order. Within each block, the two confirmation formats for the same task were presented back-to-back. They were set to the same error location. To reduce familiarity effects, each trial followed the same structure but used different content (e.g., one shopping trial involved office supplies, another fitness items). Participants encountered different confirmation orders across the three blocks, and intermediate-first and end-first trials were balanced at the group level. 3) Error location. Error positions were sampled from early, middle, or late steps; shopping tasks included only early and late because these tasks were shorter. To prevent participants from anticipating that every trial would contain an error, we had each participant do one no-error trial. The no-error trial was implemented with fresh task content, assigned to one of the three task types, and counterbalanced across participants in both task type and confirmation format.


\subsection{Procedure}
The study began with a short tutorial explaining the agent background and confirmation methods. Each trial started with a task goal, and the Overcooked block included an additional introduction to the game rules. Each participant experienced 7 trials, including 2 (confirmation formats) * 3 (task types) + 1 no-error trial. After the trials, participants completed a post-study survey to capture participants' reflections on the confirmation strategies. The survey included open-ended questions pertaining to perceived task success, preferred confirmation style, and overall impressions of different confirmation strategies. It also included Likert-scale items assessing participants' perceptions of intermediate checkpoints (e.g., whether they saved time, reduced burden, or disrupted flow).


\subsection{Participants}
We recruited 48 participants from a university community (18 female, 30 male). Regarding LLM usage, 41 participants (85\%) reported using LLMs frequently (weekly or daily). In terms of familiarity with AI agents, $7$ (15\%) had never used them, $9$ (18\%) were beginners, $19$ (39\%) were intermediate users, and $13$ (28\%) were advanced users. Each session lasted about 15 minutes, and participants received \$5 compensation.


\subsection{Variable Determination} 
The confirmation frequency in our model is determined by four parameters: $t_{\text{confirm}}$, $t_{\text{diagnose}}$, $t_{\text{redo}}$, and the accuracy rate. In this study, each task domain is associated with its own set of parameters. (1) Confirmation and diagnosis times were estimated from a pilot study with random checkpoints. In the main study, these averages were used as fixed inputs, with recalibration performed every 5–10 participants to prevent drift. We deliberately avoided per-user online adaptation, since stable personalization requires long-horizon usage data beyond the scope of this lab setting. (2) Execution time was estimated through formative agent runs and prior literature \cite{zhang2025optimizing, zhang2025leveraging, liu2023llm, waytowich2024atari}. We set $t_{\text{redo}} = 20$s for shopping (dominated by network and screenshot analysis), $10$s for image editing (external tool calls), and $10$s for Overcooked (agent reasoning plus game state processing). (3) For accuracy, we set fixed per-step action correctness probabilities for each task domain, derived from formative agent runs: 
$p_{\text{shopping}} = 87.5\%$, 
$p_{\text{image editing}} = 91\%$, and 
$p_{\text{overcooked}} = 93\%$. These values were applied uniformly across all steps within a task. 




\section{Model Verification Results}
In this section, we verify our model through both quantitative and qualitative analyses. While the quantitative results demonstrate what improvements intermediate confirmation achieves, the qualitative feedback explains why users value it and offers insights for future directions we discuss next.

\subsection{Quantitative Results}
\begin{figure*}
    \centering
    \includegraphics[width=1\linewidth]{quan.png}
    \caption{Average completion time under different confirmation strategies. Left: Intermediate confirmation reduced total time compared to end confirmation. Right: Effects vary by error location—large gains for early errors, minimal for mid, and slight cost for late errors. Error bars show 95\% CIs.}
    \label{fig:qual}
\end{figure*}

Across all task domains, our model reduced the completion time of a full task trial (8--16 steps) by \textbf{13.54\%}, corresponding to a savings of 35.84 seconds per task, compared to the baseline (End Confirmation) condition, $t(143) = 5.52, p < 0.001$. The baseline condition with end confirmation required an average of 264.70 seconds (95\% CI [253.43, 275.98]), while intermediate confirmation using our model required 228.86 seconds (95\% CI [219.73, 237.99]). Breaking down by task domain, the reductions were \textbf{17.44\%} (47.60s) for Shopping ($t(47) = 5.25, p < 0.001$; our model mean = 225.33s, 95\% CI [212.78, 237.88]), \textbf{7.46\%} (17.41s) for Image Editing ($t(47) = 2.34, p = 0.024$; our model mean = 215.85s, 95\% CI [203.42, 228.28]), and \textbf{15.64\%} (49.03s) for Overcooked Cooking ($t(40) = 2.14, p = 0.044$; our model mean = 264.51s, 95\% CI [236.71, 292.30]). 

The effect of confirmation frequency varied by error location. When errors occurred early, intermediate confirmation substantially reduced total time ($\approx 29\%$). For mid-task errors, the benefit was minimal ($\approx 2\%$), while for late errors, intermediate checkpoints slightly increased time ($+4.5\%$). This indicates that the value of intermediate confirmation is greatest for early error detection and diminishes as errors occur later in the task.

Across all tasks, the average completion time was $246.78$s (95\% CI [236.58, 256.98]), consisting of $13\%$ ($32.08$s, 95\% CI [28.29, 32.95]) confirmation, $9\%$ ($21.85$s, 95\% CI [16.99, 26.74]) diagnosis, and $78\%$ ($195.55$s, 95\% CI [189.30, 201.79]) execution. Shifting from end to intermediate confirmation meant spending more time on confirmations, but this was offset by large savings in diagnosis ($-17\%$) and redo ($-22\%$). Moreover, the average time per confirmation decreased by $38\%$, as users only needed to consider the most recent steps rather than reasoning over the entire task history when verifying correctness. Intermediate confirmation reduced the mental tax of each checkpoint. 

Beyond efficiency, reducing redo ($-22\%$) also cuts repeated LLM calls, lowering both financial costs and the energy footprint of unnecessary computation \cite{patterson2021carbon, luccioni2023estimating}. Although environmental impact is not our primary focus, this highlights an additional benefit: confirmation strategies that limit rework may also support more sustainable AI use.


\subsection{Qualitative Findings}


\subsubsection{Overall preference.} In our post-study survey, \textbf{81\%} of participants (39 of 48) preferred the modeled intermediate confirmation strategy, compared to 15\% (7 participants) who preferred end confirmation and 4\% (2 participants) with no preference (Fig.~\ref{fig:qual}, right). The Likert responses (Figure.~\ref{fig:qual}, left) further support this preference: most participants agreed that intermediate confirmation reduced errors (81\%), saved time (71\%), lowered cognitive burden (63\%), and helped prevent harmful errors (83\%). Notably, 77\% of participants disagreed that intermediate checkpoints disrupted their task flow.  As P18 noted, \textit{``I chose intermediate confirmation because it provides a balance between efficiency and reliability. By allowing the system to confirm at selected checkpoints, I can catch potential mistakes earlier instead of waiting until the very end, while still avoiding the constant interruptions of confirming every small step. This makes the process feel smoother but still safe.''} Overall, our modeled confirmation frequency achieved the intended goal of balancing user control with agent automation.


\subsubsection{Impact of Diagnosis.}Although diagnosis time accounted for only a small portion of total completion time in the quantitative study, participants emphasized that it had a strong influence on their experience. When errors were revealed only at the end, users reported feeling disoriented, as they had to review all steps at once without clear guidance: \textit{``If I can only confirm at the end, I feel confused about how to deal with the result, because I don’t know which step to check.''} (P39) Timely checkpoints minimized the effort needed for diagnosis, making error handling less confusing and less mentally taxing for users.

\subsubsection{Confirm at Early Stage.} The value of confirmations depends on the stage at which they occur. Based on our quantitative analysis, the largest time savings of intermediate confirmation arose when the agent made mistakes early in the task. The user survey further reinforced this finding from an experiential perspective, highlighting the importance of early-stage confirmation for both efficiency and usability. Each time a user and agent collaborate on a new task, the interaction is a dynamic process of gradually deepening mutual understanding: the agent gains a clearer understanding of the user’s needs, while the user gains a better sense of the agent’s performance and progressively builds trust in its automatic execution for that specific task. As P9 remarked:
\begin{quote}
``With an employee I haven't fully developed trust in yet, I would want to see intermediate steps in their work just to check that they've done it right. Perhaps if the employee has shown me that they can do the tasks correctly several times without errors, they can just show me their work at the end. I can't trust that it's going to do the task right all the way to the end quite yet, so I want it to show me that it has done the intermediate steps right before getting to a wrong final product.``   
\end{quote}
This perspective is also reflected in our model. We allow each step to have its own probability $p_{a_i}$, which can initially be set conservatively so that early-stage confirmations are more frequent. As users catch and correct early errors, $p_{a_i}$ increases, and the model dynamically decreases confirmation frequency. This adaptive behavior was not fully captured in our simulated environment. We will discuss in the next section how such dynamics could be more realistically incorporated.
\begin{figure*}
    \centering
    \includegraphics[width=1\linewidth]{qual.jpg}
    \caption{Participant Likert scale scores for the questions asked in our post-study survey (Left).Preferred confirmation style(Right)}
    \label{fig:qual}
\end{figure*}

\subsubsection{Subjective perception.} Participants’ preferences for confirmation strategy also reflected their perceptions of task importance and difficulty. For important tasks, they preferred more frequent checks (e.g., ``something elaborate like trip planning”), while for trivial tasks (e.g., ``adding groceries to my cart”), they preferred confirmation only at the end (P20). Perceived difficulty also mattered: users expected easier tasks to require fewer confirmations. However, user perceptions often diverged from reality. For tasks they considered simple, participants expected fewer confirmations, yet were frustrated when the agent still made errors. For example, agents frequently failed in the shopping cart task, but many users saw this task as trivial. Since our model calculates confirmation frequency from actual error rates, it correctly prescribed more frequent checkpoints. While efficient in theory, such frequent confirmations may feel disruptive when they conflict with user expectations. In the next section, we discuss how confirmation strategies can be better aligned with user perceptions in future systems.

\section{Discussion}
Our findings suggest that confirmation strategies should not be seen as a binary choice between user-driven supervision and agent-driven autonomy. Our decision-theoretic model strikes a mixed-initiative balance, where responsibility is flexibly shared between human and agent. To move toward this vision, we consider two complementary directions: (1) what else to model—expanding beyond efficiency to capture subjective costs and trust, and (2) how else to design—re-imagining confirmation as an interaction paradigm rather than a discrete interruption.

\subsection{Beyond Efficiency: Subjective Costs and Trust}
Our current model treats time as the only cost factor when determining the best confirmation points. However, in real-world applications, user decisions about when to confirm are influenced by broader considerations. As noted in Section 7.2.4, subjective perceptions play a critical role; here, we expand on this point by examining two key dimensions: real-world cost and user-perceived reliability.

\subsubsection{Real-world cost} In our CDCR model, cost in the final redo stage is defined only from the agent's perspective (i.e., the time required for the agent to recover from an error). In practice, however, human users often bear additional costs that extend far beyond execution time. Errors may lead to financial loss (e.g., rescheduling fees for a wrong flight), emotional or social consequences (e.g., accidentally sending an email to the wrong recipient), or even irreversible risks (e.g., permanently deleting a file). Future research should explore how these human-centered costs affect confirmation strategies. For high-stakes actions, mandatory checkpoints should be enforced regardless of the system’s estimated accuracy. For general actions, the replanning term can be extended into a multi-dimensional cost function that balances a heterogeneous variety of costs, such as time, money, emotional costs, and risks. 
%\begin{equation}
%\pi^* = \text{argmin} \mathbb{E} [a*Time+b*Money+c*Emotion+d*Risk+...] 
%\end{equation}

Reliability engineering provides inspiration, with established methods for integrating economic trade-offs, workforce allocation, and interdependent subsystems into cost functions \cite{ten2013optimizing, iren2014cost, zhou2015preventive}, which could guide extensions of our model.

\subsubsection{Perceived reliability} Users’ willingness to confirm is shaped not only by actual system accuracy but also by how reliable they believe the agent to be, based on their subjective experiences and agent framing \cite{lee2004trust, parasuraman2010complacency, shneiderman2020human}. For instance, positive framing can inflate perceived reliability and encourage overuse even when this harms performance \cite{cockburn2020framing, quinn2020loss, quinn2016bad}. Our own studies echoed these findings. First-time users who were unfamiliar with the agent tended to perceive it as less reliable, which naturally led them to confirm more frequently. Conversely, in tasks they considered simple, users expected that fewer confirmations would suffice. Yet this expectation often diverged from reality when the agent’s actual accuracy was lower than anticipated.

Such mismatches highlight the need for balancing actual system accuracy with user perception. Our model is already equipped to support this: by assigning each action its own correctness probability $p_{a_i}$ based on past performance data, we can introduce a weighting scheme that blends objective error rates with subjective perceptions, ensuring that confirmation policies remain both efficient and aligned with user expectations.

\subsection{Designing Confirmation as Interaction}
To better balance user control with agent automation, confirmation should be treated not only as a scheduling problem but also as an interaction design challenge. From our study, we identify three insights that illustrate new ways of structuring this interaction.

\subsubsection{Proactive engagement} In our two user studies, current agent UIs left users in a passive ``wait state'' during execution: with no meaningful hooks to monitor or steer the run, participants turned to other tasks and disengaged from supervision. This disengagement is risky and can lead to not only time and resources waste (discussed in our paper), but also harmful or unsafe actions (e.g., privacy breaches, opening phishing links, manipulative suggestions) \cite{alberts2024computers, bansal2024challenges, amodei2016concrete}. Current systems like GPT Agents \cite{openai2025agent} exemplify a reactive model, asking for clarification only when they cannot proceed, which is insufficient to prevent harmful actions. As agents become increasingly autonomous, effective mixed-initiative interaction requires what \citet{tennenhouse2000proactive} envisioned: moving humans from being ``in the loop'' to supervisory and policy-making roles. This shift—from human-centered to human-supervised design—requires agents to proactively solicit user supervision. Our model adds one dimension of proactivity by placing more checkpoints, but proactive supervision should extend beyond correctness to encompass broader forms of risk.

\subsubsection{Goal Alignment} Our findings on error location highlight the importance of prevention. Errors introduced early often stem from poorly communicated goals rather than faulty execution \cite{bansal2024challenges}. While most HCI work emphasizes co-execution or error repair \cite{feng2024cocoa, arawjo2024chainforge, epperson2025interactive}, we identify a design opportunity in early-stage goal alignment. Before execution, agents could present a structured plan for user review, drawing on concepts from formal languages that use rigorous specifications to evaluate whether task goals are satisfied \cite{aeronautiques1998pddl, shivashankar2013hierarchical, nau2003shop2}. As noted in Section 7.2.3, our model assigns each agent action its own correctness probability $p_{a_i}$, which could be adapted to support such early-stage goal alignment interfaces, enabling more frequent and dynamic checkpoints.

\subsubsection{Right amount of information} As tasks grow longer, participants in our study found confirmation increasingly difficult: extended linear logs made it hard to locate issues, leaving users disoriented. This highlights a core design question—what level of detail should agents expose for effective supervision? Current UIs that present step-by-step activity lists with screenshots are often too low-level, overwhelming users with detail without offering structure. Future interfaces should balance informativeness and abstraction, allowing users to monitor actions without being lost in execution traces. One promising direction is to summarize and restructure agent actions hierarchically, so that users can reason about plans at multiple levels of abstraction rather than scanning long linear logs. If such hierarchical structuring is adopted, our CDCR pattern remains intact, but diagnosis time may decrease in a logarithmic fashion—since users could locate errors by checking higher-level chunks first (e.g., binary or ternary splits) rather than scanning all low-level steps.


\section{Conclusion}
This paper introduced a decision-theoretic model for determining when users should confirm agent actions in long-horizon tasks. Building on insights from a formative study that surfaced the recurring CDCR pattern and dissatisfaction with confirm-at-end strategies, we conducted a controlled experiment with 48 participants and found that intermediate confirmation reduced task completion time by 13.54\% and was strongly preferred by 81\% of participants. Looking ahead, our findings suggest that confirmation should be framed not as a binary trade-off between autonomy and oversight, but as a mixed-initiative design opportunity, pointing to future work on integrating efficiency, trust, and interaction design to create more reliable and user-supervised agentic systems.

\bibliographystyle{ACM-Reference-Format}
\bibliography{sources.bib}
\end{document}
 
% \documentclass[sigconf]{acmart}
%\documentclass[manuscript]{acmart}
% \documentclass[manuscript,review,anonymous]{acmart}
\usepackage{tabularx}
\usepackage{array}
\usepackage{float}
\usepackage{hyphenat}

\usepackage[linesnumbered,ruled,vlined]{algorithm2e}
\usepackage{comment}
\setlength{\tabcolsep}{10pt}
\renewcommand{\arraystretch}{1.5}

\AtBeginDocument{%
  \providecommand\BibTeX{{%
    Bib\TeX}}}

\setcopyright{none}
\copyrightyear{2024}
\acmYear{2024}
\acmDOI{XXXXXXX.XXXXXXX}

\begin{document}

\title{When Should Users Check? A Decision-Theoretic Model of Confirmation Frequency in Multi-Step AI Agent Tasks}
% other options: 
% \title{Modeling Confirmation Frequency in Agentic AI: Balancing User Burden and Rollback Cost}

\author{Jieyu Zhou}
\affiliation{%
  \institution{Georgia Institute of Technology}
  \city{Atlanta}
  \state{Georgia}
  \country{USA}
}
\email{jzhou625@gatech.edu}

\author{Aryan Roy}
\affiliation{%
  \institution{Georgia Institute of Technology}
  \city{Atlanta}
  \state{Georgia}
  \country{USA}
}
\email{aroy389@gatech.edu}

\author{Sneh Gupta}
\affiliation{%
  \institution{Georgia Institute of Technology}
  \city{Atlanta}
  \state{Georgia}
  \country{USA}
}
\email{sgupta852@gatech.edu}

\author{Daniel Weitekamp}
\affiliation{%
  \institution{Georgia Institute of Technology}
  \city{Atlanta}
  \state{Georgia}
  \country{USA}
}
\email{dweitekamp3@gatech.edu}

\author{Christopher J. MacLellan}
\affiliation{%
  \institution{Georgia Institute of Technology}
  \city{Atlanta}
  \state{Georgia}
  \country{USA}
}
\email{cmaclell@gatech.edu}

\begin{teaserfigure}
  \centering
  \includegraphics[width=\textwidth]{diagram.png}
  \caption{Confirmation–Diagnosis–Correction–Redo (CDCR) Pattern. Based on the expected time of user interaction under the CDCR pattern, our model decides whether the agent should proceed autonomously or prompt the user to step in and supervise.}
  \Description{Confirmation–Diagnosis–Correction–Redo (CDCR)}
  \label{fig:CDCR}
\end{teaserfigure}

\begin{abstract}
Existing AI agents typically execute multi-step tasks autonomously and only allow user confirmation at the end. During execution, users have little control, making the confirm-at-end approach brittle: a single error can cascade and force a complete restart. Confirming every step avoids such failures, but imposes tedious overhead. Balancing excessive interruptions against costly rollbacks remains an open challenge. We address this problem by modeling confirmation as a minimum time scheduling problem. We conducted a formative study with eight participants, which revealed a recurring \textit{Confirmation–Diagnosis–Correction–Redo (CDCR)} pattern in how users monitor errors. Based on this pattern, we developed a decision-theoretic model to determine time-efficient confirmation point placement. We then evaluated our approach using a within-subjects study where 48 participants monitored AI agents and repaired their mistakes while executing tasks. Results show that 81\% of participants preferred our intermediate confirmation approach over the confirm-at-end approach used by existing systems, and task completion time was reduced by 13.54\%. 
\end{abstract}
\maketitle


\keywords{AI agents, mathematical model, error handling}






\section{Introduction}
With the advancement of Artificial Intelligence (AI), Agentic AI has emerged as a concept describing agents that assist humans in carrying out complex tasks \cite{durante2024agent, pallagani2024prospects, huang2024understanding, wang2024survey, yao2023react, sumers2023cognitive, schick2023toolformer, wei2022chain}. These AI agents have been deployed across diverse domains, including multi-modal devices \cite{rabbit2023, chang2024partnr}, web-based systems \cite{huq2023s,li2023using, li2017sugilite}, and game controls \cite{wang2023voyager,wang2023describe,lawley2023val}. Their rapid proliferation has reignited a long-standing HCI debate \cite{horvitz1999principles, shneiderman1997direct, lieberman1997autonomous, maes1995agents}: should agents operate under direct human supervision and control or should they operate in a mostly autonomous fashion?

Most commercial AI agents today prioritize autonomy, executing tasks end-to-end with little to no user oversight \cite{openai2025agent, anthropic2024claude, manus2024}. Yet these agents remain highly error-prone, achieving only ~30\% accuracy on multi-step benchmarks \cite{ma2024spreadsheetbench, xie2024osworld, xu2024theagentcompanybenchmarkingllmagents}, largely due to their limited capacity for generalizable reasoning and planning \cite{stechly2024chain, valmeekam2024planning}. Errors are especially costly in long-horizon agentic tasks, which often span 30 steps and 10–30 minutes of execution \cite{hu2024dawn, bonatti2024windows}. A single mistake can cascade into complete task failure, forcing users to redo the entire process—doubling execution time and, through repeated API calls, adding both monetary costs \cite{kapoor2024ai, feng2024cocoa, zhang2025leveraging} and increasing the agent's carbon footprint \cite{patterson2022carbon, luccioni2023estimating}. This situation highlights the need to engage user inspections over agent execution.

The timing of such inspections is essentially a tradeoff between autonomy and control. At one extreme, too much user control (e.g., the user confirming after every step) prevents error propagation but imposes substantial interaction overhead, reducing efficiency to the level of manual execution or even worse. At the other extreme, too much automation (e.g., the user only confirming at the end) reduces the burden on the user but increases the risks of errors and costs associated with re-executing the tasks. The challenge is deciding when user checks should occur to balance the efficiency of automation with the safeguards of user control.

Prior HCI research has largely focused on \textbf{how} to provide user control, through visualizing the reasoning process, editing execution results, and providing error feedback \cite{arawjo2024chainforge, epperson2025interactive, kazemitabaar2024improving, feng2024cocoa, lam2022more, ashktorab2019resilient, li2020multi, levy2021assessing}. Few studies examine \textbf{when} such control should occur. A notable exception is Horvitz’s seminal work on expected utility models for single-event decisions \cite{horvitz1999principles}, but these approaches do not extend to long-horizon tasks, where sequential dependencies and cumulative costs shape overall outcomes.  

In this paper, we address the open question of \textbf{when user checks should occur in long-horizon agentic tasks}. We first review representative agent benchmarks and deployed systems and present the results of a formative study we conducted with eight participants. The study revealed dissatisfaction with the confirm-at-end strategy used by existing agentic platforms. It also surfaced a recurring \textbf{Confirmation–Diagnosis–Correction–Redo (CDCR)} pattern employed by users to supervise agents. Building on these insights, we developed a decision-theoretic model that identifies periodic confirmation points that minimize total expected task completion time while ensuring task correctness. Finally, we directly validated our approach by comparing it to the confirm-at-end strategy using a within-subjects study with 48 participants. Results show that 81\% of participants (39 of 48) preferred our intermediate confirmation strategy over the confirm-at-end baseline. We also found that our strategy reduced the average task completion time by 13.54\% compared to the baseline. 

This paper makes four key contributions: 

1) A formative study identifying user expectations for alternative confirmation strategies compared to the current confirm-at-end approach, and revealing a recurring CDCR pattern;

2) A decision-theoretic model for checkpoint placement that minimizes user interaction while ensuring correct task execution;

3) An empirical evaluation with 48 participants demonstrating both strong user preference (81\%) for intermediate confirmation and a 13.54\% reduction in task completion time relative to the confirm-at-end baseline;

4) A discussion framing our model not only as a confirmation scheduling tool but also as a design probe to surface potential directions for user-supervised systems.



\section{Related Work}
HCI research has primarily examined how to give users control, while reliability engineering has modeled when to intervene. We unify these perspectives by using mathematical models grounded in user behavior to determine when human control should occur in long-horizon agentic tasks.
\subsection{Human Control in AI Agents}
The tension between system autonomy and direct user control has been a recurring theme in HCI for decades \cite{horvitz1999principles, shneiderman1997direct, lieberman1997autonomous, maes1995agents}. Current AI agents are highly automated, integrating context, tools, and memory to solve complex real-world tasks \cite{thrun2002probabilistic}. Yet these reasoning and tool use systems are highly error-prone, yielding only ~30\% accuracy on 10-step tasks \cite{wei2022chain, dziri2023faith}. Errors propagate like a snowball—growing exponentially and causing downstream failures \cite{zhao2024retrieval, ji2023survey}. Agent self-correction is inconsistent and does not scale with task complexity \cite{kadavath2022language}. These challenges highlight the necessity of human control.

To support such control, HCI researchers have proposed a wide range of tools for human involvement in agent planning and execution \cite{shao2024collaborative, das2024vime, zhao2024lightva, suh2024luminate, feng2024cocoa, lawley2023val, epperson2025interactive, kazemitabaar2024improving}. For instance, Cocoa builds on computational notebook interfaces to support co-planning and co-execution of research tasks \cite{feng2024cocoa}, AGDebugger enables pausing and editing agent behaviors in programming \cite{epperson2025interactive}, VAL lets users instruct agents to execute tasks in user-specified ways \cite{lawley2023val}, and Step-Phasewise decomposes complex analysis workflows for verification \cite{kazemitabaar2024improving}. These systems focus on how to give users control, but not when that control should be invoked in multi-step tasks.

If we want to study "when", we inevitably need mathematical models. Beyond the agentic AI context, prior work has examined the timing of user engagement through mathematical models \cite{horvitz1999principles, cockburn2022probability, todi2021adapting, zhao2022bayesian}. For instance, \citet{horvitz1999principles} proposed expected-utility models for deciding whether to invoke services, and \citet{cockburn2022probability} showed that users’ probability-weighting biases shape when they engage or ignore assistive features. However, these works focus on single-event decisions and are not applicable to agentic AI, where tasks involve many interdependent steps. To capture the long-horizon nature of agent execution, we need to observe users’ sequential behaviors around errors and build mathematical models that identify the optimal moments for user confirmation from a global, system-level perspective.

\subsection{Error Prevention and Recovery Models}
Outside HCI, reliability engineering offers inspiration. Researchers have long developed mathematical models to optimize inspection intervals for complex systems such as railway tracks, drainage infrastructure, and manufacturing plants \cite{ten2013optimizing, iren2014cost, zhou2015preventive}. These studies show how inspection strategies affect whole-system performance, considering dependencies such as workforce allocation, spare parts, and interdependent subsystems \cite{dekker1997review, wang2002survey, de2020review}. The central challenge is to balance preventive maintenance (early inspections to prevent failures) with corrective maintenance (recovering after failures occur), minimizing long-run costs by trading off inspection labor and detection probabilities against repair and downtime losses \cite{assis2021dynamic, wang2012overview, barlow1960optimum, taghipour2012optimal}.
For AI agents, the analogy is clear: deciding when to confirm execution steps likewise balances preventive checks with corrective recovery. But unlike engineering, the relevant cost is not purely economic—it is shaped by user experience, including confirmation burden and error recovery effort.

HCI work on errors has largely focused on how to prevent or repair mistakes. Examples include encouraging users to issue longer or repeated commands for accuracy \cite{lam2022more}, designing conversations that make error recovery more resilient \cite{ashktorab2019resilient}, referring to third-party apps for additional information during repair \cite{li2020multi}, and studying whether error detection should be system- or user-initiated \cite{levy2021assessing}. These studies enrich our understanding of error handling, but they do not provide a mathematical account of when intervention should occur across multi-step tasks.

In summary, while prior HCI work has richly explored how to give users control and how to prevent or repair errors, it has rarely asked when such intervention should occur across long-horizon tasks. Similarly, reliability engineering provides rigorous models of inspection timing, but they optimize purely for economic cost in physical systems. Our contribution is to bridge these two traditions: we introduce a mathematical perspective on the \textit{when} question in agentic AI, grounded not in labor or repair costs, but in the dynamics of user experience—balancing confirmation burden against recovery effort. This shift reframes human–agent interaction as a problem of scheduling when to switch initiative between agents acting and users verifying and correcting, opening new space for both theoretical modeling and practical design of adaptive confirmation strategies.



\begin{figure*}
  \centering
  \includegraphics[width=1\linewidth]{task_type.png}
  \caption{Representative Task Types and AI Agent Interface Example}
  \Description{Left: Task taxonomy generated from 12 representative benchmarks. Right: Example of an AI agent interface.}
  \label{fig:task type and interface}
\end{figure*}

\section{Grounding Confirmation Modeling in Agentic AI Contexts}
Before building the model, we conducted a formative study to address two research questions:
\begin{itemize}
    \item RQ1: How do users monitor AI agents during task execution? Is there any consistency in user behavior patterns across different task types and interface designs?
    \item RQ2: When should confirmation happen to support users in effectively monitoring AI agents?
\end{itemize}
To ensure that the observed user behaviors and the resulting modeling are grounded in realistic usage contexts, we first reviewed representative task benchmarks and existing AI agents (Section 3.1). Based on this review, we then designed a test environment that combines real-world tasks with deployed agentic systems to examine RQ1 and RQ2 (Section 3.2).

\subsection{Agentic AI Task Types and Existing Systems}
\subsubsection{Task Types}
To understand the landscape of agentic AI tasks and guide the design of our user study, we reviewed 12 benchmarks that cover a wide spectrum of real-world scenarios \cite{xie2024osworld, valmeekam2023planbench, xie2024travelplanner, zhou2023webarena, xu2024theagentcompanybenchmarkingllmagents, ma2024spreadsheetbench, jing2024dsbench, yao2022webshop, wei2025browsecomp, wang2024zsc, chang2024partnr}. These benchmarks are grounded in user-generated platforms such as StackOverflow, Reddit, and other public-facing systems, where users frequently ask how to complete complex or ambiguous tasks. Each benchmark contains around 300 pieces of task data. From these benchmarks, we identified four broad application domains—Office, Daily Life, Virtual Environment, and Mixed Workflow—each encompassing several representative task types (Figure~\ref{fig:task type and interface}, left). These domains provided the basis for selecting tasks in our formative study. Despite advances in LLMs, current models such as GPT-4 or Claude 3.5 still struggle to reliably complete long-horizon tasks. Reported success rates for benchmarks like OSWorld \cite{xie2024osworld}, SpreadsheetBench \cite{ma2024spreadsheetbench}, and TheAgentCompany \cite{xu2024theagentcompanybenchmarkingllmagents} often fall between 30\% and 60\%, with failure modes including incorrect tool use, misinterpreted instructions, or skipped steps. This relatively high error rate underscores the need for human-in-the-loop oversight, where users monitor and intervene during task execution.

\subsubsection{Existing Systems}
We tested tasks from the benchmarks mentioned above in nine agentic systems \cite{openai2025agent, anthropic2024claude, manus2024, flowith2024, crewai2024, layla2024, pokeeai2024, genspark2024, browseruse2024}. We also reviewed recent surveys on agent interface design \cite{luera2024survey, zheng2022ux, weisz2024design}. Typically, these agents operate within remote desktop environments, where the agent perceives its environment through periodic screenshots. The vision-language model (VLM) reasons about the current state and user instructions, and then performs primitive actions, such as mouse clicks and keyboard inputs. For instance, when the agent perceives a hotel booking website, it will enter the destination city on that website automatically, as shown in the middle of Figure~\ref{fig:task type and interface}. Over multiple steps, the agent decomposes a high-level goal into a sequence of fine-grained interface operations.

From the interaction perspective, we observed three common UI patterns across existing agents: (1) in some agents, each step is a clickable node linked to its corresponding screenshot \cite{manus2024, flowith2024, crewai2024, layla2024}; (2) in others, the agent’s activity or reasoning process is visually embedded within the screenshot, with a sliding progress bar to navigate through the log \cite{openai2025agent, anthropic2024claude, genspark2024}; and (3) some agents provide only screenshot playback without an activity log \cite{pokeeai2024, browseruse2024}. These UI patterns focus only on passively displaying the agent’s execution, but they do not proactively engage users to verify correctness. Given the low accuracy rates of current agents and the high cost of errors, confirmation mechanisms are exceedingly rare: most systems execute entire task sequences autonomously and pause only when they require missing information (e.g., payment credentials) \cite{openai2025agent}. Most errors (e.g., failed login, page not loading) are not detected in real time, and agents simply stop without notifying the user or recovering \cite{kadavath2022language}. These gaps motivated us to explore how to schedule intermediate confirmations in ways that ensure timely user intervention without introducing unnecessary overhead.


\subsection{Designing the Formative Study Environment}
Drawing from prior benchmarks and real-world agent deployments, we selected representative tasks to cover a broad spectrum of domains. By using real-world tasks and existing systems, we aim to replicate users’ authentic experiences, ensuring that subsequent model development would be grounded in realistic usage contexts.

\subsubsection{Environment Setup}
We selected five tasks to represent a broad spectrum of domains, including everyday activities, office workflows, and virtual environments. To cover the three UI patterns described above, we selected one representative agent for each pattern and randomly assigned tasks to these agents:
\begin{itemize}
    \item Agent 1 Manus \cite{manus2024} (node-based navigation) — (1) Travel planning: booking hotels, flights, taxis, and restaurants; (2) Shopping cart: adding ingredients for making pizza to an online shopping cart.
    \item Agent 2 ChatGPT Agent \cite{openai2025agent} (embedded sequential navigation) — (3) Document processing: converting scanned invoices into a spreadsheet and emailing it to the reimbursement office; (4) Image editing: removing unrelated objects, changing backgrounds, and replacing objects in an image.
    \item Agent 3 Browser Use \cite{browser_use2024} (screenshot-only playback) — (5) Video game play: navigating through different terrains, boiling ingredients, and plating meals in the game Overcooked.
\end{itemize}
This diversity of agents and tasks allows us to examine whether consistent patterns of user supervision behavior emerge across different UI designs and task types.

\subsubsection{Procedure}
We recruited eight participants for a 45-minute user study, structured in two phases. Each participant completed all five tasks, allowing us to compare how both task type and confirmation frequency influenced user behavior within the same individuals.

Phase 1 — RQ1 (User Behavior): In Task 1, participants were instructed to check every step to familiarize themselves with the agent. In Tasks 2–5, they confirmed results only at the end of task execution. Participants were asked to think aloud, verbalizing their judgments about correctness and describing how they located errors in the sequence of steps. Phase 2 — RQ2 (Confirmation Preferences): We conducted a semi-structured interview to explore participants’ desired confirmation frequency, comparing their experiences with end-of-task versus step-by-step confirmation, and eliciting their ideal frequency. 

All sessions were recorded and transcribed. We conducted a thematic analysis \cite{inbook, beyer1999contextual}, in which two researchers collaboratively coded the transcripts. We refined emerging themes through the lens of our research questions (RQ1 \& 2).



\section{Formative Study Findings}

\begin{comment}
    change the order of rq1 and rq2
\end{comment}
Our thematic analysis of the formative study data identified a dominant Confirmation–Diagnosis–Correction–Redo (CDCR) pattern where participants dealt with AI agent errors by reviewing the execution history from beginning to end. This strategy was evident across diverse task types and UI designs. We also observed some exceptions to this pattern (RQ1). For RQ2, participants expressed a desire for confirmation strategies beyond the prevalent ``confirm-at-the-end'' norm, suggesting opportunities for intermediate checkpoints. These insights informed the framing of our confirmation frequency model (section 4.3).

\subsection{Confirm-at-end is not enough}
Seven out of eight participants expressed their dissatisfaction with the current confirm-at-end strategy. As P4 noted:
\begin{quote}
``I did not like confirming at the end. Because the results were so far off from what I originally intended, that was a waste of time to do. Looking through the trace to try and find the error, I see that it was having a lot of other problems. If it had stopped there, maybe I could have clarified what my intentions were when I said the thing. Instead of wasting its time and getting something that was so far off from what I actually wanted to get.  And if we could maybe confirm at least a little bit more frequently, I think that would really help my confidence in the agent, [and improve] my communication [with the agent].''
\end{quote}
These findings suggest the dominant ``confirm-at-the-end'' approach should be reconsidered. Introducing intermediate confirmations can improve efficiency by allowing users to ensure the agent stays on the right track. Confirming every step is impractical: ``If I have to check every step my agent is making, I'd rather do it myself. (P6)'' The challenge, therefore, is to strike a balance between the fixed time cost of reviewing every agent action and the expected time cost of needing to confirm task completion at the end and then possibly undertake a long post-hoc diagnostic and redo process if the agent failed. We discuss our model for balancing between these two extremes with checkpoint scheduling in the next section. 


Scheduling confirmations requires consideration of two components: (1) the time costs for each part of the Confirmation\hyp{}Diagnosis\hyp{}Correction\hyp{}Redo (CDCR) cycle, including the time associated with waiting for the agent to generate actions, and (2) the agent's accuracy. We found that when accuracy was high, users were fine with less frequent confirmation. 


\subsection{Common Usage Patterns}
\subsubsection{Common Agent Errors} 
The most frequent agent error in our formative study was a mismatch between the activity described by the agent (i.e., the action listed) and what was actually executed (i.e., the screenshot). For example, in the travel planning task, the activity list indicated ``click July 4th''as the start date, but the screenshot showed July 9th selected. Another frequent issue was that the agent did not always execute actions in full alignment with the user’s intended goal.

\subsubsection{The CDCR Pattern}
We identified a recurring Confirmation\hyp{}Diagnosis\hyp{}Correction\hyp{}Redo (CDCR) sequence (Figure \ref{fig:CDCR}). After the agent completes a task, participants typically \textit{confirm} that the final screenshot matches their expected result. If not, they first \textit{diagnose} the error by scanning the entire execution history, often skimming the activity list or sliding through the screenshot recording. Once they get the whole picture of the execution process, they scan the execution sequence step-by-step from the beginning, reviewing the activity descriptions, reasoning traces, and associated screenshots along the way. After locating the erroneous step, participants usually \textit{correct} the agent by telling it how to fix the mistake. In the final \textit{redo} stage, participants wait for the agent to replan from the erroneous step to the end based on their correction.

\subsubsection{Exceptions to the CDCR pattern}
We found that interface differences did not fundamentally change user behavior. However, variations in task type and user familiarity revealed several edge cases where this pattern did not hold. (1) Ambiguous confirmability: In some tasks, there was no clear ground truth for what counted as an error. For example, in hotel booking, the agent might choose a location that was slightly more expensive but offered better transportation access. While arguably suboptimal, this choice was not objectively wrong. (2) Non-linear diagnosis: Some users skipped intermediate steps during diagnosis. For example, P2, who was highly experienced with Overcooked, noticed the player placing soup on the wrong pad and bypassed the cooking phase, jumping directly to the serving phase. (3) Partial correction: When errors were minor and unlikely to affect subsequent planning, users preferred to download the result file and fix the error themselves rather than having the agent re-execute the task.

\subsection{Implications for Model Scope and Applicability}


Building on the recurring confirmation–diagnosis–correction–redo pattern, our model is grounded in three key assumptions:
\begin{enumerate}
    \item The task has a clear distinction between correct and incorrect outputs.
    \item The user is not deeply familiar with the task, and thus needs to review from the beginning to understand the agent’s actions. 
    \item Errors can propagate, and without timely correction, they affect subsequent similar actions.
\end{enumerate}
While factors such as task structure and user preference are important for real-world deployment, our current modeling abstracts away these user-specific and task-structural influences to focus on the core trade-off between confirmation cost and error diagnosis cost. This simplification positions our model within a broader landscape of supervision patterns, while leaving these additional dimensions for further investigation.

Our intended application scenarios involve long, multi-step tasks with repeated sub-actions where each sub-action carries a non-trivial probability of error. For example, in image editing workflows involving repeated operations such as removing objects, changing backgrounds, and cropping images, each step is similar in structure but may still fail unpredictably. Our model aims to identify confirmation strategies that maximize efficiency in such contexts.

\section{Modeling Confirmation as a Minimum Time Scheduling Problem}

\begin{figure*}
    \centering
    \includegraphics[width=1\linewidth]{example_interval.png}
    \caption{An example of the interval between the last verified state and the next confirmation point. The colors denote the correctness of the plan execution at that point, but this correctness is only observed when the user inspects them (denoted by solid circles). When the user finds an error in state $s_j$, they must go back to $s_i$ and check each state one at a time until they find the first error. This requires them to inspect $m-1$ states. Once the first error is identified, the agent then has to redo $j-m+1$ actions. Note, the dashed circles signify that the correctness of a state is not known until diagnosis.}
    \label{fig:example-interval}
\end{figure*}

This section introduces a formal model of confirmation scheduling. Our goal is to determine \emph{when} the system should prompt the user to confirm correctness during a multi-step agentic task so as to minimize the user time needed to execute a whole task correctly. In other words, at each step, our system decides whether the agent should proceed autonomously or whether the user should be asked to step in and supervise. We proceed in three stages: (i) we define the environment (states, actions, transitions); (ii) we derive the expected time of a single confirmation interval using the \textsc{CDCR} decomposition; and (iii) we present a dynamic programming algorithm to identify the minimum time checkpoints. Finally, we describe how our model's parameters can be personalized.

\subsection{Problem Formulation}
We can formulate the scheduling of confirmations as a stochastic decision process:
\[
\mathcal{P} = (\mathcal{S}, \mathcal{A}, \mathcal{T}),
\]
where $\mathcal{S}$ is the state space, $\mathcal{A}$ is the action space, and $\mathcal{T}$ is the transition function. 

\subsubsection{State space $\mathcal{S}$} Given an agent action sequence of length N, there will be $2 \times (N+1)$ states in the space. Among these, $2N$ states correspond to the points after each action has been executed, which we denote as $s_i = \texttt{correct}$ or $s_i = \texttt{incorrect}$ for $i\in {1,\dots,N}$. The correctness value of $s_i$ is determined by whether all actions up to step $i$ have been executed correctly by the agent.
For example, if the system is in state $s_3=\texttt{correct}$, this means that the agent's first three actions have been executed correctly. In addition, there are two special states that are always correct: (1) an initial state $s_0$, representing the point before any actions have been executed, and (2) a terminal state $s_{\texttt{done}}$, indicating successful completion of the agent's entire action sequence.

At any given point, it knows that all agent actions up to the last user-verified state are correct, but it does not know whether the states beyond this point are correct (see Figure~\ref{fig:example-interval}).


\subsubsection{Action space $\mathcal{A}$.} In any given state $s_i$, the system can choose between one of two actions. It can either execute action $i+1$ from the agent's plan (\texttt{execute}) or request confirmation from the user (\texttt{verify}). Note that the system does not choose the agent's action (e.g., button clicks, scrolling, tool use), but only whether to proceed autonomously with execution or seek user verification. In this way, the system determines whether the agent or the user takes the lead. When the system is in state $s_0$ it can only choose to \texttt{execute} because this state is always correct by default, so confirmation is unnecessary. Similarly, when the system is in the final state $s_{N}$, it can only \texttt{verify} to transition to the terminal $s_{\texttt{done}}$ state because there are no more agent actions to execute.


\subsubsection{Transition function $\mathcal{T}$.}
The transition function depends on the two action options chosen by the system. We distinguish two cases:

\textit{(1) Execute.} When the system chooses to $\texttt{execute}$ it will transition from one state to another according to the following transition probabilities:
\begin{itemize}
    \item $p(s_{i}=\texttt{correct}|s_{i-1}=\texttt{correct}, \texttt{execute}) = p_{a_i}$
    \item $p(s_{i}=\texttt{incorrect}|s_{i-1}=\texttt{correct}, \texttt{execute}) = 1- p_{a_i}$
    \item $p(s_{i}=\texttt{correct}|s_{i-1}=\texttt{incorrect}, \texttt{execute}) = 0$
    \item $p(s_{i}=\texttt{incorrect}|s_{i-1}=\texttt{incorrect}, \texttt{execute}) = 1$
\end{itemize}
\noindent where $1 < i \leq N$ and $p_{a_i}$ corresponds to the probability that the agent will execute action $i$ from its plan correctly. These transitions also capture the cumulative probability of errors; i.e., once an agent makes a mistake, all subsequent states will be incorrect. Note that there is no \texttt{execute} transition out of the $s_N$ state. An important point here is that the correctness of all states after the last user verified state is not directly observable to the system; it becomes known only when the user inspects them. To compute the probability that any given state $s_i$ is correct, the system must weigh the transition from the previous state by its beliefs about the correctness of the prior state: 
$p(s_i=\texttt{correct}) = p(s_{i-1}=\texttt{correct})p(s_i=\texttt{correct}|s_{i-1}=\texttt{correct}, \texttt{execute})$. By chaining over transitions, we can compute the probability that any state $j$ is correct given the last user verified state $i$, where $i < j$: $p(s_j=\texttt{correct}|s_i=\texttt{correct}) = \prod_{m=i+1}^j p_{a_m}$. Note, we do not need to consider actions before the last user verified step because we already know they are all correct.


\textit{(2) Verify.} When the system chooses $\texttt{verify}$, the user inspects the current state $s_j$ and determines if it is \texttt{correct}.  
\begin{itemize}
    \item If $s_j=\texttt{correct}$ and $j<N$, then $p(s_m=\texttt{correct})$ is set to 1 for all $m < j$ and system remains in $s_j$.
    % The system proceeds to decide on action $a_{i+1}$.  
    \item If $s_j=\texttt{correct}$ and $j=N$, the system transitions to the terminal state $s_{\texttt{done}}$.
    \item If $s_j=\texttt{incorrect}$, then the user goes back to the last verified state and starts inspecting each of the following states to diagnose where the first error state occurs. If the last state verified by the user was $s_i=\texttt{correct}$ and during diagnosis the user determines that the first incorrect state is $s_m=\texttt{incorrect}$, then the system sets $p(s_k=\texttt{correct}) = 1$ for all $i<k<m$ and transitions to state $s_{m-1}$ (e.g., see Figure \ref{fig:example-interval}). The probability that the first incorrect state will occur at state $s_m$ follows a geometric distribution:     $p(\text{first \texttt{incorrect} at }s_m|s_i=\texttt{correct}) = 
    p(s_{i+1}=\texttt{correct}, \cdots, s_{m-1}=\texttt{correct}, s_m=\texttt{incorrect})
    = \left(\prod_{k=i+1}^{m-1} p_{a_k}\right)\,(1-p_{a_m})$.
\end{itemize}



\begin{figure*}[h]
    \centering
    \includegraphics[width=1\linewidth]{dp_diagram.png}
    \caption{Dynamic programming algorithm and results for an agent action sequence of length $N=5$ with action success probabilities $p=[0.7,0.7,0.9,0.85,0.85]$. 
    %Costs are parameterized as $t_{\text{confirm}}[0:5]=1$, $t_{\text{diagnose}}[0:5]=1$, and $t_{\text{redo}}[0:5]=1$. 
    Costs are parameterized as $t_{\text{confirm}}[k] = 1$, 
$t_{\text{diagnose}}[k] = 1$, and $t_{\text{redo}}[k] = 1$, 
for all $k \in \{0,\dots,N\}$.
    Each cell $[\textit{start}, \textit{end}]$ specifies the expected user time to correctly execute the rest of the action sequence from $start$ to $N$, if the system has a verification checkpoint at state \textit{end}. 
    White checkmarks indicate the $next\_ckpt$ locations for each $start$. The right-hand annotations summarize the resulting checkpoint policy; }
    \label{fig:dp}
\end{figure*}

\subsection{Time Cost}
The system measures cost in terms of \textit{user interaction time} (seconds), 
following the Confirmation--Diagnosis--Correction--Redo (CDCR) pattern. 
We first derive the expected time cost from step $i$ to its next confirmation point $j$. The interval begins at a state $s_i=\texttt{correct}$, which corresponds to the last state that the user has inspected and determined to be correct. The system then repeatedly chooses to
\texttt{execute} actions until it reaches state $s_j$. At this point the system chooses to ask the user to \texttt{verify} the correctness of $s_j$.

First, we denote the per-step time components in the CDCR pattern as follows:
\begin{itemize}
\item $t_{\text{confirm}}^i$: time for the user to confirm whether state $s_i$ is correct. 
  Concretely, the user compares the screenshot of state with the agent’s activity list 
  (i.e., the sequence of executed actions) to verify correctness.

\item $t_{\text{diagnose}}^m$: the time required to inspect each intermediate state $s_m$ by looking at its screenshot and the description of the single agent action executed immediately before.\footnote{This is different from $t^i_{\text{confirm}}$, which requires evaluating a sequence of agent actions rather than a single agent action.}
%    \item $t_{\text{diagnose}}$: the time required to inspect an intermediate state, via its screenshot and corresponding agent action description.

  \item $t_{\text{correct}}^m$: time for the user to specify what went wrong with action $a_m$ and 
  provide updated instructions to the agent to correct the mistake.

  \item $t_{\text{redo}}^m$: time for the agent to re-execute the action $a_m$ after the correction.
\end{itemize}

If we know that the user has verified all the states up to $s_i$, then we can compute the expected time to go from $i$ to $N$ if the next checkpoint is at $j$, where $i<j \leq N$, as:


\begin{align}
T[i,j] &= t_{\text{confirm}}^{j}
+ p(s_j=\texttt{correct}\mid s_i=\texttt{correct}) \min_{k} T[j,k] \notag\\
&\quad + \sum_{m=i+1}^{j} p(\text{first \texttt{incorrect} at } s_m\mid s_i=\texttt{correct})\notag\\
&\quad
\Bigl(
  \sum_{k=i+1}^{m} t_{\text{diagnose}}^{k} + t_{\text{correct}}^{m}
  + \sum_{k=m}^{j} t_{\text{redo}}^{k} + \min_{k} T[m-1,k]
\Bigr)
\end{align}





The first term corresponds to the time it takes the user to confirm state $s_j$. The second term corresponds to the case where the state $s_j$ is correct, which occurs with probability $p(s_j=\texttt{correct}|s_i=\texttt{correct})$. If this happens, then the expected remaining time is recursively defined as the expected time needed to traverse from $j$ to $N$ when the system chooses the next checkpoint $k$ such that this time is minimized (denoted as $\min_k T[j,k]$). The third term corresponds to the cases where $s_j=\texttt{incorrect}$. If this happens, we sum over all the possible locations of the first error weighted by their probability. For each first incorrect state $s_m$, the  expected remaining time is composed of four parts: the time needed to diagnose the error, the time needed to correct the mistake, the time needed to redo the incorrect steps, and the expected time needed to go from $m$ to $N$ if the system chooses the next checkpoint $k$ such that this time is minimized.




\subsection{Dynamic Programming Checkpoint Solver}


Given the recurrent format of T[i, j], we developed a dynamic programming algorithm to efficiently identify the best verification locations that minimize the total user time, see Figure \ref{fig:dp}. 

The algorithm starts at the end of the sequence, since there is always a verification at the end. It then works backwards populating the values of $T[:,:]$ according to the previously defined formula.\footnote{We drop $t_{\text{correct}}^m$ (the time needed for the user to instruct the agent on how to fix $a_m$) from the dynamic programming update because all errors need to be fixed no matter when verifications happen, so it turns out that dropping this term does not change the checkpoint locations.}  At each iteration, the algorithm computes the best next checkpoint for each starting location $i$, denoted as $next\_ckpt[i]$.

The populated $next\_ckpt$ vector can be used to determine the best checkpoint locations. For a given starting location $i$ (which assumes $s_i=\texttt{correct}$), $next\_ckpt[i]$ specifies the location of the best next checkpoint. For the example in Figure \ref{fig:dp}, when starting at state 1, the best next checkpoint is at state 2, if starting at state 2, the best next checkpoint is at 3, and finally when starting at state 3, the next best checkpoint is at state 5.

If the user determines that a state is incorrect during one of the checkpoint verifications, then they follow the CDCR procedure. This might result in them ending up in an earlier state. If the agent action sequence does not change, then the precomputed $T$ and $next\_ckpt$ can be reused without computation. If, however, the action sequence changes, then these need to be recomputed.


\subsection{Modeling Personalization Capability}
In our system, we have both agent-side and human-side quantities. On the agent side, evaluating agent performance is an active research area, with benchmarks reporting both task accuracy \cite{xie2024osworld, valmeekam2023planbench, xie2024travelplanner, zhou2023webarena, xu2024theagentcompanybenchmarkingllmagents, ma2024spreadsheetbench} and execution time \cite{zhang2025optimizing, zhang2025leveraging, liu2023llm, waytowich2024atari}. Accordingly, in our model, the step-level accuracy $p_{a_i}$ and the execution time $t_{\text{redo}}^k$ can be grounded in these benchmark results. For example, in image editing tasks, benchmarks report the success rate of invoking a given tool as well as the average time required to execute a single operation.

On the human side, our framework allows both population-level and per-user parameterization of timing variables such as $t_{\text{confirm}}^k$ and $t_{\text{diagnose}}^k$. These quantities can capture hierarchical effects, with individual reaction times treated as adjustments to the population average and refined as more user data become available. In this paper, however, we focus on the population-level model and leave personalization to future work.


\section{Model Verification Study Design}
We compared two confirmation formats: intermediate confirmation, guided by our minimum time scheduling model, and confirm-at-end, the baseline in current AI agents. The study tested whether intermediate confirmation improves efficiency and user experience.

\subsection{Simulated Environment}
Real AI agents vary widely in execution time and outcome quality. We built a simulated environment that fixed execution time and accuracy for each task, allowing us to isolate confirmation frequency as the primary factor. The user interface used a clickable-node tracking design (shown in Figure~\ref{fig:env}), which participants in our formative study reported as helpful for navigation and understanding task progress. We selected three representative task domains: shopping cart management, image editing, and the Overcooked game. These cover three broad categories: daily life, work-related, and virtual tasks, while also differing substantially in execution time, tool usage, and workflow complexity. Error cases were drawn from real agent failures observed in prior experiments: in shopping, the agent failed to replace an out-of-stock item; in image editing, the agent applied an incorrect mask that prevented object removal; and in Overcooked, the agent grounded to the wrong ingredient dispenser (e.g., selecting the tomato instead of the onion).

\begin{figure*}
    \centering
    \includegraphics[width=1\linewidth]{env.png}
    \caption{Example of the simulated shopping task interface with intermediate confirmation, the activity list on the left and the current action screenshot on the right, enabling users to compare and identify the first error.}
    \label{fig:env}
\end{figure*}

\subsection{Within-Subjects Design}
We employed a within-subjects design, where each participant completed two trials (intermediate and end) in all three domains. To control for potential biases, we introduced three layers of counterbalancing: task order, confirmation comparison order, and error location. 1) Task order. Each domain was treated as a block. We counterbalanced block order using a $3 \times 6$ Williams design, ensuring that each task domain appeared equally often in each position and that domain sequences were balanced across participants. 2) Confirmation strategy order. Within each block, the two confirmation formats for the same task were presented back-to-back. They were set to the same error location. To reduce familiarity effects, each trial followed the same structure but used different content (e.g., one shopping trial involved office supplies, another fitness items). Participants encountered different confirmation orders across the three blocks, and intermediate-first and end-first trials were balanced at the group level. 3) Error location. Error positions were sampled from early, middle, or late steps; shopping tasks included only early and late because these tasks were shorter. To prevent participants from anticipating that every trial would contain an error, we had each participant do one no-error trial. The no-error trial was implemented with fresh task content, assigned to one of the three task types, and counterbalanced across participants in both task type and confirmation format.


\subsection{Procedure}
The study began with a short tutorial explaining the agent background and confirmation methods. Each trial started with a task goal, and the Overcooked block included an additional introduction to the game rules. Each participant experienced 7 trials, including 2 (confirmation formats) * 3 (task types) + 1 no-error trial. After the trials, participants completed a post-study survey to capture participants' reflections on the confirmation strategies. The survey included open-ended questions pertaining to perceived task success, preferred confirmation style, and overall impressions of different confirmation strategies. It also included Likert-scale items assessing participants' perceptions of intermediate checkpoints (e.g., whether they saved time, reduced burden, or disrupted flow).


\subsection{Participants}
We recruited 48 participants from a university community (18 female, 30 male). Regarding LLM usage, 41 participants (85\%) reported using LLMs frequently (weekly or daily). In terms of familiarity with AI agents, $7$ (15\%) had never used them, $9$ (18\%) were beginners, $19$ (39\%) were intermediate users, and $13$ (28\%) were advanced users. Each session lasted about 15 minutes, and participants received \$5 compensation.


\subsection{Variable Determination} 
The confirmation frequency in our model is determined by four parameters: $t_{\text{confirm}}$, $t_{\text{diagnose}}$, $t_{\text{redo}}$, and the accuracy rate. In this study, each task domain is associated with its own set of parameters. (1) Confirmation and diagnosis times were estimated from a pilot study with random checkpoints. In the main study, these averages were used as fixed inputs, with recalibration performed every 5–10 participants to prevent drift. We deliberately avoided per-user online adaptation, since stable personalization requires long-horizon usage data beyond the scope of this lab setting. (2) Execution time was estimated through formative agent runs and prior literature \cite{zhang2025optimizing, zhang2025leveraging, liu2023llm, waytowich2024atari}. We set $t_{\text{redo}} = 20$s for shopping (dominated by network and screenshot analysis), $10$s for image editing (external tool calls), and $10$s for Overcooked (agent reasoning plus game state processing). (3) For accuracy, we set fixed per-step action correctness probabilities for each task domain, derived from formative agent runs: 
$p_{\text{shopping}} = 87.5\%$, 
$p_{\text{image editing}} = 91\%$, and 
$p_{\text{overcooked}} = 93\%$. These values were applied uniformly across all steps within a task. 




\section{Model Verification Results}
In this section, we verify our model through both quantitative and qualitative analyses. While the quantitative results demonstrate what improvements intermediate confirmation achieves, the qualitative feedback explains why users value it and offers insights for future directions we discuss next.

\subsection{Quantitative Results}
\begin{figure*}
    \centering
    \includegraphics[width=1\linewidth]{quan.png}
    \caption{Average completion time under different confirmation strategies. Left: Intermediate confirmation reduced total time compared to end confirmation. Right: Effects vary by error location—large gains for early errors, minimal for mid, and slight cost for late errors. Error bars show 95\% CIs.}
    \label{fig:qual}
\end{figure*}

Across all task domains, our model reduced the completion time of a full task trial (8--16 steps) by \textbf{13.54\%}, corresponding to a savings of 35.84 seconds per task, compared to the baseline (End Confirmation) condition, $t(143) = 5.52, p < 0.001$. The baseline condition with end confirmation required an average of 264.70 seconds (95\% CI [253.43, 275.98]), while intermediate confirmation using our model required 228.86 seconds (95\% CI [219.73, 237.99]). Breaking down by task domain, the reductions were \textbf{17.44\%} (47.60s) for Shopping ($t(47) = 5.25, p < 0.001$; our model mean = 225.33s, 95\% CI [212.78, 237.88]), \textbf{7.46\%} (17.41s) for Image Editing ($t(47) = 2.34, p = 0.024$; our model mean = 215.85s, 95\% CI [203.42, 228.28]), and \textbf{15.64\%} (49.03s) for Overcooked Cooking ($t(40) = 2.14, p = 0.044$; our model mean = 264.51s, 95\% CI [236.71, 292.30]). 

The effect of confirmation frequency varied by error location. When errors occurred early, intermediate confirmation substantially reduced total time ($\approx 29\%$). For mid-task errors, the benefit was minimal ($\approx 2\%$), while for late errors, intermediate checkpoints slightly increased time ($+4.5\%$). This indicates that the value of intermediate confirmation is greatest for early error detection and diminishes as errors occur later in the task.

Across all tasks, the average completion time was $246.78$s (95\% CI [236.58, 256.98]), consisting of $13\%$ ($32.08$s, 95\% CI [28.29, 32.95]) confirmation, $9\%$ ($21.85$s, 95\% CI [16.99, 26.74]) diagnosis, and $78\%$ ($195.55$s, 95\% CI [189.30, 201.79]) execution. Shifting from end to intermediate confirmation meant spending more time on confirmations, but this was offset by large savings in diagnosis ($-17\%$) and redo ($-22\%$). Moreover, the average time per confirmation decreased by $38\%$, as users only needed to consider the most recent steps rather than reasoning over the entire task history when verifying correctness. Intermediate confirmation reduced the mental tax of each checkpoint. 

Beyond efficiency, reducing redo ($-22\%$) also cuts repeated LLM calls, lowering both financial costs and the energy footprint of unnecessary computation \cite{patterson2021carbon, luccioni2023estimating}. Although environmental impact is not our primary focus, this highlights an additional benefit: confirmation strategies that limit rework may also support more sustainable AI use.


\subsection{Qualitative Findings}


\subsubsection{Overall preference.} In our post-study survey, \textbf{81\%} of participants (39 of 48) preferred the modeled intermediate confirmation strategy, compared to 15\% (7 participants) who preferred end confirmation and 4\% (2 participants) with no preference (Fig.~\ref{fig:qual}, right). The Likert responses (Figure.~\ref{fig:qual}, left) further support this preference: most participants agreed that intermediate confirmation reduced errors (81\%), saved time (71\%), lowered cognitive burden (63\%), and helped prevent harmful errors (83\%). Notably, 77\% of participants disagreed that intermediate checkpoints disrupted their task flow.  As P18 noted, \textit{``I chose intermediate confirmation because it provides a balance between efficiency and reliability. By allowing the system to confirm at selected checkpoints, I can catch potential mistakes earlier instead of waiting until the very end, while still avoiding the constant interruptions of confirming every small step. This makes the process feel smoother but still safe.''} Overall, our modeled confirmation frequency achieved the intended goal of balancing user control with agent automation.


\subsubsection{Impact of Diagnosis.}Although diagnosis time accounted for only a small portion of total completion time in the quantitative study, participants emphasized that it had a strong influence on their experience. When errors were revealed only at the end, users reported feeling disoriented, as they had to review all steps at once without clear guidance: \textit{``If I can only confirm at the end, I feel confused about how to deal with the result, because I don’t know which step to check.''} (P39) Timely checkpoints minimized the effort needed for diagnosis, making error handling less confusing and less mentally taxing for users.

\subsubsection{Confirm at Early Stage.} The value of confirmations depends on the stage at which they occur. Based on our quantitative analysis, the largest time savings of intermediate confirmation arose when the agent made mistakes early in the task. The user survey further reinforced this finding from an experiential perspective, highlighting the importance of early-stage confirmation for both efficiency and usability. Each time a user and agent collaborate on a new task, the interaction is a dynamic process of gradually deepening mutual understanding: the agent gains a clearer understanding of the user’s needs, while the user gains a better sense of the agent’s performance and progressively builds trust in its automatic execution for that specific task. As P9 remarked:
\begin{quote}
``With an employee I haven't fully developed trust in yet, I would want to see intermediate steps in their work just to check that they've done it right. Perhaps if the employee has shown me that they can do the tasks correctly several times without errors, they can just show me their work at the end. I can't trust that it's going to do the task right all the way to the end quite yet, so I want it to show me that it has done the intermediate steps right before getting to a wrong final product.``   
\end{quote}
This perspective is also reflected in our model. We allow each step to have its own probability $p_{a_i}$, which can initially be set conservatively so that early-stage confirmations are more frequent. As users catch and correct early errors, $p_{a_i}$ increases, and the model dynamically decreases confirmation frequency. This adaptive behavior was not fully captured in our simulated environment. We will discuss in the next section how such dynamics could be more realistically incorporated.
\begin{figure*}
    \centering
    \includegraphics[width=1\linewidth]{qual.jpg}
    \caption{Participant Likert scale scores for the questions asked in our post-study survey (Left).Preferred confirmation style(Right)}
    \label{fig:qual}
\end{figure*}

\subsubsection{Subjective perception.} Participants’ preferences for confirmation strategy also reflected their perceptions of task importance and difficulty. For important tasks, they preferred more frequent checks (e.g., ``something elaborate like trip planning”), while for trivial tasks (e.g., ``adding groceries to my cart”), they preferred confirmation only at the end (P20). Perceived difficulty also mattered: users expected easier tasks to require fewer confirmations. However, user perceptions often diverged from reality. For tasks they considered simple, participants expected fewer confirmations, yet were frustrated when the agent still made errors. For example, agents frequently failed in the shopping cart task, but many users saw this task as trivial. Since our model calculates confirmation frequency from actual error rates, it correctly prescribed more frequent checkpoints. While efficient in theory, such frequent confirmations may feel disruptive when they conflict with user expectations. In the next section, we discuss how confirmation strategies can be better aligned with user perceptions in future systems.

\section{Discussion}
Our findings suggest that confirmation strategies should not be seen as a binary choice between user-driven supervision and agent-driven autonomy. Our decision-theoretic model strikes a mixed-initiative balance, where responsibility is flexibly shared between human and agent. To move toward this vision, we consider two complementary directions: (1) what else to model—expanding beyond efficiency to capture subjective costs and trust, and (2) how else to design—re-imagining confirmation as an interaction paradigm rather than a discrete interruption.

\subsection{Beyond Efficiency: Subjective Costs and Trust}
Our current model treats time as the only cost factor when determining the best confirmation points. However, in real-world applications, user decisions about when to confirm are influenced by broader considerations. As noted in Section 7.2.4, subjective perceptions play a critical role; here, we expand on this point by examining two key dimensions: real-world cost and user-perceived reliability.

\subsubsection{Real-world cost} In our CDCR model, cost in the final redo stage is defined only from the agent's perspective (i.e., the time required for the agent to recover from an error). In practice, however, human users often bear additional costs that extend far beyond execution time. Errors may lead to financial loss (e.g., rescheduling fees for a wrong flight), emotional or social consequences (e.g., accidentally sending an email to the wrong recipient), or even irreversible risks (e.g., permanently deleting a file). Future research should explore how these human-centered costs affect confirmation strategies. For high-stakes actions, mandatory checkpoints should be enforced regardless of the system’s estimated accuracy. For general actions, the replanning term can be extended into a multi-dimensional cost function that balances a heterogeneous variety of costs, such as time, money, emotional costs, and risks. 
%\begin{equation}
%\pi^* = \text{argmin} \mathbb{E} [a*Time+b*Money+c*Emotion+d*Risk+...] 
%\end{equation}

Reliability engineering provides inspiration, with established methods for integrating economic trade-offs, workforce allocation, and interdependent subsystems into cost functions \cite{ten2013optimizing, iren2014cost, zhou2015preventive}, which could guide extensions of our model.

\subsubsection{Perceived reliability} Users’ willingness to confirm is shaped not only by actual system accuracy but also by how reliable they believe the agent to be, based on their subjective experiences and agent framing \cite{lee2004trust, parasuraman2010complacency, shneiderman2020human}. For instance, positive framing can inflate perceived reliability and encourage overuse even when this harms performance \cite{cockburn2020framing, quinn2020loss, quinn2016bad}. Our own studies echoed these findings. First-time users who were unfamiliar with the agent tended to perceive it as less reliable, which naturally led them to confirm more frequently. Conversely, in tasks they considered simple, users expected that fewer confirmations would suffice. Yet this expectation often diverged from reality when the agent’s actual accuracy was lower than anticipated.

Such mismatches highlight the need for balancing actual system accuracy with user perception. Our model is already equipped to support this: by assigning each action its own correctness probability $p_{a_i}$ based on past performance data, we can introduce a weighting scheme that blends objective error rates with subjective perceptions, ensuring that confirmation policies remain both efficient and aligned with user expectations.

\subsection{Designing Confirmation as Interaction}
To better balance user control with agent automation, confirmation should be treated not only as a scheduling problem but also as an interaction design challenge. From our study, we identify three insights that illustrate new ways of structuring this interaction.

\subsubsection{Proactive engagement} In our two user studies, current agent UIs left users in a passive ``wait state'' during execution: with no meaningful hooks to monitor or steer the run, participants turned to other tasks and disengaged from supervision. This disengagement is risky and can lead to not only time and resources waste (discussed in our paper), but also harmful or unsafe actions (e.g., privacy breaches, opening phishing links, manipulative suggestions) \cite{alberts2024computers, bansal2024challenges, amodei2016concrete}. Current systems like GPT Agents \cite{openai2025agent} exemplify a reactive model, asking for clarification only when they cannot proceed, which is insufficient to prevent harmful actions. As agents become increasingly autonomous, effective mixed-initiative interaction requires what \citet{tennenhouse2000proactive} envisioned: moving humans from being ``in the loop'' to supervisory and policy-making roles. This shift—from human-centered to human-supervised design—requires agents to proactively solicit user supervision. Our model adds one dimension of proactivity by placing more checkpoints, but proactive supervision should extend beyond correctness to encompass broader forms of risk.

\subsubsection{Goal Alignment} Our findings on error location highlight the importance of prevention. Errors introduced early often stem from poorly communicated goals rather than faulty execution \cite{bansal2024challenges}. While most HCI work emphasizes co-execution or error repair \cite{feng2024cocoa, arawjo2024chainforge, epperson2025interactive}, we identify a design opportunity in early-stage goal alignment. Before execution, agents could present a structured plan for user review, drawing on concepts from formal languages that use rigorous specifications to evaluate whether task goals are satisfied \cite{aeronautiques1998pddl, shivashankar2013hierarchical, nau2003shop2}. As noted in Section 7.2.3, our model assigns each agent action its own correctness probability $p_{a_i}$, which could be adapted to support such early-stage goal alignment interfaces, enabling more frequent and dynamic checkpoints.

\subsubsection{Right amount of information} As tasks grow longer, participants in our study found confirmation increasingly difficult: extended linear logs made it hard to locate issues, leaving users disoriented. This highlights a core design question—what level of detail should agents expose for effective supervision? Current UIs that present step-by-step activity lists with screenshots are often too low-level, overwhelming users with detail without offering structure. Future interfaces should balance informativeness and abstraction, allowing users to monitor actions without being lost in execution traces. One promising direction is to summarize and restructure agent actions hierarchically, so that users can reason about plans at multiple levels of abstraction rather than scanning long linear logs. If such hierarchical structuring is adopted, our CDCR pattern remains intact, but diagnosis time may decrease in a logarithmic fashion—since users could locate errors by checking higher-level chunks first (e.g., binary or ternary splits) rather than scanning all low-level steps.


\section{Conclusion}
This paper introduced a decision-theoretic model for determining when users should confirm agent actions in long-horizon tasks. Building on insights from a formative study that surfaced the recurring CDCR pattern and dissatisfaction with confirm-at-end strategies, we conducted a controlled experiment with 48 participants and found that intermediate confirmation reduced task completion time by 13.54\% and was strongly preferred by 81\% of participants. Looking ahead, our findings suggest that confirmation should be framed not as a binary trade-off between autonomy and oversight, but as a mixed-initiative design opportunity, pointing to future work on integrating efficiency, trust, and interaction design to create more reliable and user-supervised agentic systems.

\bibliographystyle{ACM-Reference-Format}
\bibliography{sources.bib}
\end{document}




% \newcommand{\isChecklistMainFile}{}
% \input{AnonymousSubmission/LaTeX/Reproducibility Checklist}


\end{document}

\typeout{get arXiv to do 4 passes: Label(s) may have changed. Rerun}

% %Bibliography
% \bibliographystyle{unsrt}  
% \bibliography{references}  


% \end{document}
