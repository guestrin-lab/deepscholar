\documentclass{article}
%\documentclass[sigconf]{acmart}

%%
%% \BibTeX command to typeset BibTeX logo in the docs
%% -> copied from the ACM template
%\AtBeginDocument{%mariakober.research@gmx.com
 % \providecommand\BibTeX{{%
  %  Bib\TeX}}}
    
\usepackage{arxiv}
\usepackage[T1]{fontenc}
\usepackage{graphicx}
\usepackage{textcomp}
\usepackage{xcolor}
\usepackage{listings}
\usepackage{balance}
\usepackage{tcolorbox}
\usepackage{adjustbox}
\usepackage{siunitx}
\usepackage{subcaption}
\usepackage{bbding}
\usepackage{pifont}
\usepackage{multirow}
\usepackage{listings}
\usepackage{minted}
\usepackage{ifthen}
\usepackage{enumitem}
\usepackage{url}
\usepackage{xurl}

\setminted[java]{
xleftmargin=15pt,
framesep=2mm,
linenos=true,
breaklines=true,
tabsize=2,
encoding=utf8,
frame=none,
fontsize=\scriptsize,
escapeinside=|| 
}


\lstdefinestyle{prompt}{
    commentstyle=\color{codegreen},
    keywordstyle=\color{magenta},
    stringstyle=\color{codepurple},
    basicstyle=\ttfamily\footnotesize,
    breakatwhitespace=false,         
    breaklines=true,                 
    captionpos=b,
    keepspaces=false,               
    showspaces=false,                
    showstringspaces=false,
    showtabs=false,                  
    tabsize=2,
    literate={\Ü}{{\"U}}1
             {\ä}{{\"a}}1
             {\ö}{{\"o}}1
             {\ü}{{\"u}}1
             {\≤}{{$\leq$}}1
}

\lstdefinestyle{code}{
    backgroundcolor=\color{backcolour},   
    commentstyle=\color{codegreen},
    keywordstyle=\color{magenta},
    numberstyle=\tiny\color{codegray},
    stringstyle=\color{codepurple},
    basicstyle=\ttfamily\footnotesize,
    breakatwhitespace=false,         
    breaklines=true,                 
    captionpos=b,                    
    keepspaces=true,                 
    numbers=left,                    
    numbersep=5pt,                  
    showspaces=false,                
    showstringspaces=false,
    showtabs=false,                  
    tabsize=2
}

\newcommand{\wcircle}[1]{\ding{\numexpr171 + #1}}
\newcommand{\bcircle}[1]{\ding{\numexpr181 + #1}}
%\usepackage{minted}
%\usepackage[left=1.8cm,right=1.8cm,top=2.5cm]{geometry}
%\usepackage{fancyhdr}
%\usepackage{titlesec}



% \setminted[java]{
% xleftmargin=15pt,
% framesep=2mm,
% linenos=true,
% breaklines=true,
% tabsize=2,
% encoding=utf8,
% frame=none,
% fontsize=\footnotesize,
% escapeinside=|| 
% }

\lstdefinestyle{mystyle}{
    backgroundcolor=\color{lightgray!40},   
    commentstyle=\color{green},
    keywordstyle=\color{blue},
    numberstyle=\tiny\color{gray},
    stringstyle=\color{red},
    basicstyle=\ttfamily\footnotesize,
    breakatwhitespace=false,         
    breaklines=true,                 
    captionpos=b,                    
    keepspaces=true,                 
    numbers=left,                    % Enable line numbers
    numbersep=5pt,                  
    showspaces=false,                
    showstringspaces=false,
    showtabs=false,                  
    tabsize=2
}

\lstset{style=mystyle}

\setlength{\abovecaptionskip}{5pt plus 3pt minus 2pt}
\setlength{\belowcaptionskip}{5pt plus 3pt minus 2pt}
\setlength{\textfloatsep}{5pt plus 1.0pt minus 2.0pt}
\setlength{\floatsep}{5pt plus 1.0pt minus 2.0pt}
\setlength{\intextsep}{5pt plus 1.0pt minus 2.0pt}
\setlength{\dbltextfloatsep}{5pt plus 1.0pt minus 2.0pt}
\setlength{\dblfloatsep}{5pt plus 1.0pt minus 2.0pt}

\newcommand{\dcircle}[1]{\ding{\numexpr171 + #1}}

\newcommand{\highlight}[1]{\begin{tcolorbox}[leftrule=0mm,rightrule=0mm,toprule=0mm,bottomrule=0mm,left=0pt,right=0pt,top=0pt,bottom=0pt]
%\begin{finding}
#1
%\end{finding}
\end{tcolorbox}
}

% Comment indicator.
% \newboolean{showcomments}
% \setboolean{showcomments}{true}
% %\setboolean{showcomments}{false}
% \ifthenelse{\boolean{showcomments}}
%  { \newcommand{\mynote}[2]{
%       \fbox{\bfseries\sffamily\scriptsize#1}
%         {\small$\blacktriangleright$\textsf{\emph{#2}}$\blacktriangleleft$}}}
%         { \newcommand{\mynote}[2]{}}
\newcommand{\ad}[1]{\mynote{Alioune}{\textcolor{olive}{#1}}}
\newcommand{\tb}[1]{\mynote{Bissyande}{\textcolor{red}{#1}}}
\newcommand{\jk}[1]{\mynote{Jacques}{\textcolor{blue}{#1}}} 
\newcommand{\js}[1]{\mynote{Jordan}{\textcolor{purple}{#1}}}
\newcommand{\sa}[1]{\mynote{Steven}{\textcolor{teal}{#1}}}
\newcommand{\aw}[1]{\mynote{Aïcha}{\textcolor{magenta}{#1}}}

%  \pagestyle{fancy}
% \chead{\color{gray!60}Cybersecurity4D2024  Conference (CS4D2024)}

\renewcommand{\headrulewidth}{0pt}

%\titlespacing*{\section}{1pt}{0\baselineskip}{0\baselineskip}


\begin{document}
\pagenumbering{gobble}

% We exclude all URLs from the citations to save space
%\bstctlcite{BSTcontrol}


\title{Security Evaluation of Android apps in budget African Mobile Devices}

\author{
Alioune DIALLO \\
\textit{SnT/TruX} \\
  \textit{University of Luxembourg} \\
  Kirchberg, Luxembourg \\
  alioune.diallo@uni.lu \\
   \And
    Anta DIOP\\
    \textit{UCAD/ESP} \\
    \textit{Université Cheikh Anta Diop} \\
    Dakar, Senegal \\
    antadiop1@esp.sn \\
    \And
    Abdoul Kader KABORE\\
   \textit{SnT/TruX} \\
    \textit{University of Luxemboug} \\
    Kirchberg, Luxembourg \\
    abdoulkader.kabore@uni.lu\\
    \And
    Jordan SAMHI \\
  \textit{SnT/TruX}\\
  \textit{University of Luxemboug} \\
  Kirchberg, Luxembourg \\
    jordan.samhi@uni.lu \\
  \And
    Aleksandr PILGUN\\
    \textit{SnT/TruX} \\
    \textit{University of Luxemboug} \\
    Kirchberg, Luxembourg \\
     aleksandr.pilgun@uni.lu \\
     \And
     Tegawendé F. BISSYANDE\\
        \textit{SnT/TruX} \\
        \textit{University of Luxemboug} \\
        Kirchberg, Luxembourg \\
        tegawende.bissyande@uni.lu
      \And
        Jacque KLEIN\\
        \textit{SnT/TruX} \\
        \textit{University of Luxemboug} \\
        Kirchberg, Luxembourg \\
        jacques.klein@uni.lu
}


\maketitle

\begin{abstract}
Android's open-source nature facilitates widespread smartphone accessibility, particularly in price-sensitive markets.
System and vendor applications that come pre-installed on budget Android devices frequently operate with elevated privileges, yet they receive limited independent examination. 
To address this gap, we developed a framework that extracts APKs from physical devices and applies static analysis to identify privacy and security issues in embedded software.
Our study examined 1,544 APKs collected from seven African smartphones.
The analysis revealed that 145 applications (9\%) disclose sensitive data, 249 (16\%) expose critical components without sufficient safeguards, and many present additional risks: 226 execute privileged or dangerous commands, 79 interact with SMS messages (read, send, or delete), and 33 perform silent installation operations. 
We also uncovered a vendor-supplied package that appears to transmit device identifiers and location details to an external third party. 
These results demonstrate that pre-installed applications on widely distributed low-cost devices represent a significant and underexplored threat to user security and privacy.
\\
\end{abstract}

\begin{center}
    \textbf{\textit{Keywords — Android app, sensitive data, PII, vulnerability, malware, African device}}
\end{center}

%\section{Introduction}\label{sec:introduction}
Mixture-of-Experts (MoE) is an architectural paradigm that adaptively combines predictions from multiple neural modules, known as "experts," via a learned gating mechanism. This concept has evolved from ensemble-based MoEs, where experts, jointly trained with a gating function, are often full, independent models whose outputs are combined to improve overall performance and robustness \citep{jacobs1991adaptive}. More recently, MoE layers have been integrated within larger neural architectures, with experts operating in a latent domain. These "latent MoEs" offer significant scalability benefits, especially in large language models (LLMs) \citep{shazeer2017outrageously,fedus2022switch}.
MoE makes it possible to train massive but efficient LLMs, where each token activates only a fraction of the model’s parameters, enabling specialization, better performance, and lower computational cost compared to equally sized dense models.

Regardless of their specific implementation, conventional MoE systems typically produce point estimates, lacking a direct quantification of their uncertainty. In critical applications, this absence of uncertainty information hinders interpretability, making it difficult for users to gauge the reliability of a prediction and limits informed decision-making, as the system cannot express its confidence or identify ambiguous cases. Importantly, the learned gating mechanism, which dictates the relative contribution of each expert, does not take into account expert confidence, potentially leading to suboptimal routing decisions.

In this work, we propose Mixture-of-Gaussians with Uncertainty-based Gating (MoGU), a framework for uncertainty-aware MoE architectures, which provides explicit uncertainty quantification for both individual experts and the overall MoE model. Our approach fundamentally reimagines the expert's output: instead of a point estimate, we model each expert's prediction as a random variable drawn from a normal distribution. In this setup, each expert simultaneously predicts both the mean (the label estimate) and variance of the distribution, representing its predictive uncertainty. This shift enables a more nuanced understanding of expert behavior and the derivation of the overall model's uncertainty. Furthermore, we introduce a novel gating mechanism where the estimated uncertainty of each expert directly informs its relative contribution to the overall MoE prediction, bypassing the need for a separate gating function typically found in traditional MoE setups. This creates a self-aware MoE where more confident experts naturally exert greater influence.

We evaluate MoGU on time series forecasting as our primary regression task. This choice is motivated by the inherent uncertainty in real-world time series data and the wide variety of expert architectures applicable to forecasting tasks across numerous domains \citep{time_series_survey, wang2024deep}. Our evaluation spans various expert types, forecasting benchmarks and forecasting horizon sizes, allowing for a comprehensive assessment of our method's efficacy. MoGU is shown to consistently yield more accurate forecasts compared to input-based gating MoE architectures, while simultaneously, providing uncertainty estimates that are positively correlated with prediction error. These estimates are available at both the individual expert and overall model levels. By further distinguishing between aleatoric (data-related) and epistemic (model-related) uncertainty, MoGU offers valuable insights into the source of a model's uncertainty. We also conducted a detailed ablation study to validate our key design choices.

In summary, our contributions are as follows: 
\begin{itemize}
\item \textbf{MoGU: A Novel Framework for Uncertainty-Aware MoE Architectures}: We introduce a novel framework that directly quantifies uncertainty for both individual experts and the overall model, moving beyond conventional point estimates. A key innovation is a routing mechanism that uses each expert’s estimated predictive uncertainty to dynamically determine its contribution to the final MoE output, replacing traditional input-based gating mechanisms.
\item \textbf{MoGU Improves Time Series Forecasting}: Our method effectively reduces forecasting error across various benchmarks, horizon lengths, and expert architectures.
\item \textbf{MoGU Provides Meaningful Uncertainty Estimates for Time Series Forecasting}: MoGU generates uncertainty estimates at the expert-level and overall. These estimates are positively correlated with prediction error, providing valuable insight into the model's confidence and the sources of its uncertainty.
\end{itemize}

By embedding uncertainty estimation into prediction and gating, MoGU moves beyond input-based gating  MoEs toward architectures that are more accurate, transparent, and reliable.


\section{Introduction}
 The Android operating system has enabled affordable smartphones for millions of people worldwide. Devices typically ship with manufacturer- or vendor-installed system and third-party apps, which can become a distribution vector for malware and privacy-invasive functionality, especially in developing regions~\cite{kaspersky}. Low-cost smartphones remain widely used across Africa~\cite{trustonic} and have helped reduce the digital divide by expanding access to services~\cite{alfred,dan}. For example, roughly 20–22 million people in South Africa use smartphones, accounting for one third of the population~\cite{trustonic}. Feature phones and inexpensive devices are still popular in Africa, which preserves the market opportunity for low-cost Android devices expansion. Although low-cost Android initiatives have improved device accessibility~\cite{barton}, the pre-installation process can be abused to embed harmful apps that reach large user populations.

 Multiple reports document dangerous apps shipped with devices~\cite{bobe,fakeApp,DangApp,AndMal,10.1145/2590296.2590313,builtMal}. These apps have been observed exfiltrating personal data to remote servers~\cite{bobe}, harvesting financial credentials via clipboard monitoring~\cite{fakeApp}, performing silent installations~\cite{10.1145/2590296.2590313}, or enrolling users in paid services without consent~\cite{builtMal}. Several of the reported apps appear specifically on low-cost devices, including phones sold in Africa~\cite{bobe,builtMal}.
%Many low-cost devices dumped in Africa could be potential targets since there are brands that are mainly present in Africa, not in other regions~\cite{,}.
% In a study, authors claimed that manufacturers and constructors have been paid to pre-install malware on the devices~\cite{10.1145/2590296.2590313}.

A few studies collected firmware images from vendor websites and online forums, extracting pre-installed apps for analysis~\cite {10.1145/2590296.2590313,251554,9793923,FirmwareDroid}. The extracted apps are then examined for suspicious permissions, misconfigurations in manifest files~\cite{10.1145/2590296.2590313,9793923,FirmwareDroid}, malware~\cite{10.1145/2590296.2590313}, and data leaks~\cite{251554,9152633,9519485}. 
Other studies have directly extracted pre-installed apps from physical devices~\cite{9152633,9519485}, allowing for a more comprehensive assessment across various Android brands. Yet, research on pre-installed applications remains limited since Android devices manufacturers are expected to strictly follow compliance guidelines enforced by Google~\cite{androidcertified}. Unlike globally known brands, Android devices sold in African markets are often very low‑priced. This raises concerns about the trade‑offs manufacturers make between cost and security.

% Apartently, smartphone brands, such as Infinix\footnote{Infinix website: \url{https://www.infinixmobility.com}}, itel\footnote{itel website: \url{https://www.itel-life.com}}, and Tecno\footnote{Tecno website: \url{https://www.tecno-mobile.com/home/}}, are primarily used in Africa. These brands are not known globally making them less available to security researchers.
% According to Bastien Bobe’s analysis on Medium~\cite{bobe},  these brands include pre-installed malware capable of collecting and transmitting user data, including personal information, location, and contact lists, to a remote server.
% Considering the unreliable Internet connectivity in African countries, suspicious apps could be developed to leak sensitive data using methods other than sending it through the Internet.
% In general, when an application comes from an official app store, such as Google Play, there are plenty of security mechanisms that check the safety of the application. Then, when successfully checked, the app can be considered safe.
We studied seven low-cost device models (Infinix, Tecno, itel) — all produced by TRANSSION and together accounting for 51\% of the African smartphone market in 2023~\cite{markSHared}. We extracted a dataset of \num{1544} pre-installed APKs for analysis.
Pre-installed apps on these devices have received little systematic analysis despite their prevalence. The most notable work is by Elsabagh et al.~\cite{251554} who examined pre-installed apps from Infinix and Tecno, focusing on privilege escalation. To the best of our knowledge, no prior work has systematically analyzed pre-installed apps across these low cost Android devices commonly sold in Africa.
%On the other hand, several blog posts also mentioned pre-installed malware embedded into Android devices, including some of those used in Africa~\cite{fakeApp, DangApp}.
% Most of the pre-installed apps from these devices are not present in Google Play. 
% They could come from unofficial third-party app stores or uncertified Android Open Source Project builds, which could be potentially used to distribute malicious apps.
% It means that they may not benefit from a security mechanism checking, or could contain harmful code.
% Until now, there has been no study specifically focusing on assessing the security of pre-installed apps on these devices.

%Our main objective is to investigate how common pre-installed apps on cheap Android devices dumped in Africa could be suspicious or expose sensitive data.
% We assess the security of pre-installed applications and investigate the presence of suspicious or malicious applications among them on low-cost Android devices commonly sold in Africa. 
% The primary goal is to determine the extent to which these devices exhibit embedded applications that could compromise the users' security and privacy.
% To support this investigation we designed \texttt{\textit{PiPLAnD}}, a pipeline for the systematic inspection of pre-installed apps across Android brands. \texttt{\textit{PiPLAnD}} automates this investigation through static analyses — taint-based data-leak detection, pattern-driven behavior scanning, and manifest/component inspection — to identify suspicious or malicious apps, detect sensitive-data leaks, and find insecurely exported components on the extracted apps.
To explore this gap we developed \texttt{\textit{PiPLAnD}}, a pipeline that extracts APKs from physical devices and applies taint-based leak detection, pattern-driven behavior scanning, and manifest/component inspection to find data leaks, insecurely exported components, and suspicious behaviors.
%Additionally, it analyses the Android manifest file of the app to look for security misconfiguration and assesses the permissions of third-party apps, in particular, to detect excessive privilege usage. 
%\texttt{\textit{PiPLAnD}} uses dynamic analysis to identify suspicious apps that the static technique could not identify.
% Our pipeline leverages certain of the existing state-of-the-art works related to pre-installed apps, addressing some of their limitations, improving the analysis techniques, and considering new things.
%Table~\ref{} summarizes what the literature did concerning this topic and our current work.
% The pipeline has been executed to analyse 7 low-cost Android smartphone models from brands widely used in Africa, including Infinix, Tecno, and itel.
%To provide a comparative perspective, we also analyze pre-installed apps from a device that comes from another region (e.g., Europe).

Our analysis uncovered multiple suspicious pre-installed apps across the tested devices. A substantial fraction of these apps leak sensitive information (for example, IMEI, IMSI, location) and exhibit manifest misconfigurations that expose components without adequate protection. In a notable case, \texttt{\textit{PiPLAnD}} identified the package \textit{com.transsion.statisticalsales}, which was not flagged by VirusTotal. These findings underscore the need for independent audits of pre-installed software on low-cost devices in developing regions.

Below we summarize our main contributions:
\begin{itemize}
    \item[\wcircle{1}] We present \texttt{\textit{PiPLAnD}}, a pipeline to systematically inspect pre-installed apps from Android devices.
    \item[\wcircle{2}] We collected a dataset of \num{1544} pre-installed APKs from seven low-cost device models (Infinix, Tecno, itel).
    \item[\wcircle{3}] We identify manifest misconfigurations: 249 app versions (16\%) export sensitive components without adequate protection.
    \item[\wcircle{4}] We quantify sensitive-data leakage: 145 apps (9\%) leak identifiers or location data.
    \item[\wcircle{5}] We document widespread suspicious behaviors in pre-installed apps.
\end{itemize}

\section{Background}
This short background defines terms used in the paper and clarifies our measurement scope.

\textbf{Android firmware and pre-installed apps.} Firmware boots device hardware and the Android OS~\cite{tLacoma}. Devices ship with \textit{system apps} and \textit{manufacturer-supplied} pre-installed apps~\cite{RolandH}. \textit{System apps} can run with elevated privileges or reside in privileged locations (e.g., \texttt{/system/priv-app}). Some devices in our dataset use Android Go, a lightweight Android configuration for entry-level hardware. Android Go was designed to optimise its running on low-end devices, however the security implications of this optimisation are not fully understood~\cite{androidgo}.

\textbf{Exported components and misconfigurations.} Apps declare components in AndroidManifest.xml. A component is "exported" if other apps can start or bind to it (via \texttt{android:exported} or an \texttt{intent-filter}). Exported components without proper permissions or access checks form an attack surface and are treated here as security misconfigurations~\cite{cweExpComp,exported}.

\textbf{PII, sources, sinks, and leaks.} PII includes identifiers and user data (IMEI, IMSI, phone number, location)~\cite{pii}. A SOURCE is a method that reads sensitive data (e.g., \texttt{getLastKnownLocation()}); a SINK is a method that can exfiltrate data (network send, SMS, world-readable storage). When an app allows to get this data and send it outside the app itself, in this case, we talk about data leakage (or data leak). Static taint analysis tools like FlowDroid are well known to detect leaks in Android apps~\cite{10.1145/2594291.2594299}.

\textbf{Low-cost device.}
In this study, we often use the terms "low-cost", "low-end", "cheap", and "affordable" when talking about devices that are not expensive but affordable, and specifically targeting devices primarily used in Africa.

\section{Low-cost Android devices in Africa}
\label{devices}

Upon careful investigation of the African mobile device market we identified three popular brands shipping Android devices under 100 US dollars price -- Infinix, itel, and Tecno. All these devices run Android Go Edition. We acquired 7 devices in total from these brands for our study.

\subsection{Android Go Edition}

Android Go Edition is a lightweight configuration of standard Android designed for entry-level devices with limited memory ($\leq$ 2\,GB) and storage~\cite{androidgo}. Android Go ships with resource-optimized "Go" versions of Google apps, but users can still install apps from the Play Store. Android Go may receive updates less frequently than standard Android~\cite{goVSstandard,androidgo}. To reduce resource consumption, Android Go disables several features by default~\cite{androidgo}:
\begin{itemize}
    \item Picture-in-picture support;
    \item \texttt{SYSTEM\_ALERT\_WINDOW} permission (display over other apps);
    \item Split-screen / multi-window;
    \item Live wallpapers;
    \item Multi-display;
    \item Launcher shortcuts (deep shortcuts);
    \item Reduced maximum width/height for images in remote views;
    \item VR mode.
\end{itemize}

Android Go is not necessarily less secure than standard Android. In fact, users may gain security benefits from the removal of the \texttt{SYSTEM\_ALERT\_WINDOW} permission, which has often been abused by malicious apps~\cite{fratantonio2017cloak}. We did not find literature that evaluates the security implications of the other optimizations introduced in Android Go. However, Android Go devices typically receive fewer updates and have a shorter support lifecycle, which can leave users without critical security patches~\cite{acar202450}.

Many pre-installed apps on some device brands are not available on Google Play. They may come from unverified third-party vendors that may potentially distribute malicious apps to the end users.
%Several brands, such as Infinx, itel, and Tecno, have most of their preinstalled apps not in Google Play Store.

\subsection{Devices and Pre-installed apps}

From our seven devices (Infinix, itel, and Tecno) we extracted \num{1544} pre-installed APK files, including both system and third-party apps, which form the dataset used in this study.
% Table~\ref{tabDevice} gives a detailed description of these device models.
% \begin{table}[htbp]
% \caption{Details about the devices chosen for analysis.}
% \begin{center}
% \begin{adjustbox}{width=\linewidth}
% \begin{tabular}{|c|c|c|c|}
% \hline
% \textbf{Device}&\multicolumn{3}{|c|}{\textbf{Device brands}} \\
% \cline{2-4} 
% \textbf{Feature} & \textbf{\textit{Infinix}}& \textbf{\textit{itel}}& \textbf{\textit{Tecno}} \\
% \hline
% Model& SMART 8 X6525 & A50 & POP 8 BG6m \\
% \hline
% OS version& 13 (Go edition) & 14 (Go edition) & 14 (Go edition) \\
% \hline
% Processor& T606 & T603 & T606 \\
% \hline
% Network& 4G & 4G & 4G \\
% \hline
% Storage& 64GB & 64GB & 64GB \\
% \hline
% Memory& 6GB & 6GB & 4GB \\
% \hline
% Battery capacity& 5000mAh & 5000mAh & 5000mAh \\
% \hline
% Price (Avg)& €80 & & €85 \\
% % copy& More table copy$^{\mathrm{a}}$& &  \\
% \hline
% \textbf{Pre-installed APKs} & \textbf{235} & \textbf{248} & \textbf{244} \\
% \hline
% %\multicolumn{4}{l}{$^{\mathrm{a}}$Sample of a Table footnote.}
% \end{tabular}
% \end{adjustbox}
% \label{tabDevice}
% \end{center}
% \end{table}
%We have extracted 235 APK files from Infinix, 248 from itel, and 244 from Tecno for analysis.
% This dataset includes system apps as well as third-party apps shipped with the phones. 
%We have found third-party apps only on Infinix (3 apps) and on itel (12 apps).
Most of the pre-installed apps extracted cannot be found in the Google Play Store, as shown in Figure~\ref{fig:AppGP}.
\begin{figure}[!h]
\centerline{\includegraphics[width=0.45\linewidth]{appGP_plot.png}}
\caption{App present in Google Play vs. App not present in Google Play}
\label{fig:AppGP}
\end{figure}
Several brands provide their own app stores. In particular, the Palm Store is reported to be the official app distribution platform for Infinix, Tecno, and itel~\cite{palmstore}. Palm Store is preinstalled on the devices we examined and allows users to install, uninstall, and update apps. According to the provider, the store performs automated and manual security checks, compatibility testing, and content-compliance monitoring~\cite{palmstore}.

System apps are signed by multiple different certificate authorities, and the dominant signer varies by device. For example, Table~\ref{tab:Sig} shows that on the Infinix SMART8, 68\% of system apps are signed by Infinix; 20\% by Google; 4\% by Transsion; 2\% by Tecno; 1\% by Facebook; 1\% with the default AOSP certificate; and 4\% by other authorities.
\begin{table}[!h]
\caption{System apps grouped by certificate authority for some devices.}
\begin{center}
\begin{adjustbox}{width=0.5\linewidth}
\begin{tabular}{|c|c|c|c|}
\hline
% \textbf{Signed by }&\multicolumn{3}{|c|}{\textbf{Device brands}} \\
% \cline{2-4} 
\textbf{Signed by} & \textbf{\textit{Infinix SMART8}}& \textbf{\textit{itel A50}}& \textbf{\textit{Tecno POP8}} \\
\hline
Infinix & 68\% & 0\% & 0\% \\
\hline
itel & 0\%& 71\% & 0\% \\
\hline
Tecno & 2\% & 1\% & 68\% \\
\hline
Transsion & 4\% & 2\% & 3\% \\
\hline
Google & 20\% & 22\% & 21\% \\
\hline
Facebook & 1\% & 1\% & 2\% \\
\hline
Default & 1\% & 1\% & 1\% \\
\hline
SW & 0\% & 0\% & 2\% \\
\hline
Others & 4\% & 2\% & 3\%  \\
% copy& More table copy$^{\mathrm{a}}$& &  \\
\hline
%\multicolumn{4}{l}{$^{\mathrm{a}}$Sample of a Table footnote.}
\end{tabular}
\end{adjustbox}
\label{tab:Sig}
\end{center}
\end{table}
System apps are usually found on \textit{/system/app} and \textit{/system/priv-app} folders.
However, when extracting the APK files, we have found that the apps have been distributed not only in these two folders, but in several others as well.
Figure~\ref{fig:locFolder} shows an example of the folders on Infinix.
\begin{figure}[!h]
\centerline{\includegraphics[width=0.5\linewidth]{folderInfinix.png}}
\caption{Location folders of the system apps in Infinix}
\label{fig:locFolder}
\end{figure}
Many apps for some brands have been pre-installed based on the regional needs. For example, the supply chain of Transsion devices suggests that Transsion works with partners in Africa, which could pre-install apps and services tailored for the regional needs~\cite{regneed}.
Since many steps exist to pre-install apps, this could be an entry point for introducing malicious apps on these devices~\cite{preinstMal}.

%% Explain environment/problem
% To address the problem of bridging the gap between simulated and real-world network data, we propose leveraging transfer learning with the RouteNet-Fermi model~\cite{ferriolgalmés2022routenetfermi}. Our approach involves two main components: (1) training the RouteNet-Fermi model on simulated data and (2) fine-tuning it with real-world network data. This methodology enables us to combine the broad generalization capabilities gained from simulation with real-world specificity.

Our approach involves two main components: (1) training the RouteNet-Fermi~\cite{ferriolgalmés2022routenetfermi} model on simulated data and (2) fine-tuning it with real-world network data. This methodology enables us to combine the broad generalization capabilities gained from simulation with real-world specificity.

\subsection{Model architecture}

We use a modified RouteNet-Fermi architecture to predict network performance metrics. The architecture can be decomposed into three main blocks:
\begin{enumerate}
    \item Encoding: Multi-layer perceptrons (MLPs) generate initial embeddings for network elements.
    \item Message Passing Algorithm (MPA): The embeddings are refined using the relationships between network elements by employing Gated Recurrent Units (GRU)~\cite{cho2014learningphraserepresentationsusing}.
    \item Readout: The final flow embeddings are used to predict performance metrics via an MLP.
\end{enumerate}

It should be noted that RouteNet-Fermi assumes stationary traffic, a condition that does not always apply to real-world network data. In turn, we adapt RouteNet-Fermi to non-stationary traffic by splitting network scenarios into temporal windows and predicting performance metrics for each window individually. This ensures the stationarity assumption applies only within shorter intervals, allowing the model to adapt to changing traffic conditions as in real-world scenarios.

Adapting the architecture requires two key modifications. First, the input features are adjusted to include window-specific attributes rather than global flow-level parameters. This includes features such as flow bandwidth and packet rate per window. Second, we introduce a GRU neural network during the MPA phase to capture inter-window dependencies. This mechanism updates queue embeddings in each window using those from the previous window, enabling the model to propagate temporal information. Overall, these measures aim to improve accuracy under non-stationary traffic conditions.

% \mpp{Adapting the architecture requires two key modifications. First, the input features are adjusted to include window-specific attributes, such as average bandwidth and packet rate within each temporal window. These attributes are crucial for capturing transient traffic dynamics in non-stationary network scenarios, where traffic conditions evolve over time. By focusing on shorter temporal intervals, the model can better track and respond to fluctuations in network performance, which static, aggregated metrics would miss.}

% \mpp{Second, to capture inter-window dependencies, we introduce a Gated Recurrent Unit (GRU) neural network during the Message Passing Algorithm (MPA) phase. The GRU is designed to retain information from previous windows and propagate it across the network, enabling the model to account for temporal patterns and trends. This enhancement allows the architecture to adapt to traffic changes over time, improving its predictive accuracy in real-world scenarios.}

% Types of TL used
\subsection{Manual transfer learning}
\label{sec:proposed_ft}
To summarize, we currently have an ML-model architecture, RouteNet-Fermi, a large dataset of simulated network scenarios, and a small dataset of real-world network scenarios. We propose using transfer learning to leverage the strengths of the simulated dataset while adapting the model to real-world conditions.
% \textbf{FIXME: This paragraph is redundant, I suggest to summarize it to 1 or 2 sentences} To summarize, we currently have an ML-model architecture, RouteNet-Fermi, and two datasets to work with. On one hand, we have a dataset made from real-world network scenarios, which is limited in size and variability, making it challenging to train a robust model directly. On the other, we have a dataset composed of simulated scenarios, that while abundant and diverse, result in sub-optimal models due to domain mismatch. To address this, we propose using transfer learning to leverage the strengths of the simulated dataset while adapting the model to real-world conditions, all without introducing excessive bias from the simulated samples.
% Reviewing Figure~\ref{fig:transfer_learning_pipeline}, our approach consists in transferring the weights from a pre-trained network model with the simulated data to act as a foundation for the network model for real-network data. By reusing the learned knowledge encoded in the pre-trained weights, the receiver model gains a significant advantage in adapting to the real-world environment with minimal data.
We start by training the network model with the simulated data to act as a foundation for the network model for real-network data. When fine-tuning, a critical decision is determining how to handle the weights of each block in the network.
Following the principles outlined in \cite{zeiler2013visualizingunderstandingconvolutionalnetworks}, we evaluate those configurations that adhere to the following guidelines:
\begin{itemize}
    \item Layer dependencies: We avoid configurations where a block is frozen or fine-tuned if preceded by a re-trained block. Otherwise, it would disrupt the natural flow of learned representations. We also avoid configurations where a block is frozen if preceded by a fine-tuned block.
    \item Trainable weights: We never freeze all blocks, as this would leave no trainable parameters for adaptation.
    \item Always transfer something: We never re-train all blocks, as it is equivalent to training the model from scratch.
\end{itemize}
% Following the principles outlined in \cite{zeiler2013visualizingunderstandingconvolutionalnetworks}, we evaluate those configurations that respect layer dependencies. That is, avoid configurations where a block is frozen or fine-tuned if preceded by a re-trained block as otherwise, it would disrupt the natural flow of learned representations. For the same reason we also avoid configurations where a block is frozen if preceded by a fine-tuned block.

The resulting testbed configurations are listed in Table~\ref{tab:results} in the evaluation. We split the network into blocks rather than individual layers to align with RouteNet-Fermi’s architecture. Unlike traditional NNs like MLPs, which are sequential, RouteNet-Fermi operates more like an ensemble of smaller NNs that work in parallel. For instance, the MLPs in the encoding block process individual network elements independently. Grouping these layers into blocks provides a structured approach to fine-tuning while ensuring that dependencies between blocks are respected. Furthermore, the shallow depth of the RouteNet-Fermi's internal NNs, with the deepest component being a 3-layered MLP, limits the benefit of fine-tuning individual layers.

\begin{figure}[t]
    \centering
    \includegraphics[width=0.8\linewidth]{images/fine_tuning_example.pdf}
    \caption{Visual example of fine-tuning a RouteNet-Fermi~\cite{ferriolgalmés2022routenetfermi} model, where the Encoding is frozen, the MPA is fine-tuned, and the Readout is re-trained.}
    \label{fig:fine_tuning_example}
\end{figure}

In Figure~\ref{fig:fine_tuning_example}, we show an example of how the model can be fine-tuned. In this example, the chosen fine-tune configuration was to freeze the Encoding block, fine-tune the MPA block, and re-train the Readout block. As a result, we only transfer the Encoding and MPA weights from the donor model, while the Readout block's weights are randomly initialized as in traditional training. Then, during the fine-tuning training, the Encoding block is excluded so as not to modify its weights. Note that the fine-tuning training is otherwise similar to the original training process, but using the real-network samples and with a diminished learning rate ($\approx10\times$  smaller).

\subsection{Automated transfer learning}
\label{sec:automated_ft}

In addition to the previous manual configurations, we also test our approach using automated fine-tuning approaches from the state-of-the-art. These do not require manually deciding which blocks to freeze, fine-tune, or retrain, reducing trial-and-error:

\begin{itemize}
    \item Autofreeze~\cite{liu2021autofreeze}: This method consists of loading all donor weights and setting them as trainable. During training, blocks whose weight gradients fall under a threshold are frozen, allowing weights to be adjusted while minimizing computational costs.
    \item L2-SP~\cite{pmlr-v80-li18a}: This method consists of adding a regularization term in the loss function involving the L2-distance between the receiver and donor weights. This guides the learning of the receiver model and is more effective at avoiding overfitting than the standard L2 regularization.
    \item GTOT-Tuning~\cite{zhang2022fine}: A more advanced version of L2-SP meant for Graph Neural Networks (GNNs). Instead of comparing weights, it measures the differences between node embeddings after the MPA using the Masked Wasserstein Distance (MWD). The mask in the MWD allows it to incorporate relational information.
\end{itemize}

\subsection{Testbed}
\label{subsec:testbed}
\begin{figure}[t]
    \centering
    \includegraphics[width=0.90\linewidth]{images/testbed.pdf}
    \caption{Diagram summarizing the testbed's structure.}
    \label{fig:testbed}
\end{figure}

To collect real network samples, we use a custom testbed with up to 8 real routers and traffic generators connected via switches. VLAN-based configurations allow emulation of diverse network topologies. Links range from 1 to 40 Gbps to simulate modern network conditions. Traffic is generated using 2–8 servers running MGEN and Tcpreplay. An optical splitter enables passive traffic capture for analysis, ensuring low interference. Figure~\ref{fig:testbed} shows the testbed's structure.


% \begin{itemize}
%     \item The testbed comprises up to 8 Huawei NetEngine 8000 M1A routers, interconnected via two Huawei S5732-H48UM 2CC 5G Bundle switches. A third Cisco WS-C4506-E switch links the traffic generators to one of the router switches, enabling flexible traffic configurations.
%     \item Traffic is generated using 2 to 8 servers running the MGEN~\cite{MGEN} and Tcpreplay~\cite{Tcpreplay} software. These servers create traffic for each source/destination pair, allowing the testbed to replicate a variety of network topologies through VLAN-based routing paths.
%     \item An optical splitter duplicates both ingoing and outgoing traffic from the traffic generators to a packet capture server. This server processes the captured data for subsequent analysis. By using optical splitters, we minimize any potential impact on traffic transmission or recording accuracy, ensuring high-fidelity measurements.
%     \item The routers and traffic generators are connected to the switches using 1 Gbps links, representative of modern network conditions. To prevent bottlenecks in the testbed's control plane, higher-capacity links are employed: 10 Gbps links connect the traffic generator switch to one of the router switches, and two 40 Gbps links, configured in trunk mode, connect the two router switches.
% \end{itemize}
\section{Methodology}
\label{method}

\subsection{Research questions}
Understanding the potential risks of pre-installed apps is essential.
To guide our research, we have established the following key questions.

\textbf{RQ1: To what extent do pre-installed apps leak sensitive data on low-cost devices?}
This question focuses on checking whether pre-installed apps perform activities, such as leaking sensitive data via the Internet or using other ways.
This helps us to understand not just if these apps pose risks, but also how they do so.

% \textbf{RQ2: To what extent do pre-installed apps on low-cost devices exhibit vulnerable libraries?} 
% This RQ allows us to identify vulnerable native libraries used by pre-installed apps by considering more recent vulnerabilities with known CVEs (Common Vulnerabilities and Exposures) related to Android native libraries. 

\textbf{RQ2: To what extent do pre-installed apps exhibit suspicious behaviors on low-cost devices?}
The aim of this question is to determine how widespread suspicious pre-installed apps are on affordable devices. Although previous research has shown that some of these apps may contain harmful code~\cite{10.1145/2590296.2590313}, a more comprehensive assessment is needed to quantify the extent of the problem in different manufacturers and regions.Through this question, we will explore the devices to identify apps having suspicious behaviors (including malware) as well as those using malicious URLs.

\textbf{RQ3: How prevalent are security misconfigurations in the manifest files of pre-installed apps on low-cost devices?}
Through this question, we explore apps to identify those exposing sensitive data across the exported components. 
% Additionally, we investigate the use of dangerous permissions in pre-installed third-party apps. 
% Since some permissions can be exploited to access sensitive data or to perform unauthorized actions, we analyze both the permissions declared by these apps and those that are automatically granted. 
% This helps us to understand how much control these apps really have.
% Finally, we identify the apps misusing deep links on their manifest files.
%How suspicious are pre-installed apps from affordable devices compared to those from non-affordable devices?
%Finally, RQ4 extends this analysis by comparing pre-installed apps on low-cost devices to those found on top-end models. 
%If affordability comes at the cost of weaker security practices, it is important to quantify these differences and highlight potential differences in consumer protection between different market segments.

% \textbf{RQ5: To what extent are dangerous permissions used in pre-installed third-party apps?}
% This RQ investigates the use of dangerous permissions in pre-installed third-party apps. 
% Since some permissions can be exploited to access sensitive data or to perform unauthorized actions, we analyze both the permissions declared by these apps and those that are automatically granted. 
% This helps us to understand how much control these apps really have.

By addressing these research questions, our study provides a clearer picture of the security landscape around pre-installed apps on low-cost mobile devices and highlights potential areas where better security measures are needed.

\subsection{PiPLAnD Design}
This section outlines the workflow of \texttt{\textit{PiPLAnD}}, a pipeline designed to inspect pre-installed apps using static analysis approaches. 
Figure~\ref{fig:PiPLAnD} gives an overview of the pipeline.
The source code of \texttt{\textit{PiPLAnD}} is available on our github repository\footnote{PiPLAnD source code: \url{https://anonymous.4open.science/r/PiPLAnD-9C86}}.
We present the details of the workflow of each module in the following sections.

\begin{figure}
\centerline{\includegraphics[width=0.7\linewidth]{PiPLAnD_design.png}}
%\end{adjustbox}
\caption{Overview of \texttt{\textit{PiPLAnD}}'s workflow.}
\label{fig:PiPLAnD}
\end{figure}

\textbf{Data collection.}
We have used seven (7) low-cost Android devices in this study.
\texttt{\textit{PiPLAnD}} first automatically extracts the pre-installed apps from a physical device using ADB commands when we plug the device into the computer. 
After extraction, we got a dataset of \num{1544} APK files from the overall devices.
This dataset includes system and third-party apps.

% \subsubsection{Static analysis}
\textbf{Module 1: Data leak detection.}\\
We integrate FlowDroid into \texttt{\textit{PiPLAnD}} to identify pre-installed apps that leak sensitive data.
FlowDroid is based on a list of sources and sinks for the detection.
This list contains Android API calls and Java methods.  
Thus, by default, it cannot detect data leaked from methods other than Android API calls and Java methods.
During our study, we have noticed that there are pre-installed apps that use custom methods from third-party libraries to send data over the Internet.
Because of that, we have identified apps that access to Internet by looking for \textit{android.permission.INTERNET} permission on their manifest files.
For these apps, we extracted all their methods that we have given to LLMs for source and sink categorization.
The resulting list is used, in combination with the default list of sources and sinks, to identify Internet apps leaking sensitive data over the Internet.
The default Source and Sink list is also used for detecting other types of data leakage in other apps.
% With our methodology, which consists of taking into account third-party libraries, we are able to identify this app.

% \subsubsection{Module 2: Data leak detection}
\textbf{Module 2: Behavior analysis.}\\
The behavior analysis focuses on pattern detection.
More specifically, we looked for patterns such as \textit{"pm install"}, \textit{"installPackage"}, etc., to detect installation behaviors, \textit{"logcat"} for apps collecting log data, \textit{"content://sms", "delete", "sendTextMessage", "Telephony.SMS\_RECEIVED"} for accessing SMS, deleting, sending, and listening to received SMS, etc.
Table~\ref{tab:behavPatterns} shows the full list of the patterns we used.
\begin{table}[!h]
    \centering
    \caption{The list of patterns}
    \begin{adjustbox}{width=0.5\linewidth}
    \begin{tabular}{|l|l|}
        \hline
         \textbf{Behaviors} & \textbf{Patterns} \\
         \hline
          & content://sms \\
          & delete \\
          Access / Delete / Send SMS & sendTextMessage \\
          & Telephony.SMS\_RECEIVED \\
          & Telephony.Sms \\
         \hline
         & "am start " \\
          & "chmod " \\
          Dangerous commands & "su "  \\
          & "sudo " \\
          & "rm -rf " \\
          %& Telephony.Sms \\
         \hline
          Log collection & logcat  \\
         \hline
          & "pm install " \\
         Installation indicators & installPackage \\
          &   \\
         \hline
         % & getPrimaryClip \\
         % Access clipboard data & setPrimaryClip\\
         % &  \\
         % \hline
    \end{tabular}
    \end{adjustbox}
    \label{tab:behavPatterns}
\end{table}
\texttt{\textit{PiPLAnD}} helps with this analysis by decompiling the APK file using the Androguard framework and by checking for patterns that match our list in the app code.
We only consider these patterns since they are the most obvious and are based on the existing literature, as well.
Then, it reports the findings in a JSON file.
This file is used for a manual analysis to confirm the behavior of the app based on the patterns detected in its code.

% \subsubsection{Module 3: Exported sensitive component identification}
\textbf{Module 3: Security misconfigurations on Manifest files.}\\
This module allows for the identification of security misconfigurations in the Android manifest files.
In this study, we only focus on identifying components exported that allow access to sensitive information.
Android allows restricting access to an exported component by using permissions.
We first extracted exported components from the manifest files for each app.
Then, we filter out and keep those who did not have permission to protect and restrict their access.
This gives us a list of class names with their full path in the app code.
Besides that, we have constructed a list of sensitive API calls based on the source and sink list from FlowDroid, and from the Androwarn study~\cite{debize2012androwarn}, in which authors look for sensitive data accessed.
We explored the list containing the classes of the exported components, and for each class, we analysed it using the SOOT framework~\cite{lam2011soot}.
For the analysis with SOOT, we used the sensitive API list and looked for each of them in the code of the corresponding class to know whether this component allows access to sensitive data.
If we did not find the presence of a sensitive API in the code, our analysis program constructs a call graph (CG) of the corresponding class to search whether the sensitive API is used in a method called in the code of the exported component.
The results are put into a JSON file, in which we have the corresponding class, the sensitive API, and the method where we found this API.

\section{Analysis Results}
\label{result}

In this section, we present the findings from seven (7) identified affordable devices that are primarily used in Africa. 
The main goal of this study is to evaluate the security of the pre-installed apps and to investigate the prevalence of suspicious one on low-cost devices.
Our analysis has revealed interesting findings.
Table~\ref{tabResults} summarizes these findings. %, which we will explore in the next paragraphes.
This section shows the results of the analysis by answering the different research questions.
\begin{table*}[!htbp]
\caption{Summary of the results.}
\begin{center}
\begin{adjustbox}{width=0.95\linewidth}
\begin{tabular}{ll|c|}
\cline{1-3}
%& & \multicolumn{3}{ c| }{\textbf{\# of apps (\%) in}} \\ \cline{1-5}
\multicolumn{1}{ |l| } {\textbf{Analysis Modules}} & \textbf{Behaviors}& \textbf{\# of apps (\%)} \\ \cline{1-3}
\multicolumn{1}{ |l| }{Exported sensitive components} & Components that allow to access sensitive data& \footnotesize{249 (16\%)} \\ \cline{1-3}
\multicolumn{1}{ |l| } {Leak of sensitive data}  & Leak of sensitive data & \footnotesize{145 (9\%)} \\ \cline{1-3}
% \multicolumn{1}{ |l| }{Library analysis} & Vulnerable native libs & \footnotesize{0 (0)} & \footnotesize{0 (0)} & \footnotesize{0 (0)} \\ \hline
% \multicolumn{1}{ |l| }{Manifest misconfiguration} &  &  &  &  \\ \cline{1-5}
\multicolumn{1}{ |l  }{\multirow{2}{*}{} } &
\multicolumn{1}{ |l| }{Dangerous commands} & \footnotesize{226 (15\%)} \\ \cline{2-3}
\multicolumn{1}{ |l  }{}                        &
\multicolumn{1}{ |l| }{Log collection} & \footnotesize{10 (0.7\%)} \\ \cline{2-3}
\multicolumn{1}{ |l  }{Suspicious behaviors}                        &
\multicolumn{1}{ |l| }{Silent installation behaviors} & \footnotesize{33 (2\%)} \\ \cline{2-3}
\multicolumn{1}{ |l  }{}                        &
% \multicolumn{1}{ |l| }{Read data from clipboard} &  \\ \cline{2-3}
% \multicolumn{1}{ |l  }{}                        &
\multicolumn{1}{ |l| }{Access / Send / Delete SMS} & \footnotesize{79 (5\%)} \\ \cline{1-3}
% \multicolumn{1}{ |l| }{Suspicious apps} & Dynamic analysis &  &  &  \\ \cline{1-5}
\end{tabular}
\end{adjustbox}
\label{tabResults}
\end{center}
\end{table*}

\subsection{Apps leaking sensitive data}
% \ap{Please give more details what PII is leaked and where, perhaps to provide package names.}\\
% \ad{Done.}

Mobile apps often handle sensitive data on the device. 
With the privileges that pre-installed apps have, they can perform harmful activities, including leaking this data.
%We have analysed the apps to identify those that leak sensitive data.
Our analysis, consisting of identifying apps that leak sensitive data, has identified several pre-installed apps on seven 7 different devices leaking data, such as Mobile Country Code (MCC), user location (longitude and latitude), device info, International Mobile Subscriber Identity (IMSI), IMEI (International Mobile Equipment Identity), etc.
Sensitive data is leaked in different ways, such as in SharedPreferences, in logs, in Intents, as well as on the network.
Overall, we have found around 9\% of the pre-installed apps on the devices leaking sensitive data, which represent 145 pre-installed apps. 
As an example, the app (\textit{com.transsion.statisticalsales}) sends sensitive information to a remote host by using third-party libraries with customized methods. 
As illustrated in Listing~\ref{listing:5}, the app collects location info, IMSI, IMEI, phone version, etc., and sends them to a remote server.
Pre-installed apps also leak data using other ways, such as SharedPreferences storage, device logs, etc.
These practices expose the user to many security and privacy problems. %Malicious apps could retrieve this data 
\begin{listing*}[!h]
\caption{App sending sensitive data to a remote server}
\label{listing:5}
\begin{minted}[fontsize=\scriptsize, baselinestretch=0.9, breaklines, breakanywhere]{java}
[...]
public class SSHttpClient {
    public static final String BASE_URLFLAG = "PCHttpClient";
    private static final String DEFAULT_BASE_SERVER = "https://asv.transsion.com:443/SaleStatistics/sendsale/sendSale";
    private static final String DEFAULT_INDIA_SERVER = "https://asvin.transsion.com:8080/SaleStatistics/sendsale/sendSale";
    protected static final String TAG = "SSHttpClient_";
    [...]
    public void RegisterInformation(final HttpCallback<HttprequestResult> httpCallback, boolean z) {
        RequestParams requestParams = new RequestParams();
        requestParams.put("ua", this.requestInfo.getUa());
        requestParams.put("screen", this.requestInfo.getScreen());
        requestParams.put("imsi", this.requestInfo.getImsi());
        requestParams.put("imei", this.requestInfo.getImei());
        requestParams.put("phone_version", this.requestInfo.getPhone_version());
        requestParams.put("platform", this.requestInfo.getPlatform());
        requestParams.put("device", this.requestInfo.getDevice());
        requestParams.put("lang", this.requestInfo.getLang());
        requestParams.put("timeStamp", this.requestInfo.getTimeStamp());
        requestParams.put("auth", this.requestInfo.getAuth());
        requestParams.put("lat", this.requestInfo.getLat());
        requestParams.put("lng", this.requestInfo.getLng());
        requestParams.put("client_type", this.requestInfo.getClient_type());
        requestParams.put("phone", this.requestInfo.getPhone());
        requestParams.put("client_version", this.requestInfo.getClient_version());
        requestParams.put("lac", this.requestInfo.getCELL_LAC());
        requestParams.put("cid", this.requestInfo.getCELL_CID());
        mClient.post(z ? DEFAULT_INDIA_SERVER : DEFAULT_BASE_SERVER, requestParams, new AsyncHttpResponseHandler() { 
            [...]
        });
    }
}
\end{minted}
\end{listing*}
\begin{figure}[H]
\begin{tcolorbox}[
    colframe=brown!30!black, colback=red!2, 
    title=Answer to RQ1,
    fonttitle=\bfseries,
]
The results show that several pre-installed apps on the three devices exhibit harmful activities such as leaking sensitive data, exposing the users to severe risks.
\end{tcolorbox}
\end{figure}

% \subsection{Vulnerable native libraries on pre-installed apps}
% We found nothing about this question.
% In an Android app, developers often rely on native code to implement some tasks in C/C++. 
% Once compiled, the native code produces a shared object file (.so) known as a native library.
% Several native libraries are known to be vulnerable; thus, CVE numbers are assigned to them.
% To detect whether pre-installed apps utilize vulnerable native code, we compared native libraries from the apps for similarity to a set of vulnerable libraries collected from the Internet based on matching patterns.
% The result shows a single and the same app using a known native library on the three devices.
% This is a very small number, but not negligible since it could have a very big impact on compromising users' security.

% \begin{figure}[!h]
% \begin{tcolorbox}[
%     colframe=brown!30!black, colback=red!2, 
%     title=Answer to RQ2,
%     fonttitle=\bfseries,
% ]
% We have found only 1 pre-installed app; the same one, using a vulnerable native library on each low-cost device.
% \end{tcolorbox}
% \end{figure}

\subsection{Suspicious behaviors on pre-installed apps}
Since mobile devices come with pre-installed apps, these may contain harmful code that can compromise the security of users~\cite{10.1145/2590296.2590313}.
% We first extracted the URLs used by the apps from the APK files for each devices.
% Then, we have analysed the URLs on VirusTotal to identify the potentially malicious ones.
% We considered a URL as malicious only if it is detected by at least two (2) VirusTotal antivirus.
% As results, we have found one app using malicious URLs on each device.
% Each of these apps use at least two malicious URLs (2 for the apps on Infinix and Tecno, and 3 for the app on itel).
% Similar to the URL analysis, we have also analysed the APK files on VirusTotal to detect potential malware.
% The results showed that two APK files (1 for Infinix and 1 for itel) have been detected as malicious by one antivirus each of them.
% However, since we only considered as malicious apps those that have been detected by at least two antivirus programs, we did not consider these apps as malware.
We have looked for apps having suspicious behaviors.
Several pre-installed malware have been identified to have silent installation behaviors~\cite{10.1145/2590296.2590313}. Others steal sensitive data using different ways~\cite{10.1145/2590296.2590313,contDev,trojanAnd}.
To identify these kinds of malware, we focus the analysis in detecting patterns.
%Our analysis revealed several pre-installed apps having suspicious behaviors, such as collecting log data (X\% of the apps), installation behaviors (X\%), and reading data from the clipboard (X\%).
When a pre-installed app declares the \textit{INSTALL\_PACKAGES} permission, it has the ability to install an app without the user's knowledge.
It can use \textit{android.content.pm.PackageManager.installPackage()} or \textit{Runtime.exec( 'pm install')} to silently install the app~\cite{10.1145/2590296.2590313}.
Our analysis has revealed 33 pre-installed apps having this silent installation behavior on the overall devices.
Several pre-installed apps have access to the SMS provider \textit{content://sms}.
Some of them delete SMS or declare the \textit{SEND\_SMS} and \textit{RECEIVE\_SMS} permissions with the broadcast action \textit{Telephony.SMS\_RECEIVED}, allowing them to listen and send messages. We have found around 79 apps pre-installed on these low-cost devices, sending, deleting, and/or reading SMS contents.
Furthermore, at least 10 pre-installed apps can access and collect the logcat content. Since they have déclared the  \textit{READ\_LOGS} permission, they can access the overall logcat content, including logs from other applications.
In addition to this, we have found several apps that execute dangerous commands (226 apps). %and that can access and read the content of the clipboard (X).

\begin{figure}[h!]
\begin{tcolorbox}[
    colframe=brown!30!black, colback=red!2, 
    title=Answer to RQ2,
    fonttitle=\bfseries,
]
Low-cost Android devices ship pre-installed apps with suspicious behaviors, including sending/deleting/reading SMS, executing dangerous commands, accessing overall logcat memory, and having silent installation behaviors.
\end{tcolorbox}
\end{figure}

\subsection{Security misconfigurations on the manifest files}
%In this section, we checked exported components and data leakage.
\textbf{Exported sensitive components.}
Android apps often export components, such as activities, services, receivers, and providers, explicitly by setting \textit{android:exported="true"} or implicitly by declaring an \textit{intent-filter} in the manifest file.
When it is the case, the app allows other apps to launch the component~\cite{exported}.
The access to exported components is often restricted using permissions~\cite{component}. If there is permission for an exported component, the app that wants to launch it should declare this permission.
If an app exports a component without properly enforcing permission, any app could launch it or access sensitive data it contains~\cite{cweExpComp}.
When a component allows access to sensitive data, we call it a sensitive component.
Our analysis has revealed that several pre-installed apps have sensitive components exported.
As illustrated in Table~\ref{tabResults}, we have found around 16\% of the pre-installed apps, representing 249 different app versions, having exported sensitive components on the low-cost devices, without any restriction or protection mechanism.
It means that these devices embed apps that allow other apps to potentially access sensitive information, exposing the user to security and privacy problems.
For example, we have found a pre-installed app (\textit{com.transsion.carlcare}) having this method (Listing~\ref{listing:1}), from an exported activity (\textit{com.transsion.carlcare.WarrantyCardActivity}).
This method allows access to location information (lines 13 and 14).
In the Android manifest file of the app, this component is clearly exported; however, it is protected by no mechanism, facilitating its easy access and its potential exploitation.
Another example shows an exported ContentProvider (\textit{com.sprd.providers.photos.SpecialTypesProvider}) found in the app (\textit{com.sprd.providers.photos}) that allows access to media files, contained in an external storage, from a URI (Listing~\ref{listing:3}, line 10). 
This exported component does not have a mechanism that restricts its access by other apps.

\begin{listing}[!h]
\caption{Exported activity accessing sensitive information}
\label{listing:1}
\begin{minted}[fontsize=\scriptsize]{java}
public void R1() {
    [...]
    HashMap<String, String> map = new HashMap<>();
    if (TextUtils.isEmpty(this.f15263g0)) {
        map.put("imei", listA.get(0));
    } else {
        map.put("imei", this.f15263g0);
    }
    map.put("imsi", wd.c.g());
    map.put("lang", getResources().getConfiguration().locale.
                                        toString());
    if (this.f15282z0 != null) {
        map.put("lat", this.f15282z0.getLatitude() + "");
        map.put("lng", this.f15282z0.getLongitude() + "");
    }
    map.put("phone_version", a2());
    map.put("screen", getResources().getDisplayMetrics().widthPixels + "*" + getResources().getDisplayMetrics().heightPixels);
    map.put("ua", Build.BRAND + "-" + Build.MODEL);
    [...]
}
\end{minted}
\end{listing}
% \begin{listing}[!h]
% \caption{The manifest of the exported activity}
% \label{listing:2}
% \begin{minted}[fontsize=\tiny]{java}
% <?xml version="1.0" encoding="utf-8"?>
% <manifest xmlns:android="http://schemas.android.com/apk/res/android"
%     [...]
%     <activity android:name="com.transsion.carlcare. WarrantyCardActivity"
%         android:exported="true"
%         android:launchMode="singleTask"
%         android:screenOrientation="behind" android:configChanges="smallestScreenSize|...">
%         <intent-filter>
%             <action android:name="com.transsion.carlcare.action. WARRANTY_CARD"/>
%             <category android:name="android.intent.category.DEFAULT"/>
%         </intent-filter>
%         <intent-filter>
%             <action android:name="android.intent.action.VIEW"/>
%             <category android:name="android.intent.category.DEFAULT"/>
%             <category android:name="android.intent.category.BROWSABLE"/>
%             <data
%                 android:scheme="carlcare"
%                 android:host="app.transsion.com"
%                 android:path="/warranty/detail"/>
%         </intent-filter>
%     </activity>
%     [...]
% </manifest>
% \end{minted}
% \end{listing}
\begin{listing}[!h]
\caption{Exported ContentProvider accessing external storage files}
\label{listing:3}
\begin{minted}[fontsize=\tiny]{java}
[...]
public final class SpecialTypesProvider extends ContentProvider {
    [...]
    private static final Uri EXTERNAL_CONTENT_URI = MediaStore.Files.getContentUri("external");
    private static final String[] SPECIAL_TYPE_PROJECTION = {"_data", "owner_package_name"};
    [...]
    private int getCameraType(long j) {
        Log.d(TAG, "mediaStoreId = " + j);
        boolean z = true;
        Cursor cursorQuery = getContext().getContentResolver().query( EXTERNAL_CONTENT_URI, SPECIAL_TYPE_PROJECTION, "_id=?", new String[]{String.valueOf(j)}, null);
        if (cursorQuery != null) {
            try {
                if (cursorQuery.moveToFirst()) {
                    String string = cursorQuery.getString(0);
                    Log.d(TAG, "mediaPath = " + string);
                    String string2 = cursorQuery.getString(1);
                    if (!GOOGLE_PHOTOS_PACKAGE_NAME.equals( string2) && !GOOGLE_GALLERY_PACKAGE_NAME.equals( string2)) {
                        z = false;
                    }
                    Log.d(TAG, "ownerPackageName = " + string2 + ", isGoogleCreatedImage = " + z);
                    try {
                        ExifInterface exifInterface = new ExifInterface();
                        exifInterface.readExif(string);
                        iIntValue = z ? 0 : exifInterface.getTagIntValue( ExifInterface.TAG_CAMERATYPE_IFD). intValue();
                        Log.d(TAG, "getCameraType cameraType = " + iIntValue);
                    } catch (Exception e) {
                        Log.d(TAG, "Exception occurs, mediaPath = " + string + ". ex = " + e);
                    }
                }
            } finally {
                if (cursorQuery != null) {
                    cursorQuery.close();
                }
            }
        }
        return iIntValue;
    }
    [...]
}
\end{minted}
\end{listing}
% \begin{listing}[!h]
% \caption{The manifest of the exported ContentProvider}
% \label{listing:4}
% \begin{minted}[fontsize=\tiny]{java}
% <?xml version="1.0" encoding="utf-8"?>
% <manifest xmlns:android="http://schemas.android.com/apk/res/android"
%     [...]
%     <provider
%         android:label="@string/provider_label" android:name="com.sprd.providers.photos. SpecialTypesProvider"
%         android:exported="true" android:authorities="com.sprd.android.providers. SpecialTypesProvider"
%         android:syncable="false"
%         android:grantUriPermissions="true">
%         <intent-filter>
%             <action android:name="com.google.android.apps.photos. OEM_PROVIDER"/>
%             <action android:name="com.google.android.apps.photosgo. OEM_PROVIDER"/>
%         </intent-filter>
%     </provider>
%     [...]
% </manifest>
% \end{minted}
% \end{listing}

% \textbf{Permissions on third-party pre-installed apps.}
% Manufacturers also pre-install third-party apps (non-system apps) to provide additional services.
% These apps could be over-privileged and access sensitive information with dangerous permissions.
% %PiPLAnD extracts these permissions from these apps to see the extent of their usage.
% When extracting APK files, we have found pre-installed 3rd-party apps on only Infinix and itel, with 3 and 12 apps respectively.
% As we can see in Fig.~\ref{Permsitel3rd}, every 3rd-party app declared dangerous permissions, with an average of 6.7 dangerous permissions per app in itel, representing 23\% of the average of overall permissions declared by each app (resp. 8.3 per app in Infinix, representing 23\% as well).
% Tables~\ref{tabPermsInf} and~\ref{tabPermsitel} give an overview of these dangerous permissions for pre-installed 3rd-party apps in Infinix and itel.

% \begin{figure}[!h]
% \centerline{\includegraphics[width=\linewidth]{figures/itel_permissions.png}}
% \caption{Number of permissions (dangerous and non-dangerous) declared by 3rd-party apps in itel. \textit{Note: The bars framed in green represent the only pre-installed 3rd-party apps in Infinix, and they have the same number of permissions as in itel.}}
% \label{Permsitel3rd}
% \end{figure}
% % \begin{figure}[!h]
% % \begin{tcolorbox}[
% %     colframe=red!30!black, colback=red!3, 
% %     title=Answer to RQ3,
% %     fonttitle=\bfseries,
% % ]
% % Pre-installed third-party apps always use dangerous permissions.
% % These allow them to access more sensitive resources in the Android framework.
% % \end{tcolorbox}
% % \end{figure}

% \begin{table}[!h]
% \caption{Overview of dangerous permissions used by pre-installed 3rd-party apps in Infinix.}
% \begin{center}
% \begin{adjustbox}{width=\linewidth}
% \begin{tabular}{|c|c|p{3in}|}
% \hline
% \textbf{App name}&\textbf{Package name} & \textbf{Dangerous permissions} \\
% %\textbf{}&\textbf{found} & \textbf{analyzed} & \textbf{URLs (\%)} \\
% \hline
%  & & GET\_ACCOUNTS, WRITE\_CONTACTS, CALL\_PHONE, RECORD\_AUDIO, READ\_CALENDAR, BLUETOOTH, \\
% FacebookLite & com.facebook.lite & WRITE\_CALENDAR, READ\_PHONE\_STATE, READ\_PHONE\_NUMBERS, ACCESS\_COARSE\_LOCATION,\\
%  & & ACCESS\_FINE\_LOCATION, READ\_CONTACTS , CAMERA,  WRITE\_EXTERNAL\_STORAGE,\\
%  & & READ\_MEDIA\_VISUAL\_USER\_SELECTED\\
% \hline
% & & RECORD\_AUDIO,  CAMERA, READ\_MEDIA\_IMAGES, WRITE\_EXTERNAL\_STORAGE, POST\_NOTIFICATIONS,\\
%  TikTok Lite& com.zhiliaoapp.musically.go&READ\_MEDIA\_VIDEO, READ\_CONTACTS , WRITE\_EXTERNAL\_STORAGE\\
% \hline
% Compass& com.transsion.compass& ACCESS\_COARSE\_LOCATION, ACCESS\_FINE\_LOCATION\\
% \hline
% \end{tabular}
% \end{adjustbox}
% \label{tabPermsInf}
% \end{center}
% \end{table}

% \begin{table}[!h]
% \caption{Overview of dangerous permissions used by pre-installed 3rd-party apps in itel.}
% \begin{center}
% \begin{adjustbox}{width=\linewidth}
% \begin{tabular}{|c|c|p{3in}|}
% \hline
% \textbf{App name}&\textbf{Package name} & \textbf{Dangerous permissions} \\
% %\textbf{}&\textbf{found} & \textbf{analyzed} & \textbf{URLs (\%)} \\
% \hline
%  & & GET\_ACCOUNTS, WRITE\_CONTACTS, CALL\_PHONE, RECORD\_AUDIO, READ\_CALENDAR, BLUETOOTH, \\
% FacebookLite & com.facebook.lite & WRITE\_CALENDAR, READ\_PHONE\_STATE, READ\_PHONE\_NUMBERS, ACCESS\_COARSE\_LOCATION,\\
%  & & ACCESS\_FINE\_LOCATION, READ\_CONTACTS , CAMERA,  WRITE\_EXTERNAL\_STORAGE,\\
%  & & READ\_MEDIA\_VISUAL\_USER\_SELECTED\\
% \hline
% & & RECORD\_AUDIO,  CAMERA, READ\_MEDIA\_IMAGES, WRITE\_EXTERNAL\_STORAGE, POST\_NOTIFICATIONS,\\
%  TikTok Lite& com.zhiliaoapp.musically.go&READ\_MEDIA\_VIDEO, READ\_CONTACTS , WRITE\_EXTERNAL\_STORAGE\\
% \hline
%  & & ACCESS\_BACKGROUND\_LOCATION, \\
%  Weather & com.rlk.weathers & ACCESS\_COARSE\_LOCATION, \\
%  &  & ACCESS\_FINE\_LOCATION, POST\_NOTIFICATIONS \\
% \hline
% && BLUETOOTH\_CONNECT, READ\_MEDIA\_IMAGES, BLUETOOTH, READ\_MEDIA\_AUDIO, BLUETOOTH\_SCAN, \\
% && READ\_CALL\_LOG, READ\_PHONE\_STATE, CAMERA, READ\_CONTACTS, ACCESS\_FINE\_LOCATION, \\
% Welife & com.smartlife.nebula& ACCESS\_COARSE\_LOCATION, POST\_NOTIFICATIONS, BLUETOOTH\_ADVERTISE, ANSWER\_PHONE\_CALLS, READ\_EXTERNAL\_STORAGE, READ\_MEDIA\_VIDEO,  \\
%  &  & WRITE\_EXTERNAL\_STORAGE, RECORD\_AUDIO \\
% \hline
% && BLUETOOTH\\
% WOW FM & com.funbase.xradio & WRITE\_EXTERNAL\_STORAGE \\
% && RECORD\_AUDIO \\
% \hline
% && READ\_MEDIA\_AUDIO, READ\_EXTERNAL\_STORAGE, READ\_MEDIA\_VIDEO, POST\_NOTIFICATIONS, \\
% Notepad & com.transsion.notebook & READ\_MEDIA\_IMAGES, WRITE\_EXTERNAL\_STORAGE,\\
% && RECORD\_AUDIO\\
% \hline
% && WRITE\_CALENDAR, READ\_CONTACTS, \\
% Calendar & com.transsion.calendar & POST\_NOTIFICATIONS, \\
%  & & GET\_ACCOUNTS, READ\_CALENDAR \\
% \hline
% && WRITE\_EXTERNAL\_STORAGE \\
% Visha Player & com.transsion.magicshow & POST\_NOTIFICATIONS  \\
% && READ\_EXTERNAL\_STORAGE  \\
% \hline
% && POST\_NOTIFICATIONS, READ\_EXTERNAL\_STORAGE\\
% AHA Games & net.bat.store & READ\_MEDIA\_IMAGES, RECORD\_AUDIO, CAMERA, WRITE\_EXTERNAL\_STORAGE,  \\
% \hline
% && BLUETOOTH, READ\_PHONE\_STATE, RECORD\_AUDIO, \\
% Recorder & com.transsion.soundrecorder & READ\_MEDIA\_AUDIO, WRITE\_EXTERNAL\_STORAGE, \\
% && POST\_NOTIFICATIONS, READ\_EXTERNAL\_STORAGE\\
% \hline
% Compass& com.transsion.compass& ACCESS\_COARSE\_LOCATION, ACCESS\_FINE\_LOCATION\\
% \hline
% Calculator & com.transsion.calculator & ACCESS\_COARSE\_LOCATION \\
% \hline
% \end{tabular}
% \end{adjustbox}
% \label{tabPermsitel}
% \end{center}
% \end{table}

% \textbf{Deep links.} A deep link allows access to specific content within an
% app.
% There are different types of deep links~\cite{deepLinkMASTG}:
% \begin{itemize}
%     \item Deep links with custom URL schemes:
%      \begin{verbatim}
%     <intent-filter>
%         ...
%         <data android:scheme="appname" 
%         android:host="path" />
%     </intent-filter>
%     \end{verbatim}

%     Here, \textit{appname://path} is used to access the app content.
%     This type of deep link, also called Scheme URL, is the least secure~\cite{203702}. 
%     \item Deep links with http:// or https:// schemes:
%     \begin{verbatim}
%     <intent-filter>
%         ...
%         <data android:scheme="http" 
%         android:host="www.myapp.com" 
%         android:path="/my/app/path" />
%         <data android:scheme="https" 
%         android:host="www.myapp.com" 
%         android:path="/my/app/path" />
%     </intent-filter>
%     \end{verbatim}
%     This one, to be securely implemented, should include \textit{android:autoVerify="true"} attribute, which allows (1) the Android system to automatically verify the declared deep links during the installation and (2) the app to self-designate as the default handler of the given link~\cite{deepLinkAndDoc,203702}. 
% \end{itemize}
% Table~\ref{tab:deepLink} shows that 21, 25, and 23 apps in Infinix, itel, and Tecno, respectively,  use custom URL schemes to access app content without restrictions.
% Furthermore, few apps (9 for Infinix, 3 for itel, and 2 for Tecno) use deep links with HTTP/HTTPS schemes without integrating the \textit{"autoVerify"} flag. 
% These two types of deep links are not verified by the Android system to ensure their reliability.

% \begin{table}[!h]
% \caption{Statistics about deep links}
% \begin{center}
% \begin{adjustbox}{width=\linewidth}
% \begin{tabular}{|l|c|c|c|c|}
% \hline
% % \textbf{Signed by }&\multicolumn{3}{|c|}{\textbf{Device brands}} \\
% % \cline{2-4} 
% \textbf{} & \textbf{\textit{\# App in}}& \textbf{\textit{\# App in}}& \textbf{\textit{\# App in}} \\%& \textbf{\textit{\# success.}} \\
% \textbf{Types of deep links} & \textbf{\textit{Infinix (\%)}}& \textbf{\textit{itel (\%)}}& \textbf{\textit{Tecno (\%)}} \\%& \textbf{\textit{ verified (\%)}} \\
% \hline
% Custom URL schemes & 21 (10\%) & 25 (10\%) & 23 (9\%) \\%& 0 (0\%) \\
% \hline
% HTTP/HTTPS w/o "autoVerify" & 9 (4\%)& 3 (1\%) & 2 (1\%) \\%& 0 (0\%) \\
% \hline
% HTTP/HTTPS w/ "autoVerify" & 16 (7\%) & 5 (2\%) & 5 (2\%) \\%& 16 (100\%) \\
% \hline
% \end{tabular}
% \end{adjustbox}
% \label{tab:deepLink}
% \end{center}
% \end{table}
\begin{figure}[!h]
\begin{tcolorbox}[
    colframe=brown!30!black, colback=red!2, 
    title=Answer to RQ3,
    fonttitle=\bfseries,
]
The results have revealed several pre-installed apps exporting sensitive components, including activities, services, content providers, and receivers, on low-cost devices. This potentially puts users at risk, as their data could be accessed by third parties.
% Additionally, they use deep links that increase the attack surface.
% Besides that, pre-installed third-party apps always use plenty of dangerous permissions, allowing them to access more sensitive resources in the Android framework.
% In sum, many pre-installed apps in low-cost devices pose risks to the user by exposing sensitive data.
\end{tcolorbox}
\end{figure}

%\section{Discussion}
\label{sec:discuss}


\begin{comment}
    

In this section, we discuss relevant aspects of the our approach.

\textbf{Can LLMs Interpret Ethical Information in Textual Prompts? \todo{OR} Ethical reasoning capabilities of LLMs (since this is similar with RQ1.)}

\todo{The answer will be a summary of the results, not sure. Patrizio, please check this.}


\textbf{Toward Model-Based Ethical Evaluation}

This is also 


\textbf{Constructing Ethical Profiles through Passive Interaction}

The end goal of our approach is to automatically generate user ethical profiles, leveraging the ethical reasoning abilities of LLMs to analyze user behavior in real time. If LLMs can consistently interpret ethically relevant scenarios, then they can serve as inference engines for future systems that monitor user interactions (e.g., via AR glasses or smart agents) and extract preference signals from context.

These profiles need not be static or deterministic. Rather, they can evolve as a form of passive ethical memory that captures the user’s stance across dimensions such as fairness, harm, or loyalty. This enables the design of agents that reason within user-defined moral constraints, without requiring explicit rule engineering or ongoing manual feedback.


\textbf{Implications for Software Engineering Practice}

From a software engineering perspective, our approach serves as a practical and scalable method to incorporate an ethical reasoning mechanism into the development of software systems. By using LLMs as modular evaluators, ethical reasoning mechanisms can be embedded as an operational layer in software systems, enabling them to make decisions that align with user preferences.


*******************
This framework aligns with software engineering research in two directions. First, it provides a method for rapidly probing the ethical capabilities of generative systems that may later be integrated into user-facing software agents. Second, it sets the foundation for ethically-informed profiling methods that can be automated, auditable, and adaptable to different application domains, including healthcare assistants, AR interfaces, and autonomous agents. By decoupling the evaluation from manual supervision, we advance toward scalable, trustworthy AI modules that operate within well-defined moral boundaries.

\end{comment}


%\subsection{RQ1: Do LLMs demonstrate the capacity for ethical reasoning when presented with ethically charged scenarios?}

\textbf{RQ1.} Our findings provide evidence that state-of-the-art LLMs can engage in ethical reasoning when presented with complex, real-world made \major{explanations of acceptability}. Without fine-tuning or examples, models consistently identified the most applicable ethical theory and made acceptability explanations with substantial inter-model agreement. This capability suggests that LLMs possess an implicit grasp of moral reasoning principles, grounded in their pretraining on large-scale textual corpora. From an SE perspective, this opens the door to using LLMs as ethical reasoning modules in decision-making pipelines, such as requirement negotiation, user modeling, or system auditing.

%\subsection{RQ2: Do LLMs Reason Consistently Across Models and Scenarios?}

\smallskip

\textbf{RQ2.} Quantitative results showed that models converge more strongly on binary moral acceptability (86.7\% BAR) than on ethical theory classification (73.3\% TCR). While this difference reflects the higher abstraction level of theoretical judgments, the level of agreement observed is non-trivial. The scenario-dependent variability in TCR reveals an important feature: model disagreement tends to reflect ethical ambiguity inherent in the scenario rather than arbitrary noise. This suggests that ensemble disagreement can be used as a proxy for moral uncertainty, enabling software systems to trigger escalation or human intervention when LLMs disagree sharply.

%\subsection{RQ3: How Do LLMs Compare to Human Experts?}

\smallskip

\textbf{RQ3.} %Our comparison revealed substantial alignment between LLMs and human experts. 
\major{Overall, LLMs exhibit non-trivial agreement with experts that is more pronounced in prevalent classes and weaker on rare or edge cases.} Scenarios that elicited strong agreement among experts tended to also show high inter-model LLM agreement, and vice versa. This convergence reinforces the reliability of LLMs in interpreting familiar or structurally simple moral scenarios. Divergences, especially in edge cases, underscore the importance of hybrid systems that combine automated reasoning with human oversight. For SE applications involving legal, regulatory, or safety-critical implications, LLM-based profiling should not be deployed as an isolated decision-maker but as a complementary module.

%\subsection{RQ4: What Characterizes the Structure of LLM Moral Expanations?}

\smallskip

\textbf{RQ4.} Qualitative analyses demonstrated that LLMs generate explanations that are lexically diverse but conceptually coherent. Despite low textual similarity across models, explanations consistently aligned with the chosen moral theory in over 90\% of cases. Models blended terminology from multiple ethical traditions in natural, context-sensitive ways, %mirroring 
\major{tending to reflect} how human reasoners combine principles, consequences, and character-based considerations. This expressive flexibility is critical for ethical profiling, as it enables the detection of user-aligned reasoning patterns across different moral framings. Moreover, the compactness of most explanations (single sentences) and their theoretical consistency suggest that LLMs are capable of producing tractable, auditable moral outputs suitable for runtime interpretation and logging.
%required by metareview
\begin{tcolorbox}[colback=gray!10,
                  colframe=black,
                  arc=4mm,
                  boxrule=0.8pt,
                  left=2mm, right=2mm, top=1mm, bottom=1mm]
                  \major{\textbf{Agreement $\ne$ correctness.} Our design surfaces stability and ambiguity signals, it does not certify normative accuracy. In SE practice, high agreement supports automation with audit, while low agreement recommends human-in-the-loop escalation.}
\end{tcolorbox}                  

\smallskip


\textbf{Limitations and Scope of Validity.} While promising, our findings are bounded by some limitations:

\smallskip

\noindent\textit{Theoretical coverage.} We focus on three major ethical theories utilitarianism, deontology, and virtue ethics due to their widespread adoption in software engineering practice and education~\cite{vaniea2018securitytrolley}. These ethical theories provide well-established foundations for analyzing ethical dilemmas in technology contexts. While alternative theories such as care ethics or contractualism are less commonly applied, they offer valuable perspectives that could enrich ethical analyses. Future work may explore the integration of these additional frameworks to capture a broader spectrum of moral reasoning in software engineering.

\smallskip

\noindent\textit{Scenario framing.} Our prompts use concise, decontextualized scenarios. Richer formats (e.g., dialogues, system logs) may affect model interpretation. Our current prompts are decontextualized statements. An important next step is to apply the same ethical reasoning pipeline (Figure~\ref{fig:approach2}) to richer input modalities, including:
(i) chat transcripts from developer-agent interactions;
(ii) logs of user decisions in ethically sensitive configurations;
(iii) behavioral signals from simulation environments or system telemetry.
This would move the profiling process closer to real-time, context-aware ethical inference.

\smallskip

\noindent\textit{Zero-shot constraints.} All reasoning is performed without memory or clarification. Interactive or multi-turn reasoning may yield different profiles. An ethical profile need not be static. As users interact with a system, their decisions may reveal shifts in priorities, trade-offs, or ethical boundaries. Future work should implement an ethical memory module that incrementally updates a user's profile over time, capturing both stable dispositions and contextual shifts. This requires designing a temporal profiling architecture that tracks ethical indicators across scenarios and resolutions.

\smallskip
   
\noindent\textit{Agreement $\ne$ correctness.} Convergence does not imply normative accuracy. Human biases and model alignment may coincide but remain ethically questionable. In real deployments, users may reject or revise the moral judgments made by the system. Building on our current architecture, we envision an interactive loop in which: (i) the system proposes an ethical explanation; (ii) the user confirms, modifies, or rejects the reasoning; (iii) the profile is updated accordingly. This would enable both user agency and model refinement over time, reducing the risk of misaligned ethical personalization. 

\smallskip

\major{\noindent\textit{Prompt sensitivity.} Our zero-shot, single-turn protocol deliberately controls for instruction complexity; however, model behavior can still be sensitive to seemingly innocuous variations in prompt phrasing, formatting, or input length. We therefore treat prompt sensitivity as a threat to validity and an explicit boundary of our claims. A systematic sensitivity analysis is left as future work. In practice, we recommend freezing prompt templates in repositories and reporting all formatting details that might affect reproducibility.}

\smallskip

Even if the findings support the potential viability, these limitations suggest caution in direct deployment and highlight the need for further validation before integrating LLM-based profiling into high-stakes SE systems. Our results position LLMs as viable components for modular ethical reasoning in SE. Possible use cases include: \textit{decision auditing} for moral rationales generation for SE tool outputs (e.g., in requirements prioritization or resource allocation); \textit{autonomy triage} to route decisions to humans when LLMs disagree, reducing risk in ethically charged contexts; \textit{agent personalization} to tailor behavior of autonomous SE agents based on learned ethical user profiles. More broadly, the ability to extract consistent moral structure from language enables a shift from static ethics-as-checklists to adaptive, traceable, and user-aligned ethical cognition in engineered systems.
\section{Discussions}
\label{discuss}
In this section, we discuss the results from the analysis of the devices. 

\textbf{Sensitive data leakage.}
During our analysis, we have found several pre-installed apps leaking sensitive data.
Several of them send the data to a remote host by using Android API calls.
Others use third-party libraries that use customized methods to avoid existing detection (e.g, \textit{com.transsion.statisticalsales}).
We have considered these customized methods and added them to FlowDroid, and we have detected apps sending sensitive data over the Internet with these methods.
Considered as malware by some blog posts~\cite {bobe}, the app (\textit{com.transsion. statisticalsales}) is not detectable by malware scanners such as VirusTotal.
It silently collects data (phone version, IMSI, user location, CID (Cell Tower ID), LAC (Location Area Code), etc.) and sends it to a remote host. This suspicious behavior compromises users' security and privacy.
Several other apps get and store sensitive data in internal storage, such as SharedPreferences, or log it in the logcat memory. 
These leaked data can be used by malicious actors for user profiling, user tracking~\cite{leakApp}, or accessed by other apps since pre-installed apps can share each other data when they have the same \textit{sharedUserId} and have signed with the same  certificate~\cite{10.1145/2590296.2590313}.

\textbf{Suspicious behaviors.}
We have found several pre-installed apps having suspicious behaviors, including silent installation behavior, send/delete SMS, etc.
These are done without the user's knowledge and may have negative consequences. 
Indeed, when an app is able to install another app silently, it may install a malicious app from a malicious remote server or via dynamic code loading.
This technique is often used by malicious actors to infect Android devices and avoid detection~\cite{10.1145/2590296.2590313}.
This is possible only with system apps, since non-system apps cannot declare the required permission.
When a pre-installed app has the ability to send and delete messages without the user's interaction, this can lead to the theft of sensitive data.
Several malware are known to use this technique. 
For instance, a variant of the Triada malware has been found pre-installed in Android devices~\cite{contDev}. 
This malware performs several actions, among which we have enabling premium SMS services, intercepting, sending, and deleting messages.

\textbf{Security misconfiguration.}
Our analysis has revealed several apps exporting components that allow other apps to potentially access sensitive data, without any protection.
The Android framework proposes permission levels, including normal, dangerous, and signature, to protect and restrict access to components~\cite{permsComp}.
The absence of protecting exported components is a known vulnerability from the community~\cite{permsComp,cweExpComp}.
When an app exports a component without any restriction, it allows other apps to access it. From this, a malicious app may access to sensitive resources,  as mentioned in the Common Weakness Enumeration (CWE - 926)~\cite{cweExpComp}.
The app can be victim to an attack named confused deputy attack~\cite{10.5555/2028067.2028089,9152633}, as well as a malicious app can abuse the privileges that these components have to gain unauthorized access to these resources~\cite{permsComp,9043935}.

\section{Related Work}
\label{related}

Zheng et al.~\cite{10.1145/2590296.2590313} presented a tool named DroidRay that extracts statically and dynamically pre-installed apps from 250 firmware downloaded from forums and website. 
The static extraction consists of extracting pre-installed apps directly from the firmware images. As for the dynamic extraction, it consists of flashing the image into a device and then extracting the pre-installed apps using ADB commands.
DroidRay performs pre-installed app analysis and system analysis.
For the pre-installed app analysis,  they extracted the “SharedUserId” attribute from the AndroidManifest.xml and the signature information from the RSA file. Then, they compare them with the default signatures they found from the AOSP. They also analyzed the apps in VirusTotal do detect potential malware, and applied a filter to retain apps having dangerous permissions or silent installation behavior.
For the system analysis, DroidRay performs static and dynamic analysis of the Android firmware by doing a system signature vulnerability detection, a network security analysis, and a privilege escalation vulnerability detection. 
In our proposed pipeline, we leveraged the technique of dynamic extraction to directly extract pre-installed apps from a physical device using ADB commands rather than collecting firmware and flashing them into a device.
We also used the same technique concerning malware detection, consisting of analysing APK files in VirusTotal.

Mitchell et al.~\cite{10.1145/2557547.2557557} designed DexDiff, a system for assessing the security impacts of vendor customization to the official Android system.
DexDiff helps the security analyst, who first retrieves the pre-installed apps and libraries from the phone and then builds their corresponding base binaries from the release branch in AOSP on which the phone is based.  
DexDiff compares each pair of these binaries obtained and evaluates the security impacts of individual modifications.
The bit and only similarity that this study has compared to our approach is that it extracts pre-installed apps from the phone. However, the proposed tool did not automate this process. The apps are supposed to be extracted before using DexDiff.
Contrary to our approach, PiPLAnD automated the process of extraction and analysis.


Elsabagh et al.~\cite{251554} proposed a static analysis tool named FIRMSCOPE, to identify unwanted functionality in pre-installed apps by analyzing the Android firmware.
FIRMSCOPE extracts pre-installed apps from the Android firmware and then performs a taint analysis with context-sensitive, flow-sensitive, field-sensitive, and partially object-sensitive.
Specifically, it focuses exclusively on identifying the increase in privileges. 


Gamba et al.~\cite{9152633}  presented a large-scale study of Android pre-installed using crowd-sourcing methods.
In this study, the authors built an Android app, Firmware Scanner, that looks for and extracts pre-installed apps when installed on a device.
This study performs permission analysis using Androguard~\cite{desnos2011androguard}, static analysis leveraging existing tools such as Androwarn, FlowDroid~\cite{10.1145/2594291.2594299}, and Amandroid~\cite{wei2018amandroid}, as well as apktool~\cite{apktool} and Androguard frameworks to identify unwanted behaviors, and traffic analysis using the crowd-sourced Lumen mobile traffic dataset to see app real-world behaviors. 
Compared to this work, we did the same by extracting pre-installed apps from the physical device but using ADB commands rather than an installed app, and we also performed a taint analysis using FlowDroid.

Bl\'azquez et al.~\cite{9519485} proposed FOTA (Firmware-Over-The-Air) Finder to automatically classify a given APK as FOTA or not based on Androguard, using the dataset of pre-installed apps from Firmware Scanner~\cite{9152633}. 
Then, they performed behavior analysis relying on FlowDroid and Amandroid for a taint analysis and a modification of Androwarn~\cite{androwarn} to analyze the use of API calls.
This part of the work is a bit similar to our data leak detection using FlowDroid. However, we considered all the pre-installed apps extracted from devices, and performed also malware detection.

Hou et al.~\cite{9793923} performed a study in which they collected firmware images from vendors, official websites and open source repositories, and CVE data to link them with pre-installed apps. 
In this study, they proposed a tool named AndScanner that automates the extraction of pre-installed apps from firmware images before analyzing them.
AndScanner proposes an analysis of the security patches of the firmware to know if it has been patched in time and if the security issues have been fixed. 
It also performs app analysis by analyzing the pre-installed apps using Androguard to identify misconfiguration in the manifest file and CryptoGuard~\cite{CryptoGuardOSS} to detect cryptography misuses.
Our study is completely different since we did not focus on analyzing the manifest files and cryptography misuses. However, we leveraged on Androguard framework in our pipeline.

More recently, Sutter and Tellenbach~\cite{FirmwareDroid} proposed FirmwareDroid, an automated static analysis tool for pre-installed apps.
This tool automates the process of extracting pre-installed apps from firmware images and their analysis using existing tools including Androgurad and Exodus.
The study identifies the advertising tracker libraries used with Exodus and the permissions pre-installed apps inherited with Androguard.
The authors have integrated 8 open source static analysis tools in FirmwareDroid which could be used for more analysis.

Almost all of these studies are a bit similar since they follow almost the same approach, such as collecting firmware from the Internet, extracting the pre-installed apps, and analyzing them.
Our approach allows the extraction of pre-installed apps from physical devices, the detection of malware and dangerous permissions, the detection of data leakage, and the use of malicious URLs in pre-installed apps. 
Furthermore, our approach is particularly tested on low-cost devices sold in Africa.
However, with this approach, every brand and every Android device can be inspected and analyzed. Consequently, we do not have a limitation related to missing some brands or firmware.
 
%\section{Conclusion}

In this work, we presented a full-stack investigation of LLM unlearning, encompassing methodology, evaluation, and robustness. We established a principled taxonomy that organizes twelve representative unlearning methods into three families: {\MDiv}, {\MRep}, and {\MRej}, providing a systematic lens to understand their underlying mechanisms. Our analysis revealed that conventional multiple-choice questioning (MCQ) evaluations of unlearning effectiveness (UE) and utility retention (UT) offer an incomplete picture, and we introduced open question answering (Open-QA) as a complementary paradigm to better capture generative behaviors and expose the strengths and limitations of different methods. Furthermore, we provide a comprehensive robustness assessment across model-level and input-level attacks, revealing nuanced relationships among in-domain relearning, out-of-domain fine-tuning, quantization, and jailbreak attacks. These findings clarify the trade-offs of current unlearning algorithms and guide the design of future methods that are both effective and robust. The use of LLM, limitation and broader impact are further discussed in \textbf{Appendix\,\ref{appx:llm_usage}}, \textbf{Appendix\,\ref{appx:limit}} and \textbf{Appendix\,\ref{appx:impact}}.

\section{Conclusion}
\label{concl}
This study investigates and analyzes Android pre-installed apps, in particular, those shipped with phones sold in Africa.
For this purpose, we have proposed a pipeline that allows the extraction of APK files from a physical device and inspects them to look for different suspicious behaviors, including pre-installed malware, apps exposing sensitive data, and apps sending personal data to remote hosts.
The pipeline is tested by inspecting three different low-cost Android devices bought in Africa.
The results show interesting findings, such as (1) the leak of sensitive data, (2) sensitive data exposure through exported components and (3) apps having suspicious behaviors.
%Furthermore, the pipeline detected the use of dangerous permissions from pre-installed third-party apps (that is, the manufacturer's pre-installed apps that are not system apps) in at least X\% of the third-party applications on the three devices.
%We have also compared these findings with the results from a non-low-cost device (i.e., Google Pixel 7) to point out the scope of the findings from low-cost devices. This comparison has shown there is no application in the Google Pixel device that leaks personal or sensitive data.
%In addition, there is no application that uses malicious URLs on this phone.
%It remains to determine whether more apps have suspicious behaviors or not and that static analysis was unable to detect them. 
As future research, we planned to go deeper into these pre-installed apps by analysing the URLs they use to identify malicious ones, the native libraries used, and by looking for further suspicious behaviors.
Our future study will extend the number of low-cost devices and compare the results from the analysis of pre-installed apps with those from European devices, as well.
In addition, we planned to perform a dynamic analysis by implementing a solution that monitors the system log and detects suspicious behaviors using AI models.

% \section*{Acknowledgments}

% This work is supported by the Luxembourg Ministry of Foreign and European Affairs through their Digital4Development (D4D) portfolio under the project LuxWAyS (Luxembourg/West-Africa Lab for Higher Education Capacity Building in CyberSecurity and Emerging Topics in ICT4Dev.)

\bibliographystyle{splncs04}
\bibliography{sample}

\end{document}
