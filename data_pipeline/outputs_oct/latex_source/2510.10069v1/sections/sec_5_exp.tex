
\begin{figure}
    \centering
    \includegraphics[width=\textwidth]{assets/dubbing_quality.png}
    \caption{Qualitative visual comparison: LatentSync vs. our approach.}
    \label{fig:dubbing}
\end{figure}

\section{Experiments}

\paragraph{Implementation details.}
\textit{Backbones.} The visual encoder follows the Sapiens-0.3B configuration and is initialized from its pretrained weights; the decoder adopts the ViTMAE-B design with a cross-attention sublayer in every block to attend to the identity/ambient/vocal prompt tokens. The audio branch uses a pretrained wav2vec~2.0 base encoder kept frozen during pretraining. 
\textit{Pretraining.} SyncLipMAE is trained on 128 GPUs with a per-GPU batch size of 48 (global $=128{\times}48$) for 400k steps (~15 days). 
\textit{Video dubbing fine-tuning.} We initialize from WanVACE-1.3B, freeze Wan backbone blocks, and optimize only VACE blocks (~0.7B trainable params). Training uses 128 GPUs, batch size 1 per GPU, for ~30k steps (~3 days). Each sample contains $17{+}1{+}1$ frames: 17 target frames, one history frame (for temporal continuity), and one identity frame (appearance prior).


\paragraph{Pretraining data.}
We pretrain \name on a mixture of talking-face corpora: Hallo3~\cite{hallo3}, CelebV-HQ~\cite{celebvhq}, CelebV-Text~\cite{celebvtext}, MEAD~\cite{mead}, VFHQ~\cite{vfhq}, HDTF~\cite{hdtf} and RAVDESS~\cite{ravdness}.
Unless specified, audio is mono 16\,kHz and videos are 25\,fps. We adopt the official train–val–test split when available; otherwise, we randomly sample 100 instances each for validation and testing, and use the remainder for training.

% ======================= 1) LIP SYNCHRONISATION =======================
\paragraph{Audio--visual stream synchronization.}
We evaluate alignment under two complementary protocols that require no additional training: (i) \emph{temporal lag detection} by sliding the offset between the audio token sequence $A=\{a_t\}_{t=1}^{T}$ and the visual token sequence $V=\{v_t\}_{t=1}^{T}$, and (ii) \emph{audio$\to$video token matching} reported as R-precision@$k$.

\textit{Temporal lag detection.}
For a candidate lag $\tau\!\in\!\mathcal{T}$ (positive: video lags audio), define the average distance
\[
D(\tau)=\frac{1}{T_\tau}\sum_{t=1}^{T_\tau} d\!\big(a_t,\,v_{t+\tau}\big),
\]
with $T_\tau$ chosen so indices are valid and $d(\cdot,\cdot)$ a feature distance (e.g., cosine or $\ell_2$). The estimated lag is
\[
\hat{\tau}=\arg\min_{\tau\in\mathcal{T}} D(\tau).
\]
Given ground-truth lag $\tau^\star$, we report
\[
\mathrm{Offset}=\tfrac{1}{N}\sum_{s=1}^N|\hat{\tau}^{(s)}-\tau^{(s)}|\quad\text{and}\quad
\mathrm{Acc}_{\pm K}=\tfrac{1}{N}\sum_{s=1}^N \mathbf{1}\!\big(|\hat{\tau}^{(s)}-\tau^{(s)}|\le K\big).
\]

\textit{Audio$\to$Video token matching (R-precision@$k$).}
We additionally report this retrieval-style metric because in segments with speaker silence or very small mouth motion the lag-based signal (Offset/Acc) can become weak or ambiguous, whereas a ranked retrieval test still probes cross-pair discriminability among distractors.
Within a minibatch of size $B$, let $\{(a_i,v_i)\}_{i=1}^B$ be audio/visual token pairs and let $\mathcal{P}_i$ be the candidate pool consisting of the true $v_i$ plus non-matching video tokens (pool size fixed by the batch or a pre-defined sampler). Rank $\mathcal{P}_i$ by distance to $a_i$ and denote the rank of $v_i$ by $\mathrm{rank}_i(v_i)$ (1 is best). The retrieval metric is
\[
\mathrm{R\text{-}precision@}k \;=\; \frac{1}{B}\sum_{i=1}^B \mathbf{1}\big(\mathrm{rank}_i(v_i)\le k\big),
\]
reported for small $k$ (e.g., $k\!=\!1,2,3$) and fixed pool size (e.g., 32).


\textit{Results.}
~\tabref{tab:sync_hallo3} and ~\tabref{tab:sync_vfhq} summarise results on Hallo3 and VFHQ. 
Across both datasets, \name consistently improves temporal alignment (higher $\mathrm{Acc}_{\pm K}$, lower \emph{Offset}) and cross-modal retrieval (higher R-precision@k). 
Notably, these gains are achieved with single-frame visual conditioning ($n{=}1$), whereas baselines rely on wider temporal windows ($n{=}5$ or $16$), indicating stronger per-frame audio–visual correspondence and reduced dependence on temporal aggregation.


% ======================= 2) FACIAL UNDERSTANDING =======================
\paragraph{Facial understanding (emotion \& actions).}
We evaluate on \textbf{CelebV-HQ}~\cite{celebvhq} using the \textit{emotion} and \textit{action} attribute sets, reporting video-level Accuracy and AUC (higher is better). 
Emotion is treated as a \textit{single-label}, 8-way classification task, whereas action is a \textit{multi-label} problem over 35 categories (one-vs-rest scoring aggregated at the video level). 
To match common practice, clip-based baselines (e.g., VideoMAE/MARLIN) adopt the \(a{\times}b\) protocol—uniformly sample \(a\) clips of \(b\) frames and average their predictions following ~\cite{videomae}; frame-averaging models use \(n\) uniformly sampled frames and average per-frame predictions~\cite{scsampler}.
All methods use a single linear classifier on frozen features.
Let \(\{\mathbf{x}_i\}_{i=1}^{m}\) be the sampled clips (or frames), \(f(\cdot)\) the frozen encoder, and \(W\) the linear head; the video-level logits are
\[
\bar{\ell}(V)=\frac{1}{m}\sum_{i=1}^{m} W\,f(\mathbf{x}_i),\qquad 
\hat{y}(V)=\arg\max_k \mathrm{softmax}_k\!\big(\bar{\ell}(V)\big),
\]
with Accuracy/AUC computed from \(\hat{y}(V)\) at the video level.

\textit{Results.}
\tabref{tab:celebvhq_emotion_action} summarises CelebV-HQ emotion and action. 
On emotion, \name clearly surpasses the best baseline with 16 frames and improves further at 80 frames, indicating benefits from longer temporal context. 
On action—where baselines are already strong—\name provides consistent but modest gains, with a slight boost when using 80 frames. 
Overall, \name delivers substantial improvements on emotion and steady gains on action.



\paragraph{Visual Speech Recognition (VSR)}
We evaluate \name\ on HDTF, VFHQ, and Hallo3, predicting text from silent videos. 
Following standard practice, accuracy is measured by Word Error Rate (WER), computed from word-level substitutions (\(S\)), deletions (\(D\)), and insertions (\(I\)) against the reference of length \(N\):
\[
\mathrm{WER}=\frac{S+D+I}{N}.
\]

\textit{Results.}
\tabref{tab:avsr} reports WER on VFHQ, Hallo3, and HDTF. The visual-only head built on \name’s tokens achieves the best WER on all three sets: \(7.37\) on VFHQ, \(5.99\) on Hallo3, and \(5.78\) on HDTF, improving over Auto-AVSR by absolute \(0.76\), \(0.24\), and \(0.13\) respectively, and substantially outperforming AV-HuBERT on each set. These gains indicate that \name’s pretraining yields visual motion features that transfer effectively to sentence-level lip reading under the standard hybrid CTC/attention recipe. 


% ======================= 4) LIP-SYNC GENERATION =======================
\paragraph{Video dubbing.}
We evaluate \emph{video dubbing}—re-synchronising the mouth region of a source video to a target speech while preserving identity, pose, and expressions—using a WanVACE-based model that supports both audio- and video-driven conditioning via a lightweight vocal-cue injection (details in~\secref{subsec:dubbing}). We report the metrics in~\tabref{tab:dubbing}. 
\textbf{$\mathrm{Sync}_{\mathrm{conf}}$} (identical to LSE-C) is computed with the \emph{stable SyncNet} evaluator from LatentSync~\cite{latentsync}, replacing the commonly used vanilla SyncNet whose scores are less reliable in our experiments~\cite{wav2lip}. Following SyncNet~\cite{syncnet}, for each short sliding window we form an audio–video distance curve over temporal offsets $\Delta\!\in\![-L,L]$:
\[
d_t(\Delta)\;=\;\big\|\,\phi_a\!\big(a_{t:t+w-1}\big)\;-\;\phi_v\!\big(v_{t+\Delta:\,t+\Delta+w-1}\big)\,\big\|_2,
\]
and define the per-window confidence as the \emph{median–minimum gap} on this curve,
\[
\mathrm{Conf}_t\;=\;\mathrm{median}_{\Delta}\, d_t(\Delta)\;-\;\min_{\Delta}\, d_t(\Delta),
\qquad
\mathrm{Sync}_{\mathrm{conf}}\;=\;\frac{1}{T}\sum_{t=1}^{T}\mathrm{Conf}_t,
\]
where larger values indicate a clearer minimum at the correct offset (stronger A/V synchrony)~\cite{syncnet,wav2lip}. 
\textbf{FID} measures per-frame perceptual quality via the Fréchet distance between Inception features of real vs.\ generated frames (lower is better)~\cite{fid}; 
\textbf{FVD} extends this to spatiotemporal video features to capture visual quality and temporal coherence jointly (lower is better)~\cite{fvd}.


\paragraph{Results.}
Qualitatively (Fig.~\ref{fig:dubbing}), \name preserves identity and fine lip details (e.g., teeth, beard) and is more robust to occlusions than LatentSync/MuseTalk. 
Quantitatively (Tab.~\ref{tab:dubbing}), consistent with these observations, the audio-driven variant (\textbf{A-\name}) attains the best overall scores on synchrony ($\mathrm{Sync}_{\mathrm{conf}}$) and perceptual quality (FID/FVD) on both VFHQ and Hallo3. 
The video-driven variant (\textbf{V-\name}) closely matches the audio-driven performance and surpasses prior methods, indicating that the shared token space indeed aligns modalities and enables no-difference audio- or video-driven control within a single model.



\paragraph{Ablation Studies.} We perform ablations on the key components of \name, examining: (i) whether to enable Two-Bypass Face-Aware Masking; (ii) the strategy for adapting audio features; and (iii) the inclusion of cross-attention in the decoder. Detailed results are provided in the Appendix.

 
