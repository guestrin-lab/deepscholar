
\section{\name: Pretraining Objective and Architecture}
\label{sec:pretrain}

\subsection{Overview and Design Goals}
\label{subsec:overview}
\name aims to learn synchronization-aware talking-face representations from unlabeled audio--visual streams.
As illustrated in~\figref{fig:ovewview}(a), our approach couples MAE-style masked image/video modeling on the visual stream with a CLIP-style cross-modal contrastive alignment that treats time-aligned \textit{vocal-motion} and \textit{audio} tokens as positives and temporally misaligned pairs as negatives, driving both modalities into a shared embedding space and yielding token-level audio--visual stream synchronization (\S\ref{subsec:objectives}).
Per frame, a ViT-based encoder exposes three prompt tokens that explicitly encode the essential factors of a talking-face frame—identity $\mathbf{z}^{\mathrm{id}}$, vocal motion $\mathbf{z}^{\mathrm{voc}}$, and ambient motion $\mathbf{z}^{\mathrm{amb}}$—and the decoder reconstructs frames via cross-attention conditioned on these prompts.


\paragraph{Inputs and Notation.}
Let $\mathbf{x}_{1:T}\in\mathbb{R}^{T\times C\times H\times W}$ be video frames and $\mathbf{a}\in\mathbb{R}^{N}$ mono audio at sampling rate $f_s$.
Each frame is patchified into $N_p$ patches and embedded to $D$-dimensional tokens.
For MAE-style masking, denote by $\mathcal{V}_t\subset\{1,\dots,N_p\}$ the indices of visible patches at time $t$ and by $\mathcal{M}_t$ the masked ones, with $|\mathcal{V}_t|=N_{\mathrm{vis}}$ and $\mathcal{V}_t\cup\mathcal{M}_t=\{1,\dots,N_p\}$.
The per-frame visible patch tokens are $\mathbf{P}^{\mathrm{vis}}_t\in\mathbb{R}^{N_{\mathrm{vis}}\times D}$.
Let $\mathbf{e}^{\mathrm{mask}}\in\mathbb{R}^{D}$ be a learnable mask token; a restoration operator places $\mathbf{P}^{\mathrm{vis}}_t$ and copies of $\mathbf{e}^{\mathrm{mask}}$ at indices $\mathcal{M}_t$ to form a length-$N_p$ sequence for decoding.
Per frame, the encoder also outputs three prompt tokens,
$\mathbf{z}^{\mathrm{id}}_t,\ \mathbf{z}^{\mathrm{amb}}_t,\ \mathbf{z}^{\mathrm{voc}}_t\in\mathbb{R}^{D}$.
For audio, we employ a pretrained speech encoder with $L$ layers that is kept frozen during pretraining; let $\{\mathbf{H}^{(\ell)}\}_{\ell=1}^{L}$ denote the hidden-state sequences from all layers.
To obtain frame-wise alignment with the video, we resample the input waveform so that the encoder emits $T$ tokens per utterance (one token per video frame) at every layer, yielding $\mathbf{A}^{(\ell)}=\{\mathbf{a}^{(\ell)}_t\}_{t=1}^{T}$.

\subsection{Model Design}
\label{subsec:arch}

\paragraph{Two-Bypass Face-Aware Masking.}
Each frame is processed by two masking \textit{bypasses}, and the visual encoder is invoked twice per iteration to consume the corresponding visible patches. 
\textbf{Bypass 1 (Uniform)} applies a random uniform mask hiding $75\%$ of patches, leaving the visible set $\mathcal{V}^{(1)}_t$ and visible patch tokens $\mathbf{P}^{\mathrm{vis},(1)}_t\!\in\!\mathbb{R}^{N_{\mathrm{vis}}\times D}$. We use the \textit{identity} prompt $\mathbf{z}^{\mathrm{id}}_t$ from this pass together with these visible patch tokens for decoding; the \textit{vocal/ambient} prompts from this pass are ignored. 
\textbf{Bypass 2 (Face-Preserving)} also masks $75\%$ of patches but retains facial regions with higher keep probability, forming $\mathcal{V}^{(2)}_t$. To suppress appearance leakage while preserving motion cues, we apply color/brightness/saturation perturbations on this view. We use the \textit{vocal-motion} and \textit{ambient-motion} prompts $\mathbf{z}^{\mathrm{voc}}_t,\mathbf{z}^{\mathrm{amb}}_t$ from this pass for decoding, while its visible patch tokens and identity prompt are ignored. In both passes the encoder consumes only visible patches. Before decoding, a learnable mask token is inserted at indices $\mathcal{M}^{(1)}_t$ (the complement of $\mathcal{V}^{(1)}_t$) to restore a length-$N_p$ sequence per frame, consistent with MAE.


\begin{table}[t]
  \centering
  \caption{Lip synchronisation on \textbf{VFHQ}.}
  \label{tab:sync_vfhq}
  \setlength{\tabcolsep}{5pt} % 可微调列间距
  \small
  \begin{adjustbox}{max width=\linewidth}
    {\rowcolors{2}{gray!10}{white}%
     \begin{tabular}{@{} l
                         cc  cc  cc  ccc @{}}
       \toprule
       \multirow{2}{*}{Method} &
         \multicolumn{2}{c}{K = 1} &
         \multicolumn{2}{c}{K = 5} &
         \multicolumn{2}{c}{K = 15} &
         \multicolumn{3}{c}{R-precision (32)} \\
       \cmidrule(lr){2-3}\cmidrule(lr){4-5}\cmidrule(lr){6-7}\cmidrule(lr){8-10}
       & Acc $\uparrow$ (\%) & Offset $\downarrow$
         & Acc $\uparrow$ (\%) & Offset $\downarrow$
         & Acc $\uparrow$ (\%) & Offset $\downarrow$
         & Top1 $\uparrow$ & Top2 $\uparrow$ & Top3 $\uparrow$ \\
       \midrule
       SyncNet\,-5           & 22.16 & 5.14 & 26.47 & 4.34 & 31.09 & 3.29  & 9.40  & 16.91 & 23.09 \\
       VocaLiST\,-5          & 21.76 & 5.14 & 24.34 & 4.57 & 29.23 & 3.62  & 10.18 & 16.56 & 21.98 \\
       StableSyncNet\,-16    & 32.40 & \textbf{3.17} & 33.06 & 3.07 & 35.43 & 2.90  & 23.54 & 37.65 & 46.77 \\
       \addlinespace[2pt]
       \textbf{\name\ -1 (ours)} & \textbf{38.58} & 3.80 & \textbf{48.95} &
         \textbf{3.06} & \textbf{61.78} & \textbf{2.16} &
         \textbf{35.52} & \textbf{52.98} & \textbf{63.37} \\
       \bottomrule
     \end{tabular}}
  \end{adjustbox}
\end{table}


\paragraph{Visual Encoder and Prompt Tokens.}
A ViT-style encoder applied to the visible patches outputs per-frame patch tokens (from Bypass 1) and three prompt tokens
\[
\mathbf{Z}^{\text{prompt}}_t=\big[\mathbf{z}^{\text{id}}_t,\ \mathbf{z}^{\text{amb}}_t,\ \mathbf{z}^{\text{voc}}_t\big]\in\mathbb{R}^{3\times D},
\]
where $\mathbf{z}^{\mathrm{id}}_t$ is taken from Bypass 1 and $\mathbf{z}^{\mathrm{voc}}_t,\mathbf{z}^{\mathrm{amb}}_t$ from Bypass 2 as defined above.

\paragraph{Identity-Consistent Prompt Shuffling.}
Within a mini-batch, to promote motion-invariant identity embeddings, we randomly swap $\mathbf{z}^{\mathrm{id}}$ among frames that share the same subject identity but exhibit different facial motions, weakening incidental coupling between identity prompts and short-term dynamics.

\paragraph{Audio Features and Adapter.}
A pretrained $L$-layer speech encoder (kept frozen) produces per-layer hidden sequences; we resample the input waveform so that each layer emits $T$ tokens (one per video frame), yielding aligned streams $\{\mathbf{a}^{(\ell)}_t\}_{t=1}^{T}$. To preserve both low-level acoustic detail and higher-level linguistic cues, we concatenate the aligned per-layer tokens along the feature dimension at each time step,
\[
\tilde{\mathbf{a}}_t=\big[\mathbf{a}^{(1)}_t \,\|\, \cdots \,\|\, \mathbf{a}^{(L)}_t\big],
\]
and pass them through an audio adapter that projects to the shared width $D$, producing $\mathbf{A}_t\in\mathbb{R}^{D}$. This design mitigates the bias of top encoder layers toward semantic content induced by ASR pretraining while retaining waveform-proximal information from earlier layers.

\paragraph{Decoder with Prompt Cross-Attention.}
We use an MAE-style decoder with $N_{\text{dec}}$ Transformer blocks. Its input at each frame is the restored token sequence formed by encoded visible patches from Bypass~1 plus mask tokens (with positional embeddings). Each block first applies self-attention over this full sequence (visible patches + mask tokens), and then cross-attends to the prompt tokens—identity, ambient, and a conditioning token $\mathbf{c}_t$. We perform two decoding passes per frame that share decoder weights: (i) a video-driven pass with $\mathbf{c}_t=\mathbf{z}^{\mathrm{voc}}_t$ producing $\hat{\mathbf{x}}^{\mathrm{voc}}_t$, and (ii) an audio-driven pass with $\mathbf{c}_t=\mathbf{A}_t$ producing $\hat{\mathbf{x}}^{\mathrm{aud}}_t$. Both passes reconstruct the same target frame $\mathbf{x}_t$, providing symmetric supervision that encourages the vocal-motion and audio tokens to encode consistent, alignable information. A linear head maps decoded tokens back to pixels.


\subsection{Pretraining Objectives}
\label{subsec:objectives}

\paragraph{Pixel-Space Reconstruction (MSE).}
We minimize per-frame mean squared error for the two decoding passes (video-driven and audio-driven; see \S\ref{subsec:arch}):
\[
\mathcal{L}_{\text{pix}}^{\text{voc}}=\tfrac{1}{T}\sum_{t=1}^{T}\!\big\|\hat{\mathbf{x}}^{\text{voc}}_t-\mathbf{x}_t\big\|_2^2,\qquad
\mathcal{L}_{\text{pix}}^{\text{aud}}=\tfrac{1}{T}\sum_{t=1}^{T}\!\big\|\hat{\mathbf{x}}^{\text{aud}}_t-\mathbf{x}_t\big\|_2^2.
\]

\paragraph{Audio–Vocal Contrastive Alignment.}
Unlike prior audio–visual synchronisation systems that cast the task as in-/out-of-sync \textit{binary} classification or as \textit{pairwise} contrastive matching, we align per-frame audio tokens $\mathbf{A}_t$ with vocal-motion tokens $\mathbf{v}_s=\mathbf{z}^{\mathrm{voc}}_s$ using a CLIP-style symmetric InfoNCE.
Let $p_{t\to s}=\frac{\exp(\langle \mathbf{A}_t,\mathbf{v}_s\rangle/\tau)}{\sum_{u=1}^{T} \exp(\langle \mathbf{A}_t,\mathbf{v}_u\rangle/\tau)}$ be the softmax over vocal candidates for audio index $t$, and $P_t$ the positive set (aligned index and optional neighbors within $\pm k$, expanded by an audio self-similarity threshold).
The loss is
\[
\mathcal{L}_{\mathrm{CL}}
= -\frac{1}{2T} \sum_{t=1}^{T} \Big[
\log \textstyle\sum_{s\in P_t} p_{t\to s}
\;+\;
\log \textstyle\sum_{s\in P_t} p_{s\to t}
\Big],
\]
with temperature $\tau$.
This preserves CLIP-style bidirectional alignment while allowing multiple positives per anchor.



\paragraph{Cross-Covariance Decorrelation.}
To reduce leakage across factor-specific embeddings, we penalize cross-covariance between centered tokens. For $\mathcal{P}=\{\mathrm{id},\,\mathrm{amb},\,\mathrm{voc}\}$:
\[
\mathcal{L}_{\text{cov}}
=\frac{1}{D^2}\sum_{\{p,q\}\subset\mathcal{P}}
\big\|\mathrm{Cov}\!\big(\mathbf{z}^{p},\mathbf{z}^{q}\big)\big\|_F^{2},
\]
where covariance is computed after centering along the batch/time axis.

\paragraph{Total Objective.}
The overall loss is
\[
\mathcal{L}
= \lambda_{\text{pix}}\big(\mathcal{L}_{\text{pix}}^{\text{voc}} + \mathcal{L}_{\text{pix}}^{\text{aud}}\big)
+ \lambda_{\text{CL}}\,\mathcal{L}_{\text{CL}}
+ \lambda_{\text{cov}}\,\mathcal{L}_{\text{cov}}.
\]

% ====== Alternating colors only on DATA rows: CelebV-HQ Emotion & Action ======
% Requires in preamble:
% \usepackage[table]{xcolor}
% \usepackage{booktabs}
% \usepackage{adjustbox}
% \usepackage{multirow}

\begin{table}[t]
  \centering
  \caption{CelebV-HQ emotion and action classification. Higher is better (Accuracy/AUC).}
  \label{tab:celebvhq_emotion_action}
  \setlength{\tabcolsep}{5pt}
  \small
  \begin{adjustbox}{max width=\linewidth}
  {\rowcolors{3}{gray!10}{white}%
   \begin{tabular}{@{} l c cc cc @{}}
     \toprule
     \multicolumn{1}{c}{\multirow{2}{*}{Method}} &
     \multicolumn{1}{c}{\multirow{2}{*}{Frames}} &
     \multicolumn{2}{c}{Emotion} & \multicolumn{2}{c}{Action} \\
     \cmidrule(lr){3-4}\cmidrule(lr){5-6}
      & & Accuracy $\uparrow$ & AUC $\uparrow$ & Accuracy $\uparrow$ & AUC $\uparrow$ \\
     \midrule
     VideoMAE             & $1\times16$ & 68.63 & 60.32 & 93.97 & 84.31 \\
     VideoMAE             & $5\times16$ & 68.63 & 66.60 & 94.54 & 86.01 \\
     MARLIN               & $1\times16$ & 66.67 & 58.15 & 94.69 & 82.26 \\
     MARLIN               & $5\times16$ & 69.61 & 64.98 & 94.86 & 83.90 \\
     \addlinespace[2pt]
     \textbf{\name\ (ours)} & 16        & \textbf{76.47} & \textbf{73.94} & \textbf{95.06} & \textbf{86.45} \\
     \textbf{\name\ (ours)} & 80        & \textbf{77.45} & \textbf{78.33} & \textbf{95.11} & \textbf{86.69} \\
     \bottomrule
   \end{tabular}}
  \end{adjustbox}
\end{table}

