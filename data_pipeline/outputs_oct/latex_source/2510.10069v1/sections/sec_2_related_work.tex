
\section{Related Work}


\paragraph{Audio--visual synchronisation and correspondence.}
A long line of work formulates A/V learning as correspondence or temporal alignment. ~\cite{l3net} learns generic audio–visual correspondence from unlabelled video, while ~\cite{avts} casts synchronisation as in-time vs.\ out-of-time discrimination within the same clip. For faces, ~\cite{syncnet} introduces a two-stream embedding for lip–audio alignment and lag estimation; recent transformers such as ~\cite{vocalist} improve robustness across speech/singing with longer-range context. 

\paragraph{Masked modelling and contrastive learning at scale.}
Contrastive pretraining aligns paired modalities at scale (e.g., CLIP ~\cite{clip}), while masked autoencoders (MAE~\cite{mae}/VideoMAE~\cite{videomae}) show that reconstructing heavily masked inputs yields strong visual/video representations. These paradigms are complementary and now standard building blocks for multimodal pretraining. 


\paragraph{Facial video representation pretraining.} Closer to our setting, MARLIN~\cite{marlin} applies masked autoencoding to facial videos, using facial-region-guided masking to learn a universal face encoder transferable to expression recognition, deepfake detection, etc. Subsequent variants adapt MAE-style pretraining to dynamic facial expression recognition under limited labels~\cite{mae_dfer}. While these works focus on reconstruction-only objectives within the facial domain, we couple MAE with audio–visual contrast to align speech-driven dynamics, and structure the representation to separate identity and motion factors.


\paragraph{Visual speech recognition and audio--visual ASR.}
From early end-to-end lipreading (LipNet) to large-scale self-supervised A/V pretraining (AV-HuBERT) and automated VSR/AVSR recipe design (Auto-AVSR~\cite{autoavsr}), results consistently show that robust visual streams improve recognition and benefit from pretraining on unlabelled A/V data.

\paragraph{Talking-face generation and lip-sync synthesis.}
Audio-driven talking-head synthesis has progressed from GAN-based pipelines with explicit sync critics (e.g., Wav2Lip~\cite{wav2lip}) to diffusion in latent/image space (e.g., LatentSync~\cite{latentsync}) and efficient diffusion heads (MuseTalk~\cite{musetalk}). In parallel, foundation video generators and unified conditional DiT~\cite{dit} frameworks (WAN~\cite{wan}; VACE~\cite{vace}) provide scalable backbones and conditioning interfaces (Video Condition Unit, context adapters) for editing or T2V, which our dubbing setup leverages.


\begin{table}[t]
  \centering
  \caption{Lip synchronisation on \textbf{Hallo3}. Higher is better for Acc and R-Precision; lower is better for Offset.}
  \label{tab:sync_hallo3}
  \setlength{\tabcolsep}{5pt} % 列间距可微调
  \small
  \begin{adjustbox}{max width=\linewidth}
    {\rowcolors{2}{gray!10}{white}%
     \begin{tabular}{@{} l
                         cc  cc  cc  ccc @{}}
       \toprule
       \multirow{2}{*}{Method} &
         \multicolumn{2}{c}{K = 1} &
         \multicolumn{2}{c}{K = 5} &
         \multicolumn{2}{c}{K = 15} &
         \multicolumn{3}{c}{R-precision (32)} \\
       \cmidrule(lr){2-3}\cmidrule(lr){4-5}\cmidrule(lr){6-7}\cmidrule(lr){8-10}
       & Acc $\uparrow$ (\%) & Offset $\downarrow$
         & Acc $\uparrow$ (\%) & Offset $\downarrow$
         & Acc $\uparrow$ (\%) & Offset $\downarrow$
         & Top1 $\uparrow$ & Top2 $\uparrow$ & Top3 $\uparrow$ \\
       \midrule
       SyncNet\,-5           & 20.37 & 5.04 & 24.18 & 4.03 & 30.03 & 3.02  & 7.47  & 12.77 & 17.72 \\
       VocaLiST\,-5          & 21.85 & 4.34 & 24.56 & 3.63 & 28.82 & 2.98  & 8.80  & 15.08 & 19.86 \\
       StableSyncNet\,-16    & 27.95 & 2.78 & 28.60 & 2.72 & 31.66 & 2.72  & 14.40 & 25.65 & 34.52 \\
       \addlinespace[2pt]
       \textbf{\name\,-1 (ours)} & \textbf{52.53} & \textbf{2.66} & \textbf{68.41} &
         \textbf{1.73} & \textbf{82.27} & \textbf{0.93} &
         \textbf{40.18} & \textbf{61.48} & \textbf{73.49} \\
       \bottomrule
     \end{tabular}}
  \end{adjustbox}
\end{table}

