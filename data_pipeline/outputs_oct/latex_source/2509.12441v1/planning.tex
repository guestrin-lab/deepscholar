\section{Automatic Network Planning}
\label{sec:planning}
In this section, we introduce the automatic network planning algorithm by leveraging the aforementioned DRT.

\subsection{Problem}
The goal of network planning is to obtain the optimal base station parameters (e.g., location and orientation) that maximizes the specified network performance (e.g., coverage and capacity) in the aforementioned DRT, under a given geographic area.

As illustrated in Fig. \ref{fig:figure1}, we consider a geographic region $\mathcal{A} \subset \mathbb{R}^2$ with given 3D terrain, such as buildings, trees, and walkways.
Here, we denote $\hat{\mathcal{B}} = {(\hat{x}_m, \hat{y}_m)}_{m=1}^M$ as the deployed $M$ base stations (BSs).  
To further improve coverage and expand capacity, we aim to deploy an additional $N$ BSs at locations $\bar{\mathcal{B}} = {(\bar{x}_n, \bar{y}_n)}_{n=1}^N$, selected from a feasible region $\mathcal{F} \subseteq \mathcal{A}$ that meets practical placement constraints (e.g., rooftop availability, and height limits). 
The complete BS set after deployment is $\mathcal{B} = \hat{\mathcal{B}} \cup \bar{\mathcal{B}}$, and the goal is to determine $\bar{\mathcal{B}}$ to maximize overall network performance in $\mathcal{A}$.

\begin{figure}[!t]
	\centering
	\includegraphics[width=3.0in]{fig1.pdf}
	\caption{Illustration of BSs deployment in feasible region $\mathcal{F}$.}
	\label{fig:figure1}
\end{figure}

Next, we introduce two network performance metrics, including coverage and capacity. 
We discretize $\mathcal{A}$ into a uniform square grid with resolution $a\times a$ meters, yielding
$L = \left\lceil |\mathcal{A}|/{a^2} \right\rceil$ sampling points, where $|\mathcal{A}|$ is the area of $\mathcal{A}$, and $\lceil\cdot\rceil$ represents the ceiling function (i.e., rounding up to the nearest integer).
For each sampling point $\ell \in {1, \dots, L}$, the received signal characteristics are obtained by using the aforementioned DRT that reflects the real-world propagation environment. 
The querying of the DRT is governed by a set of tunable parameters $\boldsymbol{\Theta}$, such as material conductivity and permittivity, which influence signal behavior and modelling accuracy. 

We evaluate the network coverage by using $r_\ell(\boldsymbol{\Theta})$, which denotes the reference signal received power (RSRP) at sampling point $\ell$, corresponding to the strongest signal received from all BSs in the set $\mathcal{B}$.
Moreover, we evaluate the network capacity by using the signal-to-noise ratio (SNR), which captures signal quality in the presence of interference. 
Let $s_\ell(\boldsymbol{\Theta})$ represent the SNR at point $\ell$, computed under the current parameter setting $\boldsymbol{\Theta}$ in the DRT.
Given the parameter of the DRT, we can obtain RSRP and SNR values at all sampling points, and then define the following performance metrics. 
Given a predefined RSRP threshold $r_{\mathrm{th}}$, the coverage rate is defined as:
\begin{equation}
C(\mathcal{B}) = \frac{1}{L} \sum_{\ell=1}^{L} \mathds{1}\left[ r_\ell(\boldsymbol{\Theta}) > r_{\mathrm{th}} \right],
\end{equation}
where $\mathds{1}[\cdot]$ is the indicator function, returning 1 if the condition is satisfied and 0 otherwise. 
The network capacity, reflecting the average spectral efficiency across all sampling points, is expressed by:
\begin{equation}
S(\mathcal{B}) = \frac{1}{L} \sum_{\ell=1}^{L} \log_2\left(1 + s_\ell(\boldsymbol{\Theta})\right).
\end{equation}
Finally, we combine these two metrics into a composite objective function:
\begin{equation}
\label{eq:eq3}
T(\mathcal{B}) = \alpha \cdot C(\mathcal{B}) + S(\mathcal{B}),
\end{equation}
where $\alpha > 0$ is a tunable weight controlling the trade-off between coverage and capacity. 
Here, a larger $\alpha$ places greater emphasis on coverage, whereas a smaller value prioritizes capacity.

Hence, the goal of network planning is further concretized to determine the optimal set of additional deployment locations $\bar{\mathcal{B}}^*$ that maximizes the overall objective:
\begin{equation}
\label{eq:eq4}
\bar{\mathcal{B}}^* = \arg\max_{\bar{\mathcal{B}} \subseteq \mathcal{F}} T(\hat{\mathcal{B}} \cup \bar{\mathcal{B}}),
\end{equation}
where $\hat{\mathcal{B}}$ denotes the set of existing BSs and $\mathcal{F} \subseteq \mathcal{A}$ represents the feasible deployment region.



\subsection{Challenge}
The key technical challenge lies in the unknown correlation between the input of base station locations and the output of the network performance.
This is attributed to the complex computation (e.g., ray tracing and radio effects) in the DRT, which cannot be mathematically represented.
Specifically, we can obtain only the output of network performances by querying the DRT (non-negligible querying costs) under the input base station locations.
In other words, the objective function $T(\mathcal{B})$ is unknown, which falls into the realm of blackbox optimization. 


\subsection{Solution}
To address the above problem, we design an automatic network planning method by leveraging Bayesian optimization (BO).
BO~\cite{wang2023recent, garnett2023bayesian} is the state-of-the-art global searching approach (especially for handling expensive querying costs), generally consisting of a surrogate model and acquisition function.
The surrogate model aims to online learn the blackbox function (e.g., mean and variance), where candidates include Gaussian process~\cite{rasmussen2003gaussian} and Bayesian neural networks~\cite{li2023study}.
Based on the updated surrogate model, the acquisition function evaluates the customized utility of different actions (i.e., BS locations) in the action space (i.e., $\mathcal{F}$).
In each iteration, the next action is selected by maximizing the given acquisition function (e.g., EI and PI)~\cite{frazier2018tutorial}.


To support multiple base station deployment, a naive approach in BO is to use the surrogate model to learn the combinational action space (i.e., all candidate BS locations) and overall network performance.
However, this naive approach exponentially expands the action space, and practically leads to much slow convergence speed and more cumulative querying costs, in terms of time consumption and computation complexity.
Hence, we converts the problem to be a series of incremental base station deployment problems.
In other words, we only apply the Bayesian optimization to search for one additional base station deployment at a time.
Once the base station location is determined, we put it into the existing deployed base station sets (i.e., $\hat{\mathcal{B}}$), and then create another Bayesian optimization searching for the next base station.
This incremental base station stops until the total number of base stations is all deployed.

In BO, the true objective function \eqref{eq:eq3} is replaced by a surrogate model that is cheaper to evaluate but still captures the underlying spatial correlation of the objective.
Specifically, we use a Gaussian process (GP) as the surrogate model, which is widely adopted and time-evaluated in multiple application domains.
Here, we model the target function as:
\begin{equation}
T(\hat{\mathcal{B}} \cup \bar{\mathcal{B}}) \sim \mathcal{GP}\left(\mu(\bar{\mathcal{B}}), \sigma^2(\bar{\mathcal{B}}) \right),
\end{equation}
where $\mu(\bar{\mathcal{B}})$ and $\sigma^2(\bar{\mathcal{B}})$ are the posterior mean and variance of the GP given the observations collected in previous iterations.
The GP surrogate allows us to predict both the expected performance of a candidate deployment $\bar{\mathcal{B}}$ and the uncertainty of this prediction, which is crucial for balancing exploration and exploitation.

The next action to query is generally determined by maximizing the selected acquisition function, which defines the utility of all the actions in the action space while balancing exploration and exploitation. 
Specifically, we adopt the EI as the acquisition function, which quantifies the expected gain over the current best performance $T_{\text{best}}$:
\begin{equation}
\alpha(\bar{\mathcal{B}}) = \mathbb{E} \left[ \max\left(0, T(\hat{\mathcal{B}} \cup \bar{\mathcal{B}}) - T_{\text{best}}\right) \right].
\end{equation}
This criterion favours actions that are likely to improve the objective while still exploring regions of high uncertainty.
At each iteration $t$, we select the next set of deployment locations by maximizing the acquisition function over the feasible region $\mathcal{F}$:
\begin{equation}
\bar{\mathcal{B}}_{t+1} = \arg\max{\bar{\mathcal{B}} \subseteq \mathcal{F}} \alpha (\bar{\mathcal{B}}).
\end{equation}
The newly selected action is then evaluated using the derived DRT model (see Sec.~\ref{sec:twinning}), and the resulting observation is incorporated into the GP to update its posterior distribution. 
This iterative process continues until a stopping criterion, such as a fixed iteration limit or convergence threshold, is met. 




