\documentclass[10pt, conference, letterpaper]{IEEEtran}
% \usepackage{algorithmicx}
\usepackage[ruled,vlined,linesnumbered]{algorithm2e}
\usepackage{hhline}
\usepackage{amsmath,mathtools}
\usepackage{amsfonts,amssymb}
\usepackage{mathrsfs}
\usepackage{gensymb} % for degree
% \usepackage{caption}% http://ctan.org/pkg/caption
\usepackage{graphicx} 
\usepackage{multirow}
\usepackage{enumitem,color}
\usepackage{algpseudocode}
\usepackage{makecell}
\usepackage{booktabs}
\usepackage{dsfont} 
\usepackage{caption}
\usepackage{subfigure} 
\usepackage{hyperref}
\usepackage{geometry}
\geometry{left=0.7in,top=0.7in,right=0.7in,bottom=1in,}

% for spacing details, see https://tex.stackexchange.com/questions/26521/how-to-change-the-spacing-between-figures-tables-and-text
\setlength{\textfloatsep}{2pt}  % this sets the gap between float obj (table, alg, figure, etc.) and the text
\setlength{\intextsep}{2pt}  % this sets the gap between float obj (table, alg, figure, etc.) and the text
\setlength{\floatsep}{2pt}  % this sets the gap between float obj (table, alg, figure, etc.) and the text
\setlength{\dbltextfloatsep}{2pt}  % this sets the gap between float obj (table, alg, figure, etc.) and the text
\setlength{\dblfloatsep}{2pt}  % this sets the gap between float obj (table, alg, figure, etc.) and the text
\setlength{\abovedisplayskip}{2pt} % this set the gap between equation and text
\setlength{\belowdisplayskip}{2pt} % this set the gap between equation and text
\setlength{\abovecaptionskip}{2pt} % Chosen fairly arbitrarily
\setlength{\abovedisplayshortskip}{2pt}
\setlength{\belowdisplayshortskip}{2pt}
\usepackage{titlesec}
\titlespacing*{\section}{1pt}{0.5ex}{0.5ex}
\titlespacing*{\subsection}{1pt}{0.5ex}{0.5ex}
\titlespacing*{\subsubsection}{1pt}{0.5ex}{0.5ex}
\titleformat{\subparagraph}[runin]{\normalfont\normalsize\bfseries}{\thesubparagraph}{1em}{}

\begin{document}
%
% paper title
% Titles are generally capitalized except for words such as a, an, and, as,
% at, but, by, for, in, nor, of, on, or, the, to and up, which are usually
% not capitalized unless they are the first or last word of the title.
% Linebreaks \\ can be used within to get better formatting as desired.
% Do not put math or special symbols in the title.
\title{\huge Automatic Network Planning with Digital Radio Twin\vspace{-0.1in}}



\author{
\IEEEauthorblockN{
Xiaomeng Li\IEEEauthorrefmark{1}, 
Yuru Zhang\IEEEauthorrefmark{1}, 
Qiang Liu\IEEEauthorrefmark{1}, 
Mehmet Can Vuran\IEEEauthorrefmark{1}, \\
Nathan Huynh\IEEEauthorrefmark{1},
Li Zhao\IEEEauthorrefmark{1}, 
Mizan Rahman\IEEEauthorrefmark{2}, 
Eren Erman Ozguven\IEEEauthorrefmark{3} \\
\IEEEauthorrefmark{1} University of Nebraska-Lincoln,
\IEEEauthorrefmark{2} University of Alabama,
\IEEEauthorrefmark{3} Florida State University
\vspace{-0.2in}
}
}



\maketitle

\begin{abstract}
Network planning seeks to determine base station parameters that maximize coverage and capacity in cellular networks. However, achieving optimal planning remains challenging due to the diversity of deployment scenarios and the significant simulation-to-reality discrepancy.
In this paper, we propose \emph{AutoPlan}, a new automatic network planning framework by leveraging digital radio twin (DRT) techniques.
We derive the DRT by finetuning the parameters of building materials to reduce the sim-to-real discrepancy based on crowdsource real-world user data.
Leveraging the DRT, we design a Bayesian optimization based algorithm to optimize the deployment parameters of base stations efficiently.
Using the field measurement from Husker-Net, we extensively evaluate \emph{AutoPlan} under various deployment scenarios, in terms of both coverage and capacity.
The evaluation results show that \emph{AutoPlan} flexibly adapts to different scenarios and achieves performance comparable to exhaustive search, while requiring less than 2\% of its computation time.
\end{abstract}

\begin{IEEEkeywords}
Network Planning, Digital Radio Twin, Bayesian Optimization
\end{IEEEkeywords}

% TODO: usage of DRT future 

% reference: AutoBS

\section{Introduction}
\label{sec:introduction}
Network planning constitutes a fundamental stage in the design of cellular networks, wherein critical base station (BS) parameters (e.g.,  geographic placement, antenna orientation, transmission power, and tilt) are optimized to ensure desired network coverage and performance.
With the densification of cellular networks~\cite{andreev2019future}, network planning has become increasingly intricate, requiring the joint optimization of heterogeneous base stations under site-specific propagation effects induced by irregular 3D terrain, complex building geometries, and diverse material compositions.
Conventional cellular network planning relies on a combination of field measurements (e.g., drive tests) and model-based simulations (e.g., calibrated radio propagation models), which is laborious and time-consuming. 

Digital radio twin (DRT)~\cite{khan2022digital} emerges as a promising technique to facilitate network planning.
DRT aims to replicate the real-world radio propagation in the scene, with the attributes of fidelity (imitating as closely as possible), tractability (querying as cheaply as possible), and synchronicity (tracking as timely as possible)~\cite{almasan2022network}.
By leveraging a DRT, network operators can perform trial-and-error evaluations to optimize base station parameters and assess coverage, capacity, and interference trade-offs, prior to costly field implementation.

However, it is challenging to derive a DRT with all the essential attributes.
On the one hand, existing simulators (e.g., Sionna RayTracing~\cite{hoydis2023sionna} and Wireless InSite) suffer from non-trivial simulation-to-reality (sim-to-real) discrepancy~\cite{manalastas2024simulators}, necessitating extensive calibration to improve fidelity.
On the other hand, the synchronicity attribute is difficult to achieve, as it is not cost-efficient to conduct periodic field measurements for simulator calibration.
Note that, given the complex computation (e.g., raytracing and radio effects), we can only obtain the network performance (e.g., coverage and capacity) after querying the DRT. 
Hence, it is difficult to derive the optimal base station parameters, under high-dimensional combination spaces, even if the DRT can be queried in seconds.



In this paper, we propose \emph{AutoPlan}, a new automatic network planing framework by leveraging digital radio twin techniques.
On the one hand, we derive the DRT with the following steps. 
First, we crowdsource abundant real-world data from mobile users under existing deployed base stations (e.g., prior deployments or those of other network operators). 
Second, with the online data, we continuously calibrate the parameters (particularly, the material of buildings) of the existing simulator (i.e., Sionna RT), to reduce the sim-to-real discrepancy.
Third, we accelerate the query of the calibrated simulator with GPU parallel computing to improve its tractability.
On the other hand, we design a network planning algorithm to derive the optimal base station parameters (particularly, location and power).
Specifically, we leverage Bayesian optimization to tackle the non-negligible querying costs of the DRT by creating a Gaussian process as the surrogate model and adopting expected improvement (EI) as the acquisition function.
Using the field measurement from Husker-Net (a private 5G network at the University of Nebraska-Lincoln), we extensively evaluate \emph{AutoPlan} under different deployment scenarios, in terms of both coverage, capacity, and efficiency.


Overall, we summarize the contribution of this paper as:
\begin{itemize}
    \item The DRT with the attribute of fidelity, tractability, and synchronicity, based on the simulator of Sionna RayTracing; 
    \item An automatic network planning algorithm, based on Bayesian optimization, to derive optimal base station parameters, assisted by the derived DRT;  
    \item An extensive campus-scale evaluation, showing comparable network performance with state-of-the-art works.
\end{itemize}




% The rapid evolution of fifth-generation (5G) and emerging sixth-generation (6G) cellular networks demands accurate and efficient methodologies for base station (BS) deployment planning.
% Traditionally, planning has relied on field measurements. 
% However, exhaustively testing all feasible configurations is prohibitively time-consuming and cost-intensive, especially in large rural regions. 
% Recently, the concept of digital radio twins (DRTs) has emerged as a promising paradigm for bridging the gap between simulated and real-world environments~\cite{hoydis2023sionna}. 
% DRTs combine high-fidelity 3D scene models with physics-based radio propagation. 
% In some engines, they also support differentiable propagation, enabling reproducible performance evaluation and system optimization.

% Despite this promise, building a precise DRT is challenging because of uncertainty in material parameters and environmental dynamics. In practice, obtaining accurate parameters for all objects is rarely feasible, leading to discrepancies between simulation and measurements~\cite{manalastas2024simulators}. These inaccuracies limit the reliability of deployment strategies that are solely based on uncalibrated simulations. Furthermore, BS deployment is inherently a black-box optimization problem: key performance metrics such as coverage and capacity are only available after running full simulations, rendering the mapping from deployment configurations to metrics non-differentiable and analytically intractable.

% To address these challenges, we propose \emph{AutoPlan}, a new network deployment framework, that automatically determines the deployment parameters of base stations. Specifically, we use Sionna RT~\cite{hoydis2023sionna}, a differentiable ray-tracing engine, to construct a realistic simulation of the target region. First, we calibrate DRT parameters with field measurements to reduce the sim-to-real gap. Then, we apply Bayesian optimization to iteratively select BS locations, evaluating coverage and capacity via Sionna RT simulations. This joint calibration and optimization pipeline enables robust, data-driven planning that improves reliability and efficiency over heuristic or purely simulation-driven methods.

% Overall, we summarize the contribution of this paper as:
% \begin{itemize}
%     \item An automatically calibrated DRT framework based on Sionna RT, integrating data-driven parameter calibrating with physics-based ray tracing; 
%     \item A Bayesian optimization based formulation for efficiently identifying optimal BS deployment locations;  
%     \item A campus-scale validation, demonstrating improved coverage and comparable capacity over heuristics and exhaustive search, with substantially lower computational cost.  
% \end{itemize}
% ultra dense network deployment
% simuations based planning, suffer sim-to-real gap; while it is costly and human laborious to do field testing 

% our idea: use crowdsourcing data from existing networks (e.g., other operators) to build a digital radio twin, and then optimize the planning deicvision over the digital radio twin

% specifically, how we actully achieve this idea
% summary of paper contributions

\section{Digital Radio Twin}
\label{sec:twinning}
In this section, we introduce the design of the DRT by finetuning the parameters of the target scenario in the simulator.

% TODO: define sim-to-real discrepancy 
\subsection{Problem}
The goal is to derive the DRT to achieve all the attributes of fidelity, tractability, and synchronicity. 
We consider the scene is given with 3D terrain, including the full set of building objects within the region of interest. 
Each building is associated with an initial material label, such as marble or concrete, which corresponds to a specific set of material parameters. 
These material labels serve as the initial configuration for the DRT simulation. 
Since these default values may not fully reflect the actual propagation environment, a calibration process is subsequently performed to refine the material parameters using real-world signal measurements, thereby improving the fidelity and reliability of the DRT.

In this work, we focus on calibrating two key electromagnetic parameters used in DRT modeling for objects within the region $\mathcal{A}$: conductivity (in S/m) and relative permittivity. 
Specifically, conductivity characterizes how a material attenuates or reflects incident electromagnetic energy, while relative permittivity affects the bending and delay of waves as they traverse the medium, both playing critical roles in DRT modeling. 
We use $\boldsymbol{\sigma} = [\sigma_1, \sigma_2, \cdots, \sigma_K ]$ and $\boldsymbol{\epsilon} = [ \epsilon_1, \epsilon_2, \cdots, \epsilon_K ]$ to denote the conductivity and relative permittivity of $K$ objects, respectively. 
The parameter set is thus defined as:
\begin{equation}
\boldsymbol{\Theta} =
\begin{bmatrix}
\sigma_1 & \sigma_2 & \cdots & \sigma_K \\
\epsilon_1 & \epsilon_2 & \cdots & \epsilon_K
\end{bmatrix}.
\end{equation}
Based on the typical electromagnetic properties of common materials~\cite{series2015effects}, we impose the following constraints on the tunable parameters: $\sigma_k \in (0, 2)$ and $\epsilon_k \in (1, 6)$.

Hence, the goal is further concretized to seek the optimal simulation parameters to minimize the sim-to-real discrepancy.


\subsection{Challenge}

Although DRT systems can simulate real-world wireless propagation, discrepancies often arise between their simulated outputs and actual measurements—referred to as the sim-to-real gap. 
This gap does not stem from a single source, but rather results from the complex interplay of multiple factors, such as multipath effects, material properties, and environmental geometry. 
Due to this multifactor coupling, the gap cannot be eliminated by simply adjusting a single parameter. 
The key challenge is to develop a systematic methodology for characterizing and mitigating this complex, coupled discrepancy.



\subsection{Solution}
To address the above problem, we develop a materiel learning method by leveraging the differentiable property of state-of-the-art simulators.

To calibrate $\boldsymbol{\Theta}$, we utilize both the measured and simulated reference signal received power (RSRP, in dBm) values. Let $\hat{\mathbf{r}} = [\hat{r}_1, \hat{r}_2, \cdots, \hat{r}_P]$ represent the RSRP collected at $P$ locations in $\mathcal{A}$ from existing BSs. The corresponding simulated RSRP vector, computed using the DRT under $\boldsymbol{\Theta}$, is denoted by $\tilde{\mathbf{r}}(\boldsymbol{\Theta}) = [\tilde r_1(\boldsymbol{\Theta}), \tilde r_2(\boldsymbol{\Theta}), \cdots, \tilde r_P(\boldsymbol{\Theta})]$. 
The calibration task is defined as
\begin{equation}
\boldsymbol{\Theta}^* = \arg\min_{\boldsymbol{\Theta}} \| \tilde{\mathbf{r}}(\boldsymbol{\Theta}) - \hat{\mathbf{r}} \|,
\end{equation}
where $\| \cdot \|$ represents the $\ell_2$ norm between the simulated and measured RSRP vectors. In practice, we minimize the empirical loss: 
\begin{equation}
    \mathcal{L}(\boldsymbol{\Theta}) = \frac{1}{P} \sum_{p=1}^P \left\| \tilde{\mathbf{r}}(\boldsymbol{\Theta}) - \hat{\mathbf{r}} \right\|.
\end{equation}
We optimize this loss over $E$ training epochs using stochastic gradient descent~\cite{boyd2010convex}. At each epoch $e \in {1, \dots, E}$, the parameters are updated as:
\begin{equation}
\label{eq: eq8}
\boldsymbol{\Theta}^{(e)} \leftarrow \boldsymbol{\Theta}^{(e-1)} - \eta \nabla \mathcal{L}(\boldsymbol{\Theta}^{(e-1)}) ,
\end{equation}
where $\eta$ is the learning rate and $\boldsymbol{\Theta}^{(e)}$ denotes the parameters after the $e$-th epoch. The final optimized parameters $\boldsymbol{\Theta}^*$ are obtained after $E$ epochs of training.











\section{Automatic Network Planning}
\label{sec:planning}
In this section, we introduce the automatic network planning algorithm by leveraging the aforementioned DRT.

\subsection{Problem}
The goal of network planning is to obtain the optimal base station parameters (e.g., location and orientation) that maximizes the specified network performance (e.g., coverage and capacity) in the aforementioned DRT, under a given geographic area.

As illustrated in Fig. \ref{fig:figure1}, we consider a geographic region $\mathcal{A} \subset \mathbb{R}^2$ with given 3D terrain, such as buildings, trees, and walkways.
Here, we denote $\hat{\mathcal{B}} = {(\hat{x}_m, \hat{y}_m)}_{m=1}^M$ as the deployed $M$ base stations (BSs).  
To further improve coverage and expand capacity, we aim to deploy an additional $N$ BSs at locations $\bar{\mathcal{B}} = {(\bar{x}_n, \bar{y}_n)}_{n=1}^N$, selected from a feasible region $\mathcal{F} \subseteq \mathcal{A}$ that meets practical placement constraints (e.g., rooftop availability, and height limits). 
The complete BS set after deployment is $\mathcal{B} = \hat{\mathcal{B}} \cup \bar{\mathcal{B}}$, and the goal is to determine $\bar{\mathcal{B}}$ to maximize overall network performance in $\mathcal{A}$.

\begin{figure}[!t]
	\centering
	\includegraphics[width=3.0in]{fig1.pdf}
	\caption{Illustration of BSs deployment in feasible region $\mathcal{F}$.}
	\label{fig:figure1}
\end{figure}

Next, we introduce two network performance metrics, including coverage and capacity. 
We discretize $\mathcal{A}$ into a uniform square grid with resolution $a\times a$ meters, yielding
$L = \left\lceil |\mathcal{A}|/{a^2} \right\rceil$ sampling points, where $|\mathcal{A}|$ is the area of $\mathcal{A}$, and $\lceil\cdot\rceil$ represents the ceiling function (i.e., rounding up to the nearest integer).
For each sampling point $\ell \in {1, \dots, L}$, the received signal characteristics are obtained by using the aforementioned DRT that reflects the real-world propagation environment. 
The querying of the DRT is governed by a set of tunable parameters $\boldsymbol{\Theta}$, such as material conductivity and permittivity, which influence signal behavior and modelling accuracy. 

We evaluate the network coverage by using $r_\ell(\boldsymbol{\Theta})$, which denotes the reference signal received power (RSRP) at sampling point $\ell$, corresponding to the strongest signal received from all BSs in the set $\mathcal{B}$.
Moreover, we evaluate the network capacity by using the signal-to-noise ratio (SNR), which captures signal quality in the presence of interference. 
Let $s_\ell(\boldsymbol{\Theta})$ represent the SNR at point $\ell$, computed under the current parameter setting $\boldsymbol{\Theta}$ in the DRT.
Given the parameter of the DRT, we can obtain RSRP and SNR values at all sampling points, and then define the following performance metrics. 
Given a predefined RSRP threshold $r_{\mathrm{th}}$, the coverage rate is defined as:
\begin{equation}
C(\mathcal{B}) = \frac{1}{L} \sum_{\ell=1}^{L} \mathds{1}\left[ r_\ell(\boldsymbol{\Theta}) > r_{\mathrm{th}} \right],
\end{equation}
where $\mathds{1}[\cdot]$ is the indicator function, returning 1 if the condition is satisfied and 0 otherwise. 
The network capacity, reflecting the average spectral efficiency across all sampling points, is expressed by:
\begin{equation}
S(\mathcal{B}) = \frac{1}{L} \sum_{\ell=1}^{L} \log_2\left(1 + s_\ell(\boldsymbol{\Theta})\right).
\end{equation}
Finally, we combine these two metrics into a composite objective function:
\begin{equation}
\label{eq:eq3}
T(\mathcal{B}) = \alpha \cdot C(\mathcal{B}) + S(\mathcal{B}),
\end{equation}
where $\alpha > 0$ is a tunable weight controlling the trade-off between coverage and capacity. 
Here, a larger $\alpha$ places greater emphasis on coverage, whereas a smaller value prioritizes capacity.

Hence, the goal of network planning is further concretized to determine the optimal set of additional deployment locations $\bar{\mathcal{B}}^*$ that maximizes the overall objective:
\begin{equation}
\label{eq:eq4}
\bar{\mathcal{B}}^* = \arg\max_{\bar{\mathcal{B}} \subseteq \mathcal{F}} T(\hat{\mathcal{B}} \cup \bar{\mathcal{B}}),
\end{equation}
where $\hat{\mathcal{B}}$ denotes the set of existing BSs and $\mathcal{F} \subseteq \mathcal{A}$ represents the feasible deployment region.



\subsection{Challenge}
The key technical challenge lies in the unknown correlation between the input of base station locations and the output of the network performance.
This is attributed to the complex computation (e.g., ray tracing and radio effects) in the DRT, which cannot be mathematically represented.
Specifically, we can obtain only the output of network performances by querying the DRT (non-negligible querying costs) under the input base station locations.
In other words, the objective function $T(\mathcal{B})$ is unknown, which falls into the realm of blackbox optimization. 


\subsection{Solution}
To address the above problem, we design an automatic network planning method by leveraging Bayesian optimization (BO).
BO~\cite{wang2023recent, garnett2023bayesian} is the state-of-the-art global searching approach (especially for handling expensive querying costs), generally consisting of a surrogate model and acquisition function.
The surrogate model aims to online learn the blackbox function (e.g., mean and variance), where candidates include Gaussian process~\cite{rasmussen2003gaussian} and Bayesian neural networks~\cite{li2023study}.
Based on the updated surrogate model, the acquisition function evaluates the customized utility of different actions (i.e., BS locations) in the action space (i.e., $\mathcal{F}$).
In each iteration, the next action is selected by maximizing the given acquisition function (e.g., EI and PI)~\cite{frazier2018tutorial}.


To support multiple base station deployment, a naive approach in BO is to use the surrogate model to learn the combinational action space (i.e., all candidate BS locations) and overall network performance.
However, this naive approach exponentially expands the action space, and practically leads to much slow convergence speed and more cumulative querying costs, in terms of time consumption and computation complexity.
Hence, we converts the problem to be a series of incremental base station deployment problems.
In other words, we only apply the Bayesian optimization to search for one additional base station deployment at a time.
Once the base station location is determined, we put it into the existing deployed base station sets (i.e., $\hat{\mathcal{B}}$), and then create another Bayesian optimization searching for the next base station.
This incremental base station stops until the total number of base stations is all deployed.

In BO, the true objective function \eqref{eq:eq3} is replaced by a surrogate model that is cheaper to evaluate but still captures the underlying spatial correlation of the objective.
Specifically, we use a Gaussian process (GP) as the surrogate model, which is widely adopted and time-evaluated in multiple application domains.
Here, we model the target function as:
\begin{equation}
T(\hat{\mathcal{B}} \cup \bar{\mathcal{B}}) \sim \mathcal{GP}\left(\mu(\bar{\mathcal{B}}), \sigma^2(\bar{\mathcal{B}}) \right),
\end{equation}
where $\mu(\bar{\mathcal{B}})$ and $\sigma^2(\bar{\mathcal{B}})$ are the posterior mean and variance of the GP given the observations collected in previous iterations.
The GP surrogate allows us to predict both the expected performance of a candidate deployment $\bar{\mathcal{B}}$ and the uncertainty of this prediction, which is crucial for balancing exploration and exploitation.

The next action to query is generally determined by maximizing the selected acquisition function, which defines the utility of all the actions in the action space while balancing exploration and exploitation. 
Specifically, we adopt the EI as the acquisition function, which quantifies the expected gain over the current best performance $T_{\text{best}}$:
\begin{equation}
\alpha(\bar{\mathcal{B}}) = \mathbb{E} \left[ \max\left(0, T(\hat{\mathcal{B}} \cup \bar{\mathcal{B}}) - T_{\text{best}}\right) \right].
\end{equation}
This criterion favours actions that are likely to improve the objective while still exploring regions of high uncertainty.
At each iteration $t$, we select the next set of deployment locations by maximizing the acquisition function over the feasible region $\mathcal{F}$:
\begin{equation}
\bar{\mathcal{B}}_{t+1} = \arg\max{\bar{\mathcal{B}} \subseteq \mathcal{F}} \alpha (\bar{\mathcal{B}}).
\end{equation}
The newly selected action is then evaluated using the derived DRT model (see Sec.~\ref{sec:twinning}), and the resulting observation is incorporated into the GP to update its posterior distribution. 
This iterative process continues until a stopping criterion, such as a fixed iteration limit or convergence threshold, is met. 






\section{The AutoPlan Algorithm}
\label{sec:solution}

% algorithm pseudo code
% \begin{figure*}[!t]
% 	\centering
% 	\includegraphics[width=7in]{fig2.pdf}
%     \captionsetup{justification=raggedright, singlelinecheck=false}
% 	\caption{The workflow of the \emph{AutoPlan} algorithm.}
% 	\label{fig:figure2}
% \end{figure*}


\begin{algorithm}[!h]
\caption{The \emph{AutoPlan} Algorithm}
\label{alg:algorithm1}
    
    \KwIn{
    Region $\mathcal{A}$, feasible subregion $\mathcal{F}$, \\
    \quad Existing BS set $\hat{\mathcal{B}} = \{(\hat{x}_m, \hat{y}_m)\}_{m=1}^M$,\\
    \quad Measured RSRP $\hat{\mathbf{r}}$, simulated RSRP $\tilde{\mathbf{r}}(\boldsymbol{\Theta})$,\\
    \quad Grid size $a$, RSRP threshold $r_{\text{th}}$,\\
    \quad Number of new BSs $N$, training epochs $E$
    }
    
    \KwOut{
    Calibrated parameters $\boldsymbol{\Theta}^*$, new BS locations $\bar{\mathcal{B}} = \{(\bar{x}_n, \bar{y}_n)\}_{n=1}^N$
    }
    
    \textbf{Stage 1: Digital radio twin calrbating}\;
    Initialize $\boldsymbol{\Theta}$ randomly\;
    \For{$t \gets 1$ \KwTo $E$}{
        Compute simulated RSRP $\tilde{\mathbf{r}}(\boldsymbol{\Theta})$ using DRT\;
        Compute loss $\mathcal{L}(\boldsymbol{\Theta})$ between $\hat{\mathbf{r}}$ and $\tilde{\mathbf{r}}(\boldsymbol{\Theta})$\;
        Update parameters $\boldsymbol{\Theta}$\ according to (\ref{eq: eq8});
    }
    Set $\boldsymbol{\Theta}^* \gets \boldsymbol{\Theta}$\;
    
    \vspace{1mm}
    \textbf{Stage 2: Automatic network planning}\;
    Initialize $\bar{\mathcal{B}} \gets \emptyset$, observation set $\mathcal{D} \gets \emptyset$\;
    
    \For{$n \gets 1$ \KwTo $N$}{
        Fit GP surrogate model to $\mathcal{D}$\;
        Compute acquisition function $\alpha(\bar{b})$ using Expected Improvement ($\bar{b}=(\bar x,\bar y)$)\;
        Select next BS: $\bar{b}^* \gets \arg\max\limits_{\bar{b} \in \mathcal{F} \setminus (\hat{\mathcal{B}} \cup \bar{\mathcal{B}})} \alpha(\bar{b})$\;
        Evaluate $T(\hat{\mathcal{B}} \cup \bar{\mathcal{B}} \cup \{\bar{b}^*\})$ using calibrated DRT\;
        Update: $\bar{\mathcal{B}} \gets \bar{\mathcal{B}} \cup \{\bar{b}^*\}$\;
        Add $(\bar{b}^*, T)$ to $\mathcal{D}$\;
    }

\Return{$\boldsymbol{\Theta}^*$, $\bar{\mathcal{B}}$}
\end{algorithm}


In this section, we summarize the workflow of the \emph{AutoPlan} algorithm in Alg.~\ref{alg:algorithm1}.
Overall, the algorithm takes the input of 3D terrain of the geographic region and the location of deployed base stations $\hat{\mathcal{B}}$, and generates the output of the location of the $N$ additional base stations. 
In the stage of digital radio twinning, it observes the real-world RSRPs $\hat{\mathbf{r}}$ from crowdsourced mobile users, and updates the material of buildings.
Specifically, it iteratively calculates the loss function by comparing the real-world RSRPs with the simulated RSRPs from the simulator, and performs multiple gradient descent steps.
In the stage of network planning, it iteratively search the location of the next base station by 1) selecting a base station location (i.e., maximizing the EI acquisition function); 2) querying the DRT to obtain the network performance; and 3) updating the GP surrogate model with new observation.





\section{Results}
\label{sec:results}


\subsection{Methodology}
\label{sec:methodology}

\textbf{Video Diffusion Model.} We evaluate \X using the following open source, widely available video models to generate the videos:
\begin{itemize}
    \item Wan 2.1~\cite{wan} 1.3B, 14B, 480p and 720p models, at 81 frames.
    \item HunyuanVideo 720p~\cite{hunyuanvideo} at 720p. 81 frames.
\end{itemize}

All experiments are conducted using bfloat16 precision. We implement CUDA kernels for \X with the aid of device primitives from ThunderKittens~\cite{thunderkittens} for a H100 GPU. To evaluate the quality of the videos generated, we use the VBench~\cite{vbench} VLM benchmarking scores, alongside visual comparisons of frames from the generated videos. We test two configurations of \X: one using the caching strategy to determine the mask (\X-cached), and the other using the pooling strategy (\X-pooling). For the \X-cached strategy, the threshold is set to $0.5/N$, where $N$ is the number of embedding vectors in the latent space representation of the video. The attention mask is cached once every $15$ DiT iterations. We compare \X with two prior works that use block sparse attention to leverage sparsity in attention scores in DiTs: Radial Attention~\cite{radialattn} and SparseVideoGen~\cite{sparsevideogen}. SparseVideoGen~\cite{sparsevideogen} uses a local-global attention computation strategy (windowed attention) across spatio templaral tokens. Radial attention uses a static attention mask that leads to an exponentially decaying compute density along the antidiagonal of the attention map.

\subsection{End-to-end Speedup}
\label{sec:e2espeedup}

Fig.~\ref{fig:e2e_normalized} shows the end-to-end time required to generate the video, normalized to baseline. We observe that \X is able to achieve an average speedup of $1.48\times$ and up to $1.65\times$. 
\X achieves a speedup as a result of accelerating the attention computation time during training. Fig.~\ref{fig:attn_normalized} shows the average runtime needed to compute the attention of every layer, normalized to the PyTorch implementation baseline. For the attention computation, \X achieves a speedup of $1.93\times$ on average, up to $2.38\times$. \X achieves a higher speedup when generating videos at 720p.
Our approach achieves a higher speedup of $1.2\times$ compared to SparseVideoGen~\cite{sparsevideogen} and $1.22\times$ compared to RadialAttention~\cite{radialattn}. The observed speedup comes from skipping a larger fraction of attention scores. However, this advantage diminishes at higher video resolutions (720p compared to 480p). This is because, in self-attention, interactions between blocks of embeddings that correspond to distant regions of the video are typically zero. As the resolution increases, each embedding vector covers a smaller region of the input, leading to a greater number of embeddings. This increases the proportion of zero-valued attention scores, which block-sparse attention can skip. Consequently, while more scores are skipped, the relative speedup achieved by \X decreases.



\begin{figure}[!htb]
    \centering
    \includegraphics[trim=0 90 0 80, clip, width=\linewidth]{figs2/e2enorm_speedup.pdf}
    \caption{Normalized end-to-end speedup in seconds for video generation.}
    \label{fig:e2e_normalized}
\end{figure}

\begin{figure}[!htb]
    \includegraphics[trim=0 90 0 90, clip, width=\linewidth]{figs2/attnnorm_speedup.pdf}
    \caption{Normalized attention computation speedup compared to baseline.}
    \label{fig:attn_normalized}
\end{figure}

  

% \begin{figure}[!htb]
%     \centering
%     \begin{subfigure}{0.5\textwidth}
%         \includegraphics[width=\linewidth]{figs2/e2enorm_speedup.pdf}
%         \caption{Normalized end-to-end speedup in seconds for video generation.}
%         \label{fig:e2e_normalized}
%     \end{subfigure}
%     \hfill
%     \begin{subfigure}{0.5\textwidth}
%         \includegraphics[width=\linewidth]{figs2/attnnorm_speedup.pdf}
%         \caption{Normalized attention computation speedup compared to baseline.}
%         \label{fig:attn_normalized}
%     \end{subfigure}
%     \caption{Normalized performance comparison for video generation.}
%     \label{fig:normalized_comparison}
% \end{figure}


\subsection{Qualitative Analysis}
\label{sec:qualitative_analysis}


Table~\ref{tab:vbench} shows the VBench~\cite{vbench} video benchmarking results when compared to the baseline. We observe that \X achieves negligible degradation in quality when compared to the baseline.

% Please add the following required packages to your document preamble:
% \usepackage[table,xcdraw]{xcolor}
% Beamer presentation requires \usepackage{colortbl} instead of \usepackage[table,xcdraw]{xcolor}
\begin{table*}
\centering
\caption{VBench quality metrics}
\label{tab:vbench}
\begin{tabular}{|l|r|r|r|r|}
\hline
                          & \multicolumn{1}{l|}{\textit{\textbf{\begin{tabular}[c]{@{}l@{}}Aesthetic\\ Quality\end{tabular}}}} & \multicolumn{1}{l|}{\textit{\textbf{\begin{tabular}[c]{@{}l@{}}Subject\\ Consistency\end{tabular}}}} & \multicolumn{1}{l|}{\textit{\textbf{\begin{tabular}[c]{@{}l@{}}Background\\ Consistency\end{tabular}}}} & \multicolumn{1}{l|}{\textit{\textbf{\begin{tabular}[c]{@{}l@{}}Overall\\ Consistency\end{tabular}}}} \\ \hline
Wan-1.3B 480p baseline    & 0.601                                                                                              & 0.936                                                                                                & 0.958                                                                                                   & 0.23                                                                                                 \\
Wan-1.3B 480p FGAttn      & 0.605                                                                                              & 0.939                                                                                                & 0.96                                                                                                    & 0.23                                                                                                 \\ \hline
Wan-1.3B 720p Baseline    & 0.61                                                                                               & 0.944                                                                           & 0.962                                                                                                   & 0.233                                                                                                \\
Wan-1.3B 720p FGAttn      & 0.61                                                                                               & 0.944                                                                                                & 0.964                                                                                                   & 0.232                                                                                                \\ \hline
Wan-14B 480p baseline     & 0.623                                                                                              & 0.953                                                                                                & 0.97                                                                                                    & 0.25                                                                                                 \\
Wan-14B 480p FGAttn       & 0.616                                                                                              & 0.952                                                                                                & 0.975                                                                                                   & 0.247                                                                                                \\ \hline
Wan-14B 720p baseline     & 0.621                                                                                              & 0.945                                                                                                & 0.969                                                                                                   & 0.248                                                                                                \\
Wan-14B 720p FGAttn       & {\color[HTML]{000000} 0.619}                                                                       & 0.942                                                                                                & {\color[HTML]{000000} 0.961}                                                                            & 0.245                                                                                                \\ \hline
Hunyuan-13B 720p baseline & 0.62                                                                                               & 0.944                                                                                                & 0.962                                                                                                   & 0.239                                                                                                \\
Hunyuan-13B 720p FGAttn   & 0.62                                                                                               & 0.94                                                                                                 & 0.962                                                                                                   & 0.239                                                                                                \\ \hline
\end{tabular}
\end{table*}

Figs.~\ref{fig:hunyuanvideo}, \ref{fig:wan1_3b} and \ref{fig:wan14b} show the visual representation of the produced video compared to the original (the top row of each set of videos represents the baseline video) for the HunyuanVideo model, Wan 1.3B model, and the Wan 14B model, respectively. We find that across all the prompts tested here, \X can recover the original video with no quality degradation. \X also retains the generated video style and does not significantly shift the distribution captured by the underlying model. 





\subsection{Ablation Study}
\label{sec:ablation}

Fig.~\ref{fig:ablation} depicts the average attention computation time for video generation as the threshold parameter is varied. We sweep the threshold parameter from $0.1/N$ to $1/N$, where $N$ is the number of embedding vectors in the latent space representation of the video. A higher threshold enables skipping a larger amount of computation, thereby leading to a speedup.  

\begin{figure}[!htb]
    \centering
    \includegraphics[width=\linewidth]{figs2/ablation.pdf}
    \caption{Normalized video generation time at different thresholds applied to \X-cached.}
    \label{fig:ablation}
\end{figure}



\begin{figure*}[!htb]
    \centering
    \includegraphics[width=\linewidth]{figs2/hunyuan.pdf}
    \caption{Samples of videos generated using baseline HunyuanVideo model, and \X-HunyuanVideo (The baseline generates first row, second row generated using \X)}
    \label{fig:hunyuanvideo}
\end{figure*}




\begin{figure*}[!htb]
    \centering
    \includegraphics[width=\linewidth]{figs2/wan1_3b.pdf}
    \caption{Samples of videos generated using baseline Wan-1.3B model, and \X-Wan1.3B. (First row is generated by the baseline, second row is generated using \X)}
    \label{fig:wan1_3b}
\end{figure*}





\begin{figure*}[!htb]
    \centering
    \includegraphics[width=\linewidth]{figs2/wan14b.pdf}
    \caption{Samples of videos generated using baseline Wan-14B model, and \X-Wan14B (First row is generated by the baseline, second row is generated by \X)}
    \label{fig:wan14b}
\end{figure*}





\section{Related Work}

Traditional approaches to BS deployment include both manual measurement and optimization-based methods. Early efforts often relied on drive testing or site surveys to evaluate signal quality at predefined locations~\cite{molisch2022wireless}, which are labor-intensive, time-consuming, and inflexible in adapting to changing user demands. To overcome these limitations, optimisation-based methods were introduced to automate the process using mathematical formulations or heuristics. 
Chiaraviglio et al.~\cite{chiaraviglio20215g} formulated BS planning as a mixed-integer linear program under service and EMF constraints, and proposed a heuristic solution named PLATEA. While effective in balancing coverage and EMF compliance, the method relies on static environmental models and sensitive parameters, limiting its adaptability in dynamic scenarios. Similarly, Philip et al.\cite{philip2023cellular} applied meta-heuristic algorithms such as particle swarm optimization (PSO) and genetic algorithms (GA) for BS placement. Although PSO achieved better coverage and efficiency than GA, both approaches depend on simplified propagation models and static user distributions, making them less suitable for realistic and data-driven deployments.

%……%6
In contrast, AI/ML based methods aim to enhance scalability and adaptability by replacing manual engineering and handcrafted models with data-driven learning. 
AutoBS~\cite{lee2025autobs} employs reinforcement learning to select BS locations, enabling fast inference during deployment. However, it requires large-scale pretraining on labeled environment performance pairs, which is both computationally expensive and time-consuming to collect. 
OSSN~\cite{zheng2024transformer}, a Transformer-based framework, unifies radio map estimation and site selection to reduce reliance on exhaustive candidate evaluations. While it improves planning efficiency, its performance remains limited by the availability of training data and a lack of real-world validation. 
Similarly, Dai and Zhang~\cite{dai2020propagation} take a hybrid approach by integrating machine learning with heuristic optimization, using a dataset of over $0.76$ million measurements to predict signal strength without explicit propagation modeling. 
% However, it comes at the cost of high data requirements and significant computational overhead.

However, existing AI/ML-based methods require extensive labelled datasets and incur substantial data collection and training costs. 
More importantly, they tend to overlook the sim-to-real discrepancy introduced by inaccurate or incomplete modelling of physical environments, which hinders generalization in real-world deployments.




\section{Conclusion}
In this paper, we propose a new automatic network planning framework (\emph{AutoPlan}) by leveraging digital radio twin techniques.
We derive the DRT by fintuning the parameters of building materials to reduce the sim-to-real discrepancy based on crowdsource real-world user data.
Leveraging the DRT, we design a Bayesian optimization based algorithm to efficiently search for optimal base station parameters.
Using the field measurement from Husker-Net, we extensively evaluate \emph{AutoPlan} under various deployment scenarios, in terms of coverage, capacity, and efficiency.



% \section*{Acknowledgement}
% This work is supported by the US National Science Foundation under Grant No. 2333164, and No. 2428427. \textcolor{red}{XXX USDoT project info here}

\bibliographystyle{IEEEtran}
% argument is your BibTeX string definitions and bibliography database(s)
\bibliography{ref/reference, ref/qiang, ref/ref}

\end{document}

