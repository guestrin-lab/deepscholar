%%%%%%%% mlsys 2025 EXAMPLE LATEX SUBMISSION FILE %%%%%%%%%%%%%%%%%

\documentclass{article}

% Recommended, but optional, packages for figures and better typesetting:
\usepackage{microtype}
\usepackage{graphicx}
% \usepackage{subfigure}
\usepackage{subcaption}
\usepackage{booktabs} % for professional tables

% hyperref makes hyperlinks in the resulting PDF.
% If your build breaks (sometimes temporarily if a hyperlink spans a page)
% please comment out the following usepackage line and replace
% \usepackage{mlsys2025} with \usepackage[nohyperref]{mlsys2025} above.
\usepackage{hyperref}
\usepackage{amsmath}
\usepackage{tikz}


% Attempt to make hyperref and algorithmic work together better:
\newcommand{\theHalgorithm}{\arabic{algorithm}}

\newcommand*\circled[1]{\tikz[baseline=(char.base)]{
            \node[shape=circle,fill,inner sep=1pt] (char) {\textcolor{white}{#1}};}}


% Use the following line for the initial blind version submitted for review:
% \usepackage{mlsys2025}
\usepackage[accepted]{mlsys2025}



% \usepackage{xcolor}
\usepackage[x11names,dvipsnames,table]{xcolor}
% \usepackage[usenames,dvipsnames]{color}
\usepackage{url}
\usepackage{xspace}
\usepackage[capitalize]{cleveref}
\usepackage{graphicx}
\usepackage{subcaption}
\usepackage{float} % Required for the [H] specifier
\usepackage{wrapfig} % Required for wrapping

\pagestyle{plain}
 

\newcommand{\todo}[1]{\textcolor{red}{Todo: \emph{#1}}}
\newcommand\tofix[1]{\noindent{\color{magenta}{#1}}} 

\newcommand\X{\noindent{FG-Attn}\xspace} 

% \title{
% \X: Efficient Attention for DiTs Using Fine-grained Sparsity
% }



% If accepted, instead use the following line for the camera-ready submission:

% The \mlsystitle you define below is probably too long as a header.
% Therefore, a short form for the running title is supplied here:
\mlsystitlerunning{\X: Leveraging Fine-Grained Sparsity in Diffusion Transformers}

\begin{document}

\twocolumn[
\mlsystitle{\X: Leveraging Fine-Grained Sparsity in Diffusion Transformers}

% It is OKAY to include author information, even for blind
% submissions: the style file will automatically remove it for you
% unless you've provided the [accepted] option to the mlsys2025
% package.

% List of affiliations: The first argument should be a (short)
% identifier you will use later to specify author affiliations
% Academic affiliations should list Department, University, City, Region, Country
% Industry affiliations should list Company, City, Region, Country

% You can specify symbols, otherwise they are numbered in order.
% Ideally, you should not use this facility. Affiliations will be numbered
% in order of appearance and this is the preferred way.
\mlsyssetsymbol{equal}{*}

\begin{mlsysauthorlist}
\mlsysauthor{Sankeerth Durvasula}{to,vec,intern}
\mlsysauthor{Kavya Sreedhar}{goo}
\mlsysauthor{Zain Moustafa}{to}
\mlsysauthor{Suraj Kothawade}{goo}
\mlsysauthor{Ashish Gondimalla}{goo2}
\mlsysauthor{Suvinay Subramanian}{goo}
\mlsysauthor{Narges Shahidi}{equal,goo}
\mlsysauthor{Nandita Vijaykumar}{equal,to,vec}
\end{mlsysauthorlist}

\mlsysaffiliation{intern}{Sankeerth Durvasula was supported by an internship at Google}
\mlsysaffiliation{to}{Department of Computer Science, University of Toronto, Toronto, Canada}
\mlsysaffiliation{goo}{Google, Mountain View, USA}
\mlsysaffiliation{vec}{Vector Institute, Toronto, Canada}
\mlsysaffiliation{goo2}{Google, Sunnyvale, USA}
% \mlsysaffiliation{equal}{.}




\mlsyscorrespondingauthor{Sankeerth Durvasula}{sankeerth@cs.toronto.edu}
% \mlsyscorrespondingauthor{Eee Pppp}{ep@eden.co.uk}

% You may provide any keywords that you
% find helpful for describing your paper; these are used to populate
% the "keywords" metadata in the PDF but will not be shown in the document
\mlsyskeywords{Machine Learning, MLSys}

\vskip 0.3in

\begin{abstract}
Generating realistic videos with diffusion transformers demands significant computation, with attention layers becoming the central bottleneck.
Even producing a short clip requires running a transformer over a very long sequence of embeddings, e.g., more than 30K embeddings for a 5-second video. These long sequence lengths thus incur significant compute latencies. Prior work aims to mitigate this bottleneck by exploiting sparsity in the attention layers to reduce the computation required. However, these works typically rely on block-sparse attention, which skips score computation only when all entries in a \emph{block} of attention scores (corresponding to $M$ queries and $M$ keys, with $M=64$ typically) are zero. This coarse-granular skipping of attention scores does not fully exploit sparsity in the attention map and leaves significant room for improvement.


In this work, we propose \X, a sparse attention mechanism for long-context diffusion transformers that leverages sparsity at a fine granularity. 
Unlike block-sparse attention, which skips entire $M \times M$ blocks, our approach skips computations at the granularity of $M \times 1$ \emph{slices} of the attention map. Each slice is produced as a result of query-key dot products between a block of query vectors and a \emph{single key}. To implement our proposed sparse attention mechanism, we construct a new highly efficient bulk-load operation called asynchronous-gather load. This load operation gathers a sparse set of relevant key-value vectors from memory and arranges them into packed tiles in the GPU's shared memory. In this manner, only a sparse set of keys relevant to those queries are loaded into shared memory when computing attention for a block of queries, in contrast to loading full blocks of key tokens in block-sparse attention. Our fine-grained sparse attention, applied to video diffusion models, achieves an average 1.55x (up to 1.65x) speedup for 5 second, 480p videos, and an average 1.41x (up to 1.49x) for 5 second, 720p videos on a single H100 GPU.

\textbf{\textcolor{magenta}{Code:}} \url{https://github.com/sankeerth95/FG-Attn}

% \begin{center}
% \begingroup
% \endgroup
% \end{center}


\end{abstract}
]


% this must go after the closing bracket ] following \twocolumn[ ...

% This command actually creates the footnote in the first column
% listing the affiliations and the copyright notice.
% The command takes one argument, which is text to display at the start of the footnote.
% The \mlsysEqualContribution command is standard text for equal contribution.
% Remove it (just {}) if you do not need this facility.

% \printAffiliationsAndNotice{}  % leave blank if no need to mention equal contribution
\printAffiliationsAndNotice{\mlsysEqualContribution} % otherwise use the standard text.

\thispagestyle{plain}

\section{Introduction}\label{sec:introduction}
Mixture-of-Experts (MoE) is an architectural paradigm that adaptively combines predictions from multiple neural modules, known as "experts," via a learned gating mechanism. This concept has evolved from ensemble-based MoEs, where experts, jointly trained with a gating function, are often full, independent models whose outputs are combined to improve overall performance and robustness \citep{jacobs1991adaptive}. More recently, MoE layers have been integrated within larger neural architectures, with experts operating in a latent domain. These "latent MoEs" offer significant scalability benefits, especially in large language models (LLMs) \citep{shazeer2017outrageously,fedus2022switch}.
MoE makes it possible to train massive but efficient LLMs, where each token activates only a fraction of the model’s parameters, enabling specialization, better performance, and lower computational cost compared to equally sized dense models.

Regardless of their specific implementation, conventional MoE systems typically produce point estimates, lacking a direct quantification of their uncertainty. In critical applications, this absence of uncertainty information hinders interpretability, making it difficult for users to gauge the reliability of a prediction and limits informed decision-making, as the system cannot express its confidence or identify ambiguous cases. Importantly, the learned gating mechanism, which dictates the relative contribution of each expert, does not take into account expert confidence, potentially leading to suboptimal routing decisions.

In this work, we propose Mixture-of-Gaussians with Uncertainty-based Gating (MoGU), a framework for uncertainty-aware MoE architectures, which provides explicit uncertainty quantification for both individual experts and the overall MoE model. Our approach fundamentally reimagines the expert's output: instead of a point estimate, we model each expert's prediction as a random variable drawn from a normal distribution. In this setup, each expert simultaneously predicts both the mean (the label estimate) and variance of the distribution, representing its predictive uncertainty. This shift enables a more nuanced understanding of expert behavior and the derivation of the overall model's uncertainty. Furthermore, we introduce a novel gating mechanism where the estimated uncertainty of each expert directly informs its relative contribution to the overall MoE prediction, bypassing the need for a separate gating function typically found in traditional MoE setups. This creates a self-aware MoE where more confident experts naturally exert greater influence.

We evaluate MoGU on time series forecasting as our primary regression task. This choice is motivated by the inherent uncertainty in real-world time series data and the wide variety of expert architectures applicable to forecasting tasks across numerous domains \citep{time_series_survey, wang2024deep}. Our evaluation spans various expert types, forecasting benchmarks and forecasting horizon sizes, allowing for a comprehensive assessment of our method's efficacy. MoGU is shown to consistently yield more accurate forecasts compared to input-based gating MoE architectures, while simultaneously, providing uncertainty estimates that are positively correlated with prediction error. These estimates are available at both the individual expert and overall model levels. By further distinguishing between aleatoric (data-related) and epistemic (model-related) uncertainty, MoGU offers valuable insights into the source of a model's uncertainty. We also conducted a detailed ablation study to validate our key design choices.

In summary, our contributions are as follows: 
\begin{itemize}
\item \textbf{MoGU: A Novel Framework for Uncertainty-Aware MoE Architectures}: We introduce a novel framework that directly quantifies uncertainty for both individual experts and the overall model, moving beyond conventional point estimates. A key innovation is a routing mechanism that uses each expert’s estimated predictive uncertainty to dynamically determine its contribution to the final MoE output, replacing traditional input-based gating mechanisms.
\item \textbf{MoGU Improves Time Series Forecasting}: Our method effectively reduces forecasting error across various benchmarks, horizon lengths, and expert architectures.
\item \textbf{MoGU Provides Meaningful Uncertainty Estimates for Time Series Forecasting}: MoGU generates uncertainty estimates at the expert-level and overall. These estimates are positively correlated with prediction error, providing valuable insight into the model's confidence and the sources of its uncertainty.
\end{itemize}

By embedding uncertainty estimation into prediction and gating, MoGU moves beyond input-based gating  MoEs toward architectures that are more accurate, transparent, and reliable.



\section{Underlying Hardware and Failure Model}
Persistency semantics are important for a range of new memory
hardware. A well-studied example is Px86~\cite{intelx86}, the semantics of persistent memory for x86 processors, which adds a global \emph{persistent buffer} that holds stores after they exited the thread-local store buffers. Stores in the persistent buffer are written back to memory in arbitrary order at cache line granularity, and those not written back are lost upon a crash. To avoid inconsistent state caused by crashes under persistency semantics, fence and flushes instructions are often needed to impose persistency ordering between stores. 

While the problem has received much attention for
persistent memory, the same problem exists for CXL shared
 memory.  Figure~\ref{fig:memorysystem} presents a
graphical overview of CXL shared disaggregated memory.  In this
setting, processing nodes and memory nodes are connected via CXL
networking.  Memory can be coherently shared between multiple hosts
--- memory nodes contain directories that track which cache lines have
been requested by processing nodes.  CXL on x86 is intended to provide
TSO ordering guarantees across machines.

Failures of compute nodes pose a consistency problem.  Compute nodes
cache their writes to CXL shared memory and the contents of these
caches can be written back in an arbitrary order.  The CXL standard
provides for global persistent flush (GPF).  When GPF is supported, it uses
energy reserves to write the contents of the CPU cache back to the underlying storage upon a crash or system shutdown. However, GPF does not cover a wide
range of failure modes, \eg failed cables, failed CXL transceivers,
failed CPUs, failed motherboards, etc, and therefore does not provide a complete solution to the consistency problem.

If even a single machine that is updating a given CXL memory region
fails, the data in that machine's cache is corrupted for all machines
that access the CXL memory region.  In our email discussions with
Intel engineers, it is not yet decided how CXL will expose such
failures to the software layer.  One option is to report those
cache lines as poisoned and throw exceptions on access and another
is to allow software to access the copy of the data that
resides in the CXL memory node and let software manage flushing data
with flush and fence operations in a manner similar to 
PM systems.

\begin{figure}[!htbp]
\begin{center}
  \includegraphics[scale=0.26]{figures/cxl}
\end{center}
\vspace{-.2cm}
  \caption{CXL Memory System\label{fig:memorysystem}}
\end{figure}

\textbf{PM can be viewed as a special case of the CXL shared memory
model with the differences that: (1) there is only one processing node and
 (2) after a crash, the node can return having lost the state of its
local cache.}

The x86 architecture provides the following instructions to force the cache
to write data back to persistent storage: (1) the  \code{clflush} and \code{clflushopt} instructions that
flush (and evict) a cache line and (2) the 
\code{clwb} instruction that writes back a cache line potentially without eviction.  Each of these instructions takes as input the address to flush.  The \code{clflush} instruction flushes a cache line immediately while the \code{clflushopt} and \code{clwb} instructions are not guaranteed to flush or write back a cache line until the thread executes a fence instruction.



\section{Analysis}
\label{sec:motivation}

\subsection{Latency of Video DiT Models}
\label{sec:motivation_videodit_modelperformance}

% \begin{table}[!ht]
% \centering
% \caption{Amount of time required to produce a video using a single H100 GPU chip.}
% \label{tab:attention_breakdown}
% \begin{tabular}{|l|l|l|l|l|}
% \hline
%           & Wan 1.3B, 480p & Wan 1.3B, 720p & Wan 14B, 480p & Wan 14B, 720p \\
% \hline
% 49 frames & 101s & 354s & 513s  & 1727s \\
% 81 frames & 204s & 794s & 1024s & 3823s \\
% \hline
% \end{tabular}
% \end{table}
\begin{table}[!htb]

% \begin{table}[!ht]
\centering
\begin{tabular}{|l|l|l|}
\hline
               & 49 frames  & 81 frames  \\
\hline
Wan 1.3B, 480p & 101s & 204s  \\
Wan 1.3B, 720p & 354s & 794s  \\
Wan 14B, 480p  & 513s & 1024s \\
Wan 14B, 720p  & 1727s & 3823s  \\

\hline
\end{tabular}
% \end{table}
\caption{Time to produce a video using a single H100 GPU chip.}
\label{tab:attention_breakdown}

\end{table}

% \begin{wrapfigure}{r}{0.5\textwidth} % r=right, width=50% of text width



As discussed in~\cref{sec:background_diffusiontransformer}, video-DiT models first encode a set of video frames as latent embedding vectors using a vector-quantized variational autoencoder (VQVAE). The VQVAE compresses each spatiotemporal patch of $H \times W$ pixels across $F$ consecutive frames into a single latent vector. For instance, setting $H=W=8$ and $F=4$ maps every $8\times 8$ pixel block over $4$ frames into one latent embedding vector. Even a short video, such as a five-second video at $480\times 832$ resolution, gets encoded as approximately $32000$ embedding vectors. Attention layers over such a large number of embeddings require substantial computation, as each layer must process an massive number of floating-point operations to generate even a short video.

Table~\ref{tab:attention_breakdown} shows the amount of time required to produce a 5s video at 720p using the Wan 2.1 1.3B and 14B models~\cite{wan} on a single H100 GPU chip. We see that a significant amount of time, about 10 minutes using Wan 14B model, is required to produce even a short video.
\begin{figure}[!htb]
    \centering
    \includegraphics[width=\linewidth]{figs_1/attnbreakdown}
    \caption{Breakdown of time spent (in seconds) by different operations during inference of Wan 2.1 1.3B~\cite{wan} (eager mode).}
    \label{fig:attnbreakdown}
\end{figure}
In Fig.~\ref{fig:attnbreakdown}, we depict a breakdown of the time required by each operator to produce the video (note that ``others'' here indicate the operators to encode the text tokens and the initial noisy video frames using the VQVAE). 
We observe that the majority of computation time is spent evaluating the transformer model, with the attention layer accounting for most of this cost. Specifically, when processing long sequences of embeddings (of length $N$), the attention operation scales as $O(N^2)$, in contrast to components such as the feed-forward network, which scale as $O(N)$. Consequently, for video-DiT models, longer sequences—arising from higher-resolution or longer videos—lead to an even greater fraction of time being dominated by attention. For example, in Wan 2.1 1.3B~\cite{wan}, nearly $91\%$ of the runtime is spent computing attention when generating 81 frames at 720p, compared to the already-high $76\%$ for 49 frames at 480p.

\subsection{Attention Sparsity}
\label{sec:motivation_attentionsparsity}
\label{sec:motivation_blocksparseattention}

\begin{figure}[!htb]
    \centering
    \includegraphics[width=\linewidth]{figs2/fig4.pdf}
    \caption{Sparsity in attention computation: attention scores are highly sparse, and locations of negligible attention scores are irregularly distributed.}
    \label{fig:distribution_of_attnvalues}
\end{figure}

Fig.~\ref{fig:distribution_of_attnvalues} shows heatmap of the attention scores in one attention heads of the Wan 2.1 1.3B vDIT model~\cite{wan}. 
We observe that the vast majority of attention scores are close to zero. Since many $q,k$ pairs produce negligible attention scores, attention computation can be significantly accelerated by bypassing these score computations.

Prior sparse attention mechanisms such as FlexAttention~\cite{flexattn}, block-sparse attention~\cite{bsa} and FlashAttention~\cite{flashattn} skip attention score computations to enable faster execution. In practice, these methods avoid loading and computing pairwise $q-k$ dot-products between contiguous sets (blocks) of $q,k$ vectors (typically 64 queries or 64 keys). The block size $M$, or the number of queries/keys to skip computing attention score over is determined by the matrix multiplication operation dimensions in the tensor core of the accelerator (e.g., $64 \times 64$ on H100 GPUs). This allows $M \times M$ query-dot product computation to be skipped, corresponding to attention scores computed from $M$ queries and $M$ keys. However, in order to maintain accuracy, computing a block of attention scores can be skipped only when \emph{all} query–key pairs within the block produce negligible attention scores.

\begin{table}[!htb]
\centering
\caption{Block sparsity in attention scores: percentage of $M\times M$ blocks of attention scores where all scores are below a threshold. The threshold is set to $0.5/N$, where $N$ is the sequence length of the input to the transformer.}
\label{tab:attn_block_sparsity}
\begin{tabular}{|l|l|}
\hline
\textbf{Block size} & \textbf{Sparsity} \\
\hline
$128\times128$ & 5.5\% \\
$64\times64$ & 22.8\% \\
$32\times32$ & 47.7\% \\
$16\times16$  & 70.7\% \\
\hline
\end{tabular}
\end{table}


\textbf{Sparsity in attention scores is fine-grained.} 
\label{sec:motivation_skip_attn_fine_granularity}
Table~\ref{tab:attn_block_sparsity} shows the sparsity of the attention map at different block sizes of attention scores. 
We find that skipping finer-grained blocks ($16 \times 16$) yields about 70\% sparsity, whereas coarser blocks ($64 \times 64$) achieve only 22\%. This suggests that operating at finer granularity provides a greater opportunity for speedup. However, existing block-sparse attention implementations typically skip computations only at coarse block sizes. Current block sparse attention mechanisms~\cite{bsa, flexattn} are unable to leverage finer-grain sparsity (below $64\times 64$). In this work, we aim to exploit the higher sparsity available in finer-grained blocks to design a more efficient block-sparse attention mechanism that substantially reduces FLOPs without sacrificing accuracy.


\section{LLM-based Multi-Agent Blackboard System}

This section introduces an alternative communication paradigm for LLM-based multi-agent systems inspired by blackboard systems \citep{10.1145/356810.356816}, distinct from the widely used master–slave architecture. As outlined in \textsection \ref{sec:introduction}, blackboard-based multi-agent systems provide several advantages over the master-slave approach. Here, rather than directly assigning tasks to sub-agents, the main agent posts its requests (i.e., sub-tasks for which it requires assistance) on a shared blackboard, which functions as a broadcast channel accessible to all other agents. Each helper agent independently evaluates whether it can respond to a request, considering its own capabilities, availability, cost, and other factors. If an agent decides to contribute, it writes its response to the corresponding request, and the main agent then decides whether to use or ignore the provided information. \textit{This way, all agents in the system retain full autonomy over their actions, and no centralized controller forces them to execute a specific task.} While the blackboard paradigm is applicable to a wide range of multi-agent systems, we focus on data science tasks that require data discovery, where its characteristics are particularly advantageous, as discussed in \textsection \ref{sec:introduction}. The remainder of this section details our method and its design for data science problems that require information discovery.



\paragraph{Overview:} 

An overview of our proposed method is presented in Figure~\ref{fig:overview}. The system $\pi_{s}$ operates over the data lake $\sD$ by first partitioning $\sD$ into $C$ clusters of related files. Each cluster $\sD_i$ is assigned to a file agent $\pi_{f_i}$, which is responsible for handling, loading, processing, and retrieving information from the files within its cluster. In addition, a search agent $\pi_{s}$ is included to retrieve external information from the web that may be required to solve the problem. The overall system $\pi_{s}$ is composed of a main agent $\pi_{m}$, which is responsible for solving the query $q$, and a set of $C+1$ helper agents $\Pi_{\text{helper}} = \{\pi_{f_i}\}_{i=1}^{M} \cup \{\pi_{s}\}$ that provide specialized assistance. The query $q$ is presented to $\pi_{m}$, which iteratively selects an action $a \in \sA$ from the action space $\sA$, executes the chosen action, and observes the resulting outcome from the environment. Among its actions, the main agent may interact with a blackboard $\beta$, a shared communication medium where it can post a request $r$ without addressing a specific sub-agent. The helper agents $\Pi_{\text{helper}}$ continuously monitor the blackboard, determine whether they can address a posted request, and, if so, provide their outputs on the corresponding response board $\beta_{r}$. These responses are then collected and made available to $\pi_{m}$, which incorporates them into its decision-making process.\footnote{Responses are not written back to the blackboard $\beta$ to avoid dependencies where one sub-agent's output could influence the behavior of others negatively. Instead, all responses are directed exclusively to the response board $\beta_{r}$, ensuring independent operation of sub-agents and exclusive access by the main agent $\pi_{m}$.} The main agent is limited to at most $T$ sequential actions (including actions that interact with the blackboard) to solve the query $q$, ultimately producing a program $p$ in python programming language that computes the final answer to $q$.

\paragraph{Clustering Data Lake:} 

There are multiple approaches for partitioning the data lake into clusters; applying clustering algorithms over file representations, random partitioning, or other heuristic methods. For simplicity, we do not utilize file content and instead rely solely on file names during clustering. Specifically, the file names are provided to an LLM---Gemini-2.5-Pro\footnote{Available at: \url{https://cloud.google.com/vertex-ai/generative-ai/docs/models/gemini/2-5-pro}}---which using the prompt shown in Figure~\ref{fig:clustering-prompt}, clusters the files into categories based only on their names.\footnote{This method represents just one simple possible approach to clustering, chosen for simplicity; more scalable and accurate alternatives could equally be employed in real world scenarios.} An example of this clustering is provided in Figure~\ref{fig:clustering-example} in Appendix~\ref{app:case-study}, where the model successfully groups related files together. For instance, it clusters all files originating from the National Interagency Fire Center into a category labeled ``NIFC Wildfire Statistics.'' The number of automatically derived clusters for each dataset is reported in Table~\ref{tab:stats} in Appendix~\ref{app:dataset}.



% The remainder of this section details the design of the main agent and the helper agents, emphasizing how their coordination supports effective information discovery in data science tasks.

\subsection{Main Agent}
\label{sec:main-agent}

The primary role of the main agent is to solve the problem in collaboration with the helper agents. The main agent follows the ReAct framework \citep{yao2023react}, where at each step $t$, given the query $q$ and the history of actions and observations $\sH_{t-1}$, it first reasons about what is the best next action and selects an action from a predefined action space, executes the action, observes the outcome, and appends the resulting observation to update the history $\sH_{t}$.\footnote{In this work, the inputs, outputs of the model, and observations are appended directly to the prompt of the LLM, formatted according to its chat-based input template.} The prompt used by the main agent is shown in Figure~\ref{fig:main-agent-blackboard-prompt} in Appendix~\ref{app:prompts}. The agent selects one of the following predefined actions in each step, executes them, and observe their outcomes:

\begin{itemize}[leftmargin=*]
    \item \textit{\textbf{Planning:}} In this action, the LLM decomposes the problem into smaller sub-problems and outlines a plan for addressing each of them. This action has no external effect on the environment but serves as an internal reasoning step to guide the LLM's problem-solving process. In response, the system simply acknowledges the proposed plan and instructs the LLM to proceed.
    
    \item \textit{\textbf{Reasoning:}} In this action, the LLM focuses on a specific aspect of the problem and explains its reasoning, analysis, or interpretation of the available observations and steps taken so far in this process. Similar to the planning step, this action has no external effect on the environment but functions as an internal reasoning mechanism to guide the LLM's problem-solving process. In response, the system simply acknowledges the reasoning and prompts the LLM to continue.
    
    \item \textit{\textbf{Executing Code:}} In this action, the agent generates python code, which is executed using a python interpreter. If the code runs successfully, the resulting outputs are returned to the agent for observation; otherwise, the agent receives the corresponding error messages. This action enables the agent to explore the problem interactively, inspect data files, and experiment with them to gain a deeper understanding of their content and structure and how to process them.
    
    \item \textit{\textbf{Requesting Help:}} In this action, the agent formulates a request for assistance from the sub-agents, specifying, for example, the types of data files or information needed, or the resources required to apply a tool or solve a sub-problem. This request is posted on the blackboard $\beta$ for visibility by the helper agents. Once the sub-agents respond, if they respond, their responses on the response board $\beta_r$ are collected and provided back to the main agent as the outcome of this action for observation and further use in its decision-making process.
    
    \item \textit{\textbf{Answering:}} In this action, the agent concludes the problem-solving process by generating a final program that produces the answer to the query. This action terminates the process, and the output of this step constitutes the final program $p$ generated by the system to address the problem.
\end{itemize}

\subsection{Helper Agents}
\label{sec:sub-agents}

In a data science, information discovery can typically be categorized into two tasks: (1) identifying the specific files that contain the data necessary to the problem, and (2) retrieving general knowledge about concepts relevant to the problem, such as domain-specific terms or details of particular algorithms and methods. To support these, our framework employs two types of helper agents:

\paragraph{File Agent:} 

Handling all the files in a data lake with a single agent is not feasible for several reasons: it typically involve a large number of files, many of which are lengthy and may exceed the agents context window; the files span diverse topics, which can confuse the agent and hinder effective reasoning; and accessing and processing all files simultaneously can be computationally expensive and inefficient, leading to unnecessary overhead and slower problem-solving. For these reasons, in our framework each file agent is assigned responsibility for a subset of data files determined to be relevant, as described earlier in the clustering procedure. In an offline phase, the file agent $\pi_{f_i}$ takes as input a subset of the data lake $\sD_{i}$ and operates through a two-step procedure. In the first step, the agent selects a subset\footnote{When filenames indicate multiple files containing the same type of data over different time periods, the agent does not need to inspect all of them to infer the structure; a small representative sample is sufficient.} (or all) of the files to examine their content. The contents of them are presented to the agent for inspection (details of presentation are in Appendix~\ref{app:implementation}). In the second step, after observing the selected files, the agent reasons about and analyzes them, learning how they are structured, what pre-processing or transformations may be required, and how they should be processed in general. An example of such an analysis is provided in Figure~\ref{fig:file-agent-analyze-example} in Appendix~\ref{app:case-study}. Then, in the online phase, the agent listens for requests from the main agent. Upon receiving a request, based on the analysis it did earlier, it determines whether it can contribute to answering it. If so, the agent generates a detailed plan specifying which files in $\sD_i$ are relevant, how they should be loaded in Python code, what libraries to use, the steps required for data processing, and samples from the data. The prompt used to guide the file agent is shown in Figure~\ref{fig:file-agent-prompt} in Appendix~\ref{app:prompts}. 

\paragraph{Search Agent:}

Certain data science problems require task-specific knowledge about algorithms or domain expertise that the LLM may not possess. To address this, we design a web-search agent that retrieves relevant information from a search engine. This agent operates according to the prompt shown in Figure~\ref{fig:search-agent-prompt} in Appendix~\ref{app:prompts}. Given a request $r$ posted on the blackboard $\beta$, the agent first determines whether it is capable of addressing the request. It is specifically restricted to general web-based information retrieval and does not respond to requests involving access to local files or datasets. If the agent determines that the request can be answered, it enters an iterative search process with a maximum of $T_{\text{search}} = 3$ steps. At each step $t$, the agent generates a set of queries $\sQ_{t}$, which are submitted to a search engine---in this work, Google Custom Search Engine\footnote{We use Google Custom Search Engine, configured to exclude all websites associated with the datasets used in this paper to prevent data leakage: \url{https://developers.google.com/custom-search}}---to retrieve $k=3$ webpage per query. The content of the webpages are then extracted using \textit{beautifulsoup} library\footnote{Available at: \url{https://pypi.org/project/beautifulsoup4/}} to be presented to the search agent. The extracted documents are then evaluated by the agent to determine whether they provide sufficient information to answer the request. If so, the agent generates a response to the request, which is posted to the response board $\beta_r$. If the information is insufficient, a new set of queries is generated to continue gathering relevant data from the web.



\section{Results}
\label{sec:results}


\subsection{Methodology}
\label{sec:methodology}

\textbf{Video Diffusion Model.} We evaluate \X using the following open source, widely available video models to generate the videos:
\begin{itemize}
    \item Wan 2.1~\cite{wan} 1.3B, 14B, 480p and 720p models, at 81 frames.
    \item HunyuanVideo 720p~\cite{hunyuanvideo} at 720p. 81 frames.
\end{itemize}

All experiments are conducted using bfloat16 precision. We implement CUDA kernels for \X with the aid of device primitives from ThunderKittens~\cite{thunderkittens} for a H100 GPU. To evaluate the quality of the videos generated, we use the VBench~\cite{vbench} VLM benchmarking scores, alongside visual comparisons of frames from the generated videos. We test two configurations of \X: one using the caching strategy to determine the mask (\X-cached), and the other using the pooling strategy (\X-pooling). For the \X-cached strategy, the threshold is set to $0.5/N$, where $N$ is the number of embedding vectors in the latent space representation of the video. The attention mask is cached once every $15$ DiT iterations. We compare \X with two prior works that use block sparse attention to leverage sparsity in attention scores in DiTs: Radial Attention~\cite{radialattn} and SparseVideoGen~\cite{sparsevideogen}. SparseVideoGen~\cite{sparsevideogen} uses a local-global attention computation strategy (windowed attention) across spatio templaral tokens. Radial attention uses a static attention mask that leads to an exponentially decaying compute density along the antidiagonal of the attention map.

\subsection{End-to-end Speedup}
\label{sec:e2espeedup}

Fig.~\ref{fig:e2e_normalized} shows the end-to-end time required to generate the video, normalized to baseline. We observe that \X is able to achieve an average speedup of $1.48\times$ and up to $1.65\times$. 
\X achieves a speedup as a result of accelerating the attention computation time during training. Fig.~\ref{fig:attn_normalized} shows the average runtime needed to compute the attention of every layer, normalized to the PyTorch implementation baseline. For the attention computation, \X achieves a speedup of $1.93\times$ on average, up to $2.38\times$. \X achieves a higher speedup when generating videos at 720p.
Our approach achieves a higher speedup of $1.2\times$ compared to SparseVideoGen~\cite{sparsevideogen} and $1.22\times$ compared to RadialAttention~\cite{radialattn}. The observed speedup comes from skipping a larger fraction of attention scores. However, this advantage diminishes at higher video resolutions (720p compared to 480p). This is because, in self-attention, interactions between blocks of embeddings that correspond to distant regions of the video are typically zero. As the resolution increases, each embedding vector covers a smaller region of the input, leading to a greater number of embeddings. This increases the proportion of zero-valued attention scores, which block-sparse attention can skip. Consequently, while more scores are skipped, the relative speedup achieved by \X decreases.



\begin{figure}[!htb]
    \centering
    \includegraphics[trim=0 90 0 80, clip, width=\linewidth]{figs2/e2enorm_speedup.pdf}
    \caption{Normalized end-to-end speedup in seconds for video generation.}
    \label{fig:e2e_normalized}
\end{figure}

\begin{figure}[!htb]
    \includegraphics[trim=0 90 0 90, clip, width=\linewidth]{figs2/attnnorm_speedup.pdf}
    \caption{Normalized attention computation speedup compared to baseline.}
    \label{fig:attn_normalized}
\end{figure}

  

% \begin{figure}[!htb]
%     \centering
%     \begin{subfigure}{0.5\textwidth}
%         \includegraphics[width=\linewidth]{figs2/e2enorm_speedup.pdf}
%         \caption{Normalized end-to-end speedup in seconds for video generation.}
%         \label{fig:e2e_normalized}
%     \end{subfigure}
%     \hfill
%     \begin{subfigure}{0.5\textwidth}
%         \includegraphics[width=\linewidth]{figs2/attnnorm_speedup.pdf}
%         \caption{Normalized attention computation speedup compared to baseline.}
%         \label{fig:attn_normalized}
%     \end{subfigure}
%     \caption{Normalized performance comparison for video generation.}
%     \label{fig:normalized_comparison}
% \end{figure}


\subsection{Qualitative Analysis}
\label{sec:qualitative_analysis}


Table~\ref{tab:vbench} shows the VBench~\cite{vbench} video benchmarking results when compared to the baseline. We observe that \X achieves negligible degradation in quality when compared to the baseline.

% Please add the following required packages to your document preamble:
% \usepackage[table,xcdraw]{xcolor}
% Beamer presentation requires \usepackage{colortbl} instead of \usepackage[table,xcdraw]{xcolor}
\begin{table*}
\centering
\caption{VBench quality metrics}
\label{tab:vbench}
\begin{tabular}{|l|r|r|r|r|}
\hline
                          & \multicolumn{1}{l|}{\textit{\textbf{\begin{tabular}[c]{@{}l@{}}Aesthetic\\ Quality\end{tabular}}}} & \multicolumn{1}{l|}{\textit{\textbf{\begin{tabular}[c]{@{}l@{}}Subject\\ Consistency\end{tabular}}}} & \multicolumn{1}{l|}{\textit{\textbf{\begin{tabular}[c]{@{}l@{}}Background\\ Consistency\end{tabular}}}} & \multicolumn{1}{l|}{\textit{\textbf{\begin{tabular}[c]{@{}l@{}}Overall\\ Consistency\end{tabular}}}} \\ \hline
Wan-1.3B 480p baseline    & 0.601                                                                                              & 0.936                                                                                                & 0.958                                                                                                   & 0.23                                                                                                 \\
Wan-1.3B 480p FGAttn      & 0.605                                                                                              & 0.939                                                                                                & 0.96                                                                                                    & 0.23                                                                                                 \\ \hline
Wan-1.3B 720p Baseline    & 0.61                                                                                               & 0.944                                                                           & 0.962                                                                                                   & 0.233                                                                                                \\
Wan-1.3B 720p FGAttn      & 0.61                                                                                               & 0.944                                                                                                & 0.964                                                                                                   & 0.232                                                                                                \\ \hline
Wan-14B 480p baseline     & 0.623                                                                                              & 0.953                                                                                                & 0.97                                                                                                    & 0.25                                                                                                 \\
Wan-14B 480p FGAttn       & 0.616                                                                                              & 0.952                                                                                                & 0.975                                                                                                   & 0.247                                                                                                \\ \hline
Wan-14B 720p baseline     & 0.621                                                                                              & 0.945                                                                                                & 0.969                                                                                                   & 0.248                                                                                                \\
Wan-14B 720p FGAttn       & {\color[HTML]{000000} 0.619}                                                                       & 0.942                                                                                                & {\color[HTML]{000000} 0.961}                                                                            & 0.245                                                                                                \\ \hline
Hunyuan-13B 720p baseline & 0.62                                                                                               & 0.944                                                                                                & 0.962                                                                                                   & 0.239                                                                                                \\
Hunyuan-13B 720p FGAttn   & 0.62                                                                                               & 0.94                                                                                                 & 0.962                                                                                                   & 0.239                                                                                                \\ \hline
\end{tabular}
\end{table*}

Figs.~\ref{fig:hunyuanvideo}, \ref{fig:wan1_3b} and \ref{fig:wan14b} show the visual representation of the produced video compared to the original (the top row of each set of videos represents the baseline video) for the HunyuanVideo model, Wan 1.3B model, and the Wan 14B model, respectively. We find that across all the prompts tested here, \X can recover the original video with no quality degradation. \X also retains the generated video style and does not significantly shift the distribution captured by the underlying model. 





\subsection{Ablation Study}
\label{sec:ablation}

Fig.~\ref{fig:ablation} depicts the average attention computation time for video generation as the threshold parameter is varied. We sweep the threshold parameter from $0.1/N$ to $1/N$, where $N$ is the number of embedding vectors in the latent space representation of the video. A higher threshold enables skipping a larger amount of computation, thereby leading to a speedup.  

\begin{figure}[!htb]
    \centering
    \includegraphics[width=\linewidth]{figs2/ablation.pdf}
    \caption{Normalized video generation time at different thresholds applied to \X-cached.}
    \label{fig:ablation}
\end{figure}



\begin{figure*}[!htb]
    \centering
    \includegraphics[width=\linewidth]{figs2/hunyuan.pdf}
    \caption{Samples of videos generated using baseline HunyuanVideo model, and \X-HunyuanVideo (The baseline generates first row, second row generated using \X)}
    \label{fig:hunyuanvideo}
\end{figure*}




\begin{figure*}[!htb]
    \centering
    \includegraphics[width=\linewidth]{figs2/wan1_3b.pdf}
    \caption{Samples of videos generated using baseline Wan-1.3B model, and \X-Wan1.3B. (First row is generated by the baseline, second row is generated using \X)}
    \label{fig:wan1_3b}
\end{figure*}





\begin{figure*}[!htb]
    \centering
    \includegraphics[width=\linewidth]{figs2/wan14b.pdf}
    \caption{Samples of videos generated using baseline Wan-14B model, and \X-Wan14B (First row is generated by the baseline, second row is generated by \X)}
    \label{fig:wan14b}
\end{figure*}




\section{Related Work}
\label{sec:related_work}


\textbf{Block sparse attention.} 
Several implementations of block sparse attention~\cite{bsa, flashattn, flexattn, flashinfer, flashmask} propose a coarse-grained sparse attention mechanism that skips entire blocks of attention score computations at granularity of $64\times 64$ or $128\times 128$ at half-precision. Current block-sparse attention mechanisms either prevent further reduction of block size (do not compile) or cause significant hardware underutilization and performance overhead, since they are constrained by the tensor core matrix multiplication width (\cref{sec:motivation_skip_attn_fine_granularity}). Several works in the large language model literature~\cite{minference, xattn, flashdecode, nsa, seerattn} utilize block sparse attention to accelerate attention computation. 
% Several works in large language model literature ~\cite{minference, xattn, flashdecode, nsa, seerattn} use block sparse attention to speedup attention computation.

\textbf{Block sparse attention for videoDiTs.} 
Recent works, such as Radial Attention~\cite{radialattn}, X-attention~\cite{xattn}, SparseVideoGen~\cite{sparsevideogen}, and SparseVideoGen2~\cite{sparsevideogen2}, have applied block sparse attention implementations to video diffusion models. These approaches consider a fixed sparsity pattern in the attention map based on empirical observations of significant patterns. Other works, such as Video Sparse Attention~\cite{vsa}, incorporate learned sparse attention patterns by using a parameterized model to derive the attention map mask. Both approaches utilize coarse-grained sparse attention mechanisms. In contrast, our method enables fine-grained skipping of attention blocks, providing more opportunities for skipping computation. We compare \X with SparseVideoGen and Radial Attention in~\cref{sec:results}. 
Moreover, trainable sparse attention methods such as Video Sparse Attention (VSA)~\cite{vsa} can be reformulated to generate sparse masks compatible with \X \textquotesingle s attention kernel. These methods are orthogonal to \X \textquotesingle s kernel implementation and can be used in conjunction as mask-determination strategies for \X.

% While trainable sparse attention methods can outperform training-free approaches, these methods could benefit from \X's fine-grained approach.
% Recent works such as Radial attention~\cite{radialattn}, X-attention~\cite{xattn}, SparseVideoGen~\cite{sparsevideogen}, SparseVideoGen2~\cite{sparsevideogen2} apply block sparse attention implementations for video diffusion models. They consider a fixed sparsity pattern in the attention map based on empirical observation of significant. Other works such as video sparse attention~\cite{vsa} incorporate learnt sparse attention patterns. These works use a parameterized model to derive the mask of the attention maps. Both of these works make use of coarse grain sparse attention mechanisms. Our approach allows fine-grain skipping of attention blocks enabling more opportunity to skip computation. We compare \X with SVG, radial attention in~\cref{sec:results}. Trainable sparse attention methods can make  of coarse grain sparse attention methods, and could potentially benefit from \X. 

\textbf{Other techniques to accelerate video diffusion.}
SpargeAttention~\cite{spargeattn}, SageAttention~\cite{sageattn}, and SageAttention2~\cite{sageattn2} propose general attention approximation techniques, such as quantization and token compression mechanisms, that can be applied during inference for both LLM and DiT models. Token compression-based approaches may skip essential tokens relevant to the video, which could lead to inconsistent video generation (pointed out by~\cite{radialattn}). These approaches are orthogonal to our \X.% SpargeAttention~\cite{spargeattn}, SageAttention~\cite{sageattn}, SageAttention2~\cite{sageattn2}, propose general attention approximation techniques such as quantization and token compression mechanisms that can be applied at inference time to LLM and DiT models. Approaches based on token compression skip essential tokens relevant to the video and may produce inconsistent video (\tofix{add reference here}). Quantization-based techniques are orthogonal to our approach (a lower-precision version of \X can be implemented). 






\section{Discussion}
\label{sec:discuss}


\begin{comment}
    

In this section, we discuss relevant aspects of the our approach.

\textbf{Can LLMs Interpret Ethical Information in Textual Prompts? \todo{OR} Ethical reasoning capabilities of LLMs (since this is similar with RQ1.)}

\todo{The answer will be a summary of the results, not sure. Patrizio, please check this.}


\textbf{Toward Model-Based Ethical Evaluation}

This is also 


\textbf{Constructing Ethical Profiles through Passive Interaction}

The end goal of our approach is to automatically generate user ethical profiles, leveraging the ethical reasoning abilities of LLMs to analyze user behavior in real time. If LLMs can consistently interpret ethically relevant scenarios, then they can serve as inference engines for future systems that monitor user interactions (e.g., via AR glasses or smart agents) and extract preference signals from context.

These profiles need not be static or deterministic. Rather, they can evolve as a form of passive ethical memory that captures the user’s stance across dimensions such as fairness, harm, or loyalty. This enables the design of agents that reason within user-defined moral constraints, without requiring explicit rule engineering or ongoing manual feedback.


\textbf{Implications for Software Engineering Practice}

From a software engineering perspective, our approach serves as a practical and scalable method to incorporate an ethical reasoning mechanism into the development of software systems. By using LLMs as modular evaluators, ethical reasoning mechanisms can be embedded as an operational layer in software systems, enabling them to make decisions that align with user preferences.


*******************
This framework aligns with software engineering research in two directions. First, it provides a method for rapidly probing the ethical capabilities of generative systems that may later be integrated into user-facing software agents. Second, it sets the foundation for ethically-informed profiling methods that can be automated, auditable, and adaptable to different application domains, including healthcare assistants, AR interfaces, and autonomous agents. By decoupling the evaluation from manual supervision, we advance toward scalable, trustworthy AI modules that operate within well-defined moral boundaries.

\end{comment}


%\subsection{RQ1: Do LLMs demonstrate the capacity for ethical reasoning when presented with ethically charged scenarios?}

\textbf{RQ1.} Our findings provide evidence that state-of-the-art LLMs can engage in ethical reasoning when presented with complex, real-world made \major{explanations of acceptability}. Without fine-tuning or examples, models consistently identified the most applicable ethical theory and made acceptability explanations with substantial inter-model agreement. This capability suggests that LLMs possess an implicit grasp of moral reasoning principles, grounded in their pretraining on large-scale textual corpora. From an SE perspective, this opens the door to using LLMs as ethical reasoning modules in decision-making pipelines, such as requirement negotiation, user modeling, or system auditing.

%\subsection{RQ2: Do LLMs Reason Consistently Across Models and Scenarios?}

\smallskip

\textbf{RQ2.} Quantitative results showed that models converge more strongly on binary moral acceptability (86.7\% BAR) than on ethical theory classification (73.3\% TCR). While this difference reflects the higher abstraction level of theoretical judgments, the level of agreement observed is non-trivial. The scenario-dependent variability in TCR reveals an important feature: model disagreement tends to reflect ethical ambiguity inherent in the scenario rather than arbitrary noise. This suggests that ensemble disagreement can be used as a proxy for moral uncertainty, enabling software systems to trigger escalation or human intervention when LLMs disagree sharply.

%\subsection{RQ3: How Do LLMs Compare to Human Experts?}

\smallskip

\textbf{RQ3.} %Our comparison revealed substantial alignment between LLMs and human experts. 
\major{Overall, LLMs exhibit non-trivial agreement with experts that is more pronounced in prevalent classes and weaker on rare or edge cases.} Scenarios that elicited strong agreement among experts tended to also show high inter-model LLM agreement, and vice versa. This convergence reinforces the reliability of LLMs in interpreting familiar or structurally simple moral scenarios. Divergences, especially in edge cases, underscore the importance of hybrid systems that combine automated reasoning with human oversight. For SE applications involving legal, regulatory, or safety-critical implications, LLM-based profiling should not be deployed as an isolated decision-maker but as a complementary module.

%\subsection{RQ4: What Characterizes the Structure of LLM Moral Expanations?}

\smallskip

\textbf{RQ4.} Qualitative analyses demonstrated that LLMs generate explanations that are lexically diverse but conceptually coherent. Despite low textual similarity across models, explanations consistently aligned with the chosen moral theory in over 90\% of cases. Models blended terminology from multiple ethical traditions in natural, context-sensitive ways, %mirroring 
\major{tending to reflect} how human reasoners combine principles, consequences, and character-based considerations. This expressive flexibility is critical for ethical profiling, as it enables the detection of user-aligned reasoning patterns across different moral framings. Moreover, the compactness of most explanations (single sentences) and their theoretical consistency suggest that LLMs are capable of producing tractable, auditable moral outputs suitable for runtime interpretation and logging.
%required by metareview
\begin{tcolorbox}[colback=gray!10,
                  colframe=black,
                  arc=4mm,
                  boxrule=0.8pt,
                  left=2mm, right=2mm, top=1mm, bottom=1mm]
                  \major{\textbf{Agreement $\ne$ correctness.} Our design surfaces stability and ambiguity signals, it does not certify normative accuracy. In SE practice, high agreement supports automation with audit, while low agreement recommends human-in-the-loop escalation.}
\end{tcolorbox}                  

\smallskip


\textbf{Limitations and Scope of Validity.} While promising, our findings are bounded by some limitations:

\smallskip

\noindent\textit{Theoretical coverage.} We focus on three major ethical theories utilitarianism, deontology, and virtue ethics due to their widespread adoption in software engineering practice and education~\cite{vaniea2018securitytrolley}. These ethical theories provide well-established foundations for analyzing ethical dilemmas in technology contexts. While alternative theories such as care ethics or contractualism are less commonly applied, they offer valuable perspectives that could enrich ethical analyses. Future work may explore the integration of these additional frameworks to capture a broader spectrum of moral reasoning in software engineering.

\smallskip

\noindent\textit{Scenario framing.} Our prompts use concise, decontextualized scenarios. Richer formats (e.g., dialogues, system logs) may affect model interpretation. Our current prompts are decontextualized statements. An important next step is to apply the same ethical reasoning pipeline (Figure~\ref{fig:approach2}) to richer input modalities, including:
(i) chat transcripts from developer-agent interactions;
(ii) logs of user decisions in ethically sensitive configurations;
(iii) behavioral signals from simulation environments or system telemetry.
This would move the profiling process closer to real-time, context-aware ethical inference.

\smallskip

\noindent\textit{Zero-shot constraints.} All reasoning is performed without memory or clarification. Interactive or multi-turn reasoning may yield different profiles. An ethical profile need not be static. As users interact with a system, their decisions may reveal shifts in priorities, trade-offs, or ethical boundaries. Future work should implement an ethical memory module that incrementally updates a user's profile over time, capturing both stable dispositions and contextual shifts. This requires designing a temporal profiling architecture that tracks ethical indicators across scenarios and resolutions.

\smallskip
   
\noindent\textit{Agreement $\ne$ correctness.} Convergence does not imply normative accuracy. Human biases and model alignment may coincide but remain ethically questionable. In real deployments, users may reject or revise the moral judgments made by the system. Building on our current architecture, we envision an interactive loop in which: (i) the system proposes an ethical explanation; (ii) the user confirms, modifies, or rejects the reasoning; (iii) the profile is updated accordingly. This would enable both user agency and model refinement over time, reducing the risk of misaligned ethical personalization. 

\smallskip

\major{\noindent\textit{Prompt sensitivity.} Our zero-shot, single-turn protocol deliberately controls for instruction complexity; however, model behavior can still be sensitive to seemingly innocuous variations in prompt phrasing, formatting, or input length. We therefore treat prompt sensitivity as a threat to validity and an explicit boundary of our claims. A systematic sensitivity analysis is left as future work. In practice, we recommend freezing prompt templates in repositories and reporting all formatting details that might affect reproducibility.}

\smallskip

Even if the findings support the potential viability, these limitations suggest caution in direct deployment and highlight the need for further validation before integrating LLM-based profiling into high-stakes SE systems. Our results position LLMs as viable components for modular ethical reasoning in SE. Possible use cases include: \textit{decision auditing} for moral rationales generation for SE tool outputs (e.g., in requirements prioritization or resource allocation); \textit{autonomy triage} to route decisions to humans when LLMs disagree, reducing risk in ethically charged contexts; \textit{agent personalization} to tailor behavior of autonomous SE agents based on learned ethical user profiles. More broadly, the ability to extract consistent moral structure from language enables a shift from static ethics-as-checklists to adaptive, traceable, and user-aligned ethical cognition in engineered systems.

\section{Conclusion}

In this work, we presented a full-stack investigation of LLM unlearning, encompassing methodology, evaluation, and robustness. We established a principled taxonomy that organizes twelve representative unlearning methods into three families: {\MDiv}, {\MRep}, and {\MRej}, providing a systematic lens to understand their underlying mechanisms. Our analysis revealed that conventional multiple-choice questioning (MCQ) evaluations of unlearning effectiveness (UE) and utility retention (UT) offer an incomplete picture, and we introduced open question answering (Open-QA) as a complementary paradigm to better capture generative behaviors and expose the strengths and limitations of different methods. Furthermore, we provide a comprehensive robustness assessment across model-level and input-level attacks, revealing nuanced relationships among in-domain relearning, out-of-domain fine-tuning, quantization, and jailbreak attacks. These findings clarify the trade-offs of current unlearning algorithms and guide the design of future methods that are both effective and robust. The use of LLM, limitation and broader impact are further discussed in \textbf{Appendix\,\ref{appx:llm_usage}}, \textbf{Appendix\,\ref{appx:limit}} and \textbf{Appendix\,\ref{appx:impact}}.



\bibliography{refs}
\bibliographystyle{mlsys2025}

\appendix
% \section{Implementation Details - load indices}



\end{document}


% This document was modified from the file originally made available by
% Pat Langley and Andrea Danyluk for ICML-2K. This version was created
% by Iain Murray in 2018. It was modified from a version from Dan Roy in
% 2017, which was based on a version from Lise Getoor and Tobias
% Scheffer, which was slightly modified from the 2010 version by
% Thorsten Joachims & Johannes Fuernkranz, slightly modified from the
% 2009 version by Kiri Wagstaff and Sam Roweis's 2008 version, which is
% slightly modified from Prasad Tadepalli's 2007 version which is a
% lightly changed version of the previous year's version by Andrew
% Moore, which was in turn edited from those of Kristian Kersting and
% Codrina Lauth. Alex Smola contributed to the algorithmic style files.
