\section{Related Work} \label{sec:related}

\textbf{Network traffic analysis of cloud gaming services:}
As an emerging type of multimedia streaming service that exhibits high demands on network conditions, cloud gaming services have been analyzed in prior works for their network traffic characteristics, including flow anatomy and volumetric profiles. For example, Lyu \textit{et al.} \cite{lyu_network_2024} discussed the use of network flows during entire user sessions, from platform administration to server selection and gameplay, which serve as indicative signatures for network operators to detect cloud gaming sessions played on different types of user setups by their broadband/mobile subscribers. The works in \cite{lyu_do_2024,carrascosa_cloud_2022,domenico_network_2021} analyzed the bandwidth demands of cloud gaming sessions with different levels of streaming quality settings.
Furthermore, objective quality-of-experience metrics (\eg user input lag, graphic resolution, and streaming frame rates) have also been derived from network quality-of-service (QoS) attributes of game streaming flows \cite{monaco_real-time_2025,lyu_network_2024,baena_measuring_2023}.
For the network observability industry, it has been highlighted that purely using objective QoE metrics such as frame rate cannot effectively measure the quality of game streaming perceived by players, as gameplay contexts, including game titles and player activities, can significantly vary the expected/requested QoE and network QoS levels for good user experience \cite{an_tooth_2025,carvalho_qoe_2024,he_vigor_2024,wu_zgaming_2023,saha_study_2023,iqbal_dissecting_2021,lindstrom_cloud_2020}.
To bridge this gap, we develop a traffic analysis method to classify the contexts of cloud gameplay, enabling network operators to effectively measure user-perceived streaming quality when correlating with the objective QoE metrics.


\textbf{Measuring multimedia services with contexts:}
Gaining visibility into the coarse categories of multimedia streaming contexts has become important for the network observability industry to effectively measure user-perceived streaming quality as delivered by networks \cite{wang_standardizing_2024,varvello_performance_2022,bronzino_inferring_2019,barakabitze_qoe_2018,skorin-kapov_survey_2018,finamore_youtube_2011,mok_inferring_2011}. Prior works have developed methods to classify contexts in popular types of multimedia services such as video streaming \cite{wang_characterizing_2024,zhang_sensei_2021,adarsh_too_2021,bronzino_traffic_2021}, live streaming \cite{madanapalli_reclive_2021,loh_machine_2021}, video conferencing \cite{sharma_estimating_2023,michel_enabling_2022,macmillan_measuring_2021}, voice calling \cite{mauro_experimental_2020,holub_analysis_2018}, online gaming \cite{madanapalli_know_2022,schmidt_modeling_2021}, and virtual reality applications \cite{lyu_metavradar_2023,cheng_are_2022}.
For example, the works in \cite{che_packet_2012,chen_network_2006} focus on the contexts in online gaming that vary the expected network conditions for a good gameplay experience including game titles, genres, the number of players, peripheral status and team communication model.
In \cite{lyu_metavradar_2023,cheng_are_2022}, the authors measured the network demands for a good VR experience that depend on the user activity status and the number of surrounding users.
This work is specialized in classifying the contexts (\ie game titles, gameplay activity patterns, and player activity stages) of cloud game streaming sessions. As demonstrated in a large-scale deployment at an ISP hosting NVIDIA's cloud gaming servers, our method, combined with objective QoE measurement, holds unique value for network operators to better support this highly demanding service by effectively measuring the user-perceived streaming quality.
