\section{Traffic Characterization across Cloud Game Contexts}\label{sec:dissect}
In this section, using our ground-truth traffic traces of cloud gaming sessions labeled by game titles and in-game player activity stages (\S\ref{sec:dataset}), we discuss our findings into the unique group characteristics of packets during game launch across titles (\S\ref{sec:dissect_game_titles}) and the volumetric characteristics of streaming flows in the upstream and downstream directions across player activity stages (\S\ref{sec:dissect_engagement_status}).

\subsection{Dataset Collection and Overview}\label{sec:dataset}
To the best of our knowledge, existing public traffic trace datasets (\eg \cite{slivar_cgd_2022,slivar_game_2018,marchal_analysis_2023,carrascosa_cloud_2022}) on cloud gaming do not contain sufficient labels of the game contexts articulated in \S\ref{sec:background}, and/or not have comprehensive coverage of game streaming configurations. Therefore, to obtain a holistic understanding of the network traffic characteristics of cloud gameplay sessions with diversified configurations (\ie user devices, operating systems, software applications and streaming settings) across different types of cloud game contexts (\ie game titles, genres and player activity stages), we built a lab setup to collect ground-truth traffic capture (PCAP) files of cloud game streaming sessions, as visually shown in Fig.~\ref{fig:lab_setup}.

The setup consists of a diverse collection of user devices that support cloud game streaming services, including mobile phones, PCs, a gaming console and a smart TV. The devices are connected to our regional NVIDIA GeForce NOW cloud servers hosted by our partnered Internet service provider with network conditions of nearly 1 Gbps bandwidth, less than 10 ms latency and less than 0.1\% packet loss rate. They can also access cloud gaming services hosted by other platforms such as Microsoft's Xbox Cloud Gaming with nearly ideal network conditions. The Linux-based access gateway connecting user devices is configured with packet capturing capability through Wireshark and TCPdump software.


With a team comprising two of the authors and four student volunteers, we have collected labeled PCAP files for 531 cloud game streaming sessions with 67 hours of total playtime across the 13 popular game titles listed in Table~\ref{tab:game_selection}. 
The number of game streaming sessions and their total playtime of the collected traffic traces labeled by their device genres, operating systems, software types, and streaming settings are presented in Table~\ref{tab:dataset}.
To enable us to analyze traffic characteristics associated not only with per-session game titles but also with in-game player activity stages, each PCAP file that contains packet streams of an entire gameplay session is also labeled by the changing player activity stages (\ie idle, active, or passive) with timestamps.

To validate the generalizability of our insights to other commercial cloud gaming platforms, we have also collected labeled PCAP files for a representative range of gameplay sessions on three other major platforms including Xbox Cloud Gaming, Amazon Luna, and PS5 Cloud Streaming, in addition to the PCAPs for GeForce NOW. The dataset is about 800 GB in size and is shared with the Internet research community through our university cloud drive with the availability information provided in Appendix \S\ref{sec:data_availability}.

\begin{figure*}[t!]
	\centering
	\subfigure[Overwatch using HD resolution.]{
		\includegraphics[width=0.48\textwidth]{figures/volumetric_overwatch_win_hd.pdf}
		\label{fig:volumetric_overwatch_win_hd}
	}
	\subfigure[Overwatch using UHD resolution.]{
		\includegraphics[width=0.48\textwidth]{figures/volumetric_overwatch_win_uhd.pdf}
		\label{fig:volumetric_overwatch_win_uhd}
	}
	\subfigure[CS:GO using UHD resolution.]{
		\includegraphics[width=0.48\textwidth]{figures/volumetric_csgo_win_uhd.pdf}
		\label{fig:volumetric_csgo_win_uhd}
	}
	\subfigure[Cyberpunk 2077 using UHD resolution.]{
		\includegraphics[width=0.48\textwidth]{figures/volumetric_cyberpunk_mac_uhd.pdf}
		\label{fig:volumetric_cyberpunk_mac_uhd}
	}
   \vspace{-3mm}
	\caption{Time-series line chart showing the downstream throughput and upstream packet rate of game streaming flows in four representative cloud gaming sessions with background color-coded by player activity stages as \colorlaunch{launch}, \coloridle{idle}, \colorpassive{passive} and \coloractive{active}.}
	\label{fig:session_volumetric}
\end{figure*}

\subsection{Packet Group Profiles across Game Titles} \label{sec:dissect_game_titles}

During cloud gameplay, cloud servers stream the rendered gaming video and receive user inputs in real-time through standard Real-time Transport Protocol (RTP) flows \cite{nvidia_how_2023,lyu_network_2024,domenico_network_2021}.
Specifically, gaming graphics and audio are sent from cloud servers to the client devices via downstream RTP packets of the respective streaming flows, whereas player inputs such as mouse clicks, key strokes, mobile screen touches, and voice are carried in upstream RTP packets from the client devices to cloud servers. Prior works \cite{lyu_network_2024,domenico_network_2021} have also developed signatures to detect RTP flows for cloud game streaming using flow metadata, which are used as preliminary filtering conditions on the collected traffic traces for us to focus on the game streaming flows and investigate their behavioral profiles across game contexts. 


As already illustrated in Fig.~\ref{fig:main}, before a player starts actively or passively engaging in gameplay, there is always a game \textbf{launch} stage lasting from tens of seconds to one minute, in the form of an opening animation streamed from the cloud server, to keep players waiting for the initialization of the game. In traditional gaming, such launch animations are installed and rendered on local gaming devices, while in cloud gaming, the animations that are made differently for each game title are live-streamed to players, leading to unique traffic characteristics across cloud game titles.

By analyzing our traffic traces labeled with their game titles, we observe unique packet group profiles in the downstream direction for their arrival times and payload sizes that carry launching scenes.
Fig.~\ref{fig:packet_genshin_win} visually shows the payload size and arrival time of the downstream packets in the RTP streaming flow for the first 60 seconds of a Genshin Impact gameplay session on a Windows PC with full HD graphics and 60fps streaming frame rate. From the scatter plot, we intuitively see that the packets fall into three groups, including \textbf{full} packets that have the same fixed (maximum) payload size and are constantly streamed to the local device; \textbf{steady} packets that hold similar payload sizes to their adjacent packets arriving within certain seconds; and \textbf{sparse} packets with high variations in payload sizes compared to their neighboring packets. 

The presence of the three packet groups holds similar profiles (\ie timing characteristics of each packet group and relative payload sizes across intervals of the same packet group) in all gameplay sessions of the same game titles, regardless of their client device types and streaming configurations. The packet statistics become unpredictable after the launching stage when players are involved in gameplay. As evidence, Fig.~\ref{fig:packet_genshin_android} and Fig.~\ref{fig:packet_genshin_win_hd} show the scatter plots for the same game title (\ie Genshin Impact) with different user settings. It is clear that the relative payload sizes and arrival time slots of the full, sparse, and steady packet groups in Fig.~\ref{fig:packet_genshin_android} remain almost identical compared to Fig.~\ref{fig:packet_genshin_win}. For Fig.~\ref{fig:packet_genshin_win_hd}, the arrival time slots of the three packet groups are identical compared to Fig.~\ref{fig:packet_genshin_win}, with tiny variations in relative payload sizes during three time slots (4.1 seconds in total) for the steady packet group.

Gameplay sessions of different game titles exhibit distinctive profiles for the three groups of packets. An example is provided in Fig.~\ref{fig:packet_fortnite_win} for a Fortnite gameplay session. The arrival density of full packets, the arrival time slots of sparse and steady packet groups, and their relative payload sizes are all different compared to game streaming sessions of the just-discussed Genshin Impact and other popular game titles in our dataset. 

\subsection{Flow Volumetric Profiles across Player Activity Stages} \label{sec:dissect_engagement_status}

After understanding the packet characteristics across game titles that indicate the overall QoE expectation of a game streaming session \cite{bhuyan_end_2022,sabet_delay_2020}, we now discuss the volumetric profiles of streaming flows that dynamically change on player activity stages in a cloud game. As a common practice in the gaming industry, graphics rendering and gaming computation are dynamically optimized at the software level for efficient resource consumption \cite{unity_reduce_rendering,xbox_graphics_effiency}. For cloud game streaming, such optimizations lead to reduced demands on QoE metrics (\eg streaming frame rate) and network resource consumption (\eg bandwidth) for delivering a good user experience in real time \cite{lyu_do_2024,marchal_analysis_2023}.

By analyzing the collected PCAP files of cloud gameplay with labeled player activities, we identify unique volumetric patterns in both upstream and downstream directions depending on the stages of player activity from idle, passive to active, with a visual example provided in Fig.~\ref{fig:volumetric_overwatch_win_hd}. The figure shows the upstream and downstream throughput of game streaming flows for an Overwatch gameplay session with background colors indicating the ground-truth player activity stages. 

\begin{figure}[t!]
	\mbox{
		\subfigure[Spectate-and-play games]{
			\includegraphics[width=0.22\textwidth]{figures/analyze_stage_transition_spectate.pdf}
			\label{fig:analyze_stage_transition_spectate}
		}
		\subfigure[Continuous-play games]{
			\includegraphics[width=0.22\textwidth]{figures/analyze_stage_transition_continuous.pdf}
			\label{fig:analyze_stage_transition_continuous}
		}
	}
   \vspace{-3mm}
	\caption{The average percentage of total playtime spent in the \coloridle{idle}, \colorpassive{passive}, and \coloractive{active} player activity stages with their transition probabilities per game streaming session for (a) spectate-and-play and (b) continuous-play cloud games in our labeled dataset.}
	\label{fig:analyze_stage_transition}
\end{figure}

For the \textbf{idle} player activity stage, we can see that the throughput in the downstream direction remains relatively low during the first idle stage when the player was ``browsing available matching options'' and drops to a low level in both directions when the player was ``back to the hub from the matching room'' at the end of this streaming session. 
In contrast, when the player activity stage becomes \textbf{active} for ``combating in progress'', the streaming throughput jumps to the highest level of the entire session, especially in the downstream direction.
During the \textbf{passive} stage when the player was passively watching the progress of gameplay such as ``being defeated by opponents in the match and watching the activity of teammates'', the downstream throughput remains at a similarly high level to the active stage but with reduced throughput in the upstream direction from the client to the cloud server. The observations in the relative (not absolute) changes of upstream and downstream volumetric profiles during a cloud gameplay session are consistent across all 13 popular game titles. Later in \S\ref{sec:detect}, we will leverage the bidirectional volumetric patterns as discussed to develop our method to reversely infer the player activity stages.

\begin{figure*}[t!]
    \includegraphics[width=\textwidth]{pipeline.pdf}
    \vspace{-3mm}
	\caption{Our real-time network traffic analysis methodology to classify gameplay contexts of cloud game streaming sessions for effective user experience measurement.}
        \label{fig:pipeline}
\end{figure*}

As discussed in \S\ref{sec:cg_process}, due to the designed gameplay mechanism of each game title, the 13 popular games can be categorized by their gameplay activity patterns as either ``continuous-play'' or ``spectate-and-play''. We observe that the streaming sessions of the \textbf{continuous-play} game type, \eg Cyberpunk 2077 in Fig.~\ref{fig:volumetric_cyberpunk_mac_uhd}, often have a larger fraction of playtime in the active player activity stage compared to that of \textbf{spectate-and-play} games like Overwatch in Fig.~\ref{fig:volumetric_overwatch_win_uhd} and CS:GO in Fig.~\ref{fig:volumetric_csgo_win_uhd}. Also, the \textbf{transitions among player activity stages} hold similarities for gaming sessions of the same type of gameplay activity pattern. The two state transition diagrams in Fig.~\ref{fig:analyze_stage_transition} show the average fraction of playtime spent in the active, passive, and idle player activity stages per gameplay session, with their transition probabilities computed from all ground-truth sessions. For game streaming sessions of the ``spectate-and-play'' games, the active stage often takes about 40\% to 60\% of the total playtime and the passive stage accounts for the majority of the rest of the playtime. As for the ``continuous-play'' games, more than 95\% playtime is spent in the active or idle player activity stages and less than 5\% playtime is spent in the passive stage. 


Our empirical findings in the differentiation of gameplay activity patterns show the ineffectiveness of existing cloud gaming user experience measurements relying solely on objective QoE metrics, which can mislabel ``spectate-and-play'' game streaming sessions as having poor user experience due to their low volumetric profiles during passive and idle stages. The findings also inspire us to leverage the transition behaviors of the player activity stages to classify the gameplay activity patterns.