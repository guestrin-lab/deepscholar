
\section{Preliminary of User Experience Contexts in Cloud Gaming}\label{sec:background}

In this section, we provide preliminaries on contextual factors of cloud gameplay that can affect user expectations on their perceived game streaming quality, starting from an intuitive understanding of typical player activity patterns during cloud gameplay sessions (\S\ref{sec:cg_process}), to game contexts (\ie titles/genres) that exhibit unique activity patterns (\S\ref{sec:cg_contexts}).


\subsection{Player Activity Stage and Gameplay Activity Pattern}\label{sec:cg_process}
Prior works \cite{lyu_network_2024,marchal_analysis_2023} have revealed that players who access commercial cloud gaming services first connect to the respective cloud platform for administrative services, game selection, and allocation of cloud game servers before actual gameplay. However, diversified user activities during cloud gameplay sessions that can lead to different user expectations on their perceived streaming quality, network traffic characteristics, and network demands to be satisfied by network operators have not yet been studied. Therefore, we now zoom into the gameplay session to understand the various types of user activities categorized into three levels of \textbf{\textit{player activity stages}}, namely \textbf{active}, \textbf{passive} and \textbf{idle}, which can lead to various expectations on the streaming quality.

As visually shown in Fig.~\ref{fig:main}, we use two representative gameplay sessions on NVIDIA's GeForce NOW platform to discuss the \textbf{\textit{gameplay activity patterns}} each with a different profile of player activity stages. Our discussed insights are generalizable to other commercial platforms.

The first example in Fig.~\ref{fig:fortnite_process} shows a CS:GO shooter game session where players' gameplay activities can be divided into multiple \textbf{spectate-and-play} slots. 
In this shooter gameplay, after the game has been selected on the cloud gaming platform, the player first waits for the game to launch with a transition scene and then stays in the lobby until a match starts. During these periods, the player maintains an almost idle level of gameplay activity. As will be discussed later in \S\ref{sec:dissect}, such idle activities also exhibit low-profile traffic characteristics.
Once the player starts a match, the gameplay activity becomes very active, including both high-frequency graphics refreshment and user input updates. During this shooter game match, the player sometimes gets eliminated by opponents and waits to respawn, becoming passive in terms of gameplay activity by spectating with decent graphics refresh frequency but very limited user interaction. The player goes back to the idle activity stage in the lobby before starting the second match. In realistic gameplay sessions, the spectate-and-play pattern often occurs multiple times (twice in this example) before the session ends.

\begin{table}[t!]
	\caption{Thirteen popular cloud game titles played on NVIDIA's GeForce NOW platform in our geography, with their game genre, gameplay activity pattern, and popularity by their fraction of total playtime.}
	\centering
	\resizebox{\columnwidth}{!}{
		\begin{tabular}{|l|l|l|l|}
			\hline
			\rowcolor[HTML]{C0C0C0} 
			\textbf{Game title}                    & \textbf{Game genre}                                  & \textbf{Activity pattern}                    & \textbf{Popularity} \\ \hline
			Fortnite                 & Shooter                          & Spectate-and-play                         & 37.80\%    \\ \hline
			Genshin Impact                  & Role-playing                             & Continuous-play                       & 20.10\%    \\ \hline
			Baldur's Gate 3           & Role-playing                          & Continuous-play                        & 3.30\%     \\ \hline
			R6: Siege                & Shooter                                    & Spectate-and-play               & 1.24\%     \\ \hline
			Honkai: Star Rail      & Role-playing                          & Continuous-play                       & 1.16\%     \\ \hline
			Destiny 2                & Shooter                                    & Spectate-and-play                  & 1.15\%     \\ \hline
			Call of Duty             & Shooter                                    & Spectate-and-play             & 0.97\%     \\ \hline
			Cyberpunk 2077           & Role-playing                             & Continuous-play             & 0.84\%     \\ \hline
			Overwatch 2              & Shooter                                    & Spectate-and-play             & 0.74\%     \\ \hline
			Rocket League            & Sports                                 & Spectate-and-play                  & 0.64\%     \\ \hline
			CS:GO/CS2                 & Shooter                                    & Spectate-and-play                        & 0.61\%     \\ \hline
			Dota 2                   & MOBA                                   & Spectate-and-play                        & 0.55\%     \\ \hline
			Hearthstone     & Card & Spectate-and-play                  & 0.04\%     \\ \hline
	\end{tabular}}
    \vspace{-3mm}
	\label{tab:game_selection}
\end{table}

The second representative gameplay is shown in Fig.~\ref{fig:cyberpunk_process}, where the user plays Cyberpunk 2077, a role-playing game that follows a \textbf{continuous-play} activity pattern. 
For this type of gameplay activity pattern, players often engage in active gameplay activity in a continuous manner for a relatively long period, except at the beginning (\eg login and character selection) and end of the session.
Under such game mechanism, during the active gameplay stage, the player explores the game world without interruption and completes multiple missions, with only occasional short occurrences of passive or idle stages (\eg static dialogue or transition animations) featuring infrequent graphics refresh and user motion updates.
As will be discussed in \S\ref{sec:dissect}, gameplay following the continuous-play player activity pattern has its traffic profile at a constantly high level with small fluctuations.

\subsection{Cloud Game Titles and Genres}\label{sec:cg_contexts}
Having discussed the three categories of player activity stages (\ie idle, passive, and active) observable within the two types of gameplay activity patterns (\ie spectate-and-play and continuous-play), each of which can result in unique expectations on game streaming quality such as frame rate and, therefore, network bandwidth demands, we now discuss the gameplay contexts including game titles and genres that inherently lead to these distinct gameplay activity patterns.

We have collected an exhaustive list of cloud game catalog from our partnered network operator hosting NVIDIA's GeForce NOW cloud gaming servers in our geography. While we cannot disclose the full list as requested by our industry partner, we present in Table~\ref{tab:game_selection} the top 13 game titles that contribute to over 69\% of total gameplay time during a half-year period encompassing millions of gameplay sessions, with their genres defined by the gaming community, popularity in percentage of gameplay time, and their respective gameplay activity patterns observed in our study listed.

\begin{figure}[t!]
	\includegraphics[width=\columnwidth]{lab_setup.pdf}
	\vspace{-3mm}
	\caption{Lab experiment setup for collecting traffic traces of cloud gaming sessions.}
	\label{fig:lab_setup}
\end{figure}

\begin{table}[t!]
	\caption{Our lab traffic capture dataset of NVIDIA's GeForce NOW cloud gameplay sessions across the thirteen popular game titles on a diverse profiles of user configuration.}
	\centering
	\resizebox{\columnwidth}{!}{
		\begin{tabular}{|l|l|l|l|l|l|}
			\hline
			\rowcolor[HTML]{C0C0C0} 
			\textbf{Device}        & \textbf{OS}           & \textbf{Software} & \textbf{Streaming settings} & \textbf{\#Sessions} & \textbf{Playtime} \\ \hline
			&                           & Native app        & UHD-SD; 30-120 fps               & 89                  & 10.9 hours              \\ \cline{3-6} 
			& \multirow{-2}{*}{Windows} & Browser           & QHD-SD; 30-120 fps               & 60                  & 6.8 hours                   \\ \cline{2-6} 
			&                           & Native app        & UHD-SD; 30-120 fps               & 76                  & 10.5 hours                  \\ \cline{3-6} 
			\multirow{-4}{*}{PC}     & \multirow{-2}{*}{macOS}   & Browser           & QHD-SD; 30-120 fps               & 61                  & 7.7 hours                   \\ \hline
			& Android                   & Native app        & QHD-FHD; 30-120 fps                  & 73                  & 9.1 hours                   \\ \cline{2-6} 
			\multirow{-2}{*}{Mobile} & iOS                       & Browser           & FHD-SD; 30-120 fps                    & 70                  & 8.8 hours                   \\ \hline
			TV                 & AndroidTV                & Native app        & FHD-SD; 30-120 fps                    & 48                  & 6.1 hours                   \\ \hline
			Console           & Xbox                      & Browser           & FHD-SD; 30-120 fps                    & 54                  & 7.1 hours                   \\ \hline
		\end{tabular}
	}
	\label{tab:dataset}
	\vspace{-3mm}
\end{table}

The top 13 game titles belong to five game genres as defined by the gaming community based on their content, including shooter, role-playing, sports, MOBA, and card games. The two examples just discussed in \S\ref{sec:cg_process} are both in the list. By investigating into all the thirteen game titles on the GeForce NOW platform as well as popular games on three other major cloud gaming platforms, we validated that the gameplay activity patterns (\ie spectate-and-play and continuous-play) as articulated in \S\ref{sec:cg_process} are directly correlated with their game title and genre. For example, all the six shooter games in the list have their gameplay activities following the spectate-and-play pattern with a repeating combination of idle, active, and/or passive player activity stages, so do MOBA and card games. Player activities in all cloud gameplay sessions of role-playing games are constantly active and thus follow the continuous-play pattern. 

\begin{figure*}[t!]
	\mbox{
		\subfigure[Genshin Impact on Windows app using FHD 60fps.]{
			\includegraphics[width=0.485\textwidth]{figures/packet_genshin_win_compressed.pdf}
			\label{fig:packet_genshin_win}
		}
		\subfigure[Genshin Impact on Android app using FHD 60fps.]{
			\includegraphics[width=0.485\textwidth]{figures/packet_genshin_android_compressed.pdf}
			\label{fig:packet_genshin_android}
		}
	}
	\mbox{
		\subfigure[Genshin Impact on Windows app using HD 30fps.]{
			\includegraphics[width=0.485\textwidth]{figures/packet_genshin_win_hd_compressed.pdf}
			\label{fig:packet_genshin_win_hd}
		}
		\subfigure[Fortnite on Windows app using FHD 60fps.]{
			\includegraphics[width=0.485\textwidth]{figures/packet_fortnite_win_compressed.pdf}
			\label{fig:packet_fortnite_win}
		}
	}
   \vspace{-3mm}
	\caption{Time-series scatter plots for the downstream packet payload sizes arriving in the first 60 seconds (covering the initial game launching stage) of four representative cloud game streaming sessions, where the packets can be categorized into the full, steady, or sparse groups.}
    \label{fig:packet_patterns}
\end{figure*}

As measured in \cite{lyu_do_2024,bhuyan_end_2022}, cloud gaming servers render gaming graphics that are streamed back to players' devices with different quality and encoding granularity depending on the game titles. Soon in \S\ref{sec:dissect}, we will discuss that the network traffic characteristics (\eg bandwidth usage) vary across not only game titles and gameplay activity patterns, but also player activity stages within each session, leading to diversified user expectations on streaming quality and thus network demands of the particular session. Therefore, knowing the cloud game contexts (\ie game titles/genres and player activity stages) is essential for network operators to accurately understand the effective cloud gaming quality delivered over their network infrastructure to the subscribers, so that they can precisely troubleshoot those that are indeed impacted by poor network conditions for proactive network optimization.