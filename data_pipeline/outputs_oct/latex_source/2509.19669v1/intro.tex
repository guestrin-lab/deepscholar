\section{Introduction} \label{sec:intro} 

Cloud gaming has gained significant popularity as an affordable and accessible alternative to traditional gaming by shifting intensive real-time graphics rendering and gaming computation from client hardware to cloud GPU servers. Users on high-speed networks can play the latest game releases without high-end graphics computing hardware. Driven by the increasing demand for high-quality gaming, the maturity of cloud GPU infrastructure, and the increase in the capacity of carrier networks \cite{ericsson_5g_2023}, the cloud gaming industry, led by tech giants such as NVIDIA \cite{nvidia_geforcenow}, Microsoft \cite{microsoft_xcloud}, Amazon \cite{amazon_luna} and Sony \cite{sony_ps5cloud}, is expected to soar to US\$143bn by 2032 \cite{yahoo2024}.

Network operators are experimenting with ways to monetize cloud gaming, by providing network assurance for the service. Unlike traditional online games that only require a few hundred Kbps of bandwidth from the network \cite{madanapalli_know_2022}, cloud games need much higher bandwidth of tens of Mbps; simultaneously, unlike traditional video streaming that can buffer seconds or even minutes of content, cloud game streaming experience is impacted by sub-second glitches in latency or packet drops \cite{lyu_do_2024}. This highly demanding nature of cloud gaming in terms of both bandwidth and latency/loss renders it amenable for a premium (monetizable) assurance service, such as via a 5G slice dedicated to cloud gaming \cite{vodafone_cloudgaming_5g}, or via dynamic provisioning \eg using the Quality-on-Demand API being standardized by the CAMARA group in the TM Forum \cite{camara_qod_api}. As an example, Telefonica announced in 2023 that it was conducting such a trial with the Blacknut Cloud Gaming service \cite{telefonica_blacknut_qod}, followed more recently by Singtel's collaboration with Tencent Games to launch a specialized 5G slice service for cloud game streaming \cite{singtel_tencent_2025} in 2025.

\begin{figure*}[t!]
	\centering
	\subfigure[The gameplay activity follows a divided \textbf{spectate-and-play} pattern in this shooter (CS:GO) cloud gaming session.]{
		\includegraphics[width=0.485\textwidth]{csgo_process.pdf}
		\label{fig:fortnite_process}
	}
	\subfigure[The gameplay activity follows a nearly \textbf{continuous-play} pattern in this role-playing (Cyberpunk 2077) cloud gaming session.]{
		\includegraphics[width=0.485\textwidth]{cyberpunk_process.pdf}
		\label{fig:cyberpunk_process}
	}
   \vspace{-5mm}
	\caption{Examples of the two types of gameplay activity patterns including (a) a shooter game session where player activities follow a spectate-and-play pattern and (b) a role-playing game session where player activities follow a continuous-play pattern.}
	\label{fig:main}
    \vspace{-3mm}
\end{figure*}

For a network operator to assure the gaming experience, they have to be able to measure the gaming experience accurately. This is not as trivial as measuring the throughput, latency, and loss of the game video stream, because these vary depending on the title/genre of the game \cite{bhuyan_end_2022,sabet_delay_2020}, as well as the activity stage of the player within the game \cite{carrascosa_cloud_2022,moll_network_2018}, namely whether they are active (\eg shooting), passive (\eg spectating), or idle (\eg between rounds). In the absence of this context, drops in bandwidth or streaming frame rate may wrongly be associated with poor experience, when in reality the user is engaged in a less intensive role-playing game or a strategy card game, or spectating others in a shooting/sports game.
For example, players in a First Person Shooter (FPS) game often actively engage in gameplay when a competitive match starts with stringent requirements on high streaming frame rate, and have relatively lower network requirements during passive activity stage such as matchmaking and observation. As another example, players of a card game are likely to have nearly static gaming scenarios with low expectations on streaming frame rate and bandwidth during the entire gameplay.

In this paper, we develop a novel method to help network operators measure the \textbf{context} of cloud gaming sessions, specifically the title/genre of the game and the player activity stage (\ie active, passive or idle) within the gameplay. This can be combined with prior works that have developed traffic analysis methods to classify cloud gaming sessions, identify user platforms, and measure objective Quality-of-Experience (QoE) metrics from network Quality-of-Service (QoS) parameters \cite{lyu_network_2024,carrascosa_cloud_2022,domenico_network_2021}, as well as studies that have benchmarked the degradation of game streaming quality due to network conditions \cite{lyu_do_2024,marchal_analysis_2023,ky_ml_2023,graff_analysis_2021,aumont_dissecting_2021}. With our three contributions listed below, this paper fills a gap by providing network operators with an accurate measure of the \textbf{effective} user experience that takes into account the context of game title/genre and player activity stage.

Our \textbf{first} contribution (\S\ref{sec:dissect}) systematically identifies the unique network traffic characteristics of cloud game streaming sessions in various gameplay contexts. We collect and study traffic traces of over 500 cloud gaming sessions across 13 most popular games spanning 5 genres and diverse game streaming settings. The labeled traffic dataset will be shared with the community. We reveal the unique statistical distributions in packet sizes and timings during the initial launch stage of each cloud game title/genre, as well as the relative volumetric patterns within each streaming session that relate to the player activity stages in the game.

In the \textbf{second} contribution (\S\ref{sec:detect}), we develop and evaluate a real-time network traffic analysis method to classify the cloud game title within the first five seconds of game launch, and continuously measure the player activity stage thereafter. Game title classification is achieved using pre-trained machine learning models on packet sizes and timing attributes from the first five seconds, whereas the player activity stage is continuously assessed from bidirectional volumetric profiles. When the fine-grained game title cannot be confidently inferred, our method infers the coarse-grained game genre (\ie gameplay activity pattern) from the transition behaviors between player activity stages.

Our \textbf{third} contribution (\S\ref{sec:insights}) deploys our cloud gameplay context classification into a live network operated by our partner Internet service provider (ISP) that hosts NVIDIA's GeForce NOW cloud gaming servers for our region. We collect data over a three-month period, spanning hundreds of thousands of cloud gaming hours. We validate our game title/genre classification method determined in real-time against ground truth from the game server logs produced offline after the conclusion of each gameplay session. We then highlight how bandwidth demand varies depending on the game title/genre, as well as the player activity stage within the game. Lastly, we show that by providing appropriate game contexts, our method can help network operators avoid mislabeling experience drops that are attributable to the game title/genre or player activity rather than poor network conditions, thereby helping them accurately gauge and assure cloud gaming experience over their network.
