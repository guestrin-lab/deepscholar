\section{Measuring Cloud Gaming Contexts at a Large Scale}  \label{sec:insights}

We have implemented and deployed our cloud gaming context measurement method in our partner ISP that hosts NVIDIA's GeForce NOW cloud gaming servers for our geography, providing valuable insights to precisely identify and troubleshoot user experience degradations that are indeed caused by network factors rather than mistaking low-throughput sessions caused by game titles and gameplay activity patterns as having poor experience. 
The real-time inference of game contexts is also meant to help our partner ISP in assessing the efficacy of their prioritization techniques in assuring cloud gaming experience over their 5G mobile broadband network.

\newcolumntype{?}[1]{!{\vrule width #1}}

\begin{table}[t!]
    \centering
        \caption{Player activity stage classification (by time slot) and gameplay activity pattern inference (by session) accuracy for continuous-play and spectate-and-play games.}
       \vspace{-3mm}
    \resizebox{\columnwidth}{!}{
    \begin{tabular}{|l|l?{0.5mm}l|l|}
    \hline
    \rowcolor[HTML]{C0C0C0} 
    \textbf{Gameplay actv. pattern}                                         & \textbf{Accur.} & \textbf{Player actv. stage} & \textbf{Accur.} \\ \hline
    {\color[HTML]{3531FF} }                                             &                                      & \coloractive{\textbf{Active}}                    & 94.1\%                             \\ \cline{3-4} 
    {\color[HTML]{3531FF} }                                             &                                      & \colorpassive{\textbf{Passive}}                   & 92.5\%                             \\ \cline{3-4} 
    \multirow{-3}{*}{{\color[HTML]{3531FF} \textbf{Continuous-play}}}   & \multirow{-3}{*}{95.7\%}             & \coloridle{\textbf{Idle}}                      & 97.6\%                             \\ \hline
    {\color[HTML]{6200C9} }                                             &                                      & \coloractive{\textbf{Active}}                    & 96.8\%                             \\ \cline{3-4} 
    {\color[HTML]{6200C9} }                                             &                                      & \colorpassive{\textbf{Passive}}                   & 95.9\%                             \\ \cline{3-4} 
    \multirow{-3}{*}{{\color[HTML]{6200C9} \textbf{Spectate-and-play}}} & \multirow{-3}{*}{97.2\%}             & \coloridle{\textbf{Idle}}                      & 98.4\%                             \\ \hline
    \end{tabular}
    }
    \label{tab:activity_stage}
\end{table}

In this section, we discuss representative insights from a three-month (1 December 2024 to 1 March 2025) measurement of cloud game contexts. The results are selected to comply with commercial confidentiality restrictions and our university ethical approval conditions as provided in Appendix~\S\ref{sec:appendix_ethics}.
One month before our field deployment, we evaluated our game title classification method in the deployment network by validating our classified game titles with the offline cloud game server logs that were generated after the conclusion of each game session and are not available to ISPs in daily operations. The evaluation results showed that our method can effectively classify the game titles for the thirteen popular games with an overall accuracy above 95\%, consistent with our lab evaluation results. As for the player activity stages, there is no record from cloud game server logs for us to validate outside the lab environment.

We now demonstrate how our measurement of cloud game contexts can help network operators understand player activity profiles across game titles (\S\ref{sec:usage_patterns}) which can lead to different expectations on game streaming quality; benchmark bandwidth demands that are unique to game titles and gameplay activity patterns (\S\ref{sec:network_demands}); and measure effective cloud gameplay experience by corroborating gameplay contexts with objective QoE and network QoS metrics (\S\ref{sec:effective_qoe}).


\begin{figure}[t!]
	\centering
	\subfigure[Average durations of player activity stages across 13 popular game titles.]{
		\includegraphics[width=\columnwidth]{figures/average_durations_barh.pdf}
		\label{fig:average_durations_barh}
	}
	\subfigure[Average durations of player activity stages across 2 activity patterns.]{
		\includegraphics[width=\columnwidth]{figures/average_durations_barh_by_type.pdf}
		\label{fig:average_durations_barh_by_type}
	}
   \vspace{-3mm}
	\caption{Average number of minutes spent in \coloractive{active}, \colorpassive{passive}, and \coloridle{idle} player activity stages per cloud gaming session for (a) 13 popular game titles and (b) 2 gameplay activity pattern types.}
	\label{fig:average_durations}
\end{figure}


\subsection{Player Activity Stages across Cloud Game Contexts}\label{sec:usage_patterns}
Our measurement of cloud gaming contexts provides network operators with player activity profiles of streaming sessions per cloud game title and genre, which serve as references for network operators to dynamically provision network resources, \eg allocate 5G eMBB slices with prioritized QoS profiles by our partnered ISP with an expected session duration and slice capacity, upon detecting a newly commenced game streaming session. 

We report the average duration of each game streaming session and each player activity stage (\ie active, passive, and idle) across the 13 popular game titles in Fig.~\ref{fig:average_durations_barh}. The results are aggregated for the entire three-month deployment period. We can clearly observe significant differences in both streaming durations and player activity stages.

For the session duration, nine games have long average durations near or above 1 hour. Those game titles with long session durations are either role-playing games that involve extensive dialogue-based content like Baldur's Gate (95 minutes) and Cyberpunk 2077 (82 minutes), or MOBA, Sports and Shooter games like Dota 2 and Rainbow Six Siege that have either long matchmaking or match durations. 
Rocket League and CS:GO are with short per-match duration and have the shortest average game streaming durations among the popular games.
Depending on the gameplay content, each game title has its unique distribution of player activity stages, such as a relatively larger fraction (\eg 25 -- 55\%) of duration in idle and passive stages for Baldur's Gate, Cyberpunk 2077, Honkai: Star Rail and Hearthstone due to their extensive static content or unskippable dialogues. On the contrary, Fortnite and Dota 2 have most of their gameplay duration in the active stage. 

For game sessions that do not belong to the thirteen popular titles and thus are coarsely categorized as either continuous-play or spectate-and-play games, from Fig.~\ref{fig:average_durations_barh_by_type}, we can also observe different expectations on the streaming session durations and the composition of player activity stages. Continuous-play games, which are often role-playing, have a longer session duration with a significant fraction of time spent in idle (26\%) stages. Spectate-and-play games are often match-based shooter, card, strategy or sports games without lengthy in-game static scenes and dialogues, therefore, have a larger fraction of time spent in the active stage.


\subsection{Network Bandwidth Demands across Cloud Game Contexts}   \label{sec:network_demands}
Among the network resource provisioning tasks, ensuring sufficient bandwidth for multimedia streaming sessions to support good quality-of-experience (QoE) metrics is often very challenging for network operators who operate with constrained bandwidth resources for massive subscribers, especially in high-density regions (\eg a 5G base station serving a metropolitan area) during peak hours.
Cloud game streaming sessions with different contexts exhibit various network bandwidth demands, determined not only by their streaming settings but also by the gameplay activity patterns that are inherently different across game titles.

The average bandwidth demands of streaming sessions that belong to the thirteen popular game titles and the two types of gameplay activity patterns are provided in Fig.~\ref{fig:bandwidth_violin_plot_horizontal} and Fig.~\ref{fig:bandwidth_violin_plot_horizontal_by_type} as representative examples, respectively.
According to the throughput distributions of each game title, we can observe unique ranges across different games. For example, high-demand games like Baldur's Gate and Fortnite have their maximum session-level average throughput up to 68 Mbps; while low-demand games like Hearthstone require only 20 Mbps for gameplay sessions with the best streaming settings. There are many streaming sessions with very low streaming throughput (\eg less than 1 Mbps), which likely suffer from poor user experience due to constrained network conditions with large game streaming lag mostly over 70ms from the measured objective QoE metrics. Therefore, they are excluded from our analysis of network bandwidth demands.

\begin{figure}[t!]
    \centering
    \subfigure[Average throughput per session across 13 popular game titles.]{\includegraphics[width=\columnwidth]{figures/bandwidth_violin_plot_horizontal.pdf}
	\label{fig:bandwidth_violin_plot_horizontal}}
    \subfigure[Average throughput per session across 2 gameplay activity patterns.]{\includegraphics[width=\columnwidth]{figures/bandwidth_violin_plot_horizontal_by_type.pdf}
	\label{fig:bandwidth_violin_plot_horizontal_by_type}}
    \vspace{-3mm}
    \caption{Average throughput per game streaming session for (a) 13 popular game titles and (b) 2 gameplay activity patterns.}
    \label{fig:bandwidth_violin_plot}
\end{figure}

Depending on the game streaming settings such as graphic resolutions and user setups (\eg PC or mobile devices), we have observed several (two to four) clusters of bandwidth demands for each game title. For example, as shown in Fig.~\ref{fig:bandwidth_violin_plot_horizontal}, Destiny 2 has three clusters of bandwidth in the ranges of 8 -- 18 Mbps, 20 -- 30 Mbps, and 35 -- 47 Mbps, respectively, each of which is mapped to a group of graphic resolution settings depending on the player devices. Detection of the player devices has been done in prior work \cite{lyu_network_2024} and is not discussed in this paper.
In Fig.~\ref{fig:bandwidth_violin_plot_horizontal_by_type}, similar throughput distributions, which are dominantly in the range of 10 -- 25 Mbps with the highest bandwidth demand exceeding 65 Mbps for streaming in very high graphic resolutions, are observed for both continuous-play and spectate-and-play games, with slightly higher bandwidth demands seen for spectate-and-play games. 

With a concrete understanding of the throughput requested for streaming sessions of certain game titles, our partnered ISP is better equipped to prioritize premium users with the appropriate QoS (bandwidth) profiles for service quality assurance without over-provisioning the constrained network resources.


\begin{figure}[t!]
    \centering
    \subfigure[Objective/effective session QoE across 13 popular game titles.]{
        \includegraphics[width=\columnwidth]{figures/qoe_stacked_horizontal_bars_by_game.pdf}
        \label{fig:qoe_stacked_horizontal_bars_by_game}
    }
    \subfigure[Objective/effective session QoE across 2 gameplay activity patterns.]{
        \includegraphics[width=\columnwidth]{figures/qoe_stacked_horizontal_bars_by_game_type.pdf}
        \label{fig:qoe_stacked_horizontal_bars_by_game_type}
    }
    \vspace{-3mm}
    \caption{Average fraction of gameplay duration per game streaming session in good, medium and bad user experience for both objective QoE and effective QoE levels, \ie before and after calibration with gameplay contexts, across (a) 13 popular game titles and (b) 2 gameplay activity pattern types.}
    \label{fig:insights_effective_qoe}
\end{figure}


\subsection{Measuring Effective Game Streaming Experience with Contexts}  \label{sec:effective_qoe}
Most importantly, by combining our measurement of cloud gaming contexts with objective QoE metrics (\eg streaming frame rate) and standard network QoS metrics (\ie latency, packet drop rate, and throughput), our partnered network operator is now able to measure the effective game streaming experience as perceived by subscribers, without mislabeling streaming sessions of certain game titles or periods of gameplay in idle/passive player activity stages that inherently require low streaming frame rate and network bandwidth as degraded user experience.

\textbf{Calibrating user experience measurement:} 
In the three-month deployment, our measurement system is operated in parallel with an existing network observability module (the bottom gray box in Fig.~\ref{fig:pipeline}) operated by our partnered ISP, which labels the objective quality-of-experience (QoE) of game streaming sessions into three levels as bad, medium or good, by mapping the measured frame rate, throughput, latency and packet drop rate of streaming flows into the expected value ranges that are internally maintained in the observability system. For example, a session with a streaming frame rate lower than 30 FPS and/or a throughput below 8 Mbps will be labeled with bad objective QoE. Augmented with the results produced in this work, the expected value ranges of the objective QoE levels are empirically calibrated according to our real-time measurement of cloud game contexts, annotated as \textbf{effective QoE levels}. After calibration, reasonable drops in \textbf{streaming frame rate} and \textbf{throughput} due to less demanding game titles (\eg Hearthstone and Honkai: Star Rail) or player activity stages (\ie idle and passive), which intrinsically have less dynamic graphics or infrequent scene changes and hence require lower network resources, will not be mislabeled with degraded QoE levels.
The expected value ranges of \textbf{low latency} and \textbf{packet drop rate} between players and cloud servers remain unchanged from the objective QoE levels to the effective QoE levels after the context-based calibration.

In Fig.~\ref{fig:insights_effective_qoe}, we show how cloud gaming contexts help our partnered ISP reduce falsely labeled sessions with degraded game streaming user experience, \ie by transitioning from objective QoE to effective QoE levels. 
For simplicity, we show the overall QoE levels of each game streaming session, representing the majority QoE labels measured for the respective session in real-time. It is clear from Fig.~\ref{fig:qoe_stacked_horizontal_bars_by_game} that all game titles have significantly more sessions with good QoE levels after being calibrated with the gameplay contexts. For example, all game streaming sessions of Hearthstone, a low-demanding card game, have medium or bad objective QoE levels, 80\% of which are corrected to the good effective QoE level. Another example is Cyberpunk 2077, which has 40\% and 16\% of its streaming sessions labeled as medium and bad for the objective QoE levels. Although this game title has high demands for network bandwidth and frame rate during the active stage, a large fraction of its player activity stages are passive and idle, leading to 95\% sessions with good effective QoE levels after calibration. 

For streaming sessions of less popular game titles that are classified as continuous-play or spectate-and-play games, in Fig.~\ref{fig:qoe_stacked_horizontal_bars_by_game_type}, we can also observe that about half of the total sessions are corrected from medium or poor in objective QoE levels to good in effective QoE levels.

By calibrating the mapping criteria of game streaming sessions with our real-time classification of cloud gameplay contexts for effective QoE measurement, our partnered network operator is able to precisely identify the groups of under-performing game streaming sessions in a much smaller (and therefore manageable) quantity, such as those from clients connected to its 5G home broadband network via a poorly configured mobile cell. It enables precise troubleshooting without mislabeling streaming sessions with low network demands as poor user experience. In addition, knowing the effective QoE after our context-based calibration helps the ISP to reactively improve under-performing game streaming sessions served by its 5G broadband network by enforcing QoS profiles for the respective sessions via network slices with higher capacity.