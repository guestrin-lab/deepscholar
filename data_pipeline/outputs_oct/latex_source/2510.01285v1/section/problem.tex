\section{Problem Formulation}

Let $\sD = \{\evd_{i}\}_{i=1}^{N}$ denote a data lake consisting of $N$ distinct data files, each containing information potentially completely or partially relevant to answering a data science question $q$ (some examples of these questions are shown in Figures~\ref{fig:search-example-1} and \ref{fig:search-example-2} in Appendix~\ref{app:case-study}). The objective of this work is to design a generative system $\pi_s$ that, given the query $q$ and the data lake $\sD$ as input, produces a program $p \sim \pi_s(q;\sD)$ in response. When executed (e.g., using a Python interpreter in this paper), this program $p$ retrieves, loads, and processes the appropriate data from the data lake $\sD$ and solve the given problem in the question $q$ to compute the answer. To evaluate the generated program $p$, we assume the existence of an evaluation function $\mu_{\text{generation}}$ that executes $p$ to produce an output $o_{p}$, compares $o_{p}$ with the ground-truth response $y_{q}$, and assigns a corresponding score. In addition, we assume a metric $\mu_{\text{retrieval}}$, given the program $p$ and the ground-truth files $\sD_{q}$, assigns a score reflecting the performance in discovering the correct data sources.