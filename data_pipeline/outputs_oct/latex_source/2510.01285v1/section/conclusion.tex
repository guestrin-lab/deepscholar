\section{Conclusions \& Future Work}

We addressed the critical challenge of data discovery in large, heterogeneous data lakes, a key bottleneck for applying LLMs in data science. We introduced a novel multi-agent communication paradigm based on the blackboard architecture, which replaces rigid centralized task assignment with a flexible, decentralized model of agent collaboration. Extensive experiments on three data science benchmarks demonstrate that our framework consistently outperforms strong baselines, including RAG and the master–slave paradigm, achieving up to 57\% relative improvement in end-to-end task success and a 9\% relative gain in data discovery accuracy. These results highlight the importance of communication architecture in multi-agent systems and establish the blackboard paradigm as a scalable, flexible, and effective solution for complex data science workflows.


Future work could extend the blackboard architecture and paradigm beyond data science, as the proposed approach is general and applicable to a wide range of multi-agent systems and domains. Another promising direction is to investigate more adaptive strategies for data partitioning among agents, enabling the system to better handle dynamic and evolving data environments. Ultimately, our findings point toward a broader path for developing more capable, scalable, and autonomous multi-agent AI systems for real-world data analysis applications.