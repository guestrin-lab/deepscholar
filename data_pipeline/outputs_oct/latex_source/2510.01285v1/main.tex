% This class has a lot of options, so please check deepmind.cls for more details.
% This is a minimal set for most needs.
\documentclass[11pt, a4paper, logo]{googlecloud}

% Omit dates for reproducibility.
\pdfinfoomitdate 1
\pdftrailerid{redacted}

% This avoids duplicate hyperref bookmark entries when using \bibentry (e.g. via \citeas).
\makeatletter
\renewcommand\bibentry[1]{\nocite{#1}{\frenchspacing\@nameuse{BR@r@#1\@extra@b@citeb}}}
\makeatother

\RequirePackage{algorithm}
\RequirePackage{algorithmic}

\usepackage{kantlipsum, lipsum}
\usepackage{dsfont}
% \usepackage{gdm-colors}

% Sometimes you will get errors about pdflink ending up in diffrent position. Try this and
% comment it out again when you are done with your document.
%\hypersetup{draft}

% Set the bibliography options here.
\usepackage[authoryear, sort&compress, round]{natbib}

\usepackage[utf8]{inputenc} % allow utf-8 input
\usepackage[T1]{fontenc}    % use 8-bit T1 fonts
\usepackage[loose]{minitoc}
\usepackage{url}            % simple URL typesetting
\usepackage{booktabs}       % professional-quality tables
\usepackage{nicefrac}       % compact symbols for 1/2, etc.
\usepackage{microtype}      % microtypography
\usepackage{amsmath}
\usepackage{graphicx}
\usepackage{multicol}
\usepackage{hyperref}       % hyperlinks
\usepackage[nameinlink]{cleveref}
\usepackage{bbm}
\usepackage{multirow}
\usepackage{adjustbox}
\usepackage{listings}
\usepackage{subcaption}
% \usepackage{subfigure}
\usepackage{soul}
% \usepackage{floatrow}
\usepackage{float}
\usepackage{wrapfig}
\usepackage{blindtext}
\usepackage{tablefootnote}
% \usepackage{fdsymbol}. % cause an error
\usepackage{amsfonts}
\usepackage[flushleft]{threeparttable}
% \usepackage{colortbl}
\usepackage{bbding}
\usepackage{xcolor}
\usepackage{xspace}
\usepackage{bm}
\usepackage{arydshln}
% \usepackage{subfigure}
\usepackage{enumitem}
\usepackage{setspace}
% \usepackage{mathabx}
\usepackage{color}
\usepackage{longtable}
% \usepackage{booktabs}

\usepackage[normalem]{ulem}
\usepackage{ulem}
\usepackage[nomargin,inline,marginclue,draft]{fixme}
\usepackage{balance}
\usepackage{verbatim}
\usepackage{diagbox}
\usepackage{changepage}
\usepackage{amssymb}
\usepackage{pifont}

\usepackage{array}   % Optional: Improves table column alignment


\usepackage{makecell}
\usepackage{tabularray}

% \usepackage{kantlipsum, lipsum}
% \usepackage{dsfont}
% % \usepackage{gdm-colors}
% % Optional math commands from https://github.com/goodfeli/dlbook_notation.
\input{math_commands.tex}

\usepackage{tcolorbox}

\definecolor{dkgreen}{rgb}{0,0.6,0}
\definecolor{gray}{rgb}{0.5,0.5,0.5}
\definecolor{mauve}{rgb}{0.58,0,0.82}

\newcommand{\greencheck}{\textcolor{green}{\ding{51}}}
\newcommand{\redx}{\textcolor{red}{\ding{55}}}

% % custom packages and commands
% \usepackage{graphicx}
% \usepackage{hyperref}
% \usepackage{url}
% \usepackage{xcolor}
% \usepackage{tabularx}
% \usepackage{multicol}
% \usepackage{colortbl}
% \usepackage{booktabs}
% \usepackage{xspace}
% \usepackage{algorithm}
% \usepackage{algorithmic}
% \usepackage{algpseudocode}
% \usepackage{algorithmicx}
% \usepackage{bbm, mathtools}
% \usepackage{wrapfig}
% \usepackage{hyperref}
% % \hypersetup{
% %   colorlinks,
% %   citecolor=blue,
% %   linkcolor=red,
% %   urlcolor=blue
% % }

\usepackage{tcolorbox}

\definecolor{dkgreen}{rgb}{0,0.6,0}
\definecolor{gray}{rgb}{0.5,0.5,0.5}
\definecolor{mauve}{rgb}{0.58,0,0.82}

\lstset{
  frame=tb,
  aboveskip=3mm,
  belowskip=3mm,
  showstringspaces=false,
  columns=flexible,
  basicstyle={\small\ttfamily},
  numbers=none,
  numberstyle=\tiny\color{gray},
  keywordstyle=\color{blue},
  commentstyle=\color{dkgreen},
  stringstyle=\color{mauve},
  breaklines=true,
  breakindent=0ex,
  breakatwhitespace=true,
  tabsize=2
}

\definecolor{codegreen}{rgb}{0,0.6,0}
\definecolor{codegray}{rgb}{0.5,0.5,0.5}
\definecolor{codepurple}{rgb}{0.58,0,0.82}
\definecolor{backcolour}{rgb}{0.95,0.95,0.92}

\lstdefinestyle{mystyle}{
    backgroundcolor=\color{backcolour},   
    commentstyle=\color{codegreen},
    keywordstyle=\color{magenta},
    numberstyle=\tiny\color{codegray},
    stringstyle=\color{codepurple},
    basicstyle=\ttfamily\footnotesize,
    breakatwhitespace=false,         
    breaklines=true,                 
    captionpos=b,                    
    keepspaces=true,                 
    numbers=left,                    
    numbersep=5pt,                  
    showspaces=false,                
    showstringspaces=false,
    showtabs=false,                  
    tabsize=2
}
\lstset{style=mystyle}
\lstset{escapeinside={<@}{@>}}
 
\newcommand{\todo}[1]{\textcolor{red}{#1}}

\newcommand{\stitle}[1]{\noindent \textbf{#1.}}

% Drafting notes
\newcommand{\chengrun}[1]{\textcolor{cyan}{\bf \small [Chengrun: #1]}}

\title{LLM-based Multi-Agent Blackboard System for Information Discovery in Data Science}

\correspondingauthor{Alireza Salemi (asalemi@cs.umass.edu), Mihir Parmar (mihirparmar@google.com), Jinsung Yoon (jinsungyoon@google.com), and Hamid Palangi (hamidpalangi@google.com)}

% \renewcommand{\today}{}

% --- AUTHOR DEFINITIONS ---
% 1. Use \footnotemark without a number.
\author[1]{Alireza Salemi$^{\dag}$}
\author[2]{Mihir Parmar}
\author[2]{Palash Goyal}
\author[2]{Yiwen Song}
\author[2]{Jinsung Yoon}
\author[1]{Hamed Zamani}
\author[2]{Hamid Palangi$^*$}
\author[2]{Tomas Pfister$^*$}
% --------------------------


\affil[1]{University of Massachusetts Amherst}
\affil[2]{Google Cloud AI Research}

% The \author macro works with any number of authors. There are two commands
% used to separate the names and addresses of multiple authors: \And and \AND.
%
% Using \And between authors leaves it to \LaTeX{} to determine where to break
% the lines. Using \AND forces a linebreak at that point. So, if \LaTeX{}
% puts 3 of 4 authors names on the first line, and the last on the second
% line, try using \AND instead of \And before the third author name.

Generating realistic videos with diffusion transformers demands significant computation, with attention layers becoming the central bottleneck.
Even producing a short clip requires running a transformer over a very long sequence of embeddings, e.g., more than 30K embeddings for a 5-second video. These long sequence lengths thus incur significant compute latencies. Prior work aims to mitigate this bottleneck by exploiting sparsity in the attention layers to reduce the computation required. However, these works typically rely on block-sparse attention, which skips score computation only when all entries in a \emph{block} of attention scores (corresponding to $M$ queries and $M$ keys, with $M=64$ typically) are zero. This coarse-granular skipping of attention scores does not fully exploit sparsity in the attention map and leaves significant room for improvement.


In this work, we propose \X, a sparse attention mechanism for long-context diffusion transformers that leverages sparsity at a fine granularity. 
Unlike block-sparse attention, which skips entire $M \times M$ blocks, our approach skips computations at the granularity of $M \times 1$ \emph{slices} of the attention map. Each slice is produced as a result of query-key dot products between a block of query vectors and a \emph{single key}. To implement our proposed sparse attention mechanism, we construct a new highly efficient bulk-load operation called asynchronous-gather load. This load operation gathers a sparse set of relevant key-value vectors from memory and arranges them into packed tiles in the GPU's shared memory. In this manner, only a sparse set of keys relevant to those queries are loaded into shared memory when computing attention for a block of queries, in contrast to loading full blocks of key tokens in block-sparse attention. Our fine-grained sparse attention, applied to video diffusion models, achieves an average 1.55x (up to 1.65x) speedup for 5 second, 480p videos, and an average 1.41x (up to 1.49x) for 5 second, 720p videos on a single H100 GPU.

\textbf{\textcolor{magenta}{Code:}} \url{https://github.com/sankeerth95/FG-Attn}

% \begin{center}
% \begingroup
% \endgroup
% \end{center}



%\iclrfinalcopy % Uncomment for camera-ready version, but NOT for submission.
\begin{document}
\doparttoc % Tell to minitoc to generate a toc for the parts
\faketableofcontents % Run a fake tableofcontents command for the partocs
\maketitle

% --- FOOTNOTE TEXT DEFINITION ---
% This block defines the footnote text and forces the marker to be a dagger (\dag).
\begingroup
    % 2. Temporarily redefine the footnote marker to \dag
    \renewcommand\thefootnote{\dag} 
    
    % 3. Use \footnotetext to display the text. The marker is now \dag.
    \footnotetext{Work done as a Student Researcher at Google.}
    \renewcommand\thefootnote{*}
    \footnotetext{Joint last authors.}
\endgroup

% Reset the footnote counter back to 0 for the main document body, 
% although it's usually unnecessary with the \begingroup/\endgroup.
\setcounter{footnote}{0}
% ----------------------------------


\section{Introduction}\label{sec:introduction}
Mixture-of-Experts (MoE) is an architectural paradigm that adaptively combines predictions from multiple neural modules, known as "experts," via a learned gating mechanism. This concept has evolved from ensemble-based MoEs, where experts, jointly trained with a gating function, are often full, independent models whose outputs are combined to improve overall performance and robustness \citep{jacobs1991adaptive}. More recently, MoE layers have been integrated within larger neural architectures, with experts operating in a latent domain. These "latent MoEs" offer significant scalability benefits, especially in large language models (LLMs) \citep{shazeer2017outrageously,fedus2022switch}.
MoE makes it possible to train massive but efficient LLMs, where each token activates only a fraction of the model’s parameters, enabling specialization, better performance, and lower computational cost compared to equally sized dense models.

Regardless of their specific implementation, conventional MoE systems typically produce point estimates, lacking a direct quantification of their uncertainty. In critical applications, this absence of uncertainty information hinders interpretability, making it difficult for users to gauge the reliability of a prediction and limits informed decision-making, as the system cannot express its confidence or identify ambiguous cases. Importantly, the learned gating mechanism, which dictates the relative contribution of each expert, does not take into account expert confidence, potentially leading to suboptimal routing decisions.

In this work, we propose Mixture-of-Gaussians with Uncertainty-based Gating (MoGU), a framework for uncertainty-aware MoE architectures, which provides explicit uncertainty quantification for both individual experts and the overall MoE model. Our approach fundamentally reimagines the expert's output: instead of a point estimate, we model each expert's prediction as a random variable drawn from a normal distribution. In this setup, each expert simultaneously predicts both the mean (the label estimate) and variance of the distribution, representing its predictive uncertainty. This shift enables a more nuanced understanding of expert behavior and the derivation of the overall model's uncertainty. Furthermore, we introduce a novel gating mechanism where the estimated uncertainty of each expert directly informs its relative contribution to the overall MoE prediction, bypassing the need for a separate gating function typically found in traditional MoE setups. This creates a self-aware MoE where more confident experts naturally exert greater influence.

We evaluate MoGU on time series forecasting as our primary regression task. This choice is motivated by the inherent uncertainty in real-world time series data and the wide variety of expert architectures applicable to forecasting tasks across numerous domains \citep{time_series_survey, wang2024deep}. Our evaluation spans various expert types, forecasting benchmarks and forecasting horizon sizes, allowing for a comprehensive assessment of our method's efficacy. MoGU is shown to consistently yield more accurate forecasts compared to input-based gating MoE architectures, while simultaneously, providing uncertainty estimates that are positively correlated with prediction error. These estimates are available at both the individual expert and overall model levels. By further distinguishing between aleatoric (data-related) and epistemic (model-related) uncertainty, MoGU offers valuable insights into the source of a model's uncertainty. We also conducted a detailed ablation study to validate our key design choices.

In summary, our contributions are as follows: 
\begin{itemize}
\item \textbf{MoGU: A Novel Framework for Uncertainty-Aware MoE Architectures}: We introduce a novel framework that directly quantifies uncertainty for both individual experts and the overall model, moving beyond conventional point estimates. A key innovation is a routing mechanism that uses each expert’s estimated predictive uncertainty to dynamically determine its contribution to the final MoE output, replacing traditional input-based gating mechanisms.
\item \textbf{MoGU Improves Time Series Forecasting}: Our method effectively reduces forecasting error across various benchmarks, horizon lengths, and expert architectures.
\item \textbf{MoGU Provides Meaningful Uncertainty Estimates for Time Series Forecasting}: MoGU generates uncertainty estimates at the expert-level and overall. These estimates are positively correlated with prediction error, providing valuable insight into the model's confidence and the sources of its uncertainty.
\end{itemize}

By embedding uncertainty estimation into prediction and gating, MoGU moves beyond input-based gating  MoEs toward architectures that are more accurate, transparent, and reliable.


\section{Problem Formulation}

Let $\sD = \{\evd_{i}\}_{i=1}^{N}$ denote a data lake consisting of $N$ distinct data files, each containing information potentially completely or partially relevant to answering a data science question $q$ (some examples of these questions are shown in Figures~\ref{fig:search-example-1} and \ref{fig:search-example-2} in Appendix~\ref{app:case-study}). The objective of this work is to design a generative system $\pi_s$ that, given the query $q$ and the data lake $\sD$ as input, produces a program $p \sim \pi_s(q;\sD)$ in response. When executed (e.g., using a Python interpreter in this paper), this program $p$ retrieves, loads, and processes the appropriate data from the data lake $\sD$ and solve the given problem in the question $q$ to compute the answer. To evaluate the generated program $p$, we assume the existence of an evaluation function $\mu_{\text{generation}}$ that executes $p$ to produce an output $o_{p}$, compares $o_{p}$ with the ground-truth response $y_{q}$, and assigns a corresponding score. In addition, we assume a metric $\mu_{\text{retrieval}}$, given the program $p$ and the ground-truth files $\sD_{q}$, assigns a score reflecting the performance in discovering the correct data sources.
\section{LLM-based Multi-Agent Blackboard System}

This section introduces an alternative communication paradigm for LLM-based multi-agent systems inspired by blackboard systems \citep{10.1145/356810.356816}, distinct from the widely used master–slave architecture. As outlined in \textsection \ref{sec:introduction}, blackboard-based multi-agent systems provide several advantages over the master-slave approach. Here, rather than directly assigning tasks to sub-agents, the main agent posts its requests (i.e., sub-tasks for which it requires assistance) on a shared blackboard, which functions as a broadcast channel accessible to all other agents. Each helper agent independently evaluates whether it can respond to a request, considering its own capabilities, availability, cost, and other factors. If an agent decides to contribute, it writes its response to the corresponding request, and the main agent then decides whether to use or ignore the provided information. \textit{This way, all agents in the system retain full autonomy over their actions, and no centralized controller forces them to execute a specific task.} While the blackboard paradigm is applicable to a wide range of multi-agent systems, we focus on data science tasks that require data discovery, where its characteristics are particularly advantageous, as discussed in \textsection \ref{sec:introduction}. The remainder of this section details our method and its design for data science problems that require information discovery.



\paragraph{Overview:} 

An overview of our proposed method is presented in Figure~\ref{fig:overview}. The system $\pi_{s}$ operates over the data lake $\sD$ by first partitioning $\sD$ into $C$ clusters of related files. Each cluster $\sD_i$ is assigned to a file agent $\pi_{f_i}$, which is responsible for handling, loading, processing, and retrieving information from the files within its cluster. In addition, a search agent $\pi_{s}$ is included to retrieve external information from the web that may be required to solve the problem. The overall system $\pi_{s}$ is composed of a main agent $\pi_{m}$, which is responsible for solving the query $q$, and a set of $C+1$ helper agents $\Pi_{\text{helper}} = \{\pi_{f_i}\}_{i=1}^{M} \cup \{\pi_{s}\}$ that provide specialized assistance. The query $q$ is presented to $\pi_{m}$, which iteratively selects an action $a \in \sA$ from the action space $\sA$, executes the chosen action, and observes the resulting outcome from the environment. Among its actions, the main agent may interact with a blackboard $\beta$, a shared communication medium where it can post a request $r$ without addressing a specific sub-agent. The helper agents $\Pi_{\text{helper}}$ continuously monitor the blackboard, determine whether they can address a posted request, and, if so, provide their outputs on the corresponding response board $\beta_{r}$. These responses are then collected and made available to $\pi_{m}$, which incorporates them into its decision-making process.\footnote{Responses are not written back to the blackboard $\beta$ to avoid dependencies where one sub-agent's output could influence the behavior of others negatively. Instead, all responses are directed exclusively to the response board $\beta_{r}$, ensuring independent operation of sub-agents and exclusive access by the main agent $\pi_{m}$.} The main agent is limited to at most $T$ sequential actions (including actions that interact with the blackboard) to solve the query $q$, ultimately producing a program $p$ in python programming language that computes the final answer to $q$.

\paragraph{Clustering Data Lake:} 

There are multiple approaches for partitioning the data lake into clusters; applying clustering algorithms over file representations, random partitioning, or other heuristic methods. For simplicity, we do not utilize file content and instead rely solely on file names during clustering. Specifically, the file names are provided to an LLM---Gemini-2.5-Pro\footnote{Available at: \url{https://cloud.google.com/vertex-ai/generative-ai/docs/models/gemini/2-5-pro}}---which using the prompt shown in Figure~\ref{fig:clustering-prompt}, clusters the files into categories based only on their names.\footnote{This method represents just one simple possible approach to clustering, chosen for simplicity; more scalable and accurate alternatives could equally be employed in real world scenarios.} An example of this clustering is provided in Figure~\ref{fig:clustering-example} in Appendix~\ref{app:case-study}, where the model successfully groups related files together. For instance, it clusters all files originating from the National Interagency Fire Center into a category labeled ``NIFC Wildfire Statistics.'' The number of automatically derived clusters for each dataset is reported in Table~\ref{tab:stats} in Appendix~\ref{app:dataset}.



% The remainder of this section details the design of the main agent and the helper agents, emphasizing how their coordination supports effective information discovery in data science tasks.

\subsection{Main Agent}
\label{sec:main-agent}

The primary role of the main agent is to solve the problem in collaboration with the helper agents. The main agent follows the ReAct framework \citep{yao2023react}, where at each step $t$, given the query $q$ and the history of actions and observations $\sH_{t-1}$, it first reasons about what is the best next action and selects an action from a predefined action space, executes the action, observes the outcome, and appends the resulting observation to update the history $\sH_{t}$.\footnote{In this work, the inputs, outputs of the model, and observations are appended directly to the prompt of the LLM, formatted according to its chat-based input template.} The prompt used by the main agent is shown in Figure~\ref{fig:main-agent-blackboard-prompt} in Appendix~\ref{app:prompts}. The agent selects one of the following predefined actions in each step, executes them, and observe their outcomes:

\begin{itemize}[leftmargin=*]
    \item \textit{\textbf{Planning:}} In this action, the LLM decomposes the problem into smaller sub-problems and outlines a plan for addressing each of them. This action has no external effect on the environment but serves as an internal reasoning step to guide the LLM's problem-solving process. In response, the system simply acknowledges the proposed plan and instructs the LLM to proceed.
    
    \item \textit{\textbf{Reasoning:}} In this action, the LLM focuses on a specific aspect of the problem and explains its reasoning, analysis, or interpretation of the available observations and steps taken so far in this process. Similar to the planning step, this action has no external effect on the environment but functions as an internal reasoning mechanism to guide the LLM's problem-solving process. In response, the system simply acknowledges the reasoning and prompts the LLM to continue.
    
    \item \textit{\textbf{Executing Code:}} In this action, the agent generates python code, which is executed using a python interpreter. If the code runs successfully, the resulting outputs are returned to the agent for observation; otherwise, the agent receives the corresponding error messages. This action enables the agent to explore the problem interactively, inspect data files, and experiment with them to gain a deeper understanding of their content and structure and how to process them.
    
    \item \textit{\textbf{Requesting Help:}} In this action, the agent formulates a request for assistance from the sub-agents, specifying, for example, the types of data files or information needed, or the resources required to apply a tool or solve a sub-problem. This request is posted on the blackboard $\beta$ for visibility by the helper agents. Once the sub-agents respond, if they respond, their responses on the response board $\beta_r$ are collected and provided back to the main agent as the outcome of this action for observation and further use in its decision-making process.
    
    \item \textit{\textbf{Answering:}} In this action, the agent concludes the problem-solving process by generating a final program that produces the answer to the query. This action terminates the process, and the output of this step constitutes the final program $p$ generated by the system to address the problem.
\end{itemize}

\subsection{Helper Agents}
\label{sec:sub-agents}

In a data science, information discovery can typically be categorized into two tasks: (1) identifying the specific files that contain the data necessary to the problem, and (2) retrieving general knowledge about concepts relevant to the problem, such as domain-specific terms or details of particular algorithms and methods. To support these, our framework employs two types of helper agents:

\paragraph{File Agent:} 

Handling all the files in a data lake with a single agent is not feasible for several reasons: it typically involve a large number of files, many of which are lengthy and may exceed the agents context window; the files span diverse topics, which can confuse the agent and hinder effective reasoning; and accessing and processing all files simultaneously can be computationally expensive and inefficient, leading to unnecessary overhead and slower problem-solving. For these reasons, in our framework each file agent is assigned responsibility for a subset of data files determined to be relevant, as described earlier in the clustering procedure. In an offline phase, the file agent $\pi_{f_i}$ takes as input a subset of the data lake $\sD_{i}$ and operates through a two-step procedure. In the first step, the agent selects a subset\footnote{When filenames indicate multiple files containing the same type of data over different time periods, the agent does not need to inspect all of them to infer the structure; a small representative sample is sufficient.} (or all) of the files to examine their content. The contents of them are presented to the agent for inspection (details of presentation are in Appendix~\ref{app:implementation}). In the second step, after observing the selected files, the agent reasons about and analyzes them, learning how they are structured, what pre-processing or transformations may be required, and how they should be processed in general. An example of such an analysis is provided in Figure~\ref{fig:file-agent-analyze-example} in Appendix~\ref{app:case-study}. Then, in the online phase, the agent listens for requests from the main agent. Upon receiving a request, based on the analysis it did earlier, it determines whether it can contribute to answering it. If so, the agent generates a detailed plan specifying which files in $\sD_i$ are relevant, how they should be loaded in Python code, what libraries to use, the steps required for data processing, and samples from the data. The prompt used to guide the file agent is shown in Figure~\ref{fig:file-agent-prompt} in Appendix~\ref{app:prompts}. 

\paragraph{Search Agent:}

Certain data science problems require task-specific knowledge about algorithms or domain expertise that the LLM may not possess. To address this, we design a web-search agent that retrieves relevant information from a search engine. This agent operates according to the prompt shown in Figure~\ref{fig:search-agent-prompt} in Appendix~\ref{app:prompts}. Given a request $r$ posted on the blackboard $\beta$, the agent first determines whether it is capable of addressing the request. It is specifically restricted to general web-based information retrieval and does not respond to requests involving access to local files or datasets. If the agent determines that the request can be answered, it enters an iterative search process with a maximum of $T_{\text{search}} = 3$ steps. At each step $t$, the agent generates a set of queries $\sQ_{t}$, which are submitted to a search engine---in this work, Google Custom Search Engine\footnote{We use Google Custom Search Engine, configured to exclude all websites associated with the datasets used in this paper to prevent data leakage: \url{https://developers.google.com/custom-search}}---to retrieve $k=3$ webpage per query. The content of the webpages are then extracted using \textit{beautifulsoup} library\footnote{Available at: \url{https://pypi.org/project/beautifulsoup4/}} to be presented to the search agent. The extracted documents are then evaluated by the agent to determine whether they provide sufficient information to answer the request. If so, the agent generates a response to the request, which is posted to the response board $\beta_r$. If the information is insufficient, a new set of queries is generated to continue gathering relevant data from the web.

\section{Experiments}

\subsection{Experimental Setup}
\label{sec:exp-setup}

\paragraph{Benchmarks:}


To the best of our knowledge, KramaBench is the only public benchmark for data science problems that explicitly incorporates a data discovery phase, which we adopt in our evaluation. In addition, we repurpose two existing datasets, DS-Bench \citep{jing2025dsbench} and DA-Code \citep{huang-etal-2024-da}, to include in this phase. Specifically, we manually filtered out all questions that do not require any data file for answering, as well as those that lack sufficient hints for data discovery.\footnote{For example, questions that request the computation of a data science metric on a column without specifying the structure or content of the relevant file.} After filtering, we aggregated all remaining files across questions into a unified data lake, such that the model must perform discovery to identify relevant files at inference time. In this setup, only the question and the data lake are provided to the model, requiring it to identify the relevant files to answer the question, following the same protocol as KramaBench.
% \footnote{The constructed datasets will be released publicly upon acceptance of the paper.} 
Further details on this filtering process, along with dataset statistics in Table~\ref{tab:stats}, are provided in Appendix~\ref{app:dataset}.


\paragraph{Evaluation:}

To evaluate the generated programs, we execute each and compare its output against the ground-truth reference for the corresponding question. For each dataset, we adopt its standard evaluation protocol. For KramaBench, we use the official evaluation script provided in its repository.\footnote{Available at: \url{https://github.com/mitdbg/KramaBench}} For DA-Code, we likewise rely on the official evaluation script released by its authors.\footnote{Available at: \url{https://github.com/yiyihum/da-code}} For DS-Bench, we use the original evaluation method, in which an LLM serves as the judge. The generated programs output is compared against the reference answer using Gemini-2.5-Pro as the judge LLM, with the evaluation prompt shown in Figure \ref{fig:eval-ds-bench} in Appendix \ref{app:dataset}, producing a binary score.

\paragraph{Inference Setup:} We set the maximum actions of the main agent to $T = 10$. We use nucleus sampling \citep{Holtzman2020The} with a temperature of $0.1$ for more deterministic inference and default value for other hyperparameters. Proprietary models are accessed via Vertex AI,\footnote{Available at: \url{https://cloud.google.com/vertex-ai?hl=en}} while open-source models are served by vLLM.\footnote{Available at: \url{https://docs.vllm.ai/en/latest/}} At each step, we cap the number of generated tokens at 8,192. We use Gemini-2.5-Pro and -Flash \citep{comanici2025gemini25pushingfrontier}, and Claude-4-Opus \citep{anthropic2025claude4} as the proprietary and Qwen3-Coder\footnote{Available at: \url{https://huggingface.co/Qwen/Qwen3-Coder-30B-A3B-Instruct}} with 30 billion parameters \citep{qwen3technicalreport} as the open-source LLMs. Experiments are conducted on 2 NVIDIA A100 (each with 80GB VRAM) GPUs.

% \begin{table}[]
%     \centering
%     \caption{Results of our method and the baselines on the KramaBench, DS-Bench, and DA-Code benchmarks. The best results are highlighted in \textbf{bold}.}
%     \label{tab:main-result}
%     \adjustbox{max width=\textwidth}{
%     \begin{tabular}{ll|l|cccccc|c|c|c}
%         \toprule
%         \multirow{2}{*}{\textbf{Method}} & & \multirow{2}{*}{\textbf{LLM}} & \multicolumn{6}{|c|}{\textbf{KramaBench}} & \multirow{2}{*}{\textbf{DS-Bench}} & \multirow{2}{*}{\textbf{DA-Code}} & {\textbf{Average}} \\
%         \cmidrule{4-9}
%         & & & Archaeology & Astronomy & Biomedical & Environment & Legal & Wildfire & & & (macro) \\
%         \midrule
        
%         \multirow{4}{*}{DS-GRU} & (1) & Qwen3-Coder & 0.00\% & 1.80\% & 2.11\% & 1.15\% & 3.27\% & 13.54\% & 0.00\% & 0.00\% & 2.73\% \\
        
%         & (2) & Gemini 2.5 Flash & 0.00\% & 7.83\% & 0.09\% & 10.93\% & 12.46\% & 13.34\% & 5.53\% & 0.00\% & 6.38\% \\
        
%         & (3) & Gemini 2.5 Pro & 25.00\% & 6.69\% & 10.64\% & 27.47\% & 5.94\% & 39.36\% & 3.95\% & 0.00\% & 14.88\% \\
%         & (4) & Claude 4 Opus & 8.33\% & 1.38\% & 1.90\% & 8.14\% & 9.80\% & 23.14\% & 3.55\% & 0.00\% & 7.03\% \\
%         \midrule
        
%         \multirow{4}{*}{RAG} & (5) & Qwen3-Coder & 0.00\% & 3.16\% & 4.99\% & 0.54\% & 6.19\% & 16.93\% & 6.32\% & 0.00\% & 4.76\% \\
        
%         & (6) & Gemini 2.5 Flash & 16.66\% & 3.57\% & 13.98\% & 28.57\% & 10.97\% & 33.67\% & 22.92\% & 2.75\% & 17.44\% \\
        
%         & (7) & Gemini 2.5 Pro & \textbf{33.33\%} & 8.47\% & 32.53\% & 31.36\% & 25.55\% & 38.32\% & 27.27\% & 0.00\% & 25.32\% \\
        
%         & (8) & Claude 4 Opus & \textbf{33.33\%} & 11.52\% & 23.42\% & 31.61\% & 31.80\% & 45.80\% & 35.57\% & 3.85\% & 27.11\% \\
%         \midrule
        
%         \multirow{4}{*}{Master-Slave} & (9) & Qwen3-Coder & 0.00\% & 3.55\% & 3.39\% & 7.77\% & 8.90\% & 21.79\% & 7.55\% & 0.00\% & 6.61\% \\
        
%         & (10) & Gemini 2.5 Flash & 16.66\% & 3.16\% & 13.98\% & 17.46\% & 21.75\% & 25.80\% & 26.48\% & 0.55\% & 14.87\% \\
        
        
%         & (11) & Gemini 2.5 Pro & \textbf{33.33\%} & 8.47\% & 24.74\% & 32.81\% & 34.64\% & 58.98\% & 34.38\% & 5.49\% & 29.10\% \\
        
%         & (12) & Claude 4 Opus & \textbf{33.33\%} & 8.69\% & 32.28\% & 39.16\% & \textbf{44.08\%} & 48.35\% & 45.84\% & 2.75\% & 31.81\% \\
%         \midrule
        
%         \multirow{4}{*}{Blackboard} & (13) & Qwen3-Coder & 0.00\% & 7.69\% & 7.85\% & 4.47\% & 6.36\% & 23.97\% & 14.22\% & 1.11\% & 8.47\% \\
        
%         & (14) & Gemini 2.5 Flash & 16.66\% & 3.57\% & 14.78\% & 22.92\% & 27.09\% & 41.04\% & 28.06\% & 0.55\% & 18.22\% \\
        
%         & (15) & Gemini 2.5 Pro & \textbf{33.33\%} & 17.95\% & 36.83\% & \textbf{39.31\%} & 34.92\% & \textbf{62.88\%} & 38.73\% & \textbf{9.34\%} & 34.16\% \\
        
%         & (16) & Claude 4 Opus & \textbf{\textbf{33.33\%}} & \textbf{18.69\%} & \textbf{45.31\%} & 34.35\% & 42.48\% & 50.06\% & \textbf{49.80\%} & 7.14\% & \textbf{35.14\%} \\
%         \bottomrule
%     \end{tabular}}
% \end{table}



\paragraph{Baselines:} To evaluate our method against alternative approaches for solving data science problems involving data discovery, we compare it with the following baselines:
\begin{itemize}[leftmargin=*]
    \item \textit{\textbf{DS-GRU:}} We adopt the only existing baseline (to the best of our knowledge) for data discovery in data science problems, which appends all available files directly into the LLM prompt and attempts to solve the problem \citep{lai2025kramabenchbenchmarkaisystems}. This baseline uses a self-correction loop that retries when errors occur in generated codes. For details, we refer the reader to \citet{lai2025kramabenchbenchmarkaisystems}.
    
    \item \textit{\textbf{Retrieval-Augmented Generation (RAG):}} This retrieves the top 5 files\footnote{This number is chosen based on the average number of files required to solve the problems (1.6) and the length of the context window of the backbone LLMs used in this paper.} based on the file names and contents (the method for presenting a file content to the LLM is explained in Appendix~\ref{app:implementation}) from the data lake using E5-large\footnote{Available at: \url{https://huggingface.co/intfloat/e5-large-v2}} \citep{wang2022text}, a 330M-parameter embedding model and use it to solve the problem. It then follows the same procedure as the main agent described in Section~\ref{sec:main-agent}, with two key modification: 1) the retrieved files contents and addresses are presented directly to the LLM in the prompt and 2) the general help-request action is replaced with a restricted action that only allows direct requests to the search agent. This design isolates the effect of substituting the file discovery mechanism with RAG, enabling a controlled study of its impact on performance. The prompt used for this baseline is shown in Figure~\ref{fig:main-agent-rag-prompt} in Appendix~\ref{app:prompts}.
    
    \item \textit{\textbf{Master-Slave:}} This baseline follows the same procedure as the main agent described in Section~\ref{sec:main-agent}. The key difference is that, instead of posting requests on the blackboard, the agent directly invokes sub-agents (consisting of the search agent and the file agents as explained in Section~\ref{sec:sub-agents}) based on their description by referencing their names and assign task to them. The prompt used for this baseline is shown in Figure~\ref{fig:main-agent-master-slave-prompt} in Appendix~\ref{app:prompts}.
\end{itemize} 


% \begin{table}[]
%     \centering
%     \caption{Results of our method and the baselines on the KramaBench, DS-Bench, and DA-Code benchmarks. The best results for each backbone LLMs are highlighted in \textbf{bold}.}
%     \label{tab:main-result}
%     \adjustbox{max width=\textwidth}{
%     \begin{tabular}{ll|c|cccccc|c|c|c|c}
%         \toprule
%         & \multirow{2}{*}{\textbf{Method}}& \multirow{2}{*}{\textbf{LLM}} & \multicolumn{7}{|c|}{\textbf{KramaBench}} & \multirow{2}{*}{\textbf{DS-Bench}} & \multirow{2}{*}{\textbf{DA-Code}} & {\textbf{Average}} \\
%         \cmidrule{4-10}
%         & & & Archaeology & Astronomy & Biomedical & Environment & Legal & Wildfire & Average & & & (macro) \\
%         \midrule
        
%         (1) & {DS-GRU} & \multirow{4}{*}{Qwen3-Coder} & \textbf{0.00\%} & 1.80\% & 2.11\% & 1.15\% & 3.27\% & 13.54\% & 3.64\% & 0.00\% & 0.00\% & 1.21\% \\
        
%         (2) & {RAG} &  & \textbf{0.00\%} & 3.16\% & 4.99\% & 0.54\% & 6.19\% & 16.93\% & 5.30\% & 6.32\% & 0.00\% & 3.87\% \\
        
%         (3) & {Master-Slave} &  & \textbf{0.00\%} & 3.55\% & 3.39\% & \textbf{7.77\%} & \textbf{8.90\%} & 21.79\% & 7.56\% & 7.55\% & 0.00\% & 5.03\% \\
        
%         \cmidrule{1-2} \cmidrule{4-13}
        
%         (4) & {Blackboard} & & \textbf{0.00\%} & \textbf{7.69\%} & \textbf{7.85\%} & 4.47\% & 6.36\% & \textbf{23.97\%} & \textbf{8.39\%} & \textbf{14.22\%} & \textbf{1.11\%} & \textbf{7.90\%} \\
        
%         \midrule
%         \midrule
        
%         (5) & {DS-GRU} & \multirow{4}{*}{Gemini 2.5 Flash} & 0.00\% & \textbf{7.83\%} & 0.09\% & 10.93\% & 12.46\% & 13.34\% & 7.44\% & 5.53\% & 0.00\% & 4.32\% \\
        
%         (6) & {RAG} & & \textbf{16.66\%} & 3.57\% & 13.98\% & \textbf{28.57\%} & 10.97\% & 33.67\% & 17.90\% & 22.92\% & \textbf{2.75\%} & 14.52\% \\
        
%         (7) & {Master-Slave} & & \textbf{16.66\%} & 3.16\% & 13.98\% & 17.46\% & 21.75\% & 25.80\% & 16.46\% & 26.48\% & 0.55\% & 14.49\% \\
        
%         \cmidrule{1-2} \cmidrule{4-13}
        
%         (8) & {Blackboard} & & \textbf{16.66\%} & 3.57\% & \textbf{14.78\%} & 22.92\% & \textbf{27.09\%} & \textbf{41.04\%} & \textbf{21.01\%} & \textbf{28.06\%} & 0.55\% & \textbf{16.54\%} \\
        
%         \midrule
%         \midrule
        
%         (9) & {DS-GRU} & \multirow{4}{*}{Gemini 2.5 Pro} & 25.00\% & 6.69\% & 10.64\% & 27.47\% & 5.94\% & 39.36\% & 19.18\% & 3.95\% & 0.00\% & 7.71\% \\
        
%         (10) & {RAG} & & \textbf{33.33\%} & 8.47\% & 32.53\% & 31.36\% & 25.55\% & 38.32\% & 28.26\% & 27.27\% & 0.00\% & 18.51\% \\
        
%         (11) & {Master-Slave} & & \textbf{33.33\%} & 8.47\% & 24.74\% & 32.81\% & 34.64\% & 58.98\% & 32.16\% & 34.38\% & 5.49\% & 24.01\% \\
        
%         \cmidrule{1-2} \cmidrule{4-13}
        
%         (12) & {Blackboard} & & \textbf{33.33\%} & \textbf{17.95\%} & \textbf{36.83\%} & \textbf{39.31\%} & \textbf{34.92\%} & \textbf{62.88\%} & \textbf{37.53\%} & \textbf{38.73\%} & \textbf{9.34\%} & \textbf{28.53\%} \\
        
%         \midrule
%         \midrule
        
%         (13) & {DS-GRU} & \multirow{4}{*}{Claude 4 Opus} & 8.33\% & 1.38\% & 1.90\% & 8.14\% & 9.80\% & 23.14\% & 8.78\% & 3.55\% & 0.00\% & 4.11\% \\
        
        
%         (14) & {RAG} &  & \textbf{33.33\%} & 11.52\% & 23.42\% & 31.61\% & 31.80\% & 45.80\% & 29.58\% & 35.57\% & 3.85\% & 23.00\% \\


%         (15) & {Master-Slave} & & \textbf{33.33\%} & 8.69\% & 32.28\% & \textbf{39.16\%} & \textbf{44.08\%} & 48.35\% & 34.31\% & 45.84\% & 2.75\% & 27.63\% \\

%         \cmidrule{1-2} \cmidrule{4-13}

%         (16) & {Blackboard} &  & \textbf{\textbf{33.33\%}} & \textbf{18.69\%} & \textbf{45.31\%} & 34.35\% & 42.48\% & \textbf{50.06\%} & \textbf{37.37\%} & \textbf{49.80\%} & \textbf{7.14\%} & \textbf{31.43\%} \\
%         \bottomrule
%     \end{tabular}}
% \end{table}

\begin{table*}
    \centering
    \caption{Results on the KramaBench, DS-Bench, and DA-Code benchmarks. The best results for each LLM are highlighted in \textbf{bold}. The KramaBench categories are abbreviated: Arc. (Archaeology), Ast. (Astronomy), Bio. (Biomedical), Env. (Environment), Leg. (Legal), and Wild. (Wildfire).}
    \label{tab:main-result}
    \adjustbox{max width=\textwidth}{
    \begin{tabular}{ll c cccccc c c c c}
        \toprule
        & \multirow{2}{*}{\textbf{Method}}& \multirow{2}{*}{\textbf{LLM}} & \multicolumn{7}{c}{\multirow{1}{*}{\textbf{KramaBench}}} & \multirow{2}{*}{\textbf{\makecell{DS-\\Bench}}} & \multirow{2}{*}{\textbf{\makecell{DA-\\Code}}} & \multirow{2}{*}{\textbf{\makecell{Average \\ (macro)}}} \\
        \cmidrule{4-10}
        & & & Arc. & Ast. & Bio. & Env. & Leg. & Wild. & Average & & & \\
        \midrule
        
        (1) & {DS-GRU} & \multirow{4}{*}{\makecell{Qwen3-\\Coder}} & \textbf{0.00\%} & 1.80\% & 2.11\% & 1.15\% & 3.27\% & 13.54\% & 3.64\% & 0.00\% & 0.00\% & 1.21\% \\
        
        (2) & {RAG} &  & \textbf{0.00\%} & 3.16\% & 4.99\% & 0.54\% & 6.19\% & 16.93\% & 5.30\% & 6.32\% & 0.00\% & 3.87\% \\
        
        (3) & {Master-Slave} &  & \textbf{0.00\%} & 3.55\% & 3.39\% & \textbf{7.77\%} & \textbf{8.90\%} & 21.79\% & 7.56\% & 7.55\% & 0.00\% & 5.03\% \\
        
        \cmidrule{1-2} \cmidrule{4-13}
        
        (4) & {Blackboard} & & \textbf{0.00\%} & \textbf{7.69\%} & \textbf{7.85\%} & 4.47\% & 6.36\% & \textbf{23.97\%} & \textbf{8.39\%} & \textbf{14.22\%} & \textbf{1.11\%} & \textbf{7.90\%} \\
        
        \midrule
        \midrule
        
        (5) & {DS-GRU} & \multirow{4}{*}{\makecell{Gemini 2.5\\Flash}} & 0.00\% & \textbf{7.83\%} & 0.09\% & 10.93\% & 12.46\% & 13.34\% & 7.44\% & 5.53\% & 0.00\% & 4.32\% \\
        
        (6) & {RAG} & & \textbf{16.66\%} & 3.57\% & 13.98\% & \textbf{28.57\%} & 10.97\% & 33.67\% & 17.90\% & 22.92\% & \textbf{2.75\%} & 14.52\% \\
        
        (7) & {Master-Slave} & & \textbf{16.66\%} & 3.16\% & 13.98\% & 17.46\% & 21.75\% & 25.80\% & 16.46\% & 26.48\% & 0.55\% & 14.49\% \\
        
        \cmidrule{1-2} \cmidrule{4-13}
        
        (8) & {Blackboard} & & \textbf{16.66\%} & 3.57\% & \textbf{14.78\%} & 22.92\% & \textbf{27.09\%} & \textbf{41.04\%} & \textbf{21.01\%} & \textbf{28.06\%} & 0.55\% & \textbf{16.54\%} \\
        
        \midrule
        \midrule
        
        (9) & {DS-GRU} & \multirow{4}{*}{\makecell{Gemini 2.5\\Pro}} & 25.00\% & 6.69\% & 10.64\% & 27.47\% & 5.94\% & 39.36\% & 19.18\% & 3.95\% & 0.00\% & 7.71\% \\
        
        (10) & {RAG} & & \textbf{33.33\%} & 8.47\% & 32.53\% & 31.36\% & 25.55\% & 38.32\% & 28.26\% & 27.27\% & 0.00\% & 18.51\% \\
        
        (11) & {Master-Slave} & & \textbf{33.33\%} & 8.47\% & 24.74\% & 32.81\% & 34.64\% & 58.98\% & 32.16\% & 34.38\% & 5.49\% & 24.01\% \\
        
        \cmidrule{1-2} \cmidrule{4-13}
        
        (12) & {Blackboard} & & \textbf{33.33\%} & \textbf{17.95\%} & \textbf{36.83\%} & \textbf{39.31\%} & \textbf{34.92\%} & \textbf{62.88\%} & \textbf{37.53\%} & \textbf{38.73\%} & \textbf{9.34\%} & \textbf{28.53\%} \\
        
        \midrule
        \midrule
        
        (13) & {DS-GRU} & \multirow{4}{*}{\makecell{Claude 4\\Opus}} & 8.33\% & 1.38\% & 1.90\% & 8.14\% & 9.80\% & 23.14\% & 8.78\% & 3.55\% & 0.00\% & 4.11\% \\
        
        
        (14) & {RAG} &  & \textbf{33.33\%} & 11.52\% & 23.42\% & 31.61\% & 31.80\% & 45.80\% & 29.58\% & 35.57\% & 3.85\% & 23.00\% \\


        (15) & {Master-Slave} & & \textbf{33.33\%} & 8.69\% & 32.28\% & \textbf{39.16\%} & \textbf{44.08\%} & 48.35\% & 34.31\% & 45.84\% & 2.75\% & 27.63\% \\

        \cmidrule{1-2} \cmidrule{4-13}

        (16) & {Blackboard} &  & \textbf{33.33\%} & \textbf{18.69\%} & \textbf{45.31\%} & 34.35\% & 42.48\% & \textbf{50.06\%} & \textbf{37.37\%} & \textbf{49.80\%} & \textbf{7.14\%} & \textbf{31.43\%} \\
        \bottomrule
    \end{tabular}}
    % \vspace{-0.4cm}
\end{table*}


\subsection{Empirical Findings}

\paragraph{Main Results:} 


We conduct our experiments on the datasets described in Section~\ref{sec:exp-setup} using our method and the baselines. The results are presented in Table~\ref{tab:main-result}. These results demonstrate that our method, the Blackboard System, outperforms all baselines on average across all the datasets. Specifically, the Blackboard System surpasses the DS-GRU, RAG and Master-Slave approaches on all three datasets and achieves similar or higher performance in 4 out of 6 categories on KramaBench. Furthermore, we observe that the Blackboard System consistently outperforms the baselines regardless of the backbone LLM, highlighting its robustness and generalizability. We attribute this improvement to the design of the Blackboard System, where tasks are not explicitly assigned to helper agents; instead, each agent autonomously decides whether to participate based on its capabilities. This self-selection enhances both problem-solving efficiency and data discovery performance.




\paragraph{File Discovery Performance:}

To analyze the effectiveness of different methods in data discovery, we report recall, precision, and F1-score for the file discovery task, i.e., identifying the correct files required to answer each question. The results of this experiment, using Gemini 2.5 Pro as the backbone LLM, are presented in Table~\ref{tab:result-file-discovery}. The results in this table indicate that the blackboard system achieves the highest recall, precision, and F1-score compared to all baselines, both on average and across the three datasets. In particular, for KramaBench, the blackboard system attains the highest F1-score in 4 out of 6 domains. We attribute this improvement to the design of the blackboard system, where the main agent does not directly assign requests to specific file agents, as in the master–slave setup. Instead, each file agent independently decides whether it can contribute based on its capabilities and data holdings, leading to more accurate and comprehensive file discovery.




\paragraph{Effect of Web Search (Search Agent) on the Performance:}

We observed that in some cases the backbone LLM lacks the necessary domain-specific knowledge or familiarity with specialized algorithms to fully understand and solve the problem. To address this limitation, the inclusion of a search agent that can retrieve relevant external information may be beneficial. To evaluate this, we compare the blackboard system with and without the search agent. The results on KramaBench, shown in Figure~\ref{fig:search-wo-search} using Gemini 2.5 Pro as the backbone LLM, demonstrate that incorporating the search agent improves the average performance of the blackboard system. Further analysis reveals that when the main agent encounters unfamiliar concepts, it issues requests to obtain such information from the web. In these cases, the search agent typically responds by retrieving the required knowledge, thereby enabling the main agent to continue solving the problem effectively. Illustrative examples of this behavior are provided in Figures~\ref{fig:search-example-1} and \ref{fig:search-example-2} in Appendix~\ref{app:case-study}, highlighting the importance of the search agent in scenarios where external domain knowledge is essential.

\begin{table}[]
    \centering
    \caption{File discovery performance reported using recall, precision, and f1-score. The results are obtained using Gemini 2.5 Pro as the backbone LLM. The best results are highlighted in \textbf{bold}.}
    \label{tab:result-file-discovery}
    \adjustbox{max width=\textwidth}{
    \begin{tabular}{ll l cccccc c c c c}
        \toprule
        &\multirow{2}{*}{\textbf{Method}} & \multirow{2}{*}{\textbf{Metric}} & \multicolumn{7}{c}{\textbf{KramaBench}} & \multirow{2}{*}{\textbf{DS-Bench}} & \multirow{2}{*}{\textbf{DA-Code}} & {\textbf{Average}} \\
        \cmidrule{4-10}
        & & & Archaeology & Astronomy & Biomedical & Environment & Legal & Wildfire & Average & & & (macro) \\
        \midrule
        \multirow{3}{*}{(1)} & \multirow{3}{*}{RAG } & recall & 0.875 & 0.125 & \textbf{0.666} & 0.3506 & 0.127 & 0.238 & 0.396 & 0.035 & 0.257 & 0.229 \\
        & & precision & \textbf{1.000} & 0.125 & 0.666 & 0.450 & 0.133 & 0.452 & 0.471 & 0.047 & 0.456 & 0.324  \\
        & & F1 & 0.916 & 0.125 & 0.629 & 0.332 & 0.105 & 0.301 & 0.401 & 0.034 & 0.307 & 0.247 \\
        \midrule
        \multirow{3}{*}{(2)} & \multirow{3}{*}{Master-Slave} & recall & \textbf{0.916} & 0.5138 & 0.648 & {0.382} & \textbf{0.444} & \textbf{0.567} & 0.578 & 0.323 & 0.546 & 0.482 \\
        & & precision & 0.930 & \textbf{0.750} & \textbf{0.722} & {0.500} & \textbf{0.494} & \textbf{0.642} & 0.673 & 0.503 & 0.767 & 0.647 \\
        & & F1 & 0.913 & 0.577 & \textbf{0.674} & 0.389 & \textbf{0.450} & \textbf{0.576} & 0.596 & 0.358 & 0.584 & 0.513 \\
        \midrule
        \multirow{3}{*}{(3)} & \multirow{3}{*}{Blackboard} & recall & \textbf{0.916} & \textbf{0.576} & 0.648 & \textbf{0.604} & 0.383 & 0.464 & \textbf{0.598} & \textbf{0.402} & \textbf{0.600} & \textbf{0.533} \\
        & & precision & \textbf{1.000} & 0.733 & \textbf{0.722} & \textbf{0.703} & 0.302 & 0.603 & \textbf{0.677} & \textbf{0.584} & \textbf{0.837} & \textbf{0.699} \\
        & & F1 & \textbf{0.944} & \textbf{0.618} & \textbf{0.674} & \textbf{0.588} & 0.304 & 0.495 & \textbf{0.603} & \textbf{0.438} & \textbf{0.643} & \textbf{0.561} \\
        \bottomrule
    \end{tabular}}
\end{table}


\begin{figure}
    \centering
    \includegraphics[width=\textwidth]{figs/search_without_search_compare.png}
    % \vspace{-0.4cm}
    % \centering
    \caption{Performance of Blackboard System w/ and w/o search agent (Gemini 2.5 Pro).}
    % \vspace{-0.4cm}
    \label{fig:search-wo-search}
\end{figure}


\paragraph{Effect of Number of Main Agent's Actions on the Performance:}

To examine the impact of the maximum number of actions available to the main agent, we vary this parameter across ${2, 4, 6, 8, 10}$ and evaluate the blackboard system on KramaBench using Gemini 2.5 Pro as the backbone LLM. The results, presented in Figure~\ref{fig:num-actions}, indicate that increasing the action budget consistently improves the average performance of the system. This trend aligns with intuition: a larger exploration budget allows the agent to more thoroughly analyze the problem, consider alternative strategies, better investigate the solution space, and generate a better program that answers the question.

\begin{figure}
    \centering
    \includegraphics[width=\textwidth]{figs/num_actions_performance.png}
    % \vspace{-0.4cm}
    \caption{Performance of Blackboard System with various maximum actions by the main agent.}
    \label{fig:num-actions}
    % \vspace{-0.5cm}
\end{figure}

\paragraph{Case Studies:}

To qualitatively analyze the blackboard system---specifically how it formulates requests and how this process improves the generated program---we present several case studies:
\begin{itemize}[leftmargin=*]

\item \textbf{Writing Request on the blackboard:} An example of a request posted by on the blackboard is shown in Figure~\ref{fig:request-example} in Appendix~\ref{app:case-study}. In this case, the main agent, given the data science question, formulates a request that specifies the likely column names and data formats needed to solve the problem, along with some guidance for interpretation. In response, several helper agents (3 out of 8 in this example) chose to contribute. Although the relevant files were distributed across different clusters managed by different file agents, each responding agent independently provided the file addresses, code snippets for loading the data, and explanations of the data structure along with suggested preprocessing steps. Collectively, these responses covered all the ground-truth files required to answer the question. This case study demonstrates how the main agent can effectively leverage the blackboard mechanism to discover and integrate necessary information.

\item \textbf{Comparing Generated Program by Blackboard System with Master-Slave System:} To study this further, we present an example of programs generated by the Blackboard system and the Master–Slave system in Figure~\ref{fig:program-example} in Appendix~\ref{app:case-study}. In this case, the Blackboard agent achieved a better solution because it accurately interpreted the prompt and selected the correct data files. Specifically, it identified that the patients \texttt{Age} was located in the \texttt{mmc1.xlsx} file and, more importantly, that the requested \texttt{APP-Z score} was in the \texttt{mmc7.xlsx} file. In contrast, the Master–Slave agent misinterpreted the request and instead used a general protein abundance score (\texttt{APP\_log2\_abundance}) from the wrong file, \texttt{mmc2.xlsx}. This critical error in data selection led the Master–Slave agent to produce an incorrect result of \texttt{74}, while the Blackboard agents precise data discovery and reasoning yielded the correct answer of \texttt{60}.

\end{itemize}
\section{Related Work}

\noindent{\textbf{Temporal Logic for video and audio understanding. }}
\citet{yang2023specification} and \citet{Choi2024TowardsNV} (NSVS-TL) use the probabilistic model checker STORM to verify temporal properties over object detections in videos, using LTL and PCTL to represent properties respectively. 

\noindent{\textbf{Benchmarks for video and audio understanding. }}
Benchmarks for video understanding such as Video-MME~\citep{videomme}, RexTIME~\citep{chen2024rextime}, Next-qa~\citep{nextqa}, QVHighlights~\citep{qvhighlights}, TemporalBench~\citep{cai2024temporalbench} and TempCompass~\citep{liu2024tempcompass} include tasks that require temporal understanding of events in videos. Similarly, audio understanding datasets such as MMAU~\citep{sakshi2024mmau} and CompA~\citep{ghosh2023compa} evaluate temporal tasks such as detecting the order of two events. 
These tasks are fundamentally different from the QMTP benchmark which focuses on more fine-grained temporal properties. 

\noindent{\textbf{Video retrieval with temporal queries. }} 
Popular text-to-video retrieval datasets such as 
Activity Net Captions~\citep{krishna2017dense} and DiDeMo~\citep{didemo} focus on temporal segments within minute-long videos.
Our TP2VR benchmark focuses on fine-grained temporal queries over short events in videos, with many-to-many mapping between queries and videos.

Popular text-video retrieval methods include CLIP4Clip~\citep{luo2021clip4clip}, TS2-Net~\citep{liu2022ts2},~\citep{Bain21}, which employ training to improve embeddings for retrieval, and zero-shot methods such as mPLUG~\citep{li2022mplug} and ELIOT~\citep{liu-etal-2025-eliot}. Since we use off-the-shelf models with LogSTOP for retrieval, we only include the latter for comparison.
\section{Conclusion}

In this work, we presented a full-stack investigation of LLM unlearning, encompassing methodology, evaluation, and robustness. We established a principled taxonomy that organizes twelve representative unlearning methods into three families: {\MDiv}, {\MRep}, and {\MRej}, providing a systematic lens to understand their underlying mechanisms. Our analysis revealed that conventional multiple-choice questioning (MCQ) evaluations of unlearning effectiveness (UE) and utility retention (UT) offer an incomplete picture, and we introduced open question answering (Open-QA) as a complementary paradigm to better capture generative behaviors and expose the strengths and limitations of different methods. Furthermore, we provide a comprehensive robustness assessment across model-level and input-level attacks, revealing nuanced relationships among in-domain relearning, out-of-domain fine-tuning, quantization, and jailbreak attacks. These findings clarify the trade-offs of current unlearning algorithms and guide the design of future methods that are both effective and robust. The use of LLM, limitation and broader impact are further discussed in \textbf{Appendix\,\ref{appx:llm_usage}}, \textbf{Appendix\,\ref{appx:limit}} and \textbf{Appendix\,\ref{appx:impact}}.


\newpage

\bibliographystyle{abbrvnat}
\nobibliography*
\bibliography{ref}

\newpage
\appendix
\renewcommand{\partname}{}
\renewcommand{\thepart}{}
\addcontentsline{toc}{section}{Appendix} 
\part{Appendix}
\parttoc
\newpage
\let\clearpage\relax
\clearpage
\section{Appendix}
\subsection{Document Ingestion}
\label{app:ingestion}
To better situate \mmore{} within the ecosystem of document ingestion systems, Table~\ref{tab:ingestion_comparison_expanded} presents a fine-grained comparison with two representative alternatives: \textit{Docling} and \textit{NV-Ingest} (part of NeMo Retriever). We evaluate them across modality support, indexing capabilities, and RAG integration. Green cells indicate native support, while grey cells denote the absence of the corresponding capability.

\begin{table}[h]
    \centering
    \small
    \resizebox{0.5\textwidth}{!}{%
    \begin{tabular}{lccc}
        \toprule
        \textbf{Feature} & \textbf{Docling} & \textbf{NV-Ingest\tablefootnote{\href{https://docs.nvidia.com/nemo/retriever/extraction/overview/}{NeMo Retriever Documentation}}} & \textbf{MMORE} \\
        \midrule
        % ---------------------- SUPPORTED MODALITIES ----------------------
        \rowcolor{gray!20}
        \multicolumn{4}{l}{\textit{Supported Modalities}} \\
        \midrule
        PDF & \greencell & \greencell & \greencell \\
        DOCX & \greencell & \greencell & \greencell \\
        PPTX & \greencell & \greencell & \greencell \\
        XLSX / spreadsheets & \greencell & \greycell & \greencell \\
        TXT & \greencell & \greencell & \greencell \\
        HTML & \greencell & \greycell & \greencell \\
        Markdown & \greencell & \greycell & \greencell \\
        CSV & \greencell & \greycell & \greencell \\
        Images (PNG/JPEG/SVG/TIFF/BMP) & \greencell & \greencell & \greencell \\
        Audio & \greycell & \greycell & \greencell \\
        Video & \greycell & \greycell & \greencell \\
        EML & \greycell & \greycell & \greencell \\
        \midrule
        % ---------------------- INDEXING & EMBEDDING ----------------------
        \rowcolor{gray!20}
        \multicolumn{4}{l}{\textit{Indexing \& Embedding}} \\
        \midrule
        Native engine included & \greycell & \greencell & \greencell \\
        LangChain / LlamaIndex connector & \greencell & \greencell & \greencell \\
        \midrule
        % ---------------------- RAG INTEGRATION ----------------------
        \rowcolor{gray!20}
        \multicolumn{4}{l}{\textit{RAG}} \\
        \midrule
        Built--in RAG pipeline & \greycell & \greycell & \greencell \\
        Plugin--based RAG & \greencell & \greencell & \greycell \\
        \midrule
        % ---------------------- LICENSE ----------------------
        Open--Source license & MIT & Apache~2.0 & Apache~2.0 \\
        \bottomrule
    \end{tabular}%
    }
    \caption{Fine-grained comparison of Docling, NV-Ingest, and MMORE document-ingestion pipelines. Green cells indicate native support; grey cells indicate absence of the capability.}
    \label{tab:ingestion_comparison_expanded}
\end{table}


\mmore{} supports a wide range of file formats through modular extractors. For each supported type, we define a \textit{default mode} prioritizing accuracy and a \textit{fast mode} optimized for speed. When no alternative tool is available, the fast mode is left unspecified (--). A complete list of tools used per file type is shown in Table \ref{tab:ingestion_tools}.

\begin{table*}[h]
\centering
\small
\resizebox{0.99\textwidth}{!}{%
\begin{tabular}{l l l}
\toprule
\textbf{File Type} & \textbf{Default Mode Tool(s)} & \textbf{Fast Mode Tool(s)} \\
\midrule
\rowcolor{gray!5}
DOCX & \texttt{python-docx} for text and image extraction & -- \\
MD & \texttt{markdown} for text, \texttt{markdownify} for HTML conversion & -- \\
\rowcolor{gray!5}
PPTX & \texttt{python-pptx} for text and image extraction & -- \\
XLSX & \texttt{openpyxl} for table and text extraction & -- \\
\rowcolor{gray!5}
TXT & Python built-in \texttt{open()} & -- \\
EML & Python built-in \texttt{email} module & -- \\
\rowcolor{gray!5}
Audio/Video (MP4, MP3, etc.) & \texttt{moviepy} for frames, \texttt{whisper-large-v3-turbo} for transcription & \texttt{whisper-tiny} \\
PDF & \texttt{marker-pdf} for OCR/structured data & \texttt{PyMuPDF} \\
\rowcolor{gray!5}
HTML & \texttt{BeautifulSoup} & -- \\
\bottomrule
\end{tabular}%
}
\caption{Overview of supported file types and extraction tools in \mmore{}. Full URLs are included in the project documentation.}
\label{tab:ingestion_tools}
\end{table*}


% \begin{table*}[t!]
% \centering
% \small
% \resizebox{\textwidth}{!}{%
% \begin{tabular}{lccc}
% \toprule
% \textbf{Feature} & \textbf{Docling} & \textbf{NV-Ingest\tablefootnote{\href{https://docs.nvidia.com/nemo/retriever/extraction/overview/}{NeMo Retriever Documentation}}} & \textbf{MMORE} \\
% \midrule
% Model Stack & Pydantic v2 + native PDF backend + ONNX/PyTorch AI models\tablefootnote{\label{fn:docling-github}See GitHub repo} 
%             & NIM microservices (NeMo Retriever modules)
%             & Modular Python processors + Milvus indexing + LangChain RAG \\
% \rowcolor{gray!10}
% Supported Modalities & PDF, DOCX, PPTX, XLSX, TXT, HTML, Markdown, CSV, images (PNG, JPEG, TIFF, BMP, SVG)\tablefootnote{\label{docling-github}GitHub: full format list}
%                      & PDF, DOCX, PPTX, TXT, images (JPEG, PNG, SVG, TIFF)
%                      & PDF, media (video/audio), spreadsheets, EML \\
% Throughput (best) & GPU: 2.08 pages/sec (L4); CPU: 1.35 pages/sec (M3 Max)\tablefootnote{\label{docling-technical-report}See technical report}
%                   & 12 pages/sec, 900 embeddings/sec (1× H100)
%                   & 3.89 pages/sec (4× A100) \\
% \rowcolor{gray!10}
% Indexing \& Embedding & Exports JSON/Markdown; connectors for LangChain/LlamaIndex; no native engine
%                       & Optional Milvus indexing via NeMo Retriever
%                       & Integrated Milvus + LangChain \\
% RAG Integration & Via LangChain/LlamaIndex plugins
%                 & LangChain via JSON output
%                 & Built-in LangChain interface \\
% \rowcolor{gray!10}
% Open-Source License & MIT
%                    & Apache 2.0
%                    & Apache 2.0 \\
% \bottomrule
% \end{tabular}%
% }
% \caption{Comparison of Docling, NV-Ingest, and MMORE Document Ingestion Pipelines.}
% \label{tab:ingestion_comparison}
% \end{table*}


\subsection{Multimodal Sample}

The format provides a standardized representation for processed documents, combining extracted text with references to non-text elements. As shown in the example, the "text" field contains the document's content with \texttt{<attachment>} placeholders (which are configurable) marking modality locations, while the modalities array contains all embedded objects with their types and storage paths.


\label{app:data_format}
\begin{tcolorbox}[
  enhanced,
  colback=black!3,          % Light gray background inside
  colframe=black!50,        % Medium gray border
  coltitle=white,           % White title text
  fonttitle=\bfseries,      % Bold title font
  title=Format Example:, % Box title
  colbacktitle=black!50,    % Dark gray title background
  boxrule=0.4mm,
  toptitle=1mm,
  bottomtitle=1mm
]

\begin{lstlisting}[
    basicstyle=\small\ttfamily,
    numbers=none,
    backgroundcolor=\color{black!3}
]
{
  "text": "A report containing a cool image <attachment> and a chart <attachment>...",
  "modalities": [
    {
      "type": "image",
      "value": "chart_url_2.png"
    },
    {
      "type": "image",
      "value": "chart_url_1.png"
    }
  ]
}
\end{lstlisting}

\noindent\rule{\linewidth}{0.4pt}

\small{\textit{The standardized format for document processing.}}


\end{tcolorbox}

\subsection{Processing Accuracy - Metrics}
\label{app:metrics}

To quantify extraction accuracy, we used a combination of machine translation, summarization and string similarity metrics. Their definitions are given below.

\textbf{BLEU Score (bilingual evaluation understudy)} \cite{bleuscore}:  
The BLEU score evaluates the overlap between the n-grams (sequences of words of length \(n\)) between the extracted text and the ground truth. It is defined as:

\begin{equation}  
\text{BLEU} = \text{BP} \cdot \exp \left( \sum_{n=1}^{N} w_n \log p_n \right)
\end{equation}


where \(p_n\) is the precision for n-grams of length \(n\), ranging from [1 to 4], \(w_n\) are the weights (uniform), and brevity penalty (\(\text{BP}\)), given by:

\begin{equation}  
\text{BP} = 
\begin{cases} 
1 & \text{if } c > r \\ 
\exp\left(1 - \frac{r}{c}\right) & \text{if } c \leq r
\end{cases}
\end{equation}

Here, \(c\) is the length of the candidate (extracted) text, and \(r\) is the length of the reference (ground truth). BLEU considers how much of the extracted text matches the reference text in terms of word sequences, while also penalizing outputs that are too short.

\textbf{ROUGE-L (recall-oriented understudy for gisting evaluation)} \cite{lin2004rouge}:  
ROUGE-L measures the quality of the extracted text using the longest common subsequence (LCS) between the extracted text and the ground truth. The LCS is the longest sequence of words appearing in the same order in both texts (though not necessarily consecutively). ROUGE-L is calculated as:

\begin{equation}  
\text{ROUGE-L} = F_\text{measure} = \frac{(1 + \beta^2) \cdot \text{Precision} \cdot \text{Recall}}{\beta^2 \cdot \text{Precision} + \text{Recall}}
\end{equation}

where \(\beta\) is a weighting factor (set to 1 for equal weighting), and:

\begin{equation}  
\begin{aligned}
\text{Precision} &= \frac{\text{LCS}}{\text{Length of Extracted Text}}, \\
\text{Recall} &= \frac{\text{LCS}}{\text{Length of Ground Truth}}.
\end{aligned}
\end{equation}

\textbf{Levenshtein distance - character error rate (CER)} \cite{levenshtein}:  
Given two strings, \( s_1 \) (extracted text) and \( s_2 \) (ground truth), the Levenshtein distance \( d(s_1, s_2) \) measures the minimum number of insertions, deletions, or substitutions required to transform \( s_1 \) into \( s_2 \). We normalize this distance over the length of the ground truth and is defined as:

\begin{equation}
\text{CER} = \frac{d(s_1, s_2)}{|s_1|}
\end{equation}
\end{document}

