\begin{abstract}



Referring Video Object Segmentation (RVOS) aims to segment the object referred to by the query sentence in the video. Most existing methods require end-to-end training with dense mask annotations, which could be computation-consuming and less scalable. In this work, we rethink the RVOS problem and aim to investigate the key to this task. Based on existing foundation segmentation models, we decompose the RVOS task into \textit{referring}, \textit{video}, and \textit{segmentation} factors, and propose a \textit{\namebf~(\textbf{\nameshort})} framework to address the referring and video factors while leaving the segmentation problem to foundation models. To efficiently adapt image-based foundation segmentation models to referring video object segmentation, we leverage off-the-shelf object detectors and trackers to produce temporal prompts associated with the referring sentence. While high-quality temporal prompts could be produced, they can not be easily identified from confidence scores. To tackle this issue, we propose \textit{\nameselect}~to evaluate the quality of the produced temporal prompts. By taking such prompts to instruct image-based foundation segmentation models, we would be able to produce high-quality masks for the referred object, enabling efficient model adaptation to referring video object segmentation. Experiments on RVOS benchmarks demonstrate the effectiveness of the \nameshort~framework.

\end{abstract}