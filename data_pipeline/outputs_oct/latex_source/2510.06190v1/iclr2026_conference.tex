\documentclass{article} % For LaTeX2e
\usepackage{iclr2026_conference,times}

% Optional math commands from https://github.com/goodfeli/dlbook_notation.
\input{math_commands.tex}

\usepackage[colorlinks,
            linkcolor=OliveGreen,
            anchorcolor=OliveGreen,
            citecolor=OliveGreen,
            urlcolor=OliveGreen
            ]{hyperref}
\usepackage{url}

% https://github.com/st--/annotate-equations
\usepackage{annotate-equations}
\usepackage[skins,listings,most]{tcolorbox}
\renewcommand{\eqnannotationfont}{\bfseries\small}

\usepackage{booktabs}       % professional-quality tables
\usepackage{amsfonts}       % blackboard math symbols
\usepackage{nicefrac}       % compact symbols for 1/2, etc.
\usepackage{microtype}      % microtypography
\usepackage[table,dvipsnames]{xcolor}
\usepackage{graphicx}
\usepackage{subfigure}
\usepackage{multirow}
\usepackage{threeparttable}
\usepackage{makecell}
\usepackage{floatrow}
\usepackage{colortbl}
\usepackage{amsmath,amscd,amsbsy,amsfonts,latexsym,url,bm,amsthm,amssymb,dsfont}
\usepackage{wrapfig}
% \usepackage[vlined,boxed,ruled]{algorithm2e}  % Commented out to avoid conflict with algorithm+algpseudocode
\usepackage{caption}
\usepackage{fvextra}
\usepackage{csquotes}
\usepackage{natbib}
% \setcitestyle{numbers,square}
\usepackage{enumitem}

\usepackage{cleveref}

\usepackage{listings}
\usepackage{mdframed}
\usepackage[colorinlistoftodos,prependcaption]{todonotes}
\usepackage{xargs}
\usepackage{BOONDOX-uprscr}
\usepackage{thmtools}
\usepackage{lipsum}    % 生成占位文字
\usepackage{algorithm}
\usepackage{algpseudocode}
\usepackage{algorithmicx}  % For better algorithm support
\usepackage{float}         % For better float control

% Define circled numbers with proper alignment
\newcommand{\circled}[1]{\raisebox{0.15ex}{\textbf{\textcircled{\raisebox{-0.15ex}{\small #1}}}}}


\newtheorem{theorem}{Theorem}
\newtheorem{proposition}[theorem]{Proposition}
\newtheorem{lemma}[theorem]{Lemma}
\newtheorem{corollary}[theorem]{Corollary}
\newtheorem{definition}{Definition}
\newtheorem{conjecture}[theorem]{Conjecture}
\newtheorem{example}{Example}
\newtheorem{assumption}{Assumption}


\newcommand{\red}[1]{\textcolor{BrickRed}{#1}}
\newcommand{\blue}[1]{\textcolor{NavyBlue}{#1}}
\newcommand{\purple}[1]{\textcolor{DarkOrchid}{#1}}
\newcommand{\green}[1]{\textcolor{OliveGreen}{#1}}
\newcommand{\yellow}[1]{\textcolor{Dandelion}{#1}}
\newcommand{\gray}[1]{\textcolor{gray}{#1}}

% \definecolor{gray}{HTML}{7A7A7A}
% \definecolor{red}{rgb}{1.        , 0.17254902, 0.1372549 }
% \definecolor{purple}{rgb}{0.68980392, 0.40803922, 0.70921569 }
% \definecolor{blue}{rgb}{0.05555556, 0.43398693, 1.   }

\usepackage{enumitem}
\setlist[itemize,1]{leftmargin=10pt, labelindent=5pt, itemsep=3pt, parsep=3pt}
\setlist[enumerate,1]{leftmargin=10pt, labelindent=5pt, itemsep=3pt, parsep=3pt}
\setlist[itemize,2]{leftmargin=15pt, itemsep=3pt, parsep=3pt}
\setlist[enumerate,2]{leftmargin=15pt, itemsep=3pt, parsep=3pt}

% Define macros for unmask and remask with bold and colors (math mode only)
\newcommand{\f}{f}
\newcommand{\g}{\phi}
\newcommand{\nxt}{\pi}


% Define macro for mask token
\newcommand{\mask}{\textsf{M}}

\newcommand{\unmask}{\purple{\textbf{unmask}}}
\newcommand{\remask}{\blue{\textbf{remask}}}
\newcommand{\delete}{\red{\textbf{delete}}}
\newcommand{\inser}{\yellow{\textbf{insert}}}


\usepackage[normalem]{ulem}
\useunder{\uline}{\ul}{}
\newcommand{\overbar}[1]{\mkern 1.3mu\overline{\mkern-1.3mu#1\mkern-1.3mu}\mkern 1.3mu}
\newcommand{\undebar}[1]{\mkern 1.3mu\underline{\mkern-1.3mu#1\mkern-1.3mu}\mkern 1.3mu}

\newcommand*{\ldblbrace}{\{\mskip-5mu\{}
\newcommand*{\rdblbrace}{\}\mskip-5mu\}}

\newcommand\scalemath[2]{\scalebox{#1}{\mbox{\ensuremath{\displaystyle #2}}}}

\newcommandx{\cmt}[2][1=Comment]{\vspace{2pt}\todo[inline,backgroundcolor=black!5]{\textit{(#1)} \;#2}}

\newcommand{\chenxiao}[1]{{\blue{[Chenxiao: ``#1'']}}}
\newcommand{\zhiyuan}[1]{{\purple{[Zhiyuan: ``#1'']}}}
\newcommand{\david}[1]{{\red{[David: ``#1'']}}}


\crefname{section}{$\mathsection$}{$\mathsection\mathsection$}
\Crefname{section}{$\mathsection$}{$\mathsection\mathsection$}

\title{On Powerful Ways to Generate: Autoregression, Diffusion, and Beyond}

% \title{On the Surprising Power to Generate Unnaturally}

% \title{Rethinking What are Natural Ways to Generate}

% \title{Generation to the Limit}

% \title{Unmasking Masked Diffusion: Towards Generation Processes beyond Natural Language}

% \title{In Search of A Universal Generation Process: Masked Diffusion and Beyond}

% \title{In Search of A Universal Generation Process: On Computational Power of Masked Diffusion / from Auto-Regressive to Masked Diffusion}

% \title{Should Programs Be Written Line by Line}

% \title{In Search of Universal Generation Process}

% \title{How Things Should Be Generated: A Theory of Masked Diffusion }

\iclrfinalcopy

\author{
  Chenxiao Yang$^\dagger$~~~~Cai Zhou$^\S$~~~~David Wipf~~~~Zhiyuan Li$^\dagger$\\
  $\dagger$ Toyota Technological Institute at Chicago~~~~$\S$ Massachusetts Institute of Technology\\
    \small\texttt{\{chenxiao,zhiyuanli\}@ttic.edu}
}


\newcommand{\fix}{\marginpar{FIX}}
\newcommand{\new}{\marginpar{NEW}}

\begin{document}

\maketitle

\begin{abstract}
This paper formally studies generation processes, including auto-regressive next-token prediction and masked diffusion, that abstract beyond architectural specifics. At this level of abstraction, we quantify their benefits and limitations through measurable criteria such as computational hardness and learnability. In particular, we demonstrate that allowing generation to proceed beyond autoregression and current masked diffusion, with capabilities to rewrite and length-variable edit, can bring significant theoretical and empirical advantages, with important implications for frontier LLMs that aspire to tackle increasingly hard problems and work universally across domains beyond natural language, such as coding and science.\footnote{Code is available at \url{https://github.com/chr26195/AP-MDM}.}
\end{abstract}



% \vspace{-5pt}
\section{Introduction}
% \vspace{-5pt}


% What power can being natural bring? 
The underlying generation process of almost everything in nature follows a unidirectional arrow of time. Perhaps most representative of all, spoken language is produced through a sequential process where each word builds upon preceding context in causal temporal order. This generic inductive bias has been encoded into \emph{Auto-Regressive Models (ARM)}~\citep{shannon1951prediction}, through next-token generation. Despite its simplicity, ARM when scaled through training on vast corpora, has produced remarkably powerful models capable of general-purpose task completion and reasoning, like GPT~\citep{radford2018improving,radford2019language,brown2020language,achiam2023gpt}. 

Yet reality might be more convoluted. Humans, when tackling challenging tasks, naturally undergo a non-sequential process of searching for solutions, evaluating and refining them, backtracking when needed, and iterating until answers are found. Such complexity is not fully captured by current ARM. While it is debatable if human intelligence fundamentally follows this left-to-right process~\citep{lecun2023large,malach2023auto,bachmann2024pitfalls,berglund2023reversal,nagarajan2025roll}, ARM appears increasingly ill-suited when we venture beyond natural language.

For example, code generation must subject to global constraints like balanced parentheses and well-typedness. Maintaining validity at each intermediate step makes transitions from one state to another easier, thus naturally involving updates such as inserting functions, adding branches, or changing input types. In biology, many domains remain largely beyond the reach of current LLMs, as molecular structures such as proteins and genes are combinatorial objects that can be modeled as graphs, trees, or strings that satisfy physical constraints. Their generation proceeds most naturally through structure-aware edits, e.g., swapping protein domains, inserting binding motifs into sequence graphs, or recombining DNA/RNA segments~\citep{wang2023scientific}.



% ~\citep{bommasani2022opportunitiesrisksfoundationmodels}
Given the long-standing pursuit of building foundation models powerful enough to handle increasingly complex reasoning tasks and general enough to work across diverse domains beyond natural language, it becomes important and timely to rethink generation process itself, as a mechanism separate from architectural specifics, by formally asking:

\vspace*{-0.1cm}
\begin{center}
\textit{How do we formally compare various ways to generate, and what opportunities may lie beyond next-token generation?}
\end{center}
\vspace*{-0.1cm}

Recent work suggests that next-token generation is not the only viable path. \emph{Masked Diffusion Models (MDM)}~\citep{hoogeboom2021argmax,austin2021structured,lou2024discrete,sahoo2024simple,shi2024simplified} offer a compelling alternative  procedure that, instead of causally generating tokens one by one, permits any-order generation and produces multiple tokens in parallel, with recent large-scale intantiations~\citep{deepmind2025gemini,inceptionlabs2025mercury,nie2025large,ye2025dream} showing comparable performance with AR-based LLMs. Interestingly, besides faster decoding (up to $10\times$ speedups), MDM's generation process brings empirical improvements on some order-sensitive tasks such as reversed-order poem completion~\citep{nie2025large,berglund2023reversal} and Sudoku puzzles~\citep{kim2025train,shah2024causal}. This motivates us to formally study it and compare with ARM.
% Therefore, we investigate whether such new generation processes bring advantages in terms of the scope of problems they enable solving. This seems to capture 

Perhaps counterintuitively, we find \textbf{while MDM is indeed more powerful than ARM in terms of parallelism and efficiency for simple tasks, the benefits of seemingly greater flexibility are surprisingly limited.} Like ARM~\citep{merrill2023expresssive,feng2024towards,li2024chain}, MDM also achieves Turing-completeness, but does so more efficiently with optimal parallel time complexity {(\Cref{thm:main_mdm})}, thus enabling \emph{exponential} speedups for simple parallelizable problems. However, for harder reasoning tasks, MDM faces similar fundamental limitations as ARM: both struggle with problems requiring backtracking and rewriting capabilities, and cannot handle them given realistic space resources {(\Cref{thm:main_constrained})}. Moreover, when controlling for other factors including degree of parallelism and architecture, any-order generation itself does not expand what ARM can already handle {(\Cref{thm:any_order})}, since any computation performed by MDM can be reorganized into left-to-right order to align with the underlying arrow of time. Therefore we ask: 

\vspace*{-0.1cm}
\begin{center}
\textit{What are provably more powerful ways to generate?}
\end{center}
\vspace*{-0.1cm}


\begin{figure}[t]
	\centering
    % \vspace{-10pt}
	\includegraphics[width=\textwidth]{figs/intro.pdf}
    \vspace{-20pt}
	\caption{Comparison between autoregressive generation, any-order generation (standard MDM) and any-process generation (our MDM). \vspace{-10pt}}
    \label{fig:ap_mdm_compare}
\end{figure} 

As an initial step, we propose \textbf{Any-Process Generation}, inspired by natural generative mechanisms found across domains. It extends standard MDM beyond its existing $\unmask$ capability with three additional operations (see \Cref{fig:ap_mdm_compare}): $\remask$ (converting decoded tokens back to masks), $\inser$ (adding new mask tokens at any position), and $\delete$ (removing mask tokens), all learned end-to-end from data without architectural changes. Freed from conventional physics-inspired diffusion frameworks, any-process generation removes unnecessary restrictions on mask ratios, decoding steps, sequence lengths and stopping criteria, enabling structural editing and test-time scaling. With these modifications, we show that MDM brings significant promise with both encouraging theoretical and empirical results as follows.

\textbf{Scalability to Hard Problems:}~ The capability to rewrite and backtrack breaks the non-erasable limitations of ARM and standard MDM, enabling our model to achive both optimal parallel time and space complexity (\Cref{thm:main_apmdm}), thus solving many $\textsf{NP}$-hard problems with polynomial space through test-time scaling, i.e. an exponential improvement from $\textsf{P}$ achieved by ARM and standard MDM. Empirically, on Sudoku puzzles (\Cref{fig:sudoku_example}), our model achieves 99.28\% accuracy using only 100 training instances, outperforming ARM (87.18\%) and any-order MDM (89.49\%) with $5\times$ parameters trained on 1.8M instances, which is orders of magnitude more.

\textbf{Structure-Aware Generation:}~ The flexibility to rewrite, insert and delete tokens enables structure-aware generation processes that resist sequential construction, such as gene splicing in biology (\Cref{fig:dna_example}) and 2D graph generation (\Cref{fig:graph_example}). To quantify such benefits, we prove ARM cannot even generate matched parentheses (two-sided Dyck-k language) for arbitrary lengths, i.e. one of the most basic skills for code generation, while our model can (\Cref{fig:dyck_example}, \Cref{thm:apmdm_simulation}). Empirical results on a graph editing / generation task show that our approach maintains perfect accuracy for increasingly larger graphs, while ARM performance degrades significantly as graph size increases.

\textbf{Learning and (OOD) Generalization:}~ Our approach enables learning previously-impossible simpler algorithms that significantly improve learning and generalization. For parity checking (\Cref{fig:parity_example}), our model achieves 100\% generalization to arbitrary lengths after training on only length-2 sequences, while even the latest GPT models struggle on this embarrassingly simple task.

Finally, envisioning a future with access to data of the underlying generation processes of objects we wish to generate, such as code revisions, math proof drafts, or molecular formation processes, any-process MDM is theoretically and empirically more suitable than ARM (\Cref{thm:apmdm_simulation2}). 

% A comprehensive discussion of related works is in \Cref{appendix:related}.


\begin{figure}[t]
    % \vspace{-10pt}
	\centering
	{\setlength{\subfigcapskip}{-5pt}
	\subfigure[\textbf{Sudoku} (NP-Complete Problem). Scaling up inference-time computes to solve significantly harder problems by allowing rewrites and backtracking using the $\remask$ operation (\Cref{sec:advantage_1}).]{\includegraphics[width=\textwidth]{figs/sudoku.pdf}\label{fig:sudoku_example}}
	
    \vspace{-3pt}
	\subfigure[\textbf{Coding} (Matched Parentheses / Dyck-$k$). Generating any-sized two-sided Dyck-$k$ is impossible for ARM, while our model can easily do so with the $\inser$ operation (\Cref{sec:advantage_2}).]{\includegraphics[width=\textwidth]{figs/dyck.pdf}\label{fig:dyck_example}}

    \vspace{-3pt}
	\subfigure[\textbf{Parity} (Counting 1s). Length generalizing parity and counting problems are enabled by learning a simple elimination algorithm with the $\remask$ and $\delete$ operations (\Cref{sec:advantage_3}).]{\includegraphics[width=\textwidth]{figs/parity.pdf}\label{fig:parity_example}}
	
    \vspace{-3pt}
	\subfigure[\textbf{DNA Recombination} (Splicing System). An example in science / biology where DNA segments are spliced and pasted using combinations of operations, which is hard for ARM (\Cref{sec:advantage_2}).]{\includegraphics[width=\textwidth]{figs/DNA.pdf}\label{fig:dna_example}}

    \vspace{-3pt}
	\subfigure[\textbf{Graph Editing.} Editing combinatorial structures where feature / structural evolution and parallel computation are naturally integrated (\Cref{sec:advantage_3}). Using ARM to simulate is hard (\Cref{sec:advantage_4}).]{\includegraphics[width=\textwidth]{figs/graph.pdf}\label{fig:graph_example}}
	}
	\vspace{-10pt}
	\caption{Examples of any-process generation for different tasks.\vspace{-10pt}}
	\label{fig:all_examples}
\end{figure} 



\vspace{-5pt}
\section{Preliminary} \label{sec:preliminary}
\vspace{-5pt}

\textbf{Auto-Regressive Model (ARM)}~~~ Let $\Sigma$ be a finite-sized vocabulary and $\nxt: \Sigma^* \rightarrow \Sigma$ be a next-token predictor, which maps a sequence $\mathbf{x} = (x_1, x_2, \cdots, x_n) \in \Sigma^n$ to a token $x_{n+1} \in \Sigma$. An autoregressive model (ARM) is defined based on a sequence-to-sequence mapping $\f: \Sigma^* \rightarrow \Sigma^*$, concatenating input sequence $\mathbf{x}$ and the next token $\nxt(\mathbf{x})$, i.e. $f(\mathbf{x}) = (\mathbf{x}, \nxt(\mathbf{x}))$. ARM formulates generation as an iterative process by repeatedly applying $\f$ to the current sequence. In practice, $\f$ is typically parameterized by a Transformer~\citep{vaswani2017attention} with causal attention and learnable parameters $\theta$. The notion of ARM here also aligns with \emph{Chain-of-Thought (CoT)}~\citep{wei2022chain} in many other works and we will use them interchangeably throughout this paper.

\textbf{Masked Diffusion Model (MDM)}~~~ Let $\bar{\Sigma} = \Sigma \cup \{\mask\}$ be the extended vocabulary where $\mask$ is an absorbing mask token~\citep{austin2021structured}. Consider sequences $\mathbf{x}_t = (x_{t,1}, x_{t,2}, \ldots, x_{t,S}) \in \bar{\Sigma}^S$ indexed by time $t\in[T]$, where $S$ is the maximum context length, $T$ is the number of decoding steps, $\mathbf{x}_0 = \{\mask\}^S$ is the fully masked sequence and $\mathbf{x}_T \in \Sigma^S$ is the target clean sequence.\footnote{Unlike convention in diffusion model where larger $t$ denotes earlier inference steps, we use $t$ following an intuitive feed-forward ordering during inference, i.e. the focus of this paper.} A masked diffusion model (MDM)~\citep{lou2024discrete,sahoo2024simple,shi2024simplified} also relies on a sequence-to-sequence mapping $\f: \bar{\Sigma}^S \rightarrow \bar{\Sigma}^S$ with $\mathbf{x}_{t+1} = \f(\mathbf{x}_t)$, formulating generation as an iterative process by repeatedly applying $\f$ to progressively unmasks tokens from the all-mask state.

Among many MDM variants, we consider the following standard design choices from recent large language diffusion models~\citep{nie2025large}: \textbf{1)} linear noise schedule with $S = P \cdot T$ for integer $P$, where each step reveals exactly $P$ tokens; \textbf{2)} confidence-based adaptive decoding~\citep{chang2022maskgit} rather than random token selection; \textbf{3)} encoder-only Transformer architecture without timestep embedding; \textbf{4)} conditional generation where input prompt $\mathbf{x}$ of length $n$ is a prefix of $\mathbf{x}_0$, with $n$ calculated within context length $S$, aligned with reasoning problem setup for ARM. Detailed MDM introduction and encoder-only Transformer definition are in \Cref{appendix:mdm} and \Cref{appendix:encoder}, respectively.

\vspace{-5pt}
\section{A Theory of Masked Diffusion} \label{sec:theory}
\vspace{-5pt}

The generation process in MDM is unique in two different ways: it generates multiple tokens in parallel and permits any-order generation. We now investigate whether and how exactly these properties, in their own right, translate into concrete advantages.


\vspace{-3pt}
\subsection{Power of Parallelism} \label{sec:parallelism}
\vspace{-3pt}

Prior work~\citep{merrill2023expresssive,feng2024towards,li2024chain} has shown that ARM with sufficiently many intermediate steps is Turing-complete and thus can solve any computable problem. Analogous to the role of intermediate steps in ARM, two governing resources determine MDM's power: \textbf{1)} number of decoding (denoising) steps $T(n)$, and \textbf{2)} maximum context length $S(n)$ (equivalently, the maximum number of tokens available to decode). 

\begin{definition}[\MDM]
Let $\MDM(S(n), T(n))$ be the class of decision problems solvable by MDM (\Cref{sec:preliminary}) with maximum context length $S(n)$ and at most $T(n)$ decoding steps, using some constant depth and $\log(n)$ embedding size encoder-only Transformer. Also, let $\MDM(S(n))  = \bigcup\limits_{T(n)} \MDM(S(n), T(n))$.
\end{definition}
\vspace{-10pt}


To formally characterize MDM's expressivity in relation to $T(n)$ and $S(n)$, we establish a connection with the canonical parallel computation model called \emph{Parallel Random Access Machine (PRAM)}~\citep{fortune1978parallelism,jaja1992parallel}, which is the RAM model extended to multiple processors executing over shared memory. See detailed introduction and a formal definition of the variant we use in \Cref{appendix:pram}.

\begin{definition}[\PRAM]
    Let $P(n)$ be the number of processors budget, and $w(n)=\Theta(\log n)$ the word size. Define $\PRAM(P(n),T(n))$ as the class of decision problems solvable by a \emph{uniform} CREW PRAM (see \Cref{appendix:pram} for CREW specification) using at most $P(n)$ processors in at most $T(n)$ parallel time.
\end{definition}

\begin{theorem}[MDM Simulation of PRAM, Informal] \label{thm:main_mdm}
    For any PRAM program that runs on input $\mathbf{x} \in \Sigma^n$  in at most $T(n)$ parallel time with $P(n)$ maximum processors, there exists an MDM on input $\mathbf{x}$, padded to $S(n) = \mathcal O(P(n) \cdot T(n))$, that matches the PRAM output in $\mathcal O(T(n))$ decoding steps, i.e. $\PRAM(P(n),T(n)) \subseteq \MDM(\mathcal O(P(n) \cdot T(n)), \mathcal O(T(n)))$. See formal statement in \Cref{thm:main_mdm_formal}.
\end{theorem}
\vspace{-3pt}

\textbf{This demonstrates that MDM can simulate any PRAM algorithm with optimal parallel time complexity, thereby it is not only Turing-complete as ARM already achieves, but can also solve problems significantly faster with parallelization, something ARM cannot offer.} The speedup can be \emph{exponential} compared to ARM's serial time complexity: for efficiently parallelizable problems in $\textsf{NC}$~\citep{arora2009computational},\footnote{$\textsf{NC}$ is the complexity class for efficiently parallelizable problems, those that are solvable in $\text{polylog}(n)$ time using $\text{poly}(n)$ processors; $\textsf{NC} \subseteq \textsf{P}$ and it is open whether $\textsf{NC} = \textsf{P}$~\citep{greenlaw1995limits}. PRAM is the canonical model for this notion as a Turing machine is for \textsf{P}.} graph connectivity can be solved in $\mathcal{O}(\log n)$ decoding steps versus ARM's linear complexity, and context-free languages including Dyck-k require only $\mathcal{O}(\log^2 n)$ steps. These tasks have been demonstrated hard or inefficient for ARM in previous literature~\citep{strobl2024formal,zhu2025reasoning}.


\vspace{-3pt}
\subsection{(Un)Scalability to Hard Tasks} \label{sec:inherently_hard}
\vspace{-3pt}


While noteworthy, the computational power described above comes with a non-negligible cost: solving a problem requires context length $S(n)$ to scale as $\mathcal{O}(T(n) \cdot P(n))$ (the total parallel work), a quantity at least as large as the serial time complexity (with $P(n) = 1$), per Brent's Theorem~\citep{jaja1992parallel}. Particularly, in resource-constrained regimes, we have:

\begin{theorem} \label{thm:main_constrained}
$\MDM(S(n)) \subseteq \PRAM(1, \tilde{\mathcal{O}}(S^3(n)))$, where logarithmic factors are hidden in $\tilde{\mathcal{O}}$.
\end{theorem}
\vspace{-3pt}

In other words, MDM with context length $S(n)$ cannot solve problems requiring more than $\tilde{\mathcal{O}}(S^3(n))$ serial time. This limitation is also shared by ARM~\citep{yang2025pencil}.

\textbf{This implies MDM is inherently not scalable to solving hard reasoning or generation tasks}: for problems beyond $\textsf{P}$ (e.g., $\textsf{NP}$-hard problems), this would require superpolynomial context length (under standard complexity assumptions), practically intractable in terms of both memory and per-step FLOPs. The root cause lies in MDM's irreversible token generation: once decoded, those positions cannot be reused or rewritten. As reflected in the construction of \Cref{thm:main_mdm}, each memory write must be permanently stored as tokens, forcing space to scale with computation time.


In contrast, human reasoning on hard problems naturally involves  continuous revision, exploration of alternative paths, and correction of mistakes before reaching final conclusions. Generation tasks are no different: for instance, generating planar graphs (drawable on planes without edge crossings) with minimum splitting numbers is NP-complete and naturally involves iteratively adding nodes, checking planarity constraints, and backtracking when violations occur. Such process has not been captured by either ARM or MDM.



\vspace{-3pt}
\subsection{(Limited) Power of Any-Order Generation}
\vspace{-3pt}

Any-order generation seems to offer extra flexibility over auto-regression, but does it truly translates into computational advantages? To attribute gains to any-order generation itself, we control for orthogonal factors differentiating ARM and MDM: \textbf{1)} the number of tokens generated per step, and \textbf{2)} the backbone architecture (decoder v.s. encoder). The former has already been shown to confer stronger parallelism to MDM (\Cref{sec:parallelism}); the latter provides internal parallelization benefits~\citep{ewer2024entp} and improved expressiveness through padding with dummy  tokens (i.e. $\mask$)~\citep{merrill2025exact}. 

Therefore, we fix MDM to emit exactly one token per step and ARM to use the encoder‑backbone with mask tokens padding the sequence to the same length (called \emph{Masked-ARM}), or equivalently:

\begin{definition}[Masked-ARM] \label{def:masked_arm}
A Masked-ARM is defined as an MDM (\Cref{sec:preliminary}) that is forced to decode in left-to-right order and one token per step.
\end{definition}
\vspace{-3pt}

\textbf{Perhaps counterintuitively, we show that the computational benefits from any-order generation are rather limited, by itself not enabling what Masked-ARM cannot already solve}:

\begin{theorem}[Left-to-Right v.s. Any-Order, Informal] \label{thm:any_order}
For any AO-MDM with context length $S(n)$ decoding one token per step, there exists a Masked-ARM with length $\mathcal{O}(S(n))$ and extra constant layers, that can produce the same generation process for any given input $\mathbf x$, by explicitly specifying both where to write (position) and what to write (token). See formal statement in \Cref{lemma:aomdm_simulation}.
\end{theorem}
\vspace{-3pt}

Simulating any-order generation with autoregressive models is not hard because the attention mechanism is good at fetching information from any position, and re-organizing it internally in the correct order to perform the same computation. While Masked-ARM need not replicate MDM's exact final sequence, an additional post-processing step can align their outputs without affecting the theoretical conclusion. There are also some empirical evidences showing the effectiveness of ARM simulating any-order~\citep{xue2025any}.

But not all intricacies inherent in natural generation processes can be easily sequentialized. Coding for example (as well as many natural scientific processes alike), involves anywhere editing where a new valid state depends upon previous valid states that may not be contained in the final sequence.  And even when described in left-to-right temporal order, reproducing the state requires more than simple re-organization, which attention is already provably good at. Hence such a complex process is not captured by ARM or MDM as currently instantiated.



\textbf{Remark}~~ We note that MDM's observed advantages in practice may lie in discovering an optimal order~\citep{kim2025train} from data, where left-to-right ordering need not exactly correspond to the optimal temporal generation order, though computationally equivalent (\Cref{thm:any_order}).




\vspace{-5pt}
\section{{Any-Process Generation}} \label{sec:any_process}
\vspace{-5pt}


We now introduce {\emph{Any-Process Generation}}, a more powerful generation paradigm that extends the any-order masked diffusion from \Cref{sec:preliminary} (referred to as AO-MDM hereafter) by removing various restrictions to capture natural processes not present in existing generation strategies.

\textbf{A General Formulation}~~~ Let $f_\theta: \bar{\Sigma}^* \rightarrow \bar{\Sigma}^* \times \Sigma^* \times \C^*$ be a function, by default parameterized by the same Transformer architecture, which on input $\mathbf{x}_k \in \bar{\Sigma}^*$ returns the triple $(\mathbf{x}_k, \mathbf{y}_k, \ctrlvec_k)$ with $\mathbf{y}_k \in \Sigma^{|\mathbf{x}_k|}$ and $\ctrlvec_k \in \C^{|\mathbf{x}_k|}$, where $\mathbf{x}_k$ may contain one or more $\mask$ tokens (a masked sequence) while $\mathbf{y}_k$ is mask-free; $\mathbf{y}_k$ and $\ctrlvec_k$ have the same length as the input. Core to the design of this generation process is a parameter-free (and optionally non-deterministic) transition function $g: \bar{\Sigma}^* \times \Sigma^* \times \C^* \rightarrow \bar{\Sigma}^*$, which takes $(\mathbf{x}_k, \mathbf{y}_k, \ctrlvec_k)$ as input to produce the next sequence $\mathbf{x}_{k+1} \in \bar{\Sigma}^*$ that can differ in length from $\mathbf{x}_k$. The inclusion of input $\mathbf{x}_k$ itself in the output of $f_\theta(\mathbf{x}_k)$ ensures that $g$ has access to which positions are masks initially. Overall, generation is formulated as the iterative application of $f_\theta$ and $g$ until some stopping criterion is met:
\begin{align}
\mathbf{x}_{t+1} = g\big(f_\theta(\mathbf{x}_t)\big), \quad\text{and hence }\; \mathbf{x}_t = (g \circ f_\theta)^t(\mathbf{x}_0) = {g \circ f_\theta \circ g \circ f_\theta \circ \cdots \circ g \circ f_\theta}(\mathbf{x}_0).
\end{align}
Notably, unlike vanilla MDM, this framework imposes no restrictions on mask ratios at any given time 
step, therefore each decoding step can unmask an arbitrary number of tokens, and 
$\mathbf{x}_0$ can be the input prompt $\mathbf{x}$ directly with no initial mask, as in ARM. 
This framework also does not limit the maximum number of decoding steps $T$, the stopping 
criterion is flexible and need not to be a fully unmasked sequence, as in ARM. We dub this class of models as \emph{Any-Process MDM} (AP-MDM) and will detail a specific instantiation. It is not difficult to see AO-MDM is a special case of AP-MDM.

\vspace{-3pt}
\subsection{An Instantiation with $\unmask$, $\remask$, $\inser$, and $\delete$ Operations}
\vspace{-3pt}

Define $\C = \{0,1\}^3$. For each position $i$ and time step $t$, the per-position control is a 3-bit vector $\ctrl_{t,i} = (\ctrl_{t,i}[1],\,\ctrl_{t,i}[2],\,\ctrl_{t,i}[3]) \in \C$ reserved for different purposes that will be detailed below. Correspondingly, we write $\ctrlvec_t = (\ctrl_{t,1},\ldots,\ctrl_{t,|\mathbf{x}_t|}) \in \C^{|\mathbf{x}_t|}$. 

\textbf{Capability: Rewrite via Remask}~~~ We use the first bit of the per-position control  $\ctrl_{t,i}[1] \in \{0,1\}$ to control remasking (and whether to unmask) and define $\forall y \in {\Sigma}$:
\begin{align}
    \remask_{x_{t,i}, \ctrl_{t,i}}(y) = \begin{cases}
\mask & \text{if } \ctrl_{t,i}[1] = 1 \\
y & \text{if } x_{t,i} = \mask \text{ and } \ctrl_{t,i}[1] = 0 \\
x_{t,i} & \text{otherwise}
\end{cases}
\end{align}
In other words, when $\ctrl_{t,i}[1] = 1$, position $i$ is a mask after decoding regardless of its previous state; otherwise, standard unmasking follows as usual. This operation enables self-correction and \emph{test-time scaling}, allowing models to scale computation exponentially in $S(n)$ before state repetition occurs. Additionally, since the remasking signal can be learned from data, models can adaptively determine both decoding order and parallelization degree at each step.


\textbf{Capability: Length-Variable Edit via Insert / Delete}~~~ We use the second and third bits of the per-position control $\ctrl_{t,i}$ to govern insertion and deletion, respectively. Define $\forall y \in \bar{\Sigma} \cup \{\epsilon\}$:
\begin{align}
\inser_{\ctrl_{t,i}}(y) = \begin{cases}
(y, \mask) & \text{if } \ctrl_{t,i}[2] = 1 \\
y & \text{otherwise}
\end{cases},~ \delete_{x_{t,i}, \ctrl_{t,i}}(y) = \begin{cases}
    \epsilon & \text{if } x_{t,i} = \mask \text{ and } \ctrl_{t,i}[3] = 1 \\
    y & \text{otherwise}
    \end{cases}
\end{align}
where $\epsilon$ denotes the empty string.
In other words, insert adds a mask token after position $i$ when $\ctrl_{t,i}[2] = 1$, and delete removes position $i$ when it was originally a mask token ($x_{t,i} = \mask$) and $\ctrl_{t,i}[3] = 1$. These operations enable dynamic sequence length adjustment based on problem complexity, with the insert operation allowing sequence length to grow exponentially as \emph{each mask token can spawn additional masks}. Furthermore, the delete operation provides computational efficiency by freeing space during stages that require less extensive computation, reducing overall FLOPs waste. This mechanism can work orthogonally with semi-autoregression~\citep{arriola2025block} that expands space at the end of the sequence.


\textbf{Composition}~~~ To summarize, each decoding step applies the three operations coordinate-wise and concatenates the resulting segments:
\begin{align} \label{eq:apmdm_implementation}
\begin{split}
    &f_\theta(\mathbf{x}_t) = (\mathbf{x}_t, \mathbf{y}_t, \ctrlvec_t), \quad g(\mathbf{x}_t, \mathbf{y}_t, \ctrlvec_t) = (s_{t,1}, s_{t,2}, \ldots, s_{t,|\mathbf{x}_t|}) = \mathbf{x}_{t+1} \\
    &\text{where } s_{t,i} = \inser_{\ctrl_{t,i}} \circ \delete_{x_{t,i}, \ctrl_{t,i}} \circ \remask_{x_{t,i}, \ctrl_{t,i}}(y_{t,i}), \quad \forall i \in [|\mathbf{x}_t|]
\end{split}
\end{align}

An algorithmic description of any-process generation with these operations is given in \Cref{alg:apmdm_sampling}, \Cref{appendix:algorithm}. And the stopping criterion can be flexibly defined, e.g. one can use generation of an EOS token (as in ARM) or convergence to a repeated sequence (loop occurs). Architecturally, implementing these capabilities requires no changes to the Transformer structure, only three additional logit dimensions are needed for producing control signals, i.e. three extra linear heads (details in \Cref{appendix:apmdm_architecture}).


\textbf{Pre-Training / Fine-Tuning / Data Availability}~~~ All three operations can be trained end-to-end from text corpora using self-supervised objectives, preserving MDM's scalability to large-scale training (details in \Cref{appendix:training}). The minimal architectural changes also enable direct fine-tuning from existing large diffusion models, possibly using supervised data where the underlying generation process is explicitly constructed (details in \Cref{appendix:training2}). This training flexibility separates our approach from alternatives like looped Transformers~\citep{dehghani2018universal,giannou2023looped}, which are expressive but notoriously difficult to train due to lack of intermediate supervision. 


\textbf{Design Considerations}~~~ The proposed three operations all revolve around the mask token $\mask$, leveraging existing MDM's strong unmasking capability while only adding modular extensions that are easier to pretrain or fine-tune than learning harder operations such as inversion of uniform noise~\citep{sahoo2025diffusion}. Moreover, while the definitions of $g$ and $\ctrlvec_t$ in (\ref{eq:apmdm_implementation}) suffice for achieving theoretical benefits detailed later, they are not necessary conditions; other designs are possible.

We note that the idea of remasking and editing have been individually explored in some prior/concurrent works~\citep{wang2025remasking,peng2025path,von2025generalized,havasi2025edit,wu2025dreamon}. However, there has not been a systematic study of guidance principles for yielding provable computational benefits and their practical implications. See discussions in \Cref{appendix:related}. 





% \begin{figure}[t]
% 	\centering
% 	\includegraphics[width=1\textwidth]{figs/tasks.pdf}
% 	\caption{This is a figure.} \label{fig:sudoku_example}
% \end{figure} 

\vspace{-5pt}
\section{The Power of Any-Process Generation}
\vspace{-5pt}

We now show how any-process generation circumvents various difficulties that current ARM and MDM encounter when handling tasks across different domains and complexities.

% , providing both theoretical and empirical support.

\vspace{-3pt} 
\subsection{Universally Efficient Computation} \label{sec:advantage_1}
\vspace{-3pt}

\begin{tcolorbox}[
    enhanced,
    colback=Gray!3!white,
    colframe=Gray,
    arc=4mm,
    boxrule=1pt,
    left=4pt,
    right=4pt,
    top=4pt,
    bottom=4pt,
    drop shadow={opacity=0.15, xshift=1.5pt, yshift=-1.5pt},
    fonttitle=\bfseries,
    coltitle=Gray,
    before skip=8pt,
    after skip=8pt
]
\textit{\textbf{Benefit 1}: Scalability to significantly harder problems through rewriting and backtracking.}
\end{tcolorbox} 

As discussed in \Cref{sec:inherently_hard}, inherently hard problems (e.g. many NP-hard tasks) typically do not admit sequential processes but require iterative ``search–verify–revise" loops, which hold across various domains: from theorem proving, solving code challenges to synthesis of complex structures in nature. The pathological way current generation paradigms let discardable tokens accumulate indefinitely creates scaling barriers, where space explodes and each step incurs ever-increasing computational cost (\Cref{thm:main_constrained}). We now demonstrate how AP-MDM resolves this:


\begin{theorem}[AP-MDM Simulation of PRAM, Informal] \label{thm:main_apmdm}
    For any PRAM program that runs in at most $T(n)$ parallel time, $P(n)$ processors and $S(n)$ memory usage, there exists an AP-MDM that matches PRAM output on any given input with $\mathcal{O}(S(n))$ context length and $\mathcal{O}(T(n))$ decoding steps. See formal statement in \Cref{thm:main_apmdm_formal}.
\end{theorem}\vspace{-3pt}

By comparison, standard MDM requires space scaling with the total work $\mathcal{O}(P(n) \cdot T(n))$ (\Cref{thm:main_mdm}), whereas AP-MDM requires only the actual space needed, achieving both optimal parallel time and space complexity (\Cref{thm:main_apmdm}). This implies AP-MDM not only retains the efficiency of parallelization, but also dramatically expands the range of solvable problems.  In particular, given polynomial context length, AP-MDM can solve problems in \textsf{PSPACE} rather than just \textsf{P}, which is an exponential improvement that makes many \textsf{NP}-hard problems solvable with test-time scaling. 



\textbf{Experiment: Sudoku Puzzles}~~ We conduct experiments on Sudoku puzzles, i.e. an \textsf{NP}-complete problem when generalized to $n^2 \times n^2$ grids, requiring both the capability of any-order generation, and the capability to rewrite. As illustrated in \Cref{fig:sudoku_example}, AP-MDM can use the $\remask$ (and standard $\unmask$) operations to choose the easiest position to fill first, and also erase failed branches and try alternative assignments, effectively scaling computates to solve harder instances. 


We follow the experimental setup from~\citep{kim2025train} but with a key difference: while the original work used 1.8M training puzzles, we use only 100 (moderately hard) instances for training AP-MDM. Despite this significantly reduced dataset, our approach achieves near-perfect accuracy (99.28\%) on most Sudoku puzzles, outperforming both AO-MDM and ARM that use substantially more samples and larger model sizes, as shown in \Cref{fig:sudoku_combined}. Any-process generation is sample more efficient because if the model is allowed to conduct more computes and more steps to solve the same problem, each step would become easier to learn. Orthogonally, we find from the training dynamics in \Cref{fig:sudoku_combined} that the model quickly learns to identify and fill the easiest positions (unmasking loss drops rapidly), while learning the order (which position to fill next) proves more challenging.



\begin{figure}[t!]
    % \vspace{-15pt}
    \centering
    % \vspace{-0.5cm}
    \begin{minipage}[t]{0.5\linewidth}
        \vspace{0pt}
        \centering
        \resizebox{\textwidth}{!}{%
        \setlength{\tabcolsep}{1mm}{%
        \begin{tabular}{@{}lccc@{}}
        \toprule
        \textbf{Method} & \textbf{\# Samples} & \textbf{\# Param} & \textbf{Accuracy} \\
        \midrule
        ARM {\small(w/o ordering)} & 1.8M & 42M & 9.73\% \\
        ARM {\small(with ordering)} & 1.8M & 42M & 87.18\% \\
        \midrule
        AO-MDM {\small(vanilla)} & 1.8M & 6M & 6.88\% \\
        AO-MDM {\tiny(Top-K probability)}  & 1.8M & 6M & 18.51\% \\
        AO-MDM {\tiny(Top-K prob. margin)} & 1.8M & 6M & 89.49\% \\
        \midrule
        AP-MDM & 100 & 1.2M & 99.28\% \\
        \bottomrule
        \end{tabular}}}\\[10pt]
        {\textbf{(a)} Comparison of accuracy on Sudoku.}
        \label{tab:sudoku_accuracy}
    \end{minipage}
    \hspace{10pt}
    \begin{minipage}[t]{0.42\linewidth}
        \vspace{0pt}
        \centering
        \includegraphics[width=\linewidth]{figs/sudoku_curve.pdf}
        \\[-6pt]
        {\textbf{(b)} Convergence of losses.}
        \label{fig:sudoku_process}
    \end{minipage}
    \vspace{-5pt}
    \caption{Experimental results on Sudoku puzzles. Results of ARM and AO-MDM are taken from ~\citet{kim2025train}. Losses are defined in \Cref{appendix:training}.\vspace{-12pt}}
    \label{fig:sudoku_combined}
\end{figure}



\vspace{-3pt}
\subsection{Structural Generation: Examples in Coding and Science} \label{sec:advantage_2}
\vspace{-3pt}


\begin{tcolorbox}[
    enhanced,
    colback=Gray!3!white,
    colframe=Gray,
    arc=4mm,
    boxrule=1pt,
    left=4pt,
    right=4pt,
    top=4pt,
    bottom=4pt,
    drop shadow={opacity=0.15, xshift=1.5pt, yshift=-1.5pt},
    fonttitle=\bfseries,
    coltitle=Gray,
    before skip=8pt,
    after skip=8pt
]
\textit{\textbf{Benefit 2}: Generating (or reasoning over) complex structured objects that evolve non-sequentially, common across domains beyond natural language (e.g. coding, science).}
\end{tcolorbox} 


When the evolving object involves some complex structures (e.g. trees, graphs, strings with constraints) that do not inherently build up linearly, forcing the generation into a sequential procedure can introduce unnecessary computational difficulties. Such scenarios are especially common across domains beyond natural language (e.g. coding, biology), which current LLMs struggle with.

\textbf{Example 1: Coding}~~~ Programs generally require satisfying global constraints like syntax and semantics at every intermediate state during development, since building each state upon the previous valid one is easier than jumping directly to the final solution. To illustrate this, consider the basic task of generating matched parentheses (the Dyck-$k$ language with $k$ types of parentheses), as illustrated in \Cref{fig:dyck_example}. Natural generation involves inserting parentheses anywhere without breaking balance constraints, while left-to-right generation requires global foresight and constant validity checking during generation, provably impossible for Transformers at scale. Particularly, we consider a variant called the two-sided Dyck-k (Definition in \Cref{appendix:proof_apmdm_simulation}):

\begin{theorem}[Generating Two-Sided Dyck-$k$, Informal] \label{thm:apmdm_simulation}
For any $k \ge 2$, there exists a stochastic AP-MDM whose generation distribution has support exactly equal to the two-sided Dyck-$k$ language, i.e., a string has positive generation probability if and only if it is in the language. Conversely, for any fixed ARM, there exists a length threshold beyond which the ARM cannot guarantee positive probability for all strings in two-sided Dyck-$k$. See formal statement in \Cref{thm:apmdm_simulation_formal}.
\end{theorem}
% \vspace{-3pt}


This result holds because generating Dyck language for arbitrary length is as hard as recognizing it, i.e., deciding if the current sequence has matched parentheses, which Transformer restricted in $\textsf{TC}^0$ expressivity cannot do. The vanilla MDM faces similar difficult as it has to predetermine the number of tokens between each matched parentheses. AP-MDM fundamentally circumvents this difficulty.



\textbf{Example 2: Science / Biology}~~~ Consider linear splicing~\citep{head1987formal,puaun1996splicing}, which is DNA recombination abstracted (and perhaps over-simplified) as cutting two strings and cross-pasting their halves, as illustrated in \Cref{fig:dna_example}. Iterating such rules from a finite seed set generates a splicing language, and any regular language with a marker added to the left side can be generated by such a system~\citep{head1998splicing,kari2012deciding}, while regular language has been proven impossible for constant-depth Transformers~\citep{liu2022transformers,li2024chain}. 



\textbf{Experiment: Graph Generation}~ To illustrate the power of generating complex structures by editing, we consider a graph generation task.  Given a directed graph and a prompt specifying source and target nodes $s$ and $t$, the model is required to generate a modified graph that disconnects $s$ and $t$ by removing the minimum number of edges. This is equivalent to finding the min-cut. Efficient algorithms for generation typically involve iterative editing: \textbf{1)} Use BFS to find a path from $s$ to $t$; \textbf{2)} Augment this path and modify the graph structure; \textbf{3)} Repeat until $s$ and $t$ are disconnected, then remove the min-cut edges. Any-process generation is naturally suited for such graph editing tasks, leveraging $\inser$/$\delete$/$\remask$ operations for adaptive structural and feature modifications and MDM's parallel capabilities, as illustrated in \Cref{fig:graph_example}. As shown in \Cref{fig:parity_graph}, our model achieves almost perfect accuracy for increasingly larger graphs. Meanwhile, when we train ARM to simulate the same process, they fail to perform well as graph size increases.

\vspace{-3pt}
\subsection{Learning and OOD Generalization} \label{sec:advantage_3}
\vspace{-3pt}




\begin{tcolorbox}[
    enhanced,
    colback=Gray!3!white,
    colframe=Gray,
    arc=4mm,
    boxrule=1pt,
    left=4pt,
    right=4pt,
    top=4pt,
    bottom=4pt,
    drop shadow={opacity=0.15, xshift=1.5pt, yshift=-1.5pt},
    fonttitle=\bfseries,
    coltitle=Gray,
    before skip=8pt,
    after skip=8pt
]
\textit{\textbf{Benefit 3}: Enabling the use of simpler algorithms to solve problems, thereby {improving sample efficiency and (out-of-distribution) generalization}.}

\end{tcolorbox} 


% The benefit could go beyond computational arguments discussed so far.

\textbf{Experiment: Parity}~~~ Given a binary sequence $\mathbf{x} \in \{0, 1\}^n$, parity involves determining if there are an even or odd number of 1s. This task is conceptually trivial but embarrassingly difficult for LLMs. Intuitively, the difficulty arises because ARM is forced to attend the entire input and learn a position-invariant target function, which is hard on training sequences with finite-length. With any-process generation, the model circumvents this difficulty by learning a simple elimination algorithm: examine the first two tokens, $\delete$ all 0s if the pair contains any 0 or $\delete$ the pair if both are 1s, then repeat until all are processed (\Cref{fig:parity_example}). The answer is true if any 1s remain. This mimics how humans solves the problem, a simple length-generalizable approach only possible with deletion. 
% See details in \Cref{appendix:exp_parity}.

As shown in \Cref{fig:parity_graph}, our model achieves 100\% accuracy on \textbf{any length} after training on only $n=2$ length sequences with a tiny model ($\sim$200 parameters), while ARM with orders of magnitude more parameters and samples fails to generalize beyond training lengths.


\begin{figure}[t!]
    % \vspace{-8pt}
    \centering
    \begin{minipage}[t]{0.48\linewidth}
        \vspace{0pt}
	\centering
        \resizebox{\textwidth}{!}{%
        \setlength{\tabcolsep}{2.5mm}{%
        \begin{tabular}{@{}c|c|c@{}}
        \toprule
        \textbf{Graph Size (\# Nodes)} & \textbf{ARM Acc.} & \textbf{AP-MDM Acc.} \\
        \midrule
        \textbf{4} & 90.32\% & 100\% \\
        \textbf{5} & 43.04\% & 100\% \\
        \textbf{6} & 0.30\% & 100\% \\
        \textbf{7} & N/A & 100\% \\
        \textbf{8} & N/A & 99.99\% \\
        \textbf{9} & N/A & 99.97\% \\
        \textbf{10} & N/A & 99.92\% \\
        \bottomrule
        \end{tabular}}}\\[8pt]
        {\textbf{(a)} Graph generation via editing.}
    \end{minipage}
    \hspace{10pt}
    \begin{minipage}[t]{0.41\linewidth}
        \vspace{0pt}
        \centering
        \includegraphics[width=\linewidth]{figs/parity_generalization.pdf}\\[-6pt]
        {\textbf{(b)} Length generalization on parity.}
    \end{minipage}
    \vspace{-8pt}
    \caption{Graph generation and parity task results.\vspace{-15pt}}
    \label{fig:parity_graph}
\end{figure} 


%\vspace{-3pt}
\subsection{Hardness of Being Simulated} \label{sec:advantage_4}
%\vspace{-3pt}

\begin{tcolorbox}[
    enhanced,
    colback=Gray!3!white,
    colframe=Gray,
    arc=4mm,
    boxrule=1pt,
    left=4pt,
    right=4pt,
    top=4pt,
    bottom=4pt,
    drop shadow={opacity=0.15, xshift=1.5pt, yshift=-1.5pt},
    fonttitle=\bfseries,
    coltitle=Gray,
    before skip=8pt,
    after skip=8pt
]
\textit{\textbf{Benefit 4}: If in the future we have access to data of underlying generation processes (e.g. revision history of code, articles, math proof drafts, molecular formation processes), any-process MDM is more suitable than ARM for practical training.}
\end{tcolorbox} 

Besides scalability to harder tasks (\Cref{sec:advantage_1}) and universality across domains (\Cref{sec:advantage_2}), a crucial question remains: suppose given access to datasets containing revision histories of the objects we wish to generate, would AP-MDM be the most appropriate model for such data and large-scale training? To answer this, we consider ARM as a competitor as it is Turing-complete, and equally expressive as AO-MDM (\Cref{thm:any_order}) when controlled for orthogonal factors.

We next show ARM is inherently unsuitable for training on editing datasets in two ways.  \textbf{Firstly}, unlike any-order generation (\Cref{thm:any_order}), AP-MDM's editing operations is hard to be simulated by ARM by explicitly specifying editing operations applied at each decoding step; particularly

\begin{theorem}[Hardness of Simulating AP-MDM, Informal] \label{thm:apmdm_simulation2}
There exists a constant-depth AP-MDM, such that no constant-depth ARM can simulate the generation process of that AP-MDM using a sequence of triplets, i.e., what operation to use ($\unmask$, $\remask$, $\inser$, $\delete$), where to apply the operation  (position) and what to write for the unmask operation (token), on any input $\mathbf x$, under some complexity hardness assumptions in \Cref{appendix:proof_apmdm_simulation2}. See formal statement in \Cref{thm:apmdm_simulation2_formal}.
\end{theorem}
%\vspace{-3pt}

Empirically, we show that representing our generation process using triplets described above for ARM simulation indeed becomes increasingly difficult to train as sequence length grows, as demonstrated in the graph generation task in \Cref{sec:advantage_2}.

\textbf{Secondly}, if we disregard the resource constraints from \Cref{sec:advantage_1} and \Cref{sec:advantage_2}, simulation becomes possible through additional intermediate steps, but this could require highly contrived trajectories that defeat the purpose of practical training, e.g. periodically summarize the current state, or using more than constant tokens to represent each application of an operation.

\vspace{-5pt}
\section{Conclusion}
\vspace{-5pt}

This paper provides formal analysis of generation processes and shows, provably and empirically, that moving beyond standard autoregression and current masked diffusion yields more powerful models. These results suggest concrete design principles for frontier LLMs, pointing to training and decoding schemes that scale to increasingly hard tasks and generalize across domains such as code and the sciences. See further contextualization with respect to related work in \Cref{appendix:related}. 




\bibliography{ref}
\bibliographystyle{iclr2026_conference}

\clearpage
\appendix




\tableofcontents
\clearpage


\section{Related Work} \label{appendix:related}

Masked diffusion models~\citep{hoogeboom2021argmax,austin2021structured,lou2024discrete,sahoo2024simple,shi2024simplified} extend continuous diffusion models~\citep{sohl2015deep,ho2020denoising,song2020score} to discrete data. Early work applied these models to specialized domains such as graph generation~\citep{vignac2022digress,sun2023difusco}, protein design~\citep{gruver2023protein}, and drug discovery~\citep{lee2025genmol}, where non-sequential generation provides natural advantages. The field has evolved with recent commercial-scale language models like Gemini Diffusion~\citep{deepmind2025gemini} and Mercury~\citep{inceptionlabs2025mercury}, which demonstrate competitive performance on language generation, reasoning, and coding tasks. This suggests that MDMs can serve as viable alternatives to the autoregressive models that currently dominate LLMs. Against this background, this paper investigates the fundamental computational differences between generation paradigms and explores whether more powerful generation methods exist.

Several works have explored extensions to standard MDM through mechanisms that enable rewriting and editing~\citep{vonrutte2025generalized,wang2025remasking,peng2025path,havasi2025edit,wu2025dreamon,kim2025any}, which relate to our any-process generation framework. \citet{wang2025remasking} introduces random remasking during inference, though this capability is not learned from data. \citet{lou2024discrete,vonrutte2025generalized,sahoo2025diffusion} propose adding uniform noise in the forward process rather than using masks, with models learning to revert them in the backward process, but this approach generally underperforms since modifying tokens directly appears more difficult than unmasking. \citet{peng2025path} introduces path planning to control generation, though the planner is not trained end-to-end with the base model. Current with ours: \citet{havasi2025edit} introduces edit operations to flow matching frameworks but faces similar limitations as uniform noise approaches; \citet{kim2025any} introduces to insert tokens at any 
position while \citet{wu2025dreamon} proposes expansion 
and delete, but these capabilities per se are 
insufficient for handling hard reasoning tasks as 
discussed in \Cref{sec:theory}. 


\section{Background: Masked Diffusion Model} \label{appendix:mdm}

We introduce the preliminaries of diffusion language models or masked diffusion models (MDM), following the notation established in \Cref{sec:preliminary}. Let $\bar{\Sigma} = \Sigma \cup \{\mask\}$ be the extended vocabulary where $\mask$ is the mask special token. Consider sequences $\mathbf{x}_t = (x_{t,1}, x_{t,2}, \ldots, x_{t,S}) \in \bar{\Sigma}^S$ indexed by time $t\in[T]$, where $S$ is the maximum context length, $T$ is the number of decoding steps, $\mathbf{x}_0 = \{\mask\}^S$ is the fully masked sequence and $\mathbf{x}_T \in \Sigma^S$ is the target clean sequence.

\paragraph{Forward Noising Process}
The forward noising process constructs training data by generating noisy versions $\mathbf{x}_t$ from clean sequences $\mathbf{x}_T$. Unlike the discrete inference steps in \Cref{sec:preliminary}, training uses continuous time $t \in [0, T]$ with larger $t$ denoting later denoising steps. MDM employs the \emph{absorbing mask kernel}~\citep{austin2021structured} where the signal ratio $\alpha_t = t/T$ represents the marginal probability that a token remains unmasked. Since $\alpha_t$ increases monotonically with $t$ (i.e., $\alpha_s < \alpha_t$ for $s < t$), later timesteps preserve more original tokens, consistent with our convention where $t=0$ is fully masked and $t=T$ is clean. At each position $i$, tokens either stay unchanged or become $\mask$, and once masked, they ``absorb'' into this state. The marginal distribution is:
\begin{align}
q(x_{t,i} \mid x_{T,i}) = \operatorname{Cat}(x_{t,i}; \alpha_t \mathbf{e}_{x_{T,i}} + (1-\alpha_t) \mathbf{e}_{\mask})
\label{eq:forward_marginal}
\end{align}
where $\mathbf{e}_v$ denotes the one-hot vector for token $v \in \bar{\Sigma}$. To obtain noised sequence $\mathbf{x}_t$ from $\mathbf{x}_T$, we compute the masking probability $1-\alpha_t$ and mask each position independently with this probability. 

\paragraph{Training Objective} 
The reverse process aims to recover $\mathbf{x}_T$ from $\mathbf{x}_0$. MDM parameterizes $p_\theta(\mathbf{x}_T \mid \mathbf{x}_t, t)$ to predict the clean data directly (but as mentioned in \Cref{sec:preliminary}, many recent large-scale MDMs omit explicit timestep embeddings). For the absorbing mask kernel, the true posterior $q(\mathbf{x}_s \mid \mathbf{x}_t, \mathbf{x}_T)$ for $s < t$ has an analytical form. For each position $i$:
\begin{align}
q(x_{s,i} \mid x_{t,i}, x_{T,i}) = \begin{cases}
\text{Cat}(x_{s,i}; \mathbf{e}_{x_{t,i}}), & \text{if } x_{t,i} \neq \mask \\
\text{Cat}\left(x_{s,i}; \frac{(1-\alpha_s)\mathbf{e}_{\mask} + (\alpha_t-\alpha_s)\mathbf{e}_{x_{T,i}}}{1-\alpha_s}\right), & \text{if } x_{t,i} = \mask
\end{cases}
\label{eq:true_posterior}
\end{align}
This means that, if position $i$ is not masked at time $t$, it remains unchanged at time $s$; if position $i$ is masked at time $t$, it transitions probabilistically between the original token and mask. The training objective is derived from the variational lower bound. For the absorbing mask kernel, it simplifies to:
\begin{align}
\mathcal{L}_{\text{CE}}(\theta) = \mathbb{E}_{t \sim \mathcal{U}(0,T), \mathbf{x}_T \sim p_{\text{data}}, \mathbf{x}_t \sim q(\mathbf{x}_t|\mathbf{x}_T)} \left[ -\frac{1}{|\{i: x_{t,i} = \mask\}|} \sum_{i: x_{t,i} = \mask} \log p_\theta(x_{T,i} \mid \mathbf{x}_t, t) \right]
\label{eq:ce_loss}
\end{align}
The loss is computed only on masked positions and averaged over the number of masked tokens, making this equivalent to conditional masked language modeling with proper normalization. Here, $t$ is sampled uniformly from the continuous interval $[0, T]$ during training, $\mathbf{x}_T$ is sampled from the data distribution, and $\mathbf{x}_t$ is obtained by applying the forward noising process.

\section{LLM-based Multi-Agent Blackboard System}

This section introduces an alternative communication paradigm for LLM-based multi-agent systems inspired by blackboard systems \citep{10.1145/356810.356816}, distinct from the widely used master–slave architecture. As outlined in \textsection \ref{sec:introduction}, blackboard-based multi-agent systems provide several advantages over the master-slave approach. Here, rather than directly assigning tasks to sub-agents, the main agent posts its requests (i.e., sub-tasks for which it requires assistance) on a shared blackboard, which functions as a broadcast channel accessible to all other agents. Each helper agent independently evaluates whether it can respond to a request, considering its own capabilities, availability, cost, and other factors. If an agent decides to contribute, it writes its response to the corresponding request, and the main agent then decides whether to use or ignore the provided information. \textit{This way, all agents in the system retain full autonomy over their actions, and no centralized controller forces them to execute a specific task.} While the blackboard paradigm is applicable to a wide range of multi-agent systems, we focus on data science tasks that require data discovery, where its characteristics are particularly advantageous, as discussed in \textsection \ref{sec:introduction}. The remainder of this section details our method and its design for data science problems that require information discovery.



\paragraph{Overview:} 

An overview of our proposed method is presented in Figure~\ref{fig:overview}. The system $\pi_{s}$ operates over the data lake $\sD$ by first partitioning $\sD$ into $C$ clusters of related files. Each cluster $\sD_i$ is assigned to a file agent $\pi_{f_i}$, which is responsible for handling, loading, processing, and retrieving information from the files within its cluster. In addition, a search agent $\pi_{s}$ is included to retrieve external information from the web that may be required to solve the problem. The overall system $\pi_{s}$ is composed of a main agent $\pi_{m}$, which is responsible for solving the query $q$, and a set of $C+1$ helper agents $\Pi_{\text{helper}} = \{\pi_{f_i}\}_{i=1}^{M} \cup \{\pi_{s}\}$ that provide specialized assistance. The query $q$ is presented to $\pi_{m}$, which iteratively selects an action $a \in \sA$ from the action space $\sA$, executes the chosen action, and observes the resulting outcome from the environment. Among its actions, the main agent may interact with a blackboard $\beta$, a shared communication medium where it can post a request $r$ without addressing a specific sub-agent. The helper agents $\Pi_{\text{helper}}$ continuously monitor the blackboard, determine whether they can address a posted request, and, if so, provide their outputs on the corresponding response board $\beta_{r}$. These responses are then collected and made available to $\pi_{m}$, which incorporates them into its decision-making process.\footnote{Responses are not written back to the blackboard $\beta$ to avoid dependencies where one sub-agent's output could influence the behavior of others negatively. Instead, all responses are directed exclusively to the response board $\beta_{r}$, ensuring independent operation of sub-agents and exclusive access by the main agent $\pi_{m}$.} The main agent is limited to at most $T$ sequential actions (including actions that interact with the blackboard) to solve the query $q$, ultimately producing a program $p$ in python programming language that computes the final answer to $q$.

\paragraph{Clustering Data Lake:} 

There are multiple approaches for partitioning the data lake into clusters; applying clustering algorithms over file representations, random partitioning, or other heuristic methods. For simplicity, we do not utilize file content and instead rely solely on file names during clustering. Specifically, the file names are provided to an LLM---Gemini-2.5-Pro\footnote{Available at: \url{https://cloud.google.com/vertex-ai/generative-ai/docs/models/gemini/2-5-pro}}---which using the prompt shown in Figure~\ref{fig:clustering-prompt}, clusters the files into categories based only on their names.\footnote{This method represents just one simple possible approach to clustering, chosen for simplicity; more scalable and accurate alternatives could equally be employed in real world scenarios.} An example of this clustering is provided in Figure~\ref{fig:clustering-example} in Appendix~\ref{app:case-study}, where the model successfully groups related files together. For instance, it clusters all files originating from the National Interagency Fire Center into a category labeled ``NIFC Wildfire Statistics.'' The number of automatically derived clusters for each dataset is reported in Table~\ref{tab:stats} in Appendix~\ref{app:dataset}.



% The remainder of this section details the design of the main agent and the helper agents, emphasizing how their coordination supports effective information discovery in data science tasks.

\subsection{Main Agent}
\label{sec:main-agent}

The primary role of the main agent is to solve the problem in collaboration with the helper agents. The main agent follows the ReAct framework \citep{yao2023react}, where at each step $t$, given the query $q$ and the history of actions and observations $\sH_{t-1}$, it first reasons about what is the best next action and selects an action from a predefined action space, executes the action, observes the outcome, and appends the resulting observation to update the history $\sH_{t}$.\footnote{In this work, the inputs, outputs of the model, and observations are appended directly to the prompt of the LLM, formatted according to its chat-based input template.} The prompt used by the main agent is shown in Figure~\ref{fig:main-agent-blackboard-prompt} in Appendix~\ref{app:prompts}. The agent selects one of the following predefined actions in each step, executes them, and observe their outcomes:

\begin{itemize}[leftmargin=*]
    \item \textit{\textbf{Planning:}} In this action, the LLM decomposes the problem into smaller sub-problems and outlines a plan for addressing each of them. This action has no external effect on the environment but serves as an internal reasoning step to guide the LLM's problem-solving process. In response, the system simply acknowledges the proposed plan and instructs the LLM to proceed.
    
    \item \textit{\textbf{Reasoning:}} In this action, the LLM focuses on a specific aspect of the problem and explains its reasoning, analysis, or interpretation of the available observations and steps taken so far in this process. Similar to the planning step, this action has no external effect on the environment but functions as an internal reasoning mechanism to guide the LLM's problem-solving process. In response, the system simply acknowledges the reasoning and prompts the LLM to continue.
    
    \item \textit{\textbf{Executing Code:}} In this action, the agent generates python code, which is executed using a python interpreter. If the code runs successfully, the resulting outputs are returned to the agent for observation; otherwise, the agent receives the corresponding error messages. This action enables the agent to explore the problem interactively, inspect data files, and experiment with them to gain a deeper understanding of their content and structure and how to process them.
    
    \item \textit{\textbf{Requesting Help:}} In this action, the agent formulates a request for assistance from the sub-agents, specifying, for example, the types of data files or information needed, or the resources required to apply a tool or solve a sub-problem. This request is posted on the blackboard $\beta$ for visibility by the helper agents. Once the sub-agents respond, if they respond, their responses on the response board $\beta_r$ are collected and provided back to the main agent as the outcome of this action for observation and further use in its decision-making process.
    
    \item \textit{\textbf{Answering:}} In this action, the agent concludes the problem-solving process by generating a final program that produces the answer to the query. This action terminates the process, and the output of this step constitutes the final program $p$ generated by the system to address the problem.
\end{itemize}

\subsection{Helper Agents}
\label{sec:sub-agents}

In a data science, information discovery can typically be categorized into two tasks: (1) identifying the specific files that contain the data necessary to the problem, and (2) retrieving general knowledge about concepts relevant to the problem, such as domain-specific terms or details of particular algorithms and methods. To support these, our framework employs two types of helper agents:

\paragraph{File Agent:} 

Handling all the files in a data lake with a single agent is not feasible for several reasons: it typically involve a large number of files, many of which are lengthy and may exceed the agents context window; the files span diverse topics, which can confuse the agent and hinder effective reasoning; and accessing and processing all files simultaneously can be computationally expensive and inefficient, leading to unnecessary overhead and slower problem-solving. For these reasons, in our framework each file agent is assigned responsibility for a subset of data files determined to be relevant, as described earlier in the clustering procedure. In an offline phase, the file agent $\pi_{f_i}$ takes as input a subset of the data lake $\sD_{i}$ and operates through a two-step procedure. In the first step, the agent selects a subset\footnote{When filenames indicate multiple files containing the same type of data over different time periods, the agent does not need to inspect all of them to infer the structure; a small representative sample is sufficient.} (or all) of the files to examine their content. The contents of them are presented to the agent for inspection (details of presentation are in Appendix~\ref{app:implementation}). In the second step, after observing the selected files, the agent reasons about and analyzes them, learning how they are structured, what pre-processing or transformations may be required, and how they should be processed in general. An example of such an analysis is provided in Figure~\ref{fig:file-agent-analyze-example} in Appendix~\ref{app:case-study}. Then, in the online phase, the agent listens for requests from the main agent. Upon receiving a request, based on the analysis it did earlier, it determines whether it can contribute to answering it. If so, the agent generates a detailed plan specifying which files in $\sD_i$ are relevant, how they should be loaded in Python code, what libraries to use, the steps required for data processing, and samples from the data. The prompt used to guide the file agent is shown in Figure~\ref{fig:file-agent-prompt} in Appendix~\ref{app:prompts}. 

\paragraph{Search Agent:}

Certain data science problems require task-specific knowledge about algorithms or domain expertise that the LLM may not possess. To address this, we design a web-search agent that retrieves relevant information from a search engine. This agent operates according to the prompt shown in Figure~\ref{fig:search-agent-prompt} in Appendix~\ref{app:prompts}. Given a request $r$ posted on the blackboard $\beta$, the agent first determines whether it is capable of addressing the request. It is specifically restricted to general web-based information retrieval and does not respond to requests involving access to local files or datasets. If the agent determines that the request can be answered, it enters an iterative search process with a maximum of $T_{\text{search}} = 3$ steps. At each step $t$, the agent generates a set of queries $\sQ_{t}$, which are submitted to a search engine---in this work, Google Custom Search Engine\footnote{We use Google Custom Search Engine, configured to exclude all websites associated with the datasets used in this paper to prevent data leakage: \url{https://developers.google.com/custom-search}}---to retrieve $k=3$ webpage per query. The content of the webpages are then extracted using \textit{beautifulsoup} library\footnote{Available at: \url{https://pypi.org/project/beautifulsoup4/}} to be presented to the search agent. The extracted documents are then evaluated by the agent to determine whether they provide sufficient information to answer the request. If so, the agent generates a response to the request, which is posted to the response board $\beta_r$. If the information is insufficient, a new set of queries is generated to continue gathering relevant data from the web.


\section{Implementation Details} \label{appendix:exp}

We provide implementation details for the experiments described in the main paper. All models are built on encoder-only Transformer architecture as described in \Cref{appendix:encoder}, with task-specific configurations detailed below.

\paragraph{Model Architecture}
All models are based on encoder-only Transformer architecture with rotary positional embeddings (RoPE). For AP-MDM, we add three binary classification heads (remask, insert, delete) on top of the standard unmask head. We use no timestep embedding and set time conditioning to zero during supervised training. For Sudoku, we use 6 layers, 4 attention heads, hidden dimension $d=256$, feed-forward dimension $4d=1024$, maximum sequence length 400, vocabulary size 31. For Parity, we use 1 layer, 1 attention head, hidden dimension $d=4$, feed-forward dimension $4d=16$, maximum sequence length 3, vocabulary size 6, with approximately 200 total parameters. For Graph, we use 8 layers, 4 attention heads, hidden dimension $d=256$, feed-forward dimension $4d=1024$, maximum sequence length set to accommodate graphs of varying sizes, vocabulary size 55; the same configuration is used for the ARM baseline.

\paragraph{Training Data Generation}
For supervised training as described in \Cref{appendix:training2}, training data consists of state-transition tuples $(\mathbf{x}_k, \mathbf{y}^*, \ctrlvec^*)$ generated by simulating the natural solving or generation algorithms for each task. Starting from ground-truth target solutions, we execute task-specific algorithms and record the intermediate computation states at each step. For each state, we capture the current sequence $\mathbf{x}_k$ (with unknown or unfilled positions represented as $\mask$), the target values $\mathbf{y}^*$ to be revealed, and the control decisions $\ctrlvec^*$ indicating which operations (unmask, remask, insert, delete) should be applied. Concrete examples of data generation for Sudoku, Parity, and Graph tasks are provided in \Cref{appendix:examples_sudoku}, \Cref{appendix:examples_parity}, and \Cref{appendix:examples_graph}, respectively.

\paragraph{Training Details}
We train all models using AdamW optimizer with learning rate $10^{-4}$, $(\beta_1, \beta_2) = (0.9, 0.999)$, weight decay $0.01$, and batch size 256. Learning rate follows a constant schedule with 250 warmup steps. We use mixed precision training (bfloat16) with gradient clipping at 1.0. Loss weights are set to $\lambda_r = \lambda_i = \lambda_d = 1.0$ by default. Models are trained for up to 1M steps or until convergence. For Graph generation, we use 100K graph instances for training both AP-MDM and the ARM baseline. For Parity, we use only 4 training samples for AP-MDM, while the ARM baseline is trained with up to 10K instances.




% \begin{figure}[t]
% 	\centering
% 	\includegraphics[width=1.0\textwidth]{figs/expanded_sample_5.pdf}
% 	\caption{This is a figure.}
% \end{figure} 

% \subsection{Sudoku Puzzle} \label{appendix:exp_sudoku}



% \subsection{Parity} \label{appendix:exp_parity}

% \subsection{Graph Generation} \label{appendix:exp_graph}

\section{Parallel Random Access Machine} \label{appendix:pram}

The Random Access Machine (RAM)~\citep{arora2009computational} serves as the foundational theoretical model for sequential computation, featuring a single processor that can access any memory location in unit time regardless of address—hence ``random access", along with a finite set of registers and basic arithmetic/logical operations. This contrasts with models like Turing machines where memory access is sequential. The RAM's key strength lies in its realistic abstraction of modern computers: it captures the essential computational primitives (arithmetic, memory access, conditional branching) while abstracting away hardware details, making it ideal for algorithm analysis.

The Parallel Random Access Machine (PRAM)~\citep{fortune1978parallelism,jaja1992parallel} extends this familiar RAM model to parallel computation by allowing $P(n)$ processors to operate synchronously on shared memory with $\mathcal{O}(\log n)$-bit word size.~\footnote{The $\mathcal{O}(\log n)$-bit word size choice ensures that pointer arithmetic, indexing, and basic integer operations on polynomially bounded values are unit-time, matching the standard RAM assumptions and avoiding artificial speedups due to unrealistically wide words.} Each processor in PRAM maintains its own program counter and unique identifier, enabling conditional branching and coordinated computation. The model operates in discrete synchronous time steps where all active processors execute simultaneously, inheriting RAM's unit-cost random access property while adding the complexity of concurrent memory operations. 

\paragraph{PRAM Variants} PRAM has several variants, which differ in their memory access discipline, forming a hierarchy with precise complexity relationships. Let $\EREW$, $\CREW$, $\CRCWCommon$, $\CRCWArbitrary$ and $\CRCWPriority$ denote the classes of problems solvable in polynomial parallel time with polynomially many processors under each model, listed in order of increasing expressivity:
\begin{itemize}[leftmargin=1.2em,topsep=2pt,itemsep=2pt]
    \item \textbf{\EREW} (Exclusive Read, Exclusive Write): No concurrent access to any memory cell. Most restrictive but captures essential parallelism.
    \item \textbf{\CREW} (Concurrent Read, Exclusive Write): Multiple processors may read the same cell simultaneously. Enables broadcast in $\mathcal{O}(1)$ time vs. $\Theta(\log n)$ in EREW.
    \item \textbf{\CRCWCommon}: Concurrent writes allowed only if all writers agree on the value. Boolean OR computable in $\mathcal{O}(1)$ time.
    \item \textbf{\CRCWArbitrary}: Any concurrent writer may succeed; the choice is made arbitrarily (often modeled as random selection).
    \item \textbf{\CRCWPriority}: Concurrent writes resolved by processor priority with various schemes (e.g., minimum/maximum index, sum of conflicting values).
\end{itemize}
Crucially, any algorithm in a stronger model can be simulated in a weaker model with at most $\mathcal{O}(\log n)$ parallel time overhead~\citep{jaja1992parallel}. This polylogarithmic separation appears in basic primitives, broadcast requires $\Theta(\log n)$ rounds in $\EREW$ but $\mathcal{O}(1)$ in $\CREW$, yet the models remain polynomially equivalent for most complexity-theoretic purposes. \textbf{We adopt the $\CREW$ model throughout this paper}, where different processors are not allowed to write to the same memory cell simultaneously.

PRAM, as an idealized abstraction of shared-memory multiprocessor systems, enables precise analysis of parallel algorithms and gives rise to parallel complexity classes such as \textsf{NC}~\citep{arora2009computational} (problems solvable in polylogarithmic parallel time using polynomially many processors). For example, PRAM can simulate algorithms on trees, linear arrays, meshes, and hypercubes without loss of parallel time, while reverse simulation costs at most $\mathcal{O}(\log^2 P(n))$ overhead; Boolean circuits of depth $D$ can be simulated on $\CREW$ in $\mathcal{O}(D)$ time, making PRAM a natural model for measuring parallel time complexity in theory.

Below, we provide a more formal definition that will be used in proofs.

\subsection{Definition and Execution Process of Word-RAM}

We formalize the standard word-RAM that matches a single-processor PRAM (i.e., $P(n)=1$). Throughout, let the input length be $n$ and fix the word size $w(n)=\Theta(\log n)$.

\paragraph{Word Size, Universe, and Addresses.}
Let the word universe be $\mathbb{U}=\{0,1,\dots,2^{w}-1\}$ with arithmetic modulo $2^{w}$
(two's-complement semantics). The address space is $\mathcal{A}=\{0,1,\dots,S(n)-1\}$ for some
$S(n)\le n^{\mathcal{O}(1)}$. Memory is a mapping $M:\mathcal{A}\to\mathbb{U}$, \emph{zero-initialized}.

Let $a(n)=\lceil\log_2 S(n)\rceil$ denote the address width. We adopt the transdichotomous condition:
\begin{equation}
w(n)\ \ge\ \max\{\lceil\log_2 S(n)\rceil,\ \lceil\log_2 P(n)\rceil\}
\qquad\text{and}\qquad w(n)=\Theta(\log n).
\end{equation}
This ensures every address and processor ID fits in one word, enabling well-typed register-indirect addressing and processor identification.

\paragraph{Instruction Set and Semantics.}
The machine operates with register names $\Reg=\{0,1,\ldots,k\}$ (for constant $k$), register file $R\in\mathbb{U}^{k+1}$, immediate constants $\Imm\subset\mathbb{Z}$ (a fixed finite set independent of $n$ and $w$), and label identifiers $\Lab$ for jump targets.\footnote{A label is a human-readable name for a program location (instruction index) serving as a jump/branch target.} We assume a constant-size register file with $|\Reg| \ge 2$ in the proof (any constant $\ge 2$ suffices up to constant factors). Programs define a partial label table $\mathrm{addr}:\Lab\rightharpoonup\{0,\ldots,\ell\}$ mapping each declared label to its instruction index (injective). 

The instruction alphabet $\Instr$ consists of the following parameterized forms ($r,s\in\Reg$, $c\in\Imm$, $L\in\Lab$):
\[
\begin{aligned}
\Instr \;=\;&
\{\ \mathtt{LOAD}\ r,\,[s],\ \mathtt{STORE}\ [s],\,r,\ \mathtt{LOADI}\ r,\,c\ \}\ \\
&\cup\ \{\ \mathtt{ADD}\ r,\,s,\ \mathtt{SUB}\ r,\,s,\ \mathtt{AND}\ r,\,s,\ \mathtt{XOR}\ r,\,s,\ \mathtt{SHL}\ r,\,s,\ \mathtt{SHR}\ r,\,s\ \}\ \\
&\cup\ \{\ \mathtt{BRZ}\ r,\,L,\ \mathtt{JMP}\ L,\ \mathtt{HALT}\ \}.
\end{aligned}
\]
Unbracketed registers $r,s$ denote their word values $R_r,R_s\in\mathbb{U}$. The bracketed form $[s]$ denotes register-indirect addressing: $\texttt{LOAD } r,[s]$ reads $M[R_s]$ into $R_r$, and $\texttt{STORE } [s],r$ writes $R_r$ to $M[R_s]$. Bracketed operands are only allowed in $\mathtt{LOAD}$/$\mathtt{STORE}$; nested or arithmetic addressing (e.g., $[[s]]$, $[r{+}c]$) is not part of this ISA. If $R_s\notin\mathcal{A}$, execution traps. Immediates are loaded as $c\bmod 2^w$.


The semantics of the instructions are as follows. We write $\sigma\to\sigma'$ for one execution step. Unless a jump changes it, set
$\mathrm{pc}\gets \mathrm{pc}+1$ where $\mathrm{pc}\in\{0,\ldots,\ell\}\cup\{\mathtt{HALT}\}$ is the program counter. Let $\oplus$ and $\wedge$ denote bitwise XOR and AND;
let $\ll$ and $\gg$ denote logical shifts; all arithmetic is modulo $2^w$. 

\begin{itemize}
\item $\mathtt{LOAD}\ r,\,[s]$:\;
      $a\gets R_s$; $R_r\gets M[a]$.

\item $\mathtt{STORE}\ [s],\,r$:\;
      $a\gets R_s$; $M[a]\gets R_r$.

\item $\mathtt{LOADI}\ r,\,c$:\;
      $R_r\gets c \bmod 2^w$.

\item $\mathtt{ADD}\ r,\,s$ / $\mathtt{SUB}\ r,\,s$:\;
      $R_r\gets (R_r \pm R_s)\bmod 2^w$.

\item $\mathtt{AND}\ r,\,s$ / $\mathtt{XOR}\ r,\,s$:\;
      $R_r\gets R_r \wedge R_s$ \; / \; $R_r\gets R_r \oplus R_s$.

\item $\mathtt{SHL}\ r,\,s$ or $\mathtt{SHR}\ r,\,s$:\;
      $h\gets R_s \bmod w$; \; $\mathtt{SHL}$: $R_r\gets (R_r \ll h)\bmod 2^w$;\;
      $\mathtt{SHR}$: $R_r\gets \lfloor R_r/2^h \rfloor$ (logical right shift, zero fill).

\item $\mathtt{BRZ}\ r,\,L$:\;
      If $R_r=0$ then $\mathrm{pc}\gets \mathrm{addr}(L)$ else (no change to $\mathrm{pc}$ beyond $+1$).

\item $\mathtt{JMP}\ L$:\;
      $\mathrm{pc}\gets \mathrm{addr}(L)$.

\item $\mathrm{pc}\gets \mathtt{HALT}$ and execution stops.
\end{itemize}

Intuitively, $\mathtt{LOAD}$ and $\mathtt{STORE}$ handle memory access through register-indirect addressing, $\mathtt{LOADI}$ loads immediate constants, $\mathtt{ADD}$/$\mathtt{SUB}$ perform modular arithmetic, $\mathtt{AND}$/$\mathtt{XOR}$ enable bitwise operations, $\mathtt{SHL}$/$\mathtt{SHR}$ provide bit shifts.  $\mathtt{BRZ}$ (branch 
if zero) enables conditional branching, $\mathtt{JMP}$ provides unconditional jumps, and $\mathtt{HALT}$ terminates execution.

\paragraph{Programs and Configurations.}
A \emph{program} is a pair $\mathcal{P}=(I_0,\ldots,I_\ell,\mathrm{addr})$ with $I_i\in\Instr$ and a partial label table $\mathrm{addr}:\Lab\rightharpoonup\{0,\ldots,\ell\}$ mapping each declared label to its
instruction index (injective). The program is
\emph{well formed} if whenever some $I_i$ equals $\mathtt{JMP}\ L$ or $\mathtt{BRZ}\ r,L$, then
$L\in\operatorname{dom}(\mathrm{addr})$. Code is immutable during execution and independent of~$n$ (and thus the RAM model considered here is uniform). A \emph{configuration} is $\sigma=(\mathrm{pc},R,M)$ where $\mathrm{pc}\in\{0,\ldots,\ell\}\cup\{\mathtt{HALT}\}$ is the program counter, $R\in\mathbb{U}^{k+1}$ is the register file, and $M:\mathcal{A}\to\mathbb{U}$ is memory. Input occupies $M[0..n-1]$; output is read from a designated location upon termination.

\paragraph{Initialization.}
Given an input instance of length $n$, initialization proceeds as follows:
\begin{enumerate}[leftmargin=1.2em, itemsep=1pt, topsep=2pt]
\item Build the label table $\mathrm{addr}$ from the loaded code and check well-formedness (every label operand in the code must appear exactly once as a declared label).
\item Zero-initialize memory $M$ and write the input into a designated block (e.g., $M[0..n-1]$) using the agreed-upon encoding.
\item Set all registers to zero: $R_i\gets 0$ for $i\in\{0,\ldots,k\}$.
\item Set the program counter to the first instruction: $\mathrm{pc}\gets 0$.
\end{enumerate}
The choice $w\ge \lceil\log_2 S(n)\rceil$ ensures that register-indirect addressing is well-typed: a bracketed operand $[s]$ uses $R_s$ as an address in $\mathcal{A}$.

\paragraph{Execution Cycle.}
While $\mathrm{pc}\neq\mathtt{HALT}$ and no trap occurs, the machine advances in discrete steps. Each successful step costs one time unit. Let $I_{\mathrm{pc}}$ denote the instruction at index $\mathrm{pc}$. Each step follows the fetch-decode-execute-commit cycle:

\begin{enumerate}[leftmargin=1.2em, itemsep=1pt, topsep=2pt]
\item \textbf{Fetch}: Read the current instruction $I \leftarrow I_{\mathrm{pc}}$. If $\mathrm{pc}\notin\{0,\ldots,\ell\}$, the run is invalid and we define this as a trap.

\item \textbf{Decode and read operands}: Parse the opcode and operands of $I$ without changing the machine state. Unbracketed registers $r,s$ denote their current word values $R_r,R_s\in\mathbb{U}$ (used as data). A bracketed operand $[s]$ denotes the candidate address $a\gets R_s$. An immediate $c\in\Imm$ is interpreted as $c\bmod 2^w$. A label $L$ resolves to $\mathrm{addr}(L)$ (guaranteed by well-formedness). No writes occur in this phase.

\item \textbf{Execute}: Apply the instruction semantics of $I$ to compute a finite \emph{write-set} $W$ (register and/or memory locations with their new values) and the \emph{next program counter} $\mathrm{pc}_{\text{next}}$. For memory-referencing instructions, a bracketed operand $[s]$ is valid only if $a=R_s\in\mathcal{A}$; otherwise a trap occurs. By default $\mathrm{pc}_{\text{next}}=\mathrm{pc}+1$, except for jumps/branches/halting which set it to $\mathrm{addr}(L)$ (or $\mathtt{HALT}$).

\item \textbf{Commit (writeback)}: Atomically apply the writes in $W$ to $(R,M)$ and then set $\mathrm{pc}\gets \mathrm{pc}_{\text{next}}$. Atomicity means all effects of the step become visible only at the end of the step.

\item \textbf{Cost and continuation}: If no trap occurred, charge one unit of time for this step and proceed to the next; otherwise the run aborts (abnormal termination), and only successfully committed steps are counted.
\end{enumerate}

\paragraph{Termination and Complexity.}
Execution halts when $\mathrm{pc}=\mathtt{HALT}$. The algorithm's output is read from the designated output location(s) in memory (or registers) as specified by the program. Under the assumptions above and for well-formed programs with legal memory accesses, the step relation is deterministic and yields a unique next state at each iteration. The \emph{time complexity} of an algorithm is the number of executed instructions before halting. A \emph{trap} aborts the run immediately (abnormal termination); only successfully committed steps are counted in time.

The RAM model defined here is polynomially equivalent to
bit-complexity RAM (a $\Theta(\log n)$ factor separates their running times) and to richer word-RAMs
that add $\mathtt{MUL}/\mathtt{DIV}/\mathtt{POPCNT}/\mathtt{CLZ}$ (whose presence typically improves
only by constant or $\log\log n$ factors).



\begin{algorithm}[t]
   \caption{Single-Processor Execution (Word-RAM semantics with PID init)}
   \label{alg:single-processor}
   \emph{Note.} In PRAM, $\mathtt{STORE}$ generates a pending write committed at the end of the round under the $\CREW$ rule. In the single-processor case, the store can be applied immediately.
   \begin{algorithmic}[1]
   \Require Program $\mathcal{P}=(I_0,\dots,I_\ell,\mathrm{addr})$, shared memory $M:\mathcal{A}\!\to\!\mathbb{U}$, word size $w$, processor id $\mathsf{pid}$, processor budget $P(n)$ (optional)
   \State \textbf{Init:} $pc \gets 0$;\quad $R[j]\gets 0$ for all $j$;\quad $R[0]\gets \mathsf{pid}$;\quad \textbf{optional: }$R[1]\gets P(n)\bmod 2^w$
   \While{$pc \neq \mathtt{HALT}$}
     \State $I \gets I_{pc}$ \Comment{fetch}
     \State $pc_{\text{next}} \gets pc + 1$ \Comment{default fall-through}
     \If{$I$ is \texttt{LOAD} $r,[s]$} \Comment{decode}
        \State $a \gets R[s]$
        \If{$a \notin \mathcal{A}$} \State \textbf{trap} \EndIf
        \State $R[r] \gets M[a]$ \Comment{execute}
     \ElsIf{$I$ is \texttt{STORE} $[s],r$}
        \State $a \gets R[s]$
        \If{$a \notin \mathcal{A}$} \State \textbf{trap} \EndIf
        \State $M[a] \gets R[r]$ \Comment{in PRAM semantics, this is a write event to be committed this round}
     \ElsIf{$I$ is \texttt{LOADI} $r,c$}
        \State $R[r] \gets c \bmod 2^w$
     \ElsIf{$I$ is \texttt{ADD} $r,s$}
        \State $R[r] \gets (R[r] + R[s]) \bmod 2^w$
     \ElsIf{$I$ is \texttt{SUB} $r,s$}
        \State $R[r] \gets (R[r] - R[s]) \bmod 2^w$
     \ElsIf{$I$ is \texttt{AND} $r,s$}
        \State $R[r] \gets R[r] \wedge R[s]$ \Comment{bitwise AND}
     \ElsIf{$I$ is \texttt{XOR} $r,s$}
        \State $R[r] \gets R[r] \oplus R[s]$ \Comment{bitwise XOR}
     \ElsIf{$I$ is \texttt{SHL} $r,s$}
        \State $h \gets R[s] \bmod w$
        \State $R[r] \gets (R[r] \ll h)\bmod 2^w$
     \ElsIf{$I$ is \texttt{SHR} $r,s$}
        \State $h \gets R[s] \bmod w$
        \State $R[r] \gets \lfloor R[r]/2^h \rfloor$ \Comment{logical right shift, zero-fill}
     \ElsIf{$I$ is \texttt{BRZ} $r,L$}
        \If{$R[r] = 0$} \State $pc_{\text{next}} \gets \mathrm{addr}(L)$ \EndIf
     \ElsIf{$I$ is \texttt{JMP} $L$}
        \State $pc_{\text{next}} \gets \mathrm{addr}(L)$
     \ElsIf{$I$ is \texttt{HALT}$\;$}
        \State $pc_{\text{next}} \gets \mathtt{HALT}$
     \Else
        \State \textbf{trap} \Comment{unknown opcode or malformed operands}
     \EndIf
     \State $pc \gets pc_{\text{next}}$ \Comment{commit PC; regs/memory updated in each branch above}
   \EndWhile
   \end{algorithmic}
   \end{algorithm}
   
   




\subsection{Extension to CREW PRAM}

We extend the Word-RAM defined above to a parallel machine with a processor-budget function $P:\mathbb{N}\to\mathbb{N}$ (typically $P(n)\le n^{\mathcal{O}(1)}$). All word-size/address-width assumptions, the instruction alphabet $\Instr$, the immediate-set restriction, and the \emph{single-processor} instruction semantics are exactly as in the Word-RAM subsection.

\paragraph{Processors and Shared State.}
Processors are indexed by $i\in\{0,\ldots,P(n)-1\}$. Each processor has its own program counter and register file; memory is shared:
\[
\Sigma \;=\; \bigl( (\pc_0,\ldots,\pc_{P(n)-1}),\ (R^0,\ldots,R^{P(n)-1}),\ M \bigr),
\]
where $\pc_i\in\{0,\ldots,\ell\}\cup\{\mathtt{HALT}\}$ and $R^i=(R^i_0,\ldots,R^i_k)\in\mathbb{U}^{k+1}$. All processors run the same program $\mathcal{P}=(I_0,\ldots,I_\ell,\addr)$.

\paragraph{Initialization (with Processor IDs).}
At time $t=0$:
\begin{enumerate}[leftmargin=1.2em, itemsep=1pt, topsep=2pt]
\item Build $\addr$ and check well-formedness (as in Word-RAM); zero-initialize $M$ and write the input block.
\item For each $i\in\{0,\ldots,P(n)-1\}$, set $\pc_i\gets 0$ and clear registers; then write \emph{processor-local identifiers:} \underline{$R^i_0\gets i$} and, if $P(n)\le 2^{w(n)}$, optionally \underline{$R^i_1\gets P(n)\bmod 2^{w(n)}$}. All other $R^i_j\gets 0$.
\end{enumerate}
These two words are provided so that processors can branch, partition work, and self-disable if unused.

\paragraph{Concurrent-Access Policy (CREW).}
Multiple processors may \emph{read} the same address in the same round; \emph{writes must be exclusive}: if two or more writes target the same address in a round, the run traps (abnormal termination).

\paragraph{Round Semantics (Referencing the Word-RAM Step).}
Each active processor executes exactly one instruction using the single-processor Word-RAM step semantics; the only new aspects are (i) simultaneous execution by many processors and (ii) end-of-round memory commit subject to the $\CREW$ policy. Execution proceeds in synchronous rounds $t=0,1,2,\ldots$ with state $\Sigma_t=((\pc_i^t)_i,(R^{i,t})_i,M^t)$. In round $t$, each active processor $i$ with $\pc_i^t\in\{0,\ldots,\ell\}$ executes instruction $I_{\pc_i^t}$ on its local snapshot $(\pc=\pc_i^t,\ R=R^{i,t})$ and shared memory $M^t$. After all processors compute their local effects, the round commits: register writebacks $R^{i,t}\to R^{i,t+1}$ (independently), then memory writes to $M^{t+1}$ under $\CREW$ constraints, finally program counter updates $\pc_i^{t+1}$.

\paragraph{Termination and Cost Measures.}
The parallel run terminates when $\pc_i^t=\mathtt{HALT}$ for all $i$ (or traps on an invalid access/conflict). One round costs one unit of \emph{parallel time}. The \emph{work} is the total number of executed instructions $W(n)=\sum_t |\{\,i:\pc_i^t\in\{0,\ldots,\ell\}\,\}|$, and the \emph{span} is $T_\infty(n)$ (the critical-path length). For $P(n)$ processors the Brent bound holds~\citep{jaja1992parallel}:
\[
T_{P(n)}(n)\ \le\ \Big\lceil \frac{W(n)}{P(n)} \Big\rceil + T_\infty(n).
\]
A single-processor run ($P(n)=1$) coincides with the Word-RAM model.

\paragraph{Remarks (on $P(n)$, Processor IDs, and Unused Processors).}
(1) \emph{Uniformity:} the code $\mathcal{P}$ and the immediate set $\Imm$ are independent of $n$; only the hardware parameters $w(n)$, $S(n)$, and $P(n)$ scale with input size. (2) \emph{Processor IDs:} the values $i$ and $P(n)$ are provided via initialization registers ($R^i_0$ and optionally $R^i_1$) for branching and work partitioning; programs may copy/overwrite them. (3) \emph{Unused processors:} if an algorithm needs only $m(n)\le P(n)$ processors, each processor executes a short self-filter based on $i$ (e.g., \texttt{if $i\ge m(n)$ then HALT}), or computes its assigned block; processors with empty assignment halt in $O(1)$ rounds, which does not affect the asymptotic parallel time.

The algorithm for a single-processor in PRAM is shown in \Cref{alg:single-processor}.


\section{Encoder Transformer Architecture} \label{appendix:encoder}

This section presents encoder-only Transformers, which form the backbone of MDM. We will first establish the sequence-wise extension operation, then define the core components, including bidirectional self-attention, multi-head mechanisms, and feed-forward layers, before assembling the complete architecture.

We consider an encoder-only Transformer with $H$ heads, $L$ layers, hidden size $d$, and feed-forward width $w$. We will use the following notations:

\begin{definition}[Position-Indexed Seq-to-Embedding Function]\label{def:Fpos}
    For a set $B$, let $\F(B)$ denote the set of all functions $\psi$ such that for every sequence $x=(x_1,\ldots,x_n)\in\Sigma^*$ and every index $i\in[n]$, the value $\psi(x,i)\in B$ is defined. We write this succinctly as
    \begin{equation}
        \psi:(\Sigma^*,\mathbb N)\to B.
    \end{equation}
and call this \emph{position-indexed seq-to-embedding function}. We also define $\F = \cup_{d\in \mathbb{N}^+} \F(\mathbb{R}^d)$ as the union of all such classes across real spaces of all output dimensions.
\end{definition}
    
\begin{definition}[Canonical Extension to Seq-to-Seq Function]\label{def:seq-ext}
Given a position-indexed seq-to-embedding function $\psi \in \F(B)$, its \emph{canonical extension} is defined as:
\begin{equation}
    \overline{\psi}:\Sigma^* \to B^*
    \quad\text{where}\quad
    [\overline{\psi}(x)]_i = \psi(x,i)\qquad(i\in[|x|]).
\end{equation}
For elementwise functions $g:\mathbb{R}^d\to\mathbb{R}^{d'}$, we define $\overline{g}:(\mathbb{R}^d)^*\to(\mathbb{R}^{d'})^*$ by $[\overline{g}(h_{1:n})]_i=g(h_i)$, which is a special case where $\psi(h_{1:n},i) = g(h_i)$ (ignoring cross-position context). When the arity is clear, we reuse the bar notation for both position-indexed and elementwise extensions.
\end{definition}

We now define the individual components of encoder Transformers:

\paragraph{Bidirectional Self-Attention.} The key difference from decoder Transformers is bidirectional attention, where each position can attend to all positions in the sequence. Let $d_h$ be the head dimension. For $W_Q,W_K,W_V\in\mathbb{R}^{d_h\times d}$ and $W_O\in\mathbb{R}^{d\times d_h}$, we define single-head attention on sequence of embeddings $h_{1:n}\in(\mathbb{R}^d)^n$ for any $n\in\mathbb{N}^+$:
\begin{align}
q_i&=W_Q h_i,\quad k_j=W_K h_j,\quad v_j=W_V h_j \\
[\sa_\theta(h_{1:n})]_i &= W_O\sum_{j=1}^n \alpha_{ij} v_j
\end{align}
with $\alpha_{i,\cdot}=\softmax\big((q_i^\top k_j)_{j=1}^n\big)$, $\theta=(W_Q,W_K,W_V,W_O)$. Position $i$ attends to all $j\in[n]$ without causal restrictions. We use standard $1/\sqrt{d_h}$ scaling.

\paragraph{Multi-Head Attention.} With $\theta_{\mha}=(\theta^{(1)},\ldots,\theta^{(H)})$, we combine heads via summation:
\begin{equation}
[\mha_{\theta_{\mha}}(h_{1:n})]_i = \sum_{t=1}^{H} [\sa_{\theta^{(t)}}(h_{1:n})]_i
\end{equation}
for any $i\in[n]$. Note that this differs from practical implementations which concatenate heads with dimension $d/H$ each, but maintains equivalent theoretical expressivity.

\paragraph{Feed-Forward and Projection.} Let $w=d_{\ff}$. For $W_1\in\mathbb{R}^{w\times d}$ and $W_2\in\mathbb{R}^{d\times w}$:
\begin{equation}
\ff_\theta(h)=W_2\,\sigma(W_1 h)
\end{equation}
For output projection, $\proj_\vartheta:\mathbb{R}^d\to\mathbb{R}^{|\Sigma|}$ with $\proj_\vartheta(h)=\vartheta h$ and $\vartheta\in\mathbb{R}^{|\Sigma|\times d}$. We apply these via sequence-wise extension: $\overline{\ff}$ and $\overline{\proj}$. 

For AP-MDM as described in \Cref{sec:any_process}, besides the above heads for $\unmask$, it would require three additional binary classification heads on top of the final layer: $\proj_R: \mathbb{R}^d \to \mathbb{R}$ for $\remask$, $\proj_I: \mathbb{R}^d \to \mathbb{R}$ for $\inser$, and $\proj_D: \mathbb{R}^d \to \mathbb{R}$ for $\delete$ operations, each followed by sigmoid activation. Therefore, $\proj_\vartheta$ is a mapping from $\mathbb{R}^d$ to $\mathbb{R}^{|\Sigma|+3}$.

\paragraph{Embeddings.} Define token embedding $\te:\Sigma\to\mathbb{R}^d$ and positional embedding $\pe:\mathbb{N}^+\to\mathbb{R}^d$ (which can be flexibly chosen and will be specified when used). Combined as $\overline{\te}+\overline{\pe}$. We write $\te,\pe$ for their sequence-wise extensions when clear from context.

\paragraph{Residual Connections.} The identity function $\Id_d: \mathbb{R}^d \to \mathbb{R}^d$ is defined by $\Id_d(x) = x$. A \emph{residual connection} is defined as $f + \Id_d$, where $\Id_d$ is the identity function.


Next, we assemble these components into the encoder transformer architecture:

\begin{definition}[Encoder Transformer layer]
\label{def:enc-layer}
An encoder layer is defined as:
\begin{equation}
\EncTF_{\mha,\ff} = (\overline{\ff_{\ff}}+\overline{\Id_d})\circ(\mha_{\theta_\mha}+\overline{\Id_d}):(\mathbb{R}^d)^*\to(\mathbb{R}^d)^*
\end{equation}
\end{definition}

\begin{definition}[Encoder Transformer]
\label{def:encoder}
With parameters $\theta=(\theta_{\te}$,$\theta_{\pe}$,$(\theta^{(\ell)}_{\mha})_{\ell=1}^L$,$ (\theta^{(\ell)}_{\ff})_{\ell=1}^L$,$\theta_{\proj})$, the encoder transformer is:
\begin{equation}
\Enc_\theta = \overline{\proj_{\theta_{\proj}}}\circ\left(\bigcirc_{\ell=1}^L \EncTF_{\theta^{(\ell)}_{\mha},\theta^{(\ell)}_{\ff}}\right) \circ\big(\te_{\theta_{\te}}+\pe_{\theta_{\pe}}\big)
\end{equation}
The model applies embeddings, then $L$ encoder layers, then position-wise projection to vocabulary logits. Output length equals input length $n$.
\end{definition}



\section{Key Tool: Encoder Full-Access Sequence Processing (\efasp)} \label{appendix:fasp}


In this section, we develop Full-Access Sequence Processing for encoders (\efasp), a programming language whose programs describe the construction process of seq-to-embedding functions that are equivalent to those computed by encoder-only Transformers. This extends the \texttt{FASP} framework originally developed for decoder-only Transformers in \citet{yang2025pencil}. Similar connections have also been established in \citet{weiss2021thinking,yang2024counting}.

$\efasp$ is the key technical tool that will be used to prove \Cref{thm:main_mdm}, \Cref{thm:any_order} and \Cref{thm:main_apmdm}.



\subsection{Definition of $\efasp$}\label{subsec:encoder_fasp_definition}

\paragraph{Notations.} Recall that in \Cref{appendix:encoder}, we defined the position-indexed seq-to-embedding function space $\F(B)$ as the set of all functions $\psi$ that map a sequence and a position index to an element in $B$:
\begin{equation}
    \psi:(\Sigma^*,[\cdot])\to B
\end{equation}
That is, for every sequence $\mathbf{x}=(x_1,\ldots,x_n)\in\Sigma^*$ and every index $i\in[n]$, we have $\psi(\mathbf{x},i)\in B$. We also define $\F = \cup_{d\in \mathbb{N}^+} \F(\mathbb{R}^d)$ as the union of all such classes across real spaces of all output dimensions. For any position-indexed function $\psi \in \F(B)$, its canonical extension $\seq{\psi}:\Sigma^* \to B^*$ is defined by $[\seq{\psi}(\mathbf{x})]_i = \psi(\mathbf{x},i)$ for $i\in[|\mathbf{x}|]$. This allows us to convert position-indexed functions to sequence-to-sequence functions when needed. 

Also recall that $\pe:\mathbb{N}^+\to\mathbb{R}^{d}$ is a positional embedding, and we additionally define $\op_{\act}$ as a class of activation functions. We formally define $\efasp$ as follows:

\begin{definition}[Encoder-FASP]\label{def:efasp}   
    An $\efasp$ program is a sequence of position-indexed seq-to-embedding functions $\psi_1,\ldots,\psi_T$ constructed inductively. At each step $t\in[T]$, the program maintains a set of defineable position-indexed seq-to-embedding functions $\dfnb_t$, and defines a new function by applying operators to functions in $\dfnb_t$. We define the defineable functions at step $t\in[T]$:
    \begin{align}
        \dfnb_t \triangleq \{\te, \pe\} \cup \{\psi_i \mid  1\le i\le t-1\}
    \end{align}
    where $\te(\mathbf{x},i) = \te(x_i)$ and $\pe(\mathbf{x},i) = \pe(i)$ are the token and positional embedding functions respectively, viewed as position-indexed seq-to-embedding functions. $\psi_t$ at step $t$ has to be defined by applying one of the following four \emph{primitive} operators on already-defined functions from $\dfnb_t$:
    
    \begin{enumerate}
    \item \textbf{Concatenation}: For $\psi, \psi' \in \dfnb_t$ with $\psi \in \F(\mathbb{R}^{d_1})$ and $\psi' \in \F(\mathbb{R}^{d_2})$, define
    \begin{equation}
        [\psi, \psi'](\mathbf{x}, i) = (\psi(\mathbf{x}, i) \| \psi'(\mathbf{x}, i)) \in \mathbb{R}^{d_1 + d_2}
    \end{equation}
    where $\|$ denotes vector concatenation.
    
    \item \textbf{Linear Projection}: For $\psi \in \dfnb_t$ with $\psi \in \F(\mathbb{R}^d)$ and matrix $W \in \mathbb{R}^{d' \times d}$, define
    \begin{equation}
        (W \circ \psi)(\mathbf{x}, i) = W \cdot \psi(\mathbf{x}, i) \in \mathbb{R}^{d'}
    \end{equation}
    
    \item \textbf{Nonlinear Activation}:\footnote{We allow multi-variable activation functions like Gated ReLU (ReGLU), $x,y\mapsto x[y]_+$.} For $\psi \in \dfnb_t$ with $\psi \in \F(\mathbb{R}^d)$ and $\sigma \in \op_\act$, define
    \begin{equation}
        (\sigma \circ \psi)(\mathbf{x}, i) = \sigma(\psi(\mathbf{x}, i))
    \end{equation}
    
    
    \item \textbf{Encoder Average-Hard Attention}: For $q, k \in \F(\mathbb{R}^d)$ and $v \in \F(\mathbb{R}^{d'})$ where $q, k, v \in \dfnb_t$, define
    \begin{equation} \label{eq:aha}
        \aha(q, k, v)(\mathbf{x}, i) = \frac{1}{|A_i|} \sum_{j \in A_i} v(\mathbf{x}, j)
    \end{equation}
    where $A_i = \arg\max_{j \in [|\mathbf{x}|]} \langle q(\mathbf{x}, i), k(\mathbf{x}, j) \rangle$ and ties are averaged uniformly. This attention can be seen as a pecial case of standard softmax attention with temperature approaching 0~\citep{merrill2022saturated}.
    \end{enumerate}
    
Finally, when we want to use $\efasp$ to define a function mapping from a sequence of tokens $\Sigma^*$ and a position index $i$ to a single token in $\Sigma$, we can define $\psi \in \F(\mathbb{R}^{|\Sigma|})$ and return $\arg\max \psi(\mathbf{x}, i)$ (the token corresponding to the largest logit at the position $i$).\footnote{We could assume an arbitrary order to break ties, but we omit this for simplicity. In our examples we always ensure the argmax is unique.}
\end{definition}

We denote the set of all position-indexed seq-to-embedding functions defineable by $\efasp$ with position embedding $\pe$ and activation functions $\op_\act$ as $\efasp[\pe;\op_{\act}]$, where $\pe$ can be either $\bipe$ or $\SEQ$. The expressivity of $\efasp$ depends on the specific positional embedding and activation functions used.


\subsection{Equivalence with Encoder Transformer}
\label{subsec:equiv_encoder_tf}

We now establish the equivalence between $\efasp$ and encoder-only Transformers, and define the specific instantiation considered in the proof of this paper.

\begin{definition}[Encoder Transformer Function Class]\label{def:encoder_tf_class}
Let $\F_{\text{EncTF}[\pe;\op_{\act}]}$ be the class of seq-to-embedding functions that can be expressed by encoder-only Transformers of finite depth, where the positional embedding uses $\pe$ (either $\bipe$ or $\SEQ$), feed-forward layers use activation functions from $\op_{\act}$, attention layers use average-hard attention as defined in \Cref{eq:aha}, and all intermediate computations use finite precision arithmetic.
    \end{definition}
    
It is straightforward to see that both variants are equivalent to their corresponding Transformer function classes: 

\begin{lemma}[Equivalence of $\efasp$ and Encoder Transformers]\label{thm:efasp_equiv_encoder}
For any positional embedding $\pe \in \{\bipe, \SEQ\}$ and activation function class $\op_{\act}$, the following equivalence holds:
\begin{equation}
\efasp[\pe;\op_{\act}] = \F_{\text{EncTF}[\pe;\op_{\act}]}
\end{equation}
\end{lemma}

\begin{proof}[Proof Sketch]
\textbf{Forward direction}: Each $\efasp$ primitive operator (concatenation, linear projection, nonlinear activation, encoder attention) directly corresponds to operations in encoder Transformers. Concatenation involves merging multiple smaller Transformers into a larger Transformer that produces the same output. \textbf{Reverse direction}: Any encoder Transformer can be expressed as an $\efasp$ program by decomposing each layer into primitive operations.

The detailed proof, including the treatment of closed operators and inductive construction, is invariant to decoder or encoder Transformers, and thus is identical to \citet{yang2025pencil}; we omit the details.
\end{proof}

Intuitively, this equivalence holds because $\efasp$ 
programs capture the computational 
structure of encoder Transformers. Each step in an 
$\efasp$ program corresponds to defining a new seq-to-
embedding function by applying primitive operators to 
previously defined functions, which mirrors how 
smaller Transformers are constructed into a deeper  and wider Transformer that produces the same output. The Transformer corresponding to the program is of depth $\mathcal{O}(1)$ (given constant $T$) and embedding size $\mathcal{O}(\max\{d_\pe, d_\te\})$, by construction in the proof of \Cref{thm:efasp_equiv_encoder}.

\subsection{Two Variants of $\efasp$}

Throughout this paper, we consider two variants of $\efasp$ based on different positional embeddings, both using the same activation functions.

\paragraph{Variant 1: Binary Positional Encoding}
We define $\efasp$ with binary positional embedding $\bipe: \mathbb{N}^+ \to \{0,1\}^{\lceil \log_2 S(n) \rceil}$:
\begin{equation}
\bipe(i) = \text{binary representation of } i \text{ using } \lceil \log_2 S(n) \rceil \text{ bits}
\end{equation}
This representation uses $\lceil \log_2 S(n) \rceil$ bits to represent all possible positions within the maximum context length $S(n)$, and aligns with the address representation in PRAM (\Cref{appendix:pram}) for efficient bitwise arithmetic operations. 
% \zhiyuan{here we should restrict the domain range to $S(n)$ instead of $\mathbb{N}^+$. consider call it a different name, maybe $\efasp_n$ }

\paragraph{Variant 2: Integer Positional Encoding}
We also define $\efasp$ with integer positional embedding $\SEQ: \mathbb{N}^+ \to \mathbb{N}^+$:
\begin{equation}
\SEQ(i) = i
\end{equation}
This is the identity mapping over $\mathbb{N}^+$ that directly uses the position index as a scalar feature, as considered in the original decoder-only \texttt{FASP} framework \citep{yang2025pencil}.

\paragraph{Activation Functions}
Both variants use the same class of activation functions $\op_{\act} = \{\text{ReGLU}\}$, where Gated ReLU (ReGLU)~\citep{dauphin2017language} is defined as $\text{ReGLU}(x,y) = x \cdot [y]_+ = x \cdot \max(y, 0)$ for $x,y \in \mathbb{R}$. With Gated ReLU as the primitive activation, we can express ReLU and multiplication operations through the following identities:
\begin{align}
\text{ReLU}(x) = \text{ReGLU}(x,1), \quad x \times y = \text{ReGLU}(x,y) - \text{ReGLU}(x,-y)
\end{align}
Therefore, having ReGLU allows us to express both ReLU and multiplication (reverse is also true), making both variants equivalent:
\begin{align}
\efasp[\bipe; \text{ReGLU}] &= \efasp[\bipe; [\cdot]_+, \times] \\
\efasp[\SEQ; \text{ReGLU}] &= \efasp[\SEQ; [\cdot]_+, \times]
\end{align}
where $[\cdot]_+$ and $\times$ are the ReLU and multiplication respectively.

\subsection{Original Supported Operators}

With the four primitive operators in $\efasp$ and the activation functions defined above, the following operators can be included in both variants $\efasp[\bipe; [\cdot]_+, \times]$ and $\efasp[\SEQ; [\cdot]_+, \times]$, adapted from the decoder version of \texttt{FASP}:

\emph{Arithmetic Operators}
\begin{itemize}
\item $\texttt{add}(\psi_1, \psi_2) = \psi_1 + \psi_2$: Element-wise addition
\item $\texttt{minus}(\psi_1, \psi_2) = \psi_1 - \psi_2$: Element-wise subtraction  
\item $\texttt{multi}(\psi_1, \psi_2) = \psi_1 \times \psi_2$: Element-wise multiplication
\item $\texttt{max}(\psi_1, \psi_2)$: Element-wise maximum
\item $\texttt{min}(\psi_1, \psi_2)$: Element-wise minimum
\end{itemize}

\emph{Boolean Operators}
For $\psi_1, \psi_2 \in \F(\{0,1\})$:
\begin{itemize}
\item $\texttt{and}(\psi_1, \psi_2) = \min(\psi_1, \psi_2)$: Logical AND
\item $\texttt{or}(\psi_1, \psi_2) = \lnot(\lnot \psi_1 \land \lnot \psi_2)$: Logical OR
\item $\texttt{not}(\psi) = 1 - \psi$: Logical NOT
\item $\texttt{xor}(\psi_1, \psi_2)$: Logical XOR
\end{itemize}

\emph{Comparison Operators}
For $\psi_1, \psi_2 \in \F(\mathbb{Z})$:
\begin{itemize}
\item $\texttt{leq}(\psi_1, \psi_2) = [\psi_2 - \psi_1 + 1]_+ - [\psi_2 - \psi_1]_+$: Less than or equal
\item $\texttt{geq}(\psi_1, \psi_2) = \texttt{leq}(\psi_2, \psi_1)$: Greater than or equal
\item $\texttt{eq}(\psi_1, \psi_2) = \texttt{leq}(\psi_1, \psi_2) \land \texttt{leq}(\psi_2, \psi_1)$: Equality
\item $\texttt{lt}(\psi_1, \psi_2) = \texttt{leq}(\psi_1, \psi_2 - 1)$: Less than
\item $\texttt{gt}(\psi_1, \psi_2) = \texttt{lt}(\psi_2, \psi_1)$: Greater than
\end{itemize}

\emph{Sequence Aggregation Operators}
\begin{itemize}
\item $\texttt{seq\_max}(\psi)$: Returns the maximum value across all positions in the sequence
\item $\texttt{seq\_min}(\psi)$: Returns the minimum value across all positions in the sequence  
\item $\texttt{seq\_and}(\psi) = \texttt{seq\_min}(\psi)$: Logical AND across all positions
\item $\texttt{seq\_or}(\psi) = \texttt{seq\_max}(\psi)$: Logical OR across all positions
\item $\texttt{seq\_sum}(\psi)$: Sum of values across all positions (requires $\log n$ positional embedding)
\item $\texttt{seq\_avg}(\psi) = \frac{1}{n}\sum_{j=1}^{n} \psi(x_{1:j})$: Average across all positions
\end{itemize}

\emph{Positional Operators}
\begin{itemize}
\item $\texttt{is\_first}(i) = \mathbf{1}[i=1]$: Indicator for first position
\item $\texttt{inv\_seq\_len}(i) = 1/n$: Inverse of sequence length
\item $\texttt{is\_pos\_k}(i) = \mathbf{1}[i=k]$: Indicator for position $k$
\end{itemize}

\emph{Control Flow Operators}
\begin{itemize}
\item $\texttt{if\_then\_else}(\psi_{\text{cond}}, \psi_{\text{true}}, \psi_{\text{false}})$ or $\texttt{ite}(\psi_{\text{cond}}, \psi_{\text{true}}, \psi_{\text{false}})$: If-then-else conditional selection
\end{itemize}

\emph{Attention Variants}
\begin{itemize}
\item $\texttt{aha}(q, k, v)$: Standard average-hard attention (encoder bidirectional)
\item $\texttt{rha}(q, k, v)$: Rightmost-hard attention (breaks ties by selecting rightmost position)
\item $\texttt{rightmost\_exact\_match}(q, k, v)$: Rightmost exact match (returns default if no exact match)
\end{itemize}

\subsection{Additional Operators and Justifications}


Next we give the semantics of some additional operators used in the PRAM simulation programs and justify their closure in the $\efasp$ framework.

% \paragraph{Token and Embedding Operators.}
% These operators handle token-level operations and embedding conversions, all implementable using $\efasp$ primitives:
% \begin{itemize}
% \item $\texttt{embed}(\sigma)$: For token $\sigma \in \Sigma$, returns the embedding vector $\te(\sigma) \in \mathbb{R}^d$. This is a primitive operator in $\efasp$.
% \item $\texttt{bin}(\psi)$: For $\psi \in \F(\mathbb{R}^d)$, converts real-valued embeddings to binary representations $\{0,1\}^d$ using ReLU thresholding. Implemented as $\texttt{bin}(\psi) = [\psi - 0.5]_+$ followed by $\min(\cdot, 1)$.
% \item $\texttt{ite}(\psi_{\text{cond}}, \psi_{\text{true}}, \psi_{\text{false}})$: If-then-else conditional selection. Already included in original $\efasp$ operators.
% \item $\texttt{int}(\psi)$: For $\psi \in \F(\{0,1\}^m)$, converts $m$-bit binary to integer via $\sum_{k=1}^m 2^{k-1} \psi_k$. Implementable using linear projection.
% \end{itemize}

% \paragraph{Positional and Sequence Operators.}
% These operators handle position-based computations and can be expressed using attention mechanisms and arithmetic operations:
% \begin{itemize}
% \item $\texttt{is\_last}(\mathbf{x}, i)$: Indicator for last position, implementable as $\mathbf{1}[i = |\mathbf{x}|]$ using sequence length computation.
% \item $\texttt{rightmost\_sep\_pos}(\mathbf{x}, i)$: Uses $\texttt{rightmost\_exact\_match}$ with \texttt{[SEP]} token as query to find rightmost separator.
% \item $\texttt{distance\_to\_sep}(\mathbf{x}, i)$: Computed as $i - \texttt{rightmost\_sep\_pos}(\mathbf{x}, i)$ using subtraction.
% \item $\texttt{processor\_id}(\mathbf{x}, i)$: Implemented as $\lfloor \texttt{distance\_to\_sep}(\mathbf{x}, i) / 8 \rfloor$ using division and floor operations.
% \item $\texttt{inner\_processor\_id}(\mathbf{x}, i)$: Computed as $\texttt{distance\_to\_sep}(\mathbf{x}, i) \bmod 8$ using modulo operation.
% \end{itemize}

% \paragraph{PRAM-Specific Operators.}
% These operators provide PRAM simulation capabilities through learned transformations and global state tracking:
% \begin{itemize}
% \item $\texttt{get\_instruction}(\text{addr})$: Fetches instruction via learned mapping $W_{\mathrm{INSTR}}: \mathbb{R}^w \to \mathbb{R}^w$ where PRAM instruction set is hardcoded in parameters. Implementable as linear projection followed by $\texttt{bin}(\cdot)$.
% \item $\texttt{is\_sep}(\mathbf{x}, i)$: Indicator for \texttt{[SEP]} tokens, implemented as $\mathbf{1}[x_i = \texttt{[SEP]}]$ using equality comparison.
% \item $\texttt{all\_halt}(\mathbf{x})$: Global predicate using $\texttt{seq\_and}$ to check if all processor program counters equal \texttt{HALT\_CODE}.
% \end{itemize}

% \paragraph{Closure Justification.}
% All operators above are closed under the $\efasp$ framework: token operations use primitive embeddings and linear projections; positional operations combine attention mechanisms with arithmetic; PRAM-specific operations use learned parameters within linear transformations. The bit operations ($\ll$, $\gg$, $\&$, $\oplus$) used in the programs are already defined in the bitwise arithmetic operators below.

\subsubsection*{Bitwise Arithmetic Operators}

These operators are defined in the encoder-$\efasp$ framework using activations $\{[\cdot]_+, \times\}$ (equivalently ReGLU) and are independent of the specific positional embedding choice. All inputs and outputs are position-indexed seq-to-embedding functions in $\F(\{0,1\}^m)$ where $\psi(\mathbf{x}, i) \in \{0,1\}^m$ encodes an $m$-bit integer with LSB at coordinate 1. All arithmetic is modulo $2^m$.

\paragraph{Bitwise Addition.} 
Given $\psi_1, \psi_2 \in \F(\{0,1\}^m)$, write at position $(\mathbf{x}, i)$:
\begin{equation}
\psi_1(\mathbf{x}, i) =: \mathbf{a} = (a_1, \ldots, a_m), \quad \psi_2(\mathbf{x}, i) =: \mathbf{b} = (b_1, \ldots, b_m) \in \{0,1\}^m
\end{equation}
Bitwise addition is defined as adding two $m$-bit integers modulo $2^m$. This can be constructed  using the primitive operators (and other operators that are already defined) in \Cref{def:efasp}, which follows an approach similar to standard carry-lookahead, and is a constant-depth, polylogarithmic-width construction:

Define for $k \in [m]$ the local propagate/generate bits:
\begin{equation}
p_k = a_k \oplus b_k = a_k + b_k - 2a_k b_k, \quad g_k = a_k \wedge b_k = a_k b_k
\end{equation}

Let $S_0 = 0$ and $S_j = \sum_{t \leq j} p_t$ for $j \in [m]$ (computed by a single linear layer). For $1 \leq j < i \leq m$, define the interval-all-ones gate:
\begin{equation}
Q_{j,i} = \mathrm{eq}_0\left((S_{i-1} - S_j) - ((i-1) - j)\right)
\end{equation}
where $\mathrm{eq}_k(u) := 2\left([u-(k-\frac{1}{2})]_+ - 2[u-k]_+ + [u-(k+\frac{1}{2})]_+\right)$ equals 1 at $u = k$ and 0 at all other integers.

The carry into bit $i$ is:
\begin{equation}
C_i = \begin{cases}
0, & i = 1 \\
1 - \mathrm{eq}_0\left(\sum_{j=1}^{i-1} g_j Q_{j,i}\right), & i \geq 2
\end{cases}
\end{equation}

The sum bits are $s_i = p_i \oplus C_i = p_i + C_i - 2p_i C_i$. We define:
\begin{equation}
\texttt{bit\_add}_m(\psi_1, \psi_2)(\mathbf{x}, i) := \mathbf{s} = (s_1, \ldots, s_m) \in \{0,1\}^m
\end{equation}

\paragraph{Bitwise Subtraction.}
For $\psi_1, \psi_2 \in \F(\{0,1\}^m)$, define:
\begin{equation}
\texttt{bit\_minus}_m(\psi_1, \psi_2) := \texttt{bit\_add}_m(\psi_1, \neg\psi_2) \dotplus 1
\end{equation}
where $\neg$ is bitwise NOT (elementwise $1 - \cdot$) and ``$\dotplus 1$" adds the constant vector $\mathbf{e}_1 = (1, 0, \ldots, 0)$ via the same $\texttt{bit\_add}_m$.

\paragraph{Logical Shifts.}
Let $\psi \in \F(\{0,1\}^m)$ and $\tau \in \F(\{0,1\}^m)$. At position $(\mathbf{x}, i)$, write $\mathbf{a} = \psi(\mathbf{x}, i) = (a_1, \ldots, a_m)$ and define the shift amount:
\begin{equation}
t = \mathrm{int}(\tau) = \sum_{r=1}^m 2^{r-1} \tau_r \in \{0, \ldots, m\}
\end{equation}

For $k \in [m]$, we define:
\begin{align}
[\texttt{shift\_left}_m(\psi, \tau)]_k &= \sum_{s=0}^{\min\{m, k-1\}} \mathrm{eq}_s(t) \cdot a_{k-s} \\
[\texttt{shift\_right}_m(\psi, \tau)]_k &= \sum_{s=0}^{\min\{m, m-k\}} \mathrm{eq}_s(t) \cdot a_{k+s}
\end{align}
where out-of-range indices are treated as 0, and $\mathrm{eq}_s(\cdot)$ is the integer-equality gate realized by three ReLUs.

\paragraph{Complexity Analysis.}
All operators act locally at each position on $\F(\{0,1\}^m)$ without cross-position communication, and are composed from $\efasp$ primitives. Throughout $m = \Theta(\log n)$.

The witness enumeration method for bitwise addition requires: \textbf{(i)} one linear layer for $(p, g)$ and prefix sums $(S_j)$; \textbf{(ii)} one nonlinear layer for witnesses $Q_{j,i}$ (each uses 3 ReLUs) and products $g_j Q_{j,i}$; \textbf{(iii)} linear aggregation and threshold for carries $C_i$; \textbf{(iv)} local polynomial for $s_i = p_i \oplus C_i$. This achieves constant depth (3-4 layers) and width $\mathcal{O}(m^2) = \mathcal{O}((\log n)^2)$ (polylogarithmic in $n$). Bitwise subtraction uses two's complement and reuses the same addition circuit with identical complexity bounds. Logical shifts compute the shift amount $t$ and all candidate shifts in parallel, then use equality gates for selection, also achieving constant depth and $\mathcal{O}(m^2)$ width.

All constructions use only $\efasp$ primitives (linear projections, ReLU/ReGLU activations, multiplication). By the equivalence established in \Cref{appendix:fasp}, these are realizable by constant-depth encoder Transformers.

% \subsubsection*{Basic PRAM Operators}

% We now define the basic operators needed for PRAM simulation, working with the $\efasp[\SEQ; [\cdot]_+, \times]$ variant and the input format established in the Setup. All these operators are position-indexed sequence-to-embedding mappings that extract semantic information from the encoded PRAM state.

% \paragraph{Position Type Predicates.}
% These predicates identify the semantic role of each position in the input sequence.
% \begin{align}
% \mathrm{is\_pn}(\mathbf{x}, i) &:= \mathrm{is\_first}(\mathbf{x}, i) = \mathbf{1}[i = 1] \\
% \mathrm{is\_addr}(\mathbf{x}, i) &:= \neg \mathrm{is\_first}(\mathbf{x}, i) \wedge (\SEQ(\mathbf{x}, i) \bmod 2 = 0) \\
% \mathrm{is\_data}(\mathbf{x}, i) &:= \neg \mathrm{is\_first}(\mathbf{x}, i) \wedge (\SEQ(\mathbf{x}, i) \bmod 2 = 1) \\
% \mathrm{is\_mask}(\mathbf{x}, i) &:= \mathbf{1}[x_i = \mask]
% \end{align}
% These predicates partition the sequence positions into four disjoint types: $\mathrm{is\_pn}$ identifies the processor count position (position 1), $\mathrm{is\_addr}$ identifies address positions (even positions > 1), $\mathrm{is\_data}$ identifies data positions (odd positions > 1), and $\mathrm{is\_mask}$ identifies mask token positions.

% \paragraph{Global Information and Processor ID.}
% These operators provide global PRAM configuration information and assign processor IDs based on position.
% \begin{align}
% \mathrm{PN}(\mathbf{x}) &:= \mathrm{seq\_max}(\mathrm{ite}(\mathrm{is\_pn}(\mathbf{x}, \cdot), \mathrm{bin}(\te(\mathbf{x}, \cdot)), \mathbf{0}_w)) \in \{0,1\}^w \\
% \mathrm{proc\_id}(\mathbf{x}, i) &:= \SEQ(\mathbf{x}, i) \bmod \mathrm{int}(\mathrm{PN}(\mathbf{x}))
% \end{align}
% The first operator $\mathrm{PN}(\mathbf{x})$ extracts the total processor count from the sequence, while $\mathrm{proc\_id}(\mathbf{x}, i)$ assigns each position a processor ID by taking the position modulo the processor count. Here $\mathrm{int}(\cdot)$ converts the $w$-bit representation back to an integer value.

\paragraph{Instruction Access Operations.}
This operator enables instruction fetching from memory by address lookup, which is essential for PRAM simulation.
\begin{align}
\mathrm{get\_instruction}(\mathbf{x}, i) &:= \mathrm{bin}(W_{\mathrm{INSTR}} \circ \mathrm{ite}(\mathrm{is\_addr}(\mathbf{x}, \cdot), \mathrm{bin}(\te(\mathbf{x}, \cdot)), \mathbf{0}_w))(\mathbf{x}, i) \in \{0,1\}^w
\end{align}
where $W_{\mathrm{INSTR}}: \mathbb{R}^w \to \mathbb{R}^w$ is a learned linear transformation (MLP layer) that maps address bits to instruction bits. This operator first extracts the address bits from address positions (even positions > 1) by converting their token embeddings to binary representations, then applies the instruction lookup transformation $W_{\mathrm{INSTR}}$ to produce the corresponding instruction encoding. The PRAM instruction set (LOAD, STORE, ADD, SUB, etc., as defined in \Cref{appendix:pram}) is hardcoded into the parameters of $W_{\mathrm{INSTR}}$ during training, enabling the model to perform instruction fetching through learned address-to-instruction mappings.



% \paragraph{Additional operators.} We also consider the following operators that are not included in the decoder version of \texttt{FASP}, but are supported by our encoder version and used in our proofs:

% \emph{Bitwise Arithmetic Operators}

% For PRAM simulation and address computation, we define the following bitwise operations on $m$-bit binary representations where $m = O(\log n)$:

% \begin{itemize}
% \item $\texttt{bit\_add}_m(\psi_1, \psi_2)$: Binary addition of two $m$-bit addresses (with carry propagation)
% \item $\texttt{bit\_minus}_m(\psi_1, \psi_2)$: Binary subtraction using two's complement arithmetic  
% \item $\texttt{shift\_left}_m(\psi, k)$: Left logical shift by $k$ positions (equivalent to multiplication by $2^k$)
% \item $\texttt{shift\_right}_m(\psi, k)$: Right logical shift by $k$ positions (equivalent to division by $2^k$)
% \end{itemize}

% These operators can be implemented using polynomial-width MLPs with ReGLU activations by leveraging standard binary arithmetic algorithms. The key insight is that any Boolean function on $m$ bits can be computed by a two-layer network of width $2^m = \text{poly}(n)$, making these operations tractable for address-sized operands.   

\section{Proof of \Cref{thm:main_mdm}}

\begin{theorem}[MDM Simulation of PRAM, Formal] \label{thm:main_mdm_formal}
    For any PRAM program $\mathcal P = (I_0,\dots,I_\ell,\mathrm{addr})$ (with finite number of instructions $\ell$, and is uniform for all processors and input size $n$), that on input $\mathbf x_{\text{val}} \in \mathbb U^n$ with corresponding address $\mathbf x_{\text{addr}} \in \mathcal A^n$ that runs in $T(n)$ parallel time using at most $P(n)$ processors and outputs $\text{PRAM}_{\mathcal P}(\mathbf x_{\text{addr}}, \mathbf x_{\text{val}}) \in \mathbb U$ per procedure described in \Cref{appendix:pram}, there exists a bijection $\phi: \mathbb U \cup \mathcal A \rightarrow \Sigma$ and a special token $\texttt{[SEP]} \in \Sigma$, and a MDM with constant depth and $\log(n)$ embedding size encoder-only Transformer, on input $\mathbf{x} = ((\mathbf{z}_{2i}, \mathbf{z}_{2i+1})_{i=0}^{n-1}, \texttt{[SEP]}) \in \Sigma^{2n+1}$ where $\mathbf{z}_{2i} = \phi(\mathbf x_{\text{addr},i})$ and $\mathbf{z}_{2i+1} = \phi(\mathbf x_{\text{val},i})$, padded to $\mathcal O(P(n) \times T(n))$ context length, outputs $\phi(\text{PRAM}_{\mathcal P}(\mathbf x_{\text{addr}}, \mathbf x_{\text{val}}))$ with $\mathcal O(T(n))$ decoding steps.
\end{theorem}

% \zhiyuan{in the formal statement the width and depth of MDM with should definitely be mentioned, if the MDM is uniform (i.e., size does not depend on n), then the right way to state the theorem is to change the order of quantifiers, like "for any uniform PRAM, there is a MDM, such that for any n, ... "}

The proof demonstrates that AO-MDM can simulate any PRAM algorithm. The construction is based on $\efasp$, the programming language we developed, whose definable programs are equivalent to encoder-only Transformer function class (see \Cref{appendix:fasp}). We prove \Cref{thm:main_mdm} by: \textbf{(1)} defining the setup and input format for PRAM simulation; \textbf{(2)} constructing an $\efasp$ program that simulates PRAM execution in \Cref{alg:single-processor}.


\paragraph{Choice of Architecture / $\efasp$ Variant}
For this simulation, we use the $\efasp[\SEQ; [\cdot]_+, \times]$ variant with integer positional encoding rather than the binary variant. This choice is crucial because MDM's context length can be exponentially large (e.g., for NP-hard problems), while PRAM's actual memory usage remains polynomial. Using $\log n$ bits to represent positions would be insufficient when the context length $n$ itself grows exponentially with the problem size, even though PRAM addresses can still be represented in $\log S(n)$ bits. To avoid confusion between MDM's context length and PRAM's memory space, we use $S_{\text{MDM}}(n)$ to denote the maximum context length and reserve $S(n)$ for PRAM's memory space.

\paragraph{Input Format}
The input that encodes the PRAM's initial memory state is a sequence $\mathbf{x} = (x_1, \ldots, x_{2n+2}) \in \Sigma^{2n+2}$ of discrete tokens from the vocabulary $\Sigma$.

Let $w = \Theta(\log n)$ be the word width and recall from \Cref{appendix:pram} that the address width $a = \lceil \log_2 S(n) \rceil \leq w$ (addresses fit within words). The sequence has length $2n+2 = 1 + 2n + 1$ where:
\begin{align}
x_1 &\in \Sigma \quad \text{(processor count token)} \\
x_{2i} &\in \Sigma \quad \text{(address token)} \quad \text{for } i = 1, \ldots, n \\
x_{2i+1} &\in \Sigma \quad \text{(data token)} \quad \text{for } i = 1, \ldots, n \\
x_{2n+2} &= \texttt{[SEP]} \quad \text{(separator token)}
\end{align}

Through token embedding $\te: \Sigma \to \mathbb{R}^d$ and subsequent linear projections, these discrete tokens are mapped to their semantic bit representations:
\begin{align}
\te(x_1) &\mapsto P(n) \in \{0,1\}^w \\
\te(x_{2i}) &\mapsto \text{addr}_i \in \{0,1\}^w \quad \text{(address bits)} \\
\te(x_{2i+1}) &\mapsto \text{val}_i \in \{0,1\}^w \quad \text{(data bits)}
\end{align}
for $i = 1, \ldots, n$. The \texttt{[SEP]} token serves as a separator with special meaning in the computation trace, as detailed in the next subsection. Following the standard MDM notation from \Cref{sec:preliminary}, the actual input to the MDM is the padded sequence $\mathbf{x}_0 = (x_{0,1}, x_{0,2}, \ldots, x_{0,S_{\text{MDM}}(n)}) \in \bar{\Sigma}^{S_{\text{MDM}}(n)}$ where $\bar{\Sigma} = \Sigma \cup \{\mask\}$:
\begin{align}
x_{0,j} &= x_j \quad \text{for } j = 1, \ldots, 2n+2 \\
x_{0,j} &= \mask \quad \text{for } j = 2n+3, \ldots, S_{\text{MDM}}(n)
\end{align}

We aim to show that there exists an encoder-only Transformer that, given the initial memory state of a PRAM as input, can output the exact same result as the PRAM algorithm after $T_{\text{MDM}}(n) = \mathcal{O}(T(n))$ decoding steps. The Transformer will have constant depth and context length $\mathcal{O}(S_{\text{MDM}}(n))$, where $S_{\text{MDM}}(n)$ represents the MDM's context budget.


We next provide an overview of the construction:

\paragraph{Processor State and Computation Log Representation:} First, we state the representation of processor state and computation log as tokens. We represent each processor's state and one round of computation using a fixed number of tokens: program counter (1 token), register file (5 tokens), and computation log (2 tokens), for a total of 8 tokens per processor.

\emph{Program Counter (PC):} A single word encoding the current instruction address.

\emph{Register File:} We maintain exactly 5 registers, each storing one word. This provides sufficient computational capacity while keeping the representation tractable.

\emph{Computation Log:} This log is populated only when executing \texttt{STORE [s],r} instructions, recording the target address and stored value. For all other instructions, the log remains empty (represented by special tokens).

The computation trace for one parallel round can be represented as:
\begin{align}
\texttt{[SEP]} \; \langle \texttt{PC}_1, \texttt{R}_{1,1}, \ldots, \texttt{R}_{1,5}, \texttt{Addr}_1, \texttt{Val}_1 \rangle \; \langle \texttt{PC}_2, \texttt{R}_{2,1}, \ldots, \texttt{R}_{2,5}, \texttt{Addr}_2, \texttt{Val}_2 \rangle \; \ldots
\end{align}
where \texttt{[SEP]} serves as a separator token to distinguish different computation rounds and there are a total of $P(n)$ independent processors. Each processor $i \in \{1, \ldots, P(n)\}$ contributes an 8-tuple $\langle \texttt{PC}_i, \texttt{R}_{i,1}, \texttt{R}_{i,2}, \texttt{R}_{i,3}, \texttt{R}_{i,4}, \texttt{R}_{i,5}, \texttt{Addr}_i, \texttt{Val}_i \rangle$ representing its program counter, five register values, and memory write operation (address and value). The trace thus contains exactly $P(n)$ such 8-tuples per parallel round.

\paragraph{Processor Assignment and Role Identification.}
The algorithm begins by determining whether the current position contains a mask token (i.e. $\mathrm{is\_mask}$). If the position is a mask token (\textbf{Branch 1}), the algorithm continues by computing the distance to the nearest preceding \texttt{[SEP]} token. If the position is not a mask token (\textbf{Branch 2}), it indicates this position has already been unmasked (computation has already finished), and the algorithm returns the input token (or a all zero vector) which will not be unmasked per definition of MDM \Cref{sec:preliminary}).

To identify the processor ID, we find the rightmost \texttt{[SEP]} token to the left of the current position (i.e. $\mathrm{rightmost\_sep\_pos}$) and compute the distance between them (i.e. $\mathrm{distance\_to\_sep}$). If this distance $> 8 \times P(n)$ (\textbf{Branch 1.1}), the position does not participate in the current computation round as it is not a token that should be ``unmasked" in this round, in this case, the algorithm returns a special embedding (a all zero vector), which results in uniform distribution during prediction and smallest confidence, ensuring that the MDM will not select this position for unmasking; if the distance $= 8 \times P(n) + 1$ (\textbf{Branch 1.2}), this position return the embedding of \texttt{[SEP]}, preparing for the next computation round; otherwise if the distance $< 8 \times P(n) + 1$ (\textbf{Branch 1.3}), the position participates in the computation of the current round. 

For those positions participating in the current computation round. The corresponding processor ID (i.e. $\mathrm{processor\_id}$) is obtained by right-shifting $\mathrm{distance\_to\_sep}$ by 3 bits with zero-padding on the left (since each processor corresponds to exactly 8 tokens). The rightmost 3 bits represent the position within that processor (i.e. $\mathrm{inner\_processor\_id}$).

\paragraph{Initialization of Processor State.} 
We need to initialize the initial state of all processors at the beginning. This is determined by the current number of \texttt{[SEP]} tokens in the sequence. Specifically, when the current position is a mask token and there is exactly one \texttt{[SEP]} token (i.e. $\mathrm{seq\_sum}(\mathrm{is\_sep}) == 1$) (\textbf{Branch 3}), we consider this the initialization state. All program counters are set to 0, all registers are set to 0, and all memory locations are set to 0.


\paragraph{Fetch Instruction and Execution (Main Loop).} According to the processor ID and inner processor position, the algorithm fetches the instruction from the instruction memory (i.e. $\mathrm{get\_instruction}$), which is hard-coded into the parameters of the model, and execute it (using $\mathrm{execute}$). Different instructions yields different execution semantics, and sequently different $8$ token state. Finally, according to the $\mathrm{inner\_processor\_id}$, the algorithm chooses what to return. The algorithm terminates when the PC of all processors are HALT.


We now formally construct an $\efasp$ program that simulates PRAM execution (the single-processor algorithm detailed in \Cref{appendix:pram}). The semantics meanings and justifications of operators used in the program are summarized in \Cref{appendix:fasp}. The only two global operators are \texttt{seq\_sum} and \texttt{rightmost\_exact\_match} implementable by attention , otherwise are all local operators implementable by polylog-width constant-depth MLPs.

The context length of MDM used by this construction is $S_{\text{MDM}}(n) = \mathcal{O}(T_{\text{par}}(n) \times P(n))$, the decoding steps is $T_{\text{MDM}}(n) = \mathcal{O}(T_{\text{par}}(n))$, where $T_{\text{par}}(n)$ is the parallel time complexity and $P(n)$ is the processor count. For the constructed Transformer, embedding size is $\log(n)$ and depth is a constant.


{
\captionsetup[lstlisting]{labelformat=empty} % Remove "Listing X:"
\lstdefinestyle{customcode}{
    basicstyle=\small\ttfamily,
    mathescape=true,
    commentstyle=\color{green!50!black},
    language=python,
    keywordstyle=\color{blue},
    stringstyle=\color{red},
    frame=lines,
    framesep=5pt,
    tabsize=4,
    columns=fullflexible,
    breaklines=true,
    keepspaces=false,
    backgroundcolor=\color{white},
    showstringspaces=false,
    captionpos=b,
    emphstyle=[1]\color{orange},
    keywords={
        if, elif, else, equal, max, min, if_then_else, 
        not, and, or
    },
    emph={[1]get_token, get_move, get_symbol,seq_len},
    emphstyle=[2]\color{red},
    emph={[2]seq_sum,aha,rha,exact_match,seq_max,seq_min,rightmost_exact_match,seq_and,seq_or},
    literate=%
    {PE}{{{\color{orange}PE}}}{1}
    {TE}{{{\color{orange}TE}}}{1}
}

\begin{lstlisting}[style=customcode]
# ---------------------- Initialization ----------------------
is_sep   = (TE == embed([SEP]))
is_mask  = (TE == embed([MASK]))
is_init = (seq_sum(is_sep) == 1)

# Get the current and last [SEP] position
cur_sep  = rightmost_exact_match(1, is_sep, PE)      
dist_to_sep     = PE - cur_sep                               
pn = rightmost_exact_match(1, is_first, TE)
spanned_pn    = pn << 3                                   

# Skip positions not participating in computation
if (not is_mask) or (dist_to_sep > spanned_pn): return 0
if dist_to_sep == spanned_pn + 1:  return embed([SEP])
if is_init: return 0

# Initialization
pid        = (dist_to_sep - 1) >> 3     
inner_id   = (dist_to_sep - 1 )[:3]     

if is_init and inner_id == 1: return pid
if is_init and inner_id != 1: return 0

# ---------------------- Read previous round state ----------------------
prev_sep   = rightmost_exact_match(1, (is_sep and (PE < cur_sep)), PE)
prev_pid_base  = prev_sep + 1 + (pid << 3)

pos_PC     = prev_pid_base + 0
PC    = rightmost_exact_match(pos_PC, PE, TE)
pos_R1     = prev_pid_base + 1
R1    = rightmost_exact_match(pos_R1, PE, TE)
pos_R2     = prev_pid_base + 2
R2    = rightmost_exact_match(pos_R2, PE, TE)
pos_R3     = prev_pid_base + 3
R3    = rightmost_exact_match(pos_R3, PE, TE)
pos_R4     = prev_pid_base + 4
R4    = rightmost_exact_match(pos_R4, PE, TE)
pos_R5     = prev_pid_base + 5
R5    = rightmost_exact_match(pos_R5, PE, TE)

if PC == HALT_CODE: return 0

# -------------------- Fetch and execute instruction --------------------
# Decode
I_type, op_r, op_s, op_c, label_addr = get_instruction(PC)

# Source/destination register
Rs = (R1 if op_s == 1 else
        R2 if op_s == 2 else
        R3 if op_s == 3 else
        R4 if op_s == 4 else
        R5 if op_s == 5 else 0)

Rr = (R1 if op_r == 1 else
        R2 if op_r == 2 else
        R3 if op_r == 3 else
        R4 if op_r == 4 else
        R5 if op_r == 5 else 0)


# Default effect
PC_next    = PC + 1
WR_val     = Rr
writes_reg = False
ADDR_out   = 0      # Slot 6: only STORE overwrites address (a <= w, use word directly)
VAL_out    = 0      # Slot 7: only STORE overwrites value

#  Address read
if   I_type == embed([LOAD]):    
    ADDR_KEYS    = (TE if is_addr else 0)     
    ADDR_POSVAL  = (PE if is_addr else 0)    

    last_addr_pos_load = rightmost_exact_match(Rs, ADDR_KEYS, ADDR_POSVAL)
    load_val           = rightmost_exact_match(last_addr_pos_load + 1, PE, TE)
    WR_val = load_val

# Per-instruction semantics
elif I_type == embed([STORE]):   ADDR_out, VAL_out = Rs, Rr
elif I_type == embed([LOADI]):   WR_val = op_c
elif I_type == embed([ADD]):     WR_val = (Rr + Rs)
elif I_type == embed([SUB]):     WR_val = (Rr - Rs)
elif I_type == embed([AND]):     WR_val = (Rr & Rs)
elif I_type == embed([XOR]):     WR_val = (Rr ^ Rs)
elif I_type == embed([SHL]):     WR_val = (Rr << Rs)
elif I_type == embed([SHR]):     WR_val = (Rr >> Rs)
elif I_type == embed([BRZ]) and Rr == 0: PC_next = label_addr
elif I_type == embed([JMP]):     PC_next = label_addr
elif I_type == embed([HALT]):    PC_next = HALT_CODE

# Register writeback: only for {LOAD, LOADI, ADD, SUB, AND, XOR, SHL, SHR}
writes_reg = (I_type == embed([LOAD]))  or (I_type == embed([LOADI])) or \
                (I_type == embed([ADD]))   or (I_type == embed([SUB]))   or \
                (I_type == embed([AND]))   or (I_type == embed([XOR]))   or \
                (I_type == embed([SHL]))   or (I_type == embed([SHR]))

R1_next = (WR_val if (writes_reg and op_r == 1) else R1)
R2_next = (WR_val if (writes_reg and op_r == 2) else R2)
R3_next = (WR_val if (writes_reg and op_r == 3) else R3)
R4_next = (WR_val if (writes_reg and op_r == 4) else R4)
R5_next = (WR_val if (writes_reg and op_r == 5) else R5)

# ---------- Return one of 8 slots according to inner_id ----------
if   inner_id == 0:  return PC_next
elif inner_id == 1:  return R1_next
elif inner_id == 2:  return R2_next
elif inner_id == 3:  return R3_next
elif inner_id == 4:  return R4_next
elif inner_id == 5:  return R5_next
elif inner_id == 6:  return ADDR_out
else:                return VAL_out
\end{lstlisting}
}




\section{Proof of \Cref{thm:main_constrained}} \label{appendix:proof_main_constrained}


The result stems from MDM's total amount of computation being bounded by $S(n)$ in both total steps ($T(n) \leq S(n)$) and per-step capacity (polynomial embedding size), preventing it from solving problems requiring greater computational resources.

Fix an encoder-only MDM with context length $S(n)$ and $T(n)$ decoding steps. Throughout we assume constant depth/heads and log-precision arithmetic with hidden width $d=\Theta(\log(S(n)+T(n)))$ (binary positional code), as in our setup. Particularly, at each decoding step the model re-encodes a length-$S(n)$ sequence. A single forward pass is dominated by self-attention: for each position $i$ we form a query in $\mathbb{R}^d$ and take dot products with all $S(n)$ keys, then take the value-weighted sum. Counting FLOPs, one attention head costs
    \begin{align}
    \Theta\big(S(n)\cdot S(n)\cdot d\big)=\Theta\!\big(S(n)^2\log(S(n)+T(n))\big),
    \end{align}
    and the multi-head/multi-layer constants only change the leading constant. The position-wise MLP adds $\Theta(S(n)\cdot \operatorname{poly}(d))=\Theta\!\big(S(n)\operatorname{polylog}(S(n)+T(n))\big)$ FLOPs and is lower order when $S(n)\gg d$. Thus one decoding step costs
    \begin{align}
    \widetilde{\mathcal{O}}\big(S(n)^2\big)\quad\text{FLOPs,}
    \end{align}
    where $\widetilde{\mathcal{O}}(\cdot)$ suppresses polylog factors in $S(n)+T(n)$. Over $T(n)$ steps the total compute is $\widetilde{\mathcal{O}}\!\big(S(n)^2\,T(n)\big)$. In particular, when each step reveals at least one token (or a constant number), we have $T(n)\le S(n)$, yielding the unified cubic bound $\widetilde{\mathcal{O}}(S(n)^3)$. Hence any problem that needs $\omega\!\big(S(n)^3\big)$ serial time cannot be solved by MDM in the $(S(n),T(n))$ regime stated.


\section{Proof of \Cref{thm:any_order}} \label{appendix:proof_any_order}



Recall \Cref{def:masked_arm}, we defined \textbf{Masked-ARM} as an autoregressive model with encoder-only Transformer architecture that pads the input sequence with mask tokens to the maximum context length, which is also equivalent to a MDM with a fixed order (left-to-right) generation and generating one token at a time. Consider an AO-MDM with input format $\mathbf{x} = (x_1, \ldots, x_n)$ followed by a special separator token \texttt{[SEP]} at position $n+1$. 

\textbf{AO-MDM Intermediate State:} At any intermediate generation step, the AO-MDM state can be represented as $\mathbf{z} = (z_1, \ldots, z_{S(n)}) \in \bar{\Sigma}^{S(n)}$ where $\bar{\Sigma} = \Sigma \cup \{\mask\}$. The sequence structure is:
\begin{align}
z_j &= x_j \quad \text{for } j \in [n] \quad \text{(fixed input portion)} \\
z_{n+1} &= \texttt{[SEP]} \quad \text{(separator)} \\
z_j &\in \Sigma \cup \{\mask\} \quad \text{for } j \in \{n+2, \ldots, S(n)\} \quad \text{(generation region)}
\end{align}

Let $\mathcal{D} = (d_1, d_2, \ldots, d_k)$ denote the sequence of positions that have been decoded (unmasked) by the AO-MDM in chronological order, where $d_i \in \{n+2, \ldots, S(n)\}$ and $z_{d_i} \neq \mask$ for all $i \in [k]$. The ordering reflects the temporal sequence in which the AO-MDM performed the unmasking operations.

\begin{definition}[Position/Content Tokens and Address Encoding] \label{def:addr_tok}
Let $\Sigma$ be the base vocabulary. We reserve a subset $\Sigma_{\mathrm{pos}} \subseteq \Sigma$ for position tokens and define a bijection $\mathrm{encode}: \{1,\ldots,S(n)\} \to \Sigma_{\mathrm{pos}}$ with inverse $\mathrm{dec\_pos}: \Sigma_{\mathrm{pos}} \to \{1,\ldots,S(n)\}$. For each decoded position $d_i$, define
\[
\texttt{addr}_{d_i} := \mathrm{encode}(d_i) \in \Sigma_{\mathrm{pos}} \subseteq \Sigma, \qquad \texttt{tok}_{d_i} := z_{d_i} \in \Sigma.
\]
Thus both address tokens and content tokens are drawn from the original vocabulary $\Sigma$. We also reserve a subset $\Sigma_{\mathrm{op}} \subseteq \Sigma$ for operator tokens used later for AP-MDM edits (\Cref{sec:any_process}).
\end{definition}

\textbf{Masked-ARM Simulation:} For each decoded token at position $d_i \in \mathcal{D}$, the Masked-ARM represents it using a 2-tuple:
\begin{align}
\langle \texttt{addr}_{d_i}, \texttt{tok}_{d_i} \rangle = \langle \text{encode}(d_i), z_{d_i} \rangle
\end{align}
where $\text{encode}(d_i)$ is a token representation of the positional index $d_i$, and $z_{d_i}$ is the actual decoded token. The target Masked-ARM sequence to be constructed is:
\begin{align}
\mathbf{y}_{\text{ARM}} = (x_1, \ldots, x_n, \texttt{[SEP]}, \texttt{addr}_{d_1}, \texttt{tok}_{d_1}, \ldots, \texttt{addr}_{d_k}, \texttt{tok}_{d_k}, \underbrace{\mask, \ldots, \mask}_{\text{remaining positions}})
\end{align}
where the sequence $\mathcal{D} = (d_1, d_2, \ldots, d_k)$ preserves the chronological order of AO-MDM's decoding operations. The Masked-ARM sequence has total length $2S(n) - n - 1 = \mathcal{O}(S(n))$, since each AO-MDM token requires two tokens (address and content) in the Masked-ARM representation.

\textbf{Induction:} To prove We prove by induction that for any AO-MDM, there exists a corresponding Masked-ARM that can simulate the AO-MDM's generation process step by step for arbitrary input sequences.

\begin{theorem}[AO-MDM Simulation by Masked-ARM]\label{lemma:aomdm_simulation}
For any AO-MDM, there exists a corresponding Masked-ARM such that: for any input sequence $\mathbf{x} = (x_1, \ldots, x_n)$ and any intermediate state of the AO-MDM with decoded sequence $\mathcal{D} = (d_1, d_2, \ldots, d_k)$, the Masked-ARM, starting from the corresponding intermediate state $\mathbf{y}_{\text{ARM}}$, can generate the next address-token pair $\langle \texttt{addr}_{d_{k+1}}, \texttt{tok}_{d_{k+1}} \rangle$ such that:
\begin{align}
\text{encode}(d_{k+1}) &= \texttt{addr}_{d_{k+1}} \quad \text{(address matches AO-MDM's next decode position)} \\
z_{d_{k+1}} &= \texttt{tok}_{d_{k+1}} \quad \text{(token matches AO-MDM's next decode content)}
\end{align}
where $d_{k+1}$ is the position that AO-MDM will decode next, and $z_{d_{k+1}}$ is the token that AO-MDM will generate at that position.
\end{theorem}

\begin{proof}
To prove this result, we decompose the architecture of the AO-MDM into two parts: the input transformation part (which can be represented as an operator $\texttt{mdm\_embed}$) that transforms the token and position into an embedding, and the output generation part (which can be represented as an operator $\texttt{mdm\_decode}$) that transforms the embedding into logits, that is:
\begin{equation}
\text{AO-MDM}(\mathbf{x}) = \texttt{mdm\_decode}(\texttt{mdm\_embed}(\bar{\te}, \bar{\pe}))(\mathbf{x})
\end{equation}
where $\bar{\te}$ and $\bar{\pe}$ are seq-to-seq functions defined in \Cref{def:seq-ext}.

This decomposition is invariant to the choice of token and position embedding functions and AO-MDM's parameter configuration. Simulating AO-MDM's generation process boils down to the following two steps:

\textbf{Step 1: Replicating \texttt{mdm\_embed}.} We construct initial layers of the Masked-ARM that, given the Masked-ARM state $\mathbf{y}_{\text{ARM}}$, produce intermediate embeddings identical to $\texttt{mdm\_embed}(\mathbf{x})$ where $\mathbf{x}$ is the corresponding AO-MDM state. This transformation converts the address-token pair representation back into the embedding format that the AO-MDM expects, enabling the subsequent layers to perform identical computations. 
We write the $\efasp$ programs (which corresponds to the encoder Transformer construction) for the construction:


{
\captionsetup[lstlisting]{labelformat=empty} % Remove "Listing X:"
\lstdefinestyle{customcode}{
    basicstyle=\small\ttfamily,
    mathescape=true,
    commentstyle=\color{green!50!black},
    language=python,
    keywordstyle=\color{blue},
    stringstyle=\color{red},
    frame=lines,
    framesep=5pt,
    tabsize=4,
    columns=fullflexible,
    breaklines=true,
    keepspaces=false,
    backgroundcolor=\color{white},
    showstringspaces=false,
    captionpos=b,
    emphstyle=[1]\color{orange},
    keywords={
        if, elif, else, equal, max, min, if_then_else, 
        not, and, or
    },
    emph={[1]get_token, get_move, get_symbol,seq_len},
    emphstyle=[2]\color{red},
    emph={[2]seq_sum,aha,rha,exact_match,seq_max,seq_min,rightmost_exact_match,seq_and,seq_or},
    literate=%
    {PE}{{{\color{orange}PE}}}{1}
    {TE}{{{\color{orange}TE}}}{1}
}

\begin{lstlisting}[style=customcode]
# ---------------------- Get logits identical to AO-MDM ----------------------
mdm_logits = mdm_decode(embed_MDM)
tok_scores = score(mdm_logits)

# AO-MDM candidate set: positions > [SEP], still [MASK], and within valid range
cand_mask = (PE > sep_pos) and (TE_MDM == embed([MASK])) and (PE <= sn)
cand_score = (tok_scores if cand_mask else 0)

max_score  = seq_max(cand_score)
is_best    = cand_mask and (tok_scores == max_score)

# Select AO-MDM's next decode position and corresponding logits
next_pos   = rightmost_exact_match(1, is_best, PE)
logits_next = rightmost_exact_match(next_pos, PE, mdm_logits)

# ---------------------- Emit as Masked-ARM <addr, tok> order ----------------------
gen_slot = rightmost_exact_match(1,
                                    (PE > sep_pos) and (PE <= sn) and (TE == embed([MASK])),
                                    PE)
emit_addr = (((gen_slot - sep_pos)[:1]) == 0)

if PE == gen_slot:
    result = (next_pos if emit_addr else logits_next)
else:
    result = 0

return result
\end{lstlisting}
}

This completes the proof.
\end{proof}

We remark the proof relies on two assumptions: 1) the function $\bar{S}(\mathbf x) = S(|\mathbf x|)$ is deterministic and computable by encoder Transformer (this is implemented by the $\texttt{sn}$ function in the $\efasp$ program for Step 1); 2) the confidence score is also dertermistic and computable by encoder Transformer (this is implemented by the $\texttt{score}$ operator in the $\efasp$ program for Step 2).


\section{Proof of \Cref{thm:main_apmdm}} \label{sec:main_apmdm_proof}

\begin{theorem}[AP-MDM Simulation of PRAM, Formal] \label{thm:main_apmdm_formal}
Let $\mathcal P = (I_0,\dots,I_\ell,\mathrm{addr})$ be a uniform PRAM program with a finite instruction set of size $\ell$, identical across processors and input size $n$. On an initial memory state specified by address–value pairs $(\mathbf x_{\mathrm{addr}}, \mathbf x_{\mathrm{val}})$ with $\mathbf x_{\mathrm{val}} \in \mathbb U^n$ and $\mathbf x_{\mathrm{addr}} \in \mathcal A^n$, suppose $\mathcal P$ runs in parallel time $T(n)$ using at most $P(n)$ processors and at most $S(n)$ shared-memory words of $\Theta(\log n)$ bits, and outputs $\mathrm{PRAM}_{\mathcal P}(\mathbf x_{\mathrm{addr}}, \mathbf x_{\mathrm{val}}) \in \mathbb U$ (see \Cref{appendix:pram}). Then there exists a bijection $\phi: \mathbb U \cup \mathcal A \rightarrow \Sigma$ and an AP-MDM which, on input
\[
\mathbf{x} = (z_0, z_1, \ldots, z_n) \in \Sigma^{n+1}, \quad z_0 = \phi(P(n)),\; z_i = \phi(\mathbf x_{\mathrm{val},i}) \text{ for } i=1,\ldots,n,
\]
padded to context length $\mathcal O(S(n))$ (addresses provided implicitly by positional encodings), produces $\phi\!\left(\mathrm{PRAM}_{\mathcal P}(\mathbf x_{\mathrm{addr}}, \mathbf x_{\mathrm{val}})\right)$ in $\mathcal O(T(n))$ decoding steps.
\end{theorem}

We first show that AP-MDM can simulate a weaker model called {Rewrite-MDM}, which is sufficient for the result. Then we construct an $\efasp$ program that simulates PRAM execution in a space-efficient manner.


\textbf{Rewrite-MDM} follows the same framework as AP-MDM but with simplified control signals. For any token $y \in \Sigma$, define:
\begin{align}
\remask_{x_{t,i}, \ctrl_{t,i}}(y) = \begin{cases}
y & \text{if } \ctrl_{t,i}[1] = 1 \\
x_{t,i} & \text{if } \ctrl_{t,i}[1] = 0
\end{cases}
\end{align}
where $\ctrl_{t,i}[1] \in \{0,1\}$ is a binary rewrite signal. In other words, when $\ctrl_{t,i}[1] = 1$, the model rewrites position $i$ with new content $y$; when $\ctrl_{t,i}[1] = 0$, it preserves the original content $x_{t,i}$ unchanged.

We next show how each transition $\mathbf{z}_t \rightarrow \mathbf{z}_{t+1}$ in Rewrite-MDM can be simulated by exactly three steps of AP-MDM as defined in \Cref{sec:any_process}.

\begin{lemma}[AP-MDM Simulation of Rewrite-MDM] \label{lemma:apmdm_simulates_rewrite}
For any Rewrite-MDM transition $\mathbf{z}_t \rightarrow \mathbf{z}_{t+1}$ on sequence of length $n$, there exists a sequence of three AP-MDM steps that produces the identical result.
\end{lemma}

\begin{proof}
Given a Rewrite-MDM transition where we want to selectively rewrite positions in sequence $\mathbf{z}_t = (z_{t,1}, z_{t,2}, \ldots, z_{t,n})$ according to rewrite signal $\mathbf{r}_t = (r_{t,1}, r_{t,2}, \ldots, r_{t,n})$, we simulate this using the following three AP-MDM steps: $\mathbf{z}_t \rightarrow \mathbf{u}^{(1)} \rightarrow \mathbf{u}^{(2)} \rightarrow \mathbf{u}^{(3)} = \mathbf{z}_{t+1}$.

\textbf{Step 1 (Insert):} Starting from $\mathbf{z}_t$, apply $\inser$ operation at every position $i \in [n]$ to create an expanded sequence of length $2n$:
\begin{align}
\mathbf{u}^{(1)} = (g \circ f_\theta)(\mathbf{z}_t)
\end{align}
where $\ctrl^{(1)}_{i}[1] = 0$ (no remask), $\ctrl^{(1)}_{i}[2] = 1$ (insert), $\ctrl^{(1)}_{i}[3] = 0$ (no delete) for all $i \in [n]$.
This yields $\mathbf{u}^{(1)} = (z_{t,1}, \mask, z_{t,2}, \mask, \ldots, z_{t,n}, \mask)$.

\textbf{Step 2 (Unmask and Remask):} Apply AP-MDM's $(g \circ f_\theta)$ operation on $\mathbf{u}^{(1)}$ with control signals:
\begin{align}
\mathbf{u}^{(2)} = (g \circ f_\theta)(\mathbf{u}^{(1)})
\end{align}
where the control signals are set as follows:
\begin{itemize}
\item For even positions $2i$ (newly inserted masks): $\ctrl^{(2)}_{2i}[1] = 0$ (unmask), $\ctrl^{(2)}_{2i}[2] = 0$ (no insert), $\ctrl^{(2)}_{2i}[3] = 0$ (no delete)
\item For odd positions $2i-1$ (original tokens): $\ctrl^{(2)}_{2i-1}[1] = r_{t,i}$ (remask), $\ctrl^{(2)}_{2i-1}[2] = 0$ (no insert), $\ctrl^{(2)}_{2i-1}[3] = 0$ (no delete)
\end{itemize}
which yield $\mathbf{u}^{(2)} = (\mask, z_{t+1,1}, \mask, z_{t+1,2}, \ldots, \mask, z_{t+1,n})$.

\textbf{Step 3 (Delete):} Apply AP-MDM's $(g \circ f_\theta)$ operation again to $\delete$ all mask tokens at original positions:
\begin{align}
\mathbf{u}^{(3)} = (g \circ f_\theta)(\mathbf{u}^{(2)})
\end{align}
where for all positions $j$ in $\mathbf{u}^{(2)}$:
\begin{itemize}
\item For odd positions $2i-1$: $\ctrl^{(3)}_{2i-1}[1] = 0$ (no remask), $\ctrl^{(3)}_{2i-1}[2] = 0$ (no insert), $\ctrl^{(3)}_{2i-1}[3] = \mathds{1}[u^{(2)}_{2i-1} = \mask]$ (delete if mask)
\item For even positions $2i$: $\ctrl^{(3)}_{2i}[1] = 0$ (no remask), $\ctrl^{(3)}_{2i}[2] = 0$ (no insert), $\ctrl^{(3)}_{2i}[3] = 0$ (no delete)
\end{itemize}
This removes all mask tokens at odd positions and recovers the original length $n$. By construction, $\mathbf{u}^{(3)} = \mathbf{z}_{t+1}$, completing the simulation.


\textbf{State Tracking Mechanism}~~~ To enable the AP-MDM to autonomously determine which of the three simulation steps to execute, we augment sequences with special boundary tokens $\texttt{[BOS]}$ and $\texttt{[EOS]}$. The model identifies the current phase by examining the boundary token configuration:
\begin{itemize}
\item \textbf{Step 1 (Insert):} Normal state with $\texttt{[BOS]}$ at the beginning and $\texttt{[EOS]}$ at the end
\item \textbf{Step 2 (Unmask and Remask):} $\texttt{[EOS]}$ is followed by a $\mask$ token, indicating expanded state
\item \textbf{Step 3 (Delete):} $\texttt{[BOS]}$ is preceded by a $\mask$ token, signaling cleanup phase
\end{itemize}
During Step 2, the model leverages the first bit of positional encodings (e.g. $\bipe$ introduced in \Cref{appendix:fasp}) to distinguish between original positions (odd indices) and newly inserted positions (even indices), enabling it to correctly apply remasking operations to original positions based on the rewrite signal $\mathbf{r}_t$ while unmasking new positions to write content from $\mathbf{w}_t$.

We omit the Transformer-based construction for the procedure described above for brevity, which can be done by a simple $\efasp$ program.
\end{proof}


We use the Rewrite-MDM variant established above to simulate PRAM algorithms with optimal space complexity. Here we use the $\efasp[\bipe; [\cdot]_+, \times]$ variant with binary positional encoding (\Cref{appendix:fasp}). The input that encodes the PRAM's initial memory state is a sequence $\mathbf{x} = (x_1, \ldots, x_{n+1}) \in \Sigma^{n+1}$ of discrete tokens from the vocabulary $\Sigma$:
\begin{align}
x_1 &\in \Sigma \quad \text{(processor count token)} \\
x_{i+1} &\in \Sigma \quad \text{(data token)} \quad \text{for } i = 1, \ldots, n
\end{align}

Through token embedding $\te: \Sigma \to \mathbb{R}^w$, these discrete tokens are mapped to their semantic bit representations:
\begin{align}
\te(x_1) &\mapsto P(n) \in \{0,1\}^w \\
\te(x_{i+1}) &\mapsto \text{val}_i \in \{0,1\}^w \quad \text{(data bits)}
\end{align}
for $i = 1, \ldots, n$. The actual input to the AP-MDM is the padded sequence $\mathbf{x}_0 = (x_{0,1}, x_{0,2}, \ldots, x_{0,S(n)}) \in \bar{\Sigma}^{S(n)}$ where $\bar{\Sigma} = \Sigma \cup \{\mask\}$:
\begin{align}
x_{0,j} &= x_j \quad \text{for } j = 1, \ldots, n+1 \\
x_{0,j} &= \mask \quad \text{for } j = n+2, \ldots, S(n)
\end{align}

The crucial advantage of AP-MDM is that it can dynamically rewrite the content at any position using the $\remask$ operation, allowing the simulation to use space optimally as $\mathcal{O}(S(n))$ rather than the $\mathcal{O}(P(n) \times T(n))$ required by standard MDM.


We next provide an overview of the construction:

The key difference between how Rewrite-MDM and AO-MDM simulate PRAM is that Rewrite-MDM can directly rewrite the memory at any position and each computation does not necessarily has be kept in the context forever. This enables us to get rid of the address token. Now representation of a processor can be simplified to:
\begin{align}
\ldots \langle \texttt{PC}_1, \texttt{R}_{1,1}, \texttt{R}_{1,2}, \texttt{R}_{1,3} \rangle \; \langle \texttt{PC}_2, \texttt{R}_{2,1}, \texttt{R}_{2,2}, \texttt{R}_{2,3}  \rangle \; \ldots 
\end{align}
where we only use 3 registers (this is sufficient for the proof but can be extended to any $k \geq 2$).

Additionally, we do not append processor representations to the end of input $\mathbf{x}$ as in AO-MDM, but instead will initialize them at the end of the entire sequence. The remaining part of the sequence is used as a shared memory where token embeddings are data and positional encodings are addresses, aligning more closely with PRAM.

\paragraph{Intialization.} When the last position is a mask token, we initialize the processor state and computation log at the end of the sequence. Roles of each token are calculated similarly as the construction in AO-MDM (except it is static throughout the generation process).

The execution of the program is similar to the construction in AO-MDM, except now the address is inherently associated with the positional encoding. The termination is also slightly different: the returned embedding has to contain an additional bit to indicate the rewrite operation. Also, the termination condition is no longer when all masked are unmasked but a flexibly defined one: in our case, when all processors are HALT.



Using the operators defined above, we now construct an $\efasp$ program that simulates PRAM execution. The program implements the single-processor algorithm detailed in \Cref{appendix:pram}.



{
\captionsetup[lstlisting]{labelformat=empty} % Remove "Listing X:"
\lstdefinestyle{customcode}{
    basicstyle=\small\ttfamily,
    mathescape=true,
    commentstyle=\color{green!50!black},
    language=python,
    keywordstyle=\color{blue},
    stringstyle=\color{red},
    frame=lines,
    framesep=5pt,
    tabsize=4,
    columns=fullflexible,
    breaklines=true,
    keepspaces=false,
    backgroundcolor=\color{white},
    showstringspaces=false,
    captionpos=b,
    emphstyle=[1]\color{orange},
    keywords={
        if, elif, else, equal, max, min, if_then_else, 
        not, and, or
    },
    emph={[1]get_token, get_move, get_symbol,seq_len},
    emphstyle=[2]\color{red},
    emph={[2]seq_sum,aha,rha,exact_match,seq_max,seq_min,rightmost_exact_match,seq_and,seq_or},
    literate=%
    {PE}{{{\color{orange}PE}}}{1}
    {TE}{{{\color{orange}TE}}}{1}
}
\begin{lstlisting}[style=customcode]
# ---------------------- Roles & Layout  ----------------------
is_mask = (TE == embed([MASK]))
pn      = rightmost_exact_match(1, is_first, TE)             # number of processors P(n)
last_p  = rightmost_exact_match(1, is_last,  PE)             # last position index
proc_b  = last_p - (pn << 2) + 1                             # processor region start
in_proc = (PE >= proc_b) and (PE < proc_b + (pn << 2))       # in processor region
in_mem  = (PE >= 2) and (PE < proc_b)                        # in memory region

# ---------------------- Initialization  ----------------------
last_is_mask = rightmost_exact_match(1, is_last, is_mask)
if last_is_mask and in_proc:
    inner = (PE - proc_b)[:2]                                # slot index 0..3
    # initialize processor region to 0, and R=1 (require rewrite)
    if   inner == 0: return (0, 1)                           # PC
    elif inner == 1: return ((PE - proc_b) >> 2, 1)          # R1
    elif inner == 2: return (0, 1)                           # R2
    else:            return (0, 1)                           # R3
# no rewrite for other positions in initialization step
if last_is_mask and not in_proc:
    return (TE, 0)

# ========= Processor zone update (only if in_proc) =========
if in_proc:
    pid  = (PE - proc_b) >> 2
    slot = (PE - proc_b)[:2]                                 # 0:PC 1:R1 2:R2 3:R3

    # read previous round processor state (fixed slot)
    pc_pos = proc_b + (pid << 2) + 0
    r1_pos = proc_b + (pid << 2) + 1
    r2_pos = proc_b + (pid << 2) + 2
    r3_pos = proc_b + (pid << 2) + 3
    PC = rightmost_exact_match(pc_pos, PE, TE)
    R1 = rightmost_exact_match(r1_pos, PE, TE)
    R2 = rightmost_exact_match(r2_pos, PE, TE)
    R3 = rightmost_exact_match(r3_pos, PE, TE)

    # processor already HALTed, do not update (R=0 for this slot)
    if PC == HALT_CODE: return (0, 0)

    # fetch and decode instruction
    I_type, op_r, op_s, op_c, label_addr = get_instruction(PC)
    Rs = (R1 if op_s == 1 else R2 if op_s == 2 else R3 if op_s == 3 else 0)
    Rr = (R1 if op_r == 1 else R2 if op_r == 2 else R3 if op_r == 3 else 0)

    # default
    PCn, WR, WR_en = PC + 1, Rr, 0

    # instruction semantics
    if I_type == embed([LOAD]):  
        # mem_get: address=PE, value=TE (only match in memory region)
        hitp = rightmost_exact_match(Rs, (PE if in_mem else 0), (PE if in_mem else 0))
        WR = rightmost_exact_match(hitp, PE, (TE if in_mem else 0))
        WR_en = 1

    elif I_type == embed([STORE]): WR_en = 0
    elif I_type == embed([LOADI]): WR, WR_en = op_c, 1
    elif I_type == embed([ADD]):   WR, WR_en = (Rr + Rs), 1
    elif I_type == embed([SUB]):   WR, WR_en = (Rr - Rs), 1
    elif I_type == embed([AND]):   WR, WR_en = (Rr & Rs), 1
    elif I_type == embed([XOR]):   WR, WR_en = (Rr ^ Rs), 1
    elif I_type == embed([SHL]):   WR, WR_en = (Rr << Rs)), 1
    elif I_type == embed([SHR]):   WR, WR_en = (Rr >> Rs)), 1
    elif I_type == embed([BRZ]) and (Rr == 0): PCn = label_addr
    elif I_type == embed([JMP]):   PCn = label_addr
    elif I_type == embed([HALT]):  PCn = HALT_CODE

    # unified writeback
    R1n = (WR if (WR_en and op_r == 1) else R1)
    R2n = (WR if (WR_en and op_r == 2) else R2)
    R3n = (WR if (WR_en and op_r == 3) else R3)

    # return next state for this slot and require rewrite
    if   slot == 0: return (PCn, 1)
    elif slot == 1: return (R1n, 1)
    elif slot == 2: return (R2n, 1)
    else:           return (R3n, 1)

# ========= Memory zone update (also for non MASK) =========
if in_mem:
    # for all PC slots, construct STORE stream (address, value) for this step
    is_pc_glob = (((PE - proc_b)[:2]) == 0) and (PE >= proc_b) and (PE < proc_b + (pn << 2))
    PCi  = (TE if is_pc_glob else 0)
    It, rd, rs, cimm, L = get_instruction(PCi)

    pid_i = ((PE - proc_b) >> 2)                        # only meaningful for PC slot
    r1_i  = rightmost_exact_match(proc_b + (pid_i << 2) + 1, PE, TE)
    r2_i  = rightmost_exact_match(proc_b + (pid_i << 2) + 2, PE, TE)
    r3_i  = rightmost_exact_match(proc_b + (pid_i << 2) + 3, PE, TE)
    Rs_i  = (r1_i if rs == 1 else r2_i if rs == 2 else r3_i if rs == 3 else 0)
    Rr_i  = (r1_i if rd == 1 else r2_i if rd == 2 else r3_i if rd == 3 else 0)

    STORE_KEYS = (Rs_i if (is_pc_glob and (It == embed([STORE]))) else 0)  # store address
    STORE_VALS = (Rr_i if (is_pc_glob and (It == embed([STORE]))) else 0)  # store value

    hit = rightmost_exact_match(PE, STORE_KEYS, 1, 0)
    val = rightmost_exact_match(PE, STORE_KEYS, STORE_VALS, TE)

    # all halt: do not rewrite; otherwise, if hit, rewrite this address (even if not MASK originally)
    if hit == 1: return (val, 1)
    else:        return (TE, 0)

return (TE, 0)    
\end{lstlisting}
}


\section{Proof of \Cref{thm:apmdm_simulation}} \label{appendix:proof_apmdm_simulation}

\begin{definition}[Two-Sided Dyck-$k$]
    Let $\Sigma_k=\{a_1^{\pm1},\dots,a_k^{\pm1}\}$. Define $u\Rightarrow v$ if $v$ is obtained from $u$ by deleting a factor $a_i a_i^{-1}$ or $a_i^{-1}a_i$ for some $i\in\{1,\dots,k\}$. Write $u\Rightarrow^* v$ iff there exist $m\ge 0$ and words $u=w_0,\dots,w_m=v$ with $w_j\Rightarrow w_{j+1}$ for all $j$. Then
    \begin{equation}
    \TDyck_k \;:=\; \{\, w\in\Sigma_k^* \;:\; w \Rightarrow^* \varepsilon \,\}.
    \end{equation}
    where $\varepsilon$ is the empty word.
\end{definition}

For the Two-Sided Dyck-$k$ language, we define the vocabulary as:
\begin{equation}
\Sigma = \{a_1^{+1}, a_1^{-1}, a_2^{+1}, a_2^{-1}, \ldots, a_k^{+1}, a_k^{-1}\} \cup \{\texttt{[BOS]}, \texttt{[EOS]}\}
\end{equation}
and the extended vocabulary $\bar{\Sigma} = \Sigma \cup \{\mask_1, \mask_2\}$, where $\{a_i^{\pm1}\}_{i=1}^k$ are the $2k$ bracket tokens, $\texttt{[BOS]}$ and $\texttt{[EOS]}$ are boundary tokens, and $\mask_1, \mask_2$ are two types of mask tokens used to handle an inherent limitation of vanilla masked diffusion when extended to the non-deterministic case (i.e. given two mask tokens, the model can not randomly generate $AA$ and $BB$ without also having probability to generate $AB$ and $BA$). Thus $|\Sigma| = 2k + 2$ and $|\bar{\Sigma}| = 2k + 4$.

Following \citep{merrill2022saturated}:

\begin{definition}[Hardmax]\label{def:hardmax}
For any $x \in \mathbb{R}^n$, define the zero-temperature softmax (Hardmax) as
\[
\softmax_{0}(x) \;\triangleq\; \lim_{\tau \to 0^+} \softmax_{\tau}(x),
\quad\text{where}\quad
[\softmax_{0}(x)]_i =
\begin{cases}
\frac{1}{|\arg\max_j x_j|}, & i \in \arg\max_j x_j,\\
0, & \text{otherwise.}
\end{cases}
\]
\end{definition}
    
\begin{definition}[Stochastic AP-MDM] \label{def:stochastic_apmdm}
A {stochastic AP-MDM} is defined as an AP-MDM with encoder-only Transformer backbone as in \Cref{appendix:encoder}, where instead of greedy decoding, we use sampling from Hardmax distributions. Formally, let $\Enc_\theta: \Sigma^* \to (\mathbb{R}^d)^*$ be the encoder Transformer (before the final projection layer) as defined in \Cref{def:encoder}. For each position $i$ in the input sequence, let $\mathbf{h}_i = [\Enc_\theta(\mathbf{x})]_i \in \mathbb{R}^d$ be the hidden state. Define logits $\ell(v\mid \mathbf{x},i)=\langle \mathbf{h}_i,\te(v)\rangle$ and the probability distribution over vocabulary $\Sigma$ as:
\begin{equation}
p_\theta(v \mid \mathbf{x}, i) = [\softmax_{0}(\ell(\cdot\mid \mathbf{x},i))]_v
\end{equation}
where $\softmax_0$ is the Hardmax function from \Cref{def:hardmax}. The stochastic AP-MDM samples tokens according to $v_i \sim \text{Categorical}(p_\theta(\cdot \mid \mathbf{x}, i))$. For the insert operation to support two types of mask tokens ($\mask_1$ and $\mask_2$), we use two separate classification heads $\proj_{I_1}$ and $\proj_{I_2}$ whose outputs are thresholded for inserting $\mask_1$ or $\mask_2$, with priority: $\mask_2$ takes precedence over $\mask_1$. We disable remask and delete operations for this construction.
\end{definition}


\begin{theorem}[Generating Two-Sided Dyck-$k$, Formal] \label{thm:apmdm_simulation_formal}
For any $k\ge 2$, there exists a stochastic AP-MDM as in \Cref{def:stochastic_apmdm} with constant-depth Transformer backbone such that the support of the induced distribution of AP-MDM over strings $w$ is exactly equal to $\TDyck_k$, that is, 
\begin{enumerate}
\item (\emph{Coverage}) For every $w\in \TDyck_k$, $\Pr_\theta[w]>0$.
\item (\emph{Support exactness}) For every $w\notin \TDyck_k$, $\Pr_\theta[w]=0$.
\end{enumerate}
Conversely, under the common hardness assumption that $\textsf{TC}^0\neq \textsf{NC}^1$, for any constant-depth ARM with polylogarithmic embedding dimension, there exists $N\in\mathbb{N}$ such that the support of the distribution generated ARM cannot be exactly equal to $\TDyck_k$.
\end{theorem}

\begin{proof}

For the claim about ARM, we will prove by contradiction. First we note that every $w\in\Sigma_k^*$, there exists a $w'\in \TDyck_k$ such that $w$ is a prefix of $w'$. (The existence of such $w'$ is straightforward, for example, one can construct it by taking $w' = w w^{-1}$, where the inverse is performed by viewing $w$ as an element of the corresponding free group) Now we suppose ARM can indeed generate a distribution whose support is exactly equal to $\TDyck_k$. 
% Then generating arbitrary-length matched bracket sequences requires the model to 
This implies that the ARM must be able to determine at each generation step whether the current sequence can be terminated. Specifically, the model must assess whether the currently generated sequence satisfies the complete bracket matching condition. If the sequence is properly matched, the probability of outputting $\texttt{[EOS]}$ must be non-zero to enable termination. Therefore, the difficulty of generating matched brackets reduces to the problem of recognizing whether a given sequence forms valid matched brackets. For the two-sided Dyck-$k$ language, this recognition problem is DLOGTIME-uniform $\textsf{NC}^1$-hard~\citep{robinson1993parallel}, which exceeds the computational capacity of constant-depth Transformers which is in $\textsf{TC}^0$, under the hardness assumption that $\textsf{TC}^0\neq \textsf{NC}^1$. Thus we conclude ARM cannot generate a distribution over $\Sigma_k^*$ whose support is exactly equal to $\TDyck_k$.

For the stochastic AP-MDM, we construct the following algorithm to generate all strings in the two-sided Dyck-$k$ language through the following algorithmic procedure, illustrated in \Cref{fig:dyck_example}:

\textbf{Step 1 (Probabilistic Mask Insertion):} If the current sequence contains no mask tokens, then for any sequence position $j$ not containing an end-of-sequence token, the model inserts $\mask_1$ with constant probability $p \in (0,1)$ using the insert operation from \Cref{sec:any_process}. The insertion probability for $\mask_2$ is set to zero at this stage.

\textbf{Step 2 (Uniform Token Selection):} At positions containing $\mask_1$, the model performs two operations: (i) it samples uniformly from the bracket token set $\{a_i^{\pm 1}\}_{i=1}^k$ to determine the content, and (ii) it inserts $\mask_2$ with probability 1. Due to the priority-based insertion mechanism defined in \Cref{def:stochastic_apmdm}, $\mask_2$ overrides $\mask_1$, making the original insertion probability irrelevant for subsequent processing.

\textbf{Step 3 (Context-Aware Bracket Matching):} When processing $\mask_2$ tokens, the model identifies the nearest bracket token $a_j^{\pm 1}$ to the left of the current position and generates the corresponding matching bracket according to the two-sided Dyck-$k$ reduction rules.

\textbf{Termination Condition:} The termination mechanism operates as follows: In Step 2, when the sequence contains $\mask_1$ tokens but no $\mask_2$ tokens, the model inserts $\mask_2$ at the final position with a fixed probability. Subsequently, in Step 3, when processing this final $\mask_2$ token, the model generates $\texttt{[EOS]}$ (given the binary positional encoding we considered in \Cref{appendix:pram}, the model is able to identify if the token should be decoded as $\texttt{[EOS]}$ or a matching bracket), signaling the end of the generation process.


 The above algorithm admits an $\efasp$ program implementation (see \Cref{appendix:fasp}) with the following treatment of stochastic operations:

For uniform sampling over the $2k$ bracket tokens, we exploit the fixed vocabulary size by assigning each bracket token $a_i^{\pm 1}$ to distinct dimensions in the $d$-dimensional embedding space. Specifically, when the $\efasp$ program needs to output a uniform distribution over a subset $S \subseteq \{a_i^{\pm 1}\}_{i=1}^k$, it returns a hidden state $\mathbf{h} \in \mathbb{R}^d$ where $\langle \mathbf{h}, \te(v) \rangle = c$ for all $v \in S$ (for some constant $c$) and $\langle \mathbf{h}, \te(v') \rangle < c$ for $v' \notin S$. The hardmax from \Cref{def:stochastic_apmdm}, ensures that the probability mass concentrates uniformly over $S$, achieving the desired uniform sampling behavior.

It is easy to see that this generation procedure can produce any string in the two-sided Dyck-$k$ language, and since any token that would violate Dyck constraints always has strictly smaller logit and hence probability $0$ under Hardmax, the support of the distribution is exactly $\TDyck_k$.
\end{proof}

We note the introduction of two mask tokens is for the model to distinguish between different steps, but this is not necessary if we allow some random seeds in input which mitigates the limitation of MDM when extending to the non-deterministic case.




\section{Proof of \Cref{thm:apmdm_simulation2}} \label{appendix:proof_apmdm_simulation2}


\paragraph{Edit Triplet Encoding.}
We encode each elementary edit as a triplet of tokens $\langle \texttt{op},\, \texttt{pos},\, \texttt{val} \rangle \in \Sigma_{\mathrm{op}} \times \Sigma_{\mathrm{pos}} \times \Sigma$, where
\begin{equation}
\Sigma_{\mathrm{op}}=\{\texttt{UNMASK},\texttt{INSERT},\texttt{DELETE},\texttt{REMASK}\}
\end{equation}
The position token set $\Sigma_{\mathrm{pos}}$ reuses the earlier address/position encoding in \Cref{def:addr_tok}: a position $i \in [S(n)]$ is encoded as $\texttt{pos}=\texttt{encode}(i)$ with inverse decoding $\texttt{dec\_pos}(\texttt{pos})=i$. 

The semantics of the triplet follows the instantiation of AP-MDM considered in \Cref{sec:any_process}.

\begin{definition}[Editing Sequence]\label{def:editing_sequence}
Given an input sequence $\mathbf{x}\in \bar{\Sigma}^*$, an editing sequence is a finite sequence of triplets
\begin{align}
    T(\mathbf{x}) = \big(\langle \mathrm{op}_j,\, \mathrm{pos}_j,\, \mathrm{val}_j \rangle\big)_{j=1}^{m(\mathbf{x})},\quad\mathrm{op}_j\in\Sigma_{\mathrm{op}},\; \mathrm{pos}_j\in\Sigma_{\mathrm{pos}},\; \mathrm{val}_j\in\Sigma.
\end{align}

Its application to $\mathbf{x}$ is defined recursively by $\mathbf{x}^{(0)}=\mathbf{x}$ and
\begin{equation}
\mathbf{x}^{(j)} = \mathrm{Apply}\big(\langle \mathrm{op}_j,\mathrm{pos}_j,\mathrm{val}_j\rangle,\, \mathbf{x}^{(j-1)}\big),\quad j=1,\ldots,m(\mathbf{x}).
\end{equation}
We write
\begin{equation}
\mathrm{Apply\_Triplets}\big(T(\mathbf{x}),\, \mathbf{x}\big) := \mathbf{x}^{(m(\mathbf{x}))}.
\end{equation}
An editing sequence is valid iff every intermediate application is well-defined under the triplet semantics (e.g., $\texttt{UNMASK}$ applies only to masks).
\end{definition}

\begin{theorem}[Hardness of Simulating AP-MDM, Formal] \label{thm:apmdm_simulation2_formal}
There exists an AP-MDM $F$ with a constant-depth encoder-only Transformer backbone such that no ARM or Masked-ARM $G$ (\Cref{def:masked_arm}) with a constant-depth decoder-only Transformer backbone can, on every input $\mathbf{x}$, produce an editing sequence $T_G(\mathbf{x})$ (\Cref{def:editing_sequence}) that realizes $F$'s generation process; i.e., under the assumption that constant-depth Transformers do not include $\textsf{TC}^0$,
\[
\forall G\;\exists\, \mathbf{x}\in\bar{\Sigma}^*:\; \mathrm{Apply\_Triplets}\big(T_G(\mathbf{x}),\, \mathbf{x}\big) 
\;\neq\; 
\mathrm{Apply\_Triplets}\big(T_F(\mathbf{x}),\, \mathbf{x}\big),
\]
or $T_G(\mathbf{x})$ is invalid.
\end{theorem}

The ARM in the above result can be replaced by the Masked-ARM with encoder architecture used in \Cref{thm:any_order} without affecting the result.

\begin{proof}
Fix $L\in\mathbb N$. Let $\mathbf u\in\Sigma^L$ be the base string, let $T$ be a valid editing sequence (\Cref{def:editing_sequence}), and let $q\in\Sigma_{\mathrm{pos}}$ be a query position token with index $i=\mathrm{dec\_pos}(q)$. Encode the input as
\begin{equation}
\mathbf x 
= (\mathbf u,\; \texttt{[SEP]},\; \mathrm{flatten}(T),\; \texttt{[SEP]},\; q,\; \texttt{[SEP]}) 
\in \Sigma^{L+3+3m(\mathbf u)}.
\end{equation}
Here
\begin{equation}
\mathrm{flatten}(T)=(\mathrm{op}_1,\mathrm{pos}_1,\mathrm{val}_1,\ldots,\mathrm{op}_{m},\mathrm{pos}_{m},\mathrm{val}_{m}).
\end{equation}
The task is to output the queried symbol after applying the editing history:
\begin{equation}
 y\;=\; \big[\mathrm{Apply\_Triplets}\big(T(\mathbf u),\, \mathbf u\big)\big]_i\;\in\;\Sigma.
\end{equation}
That is, the instance provides (i) a base string $\mathbf u$, (ii) an editing history $T$ as a sequence of triplets, and (iii) a query position token $q$. The model must simulate $T$ on $\mathbf u$ and return the symbol at the queried position $i$ in the resulting string. For AP-MDM, the simulation is intuitive and can be proven by simple $\efasp$ program which we skip in this proof.

For ARM, due to the construction of the problem, the simulation process 
is exactly copying the editing sequence part in the input, therefore 
solving the problem is equivalent to directly answer the query, which we 
show the equivalence to a $\mathsf{TC}^0$-hard task:

\begin{definition}[{\sc Preserves}~\citep{allender2006grid}]
    Let $A$ be an ordered list (1-indexed). The update alphabet is
    \begin{equation}
    \mathcal{U}=\{\texttt{insert}(i),\,\texttt{delete}(i)\mid i\in\mathbb{N}\}.
    \end{equation}
    For an initial list $A_0$ and an update sequence $s\in\mathcal{U}^*$, let $A_t$ be the list after applying the first $t$ updates of $s$.
    For indices $i,j\in\mathbb{N}$, define
    \begin{equation}
    \mathrm{PRESERVES}(A_0,s,i,j)\iff
    \text{the item at position }i\text{ in }A_0\text{ still exists after }s\text{ and is at position }j\text{ in }A_{|s|}.
    \end{equation}
    The decision problem {\sc Preserves} asks, given $(A_0,s,i,j)$, whether $\mathrm{PRESERVES}(A_0,s,i,j)$ holds.
\end{definition}

\begin{conjecture} \label{conj:preserves_hardness}
The {\sc Preserves} problem is $\mathsf{NC}^1$-hard under DLOGTIME-uniform reductions.
\end{conjecture}
    

We reduce \textsc{Preserves} to our Editing-Query task in one step. Given $(A_0,s,i,j)$ with $|A_0|=L$, let $\mathbf u\in\Sigma^L$ list the items of $A_0$ (unique token id per item); expand each $\texttt{insert}(p,v)$ as the triplet block $\big(\langle \texttt{INSERT},\mathrm{encode}(p),\mask\rangle,\langle \texttt{UNMASK},\mathrm{encode}(p'),v\rangle\big)$ and each $\texttt{delete}(p)$ as $\big(\langle \texttt{REMASK},\mathrm{encode}(p),\bullet\rangle,\langle \texttt{DELETE},\mathrm{encode}(p),\bullet\rangle\big)$, where $p'$ is the position of the newly inserted mask under our convention and $\bullet$ is ignored; let $T$ be the concatenation over $s$ and set $q=\mathrm{encode}(j)$. Then with
\begin{equation}
 y\;=\; \big[\mathrm{Apply\_Triplets}\big(T,\mathbf u\big)\big]_j,
\end{equation}
we have
\begin{equation}
\mathrm{PRESERVES}(A_0,s,i,j)\;\Longleftrightarrow\; y=\mathrm{id}(i).
\end{equation}
Thus any model that solves Editing-Query on all inputs also decides \textsc{Preserves}. By \Cref{conj:preserves_hardness}, \textsc{Preserves} is $\mathsf{NC}^1$-hard, which places it beyond the computational capacity of constant-depth Transformers that are contained in $\mathsf{TC}^0$. Under this conjecture, no constant-depth ARM or Masked-ARM can solve our Editing-Query task on all inputs. This completes the proof.

\end{proof}


\vspace{-10pt}
\section{Experiment: Sudoku Puzzle} \label{appendix:examples_sudoku}
\vspace{-10pt}


We provide detailed description of training data generation for the Sudoku puzzle experiment. The data generation process simulates a backtracking-based solving algorithm and records the intermediate states as supervised training trajectories for AP-MDM.

\paragraph{State Representation}
The $9 \times 9$ Sudoku grid is represented as a sequence of 324 tokens, where each cell is encoded using 3 consecutive tokens: $(\text{value}, \text{color}, \text{marker})$. The vocabulary consists of 32 tokens including: EMPTY (unfilled cell), MASK (unknown position to be predicted), digits 1-9 (filled values), WHITE (default color indicating no branch), 15 branch COLOR tokens (tracking different branching paths), NORMAL (standard state), SKULL (failed branch point), BRANCH (active branch starting position), and SEPARATOR (structural delimiter between cells). The value token can be EMPTY (for unfilled cells), a digit 1-9 (for filled cells), or MASK (for positions being predicted). The color token tracks which branching decision path the cell belongs to, using WHITE for non-branched cells and one of 15 COLOR tokens for cells within branches. The marker token indicates the cell's computational status: NORMAL for standard cells, SKULL for positions that caused backtracking, and BRANCH for branch starting points.

The solving algorithm consists of several atomic operations, each translated into state-transition tuples for AP-MDM training:

\textbf{Assign and Branch:} The two most fundamental operations in Sudoku solving are deterministic assignment and branch creation, both implemented through combinations of $\remask$ and $\unmask$ operations. For Assign, when deterministically filling a cell with value $v$, we generate a 2-step transition: first, the three tokens at the target position are remasked, converting them to $(\mask, \mask, \mask)$; second, these masks are unmasked to the target values $(v, c, \text{NORMAL})$ where $c$ is the appropriate color token determined by the current branch context. For Branch, when creating a branch at position $(r, c)$ with candidate value $v$ and branch identifier $b$, we similarly generate a 2-step transition: first, remask the cell tokens to $(\mask, \mask, \mask)$; second, unmask to $(v, \text{COLOR}_b, \text{BRANCH})$ to mark this cell as a branch starting point with the corresponding branch color. 

Concrete examples visualizing these two operations are shown in \Cref{fig:sudoku_assign_branch}.

\textbf{Contradiction Marking, Backtrack, and Recovery:} When the solver detects a contradiction (a cell with no valid candidates), before backtracking, we mark this contradicting position through a 2-step transition: first remask the cell to $(\mask, \mask, \mask)$, then unmask to $(\text{EMPTY}, \text{COLOR}_b, \text{SKULL})$ to visually indicate the contradiction location. When backtracking from a failed branch, multiple cells need to be cleared through a 2-step transition that handles both ordinary cells and the branch starting point differently: in the first step, ordinary cells are remasked in all three token positions while the branch starting point only has its marker token remasked; in the second step, ordinary cells are unmasked to $(\text{EMPTY}, \text{WHITE}, \text{NORMAL})$ while the branch starting point undergoes simultaneous remask and unmask operations with the first two positions remasked to $(\mask, \mask)$ and the third position unmasked to store the failed value. After backtracking, when filling the failed position with a new value, we generate a 2-step transition to convert from the SKULL state: the first step unmasks the value and color while remasking the marker to $(\mask, \mask, v_{\text{old}}) \to (v_{\text{new}}, c, \mask)$, and the second step unmasks the marker to either NORMAL or BRANCH depending on whether this creates a new branch. Concrete examples visualizing these operations are shown in \Cref{fig:sudoku_other_ops}.


\paragraph{Training Data Statistics}
Following this generation process, each Sudoku puzzle produces 1421.3 state-transition tuples (i.e. from $\mathbf x_t$ to $\mathbf x_{t+1}$) on average depending on the puzzle difficulty and the number of backtracking steps required. For our experiments, we generated training data from 100 hard Sudoku puzzles, each yielding 25,022.6 training state-transition tuples on average.

\begin{figure}[t]
	\centering
	\subfigure[]{\includegraphics[width=1.0\textwidth]{figs/sample_595.pdf}}
	
	
	\subfigure[]{\includegraphics[width=1.0\textwidth]{figs/sample_596.pdf}}
	\caption{Demonstration of value assignment.}
	\label{fig:sudoku_assign_branch}
\end{figure} 

\begin{figure}[t]
	\centering
    \ContinuedFloat
	\subfigure[]{\includegraphics[width=1.0\textwidth]{figs/sample_611.pdf}}
	
	\subfigure[]{\includegraphics[width=1.0\textwidth]{figs/sample_612.pdf}}
	\caption{Demonstration of backtracking and recovery from failure operations (Part 1). (Continued on next page)}
	\label{fig:sudoku_other_ops}
\end{figure} 

\begin{figure}[p]
	\ContinuedFloat
	\centering
	\subfigure[]{\includegraphics[width=1.0\textwidth]{figs/sample_613.pdf}}
	
	\subfigure[]{\includegraphics[width=1.0\textwidth]{figs/sample_614.pdf}}
	\caption{(Continued) Part 2.}
\end{figure} 

\begin{figure}[p]
	\ContinuedFloat
	\centering
	\subfigure[]{\includegraphics[width=1.0\textwidth]{figs/sample_615.pdf}}
	
	\subfigure[]{\includegraphics[width=1.0\textwidth]{figs/sample_616.pdf}}
	\caption{(Continued) Part 3.}
\end{figure} 

\clearpage


\section{Experiment: Graph Generation} \label{appendix:examples_graph}

We consider a graph editing task that requires computing the minimum edge set to disconnect two specified nodes. Formally, given a directed graph $G = (V, E)$ with unit-capacity edges and two designated nodes $s, t \in V$ (source and target), the task is to generate a modified graph $G' = (V, E')$ where $E' \subseteq E$ such that there exists no path from $s$ to $t$ in $G'$, and $|E \setminus E'|$ (the number of removed edges) is minimized. This is equivalent to computing the minimum $s$-$t$ cut, which by the max-flow min-cut theorem equals the maximum flow from $s$ to $t$. The generation process involves iteratively finding augmenting paths, modifying the graph structure by reversing edge directions, and tracking intermediate states until no more augmenting paths exist. The final output is the graph with min-cut edges removed, effectively disconnecting $s$ and $t$.

We provide detailed description of training data generation for this graph editing task based on the Edmonds-Karp algorithm. The data generation process simulates the BFS-based augmenting path search and records the intermediate algorithmic states as supervised training trajectories for AP-MDM.

\paragraph{State Representation}
A directed graph with $n$ nodes and $m$ edges is represented as a token sequence with three main components: prompt (source and target nodes), graph data (edge list with features), and node data (node list with features). Each edge is encoded with its endpoints $(u, v)$ and two feature slots tracking the edge's directional availability: slot1 represents forward direction availability (initially FB for forward-backward) and slot2 represents reverse direction availability (initially MASK indicating unavailable). Each node is encoded with its ID and two features: level (BFS layer, initially INF for unvisited or LVL0 for source) and parent (parent node in BFS tree, initially NIL). The vocabulary includes structural tokens (PROMPT, SRC, TGT, GRAPH, NODES, parentheses), edge feature tokens (FB, MASK), node feature tokens (LVL0-LVL9, INF, NIL, PAR), node IDs (0-299), and termination tokens (EOA, EOS).

\paragraph{Atomic Operations}
The Edmonds-Karp algorithm is decomposed into atomic operations, each translated into state-transition tuples for AP-MDM training. The algorithm consists of four phases:

\textbf{Feature Expansion:} Before the algorithm begins, edge and node feature slots must be initialized through a 3-step process following a expansion-then-unmask paradigm. First, MASK tokens are inserted after each edge's second node and after each node's ID ($\inser$ operation). Second, another MASK is inserted while simultaneously unmasking the first MASK to FB for edges and to the appropriate level for nodes ($\inser$ + $\unmask$ operations). Third, the second MASK is unmasked to MASK for edges and to NIL for nodes ($\unmask$ operation). See \Cref{fig:graph_feature_expansion}.

\textbf{Breadth-First Search:} The breadth-first search proceeds by discovering nodes layer by layer. When a new node is discovered through an edge, we generate a 2-step transition: first, remask the node's level and parent features to MASK ($\remask$ operation); second, unmask these MASKs to the new level and parent ID ($\unmask$ operation). This continues until either the target node is reached (proceed to augmentation) or no new nodes can be discovered (algorithm terminates). See \Cref{fig:graph_bfs}.

\textbf{Path Augmentation:} When an augmenting path from source to target is found, we generate a 2-step transition to flip edges along the path and reset node features. First, for each edge on the path, swap its slot1 and slot2 values (reversing the direction), and simultaneously remask all node features to MASK ($\remask$ operation). Second, unmask all node MASKs back to their initial values: INF for non-source nodes and LVL0 for the source node, with all parents set to NIL ($\unmask$ operation). After augmentation, the algorithm returns to BFS phase to search for the next augmenting path. See \Cref{fig:graph_augmentation}.

\textbf{Termination and Structural Editing:} When no more augmenting paths exist, the final BFS identifies two disjoint sets $S$ and $T$ where $S$ contains the source and $T$ contains the target. The min-cut edges are those directed from nodes in $S$ to nodes in $T$ in the original graph, and these edges must be removed to disconnect source and target. This is accomplished through a 3-step process. First, remask all tokens representing the min-cut edges to MASK ($\remask$ operation). Second, delete these edges while simultaneously expanding a MASK token after EOA ($\delete$ + $\inser$ operations). Third, unmask the expanded MASK to EOS ($\unmask$ operation), marking algorithm completion. See \Cref{fig:graph_termination}.

% Following this generation process, each graph instance produces state-transition tuples proportional to the number of augmenting paths and the BFS steps required to find them. Our training set includes graphs with varying structures, densities, and node IDs, providing diverse algorithmic scenarios for the model to learn from.

\paragraph{ARM Baseline}
For comparison with AP-MDM, we train an autoregressive baseline that learns to generate the complete solving trajectory as a single sequence. The ARM baseline takes the initial graph state as input and must generate the full sequence of operations needed to solve the task. To enable this, we convert the AP-MDM training data into an ARM-compatible format by representing each operation as a triplet $(p, o, v)$ where $p$ is the position index (POSE0-POSE399), $o$ is the operation type (REMASK, UNMASK, INSERT, DELETE), and $v$ is the value (or NONE for operations without values). The sequence format is: $[\text{initial state } \mathbf{x}_0]$ STEP $[\text{operations}_1]$ STEP $[\text{operations}_2]$ STEP $\cdots$ ANSWER $[\text{final state } \mathbf{x}_{\text{final}}]$, where each STEP separates consecutive state transitions and ANSWER marks the beginning of the final output. For ARM training, we use the standard next-token prediction objective with teacher forcing, where the model learns to autoregressively generate the entire operation sequence given the initial state.


\begin{table}[t]
	\centering
\caption{Sequence length statistics for graphs of different sizes. AP-MDM sequences contain state-transition tuples, while ARM sequences enumerate all operations explicitly.}
\label{tab:graph_sequence_stats}
\resizebox{\textwidth}{!}{
\begin{tabular}{c|c|cc|cc}
\toprule
\multirow{2}{*}{\textbf{\# Nodes}} & \multirow{2}{*}{\textbf{\# Edges}} & \multicolumn{2}{c|}{\textbf{AP-MDM}} & \multicolumn{2}{c}{\textbf{ARM}} \\
& & \textbf{Avg. Seq. Length} & \textbf{Max Seq. Length} & \textbf{Avg. Seq. Length} & \textbf{Max Seq. Length} \\
\midrule
4 & 12 & 56 & 65 & 932 & 932 \\
5 & 17 & 81 & 88 & 1,375 & 1,403 \\
6 & 23 & 114 & 129 & 2,083 & 2,140 \\
7 & 29 & 148 & 170 & 2,687 & 2,874 \\
8 & 36 & 189 & 217 & 3,529 & 3,865 \\
9 & 43 & 236 & 252 & 5,586 & 6,597 \\
10 & 50 & 270 & 305 & 4,915 & 5,392 \\
\bottomrule
\end{tabular}
}
\end{table} 


As shown in \Cref{tab:graph_sequence_stats}, the sequence length grows with graph size for both AP-MDM and ARM, but ARM requires significantly longer sequences due to the explicit operation enumeration. 

\begin{figure}[t]
	\centering
	\setcounter{subfigure}{0}
	\subfigure[]{\includegraphics[width=1.0\textwidth]{figs/graph_0.pdf}}
	
	\vspace{0.2cm}
	
	\subfigure[]{\includegraphics[width=1.0\textwidth]{figs/graph_1.pdf}}
	
	\vspace{0.2cm}
	
	\subfigure[]{\includegraphics[width=1.0\textwidth]{figs/graph_2.pdf}}
	\caption{Feature Expansion phase in graph generation, showing the initialization process for edge and node feature slots using $\inser$ and $\unmask$ operations.}
	\label{fig:graph_feature_expansion}
\end{figure} 

\begin{figure}[t]
	\centering
	\setcounter{subfigure}{0}
	\subfigure[]{\includegraphics[width=1.0\textwidth]{figs/graph_3.pdf}}
	
	\vspace{0.2cm}
	
	\subfigure[]{\includegraphics[width=1.0\textwidth]{figs/graph_4.pdf}}
	\caption{Parallelized BFS phase in graph generation, showing layer-by-layer node discovery with parallel processing using $\remask$ and $\unmask$ operations.}
	\label{fig:graph_bfs}
\end{figure}

\begin{figure}[t]
	\centering
	\setcounter{subfigure}{0}
	\subfigure[]{\includegraphics[width=1.0\textwidth]{figs/graph_5.pdf}}
	
	\vspace{0.2cm}
	
	\subfigure[]{\includegraphics[width=1.0\textwidth]{figs/graph_6.pdf}}
	\caption{Path Augmentation phase in graph generation, showing edge reversal and node feature reset after finding an augmenting path using $\remask$ and $\unmask$ operations.}
	\label{fig:graph_augmentation}
\end{figure}

\begin{figure}[t]
	\centering
	\setcounter{subfigure}{0}
	\subfigure[]{\includegraphics[width=1.0\textwidth]{figs/graph_29.pdf}}
	
	\vspace{0.2cm}
	
	\subfigure[]{\includegraphics[width=1.0\textwidth]{figs/graph_30.pdf}}
    \vspace{0.2cm}
	
	\subfigure[]{\includegraphics[width=1.0\textwidth]{figs/graph_31.pdf}}
	\caption{Termination and Structural Editing phase, showing the deletion of min-cut edges and algorithm completion using $\remask$ and $\delete$ operations.}
	\label{fig:graph_termination}
\end{figure}

\section{Experiment: Parity}  \label{appendix:examples_parity}


The parity task requires determining whether a binary sequence contains an even or odd number of 1s. Formally, given an input sequence $\mathbf{x} = (x_1, x_2, \ldots, x_n) \in \{0, 1\}^n$, the task is to compute $\bigoplus_{i=1}^n x_i$ (XOR of all bits), outputting 0 for even parity and 1 for odd parity. 

\paragraph{Algorithm and Data Generation}
For AP-MDM training, we implement an elimination algorithm that mimics how humans naturally solve parity: repeatedly remove pairs of identical elements until only the result remains. The vocabulary consists of 5 tokens: BOS (sequence start), EOS (sequence end), MASK, digit 0, and digit 1. The elimination process follows these rules: when encountering 0s in the sequence, convert them to MASK; when encountering a pair of 1s, convert both to MASK; then delete the MASK tokens. This process repeats until only BOS remains (for even parity) or BOS followed by a single 1 remains (for odd parity). Each elimination step generates a state-transition tuple: converting tokens to MASK is a $\remask$ operation, and removing MASKs is a $\delete$ operation.

\paragraph{Data}
The training data consists of only 4 instances that cover all possible elimination patterns. Each sample is a single-step state transition demonstrating one atomic operation. The test set contains 1,000 randomly generated binary sequences with variying lengths. The test sequences have approximately equal distribution of even and odd parities.

\paragraph{ARM Baseline}
For the ARM baseline, we train autoregressive models with chain-of-thought reasoning where the model generates intermediate cumulative XOR values at each position before outputting the final result. The sequence structure is: BOS $[x_1 \, x_2 \, \cdots \, x_n]$ EOP $[s_1 \, s_2 \, \cdots \, s_n \, \text{result}]$ EOS, where $s_i = \bigoplus_{j=1}^i x_j$ represents the cumulative XOR up to position $i$, and result is True (for odd parity) or False (for even parity). Only the content after EOP is used for computing the training loss. We train ARM models with up to 10K training instances at various fixed lengths (e.g., length 2, 10, 50, 100) and evaluate their ability to generalize to longer unseen lengths. 









\end{document}

