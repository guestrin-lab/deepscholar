\begin{abstract}
Composed Image Retrieval (CIR), which aims to find a target image from a reference image and a modification text, presents the core challenge of performing unified reasoning across visual and semantic modalities. While current approaches based on Vision-Language Models (VLMs, e.g., CLIP) and more recent Multimodal Large Language Models (MLLMs, e.g., Qwen-VL) have shown progress, they predominantly function as ``black boxes." This inherent opacity not only prevents users from understanding the retrieval rationale but also restricts the models' ability to follow complex, fine-grained instructions. To overcome these limitations, we introduce CIR-CoT, the first end-to-end retrieval-oriented MLLM designed to integrate explicit Chain-of-Thought (CoT) reasoning. By compelling the model to first generate an interpretable reasoning chain, CIR-CoT enhances its ability to capture crucial cross-modal interactions, leading to more accurate retrieval while making its decision process transparent. Since existing datasets like FashionIQ and CIRR lack the necessary reasoning data, a key contribution of our work is the creation of structured CoT annotations using a three-stage process involving a caption, reasoning, and conclusion. Our model is then fine-tuned to produce this structured output before encoding its final retrieval intent into a dedicated embedding. Comprehensive experiments show that CIR-CoT achieves highly competitive performance on in-domain datasets (FashionIQ, CIRR) and demonstrates remarkable generalization on the out-of-domain CIRCO dataset, establishing a new path toward more effective and trustworthy retrieval systems.
\end{abstract}