\section{Appendix}
We provide additional results and details that were not included in the main text due to space limitations.
\subsection{Correlation Heatmaps: Uncertainty versus Prediction Error}
Fig.~\ref{fig:heatmaps} shows correlation heatmaps discussed in Section \label{subsec:unc_analysis} in the main text. This heatmap visualizes the relationship between the Mean Absolute Error (MAE) of MoGU's predictions and its reported uncertainties (aleatoric, epistemic, and total) for a model using three iTransformer experts.
    \begin{figure}[h!]
        \centering
        \begin{subfigure}[t]{0.24\textwidth}
            \centering
            \includegraphics[width=\linewidth]{Chapters/Figures/itransformer_etth1_heatmap_elig.png}
             \subcaption{ETTh1}
            \label{fig:heatmap_a}
        \end{subfigure}
        \hfill % Adds horizontal space between subfigures
        \begin{subfigure}[t]{0.23\textwidth}
            \centering
            \includegraphics[width=\linewidth]{Chapters/Figures/itransformer_etth2_heatmap_elig.png}
             \subcaption{ETTh2}
            \label{fig:heatmap_b}
        \end{subfigure}
        \hfill % Adds horizontal space between subfigures
        \begin{subfigure}[t]{0.23\textwidth}
            \centering
            \includegraphics[width=\linewidth]{Chapters/Figures/itransformer_ettm1_heatmap_elig.png}
             \subcaption{ETTm1}
            \label{fig:heatmap_c}
        \end{subfigure}
        \hfill % Adds horizontal space between subfigures
        \begin{subfigure}[t]{0.23\textwidth}
            \centering
            \includegraphics[width=\linewidth]{Chapters/Figures/itransformer_ettm2_heatmap_elig.png}
            \subcaption{ETTm2}
            \label{fig:heatmap_d}
        \end{subfigure}
        
        \caption{Heatmaps of the Pearson correlation between MoGU's reported uncertainties (aleatoric, epistemic, and total) and the MAE of its predictions. The correlation is displayed per variable for the ETT datasets.}
        \label{fig:heatmaps}
    \end{figure}
\subsection{Additional Ablations}\label{subsec:app_ablations}
\textbf{Resolution of Uncertainty Estimation.} We provide Table \ref{tab:ablations_time}, discussed in the main text. This table explores an alternative where the expert estimates uncertainty at the variable level ('Time-Fixed'), rather than for each individual time point ('Time-Varying').

\textbf{Loss Function.} We note that the MoGU model can also be optimized through the following MoG loss:
\begin{equation}\label{eq:mog_loss_alt}
    \mathcal{L}_{}= - \log (\sum_i w_i(x)\mathcal{N}(y;y_i(x), \sigma_i^2(x))) 
\end{equation}
where $\mathcal{N}$ is the Normal density function and the loss
has the form of a Negative Log Likelihood (NLL) of a MoG distribution.
We compare the performance of our model when using the loss presented in Eq. \ref{eq:mog_loss} and when using the aforementioned alternative (Eq. \ref{eq:mog_loss_alt}). The results of this experiment, presented in Table \ref{tab:ablations_loss} in our Appendix, suggest that optimizing with our proposed loss (Eq. \ref{eq:mog_loss}) yields more effective learning and consistently better results by imposing a stricter constraint on expert learning compared to the MoG loss.

\linespread{1.1}
\begin{table*}[h!]
\small
\caption{Ablation study on the resolution of reported uncertainty. We compare two methods for estimating variance in both MoE and MoGU: estimating it once per horizon versus estimating it for each time point (per variable in both cases). The table reports the MAE and MSE for each configuration. All results were generated using iTransformer as the expert architecture with a 96-time-point horizon.}
	\label{tab:ablations_time}
	\centering
		\begin{tabular}{cc|c|cc|cc}
    \cline{2-7}
    & \multicolumn{2}{c|}{Prediction Variance}& \multicolumn{2}{c|}{Time-Fixed}& \multicolumn{2}{c}{Time-Varying}\\
    \cline{2-7}
    & \multicolumn{2}{c|}{Metric}& MAE & MSE & MAE & MSE\\
    \cline{2-7}
\multicolumn{3}{c|}{ETTh1} & 0.401 & 0.392 & \textbf{0.400 }& \textbf{0.380}   \\
\multicolumn{3}{c|}{ETTh2} &  0.337 & 0.290 & \textbf{0.336} & \textbf{0.283} \\
\multicolumn{3}{c|}{ETTm1} & 0.360 & 0.324 & \textbf{0.356} & \textbf{0.320} \\
\multicolumn{3}{c|}{ETTm2}  &  \textbf{0.255} & \textbf{0.174}  & 0.260 & 0.179  \\
\cline{2-7}
\end{tabular}
\end{table*}
\linespread{1.0}
\linespread{1.1}
\begin{table*}[h!]
\small
\caption{Ablation study of MoGU's loss. We compare the loss formulation in Eq. \ref{eq:mog_loss}, used by MoGU to an alternative MoG loss, given in Eq. \ref{eq:mog_loss_alt} }
	\label{tab:ablations_loss}
	\centering
		\begin{tabular}{cc|c|cc|cc}
    \cline{2-7}
    & \multicolumn{2}{c|}{Loss Formulation}& \multicolumn{2}{c|}{Eq. \ref{eq:mog_loss_alt} (Alt. MoG loss)}& \multicolumn{2}{c}{Eq. \ref{eq:mog_loss} (MoGU's loss)}\\
    \cline{2-7}
    & \multicolumn{2}{c|}{Metric}& MAE & MSE & MAE & MSE\\
    \cline{2-7}
\multicolumn{3}{c|}{96} & 0.343 & 0.304 & \textbf{0.336 }& \textbf{0.283}   \\
\multicolumn{3}{c|}{192} & 0.389 & 0.378 & \textbf{0.387} & \textbf{0.361} \\
\multicolumn{3}{c|}{336} & \textbf{0.424 }& 0.422 & 0.425 & \textbf{0.415} \\
\multicolumn{3}{c|}{720}  &  \textbf{0.438} & \textbf{0.421}  & 0.442 & \textbf{0.421}  \\
\cline{2-7}
\end{tabular}
\end{table*}
\linespread{1.0}
\subsection{MoGU's Algorithm}
We provide the pseudo code for MoGU in Listing 1 to enhance clarity.
Furthermore, to ensure reproducibility, our code and the scripts needed to reproduce the main results are available at: \url{https://github.com/yolish/moe_unc_tsf}
We implemented MoGU to be highly configurable, so that users can  specify the number of experts, the expert architecture, the mixture type (MoE or MoG) and the gating mechanism.

\begin{algorithm}[t]
\label{listing:mogu}
\caption{Mixture-of-Gaussians with Uncertainty-based gating (MoGU)}
\begin{algorithmic}[1]  % line numbers
\Require Training set $X$, labels $y$
\Ensure Model parameters $\theta$

\For{each training epoch}
  \For{each mini-batch $\mathcal{B}$}
    \For{each sample $x \in \mathcal{B}$}
      \For{each expert $i=1,\dots,k$}
        \State Compute expert output $f_i(x) = \mathcal{N}(y; \mu_i(x,\theta),\sigma_i^2(x,\theta))$.
        \State Set $w_i(x)= \frac{\sigma_i^{-2}(x)}{\sum_j \sigma_j^{-2}(x)}$.
      \EndFor
     \EndFor
    %\State Compute loss $\mathcal{L} = -\sum_{x \in \mathcal{B}} \sum_{i=1}^k w_i(x) \log (f_i(x))$.
    \State Compute loss $\mathcal{L}= \sum w_i(x)\mathcal{L_{NLLG}}(y;y_i(x), \sigma_i^2(x)))$.
    \State Update model parameters.
  \EndFor
\EndFor 

\State Test time prediction is $\hat{y}(x) = \sum_i w_i(x) \mu_i(x)$.
\State Test time prediction uncertainty is: $\sum_i w_i(x) \sigma_i^2(x) +  \sum_i w_i(x) (\hat{y}(x)-\mu_i(x))^2$.
\end{algorithmic}
\end{algorithm}