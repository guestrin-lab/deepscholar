
\documentclass{article}

% if you need to pass options to natbib, use, e.g.:
%     \PassOptionsToPackage{numbers, compress}{natbib}
% before loading neurips_2025


% ready for submission
%\usepackage{neurips_2025}


% to compile a preprint version, e.g., for submission to arXiv, add add the
% [preprint] option:
\usepackage[preprint]{neurips_2025}


% to compile a camera-ready version, add the [final] option, e.g.:
%     \usepackage[final]{neurips_2025}


% to avoid loading the natbib package, add option nonatbib:
%    \usepackage[nonatbib]{neurips_2025}
% Optional math commands from https://github.com/goodfeli/dlbook_notation.
\input{math_commands.tex}

\usepackage{algorithm}
\usepackage{algpseudocode}

\usepackage{hyperref}       % hyperlinks
\usepackage{url}            % simple URL typesetting
\usepackage{booktabs}       % professional-quality tables
\usepackage{amsfonts}       % blackboard math symbols
\usepackage{nicefrac}       % compact symbols for 1/2, etc.
\usepackage{microtype}      % microtypography
\usepackage{xcolor}         % colors
\usepackage{amsmath}
% \usepackage{amssymb}
\usepackage{subcaption}
\usepackage{graphicx}
\usepackage{comment}
\usepackage{multirow}
\usepackage{enumitem}
\usepackage{float}

% for figures 
\usepackage{amsmath}
\usepackage{amsfonts}
\usepackage{amssymb}
\usepackage{tikz}
\usetikzlibrary{shapes, arrows, positioning}
% \usepackage{soul}
\usepackage{xcolor}
\usepackage[utf8]{inputenc} % allow utf-8 input
\usepackage[T1]{fontenc}    % use 8-bit T1 fonts
\usepackage{hyperref}       % hyperlinks
\usepackage{url}            % simple URL typesetting
\usepackage{booktabs}       % professional-quality tables
\usepackage{amsfonts}       % blackboard math symbols
\usepackage{nicefrac}       % compact symbols for 1/2, etc.
\usepackage{microtype}      % microtypography
\usepackage{xcolor}         % colors
\usepackage{amsmath}
% \usepackage{amssymb}
\usepackage{subcaption}
\usepackage{graphicx}
\usepackage{comment}
\usepackage{multirow}
\usepackage{enumitem}
\usepackage{float}

% for figures 
\usepackage{amsmath}
\usepackage{amsfonts}
\usepackage{amssymb}
\usepackage{tikz}
\usetikzlibrary{shapes, arrows, positioning}
% \usepackage{soul}
\usepackage{xcolor}
\usepackage[colorinlistoftodos]{todonotes}


\title{MoGU: Mixture-of-Gaussians with Uncertainty-based Gating for Time Series Forecasting}

% Authors must not appear in the submitted version. They should be hidden
% as long as the \iclrfinalcopy macro remains commented out below.
% Non-anonymous submissions will be rejected without review.

%\author{Yoli Shavit \qquad \qquad  Jacob Goldberger \\
%Faculty of Engineering, Bar Ilan University \\ Ramat-Gan, Israel\\
%{\tt\small \{yoli.shavit,jacob.goldberger\}@biu.ac.il}}

\author{Yoli Shavit  \& Jacob Goldberger\\
Faculty of Engineering\\
Bar Ilan University\\
Ramat-Gan, Israel \\
\texttt{\{yoli.shavit,jacob.goldberger\}@biu.ac.il} 
}
% The \author macro works with any number of authors. There are two commands
% used to separate the names and addresses of multiple authors: \And and \AND.
%
% Using \And between authors leaves it to \LaTeX{} to determine where to break
% the lines. Using \AND forces a linebreak at that point. So, if \LaTeX{}
% puts 3 of 4 authors names on the first line, and the last on the second
% line, try using \AND instead of \And before the third author name.

\newcommand{\fix}{\marginpar{FIX}}
\newcommand{\new}{\marginpar{NEW}}

%\iclrfinalcopy % Uncomment for camera-ready version, but NOT for submission.
\begin{document}


\maketitle

\begin{abstract}
We introduce Mixture-of-Gaussians with Uncertainty-based Gating (MoGU), a novel Mixture-of-Experts (MoE) framework designed for regression tasks and applied to time series forecasting. Unlike conventional MoEs that provide only point estimates, MoGU models each expert's output as a Gaussian distribution. This allows it to directly quantify both the forecast (the mean) and its inherent uncertainty (variance). MoGU's core innovation is its uncertainty-based gating mechanism, which replaces the traditional input-based gating network by using each expert's estimated variance to determine its contribution to the final prediction. Evaluated across diverse time series forecasting benchmarks, MoGU consistently outperforms single-expert models and traditional MoE setups. It also provides well-quantified, informative uncertainties that directly correlate with prediction errors, enhancing forecast reliability. Our code is available from: \url{https://github.com/yolish/moe_unc_tsf}.
    %Mixture-of-Experts (MoE) is an architectural paradigm that adaptively combines predictions from multiple neural modules, typically sharing a common architecture, via a learned gating mechanism. Whether experts predict in the target or latent domain, they conventionally output point estimates, fundamentally lacking direct quantification of their inherent uncertainty. This limitation extends to the overall MoE architecture, hindering interpretability and informed decision-making in critical applications. In this work, we propose Mixture-of-Gaussians with Uncertainty-based Gating (MoGU): a novel framework for MoE architectures, which directly quantifies uncertainty for both individual experts and the aggregated MoE model, and leverages expert uncertainty to inform the gating mechanism. Focusing on regression tasks, we model each expert's prediction as a random variable drawn from a normal distribution. Here, the expert not only predicts the mean (the predicted label) but also estimates its own variance (uncertainty). Instead of relying on a separate gating function, as in deterministic MoE architectures, our method directly derives each expert's relative contribution by leveraging its inherent uncertainty. We evaluate our approach using time series forecasting as the primary domain, given its diverse applications and benchmarks. Our method  outperforms the MoE setting across various forecasting benchmarks and expert architectures, while providing informative and well-quantified uncertainties, which correlate with prediction errors. Our code is available from: \url{https://anonymous.4open.science/r/moe_unc_tsf-65E1}
\end{abstract}


\section{Introduction}\label{sec:introduction}
Mixture-of-Experts (MoE) is an architectural paradigm that adaptively combines predictions from multiple neural modules, known as "experts," via a learned gating mechanism. This concept has evolved from ensemble-based MoEs, where experts, jointly trained with a gating function, are often full, independent models whose outputs are combined to improve overall performance and robustness \citep{jacobs1991adaptive}. More recently, MoE layers have been integrated within larger neural architectures, with experts operating in a latent domain. These "latent MoEs" offer significant scalability benefits, especially in large language models (LLMs) \citep{shazeer2017outrageously,fedus2022switch}.
MoE makes it possible to train massive but efficient LLMs, where each token activates only a fraction of the model’s parameters, enabling specialization, better performance, and lower computational cost compared to equally sized dense models.

Regardless of their specific implementation, conventional MoE systems typically produce point estimates, lacking a direct quantification of their uncertainty. In critical applications, this absence of uncertainty information hinders interpretability, making it difficult for users to gauge the reliability of a prediction and limits informed decision-making, as the system cannot express its confidence or identify ambiguous cases. Importantly, the learned gating mechanism, which dictates the relative contribution of each expert, does not take into account expert confidence, potentially leading to suboptimal routing decisions.

In this work, we propose Mixture-of-Gaussians with Uncertainty-based Gating (MoGU), a framework for uncertainty-aware MoE architectures, which provides explicit uncertainty quantification for both individual experts and the overall MoE model. Our approach fundamentally reimagines the expert's output: instead of a point estimate, we model each expert's prediction as a random variable drawn from a normal distribution. In this setup, each expert simultaneously predicts both the mean (the label estimate) and variance of the distribution, representing its predictive uncertainty. This shift enables a more nuanced understanding of expert behavior and the derivation of the overall model's uncertainty. Furthermore, we introduce a novel gating mechanism where the estimated uncertainty of each expert directly informs its relative contribution to the overall MoE prediction, bypassing the need for a separate gating function typically found in traditional MoE setups. This creates a self-aware MoE where more confident experts naturally exert greater influence.

We evaluate MoGU on time series forecasting as our primary regression task. This choice is motivated by the inherent uncertainty in real-world time series data and the wide variety of expert architectures applicable to forecasting tasks across numerous domains \citep{time_series_survey, wang2024deep}. Our evaluation spans various expert types, forecasting benchmarks and forecasting horizon sizes, allowing for a comprehensive assessment of our method's efficacy. MoGU is shown to consistently yield more accurate forecasts compared to input-based gating MoE architectures, while simultaneously, providing uncertainty estimates that are positively correlated with prediction error. These estimates are available at both the individual expert and overall model levels. By further distinguishing between aleatoric (data-related) and epistemic (model-related) uncertainty, MoGU offers valuable insights into the source of a model's uncertainty. We also conducted a detailed ablation study to validate our key design choices.

In summary, our contributions are as follows: 
\begin{itemize}
\item \textbf{MoGU: A Novel Framework for Uncertainty-Aware MoE Architectures}: We introduce a novel framework that directly quantifies uncertainty for both individual experts and the overall model, moving beyond conventional point estimates. A key innovation is a routing mechanism that uses each expert’s estimated predictive uncertainty to dynamically determine its contribution to the final MoE output, replacing traditional input-based gating mechanisms.
\item \textbf{MoGU Improves Time Series Forecasting}: Our method effectively reduces forecasting error across various benchmarks, horizon lengths, and expert architectures.
\item \textbf{MoGU Provides Meaningful Uncertainty Estimates for Time Series Forecasting}: MoGU generates uncertainty estimates at the expert-level and overall. These estimates are positively correlated with prediction error, providing valuable insight into the model's confidence and the sources of its uncertainty.
\end{itemize}

By embedding uncertainty estimation into prediction and gating, MoGU moves beyond input-based gating  MoEs toward architectures that are more accurate, transparent, and reliable.


\section{Related Work}
\label{sec:related_work}


\textbf{Block sparse attention.} 
Several implementations of block sparse attention~\cite{bsa, flashattn, flexattn, flashinfer, flashmask} propose a coarse-grained sparse attention mechanism that skips entire blocks of attention score computations at granularity of $64\times 64$ or $128\times 128$ at half-precision. Current block-sparse attention mechanisms either prevent further reduction of block size (do not compile) or cause significant hardware underutilization and performance overhead, since they are constrained by the tensor core matrix multiplication width (\cref{sec:motivation_skip_attn_fine_granularity}). Several works in the large language model literature~\cite{minference, xattn, flashdecode, nsa, seerattn} utilize block sparse attention to accelerate attention computation. 
% Several works in large language model literature ~\cite{minference, xattn, flashdecode, nsa, seerattn} use block sparse attention to speedup attention computation.

\textbf{Block sparse attention for videoDiTs.} 
Recent works, such as Radial Attention~\cite{radialattn}, X-attention~\cite{xattn}, SparseVideoGen~\cite{sparsevideogen}, and SparseVideoGen2~\cite{sparsevideogen2}, have applied block sparse attention implementations to video diffusion models. These approaches consider a fixed sparsity pattern in the attention map based on empirical observations of significant patterns. Other works, such as Video Sparse Attention~\cite{vsa}, incorporate learned sparse attention patterns by using a parameterized model to derive the attention map mask. Both approaches utilize coarse-grained sparse attention mechanisms. In contrast, our method enables fine-grained skipping of attention blocks, providing more opportunities for skipping computation. We compare \X with SparseVideoGen and Radial Attention in~\cref{sec:results}. 
Moreover, trainable sparse attention methods such as Video Sparse Attention (VSA)~\cite{vsa} can be reformulated to generate sparse masks compatible with \X \textquotesingle s attention kernel. These methods are orthogonal to \X \textquotesingle s kernel implementation and can be used in conjunction as mask-determination strategies for \X.

% While trainable sparse attention methods can outperform training-free approaches, these methods could benefit from \X's fine-grained approach.
% Recent works such as Radial attention~\cite{radialattn}, X-attention~\cite{xattn}, SparseVideoGen~\cite{sparsevideogen}, SparseVideoGen2~\cite{sparsevideogen2} apply block sparse attention implementations for video diffusion models. They consider a fixed sparsity pattern in the attention map based on empirical observation of significant. Other works such as video sparse attention~\cite{vsa} incorporate learnt sparse attention patterns. These works use a parameterized model to derive the mask of the attention maps. Both of these works make use of coarse grain sparse attention mechanisms. Our approach allows fine-grain skipping of attention blocks enabling more opportunity to skip computation. We compare \X with SVG, radial attention in~\cref{sec:results}. Trainable sparse attention methods can make  of coarse grain sparse attention methods, and could potentially benefit from \X. 

\textbf{Other techniques to accelerate video diffusion.}
SpargeAttention~\cite{spargeattn}, SageAttention~\cite{sageattn}, and SageAttention2~\cite{sageattn2} propose general attention approximation techniques, such as quantization and token compression mechanisms, that can be applied during inference for both LLM and DiT models. Token compression-based approaches may skip essential tokens relevant to the video, which could lead to inconsistent video generation (pointed out by~\cite{radialattn}). These approaches are orthogonal to our \X.% SpargeAttention~\cite{spargeattn}, SageAttention~\cite{sageattn}, SageAttention2~\cite{sageattn2}, propose general attention approximation techniques such as quantization and token compression mechanisms that can be applied at inference time to LLM and DiT models. Approaches based on token compression skip essential tokens relevant to the video and may produce inconsistent video (\tofix{add reference here}). Quantization-based techniques are orthogonal to our approach (a lower-precision version of \X can be implemented). 





\section{LLM-based Multi-Agent Blackboard System}

This section introduces an alternative communication paradigm for LLM-based multi-agent systems inspired by blackboard systems \citep{10.1145/356810.356816}, distinct from the widely used master–slave architecture. As outlined in \textsection \ref{sec:introduction}, blackboard-based multi-agent systems provide several advantages over the master-slave approach. Here, rather than directly assigning tasks to sub-agents, the main agent posts its requests (i.e., sub-tasks for which it requires assistance) on a shared blackboard, which functions as a broadcast channel accessible to all other agents. Each helper agent independently evaluates whether it can respond to a request, considering its own capabilities, availability, cost, and other factors. If an agent decides to contribute, it writes its response to the corresponding request, and the main agent then decides whether to use or ignore the provided information. \textit{This way, all agents in the system retain full autonomy over their actions, and no centralized controller forces them to execute a specific task.} While the blackboard paradigm is applicable to a wide range of multi-agent systems, we focus on data science tasks that require data discovery, where its characteristics are particularly advantageous, as discussed in \textsection \ref{sec:introduction}. The remainder of this section details our method and its design for data science problems that require information discovery.



\paragraph{Overview:} 

An overview of our proposed method is presented in Figure~\ref{fig:overview}. The system $\pi_{s}$ operates over the data lake $\sD$ by first partitioning $\sD$ into $C$ clusters of related files. Each cluster $\sD_i$ is assigned to a file agent $\pi_{f_i}$, which is responsible for handling, loading, processing, and retrieving information from the files within its cluster. In addition, a search agent $\pi_{s}$ is included to retrieve external information from the web that may be required to solve the problem. The overall system $\pi_{s}$ is composed of a main agent $\pi_{m}$, which is responsible for solving the query $q$, and a set of $C+1$ helper agents $\Pi_{\text{helper}} = \{\pi_{f_i}\}_{i=1}^{M} \cup \{\pi_{s}\}$ that provide specialized assistance. The query $q$ is presented to $\pi_{m}$, which iteratively selects an action $a \in \sA$ from the action space $\sA$, executes the chosen action, and observes the resulting outcome from the environment. Among its actions, the main agent may interact with a blackboard $\beta$, a shared communication medium where it can post a request $r$ without addressing a specific sub-agent. The helper agents $\Pi_{\text{helper}}$ continuously monitor the blackboard, determine whether they can address a posted request, and, if so, provide their outputs on the corresponding response board $\beta_{r}$. These responses are then collected and made available to $\pi_{m}$, which incorporates them into its decision-making process.\footnote{Responses are not written back to the blackboard $\beta$ to avoid dependencies where one sub-agent's output could influence the behavior of others negatively. Instead, all responses are directed exclusively to the response board $\beta_{r}$, ensuring independent operation of sub-agents and exclusive access by the main agent $\pi_{m}$.} The main agent is limited to at most $T$ sequential actions (including actions that interact with the blackboard) to solve the query $q$, ultimately producing a program $p$ in python programming language that computes the final answer to $q$.

\paragraph{Clustering Data Lake:} 

There are multiple approaches for partitioning the data lake into clusters; applying clustering algorithms over file representations, random partitioning, or other heuristic methods. For simplicity, we do not utilize file content and instead rely solely on file names during clustering. Specifically, the file names are provided to an LLM---Gemini-2.5-Pro\footnote{Available at: \url{https://cloud.google.com/vertex-ai/generative-ai/docs/models/gemini/2-5-pro}}---which using the prompt shown in Figure~\ref{fig:clustering-prompt}, clusters the files into categories based only on their names.\footnote{This method represents just one simple possible approach to clustering, chosen for simplicity; more scalable and accurate alternatives could equally be employed in real world scenarios.} An example of this clustering is provided in Figure~\ref{fig:clustering-example} in Appendix~\ref{app:case-study}, where the model successfully groups related files together. For instance, it clusters all files originating from the National Interagency Fire Center into a category labeled ``NIFC Wildfire Statistics.'' The number of automatically derived clusters for each dataset is reported in Table~\ref{tab:stats} in Appendix~\ref{app:dataset}.



% The remainder of this section details the design of the main agent and the helper agents, emphasizing how their coordination supports effective information discovery in data science tasks.

\subsection{Main Agent}
\label{sec:main-agent}

The primary role of the main agent is to solve the problem in collaboration with the helper agents. The main agent follows the ReAct framework \citep{yao2023react}, where at each step $t$, given the query $q$ and the history of actions and observations $\sH_{t-1}$, it first reasons about what is the best next action and selects an action from a predefined action space, executes the action, observes the outcome, and appends the resulting observation to update the history $\sH_{t}$.\footnote{In this work, the inputs, outputs of the model, and observations are appended directly to the prompt of the LLM, formatted according to its chat-based input template.} The prompt used by the main agent is shown in Figure~\ref{fig:main-agent-blackboard-prompt} in Appendix~\ref{app:prompts}. The agent selects one of the following predefined actions in each step, executes them, and observe their outcomes:

\begin{itemize}[leftmargin=*]
    \item \textit{\textbf{Planning:}} In this action, the LLM decomposes the problem into smaller sub-problems and outlines a plan for addressing each of them. This action has no external effect on the environment but serves as an internal reasoning step to guide the LLM's problem-solving process. In response, the system simply acknowledges the proposed plan and instructs the LLM to proceed.
    
    \item \textit{\textbf{Reasoning:}} In this action, the LLM focuses on a specific aspect of the problem and explains its reasoning, analysis, or interpretation of the available observations and steps taken so far in this process. Similar to the planning step, this action has no external effect on the environment but functions as an internal reasoning mechanism to guide the LLM's problem-solving process. In response, the system simply acknowledges the reasoning and prompts the LLM to continue.
    
    \item \textit{\textbf{Executing Code:}} In this action, the agent generates python code, which is executed using a python interpreter. If the code runs successfully, the resulting outputs are returned to the agent for observation; otherwise, the agent receives the corresponding error messages. This action enables the agent to explore the problem interactively, inspect data files, and experiment with them to gain a deeper understanding of their content and structure and how to process them.
    
    \item \textit{\textbf{Requesting Help:}} In this action, the agent formulates a request for assistance from the sub-agents, specifying, for example, the types of data files or information needed, or the resources required to apply a tool or solve a sub-problem. This request is posted on the blackboard $\beta$ for visibility by the helper agents. Once the sub-agents respond, if they respond, their responses on the response board $\beta_r$ are collected and provided back to the main agent as the outcome of this action for observation and further use in its decision-making process.
    
    \item \textit{\textbf{Answering:}} In this action, the agent concludes the problem-solving process by generating a final program that produces the answer to the query. This action terminates the process, and the output of this step constitutes the final program $p$ generated by the system to address the problem.
\end{itemize}

\subsection{Helper Agents}
\label{sec:sub-agents}

In a data science, information discovery can typically be categorized into two tasks: (1) identifying the specific files that contain the data necessary to the problem, and (2) retrieving general knowledge about concepts relevant to the problem, such as domain-specific terms or details of particular algorithms and methods. To support these, our framework employs two types of helper agents:

\paragraph{File Agent:} 

Handling all the files in a data lake with a single agent is not feasible for several reasons: it typically involve a large number of files, many of which are lengthy and may exceed the agents context window; the files span diverse topics, which can confuse the agent and hinder effective reasoning; and accessing and processing all files simultaneously can be computationally expensive and inefficient, leading to unnecessary overhead and slower problem-solving. For these reasons, in our framework each file agent is assigned responsibility for a subset of data files determined to be relevant, as described earlier in the clustering procedure. In an offline phase, the file agent $\pi_{f_i}$ takes as input a subset of the data lake $\sD_{i}$ and operates through a two-step procedure. In the first step, the agent selects a subset\footnote{When filenames indicate multiple files containing the same type of data over different time periods, the agent does not need to inspect all of them to infer the structure; a small representative sample is sufficient.} (or all) of the files to examine their content. The contents of them are presented to the agent for inspection (details of presentation are in Appendix~\ref{app:implementation}). In the second step, after observing the selected files, the agent reasons about and analyzes them, learning how they are structured, what pre-processing or transformations may be required, and how they should be processed in general. An example of such an analysis is provided in Figure~\ref{fig:file-agent-analyze-example} in Appendix~\ref{app:case-study}. Then, in the online phase, the agent listens for requests from the main agent. Upon receiving a request, based on the analysis it did earlier, it determines whether it can contribute to answering it. If so, the agent generates a detailed plan specifying which files in $\sD_i$ are relevant, how they should be loaded in Python code, what libraries to use, the steps required for data processing, and samples from the data. The prompt used to guide the file agent is shown in Figure~\ref{fig:file-agent-prompt} in Appendix~\ref{app:prompts}. 

\paragraph{Search Agent:}

Certain data science problems require task-specific knowledge about algorithms or domain expertise that the LLM may not possess. To address this, we design a web-search agent that retrieves relevant information from a search engine. This agent operates according to the prompt shown in Figure~\ref{fig:search-agent-prompt} in Appendix~\ref{app:prompts}. Given a request $r$ posted on the blackboard $\beta$, the agent first determines whether it is capable of addressing the request. It is specifically restricted to general web-based information retrieval and does not respond to requests involving access to local files or datasets. If the agent determines that the request can be answered, it enters an iterative search process with a maximum of $T_{\text{search}} = 3$ steps. At each step $t$, the agent generates a set of queries $\sQ_{t}$, which are submitted to a search engine---in this work, Google Custom Search Engine\footnote{We use Google Custom Search Engine, configured to exclude all websites associated with the datasets used in this paper to prevent data leakage: \url{https://developers.google.com/custom-search}}---to retrieve $k=3$ webpage per query. The content of the webpages are then extracted using \textit{beautifulsoup} library\footnote{Available at: \url{https://pypi.org/project/beautifulsoup4/}} to be presented to the search agent. The extracted documents are then evaluated by the agent to determine whether they provide sufficient information to answer the request. If so, the agent generates a response to the request, which is posted to the response board $\beta_r$. If the information is insufficient, a new set of queries is generated to continue gathering relevant data from the web.

\section{Experiments}\label{sec:experiments}
We evaluate MoGU on several multivariate time series forecasting benchmarks. We compare its performance to the standard MoE, which lacks uncertainty estimation, and to a single-expert model. Our evaluation varies the number of experts, prediction horizon length, and expert architecture. The complete experimental setup is detailed in Section \ref{sec:experimental_setup}.
The results of our evaluation are presented in Section \ref{subsec:tsf_results}. MoGU achieves competitive performance, consistently outperforming both the standard MoE and the single-expert models. We further analyze the reported uncertainty by our method in Section \ref{subsec:unc_analysis}. We find that the uncertainty estimates reported by MoGU are informative, positively correlated with prediction error, and accurately reflect the error trend.
Finally, in Section \ref{subsec:ablations}, we present an ablation study that explores alternative design choices for our gating mechanism, loss, uncertainty head architecture, and the resolution at which uncertainty is reported. The results further validate the advantage of our proposed novel uncertainty-based gating and demonstrate the robustness of our framework.
\subsection{Experimental setup}
\label{sec:experimental_setup}
\textbf{Datasets.}
We evaluate our method on eight widely used time series forecasting datasets~\citep{autoformer}: four Electricity Transformer Temperature (ETT) datasets (ETTh1, ETTh2, ETTm1, ETTm2)~\citep{informer}, as well as Electricity\footnote{
https://archive.ics.uci.edu/ml/datasets/ElectricityLoadDiagrams20112014}, Weather\footnote{
https://www.bgc-jena.mpg.de/wetter/}, Exchange~
\citep{lai2018modeling}, and Illness (ILI)\footnote{
https://gis.cdc.gov/grasp/fluview/fluportaldashboard.html}.
%\jacob{do we need to add references to the datasets? YS: added}

\textbf{Experimental Protocol.} Our experiments follow the standard protocol used in recent time series forecasting literature \citep{Yuqietal-2023-PatchTST,liu2023itransformer, wang2024deep}. For the ILI dataset, we use a forecast horizon length  $h \in \{24, 36, 48, 60\}$. For all other datasets, the forecast horizon length is selected from {96,192,336,720}. A look-back window of 96 is used for all experiments. We report performance using the Mean Absolute Error (MAE) and Mean Squared Error (MSE). We evaluate the quality of our uncertainty estimates by computing the Pearson and Spearman correlation with respect to the prediction error. Specifically, for each individual variable, we correlate the model's reported uncertainty values with the corresponding MAE across all time points. We then average these correlation coefficients to get an overall measure. 

\textbf{Expert Architecture.} MoGU is a general MoE framework compatible with various expert architectures. We evaluate it using three state-of-the-art expert models: iTransformer \citep{liu2023itransformer}, PatchTST \citep{Yuqietal-2023-PatchTST}, and DLinear \citep{dlinear}. These models represent different architectural approaches, including Transformer and MLP-based designs. 

{\bf Implementation and Training Details.}
We implemented MoGU in PyTorch~\citep{paszke2019pytorch}. For the expert architecture, we extended the existing implementations of PatchTST, iTransformer, and DLinear available from the Time Series Library (TSLib)~\citep{wang2024deep}, to incorporate uncertainty estimation as detailed in Section \ref{subsec:tsf_mogu}. For training, we used a configuration similar to the one provided by TSLib. All models were trained for a maximum of 10 epochs with a patience of 3 epochs for early stopping. We used the Adam optimizer with a batch size of 8. The learning rate was set to $\lambda=0.001$ for the Weather and Electricity datasets and $\lambda=0.0001$ for all other datasets. All experiments were conducted on a single NVIDIA A100 80GB GPU.
\linespread{1.1}
\begin{table*}[h]
\small
%\caption{Multivariate forecasting results when using a single expert (standard forecasting setup) and when using MoE  and MoG  with Uncertainty-based gating (MoGU, ours), when varying on the number of experts. We use iTransformer as the expert architecture and report the MSE for each configuration, when forecasting a horizon of 96 time points. The best MAE and MSE results for each dataset are shown in bold.}	
\caption{Multivariate forecasting results when using a single expert (standard forecasting setup) and when using MoE  and MoG  with Uncertainty-based gating (MoGU, ours), when varying on the number of experts. The best MAE and MSE results for for a 96-time point horizon are shown in bold for each dataset.}
\label{tab:main_results_num_experts}
	\centering
		\begin{tabular}{cc|c|c|cccc|cccc}
    \cline{2-12}
    & \multicolumn{2}{c|}{Configuration}& Single& \multicolumn{4}{c|}{MoE}& \multicolumn{4}{c}{MoGU (ours)} \\
     & \multicolumn{2}{c|}{}& Expert & \multicolumn{4}{c|}{}& \multicolumn{4}{c}{} \\
    \cline{2-12}
    & \multicolumn{2}{c|}{Num. Experts}&1&2&3&4&5&2&3&4&5\\
    \cline{2-12} 
     & \multicolumn{2}{c|}{ETTh1} & 0.398 & 0.391 & 0.393 & 0.398 & 0.392 & 0.385 & 0.380 & 0.382 & \textbf{0.381} \\
 & \multicolumn{2}{c|}{ETTh2} & 0.295 & 0.307 & 0.299 & 0.305 & 0.311 & 0.284 & \textbf{0.283} & 0.286 & 0.286 \\
 & \multicolumn{2}{c|}{ETTm1}& 0.341 & 0.349 & 0.332 & 0.347 & 0.339 & 0.320 & 0.320 & 0.314 & \textbf{0.312} \\
 & \multicolumn{2}{c|}{ETTm2} & 0.188 & 0.186 & 0.179 & 0.180 & 0.177 & 0.179 & 0.179 & 0.176 & \textbf{0.175} \\
 \cline{2-12} 
\end{tabular}
\end{table*}
\linespread{1.00}
\linespread{1.1}
\begin{table*}[h!]
\small
\caption{Multivariate long-term forecasting results with MoE and MoGU (ours), with iTransformer and PatchTST as the expert architectures. We report the MAE and MSE for each configuration, using prediction lengths $L\in \{24, 36, 48, 60\}$ for the ILI dataset and $L\in \{96, 192, 336, 720\}$ for the others. The best MSE results for each configuration are shown in bold.}
	\label{tab:main_results}
	\centering
		\begin{tabular}{cc|c||cc|cc||cc|cc}
    \cline{2-11}
    & \multicolumn{2}{c||}{Expert}& \multicolumn{4}{c||}{iTransformer} & \multicolumn{4}{c}{PatchTST}\\
    \cline{2-11}
    & \multicolumn{2}{c||}{Mixture Type}& \multicolumn{2}{c|}{MoE} & \multicolumn{2}{c||}{MoGU (ours)} & \multicolumn{2}{c|}{MoE} & \multicolumn{2}{c}{MoGU (ours)}\\
    \cline{2-11}
    & \multicolumn{2}{c||}{Metric}& MAE & MSE & MAE & MSE & MAE & MSE & MAE & MSE\\
    \cline{2-11}   
&\multirow{4}*{\rotatebox{90}{ETTh1}} & 96 & 0.410 & 0.393 & \textbf{0.400} & \textbf{0.380} & \textbf{0.406} & \textbf{0.386} & 0.415 & 0.409 \\
&\multicolumn{1}{c|}{} & 192 & 0.432 & 0.437 & \textbf{0.431} & \textbf{0.436} & 0.448 & 0.459 & \textbf{0.443} & \textbf{0.453} \\
&\multicolumn{1}{c|}{} & 336 & 0.472 & 0.504 & \textbf{0.454} &\textbf{ 0.479} & 0.465 & 0.485 &\textbf{ 0.459} & \textbf{0.484} \\
&\multicolumn{1}{c|}{} & 720 & \textbf{0.489} & \textbf{0.500} & 0.491 & 0.501 & 0.494 & 0.510 & \textbf{0.483} & \textbf{0.485} \\
\cline{2-11}
&\multirow{4}*{\rotatebox{90}{ETTh2}} & 96 & 0.348 & 0.299 & \textbf{0.336} & \textbf{0.283} & 0.347 & 0.298 & \textbf{0.331} & \textbf{0.277} \\
&\multicolumn{1}{c|}{} & 192 & 0.396 & 0.377 & \textbf{0.387} &\textbf{ 0.361} & 0.400 & 0.375 & \textbf{0.386} & \textbf{0.357} \\
&\multicolumn{1}{c|}{} & 336 & 0.427 & \textbf{0.413} & \textbf{0.425} & 0.415 & 0.440 & 0.422 &\textbf{ 0.423} & \textbf{0.406} \\
&\multicolumn{1}{c|}{} & 720 & 0.447 & 0.435 & \textbf{0.442 }&\textbf{ 0.421 }& 0.460 & 0.443 & \textbf{0.447} &\textbf{ 0.426} \\
\cline{2-11}
&\multirow{4}*{\rotatebox{90}{ETTm1}} & 96 & 0.367 & 0.332 & \textbf{0.356} & \textbf{0.320} & 0.371 & 0.337 & \textbf{0.362} & \textbf{0.326} \\
&\multicolumn{1}{c|}{} & 192 & 0.396 & 0.382 & \textbf{0.379} & \textbf{0.363} & 0.398 & \textbf{0.380} & \textbf{0.393} & 0.389 \\
&\multicolumn{1}{c|}{} & 336 & 0.411 & 0.407 & \textbf{0.404} & \textbf{0.400} & \textbf{0.407} & \textbf{0.400} & \textbf{0.407} & \textbf{0.400} \\
&\multicolumn{1}{c|}{} & 720 & 0.460 & 0.500 & \textbf{0.438} & \textbf{0.466} & 0.448 & 0.465 & \textbf{0.442 }& \textbf{0.460 }\\
\cline{2-11}
&\multirow{4}*{\rotatebox{90}{ETTm2}} & 96 & 0.261 &\textbf{ 0.179} & \textbf{0.260} & \textbf{0.179} & 0.264 & 0.177 & \textbf{0.259} & \textbf{0.175} \\
&\multicolumn{1}{c|}{} & 192 & 0.306 & 0.246 & \textbf{0.302} &\textbf{ 0.245} & 0.308 & 0.247 & \textbf{0.303} & \textbf{0.242} \\
&\multicolumn{1}{c|}{} & 336 & 0.345 & 0.307 &\textbf{ 0.339} & \textbf{0.301} &\textbf{ 0.346} & \textbf{0.304} & \textbf{0.346} & 0.307 \\
&\multicolumn{1}{c|}{} & 720 & 0.401 & 0.403 & \textbf{0.395} &\textbf{ 0.397} & 0.405 & 0.408 & \textbf{0.403 }& \textbf{0.405} \\
\cline{2-11}
&\multirow{4}*{\rotatebox{90}{ILI}} & 24 & 0.864 & 1.786 & \textbf{0.827} & \textbf{1.756} & 0.866 & 1.871 & \textbf{0.822} & \textbf{1.848} \\
&\multicolumn{1}{c|}{} & 36 & 0.882 & 1.746 & \textbf{0.825} & \textbf{1.629} & 0.875 & 1.875 & \textbf{0.835} & \textbf{1.801} \\
&\multicolumn{1}{c|}{} & 48 & 0.948 & 1.912 & \textbf{0.843 }& \textbf{1.634} & 0.878 & \textbf{1.798} & \textbf{0.844} & 1.818 \\
&\multicolumn{1}{c|}{} & 60 & 0.979 & 1.986 & \textbf{0.881} & \textbf{1.692} & 0.904 & 1.864 & \textbf{0.864} & \textbf{1.831 }\\
\cline{2-11}
&\multirow{4}*{\rotatebox{90}{Weather}} & 96 & 0.253 & 0.208 & \textbf{0.249} & \textbf{0.207} & 0.237 & 0.196 & \textbf{0.230} & \textbf{0.188} \\
&\multicolumn{1}{c|}{} & 192 &\textbf{ 0.283} & \textbf{0.246} & \textbf{0.283} & 0.251 & 0.268 & 0.235 & \textbf{0.265} & \textbf{0.232} \\
&\multicolumn{1}{c|}{} & 336 & \textbf{0.315} & \textbf{0.296} & 0.317 & 0.300 & 0.308 & 0.291 & \textbf{0.303} & \textbf{0.287} \\
&\multicolumn{1}{c|}{} & 720 & \textbf{0.361} & \textbf{0.369} & \textbf{0.361} & 0.371 & 0.353 & 0.363 & \textbf{0.351} & \textbf{0.361} \\
\cline{2-11}
&\multirow{4}*{\rotatebox{90}{Electricity}} & 96 & \textbf{0.235} & \textbf{0.144} & 0.238 & 0.148 & \textbf{0.248} & \textbf{0.161 }& 0.257 & 0.169 \\
&\multicolumn{1}{c|}{} & 192 & 0.254 &\textbf{ 0.162} & \textbf{0.251} & 0.163 &\textbf{ 0.258} &\textbf{ 0.170} & 0.263 & 0.179 \\
&\multicolumn{1}{c|}{} & 336 & \textbf{0.269 }& \textbf{0.175} & \textbf{0.269} & 0.179 & \textbf{0.276} & \textbf{0.188} & 0.286 & 0.200 \\
&\multicolumn{1}{c|}{} & 720 & \textbf{0.297} & \textbf{0.204} & 0.302 & 0.216 & \textbf{0.314} &\textbf{ 0.231 }& 0.319 & 0.242 \\
\cline{2-11}
& \multicolumn{2}{c||}{Num. Wins }  & 4 & 9 &\textbf{ 21 }&\textbf{ 18} & 5 & 8 & \textbf{21} & \textbf{19} \\
\cline{2-11}
\end{tabular}
\linespread{1.00}
\linespread{1.1}
\end{table*}
\vspace{-0.3in}
\linespread{1.1}
\begin{table*}[h!]
\small
\caption{Multivariate forecasting results for MoE and MoGU (ours), using DLinear, iTransformer, and PatchTST as expert architectures. The best MAE and MSE results for each configuration are shown in bold for a 96-time point horizon.}
\label{tab:main_results_exchange}
    \small
	\centering
       \scalebox{0.8}{ 
		\begin{tabular}{cc||cc|cc||cc|cc||cc|cc}
    \cline{2-14}
    \multicolumn{2}{c||}{Expert}& \multicolumn{4}{c||}{DLinear} & \multicolumn{4}{c||}{iTransformer} & \multicolumn{4}{c}{PatchTST} \\
    \cline{2-14}
    \multicolumn{2}{c||}{Mixture Type}& \multicolumn{2}{c|}{MoE} & \multicolumn{2}{c||}{MoGU (ours)}& \multicolumn{2}{c|}{MoE} & \multicolumn{2}{c||}{MoGU (ours)}& \multicolumn{2}{c|}{MoE} & \multicolumn{2}{c}{MoGU (ours)} \\
    \cline{2-14}
    \multicolumn{2}{c||}{Metric}& MAE & MSE & MAE & MSE & MAE & MSE & MAE & MSE & MAE & MSE & MAE & MSE \\
    \cline{2-14}
\multicolumn{2}{c||}{Exchange} & 0.213 & 0.086 & \textbf{0.209} & \textbf{0.080}  & 0.225 & 0.010 & \textbf{0.208} & \textbf{0.089} & \textbf{0.201} & 0.086 & {0.202} & \textbf{0.084}  \\            \cline{2-14}
\multicolumn{2}{c||}{ETTh1} & \textbf{0.400} & \textbf{0.382} & \textbf{0.400} & \textbf{0.382} & 0.410 & 0.393 & \textbf{0.400} & \textbf{0.380} & \textbf{0.406} & \textbf{0.386} & 0.415 & 0.409 \\
\cline{2-14}
\multicolumn{2}{c||}{ETTh2} & 0.373 & 0.320 & \textbf{0.366} & \textbf{0.308} & 0.348 & 0.299 & \textbf{0.336} & \textbf{0.283} & 0.347 & 0.298 & \textbf{0.331} & \textbf{0.277} \\
\cline{2-14}
\multicolumn{2}{c||}{ETTm1} & \textbf{0.360} & \textbf{0.322} & 0.363 & 0.338 & 0.367 & 0.332 & \textbf{0.356} & \textbf{0.320} & 0.371 & 0.337 & \textbf{0.362} & \textbf{0.326} \\
\cline{2-14}
\multicolumn{2}{c||}{ETTm2} & 0.285 & 0.189 &\textbf{0.271} & \textbf{0.183} & 0.261 &\textbf{ 0.179} & \textbf{0.260} & \textbf{0.179} & 0.264 & 0.177 & \textbf{0.259} & \textbf{0.175} \\
\cline{2-14}
\end{tabular}
    }
\end{table*}
\linespread{1.00}
\subsection{Results}\label{sec:results}
%main results
%\jacob{ I didnt find a comparison between MoG (with softmax weights an MOGU - Table 5 in the ablations}
%\jacob{In Table 1 we promise in the caption both MAE and MSE but we show only one measure - YS: fixed, only MSE is reported due to space limitations}
%tables and figures - cont.
\linespread{1.1}
\begin{table*}[h!]
\small
\caption{Pearson (R) correlation and Spearman ($\rho$) coefficients between  the uncertainty reported by MoGU and the MAE of its predictions. The correlations were computed per variable and then averaged, showing the relationship for a 96-time point horizon. All reported results are statistically significant with a p-value $\le 0.00001$.}
	\label{tab:unc_corr}
	\centering
		\begin{tabular}{cc|c|cc|cc|cc}
    \cline{2-9}
    & \multicolumn{2}{c|}{Uncertainty}& \multicolumn{2}{c|}{Aleatoric (A)}& \multicolumn{2}{c|}{Epistemic (E)} & \multicolumn{2}{c}{Total (A+E)}\\
    \cline{2-9}
    & \multicolumn{2}{c|}{Corr. Coeff.}&R&$\rho$&R&$\rho$&R&$\rho$\\
    \cline{2-9} 
    
&\multirow{4}*{\rotatebox{90}{iTransfor.}} & ETTh1 & 0.25 & 0.22 & 0.03 & 0.04 & 0.25 & 0.22 \\
&\multicolumn{1}{c|}{} & ETTh2 & 0.15 & 0.20 & 0.08 & 0.15 & 0.15 & 0.21\\
&\multicolumn{1}{c|}{} & ETTm1 & 0.27 & 0.29 & 0.10 & 0.13 & 0.27 & 0.30 \\
&\multicolumn{1}{c|}{} & ETTm2 & 0.15 & 0.17 & 0.13 & 0.24 & 0.16 & 0.19 \\
\cline{2-9}
&\multirow{4}*{\rotatebox{90}{PatchTST}} & ETTh1 & 0.26 & 0.23 & 0.05 & 0.05 & 0.26 & 0.23 \\
&\multicolumn{1}{c|}{} & ETTh2 & 0.14 & 0.17 & 0.12 & 0.20 & 0.14 & 0.17 \\
&\multicolumn{1}{c|}{} & ETTm1 & 0.31 & 0.30 & 0.07 & 0.11 & 0.31 & 0.30 \\
&\multicolumn{1}{c|}{} & ETTm2 & 0.11 & 0.11 & 0.14 & 0.25 & 0.11 & 0.11\\
\cline{2-9}
\end{tabular}
\end{table*}
\linespread{1.0}



        \begin{figure}[h!]
        \centering
        \begin{subfigure}[t]{0.45\textwidth}
            \centering
            \includegraphics[width=\linewidth]{Chapters/Figures/patchtst_etth1_v0s0.png}
            \subcaption{PatchTST/ETTh1}
            
            \label{fig:unc_graph_a}
        \end{subfigure}
        \hfill % Adds horizontal space between subfigures
        \begin{subfigure}[t]{0.45\textwidth}
            \centering
            \includegraphics[width=\linewidth]{Chapters/Figures/itransformer_ettm1_v0s2.png}
            \subcaption{iTransformer/ETTm1}
            
            \label{fig:unc_graph_b}
        \end{subfigure}
        \caption{Example forecasts along with the ground truth, the MAE and uncertainty reported by MoGU with three experts. The forecasts for the Etth1 dataset (a) were generated using PatchTST as the expert architecture, while those for Ettm1 (b) were generated using iTransformer.}
        \label{fig:unc_graph}
    \end{figure}
\linespread{1.1}
\begin{table*}[h!]
\small
\caption{Forecasting errors for a 96-time point horizon of MoE, MoG and MoGU models. The best results are shown in bold for each configuration and dataset.}
%\caption{A comparison between a MoE, which lacks uncertainty estimation and uses the standard input-based softmax gating, a MoG which uses the same gating as the deterministic MoE, and our  MoGU. Results show the MAE and MSE for both configurations using iTransformer and PatchTST as expert architectures, for a 96-time point horizon. The best results are shown in bold for each configuration.}
	\label{tab:ablations_gating}
	\centering
		\begin{tabular}{cc|c|cc|cc|cc}
    \cline{2-9}
    & \multicolumn{2}{c|}{Mixture Type}& \multicolumn{2}{c|}{MoE}& \multicolumn{2}{c|}{MoG} &\multicolumn{2}{c}{MoGU}\\
    \cline{2-9}
    & \multicolumn{2}{c|}{Metric}& MAE & MSE & MAE & MSE & MAE & MSE\\
    \cline{2-9}
&\multirow{4}*{\rotatebox{90}{iTransfor.}} & ETTh1 &  0.410 &  0.393 &  0.403 & 0.387 & \textbf{0.400} & \textbf{0.380}   \\
&\multicolumn{1}{c|}{} & ETTh2 & 0.348 & 0.299 & 0.340 & 0.288 & \textbf{0.336} & \textbf{0.283 } \\
&\multicolumn{1}{c|}{} & ETTm1 & 0.367 & 0.332 & 0.360 & 0.326 & \textbf{0.356} & \textbf{0.320} \\
&\multicolumn{1}{c|}{} & ETTm2 & 0.261 & 0.179 &\textbf{0.256} & \textbf{0.175} & 0.260 & 0.179  \\
\cline{2-9}
&\multirow{4}*{\rotatebox{90}{PatchTST}} & ETTh1 & \textbf{0.406} & \textbf{0.386} & 0.420 & 0.413 & {0.415} & {0.409}   \\
&\multicolumn{1}{c|}{} & ETTh2 & 0.347 & 0.298 & 0.343 & 0.291 &\textbf{ 0.331} & \textbf{0.277} \\
&\multicolumn{1}{c|}{} & ETTm1 & 0.371 & 0.337 & 0.372 & 0.337 & \textbf{0.362} & \textbf{0.326}  \\
&\multicolumn{1}{c|}{} & ETTm2 & 0.264 & 0.177 &  \textbf{0.259} & 0.176 & \textbf{0.259} & \textbf{0.175} \\
\cline{2-9}
\end{tabular}
\end{table*}
\linespread{1.0}
\linespread{1.1}
\begin{table*}[h!]
\small
\caption{Ablation study of the uncertainty head's architecture with a  96-time point horizon. The best results are shown in bold for each configuration and dataset.}
	\label{tab:ablations_head_arc}
	\centering
		\begin{tabular}{cc|c|cc|cc}
    \cline{2-7}
    & \multicolumn{2}{c|}{Head Architecture}& \multicolumn{2}{c|}{FC}& \multicolumn{2}{c}{MLP}\\
    \cline{2-7}
    & \multicolumn{2}{c|}{Metric}& MAE & MSE & MAE & MSE\\
    \cline{2-7}
&\multirow{4}*{\rotatebox{90}{iTransfor.}} & ETTh1 & \textbf{0.399} & 0.383  & 0.400 & \textbf{0.380}   \\
&\multicolumn{1}{c|}{} & ETTh2 & 0.338  & 0.286 & \textbf{0.336} & \textbf{0.283 } \\
&\multicolumn{1}{c|}{} & ETTm1 &0.357 &  0.321& \textbf{0.356} & \textbf{0.320} \\
&\multicolumn{1}{c|}{} & ETTm2  & 0.261 & 0.178  & 0.260 & 0.179  \\
\cline{2-7}
&\multirow{4}*{\rotatebox{90}{PatchTST}} & ETTh1 & 0.410 & 0.401 & \textbf{0.415} & \textbf{0.409}   \\
&\multicolumn{1}{c|}{} & ETTh2 &  0.340 & 0.285  &\textbf{ 0.331} & \textbf{0.277} \\
&\multicolumn{1}{c|}{} & ETTm1 & {0.356} & \textbf{0.320}  & \textbf{0.362} & {0.326}  \\
&\multicolumn{1}{c|}{} & ETTm2 & 0.260 & \textbf{0.174} & \textbf{0.259} & {0.175} \\
\cline{2-7}
\end{tabular}
\end{table*}
\linespread{1.0}




\subsubsection{Time Series Forecasting with MoGU}\label{subsec:tsf_results}
Table \ref{tab:main_results_num_experts} compares MoGU's performance against single-expert and standard MoE configurations on the ETT datasets. Using iTransformer as the expert architecture and varying the number of experts from 2 to 5, MoGU consistently yields more accurate predictions than both single-expert and standard MoE settings.
Tables \ref{tab:main_results} and \ref{tab:main_results_exchange} provide further comparisons between a three-expert MoE and MoGU. MoGU outperforms standard MoE in the majority of cases across different multivariate forecasting datasets and horizon lengths, utilizing iTransformer, PatchTST, and DLinear as expert architectures. 

\subsubsection{Uncertainty Quantification for Time Series Forecasting with MoGU}\label{subsec:unc_analysis}
To assess how well MoGU's reported uncertainty aligns with its actual prediction errors, we compute the Pearson (R) and Spearman ($\rho$) correlation coefficients between them. Table \ref{tab:unc_corr} presents these coefficients for the aleatoric, epistemic, and total uncertainties (as defined in Eq. \ref{eq:var-mogu}).

We observe a statistically significant positive correlation between MoGU's uncertainty estimates and the Mean Absolute Error (MAE) of its predictions. Interestingly, the correlation with aleatoric uncertainty is typically higher than with epistemic uncertainty. Since aleatoric uncertainty represents the inherent randomness in the data itself, this correlation suggests that the model can use uncertainty estimates to identify data points where irreducible randomness makes accurate predictions difficult, thereby leading to higher errors.

Fig. \ref{fig:unc_graph} illustrates the relationship between MoGU's prediction error and uncertainty estimates by showing the predicted and ground truth values alongside the MAE and reported uncertainty for representative examples. The uncertainty at each time point closely follows the prediction error. 
We further show the Pearson correlation heatmaps in Fig. \ref{fig:heatmaps} in our Appendix. These heatmaps further visualize the relationship between the Mean Absolute Error (MAE) of MoGU's predictions and its reported uncertainties (aleatoric, epistemic, and total), when using MoGU with three iTransformer experts. The analysis is presented per variable for each of the ETT datasets, highlighting the extent to which different uncertainty components correlate with predictive error. While the correlation between uncertainty and MAE varies among variables, it remains consistently positive.

\subsubsection{Ablations}\label{subsec:ablations}
We conducted an ablation study to evaluate our key design choices. For all experiments, we used a configuration with three experts.

\textbf{Gating Mechanism.} Table \ref{tab:ablations_gating} compares our MoGU to a standard input-based gating mechanism \citep{jacobs1991adaptive}, when employed by a deterministic MoE and with a MoG. The input-based method utilizes a separate neural module to predict weights by processing the input before a softmax layer. We evaluated the MoE, MoG and MoGU methods on four ETT datasets using iTransformer and PatchTST as the expert architectures. Our uncertainty-based gating consistently resulted in a lower prediction error.

\textbf{Uncertainty Head Architecture.}
We also evaluated the design of our uncertainty head, which is implemented as a shallow Multi-Layer Perceptron (MLP) with a single hidden fully connected layer. Table \ref{tab:ablations_head_arc} compares this to an alternative using only a single fully connected layer. The MLP alternative performed better in most cases, though the performance difference was relatively small.

\textbf{Resolution of Uncertainty Estimation.}
Table \ref{tab:ablations_time} in our Appendix explores an alternative where the expert estimates uncertainty at the variable level ('Time-Fixed'), rather than for each individual time point ('Time-Varying'). Predicting uncertainty at the higher resolution of a single time point yielded better results, demonstrating the advantage of our framework's ability to provide high-resolution uncertainty predictions. We note that our framework is flexible and supports both configurations.

Additional ablations for our \textbf{Loss Function} are provided in the Appendix (Section~\ref{subsec:app_ablations}).







%\input{Chapters/limitations_and_future_work}
\section{Conclusion}

In this work, we presented a full-stack investigation of LLM unlearning, encompassing methodology, evaluation, and robustness. We established a principled taxonomy that organizes twelve representative unlearning methods into three families: {\MDiv}, {\MRep}, and {\MRej}, providing a systematic lens to understand their underlying mechanisms. Our analysis revealed that conventional multiple-choice questioning (MCQ) evaluations of unlearning effectiveness (UE) and utility retention (UT) offer an incomplete picture, and we introduced open question answering (Open-QA) as a complementary paradigm to better capture generative behaviors and expose the strengths and limitations of different methods. Furthermore, we provide a comprehensive robustness assessment across model-level and input-level attacks, revealing nuanced relationships among in-domain relearning, out-of-domain fine-tuning, quantization, and jailbreak attacks. These findings clarify the trade-offs of current unlearning algorithms and guide the design of future methods that are both effective and robust. The use of LLM, limitation and broader impact are further discussed in \textbf{Appendix\,\ref{appx:llm_usage}}, \textbf{Appendix\,\ref{appx:limit}} and \textbf{Appendix\,\ref{appx:impact}}.


\bibliographystyle{iclr2026_conference}
\bibliography{moe_uncertainty_tsf}

\appendix
\section{Appendix}
We provide additional results and details that were not included in the main text due to space limitations.
\subsection{Correlation Heatmaps: Uncertainty versus Prediction Error}
Fig.~\ref{fig:heatmaps} shows correlation heatmaps discussed in Section \label{subsec:unc_analysis} in the main text. This heatmap visualizes the relationship between the Mean Absolute Error (MAE) of MoGU's predictions and its reported uncertainties (aleatoric, epistemic, and total) for a model using three iTransformer experts.
    \begin{figure}[h!]
        \centering
        \begin{subfigure}[t]{0.24\textwidth}
            \centering
            \includegraphics[width=\linewidth]{Chapters/Figures/itransformer_etth1_heatmap_elig.png}
             \subcaption{ETTh1}
            \label{fig:heatmap_a}
        \end{subfigure}
        \hfill % Adds horizontal space between subfigures
        \begin{subfigure}[t]{0.23\textwidth}
            \centering
            \includegraphics[width=\linewidth]{Chapters/Figures/itransformer_etth2_heatmap_elig.png}
             \subcaption{ETTh2}
            \label{fig:heatmap_b}
        \end{subfigure}
        \hfill % Adds horizontal space between subfigures
        \begin{subfigure}[t]{0.23\textwidth}
            \centering
            \includegraphics[width=\linewidth]{Chapters/Figures/itransformer_ettm1_heatmap_elig.png}
             \subcaption{ETTm1}
            \label{fig:heatmap_c}
        \end{subfigure}
        \hfill % Adds horizontal space between subfigures
        \begin{subfigure}[t]{0.23\textwidth}
            \centering
            \includegraphics[width=\linewidth]{Chapters/Figures/itransformer_ettm2_heatmap_elig.png}
            \subcaption{ETTm2}
            \label{fig:heatmap_d}
        \end{subfigure}
        
        \caption{Heatmaps of the Pearson correlation between MoGU's reported uncertainties (aleatoric, epistemic, and total) and the MAE of its predictions. The correlation is displayed per variable for the ETT datasets.}
        \label{fig:heatmaps}
    \end{figure}
\subsection{Additional Ablations}\label{subsec:app_ablations}
\textbf{Resolution of Uncertainty Estimation.} We provide Table \ref{tab:ablations_time}, discussed in the main text. This table explores an alternative where the expert estimates uncertainty at the variable level ('Time-Fixed'), rather than for each individual time point ('Time-Varying').

\textbf{Loss Function.} We note that the MoGU model can also be optimized through the following MoG loss:
\begin{equation}\label{eq:mog_loss_alt}
    \mathcal{L}_{}= - \log (\sum_i w_i(x)\mathcal{N}(y;y_i(x), \sigma_i^2(x))) 
\end{equation}
where $\mathcal{N}$ is the Normal density function and the loss
has the form of a Negative Log Likelihood (NLL) of a MoG distribution.
We compare the performance of our model when using the loss presented in Eq. \ref{eq:mog_loss} and when using the aforementioned alternative (Eq. \ref{eq:mog_loss_alt}). The results of this experiment, presented in Table \ref{tab:ablations_loss} in our Appendix, suggest that optimizing with our proposed loss (Eq. \ref{eq:mog_loss}) yields more effective learning and consistently better results by imposing a stricter constraint on expert learning compared to the MoG loss.

\linespread{1.1}
\begin{table*}[h!]
\small
\caption{Ablation study on the resolution of reported uncertainty. We compare two methods for estimating variance in both MoE and MoGU: estimating it once per horizon versus estimating it for each time point (per variable in both cases). The table reports the MAE and MSE for each configuration. All results were generated using iTransformer as the expert architecture with a 96-time-point horizon.}
	\label{tab:ablations_time}
	\centering
		\begin{tabular}{cc|c|cc|cc}
    \cline{2-7}
    & \multicolumn{2}{c|}{Prediction Variance}& \multicolumn{2}{c|}{Time-Fixed}& \multicolumn{2}{c}{Time-Varying}\\
    \cline{2-7}
    & \multicolumn{2}{c|}{Metric}& MAE & MSE & MAE & MSE\\
    \cline{2-7}
\multicolumn{3}{c|}{ETTh1} & 0.401 & 0.392 & \textbf{0.400 }& \textbf{0.380}   \\
\multicolumn{3}{c|}{ETTh2} &  0.337 & 0.290 & \textbf{0.336} & \textbf{0.283} \\
\multicolumn{3}{c|}{ETTm1} & 0.360 & 0.324 & \textbf{0.356} & \textbf{0.320} \\
\multicolumn{3}{c|}{ETTm2}  &  \textbf{0.255} & \textbf{0.174}  & 0.260 & 0.179  \\
\cline{2-7}
\end{tabular}
\end{table*}
\linespread{1.0}
\linespread{1.1}
\begin{table*}[h!]
\small
\caption{Ablation study of MoGU's loss. We compare the loss formulation in Eq. \ref{eq:mog_loss}, used by MoGU to an alternative MoG loss, given in Eq. \ref{eq:mog_loss_alt} }
	\label{tab:ablations_loss}
	\centering
		\begin{tabular}{cc|c|cc|cc}
    \cline{2-7}
    & \multicolumn{2}{c|}{Loss Formulation}& \multicolumn{2}{c|}{Eq. \ref{eq:mog_loss_alt} (Alt. MoG loss)}& \multicolumn{2}{c}{Eq. \ref{eq:mog_loss} (MoGU's loss)}\\
    \cline{2-7}
    & \multicolumn{2}{c|}{Metric}& MAE & MSE & MAE & MSE\\
    \cline{2-7}
\multicolumn{3}{c|}{96} & 0.343 & 0.304 & \textbf{0.336 }& \textbf{0.283}   \\
\multicolumn{3}{c|}{192} & 0.389 & 0.378 & \textbf{0.387} & \textbf{0.361} \\
\multicolumn{3}{c|}{336} & \textbf{0.424 }& 0.422 & 0.425 & \textbf{0.415} \\
\multicolumn{3}{c|}{720}  &  \textbf{0.438} & \textbf{0.421}  & 0.442 & \textbf{0.421}  \\
\cline{2-7}
\end{tabular}
\end{table*}
\linespread{1.0}
\subsection{MoGU's Algorithm}
We provide the pseudo code for MoGU in Listing 1 to enhance clarity.
Furthermore, to ensure reproducibility, our code and the scripts needed to reproduce the main results are available at: \url{https://github.com/yolish/moe_unc_tsf}
We implemented MoGU to be highly configurable, so that users can  specify the number of experts, the expert architecture, the mixture type (MoE or MoG) and the gating mechanism.

\begin{algorithm}[t]
\label{listing:mogu}
\caption{Mixture-of-Gaussians with Uncertainty-based gating (MoGU)}
\begin{algorithmic}[1]  % line numbers
\Require Training set $X$, labels $y$
\Ensure Model parameters $\theta$

\For{each training epoch}
  \For{each mini-batch $\mathcal{B}$}
    \For{each sample $x \in \mathcal{B}$}
      \For{each expert $i=1,\dots,k$}
        \State Compute expert output $f_i(x) = \mathcal{N}(y; \mu_i(x,\theta),\sigma_i^2(x,\theta))$.
        \State Set $w_i(x)= \frac{\sigma_i^{-2}(x)}{\sum_j \sigma_j^{-2}(x)}$.
      \EndFor
     \EndFor
    %\State Compute loss $\mathcal{L} = -\sum_{x \in \mathcal{B}} \sum_{i=1}^k w_i(x) \log (f_i(x))$.
    \State Compute loss $\mathcal{L}= \sum w_i(x)\mathcal{L_{NLLG}}(y;y_i(x), \sigma_i^2(x)))$.
    \State Update model parameters.
  \EndFor
\EndFor 

\State Test time prediction is $\hat{y}(x) = \sum_i w_i(x) \mu_i(x)$.
\State Test time prediction uncertainty is: $\sum_i w_i(x) \sigma_i^2(x) +  \sum_i w_i(x) (\hat{y}(x)-\mu_i(x))^2$.
\end{algorithmic}
\end{algorithm}
\end{document}
