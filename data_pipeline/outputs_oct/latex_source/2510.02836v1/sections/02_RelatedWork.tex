\section{Related Work}
\label{sec:related-work}

We surveyed various lines of research that our work draws upon, and grouped them into two
areas: \emph{i)} knowledge workers and well-being (\S\ref{subsec:rw_1}); and \emph{ii)} the role of VR in promoting well-being (\S\ref{subsec:rw_2}).

\subsection{Knowledge Work and Well-being}
\label{subsec:rw_1}
Knowledge workers face a number of challenges due to technological advances, hybrid work, and shifting organizational structures~\cite{constantinides2022future, rudnicka2020eworklife}. While these changes bring flexibility, they also introduce stressors (e.g., long hours, lack of meaningful work, and poor work relationships) which can lead to alienation~\cite{nair_exploration_2010}. Constant connectivity fragments attention with frequent interruptions~\cite{soto_observing_2021}, while open-plan offices and remote work exacerbate distractions, isolation, and blur work-life boundaries~\cite{teevan_new_2021}. Together, these factors contribute to stress, burnout, and mental health problems, which ultimately reduces engagement and work outcomes~\cite{who_burnout_2019}. Therefore, employee well-being is vital to organizational success.

Knowledge work requires environments that support autonomy and creativity~\cite{mladkova_knowledge_2011} because it involves processing complex information and making decisions without clear guidelines~\cite{septiandri2024potential}. However, organizational tools and technologies focused on standardization and efficiency can hinder these needs~\cite{karr-wisniewski_when_2010}. Tools such as task managers and AI-driven automation might reduce cognitive load~\cite{das2023focused} but could also perpetuate overwork and stress~\cite{leshed_i_2011, mark_effects_2018}. HCI research has recently increasingly moved from productivity-focused agendas to the holistic experience of technology use in work settings \cite{guillou_is_2020, kim_understanding_2019}, through well-being-focused solutions such as mindfulness apps or tools for supporting emotional resilience~\cite{howe_design_2022}. However, these technologies often face challenges in adoption as they are treated as optional add-ons rather than integral parts of work processes. Moreover, they frequently fail to provide workers with opportunities to detach fully from their workplace and tasks, limiting their ability to recharge and mentally recover.

\begin{figure*}[h!]
  \centering

  \includegraphics[width=.98\textwidth, height=3.9cm]{figures/methodology.png} 

\caption{Two-phase study methodology. In Phase 1, we conducted semi-structured interviews with 10 knowledge workers to gather design requirements that informed the development of a VR well-being app \emph{Tranquil Loom}. In Phase 2, we deployed the app in a workplace setting with 35 participants and evaluated it through pre-post well-being measures, usage data, and follow-up interviews.}
\label{fig:methodology}

\end{figure*}

\subsection{The Role of VR in Promoting Well-being}
\label{subsec:rw_2}
VR apps have been shown to support well-being through stress management~\cite{mazgelyte_effects_2021, pimentel_digital_2019}, emotional regulation~\cite{garcia-ballesteros_garden_2024, rodriguez_vr-based_2015}, and physical rehabilitation~\cite{camporesi_vr_2013}. Recent studies have explored their application in workplace well-being~\cite{Riches2023Virtual, Naylor2020A}, showing that VR interventions can reduce anxiety and negative mood states through nature-inspired or abstract environments and guided meditation~\cite{8252142, adhyaru2022virtual}. Thoondee and Oikonomou~\cite{8252142} demonstrated VR's stress-reducing effects for office workers, while Heyse et al.~\cite{heyse} enhanced relaxation by tailoring content to users’ emotions.

Nevertheless, many VR apps lack specificity for knowledge workers, often targeting general well-being instead of unique stressors and cognitive demands~\cite{8252142}. Hardware discomforts, such as prolonged headset use, limit adoption~\cite{kim_effect_2021}, and logistical issues in open-plan offices further complicate implementation. Privacy concerns surrounding data collection hinder trust~\cite{kaminska}, and sustaining engagement remains difficult, as the novelty of VR tools often wears off~\cite{Riches2023Virtual}. Despite these barriers, VR can provide distraction-free environments for relaxation and mindfulness~\cite{Naylor2019Augmented}. Features such as emotion-based adaptation~\cite{heyse} and biometric feedback~\cite{kaminska} can personalize experiences and boost engagement. Integrating VR within existing workplace tools such as calendars, can make tools more seamless and practical~\cite{chow_feeling_2024}. At the same time, transparent data practices are crucial for building trust~\cite{tahaei2023human}. By addressing these challenges, VR has the potential to become an effective and tailored solution for workplace well-being.
\smallskip

\noindent\textbf{Research Gaps.} 
%\todo{make sure this paragraph is linked with RQs and surfaces the complication that is stated in the abstract} 
Prior research on VR well-being often takes a solution-driven approach by evaluating predefined apps that target singular outcomes such as relaxation or focus \cite{adhyaru2022virtual}. However, this approach overlooks the multifaceted and fluctuating nature of workplace well-being, especially among knowledge workers whose needs can shift throughout the day. Moreover, many studies involve users only at the evaluation stage and exclude their perspectives from the design process \cite{Riches2023Virtual, heyse, 8252142}. As a result, existing tools may lack the flexibility and personal relevance needed for real-world uptake. In response, we adopted a human-centered approach integrating workers' experiences from early design to deployment.

\section{Author Positionality Statement}
We recognize that our positionality shaped the study's framing, design, and interpretation~\cite{havens2020situated}.  Our team, based at an Eastern European organization, includes one female and three male researchers with backgrounds in HCI, interaction design, AI, immersive and ubiquitous technologies.  Our positionality has shaped the study design in two ways: \emph{1)} prioritizing participants' lived experiences and perceptions over technology evaluation; and \emph{2)} balancing enthusiasm for immersive tech with critical reflection on feasibility. We also acknowledge that our institutional and personal perspectives informed our choices, while other interpretations may remain valid.

