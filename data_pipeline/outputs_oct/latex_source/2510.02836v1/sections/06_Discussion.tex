\section{Discussion}
\label{sec:discussion}

% Our work investigated how VR can support knowledge workers' well-being through short and self-directed breaks during work. Using a human-centred approach, we designed a VR well-being app called Tranquil Loom, and evaluated it with 35 knowledge workers. Quantitative data showed significant reductions in anxiety and increases in mindfulness following brief sessions. Stretching and open-ended exploration were especially engaging, which emphasize a need to balance structured and spontaneous experiences. Moreover, participants consistently emphasized the importance of sound design and environmental immersion, reinforcing that sensory qualities substantially shape VR's perceived effectiveness~\cite{adhyaru2022virtual}. 


We investigated how VR can support knowledge workers' well-being through short and self-directed breaks. Using a human-centered approach, we designed and evaluated Tranquil Loom with 35 participants. Brief VR sessions significantly reduced anxiety and increased mindfulness. Stretching and exploration were especially engaging, and highlight the need to balance structure and spontaneity. Participants also emphasized the importance of sound and immersive environments in shaping VR's effectiveness~\cite{adhyaru2022virtual}.

Additionally, our participants valued the idea of emotionally adaptive support despite usage data showed little difference between AI-selected and user-chosen activities. However, participants emphasized that AI should remain optional to preserve autonomy. Overall, these findings reinforce the value of flexible VR apps that adapt to users' momentary states without imposing expectations.





% Our approach of integrating AI-driven emotional reflection prompts and varied immersive activities opened avenues for considering how VR might adapt to individual well-being states. \revision{While usage data showed little difference between activities selected by the AI assistant and those chosen independently, participants valued the idea of personalized and emotionally adaptive support. The presence of AI, even in a simple form, prompted visions of more responsive features such as mood-based suggestions, conversational agents, and voice interactions. However, participants emphasized that AI should remain optional and supportive to preserve users' sense of autonomy.} Overall, these findings reinforce the value of flexible and emotionally attuned VR apps that adapt to users' momentary states without imposing expectations.


% We investigated how VR can support the well-being of knowledge workers through short and self-directed breaks during the workday. We employed a human-centred approach to design, develop, and evaluate Tranquil Loom, a VR app created in response to the diverse and fluctuating well-being needs identified in formative interviews and related work. \revision{Our deployment with 35 knowledge workers revealed the practical value of a VR system, even in its current modular form. Quantitative data showed significant reductions in anxiety and increases in mindfulness following short sessions. Participants found stretching and open-ended exploration particularly engaging, reflecting a balanced interest in both structured and spontaneous experiences. }

% Usage data showed no meaningful difference between activities recommended by the AI assistant and those chosen by users directly. This suggests that participants relied more on personal preference than system guidance, reinforcing the importance of agency and intuitive exploration. However, the presence of AI, even in a lightweight form, prompted participants to imagine more personalized experiences. They envisioned emotionally adaptive suggestions, custom meditations, voice interaction, and embodied AI agents that respond to mood or context. Importantly, participants wanted AI to support and not direct their choices. Suggestions should be optional, relevant, and easy to dismiss, rather than imposing structure or expectations.

% Additionally, stretching activities emerged as particularly engaging, followed closely by open-ended exploration, reflecting a balanced interest in both structured and spontaneous well-being experiences, supporting the findings of our formative study. Such patterns indicate that effective VR designs must accommodate diverse user preferences without prescribing rigid interaction modalities. Qualitative insights further deepened our understanding of this by highlighting how participants engaged playfully with the virtual environments, demonstrating a spontaneous inclination towards exploratory behaviors. Playful exploration was not only inherently satisfying but also provided a sense of mental restoration distinct from traditional, structured well-being activities. Moreover, participants consistently emphasized the importance of sound design and environmental immersion, reinforcing that sensory qualities substantially shape VR's perceived effectiveness~\cite{adhyaru2022virtual}. Reflexively, our approach of integrating AI-driven emotional reflection prompts and varied immersive activities opened avenues for considering how VR might adapt to individual well-being states.


% \todo{comment 4. Marios to move this in the design implications} \revision{However, more than the individual activities themselves, participants valued the system’s situational flexibility. \revision{Our results point to a clear preference for using VR as a \textit{drop-in tool} accessed spontaneously and frequently, without pre-booking or scheduling. Participants favored short sessions (5–20 minutes) that could fit between tasks, rather than fewer, longer breaks. Requiring bookings, advanced planning, or daily check-ins was seen as impractical and, in some cases, counterproductive. This indicates that organizations should treat VR as a situational support system, available when needed, not when assigned. From a managerial perspective, this means facilitating access without gatekeeping: dedicated spaces with available headsets and minimal setup would better support uptake.  Importantly, participants also made clear what not to do. VR well-being tools should not be positioned as productivity enhancers, tracked interventions, or compulsory practices. Booking systems, log-ins, usage quotas, progress tracking, and employer oversight were all viewed as potential barriers to engagement, undermining trust and discouraging participation. These barriers conflict with the very flexibility participants valued and should be avoided in real-world deployments.  Instead, participants wanted something personal, private, and easy to access; something they could reach for in moments of stress without explanation or commitment. For managers, this means shifting from measurement to trust: creating space for well-being without tying it to outcomes.}}




Drawing from our findings, we then discuss five design trade-offs for  VR workplace well-being apps:  assigned use \emph{vs.} spontaneous access; novelty \emph{vs.} familiarity; doing \emph{vs.} being; structure \emph{vs.} openness; and AI guidance \emph{vs.} user autonomy.

% Furthermore, our data uncovered five design trade-offs inherent in VR workplace well-being interventions: 


% Participants debated the benefits of \emph{novelty against the comfort of familiar environments}, considered \emph{active (doing) vs passive interaction (being)}, grappled with preferences for \emph{structured activities vs open-ended, playful exploration}, \revision{weighed assigned use \emph{vs.} spontaneous access, and reflected on the balance between AI guidance \emph{vs.} user autonomy}. Such tensions bring to the surface important considerations for future VR well-being designs, setting the stage for discussions on navigating these trade-offs to maximize user benefits effectively. 

\smallskip
% \noindent \textbf{Design trade-offs.}
% \smallskip
\noindent \textbf{\revision{Assigned Use \emph{vs.} Spontaneous Access.}} \revision{While structured well-being programs often rely on scheduled activities and check-ins, our findings suggest that such formalization may be counterproductive in VR. Participants preferred using the app as a drop-in tool and viewed it as a resource they could access informally and frequently, without pre-booking or planning. Short and situational uses between tasks (e.g., 5–20 minutes) were favored over fewer and longer breaks. Participants emphasized that when VR becomes another item on the to-do list, its restorative value is diminished.}

\revision{This highlights a tension between assigned use (which enables organizational oversight) and spontaneous access (which supports personal autonomy). Participants viewed booking systems, log-ins, and usage tracking as intrusive, likely to undermine trust and discourage use~\cite{bailey2006need}. Instead, they advocated for low-friction access through open spaces, minimal setup, and no monitoring. They framed well-being not as something to be scheduled or tracked, but as a right to be supported when needed.} Future implementations should explore ultra-accessible deployments such as quiet VR booths or side rooms directly within the office. Such setups could further lower barriers to use and promote spontaneous micro-breaks during the workday.


\smallskip


% \revision{This highlights a tension between assigned use (allows organizations to measure and manage participation) and spontaneous access (supports individual autonomy and emotional readiness). Booking systems, log-ins, usage quotas, and productivity-linked incentives were viewed as intrusive and undermining trust and discouraging use. These mechanisms clashed with the emotional privacy that participants sought and introduced friction that discouraged uptake~\cite{bailey2006need}. Instead, participants advocated for low-friction access: open spaces, minimal setup, and no monitoring. They framed well-being not as something to be scheduled or tracked, but as a right to be supported when needed. For managers and designers, this means shifting from measurement to trust, where emotional availability is prioritized over organizational accountability.}

\noindent \textbf{Novelty \emph{vs.} Familiarity.} Nature-based environments \cite{annerstedt_inducing_2013, browning_can_2020} and biophilic elements \cite{gattullo_biophilic_2022, garcia-ballesteros_garden_2024} are widely used in VR well-being tools, often associated with reduced stress and improved mood. Our participants similarly preferred natural scenes (i.e., forests and beaches) not for their realism, but for their emotional familiarity. The familiar settings were seen as easier to settle into and more effective for detachment during short breaks, echoing environmental psychology research that shows recognizable, low-effort environments support recovery in cognitively demanding contexts \cite{korpela_restorative_1996}. However, participants raised concerns about repetition. Even calming environments could feel stale without variation~\cite{ng_chenrezig_2023}, pointing to a tension between overstimulating dynamic scenes and static ones that become dull. Our stylized but recognizable environments appeared to strike a balance: emotionally grounding yet perceptually fresh.

The novelty effect \cite{kini_xr_2024} played a limited role. While novelty could attract initial interest, participants found too much of it would be distracting, especially in high-focus work contexts. They preferred intuitive and low-effort experiences that do not require adjusting to new interfaces or making additional choices. This aligns with prior research on the tension between stimulation and ease \cite{chirico_when_2019}, though few studies address how to maintain engagement without relying on novelty. For our participants, novelty worked best as subtle change. Playful surprises would be welcomed but not a reason for repeated use. Instead, returning to familiar environments was described as a way of emotional anchoring, with slight changes becoming a way to match changing moods or energy levels.

\smallskip

\noindent \textbf{Doing \emph{vs.} Being.} Our findings emphasize the need to balance active engagement (``doing'') with passive immersion (``being'') in VR well-being tools \cite{atherton_doing_2020}. Participants often explored their surroundings walking, attempting to pick up objects, or searching for hidden features, even when explicit interaction was not available. Such behaviors reflected spontaneous curiosity and a desire to momentarily step outside the structure of work. At other times, participants preferred stillness. We observed them sitting quietly, listening to the sounds, or focusing on visual details. These contrasting modes supported different forms of recovery: ``doing'' helped re-engage attention and counter boredom, while ``being'' offered a calming and low-demand experience that reduced stress.  Our observations resonate with established psychological frameworks, such as Kaplan’s Attention Restoration Theory \cite{kaplan_restorative_1995}, which emphasizes the restorative power of environments that balance gentle stimulation and effortless attention. The deliberate inclusion of both active and passive immersion in Tranquil Loom aligns with this theory, suggesting that successful interventions should address both. 

% Beyond this, several participants described moments of awe such as becoming absorbed in virtual spaces or feeling unexpectedly moved by sound or scenery. Awe, linked to perceptual vastness and emotional well-being~\cite{he_i_2024}, may have been enabled by Tranquil Loom's environments, especially during open exploration.
Beyond this, several participants described moments that suggested a sense of awe; becoming absorbed in vast virtual spaces, ``forgetting where they are'', or feeling unexpectedly moved by sound or scenery. Awe is often linked to perceptual vastness and a need to mentally accommodate the experience, and has been associated with reduced self-focus and increased emotional well-being \cite{he_i_2024}. While not explicitly designed for this, Tranquil Loom's stylized and expansive environments may have enabled such responses, particularly during open-ended exploration.
\smallskip

\noindent \textbf{Structure \emph{vs.} Openness.}
The tension between structure and openness highlights a core design challenge: how to support presence without prescribing behavior. Our findings suggest the value of VR well-being tools lies not only in the activities offered, but in allowing users to engage on their own terms. Participants moved fluidly between active and passive states, guided more by how they felt than by predefined tasks. This was especially evident in exploration mode, where users engaged in spontaneous, playful acts such as looking for animals, imagining hidden features, or trying to fly. These unstructured behaviors were described as refreshing and emotionally satisfying. Rather than distractions, they were expressions of well-being where goal-driven thinking is suspended, therefore allowing alternative forms of attention\cite{atherton_doing_2020}. Although play is rarely foregrounded in workplace well-being tools, our findings align with research showing that light, curiosity-driven engagement can restore attention and support emotional regulation \cite{kaplan_restorative_1995}. Participants proposed low-pressure interactive features, like Easter eggs \cite{lakier_more_2022}, as gentle prompts for discovery and delight, offering emotional connection without cognitive strain. %
\smallskip

%Together, these insights suggest that future VR well-being tools should not frame structure and unstructured activity as a binary, but consider how to support fluid movement between the two. Supporting curiosity and play is as important as offering relaxation and focus.
\smallskip
\noindent\textbf{\revision{Guidance \emph{vs.} Autonomy.}} \revision{Participants were divided on whether AI suggestions added value to their experience or interfered with their own agency. Some appreciated Loomi's ability to reduce decision fatigue, especially when feeling overwhelmed or mentally drained. The assistant was seen as a gentle prompt that helped them get started without overthinking (i.e., guiding without prescribing).}
% ~\cite{luger2016like}
\revision{Yet many others preferred to ignore Loomi's suggestions entirely, choosing instead to follow their instincts or needs in the moment. Even those who acknowledged the AI's input often reframed it as a ``nice-to-have'' rather than something they actively relied on. Participants emphasized that AI should not override user agency, but offer lightweight scaffolding, perhaps suggesting ``a good place to start'' or adapting over time as preferences evolve. The tension here is between offering guidance and supporting self-direction. However, this reflects a broader trade-off: should the tool guide users toward pre-determined ``helpful'' paths, or should it simply open space for reflection and allow the user to lead? Too much guidance risks feeling prescriptive~\cite{binns2018s}, while too little may leave users feeling unsupported, especially during vulnerable moments.}




% Throughout our research, we found that non-goal-oriented play and playful interactions significantly enhanced the value of VR as a drop-in well-being tool in workplace settings. Unlike traditional gamification, which tends to focus on structured tasks or clear objectives \cite{liarokapis2022improving}, unstructured interactions, such as open exploration, helped participants mentally disengage from work and find immediate relief. These interactions tapped into natural curiosity, a quality widely recognized in the literature as important for stress reduction and creativity \cite{atherton_doing_2020}. In our study, moments like running across virtual water or trying to interact with flowers allowed users to reconnect with a playful mindset. At the same time, the study revealed a balance must be struck between play and overstimulation. Participants suggested small and low-effort enhancements  to increase engagement without adding cognitive load. This aligns with Riches et al. \cite{Riches2023Virtual}, who emphasize the value of curiosity-driven interactivity in VR well-being tools. 



\section{Design Implications}
\label{sec:design-implications}
\revision{Drawing from the design trade-offs, we discuss key considerations for developing effective VR well-being apps for the workplace. Together, these implications call for a (re)thinking of how VR well-being apps are positioned and designed: \emph{not as solutions to optimize workers, but as emotionally intelligent tools that support autonomy, trust, and self-directed recovery}.}
% \revision{Drawing from the design trade-offs, we discuss key considerations for developing effective VR well-being apps for the workplace. These implications call for (re)thinking how VR well-being apps are positioned and designed: \emph{not to optimize workers, but to support autonomy, trust, and self-directed recovery}.}
\smallskip

\noindent\textbf{\revision{Enable spontaneous use through low-friction access.}} \revision{Participants preferred short (5-20 minutes) situational use over scheduled sessions. VR well-being apps should be easily accessible without pre-planning or tracking through dedicated spaces, minimal setup, and optional use models. Well-being should not be another task to manage, but a resource to reach for when needed.}
\smallskip

\noindent\textbf{\revision{Design for emotional continuity, not novelty.}} \revision{Our findings challenge the assumption that novelty is necessary for sustained engagement. Instead, emotionally familiar environments supported relaxation and emotional anchoring. Rather than offering entirely new experiences, designers should support subtle evolution over time (e.g., shifting light, ambient sounds) that preserves the emotional tone and, where possible, create feelings of `awe'.  The `novelty effect', when used, should support the user's sense of ease.}
\smallskip

\noindent\textbf{\revision{Support fluid transitions between modes of engagement.}}
\revision{Well-being is not a fixed state, and users' needs may shift even within a single session. Participants naturally moved between active and passive engagement based on how they felt. VR apps support different modes of presence. Designers should enable transitions between ``doing'' and ``being'' to accommodate fluctuating attention and energy levels. Passive features such as ambient audio, as well as optional play elements (e.g., hidden objects or responsive scenery) may offer restorative engagement without pressure.}
\smallskip

\noindent\textbf{\revision{Embed structure without enforcing it.}}
\revision{Our findings suggest that future VR well-being tools should not frame structure and unstructured activity as a binary, but consider how to support fluid movement between the two. Supporting curiosity and play is as important as offering relaxation and focus. While structured activities can offer purpose, they should be offered as optional scaffolding available when needed, but never imposed. Design should empower users to dip in and out, choose their own paths, and interpret well-being on their own terms. Even ``guidance'' features should default to low-commitment interactions, with space for or free-form use.}
\smallskip

\noindent\textbf{\revision{Design AI guidance as optional and contextual scaffolding.}} \revision{Participants valued AI support when it was framed as gentle and optional guidance that reduced decision fatigue during moments of stress. However, they strongly resisted prescriptive or overly directive suggestions. Designers should treat AI as a companion that offers context-aware prompts without assuming authority. This means enabling users to easily ignore, override, or disable suggestions, while still allowing for personalisation if and when users seek it. Adaptive tools should learn preferences subtly and transparently to support autonomy rather than structuring behavior.} Additionally, future VR well-being apps could leverage generative AI to allow users to co-create or modify their environments (e.g., turning a beach into a bay or adding unique features), which potentially may foster a stronger sense of ownership and encourage return visits.

% \revisionsecond{Similarly, future VR well-being apps could leverage generative AI for environment co-creation (e.g., turning a beach into a bay).}




% \revisionsecond{Additionally, future VR well-being apps could leverage generative AI to allow users to co-create or modify their environments (e.g., turning a beach into a bay or adding unique features), which potentially may foster a stronger sense of ownership and encourage return visits.}







% \noindent\textbf{\revision{Prioritise trust with transparent and minimal AI.}} \revision{Participants were open to personalisation but wary of AI features that felt intrusive, generic, or controlling. Design should favour transparency, limit behavioural tracking, and offer opt-in suggestions. AI should enhance user agency through lightweight guidance—like “here’s a good place to start”—without overriding control. Supporting well-being means protecting emotional privacy and reinforcing user autonomy.}





% \noindent\textbf{\revision{Prioritize trust through minimal tracking and ethical AI.}}
% \revision{A recurring concern was trust, particularly when AI or organizational structures were perceived as invasive. Participants were wary of AI recommendations that felt generic, systems that logged behavior, or workplace monitoring that framed well-being as a productivity tool. To earn trust, designers should clarify data boundaries, avoid opaque personalization, and resist integrating tracking as a default. Privacy-preserving mechanisms such as on-device processing or opt-in-only metrics can support personalization without compromising safety. Participants expressed interest in more personalized experiences (e.g., activity suggestions) but emphasized that such personalization must remain user-driven and transparent.}



% \todo{connect to previous section. better transition} The following implications are drawn from the design tensions surfaced in our findings. 

% \smallskip
% \noindent\textbf{1. Support spontaneous, short sessions.} The preference for frequent and unscheduled breaks over longer planned ones reflects a tension between workplace structure and individual autonomy. To accommodate this, VR tools should be designed for quick, on-demand access, without requiring booking systems, log-ins, or scheduled slots. Reducing procedural barriers makes it easier for users to reach for VR in moments of need, not obligation.

% \smallskip
% \noindent\textbf{2. Prioritise user agency in AI support.} AI should offer optional suggestions without prescribing paths. Lightweight scaffolding, such as reflective prompts or gentle recommendations, can support users without reducing autonomy. AI may personalize sound or environment, but should avoid behavioural tracking (unless the user opts in) or anything that implies surveillance.

% \smallskip
% \noindent\textbf{3. Enable movement between modes.} The trade-off between doing and being was not binary; users shifted between active exploration and passive rest depending on how they felt. Systems should not lock users into a specific activity type. Instead, they should allow fluid movement between guided and open-ended sessions, enabling users to find their own ways of engagement.

% \smallskip
% \noindent\textbf{4. Use familiar environments with subtle variation.} Familiar settings reduce cognitive effort and support relaxation, but can become repetitive. Stylised nature scenes with minor dynamic elements (like ambient sound or lighting) strike a balance. Generative variation can prevent fatigue without disrupting ease.

% \smallskip
% \noindent\textbf{5. Keep it personal, not performative.} Several concerns were raised about how VR might be framed or monitored in the workplace. The tension here lies between well-being as a personal act and as a tracked intervention. Systems should avoid linking usage to productivity or metrics. For VR to be trusted, it needs to feel private, voluntary, and disconnected from managerial oversight.
% \smallskip

% Together, these implications support a vision of workplace VR that prioritises autonomy, emotional comfort, and flexibility, without demanding users adapt to the system or conform to predefined uses.







%However, relying on familiarity alone may not be sufficient. While participants appreciated calming and recognizable environments, some raised concerns about repetition fatigue, particularly if the same scene was used repeatedly over time. This points to a tension: environments should not be overstimulating, yet if they remain static, they risk becoming uninspiring. Rather than treating novelty and familiarity as opposing forces, our design approach, using stylized but recognizable environments, seemed to offer a productive middle ground. It allowed users to feel emotionally anchored while still encountering something perceptually fresh. We argue that this notion of familiar difference, where core emotional tone remains consistent but surface-level features evolve, may be more suitable for workplace well-being than extremes of either realism or novelty.

%In well-being-focused VR, novelty can be both engaging and disruptive. Although often seen as a driver of interest, novelty may inadvertently introduce friction, especially in workplace contexts where cognitive fatigue is already high. Several participants valued not having to make decisions or adjust to new interfaces, emphasizing that too much stimulation could counteract the purpose of the break. Prior work has flagged this tension \cite{chirico_when_2019}, but few studies have examined supporting repeated, long-term use without relying on novelty as the primary hook. Our participants seemed to treat novelty as background texture rather than the main attraction, something that refreshes without requiring attention. Therefore, designers should focus less on innovation for its own sake and more on how experiences evolve over time while preserving a sense of emotional continuity. This has practical implications for integrating VR into daily work routines too. A calming environment might feel restorative on first use but may become stale or artificial if repeated without variation. Instead of relying on manual content updates or static asset libraries, future VR systems could use generative AI to introduce non-disruptive variations, such as shifting light, weather, sound, or small environmental details. The additions could also be based on the users' current mood and psychological arousal, supporting their momentary needs \cite{ng_chenrezig_2023}. These changes would maintain the emotional familiarity of a scene while keeping it perceptually fresh, similar to how open-ended games sustain engagement. Importantly, such changes should not require user decisions; the aim is to reduce mental effort, not add to it.



%Finally, we observed that while novelty was appreciated in moments of curiosity or playfulness, it was rarely the reason users stated would make them return to the system. Some participants described the ability to revisit environments that make them feel safe, such as the beach, repeatedly, as helpful for building a sense of emotional continuity. In contrast, others desired environments that would shift subtly to match their mood or energy levels. This reflects work on affective forecasting, which shows that people often mispredict what experiences will restore them \cite{wilson_affective_2005}. Rather than offering a blank canvas or tightly scripted journey, VR wellness tools could be more effective if they guide users towards environments that feel both familiar and meaningfully new, acknowledging the psychological function of repetition in creating experiences that feel safe while also recognizing the need for the environmental richness that supports return visits. In short, novelty should not be eliminated, but neither should it be the reason for returning. Instead, designers should consider how to shape experiences that feel consistently supportive while still offering enough newness to sustain engagement over time.




% The wider VR literature tends to focus on novelty in interaction as a driver of engagement, but in well-being contexts, novelty here too, can be a double-edged sword.  High levels of task switching and decision-making throughout the day can lead to cognitive fatigue, and introducing overly novel or interactive VR elements may inadvertently recreate the mental effort users are trying to escape.  Several participants mentioned that they valued not having to think too much during the experience.  Experiences that are too stimulating may undermine the goal of mental restoration, while overly static environments may lose their effectiveness over time. Prior work has flagged this tension \cite{chirico_when_2019}, yet few studies have addressed how to design for sustained use without leaning on novelty as the primary hook. Our participants seemed to navigate this by treating novelty as a backdrop rather than a feature, something that gently refreshes their experience, but does not demand attention, pointing to the need for environments that evolve subtly, maintaining a core sense of familiarity while avoiding repetition fatigue. This is particularly relevant when considering the long-term integration of VR tools into workplace routines. A nature scene that feels calming the first few times may begin to feel artificial or dull if revisited daily without variation. Rather than relying on manual content updates or static libraries, future systems could leverage generative AI to introduce small, non-disruptive variations, such as changes in time of day, weather, landscape, ambient sounds, or micro-interactions, similar to open ended games. Such diversity would preserve the emotional resonance of familiar settings while keeping the experience perceptually fresh. Importantly, such changes should not require user input or decision-making, as the goal remains reducing cognitive demand, not introducing new choices.
% While novelty is often seen as central to VR’s appeal—particularly in the context of well-being applications that aim to provide mental escape—our findings suggest this assumption deserves closer examination. Participants described initial excitement with the immersive environments, yet many also gravitated towards scenes that felt soothing rather than spectacular. The environments that supported familiarity—like forests or beaches—were not dismissed as mundane; instead, they were seen as reliable backdrops for moments of calm. This aligns with research in environmental psychology showing that familiarity with natural settings can reduce cognitive load and foster a sense of security (Korpela & Hartig, 1996), which may be particularly valuable in high-pressure work environments.

% The persistent focus on novelty in VR design for well-being overlooks how repeated exposure might serve different psychological functions over time. Novelty may boost initial engagement, but it does not necessarily promote restoration or routine use. Familiarity, in contrast, can enable users to quickly enter a mental state of relaxation without re-orienting themselves or adapting to unexpected elements. Designers aiming for long-term relevance in workplace wellness tools might do well to reconsider the emphasis on constant innovation in visuals or interactivity. Instead, they could create systems that evolve subtly or allow users to personalise their environment over time, building emotional resonance rather than relying on spectacle.

% Moreover, the desire for familiar environments speaks to a broader point about cognitive fatigue in knowledge work. Research on decision fatigue (Baumeister et al., 1998) suggests that reducing the need for micro-decisions can be a form of relief in itself. When VR experiences require users to explore, adapt, or interpret new information constantly, they risk replicating the mental effort users are trying to escape. In this sense, overemphasising novelty may inadvertently reinforce the very stressors such tools aim to alleviate. The implication here is not to reject novelty altogether, but to distinguish between novelty that stimulates curiosity and novelty that creates friction.

% Finally, we observed that some participants appreciated the option for novelty—as a form of playful exploration or mood enhancement—but preferred to return to known environments when they sought comfort. This pattern reflects research on affective forecasting, which shows that people are often poor judges of what experiences will make them feel better (Wilson & Gilbert, 2005). As such, VR wellness tools might benefit from gently guiding users based on past preferences or physiological cues, rather than offering a blank slate. Supporting repeatable, low-effort experiences that still feel personally meaningful could be more impactful than attempting to impress users with novelty alone.




% \subsection{Limitations and Future Directions} Our study has three main limitations. First, the sample size is small, with only 10 participants from a single research organization. This limits the generalizability of our findings to other professions or workplace contexts. Second, the study relied on qualitative methods and self-reported data. While this provides valuable insights into user perceptions, it does not capture the actual usability or effectiveness of VR wellness tools in practice. Third, the focus of this work was on perceptions rather than implementation. As a result, some design recommendations are speculative and would require further validation through prototyping and user testing. Future work should include larger and more diverse samples, iterative prototype development to test the feasibility of the recommendations, and longitudinal studies to evaluate the long-term effectiveness of VR wellness tools in workplace settings.

\section{Limitations and Future Work}
\revision{In Phase 1, we acknowledge two limitations. First, the sample size was small and included participants recruited through professional networks with prior experience in XR, gaming, or AI. Second, the tech-savvy sample may have been more receptive to VR, thus limiting the generalizability of the design requirements. Future work should include participants with varying levels of technological familiarity to understand broader needs and adoption barriers.}

% \revision{In Phase 2, we acknowledge five major limitations.
% First, the deployment was short-term, limited to one-off sessions in a specific workspace setting, and the participant sample was tech-savvy, which might not reflect broader populations.  Second, the tethered VR setup reduced portability compared to standalone headsets. Third, the app supported only low-effort movement and limited input styles, which might have affected appeal, and supported one activity per session, which might have disrupted the flow.
% Fourth, we prioritized practical application over technical novelty when designing the app. This was an intentional choice as the focus of our work was on exploring how existing immersive tools can be used in practical ways to support well-being at work.  Finally, the AI assistant provided static suggestions that might have limited personalization. Future studies should explore longer-term use with adaptive features, test the app in diverse work settings, and support less tech-confident users with smoother activity transitions.}

\revision{In Phase 2, we acknowledge five limitations. First, the short-term deployment in a specific workspace with relatively tech-savvy participants may limit generalizability. Second, the tethered VR setup reduced portability versus standalone headsets. Third, the app only supported low-effort movement, limited input, and one activity per session, which may have affected engagement. Fourth, we prioritized practical application over technical novelty, focusing on the use of existing immersive tools for workplace well-being. Finally, the AI assistant provided static suggestions, limiting personalization. Future work should explore longer-term, adaptive use in diverse work settings and support for less tech-confident users.}





% Third, while the app supported low-effort physical activity, it lacked full-body movement or varied input modes, which may limit appeal. Fourth, users could access only one activity per session (requiring menu navigation to switch), which may have disrupted the flow. Fifth, participants were largely tech-savvy, which may not generalize to broader user groups. Sixth, the AI assistant offered static suggestions without learning over time, thus limiting personalization. Finally, the system did not introduce new technological features or interaction techniques. This was an intentional choice: the focus of our work was not on technological novelty, but on exploring how existing immersive tools can be used in practical ways to support well-being at work. Future studies should explore longer-term deployments to understand sustained use, particularly with adaptive or context-aware systems. Testing in varied work settings with standalone devices would offer more realistic insights into everyday integration. Including less tech-confident users and enabling smoother transitions between activity types could help create more inclusive and flexible VR well-being tools.}
