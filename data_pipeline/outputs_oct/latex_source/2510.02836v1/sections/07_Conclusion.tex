\section{Conclusion}
\label{sec:conclusion}
% Our work explored how VR can support workplace well-being through Tranquil Loom, a VR app designed for short, self-directed breaks. Through a two-phase mixed-methods study, we found that even brief VR sessions can positively impact mindfulness and anxiety while offering users a surprising sense of agency and playfulness, particularly through open-ended exploration. Our participants embraced VR as a spontaneous tool to navigate fluctuating states of well-being during the workday. Our  findings revealed design tensions between AI-driven personalization and autonomy, as well as structure and openness. Well-being in VR is not only about what users do but how freely
% they are allowed to do it. Future work should explore how adaptive VR experiences might fluidly respond to changing user contexts without overriding user intent. 


We explored how VR can support workplace well-being through Tranquil Loom, which is a VR app for short and self-directed breaks. Even brief VR use improved mindfulness, reduced anxiety, and promoted agency and playfulness. However, design tensions emerged between personalization and autonomy, and between structure and openness. Future work should explore adaptive VR that responds to users' changing needs without overriding intent.

