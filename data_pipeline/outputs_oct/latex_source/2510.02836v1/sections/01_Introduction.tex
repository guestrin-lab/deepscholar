Virtual Reality (VR) is increasingly being explored as a tool for supporting well-being across a range of contexts, including the workplace \cite{Riches2023Virtual}. Its immersive nature allows users to temporarily step away from their immediate environment~\cite{slater_enhancing_2016} by offering a form of instant detachment, especially for those in mentally intensive and sedentary roles. Prior research has demonstrated the potential of VR in promoting mental restoration ~\cite{ma_effectiveness_2023}, reducing stress ~\cite{xu_effectiveness_2024}, and supporting light physical activity ~\cite{kim_effect_2021, yoo_embedding_2020}.  Additionally, there is empirical evidence that VR apps such as nature-based relaxation environments or guided meditations can promote mental clarity~\cite{ladakis2024virtual}.  As organizations pay greater attention to employee well-being, immersive technologies have emerged as potential tools for helping people disconnect temporarily from busy or screen-heavy environments, re-center during stressful moments, or re-energize between tasks~\cite{Riches2023Virtual, wagener_role_2021}. In workspace settings, VR can offer an immediate shift in environment without requiring employees to leave their workspace, which may be particularly useful for those with limited time or flexibility during the workday.

This is especially relevant for knowledge workers working in office settings. Typically, knowledge workers deal with processing, analyzing, and managing information~\cite{Surawski2019Who}. As central drivers of productivity in many sectors~\cite{Greene2011Space}, they frequently navigate cognitive overload, long periods of physical inactivity, and blurred boundaries between work and life~\cite{wepfer_work-life_2018, soto_observing_2021}. Such demands can affect mental focus, physical comfort, and overall well-being~\cite{wepfer_work-life_2018}. While organizations offer interventions including flexible schedules or wellness programs~\cite{teevan_new_2021, chow_feeling_2024}, these do not always align with how people actually manage their time and energy during the day. At the same time, many knowledge workers are comfortable experimenting with digital tools to support their productivity and focus~\cite{guillou_is_2020}. For example, a recent report by Microsoft and LinkedIn found that 75\% of knowledge workers already use AI in their roles, with most reporting increased efficiency and creativity~\cite{MicrosoftLinkedIn2024}.  

Openness to innovation presents an opportunity for wellness interventions through technology. One such technology is VR. Yet, current VR tools for the workplace often adopt a solution-focused mindset, offering single-purpose experiences such as guided meditation or virtual nature walks \cite{adhyaru2022virtual, Riches2023Virtual}. They rarely account for the complexity of real-world work where needs vary widely across individuals and time. Broader literature on VR for well-being highlights the need for more flexible and personalized approaches \cite{wagener_role_2021}. Still, little research has explored what this might mean in the specific context of everyday knowledge work. Different workers may need different forms of support at different times, whether it is movement to relieve tension, a quiet space to reset, or playful exploration to re-energize. A tool that assumes a single definition of well-being or imposes a fixed routine is unlikely to support a wide range of users in meaningful ways. Moreover, the question of how such tools could be integrated into the rhythms and constraints of working life remains largely underexplored~\cite{Riches2023Virtual}.

To address this gap, our work investigates how VR might support the well-being of knowledge workers in ways that are responsive to their everyday needs. Rather than developing new interaction modes or a novel technical app, our focus is on investigating how familiar well-being practices (i.e., stretching, meditation, and exploratory interaction) can be integrated into a single tool in ways that feel responsive to the pace and structure of everyday work. We position our work as an exploratory design study that surfaces the trade-offs and constraints of integrating diverse activity modes in VR, and reflect on how existing interventions can be adapted and situated in office settings. Rather than introducing new technical designs, we aim to surface lived user experiences and practical design implications for real-world workplace well-being.

To achieve this, we carried out a two-phase mixed-methods study. In Phase 1, we conducted semi-structured interviews with 10 knowledge workers, during which we explored how participants manage their well-being during the day, their expectations and concerns around using VR in the workplace, and the types of experiences they believed would be helpful. Their insights informed the design of \textit{Tranquil Loom}, a VR app offering short guided stretching exercises, meditation, and open-ended exploration across four calming environments (Figure \ref{fig:teaser}). Tranquil Loom was designed to support different types of well-being needs and allow users to choose how to engage, depending on how they felt at the time, also supported with suggestions by an AI assistant. In Phase 2, we deployed the app in an office workplace setting with 35 participants. We combined usage data, pre- and post-intervention well-being measures, and follow-up interviews to examine how participants used the app, what they found helpful or unhelpful, and how VR might fit into the flow of a working day (Figure \ref{fig:methodology}). Our aim was to understand how VR might be designed and integrated in ways that reflect the challenges of everyday work. To achieve this goal, our work is guided by three research questions (RQs): 

\begin{enumerate}[label=\textbf{RQ\textsubscript{\arabic*}:}, align=left, leftmargin=2.2em, labelsep=0.5em, itemsep=0.8em]
    \vspace{-0.2cm} \item Which aspects of workplace well-being can VR address?
    \vspace{-0.2cm} \item How can VR experiences be tailored to reflect the varied personal and organizational needs of knowledge workers?
    \vspace{-0.2cm} \item Which design elements and functionalities are most effective for supporting knowledge workers' well-being within VR?
\end{enumerate}

In answering our RQs, we made three main contributions:

\begin{enumerate}
    \vspace{-0.2cm} \item Through semi-structured interviews with knowledge workers (\S\ref{sec:design}), we identified six design requirements for VR apps that account for the diverse demands of cognitive, emotional, and physical well-being in real-world work settings.

     \vspace{-0.2cm} \item With these requirements at hand, we designed and deployed a VR well-being app called \emph{Tranquil Loom} (\S\ref{sec:app}). The app featured three types of well-being practices (i.e., stretching, meditation, and exploration) across four environments. 
     
    \vspace{-0.2cm} \item A mixed-methods evaluation highlighting the design tensions and opportunities in VR-based workplace well-being (\S\ref{sec:evaluation}). We found that workers embraced VR as a situational ``drop-in'' tool rather than a scheduled routine and valued playful and self-directed engagement over AI-guided suggestions.


\end{enumerate}

With our work, we shift the focus from evaluating VR well-being apps as fixed interventions to designing for the complexity of real-world needs. We identified key design trade-offs such as between structure \emph{vs.} openness, doing \emph{vs.} being, and AI guidance \emph{vs.} autonomy (\S\ref{sec:discussion}), and discussed design implications for emotionally responsive and ethically grounded tools that support trust and spontaneous use in the workplace (\S\ref{sec:design-implications}).




