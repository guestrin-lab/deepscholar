\begin{abstract}

In egocentric applications such as augmented and virtual reality, immersive iris recognition is emerging as an accurate and seamless way to identify persons. While classic systems acquire iris images on-axis, \ie, via dedicated frontal sensors in controlled settings, the immersive setup primarily captures off-axis irises through tilt-placed headset cameras, with only mild control in open scenes. This yields unique challenges, including perspective distortion, intensified quality degradations, and intra-class variations in iris texture. Datasets capturing these challenges remain scarce. To fill this gap, this paper introduces ImmerIris, a large-scale dataset collected via VR headsets, containing 499,791 ocular images from 564 subjects. It is, to the best of current knowledge, the largest public dataset and among the first dedicated to off-axis acquisition. Based on ImmerIris, evaluation protocols are constructed to benchmark recognition methods under different challenging factors. Current methods, primarily designed for classic on-axis imagery, perform unsatisfactorily on the immersive setup, mainly due to reliance on fallible normalization. To this end, this paper further proposes a normalization-free paradigm that directly learns from ocular images with minimal adjustment. Despite its simplicity, this approach consistently outperforms normalization-based counterparts, pointing to a promising direction for robust immersive recognition.


% In egocentric applications, iris recognition using immersive VR/AR headsets is emerging as an accurate and seamless way to identify persons. However, it faces unique challenges beyond classical controlled setups. Images acquired in open scenes exhibit inherent off-axis distortion, quality degradation, and large intra-class variations, while datasets capturing these challenges remain scarce. To address this gap, this paper introduces ImmerIris, a large-scale dataset collected via VR headsets, containing 564 subjects and 499,791 ocular images. It is, to the best of current knowledge, the largest public IR dataset and among the first dedicated to off-axis acquisition. Benchmarks on ImmerIris reveal that existing methods perform unsatisfactorily in immersive scenarios due to reliance on fallible normalization. A normalization-free paradigm is therefore proposed, which directly learns from ocular images with minimal adjustment. Despite its simplicity, this approach consistently outperforms normalization-based counterparts, pointing to a promising direction for robust immersive IR.

\vspace{-3mm}
\end{abstract}