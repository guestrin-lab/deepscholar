\section{Benchmarks}
\label{sec:benchmark}

% Please add the following required packages to your document preamble:
% \usepackage{multirow}
% \usepackage[table,xcdraw]{xcolor}
% Beamer presentation requires \usepackage{colortbl} instead of \usepackage[table,xcdraw]{xcolor}
\begin{table}[tbp]
\footnotesize
\centering
\begin{tabular}{clccc}
\toprule
                                    & \textbf{Method}                                               & \multicolumn{3}{c}{\textbf{Performance}}                                                         \\
\multirow{-2}{*}{\makecell[c]{\textbf{Test}\\\textbf{Protocol}}} & FRR@FAR (\%) $\downarrow$                                     & 1e-1                             & 1e-3                             & 1e-5                             \\
\midrule
\multirow{7}{*}{\makecell[l]{\textbf{(a)}\\\textbf{CASIA-T}}} & Gabor~\cite{daugman2009iris}            & 0.36                         & 1.03                         & 5.24                         \\
                                    & OM~\cite{sun2008ordinal}                & 0.24                         & 1.76                         & 5.14                         \\
                                    & Maxout~\cite{zhang2018deep}             & 1.82                         & 17.93                        & 47.26                        \\
                                    & Maxout-BA~\cite{wei2022towards}         & 2.14                         & 21.49                        & 50.38                        \\
                                    & UE-UGCL~\cite{wei2022towards}           & 1.29                         & 11.51                        & 35.20                        \\
                                    & CM~\cite{wei2022contextual}             & 1.50                         & 14.71                        & 38.76                        \\
                                    & ComplexIrisNet~\cite{nguyen2022complex} & 1.08                         & 13.74                        & 35.79                        \\
                                    % & IR-Norm                                                       & 0.50                         & 2.59                         & 6.41                         \\
                                    % & IR-BBox (ours)                                                & 0.37                         & 3.99                         & 14.75                        \\
\midrule
\multirow{7}{*}{\makecell[l]{\textbf{(b)}\\\textbf{Immer-Any}}} & Gabor~\cite{daugman2009iris}            & 32.12                        & 64.33                        & 85.47                        \\
                                    & OM~\cite{sun2008ordinal}                & 30.85                        & 72.18                        & 88.48                        \\
                                    & Maxout~\cite{zhang2018deep}             & 38.83 & 83.61 & 94.09 \\
                                    & Maxout-BA~\cite{wei2022towards}         & 36.43 & 78.73 & 91.94 \\
                                    & UE-UGCL~\cite{wei2022towards}           & 34.62 & 79.02 & 92.42 \\
                                    & CM~\cite{wei2022contextual}             & 38.68 & 78.63 & 90.90 \\
                                    & ComplexIrisNet~\cite{nguyen2022complex} & 42.25                        & 81.07                        & 93.14                        \\
                                    % & IR-Norm                                                       & 32.88                        & 76.80                        & 91.14                        \\
                                    % & IR-BBox (ours)                                                & 44.85                        & 78.85                        & 89.73                       \\
\bottomrule
\end{tabular}
\caption{Verification FRR@FAR ($\downarrow$) of SOTAs trained on CASIA-T and tested on (a) CASIA-T and (b) Immer-Any. Results are averaged over left and right eyes due to space constraints.}
\label{tab:exp-onaxis}
\vspace{-3mm}
\end{table} 

We present a comprehensive benchmark of SOTAs and our method on the ImmerIris dataset. \textbf{In short, we find:} 1) ImmerIris reveals intrinsic challenges of immersive recognition that differ from controlled setups; 2) SOTAs transfer poorly to the immersive setting, hence methodological advances are necessary; 3) normalization-free paradigm delivers robust performance and points to a promising direction.

\subsection{Experimental Setup}
\label{subsec:exp-setup}


\begin{table*}[tbp]
\centering
\footnotesize
\begin{tabular}{ll|ccc|ccc|ccc|ccc}
\toprule
\multirow{2}{*}{\textbf{Eye}} & \textbf{Method}           & \multicolumn{3}{c|}{\textbf{Immer-Control}} & \multicolumn{3}{c|}{\textbf{Immer-Fix}} & \multicolumn{3}{c|}{\textbf{Immer-Select}} & \multicolumn{3}{c}{\textbf{Immer-Any}} \\
                              & FRR@FAR (\%) $\downarrow$ & 1e-1           & 1e-3           & 1e-5           & 1e-1          & 1e-3          & 1e-5         & 1e-1           & 1e-3           & 1e-5          & 1e-1          & 1e-3          & 1e-5         \\
\midrule
\multirow{9}{*}{Left}         & Gabor~\cite{daugman2009iris}                     & 3.26       & 11.06      & 19.14      & 8.33      & 22.44     & 44.12    & 22.22      & 53.43      & 87.66     & 30.75     & 62.60     & 83.20    \\
                              & OM~\cite{sun2008ordinal}                        & 3.09       & 18.10      & 39.16      & 8.09      & 31.28     & 53.37    & 19.67      & 63.36      & 81.77     & 29.51     & 73.13     & 87.81    \\
                              & Maxout~\cite{zhang2018deep}                    & 0.74       & 8.82       & 20.95      & 3.18      & 18.61     & 34.13    & 7.48       & 45.03      & 74.33     & 9.54      & 51.48     & 77.32    \\
                              & Maxout-BA~\cite{wei2022towards}                 & 0.43       & 6.90       & 19.58      & 2.42      & 16.25     & 34.84    & 5.33       & 40.49      & 72.06     & 7.06      & 46.65     & 75.67    \\
                              & UE-UGCL~\cite{wei2022towards}                   & 0.23       & 3.17       & 10.63      & 1.38      & 10.65     & 24.36    & 3.04       & 25.61      & 56.32     & 3.84      & 31.77     & 60.57    \\
                              & CM~\cite{wei2022contextual}                        & \textbf{0.17}       & 1.93       & 7.18       & 1.18      & \underline{7.73}      & 18.35    & 2.52       & \textbf{17.52}      & \textbf{45.03}     & \underline{3.35}      & \textbf{23.39}     & \textbf{49.93}    \\
                              & ComplexIrisNet~\cite{nguyen2022complex}            & \underline{0.19}       & 2.07       & 7.32       & \underline{0.98}      & 7.97      & \underline{19.73}    & \underline{2.23}       & \underline{19.65}      & 49.13     & 3.63      & 27.81     & 57.62    \\
                              & IR-Norm                 & 0.25       & \underline{1.70}       & \underline{6.41}       & 1.92      & 9.49      & 19.77    & 4.33       & 23.48      & 49.20     & 5.48      & 28.95     & 56.63    \\
                              & \textbf{IR-BBox} (ours)          & 0.21       & \textbf{1.59}       & \textbf{5.50}       & \textbf{0.83}      & \textbf{6.28}      & \textbf{15.22}    & \textbf{2.17}       & 20.99      & \underline{47.96}     & \textbf{2.36}      & \underline{24.03}     & \underline{52.04}    \\
\midrule
\multirow{9}{*}{Right}        & Gabor~\cite{daugman2009iris}                     & 3.75       & 12.86      & 25.84      & 9.62      & 26.46     & 51.29    & 24.09      & 55.72      & 84.09     & 33.49     & 66.05     & 87.74    \\
                              & OM~\cite{sun2008ordinal}                        & 4.74       & 19.77      & 37.26      & 10.21     & 32.30     & 52.64    & 22.38      & 61.40      & 81.17     & 32.19     & 71.23     & 89.14    \\
                              & Maxout~\cite{zhang2018deep}                    & 1.25       & 11.99      & 37.21      & 3.65      & 21.52     & 39.75    & 8.30       & 47.35      & 79.66     & 12.49     & 54.94     & 80.53    \\
                              & Maxout-BA~\cite{wei2022towards}                 & 0.65       & 9.71       & 33.70      & 2.54      & 18.67     & 37.35    & 6.14       & 42.81      & 76.36     & 9.53      & 50.07     & 76.92    \\
                              & UE-UGCL~\cite{wei2022towards}                   & 0.15       & 3.83       & 14.43      & 1.17      & 10.98     & 26.91    & 2.75       & 26.89      & 59.84     & 4.38      & 33.55     & 62.80    \\
                              & CM~\cite{wei2022contextual}                        & 0.12       & 2.65       & 10.39      & 1.14      & \underline{9.07}      & \underline{20.02}    & 2.06       & \underline{18.79}      & \underline{42.69}     & \underline{3.60}      & \underline{26.37}     & \underline{50.11}    \\
                              & ComplexIrisNet~\cite{nguyen2022complex}            & \textbf{0.10}       & 2.47       & 8.55       & \underline{1.01}      & 9.67      & 23.86    & \underline{1.93}       & 20.47      & 48.14     & 3.97      & 32.24     & 60.72    \\
                              & IR-Norm                 & 0.42       & \underline{2.15}       & 8.26       & 2.34      & 10.60     & 20.75    & 4.72       & 22.78      & 46.19     & 6.20      & 30.48     & 53.82    \\
                              & \textbf{IR-BBox} (ours)          & \underline{0.11}       & \textbf{1.69}       & \textbf{4.93}       & \textbf{0.74}      & \textbf{6.41}      & \textbf{15.02}    & \textbf{1.75}       & \textbf{17.21}      & \textbf{40.64}     & \textbf{2.28}      & \textbf{23.86}     & \textbf{49.56}    \\
\midrule
\multirow{9}{*}{Dual}         & Gabor~\cite{daugman2009iris}                     & 4.66       & 14.05      & 23.17      & 7.12      & 19.62     & 31.09    & -            & -            & -           & 28.50     & 58.60     & 73.80    \\
                              & OM~\cite{sun2008ordinal}                        & 4.67       & 20.06      & 33.89      & 6.83      & 25.31     & 41.04    & -            & -            & -           & 25.50     & 63.81     & 80.28    \\
                              & Maxout~\cite{zhang2018deep}                    & 0.85       & 9.22       & 22.17      & 1.69      & 13.22     & 28.44    & -            & -            & -           & 6.38      & 40.23     & 73.14    \\
                              & Maxout-BA~\cite{wei2022towards}                 & 0.53       & 6.69       & 19.08      & 1.17      & 10.49     & 24.01    & -            & -            & -           & 4.27      & 34.30     & 66.39    \\
                              & UE-UGCL~\cite{wei2022towards}                   & 0.26       & 2.81       & 10.14      & 0.61      & 5.41      & 16.18    & -            & -            & -           & 2.02      & 19.65     & 48.01    \\
                              & CM~\cite{wei2022contextual}                        & 0.24       & 2.53       & 7.81       & 0.59      & 4.83      & \underline{12.23}    & -            & -            & -           & 1.84      & \underline{16.60}     & \textbf{38.44}    \\
                              & ComplexIrisNet~\cite{nguyen2022complex}            & \underline{0.21}       & \underline{2.18}       & 7.57       & \underline{0.50}      & \underline{4.37}      & 12.81    & -            & -            & -           & \underline{1.65}      & 18.42     & 41.71    \\
                              & IR-Norm                 & 0.48       & 2.24       & \underline{5.65}       & 1.44      & 6.85      & 14.14    & -            & -            & -           & 3.85      & 20.95     & 44.69    \\
                              & \textbf{IR-BBox} (ours)          & \textbf{0.18}       & \textbf{1.16}       & \textbf{5.17}       & \textbf{0.48}      & \textbf{3.17}      & \textbf{10.61}    & -            & -            & -           & \textbf{1.24}      & \textbf{13.29}     & \underline{40.45}   \\
\bottomrule
\end{tabular}
\caption{Verification FRR@FAR ($\downarrow$) of SOTAs and the proposed method on general test protocols of increasing difficulty. \textbf{Bold} and \underline{underline} indicate the best and second-best results, respectively; hereafter the same.}
\label{tab:exp-1v1-general}
\end{table*}

% \noindent \textbf{Compared Methods.} We evaluate 7 normalization-based SOTAs—two training-free (Gabor~\cite{daugman2009iris}, OM~\cite{sun2008ordinal}) and 5 learning-based (Maxout~\cite{zhang2018deep}, Maxout-BA, UE-UGCL, CM, ComplexIrisNet)—against our normalization-free baseline IR-BBox. An ablation, IR-Norm, replaces the bounding box with normalized irises to isolate the effect of waiving normalization.

% \noindent \textbf{Datasets and  metrics.} Experiments are conducted on CASIA-IrisV4-Thousand (CASIA-T, 20K images, 1000 subjects; 700/300 split) as the controlled setup, and on the proposed ImmerIris as the immersive setup.  For verification we report FRR@FAR (1e-1/1e-3/1e-5), and for identification rank-1 accuracy. In dual-eye protocols, a probe is accepted only if both eyes succeed. We leave detailed setups to the supplementary material.

\noindent \textbf{Compared SOTAs.} We compare 7 normalization-based SOTAs with our normalization-free baseline, IR-BBox. These include 2 training-free methods using hand-crafted filters, Gabor~\cite{daugman2009iris} and OM~\cite{sun2008ordinal}, and 5 learning-based that extract identity templates using DNNs, Maxout~\cite{zhang2018deep}, Maxout-BA~\cite{wei2022towards}, UE-UGCL~\cite{wei2022towards}, CM~\cite{wei2022contextual}, and ComplexIrisNet~\cite{nguyen2022complex}. We further ablate IR-BBox by replacing the bounding box with normalized iris, referred to as IR-Norm, to show the stand-alone effect of waiving normalization.

\noindent \textbf{Datasets.} We compare 2 setups. For the controlled setup, used as a reference, we employ CASIA-IrisV4-Thousand~\cite{casia-iris-v4} (CASIA-T), which contains 20K images from 1000 subjects. We split it into 700 subjects for training and 300 for open-set testing, and uniformly sample genuine and imposter pairs in a manner comparable to Immer-Any. For the immersive setup, we use the proposed ImmerIris.

\noindent \textbf{Metrics.} We report FRR@FAR at 1e-1/1e-3/1e-5, \ie the false rejection rate at these false acceptance rate thresholds, for iris verification, and rank-1 accuracy for iris identification. For dual-eye protocols, a probe is accepted only when both eyes are accepted or ranked first.

 \noindent \textbf{Implementations.} For fair comparison, we rerun the official codes of Maxout, Maxout-BA, UE-UGCL, and CM on our training and test datasets. For methods without available code~\cite{daugman2009iris,sun2008ordinal,nguyen2022complex}, we re-implement them to the best of our effort while acknowledging potential inconsistencies. All SOTAs are trained according to their recommended settings and parameters. Iris normalization is performed using an open-source repository\footnote{https://github.com/worldcoin/open-iris}, which produces normalized irises at a resolution of 64$\times$512. We resize them to the input shape preferred by each SOTA. IR-BBox and IR-Norm are implemented via an FR framework\footnote{https://github.com/Tencent/TFace}, using an IR-50~\cite{he2016deep} backbone, an input size of 112$\times$112 and trained with SGD for 32 epochs. Alternative normalization method and model architecture are compared later in~\cref{subsec:exp-abal}.

\subsection{Performance under Controlled Setup}
\label{subsec:exp-controlled}

\begin{table*}[tbp]
\centering
\footnotesize
\begin{tabular}{ll|ccc|ccc|ccc|ccc}
\toprule
\multirow{2}{*}{\textbf{Eye}} & \textbf{Method}           & \multicolumn{3}{c|}{\textbf{Immer-Occlusion}} & \multicolumn{3}{c|}{\textbf{Immer-Dilation}} & \multicolumn{3}{c|}{\textbf{Immer-Light}} & \multicolumn{3}{c}{\textbf{Immer-Angle}} \\
                              & FRR@FAR (\%) $\downarrow$ & 1e-1            & 1e-3            & 1e-5           & 1e-1           & 1e-3            & 1e-5           & 1e-1             & 1e-3             & 1e-5            & 1e-1           & 1e-3          & 1e-5          \\
\midrule
\multirow{9}{*}{Left}         & Gabor~\cite{daugman2009iris}                     & 12.63       & 30.92       & 43.16      & 3.33       & 11.98       & 21.19      & 6.88         & 20.90        & 37.94       & 22.50      & 53.62     & 73.12     \\
                              & OM~\cite{sun2008ordinal}                        & 8.63        & 28.45       & 45.46      & 4.62       & 16.80       & 24.23      & 7.63         & 36.12        & 65.99       & 21.23      & 68.97     & 88.93     \\
                              & Maxout~\cite{zhang2018deep}                    & 2.45        & 22.90       & 37.54      & 0.93       & 9.34        & 24.23      & 1.69         & 18.09        & 41.86       & 5.10       & 45.20     & 75.98     \\
                              & Maxout-BA~\cite{wei2022towards}                 & 1.68        & 20.70       & 35.07      & 1.02       & 9.00        & 21.56      & 1.28         & 15.30        & 39.74       & 2.98       & 39.14     & 73.71     \\
                              & UE-UGCL~\cite{wei2022towards}                   & 0.63        & 13.33       & 27.66      & 0.32       & 4.14        & 11.71      & 0.50         & 6.40         & 18.67       & 1.28       & 22.13     & 55.52     \\
                              & CM~\cite{wei2022contextual}                        & 0.50        & 7.31        & 16.59      & 0.08       & 6.59        & 12.10      & 0.34         & 5.29         & \underline{16.62}       & \textbf{0.93}       & \textbf{13.25}     & \textbf{38.92}     \\
                              & ComplexIrisNet~\cite{nguyen2022complex}            & 0.58        & 11.67       & 29.58      & \underline{0.04}       & \underline{3.34}        & \underline{8.22}       & \underline{0.20}         & \underline{4.78}         & 21.62       & 1.07       & \underline{15.31}     & \underline{40.31}     \\
                              & IR-Norm                 & \underline{0.48}        & \underline{5.01}        & \underline{12.35}      & 0.13       & 5.01        & 13.66      & 0.61         & 7.30         & 20.82       & 1.31       & 16.30     & 46.63     \\
                              & \textbf{IR-BBox} (ours)          & \textbf{0.38}        & \textbf{2.51}        & \textbf{8.23}       & \textbf{0.02}       & \textbf{2.26}        & \textbf{4.53}       & \textbf{0.42}         & \textbf{2.02}         & \textbf{9.79}        & \underline{0.97}       & 16.40     & 45.04     \\
\midrule
\multirow{9}{*}{Right}        & Gabor~\cite{daugman2009iris}                     & 10.74       & 28.40       & 42.55      & 8.87       & 25.46       & 37.90      & 12.27        & 33.00        & 55.17       & 23.61      & 53.64     & 71.72     \\
                              & OM~\cite{sun2008ordinal}                        & 8.50        & 27.47       & 46.61      & 8.62       & 29.60       & 43.44      & 9.44         & 42.12        & 63.49       & 24.95      & 66.18     & 88.05     \\
                              & Maxout~\cite{zhang2018deep}                    & 3.06        & 22.03       & 33.84      & 4.64       & 20.26       & 38.69      & 3.58         & 23.71        & 40.88       & 7.55       & 48.21     & 79.96     \\
                              & Maxout-BA~\cite{wei2022towards}                 & 1.76        & 19.88       & 30.80      & 3.31       & 17.27       & 25.31      & 2.99         & 22.88        & 40.88       & 5.31       & 42.32     & 74.85     \\
                              & UE-UGCL~\cite{wei2022towards}                   & 0.43        & 11.32       & 26.38      & 1.00       & 12.52       & 22.45      & 1.61         & 15.49        & 33.70       & 1.75       & 22.76     & 56.77     \\
                              & CM~\cite{wei2022contextual}                        & \underline{0.42}        & 7.44        & 19.11      & 0.88       & \underline{8.74}        & \underline{17.09}      & 1.13         & 12.36        & 27.82       & 1.01       & \textbf{15.32}     & 40.52     \\
                              & ComplexIrisNet~\cite{nguyen2022complex}            & 0.46        & 12.26       & 27.71      & \textbf{0.21}       & 9.59        & 29.43      & \underline{0.95}         & \underline{10.50}        & \underline{26.39}       & \underline{0.97}       & 16.93     & \textbf{40.25}     \\
                              & IR-Norm                 & 0.66        & \underline{5.31}        & \underline{12.43}      & 2.19       & 14.31       & 30.47      & 4.44         & 22.40        & 38.27       & 1.79       & \underline{16.46}     & 42.81     \\
                              & \textbf{IR-BBox} (ours)          & \textbf{0.26}        & \textbf{2.24}        & \textbf{5.52}       & \underline{0.33}       & \textbf{8.73}        & \textbf{10.85}      & \textbf{0.49}         & \textbf{9.86}         & \textbf{21.45}       & \textbf{0.76}       & 17.74     & \underline{40.41}     \\
\midrule
\multirow{9}{*}{Dual}         & Gabor~\cite{daugman2009iris}                     & 9.07        & 26.02       & 39.54      & 6.77       & 20.77       & 30.55      & 8.87         & 25.90        & 41.84       & 17.77      & 44.61     & 63.70     \\
                              & OM~\cite{sun2008ordinal}                        & 6.48        & 25.66       & 39.21      & 5.45       & 28.25       & 42.09      & 6.65         & 39.03        & 62.44       & 16.94      & 52.04     & 73.66     \\
                              & Maxout~\cite{zhang2018deep}                    & 1.29        & 15.75       & 35.30      & 1.41       & 13.17       & 24.45      & 1.25         & 12.43        & 27.52       & 3.08       & 31.32     & 60.07     \\
                              & Maxout-BA~\cite{wei2022towards}                 & 0.78        & 12.95       & 26.76      & 0.54       & 12.60       & 19.77      & 0.53         & 11.06        & 27.83       & 1.42       & 23.55     & 54.12     \\
                              & UE-UGCL~\cite{wei2022towards}                   & \textbf{0.27}        & 5.10        & 17.22      & 0.27       & 3.27        & 6.53       & \textbf{0.39}         & 5.89         & 16.55       & 0.45       & 9.32      & 34.44     \\
                              & CM~\cite{wei2022contextual}                        & \underline{0.29}        & \underline{3.77}        & \underline{10.53}      & \textbf{0.12}       & \underline{2.62}        & \underline{5.87}       & 0.53         & \underline{4.33}         & 14.99       & \underline{0.33}       & 7.25      & 25.02     \\
                              & ComplexIrisNet~\cite{nguyen2022complex}            & 0.33        & 6.12        & 18.59      & \underline{0.16}       & 3.73        & 13.71      & 0.53         & 4.68         & \underline{12.87}       & 0.43       & 7.06      & \textbf{20.79}     \\
                              & IR-Norm                 & 0.67        & 4.99        & 11.73      & 0.54       & 5.14        & 8.13       & 1.57         & 9.43         & 21.35       & 0.91       & \underline{6.49}      & \underline{21.13}     \\
                              & \textbf{IR-BBox} (ours)          & 0.31        & \textbf{1.63}        & \textbf{3.92}       & 0.28       & \textbf{1.02}        & \textbf{1.87}       & \underline{0.44}         & \textbf{1.99}         & \textbf{6.12}        & \textbf{0.32}       & \textbf{6.41}      & 23.39    \\
\bottomrule
\end{tabular}
\caption{Verification FRR@FAR ($\downarrow$) of SOTAs and the proposed method on factor-specific test protocols.}
\label{tab:exp-1v1-factor}
\end{table*}

We begin by evaluating SOTAs on the controlled setup, their default operating scenario, to establish a reference for later comparison. Specifically, we train learning-based SOTAs on the training set of CASIA-T and evaluate them together with training-free SOTAs on CASIA-T's test protocol. We report verification FRR@FAR in~\cref{tab:exp-onaxis}(a). Results indicate that Gabor and OM are highly robust, whereas learning-based SOTAs yield lower but still acceptable performance. We note that this drop is attributable not to defects of the methods, but largely to the limited training data volume. Prior literature~\cite{zhang2018deep,wei2022towards,wei2022contextual,nguyen2022complex} shows that, these learning-based SOTAs can surpass Gabor and OM when trained on sufficiently large, though publicly unavailable, datasets. Overall, SOTAs remain effective under the controlled setup.

\subsection{Divergence of Immersive Data} 
\label{subsec:exp-divergence}

We further test SOTA models trained on CASIA-T on the Immer-Any protocol of ImmerIris to examine whether they can directly generalize to immersive scenarios. The purpose is to validate the divergence between controlled and immersive iris data. Note that our employed normalization method can produce normalized irises for both on-axis and off-axis data by design, so the inputs are nominally aligned. Any performance change can therefore be primarily attributed to the inherent domain gap between datasets. As shown in~\cref{tab:exp-onaxis}(b), the performance of all SOTAs deteriorates to the point of being barely usable. This reveals a significant domain gap between iris data captured under controlled and immersive setups that cannot be mitigated through normalization. These findings suggest that our proposed ImmerIris dataset introduces a distinct and novel scenario compared to existing controlled datasets.  

\subsection{General ImmerIris Verification Performance}
\label{subsec:exp-1v1-general}

% Please add the following required packages to your document preamble:
% \usepackage{multirow}
\begin{table*}[tbp]
\centering
\footnotesize
\begin{tabular}{ll|YYYY|YYY}
\toprule
\textbf{Eye}           & \textbf{Method}  & \textbf{Control} & \textbf{Fix} & \textbf{Select} & \textbf{Any} & \textbf{Occlusion} & \textbf{Light} & \textbf{Angle} \\
\midrule
\multirow{9}{*}{Left}  & Gabor~\cite{daugman2009iris}           & 91.99          & 86.44      & 48.52         & 49.40      & 65.72            & 65.57        & 50.01        \\
                       & OM~\cite{sun2008ordinal}              & 85.55          & 79.04      & 45.15         & 44.79      & 45.58            & 62.30        & 47.16        \\
                       & Maxout~\cite{zhang2018deep}          & 97.70          & 93.97      & 76.97         & 76.12      & 77.39            & 81.97        & 79.96        \\
                       & Maxout-BA~\cite{wei2022towards}       & 98.74          & 95.79      & 82.18         & 81.78      & 89.40            & 88.52        & 85.75        \\
                       & UE-UGCL~\cite{wei2022towards}         & 99.35          & 97.27      & 89.92         & 88.83      & 93.64            & 96.72        & 93.25        \\
                       & CM~\cite{wei2022contextual}              & 99.59          & 98.19      & \underline{93.59}         & \underline{92.65}      & \underline{95.41}            & 95.08        & \textbf{96.14}        \\
                       & ComplexIrisNet~\cite{nguyen2022complex}  & \underline{99.67}          & 97.89      & 92.52         & 91.14      & 93.64            & 96.72        & 95.18        \\
                       & IR-Norm        & \textbf{99.76}          & \underline{98.26}      & 92.28         & 91.99      & 95.41            & \underline{98.36}        & 94.75        \\
                       & \textbf{IR-BBox} (ours) & 99.52          & \textbf{98.91}      & \textbf{93.87}         & \textbf{94.39}      & \textbf{98.23}            & \textbf{98.36}        & \underline{95.49}        \\
\midrule
\multirow{9}{*}{Right} & Gabor~\cite{daugman2009iris}           & 86.80          & 80.41      & 45.10         & 45.63      & 71.43            & 59.21        & 46.64        \\
                       & OM~\cite{sun2008ordinal}              & 84.01          & 77.20      & 44.06         & 44.77      & 62.86            & 55.26        & 46.26        \\
                       & Maxout~\cite{zhang2018deep}          & 93.72          & 89.91      & 70.41         & 70.34      & 86.26            & 73.68        & 72.85        \\
                       & Maxout-BA~\cite{wei2022towards}       & 95.78          & 92.21      & 76.49         & 75.65      & 88.71            & 72.37        & 79.08        \\
                       & UE-UGCL~\cite{wei2022towards}         & 98.21          & 95.60      & 86.58         & 86.41      & 95.10            & 76.32        & 89.28        \\
                       & CM~\cite{wei2022contextual}              & 99.18          & 97.49      & \underline{91.90}         & \underline{91.47}      & 96.60            & \underline{97.37}        & \underline{94.28}        \\
                       & ComplexIrisNet~\cite{nguyen2022complex}  & \underline{99.30}          & 97.07      & 91.00         & 90.04      & 96.46            & 92.11        & 94.00        \\
                       & IR-Norm        & 99.25          & \underline{97.76}      & 91.55         & 91.43      & \underline{98.50}            & 90.79        & 93.86        \\
                       & \textbf{IR-BBox} (ours) & \textbf{99.51}          & \textbf{98.86}      & \textbf{94.44}         & \textbf{94.17}    & \textbf{99.57}            & \textbf{100.00}       & \textbf{94.44}        \\
\midrule
\multirow{9}{*}{Dual}  & Gabor~\cite{daugman2009iris}           & 76.56          & 73.23      & -               & 30.57      & 58.06            & 43.08        & 40.75        \\
                       & OM~\cite{sun2008ordinal}              & 64.68          & 61.34      & -               & 22.83      & 46.93            & 16.92        & 32.42        \\
                       & Maxout~\cite{zhang2018deep}          & 90.92          & 88.55      & -               & 59.10      & 78.60            & 66.15        & 70.32        \\
                       & Maxout-BA~\cite{wei2022towards}       & 93.48          & 91.37      & -               & 65.86      & 83.74            & 72.31        & 76.20        \\
                       & UE-UGCL~\cite{wei2022towards}         & 98.01          & 96.25      & -               & 79.34      & 95.72            & 73.85        & 89.04        \\
                       & CM~\cite{wei2022contextual}              & 97.97          & 96.54      & -               & \underline{86.03}      & 94.29            & 80.00        & \underline{93.54}        \\
                       & ComplexIrisNet~\cite{nguyen2022complex}  & \underline{98.49}          & \underline{96.98}      & -               & 84.72      & 94.58            & \underline{86.15}        & 92.94        \\
                       & IR-Norm        & 98.33          & 96.98      & -               & 85.26      & \underline{96.86}            & 83.08        & 92.30        \\
                       & \textbf{IR-BBox} (ours) & \textbf{99.03}          & \textbf{98.27}      & -               & \textbf{88.93}    & \textbf{97.48}            & \textbf{98.41}        & \textbf{93.84}       \\
\bottomrule
\end{tabular}
\caption{Identification accuracy ($\uparrow$) of SOTAs and the proposed method on general and factor-specific test protocols.}
\label{tab:exp-1vn}
\end{table*}

% \begin{table*}[tbp]
% \centering
% \small
% \begin{tabular}{ll|YYYY|YYY}
% \toprule
% \textbf{Eye}           & \textbf{Method}    & \textbf{Control} & \textbf{Fix} & \textbf{Select} & \textbf{Any} & \textbf{Occlusion} & \textbf{Pupil} & \textbf{Angle} \\
% \midrule
% \multirow{9}{*}{Left}  & Gabor~\cite{daugman2009iris}              & 81.31          & 68.00      & 40.78         & 40.85      & 33.58            & 40.29           & 42.33        \\
%                        & OM~\cite{sun2008ordinal}                 & 73.55          & 59.72      & 34.48         & 33.14      & 26.77            & 23.88           & 35.64        \\
%                        & Maxout~\cite{zhang2018deep}             & 90.69          & 78.06      & 68.93         & 67.88      & 49.24            & 56.71           & 72.34        \\
%                        & Maxout-BA~\cite{wei2022towards}          & 91.45          & 78.55      & 73.42         & 72.08      & 49.74            & 55.22           & 77.04        \\
%                        & UE-UGCL~\cite{wei2022towards}            & 92.15          & 80.62      & 80.01         & 78.67      & 57.07            & 61.19           & 83.34        \\
%                        & CM~\cite{wei2022contextual}                 & 92.44          & 81.08      & 83.52         & 82.42      & 57.32            & 61.19           & 86.33        \\
%                        & ComplexIrisNet~\cite{nguyen2022complex}     & 92.42          & 80.78      & 82.96         & 80.93      & 57.07            & 62.69           & 85.66        \\
%                        & IR-Norm          & 92.70          & 77.43      & 78.65         & 78.75      & 54.82            & 53.65           & 85.37        \\
%                        & IR-BBox   (ours) & 92.93          & 78.66      & 80.55         & 81.87      & 57.08            & 54.87           & 85.89        \\
% \midrule
% \multirow{9}{*}{Right} & Gabor~\cite{daugman2009iris}              & 83.73          & 67.34      & 37.89         & 38.85      & 47.28            & 37.97           & 41.18        \\
%                        & OM~\cite{sun2008ordinal}                 & 76.31          & 60.32      & 32.92         & 33.84      & 41.40            & 25.31           & 36.11        \\
%                        & Maxout~\cite{zhang2018deep}             & 91.44          & 76.04      & 64.27         & 64.32      & 61.19            & 45.56           & 67.44        \\
%                        & Maxout-BA~\cite{wei2022towards}          & 92.88          & 77.58      & 68.85         & 68.89      & 62.44            & 46.83           & 72.39        \\
%                        & UE-UGCL~\cite{wei2022towards}            & 95.89          & 81.21      & 79.54         & 79.40      & 71.04            & 48.10           & 83.15        \\
%                        & CM~\cite{wei2022contextual}                 & 95.98          & 81.63      & 84.14         & 83.84      & 72.62            & 51.90           & 87.76        \\
%                        & ComplexIrisNet~\cite{nguyen2022complex}     & 96.17          & 82.11      & 84.27         & 83.07      & 73.08            & 50.63           & 87.66        \\
%                        & IR-Norm          & 97.09          & 79.24      & 81.85         & 81.40      & 71.01            & 39.70           & 86.16        \\
%                        & IR-BBox (ours)   & 97.86          & 81.42      & 86.60         & 86.48      & 71.51            & 48.52           & 88.67        \\
% \midrule
% \multirow{9}{*}{Dual}  & Gabor~\cite{daugman2009iris}              & 71.80          & 64.02      & -               & 30.56      & 44.77            & 43.07           & 40.75        \\
%                        & OM~\cite{sun2008ordinal}                 & 60.66          & 53.63      & -               & 22.83      & 36.19            & 16.92           & 32.42        \\
%                        & Maxout~\cite{zhang2018deep}             & 85.28          & 77.44      & -               & 59.11      & 60.61            & 66.15           & 70.33        \\
%                        & Maxout-BA~\cite{wei2022towards}          & 87.66          & 79.89      & -               & 65.86      & 64.57            & 72.30           & 76.26        \\
%                        & UE-UGCL~\cite{wei2022towards}            & 91.91          & 84.16      & -               & 79.33      & 73.81            & 73.84           & 89.04        \\
%                        & CM~\cite{wei2022contextual}                 & 91.88          & 84.42      & -               & 86.05      & 72.72            & 80              & 93.54        \\
%                        & ComplexIrisNet~\cite{nguyen2022complex}     & 92.37          & 84.80      & -               & 84.72      & 72.94            & 86.15           & 92.94        \\
%                        & IR-Norm          & 92.37          & 78.75      & -               & 81.08      & 67.00            & 75.38           & 91.87        \\
%                        & IR-BBox (ours)   & 93.27          & 81.40      & -               & 87.91      & 69.08            & 95.38           & 93.71       \\
% \bottomrule
% \end{tabular}
% \caption{Identification performance of SOTAs and the proposed method on variation-based protocols. Results are reported in rank-1 accuracy ($\uparrow$).{\color{blue} Results to be updated!}}
% \label{tab:off-axis-identification}
% \end{table*}





\Cref{subsec:exp-1v1-general,subsec:exp-1v1-factor} present the benchmark results of SOTAs and IR-BBox on the proposed ImmerIris dataset. We first train models of compared methods on the ImmerIris training set and evaluate their verification FRR@FAR under 4 general test protocols of increasing difficulty. Recall from~\cref{subsec:protocol-rationale}: 1) Immer-Control simulates a controlled setting in immersive IR, where samples are of high quality and have fixed gaze points; 2) Immer-Fix incorporates challenging samples; 3) Immer-Select further varies gaze points except for extreme angles; and 4) Immer-Any imposes no constraints on angle or quality. Results are reported in~\cref{tab:exp-1v1-general}. We highlight three-fold findings:

\noindent \textbf{Regarding dataset and benchmark design:}
1) Immer-Control is comparable to the controlled setup in difficulty except for additional off-axis distortion. By training on ImmerIris, which is 25$\times$ larger than CASIA-T, learning-based SOTAs achieve better performance here than in~\cref{tab:exp-onaxis}(a), showing their advantage from data richness. This also underscores ImmerIris's significance in providing a large-scale iris dataset.
2) All SOTAs degrade sharply as protocols incorporate challenging samples and gaze variations, confirming these as major difficulties in immersive IR.

\noindent \textbf{Regarding existing SOTAs:}
3) Training-free Gabor and OM perform unsatisfactorily on all protocols. Since their effectiveness depends primarily on normalization quality, this indicates the insufficiency of current normalization methods even in the simplest immersive setting.
4) While training on ImmerIris eliminates the domain gap and improves results on Immer-Any compared with~\cref{tab:exp-onaxis}(b), the performance of SOTAs remains barely operable, showing they are unprepared for real-world immersive setup and that methodological advances are necessary.

\noindent \textbf{Regarding normalization-free paradigm:}
5) IR-BBox performs surprisingly well, ranking first or second in almost all cases. This confirms the potential of waiving normalization. We believe that waiving normalization is effective by avoiding its fallibility and introducing global perception of high-level iris semantics, thereby improving robustness to open-scene variations.
6) The ablated IR-Norm also performs well due to the strengths of our inherited FR framework. Its gap from IR-BBox demonstrates the stand-alone benefit of normalization-free design.


% We highlight several key findings: 1) Immer-Control can be seen as a counterpart of the controlled setup with additional off-axis distortion. By training on ImmerIris, which is 25$\times$ larger than CASIA-T, learning-based SOTAs achieve better performance here than in~\cref{tab:exp-onaxis}(a), demonstrating their ability to benefit from data richness. This also underscores the value of our work in constructing a large-scale dataset. 2) In contrast, the training-free Gabor and OM perform unsatisfactorily. Since their effectiveness depends directly on the quality of normalization, this indicates that conventional normalization methods are insufficient even for handling the simplest immersive scenario. 3) The performance of all SOTAs degrades severely as the protocols incorporate challenging samples and gaze variations, indicating that these factors are indeed major difficulties in immersive IR. 4) SOTAs achieve better results on Immer-Any than in~\cref{tab:exp-onaxis}(a), as training on ImmerIris eliminates the domain gap. Nonetheless, their performance still degrades to the point of being barely operable, indicating that SOTAs are not yet prepared for real-world immersive IR and that methodological advances are necessary. 5) Our proposed IR-BBox performs surprisingly well, ranking first or second in almost all results. Since its only key change is waiving normalization, this highlights the potential of the normalization-free paradigm. The benefit lies not only in avoiding fallible normalization to yield more accurate representations, but also in providing more global context, which enhances robustness to open-scene variations. 6) The ablated IR-Norm also achieves strong performance, mainly attributed to the advances of FR frameworks in in-the-wild scenarios. Its performance gap with IR-BBox also demonstrates the stand-alone gain of being normalization-free.




% \subsection{Iris Identification on ImmerIris}
\subsection{Performance on Challenging Factors}
\label{subsec:exp-1v1-factor}

Beyond general performance, we also investigate the impact of degradation and intra-class variation on the immersive setup through dedicated test protocols from~\cref{subsec:protocol-rationale}, which differ from Immer-Control only by the added factors. Specifically: 1) Immer-Occlusion, occlusion by eyelids or eyelashes; 2) Immer-Dilation, illumination-induced pupil dilation; 3) Immer-Light, paired irises with pupil size variation; and 4) Immer-Angle, high-quality samples with isolated gaze point variations. Results are reported in~\cref{tab:exp-1v1-factor}. By comparing with Immer-Control in~\cref{tab:exp-1v1-general}, we find:

\noindent \textbf{Regarding specific challenges:} 1) Occlusion imposes a moderate challenge, with an average 9.34\% increase in FRR@FAR(1e-5). 2) Dilation alone does not substantially degrade performance, even for normalization-based SOTAs, likely because the polar transform unwraps the iris into a rectangle, where pupil size mainly alters the aspect ratio without significantly impairing information. 3) However, performance drops considerably when dilated and constricted irises are paired, suggesting that mismatched pupil sizes impair geometric consistency after normalization. 4) Gaze point variation causes an average degradation of 36.99\%, making it the top challenge in the immersive setup.

\noindent \textbf{Regarding normalization-free paradigm:} 5) IR-BBox handles degradations (\ie, occlusion and dilation) effectively and is robust to pupil size variation, ranking first or second across these protocols. It improves by 4.12–16.82\% over IR-Norm, validating the stand-alone strength of being normalization-free. 6) Nonetheless, though IR-BBox outperforms most SOTAs under gaze point variation, it lacks a decisive advantage, suggesting that dedicated improvements are required to address this factor. We highlight this as a promising direction for future research.

% By comparing to~\cref{tab:off-axis-verification-all}, we find: 1) Occlusion imposes a moderate challenge by an average FRR@FAR(1e-5) change of 9.34\%. 2) Dilation alone does not substantially degrade performance even for normalization-based SOTAs. This is likely because when the iris is normalized via the polar transform, changes in pupil size mainly affect the resolution of the rectangular shape without substantially degrading the encoded information. 3) Nonetheless, SOTAs' performance drops significantly when dilated and constricted irises are paired, suggesting the geometric consistency between them after normalization may be impaired. 4) Variations of gaze points imposes an average degradation of 36.99\%. We identify it as the most challenging factor in immersive IR. 5) IR-BBox handles degradations (occlusion and dilation) well and is robust to pupil size variation, ranking first or second on these protocols. It improves 4.12-16.82\% than IR-Norm, showing the stand-alone strength of our normalization-free paradigm. 6) Nonetheless, it handles gaze point variation better than most SOTAs but without decisive advantage. 说明normalization-free要解决angle variation还需要其它技术辅助。

% Findings: 1) Dilation 其实easier than we expected,这可能是dilation仅相当于径向拉伸normalized iris,所以哪怕normalization-based方法表现都不错;但2) pupil size variation则不好,这表明现有方法不擅长处理variation;3) angle还是最大的挑战;4) IR-BBox在前三个protocol下都很好,说明它不仅能够处理degradation,对pupil size variation也很鲁棒。注意到它相较IR-Norm有4.12-16.82\%的显著改进;5) 对所有方法而言angle variation都是最有挑战性的场景;IR-BBox在这个场景虽然较大部分方法好,但没有很明显的优势,说明normalization-free要解决angle variation还需要其它技术辅助。

% As shown in~\cref{tab:off-axis-verification-var}, SOTA methods consistently struggle under these settings. Their performance deteriorates sharply, particularly in Immer-Dilation-Mix and Immer-Angle, where multiple degradations and extreme pose lead to FRR values that render them barely operable. These results highlight the fragility of normalization-based pipelines when faced with realistic, in-the-wild conditions.

% In contrast, our proposed IR-BBox exhibits clear robustness across all challenging factors. It consistently achieves the best or second-best results, with especially large margins in Immer-Dilation and Immer-Angle. This demonstrates that waiving normalization in favor of bounding-box cropping not only benefits overall recognition but also equips the model to handle intra-class variations more effectively. Taken together, these findings reinforce IR-BBox as a promising baseline for developing robust immersive IR solutions.


\subsection{ImmerIris Identification}
\label{subsec:exp-1vn}

We further evaluate the identification performance of SOTAs and IR-BBox under the general and factor-specific protocols. Note that the Immer-Dilation protocol is undefined for identification, as the gallery set is fixed to normal pupil-to-ocular ratio samples. Full results are yielded in the supplementary material. Results hence are reported on the remaining 7 protocols by rank-1 accuracy in~\cref{tab:exp-1vn}.

Overall, we observe similar trends to those in the verification task: 1) the protocols reveal distinct challenges from degradation and intra-class variation in immersive recognition compared with the controlled setup; 2) SOTAs show considerable room for improvement under these factors; and 3) the proposed normalization-free paradigm consistently achieves gains, ranking first or second in almost all cases. See the supplementary material for further discussion.




% Similar to verification, SOTA methods exhibit substantial performance drops as the protocols become more challenging, especially under Immer-Any, Occlusion, Dilation, and Angle. In particular, traditional methods such as Gabor and OM degrade sharply, confirming their limited applicability in immersive, in-the-wild scenarios.

% We further evaluate SOTAs and IR-BBox under the same seven test protocols in the identification setting, reporting rank-1 accuracy in~\cref{tab:exp-1vn}. Similar to verification, SOTA methods exhibit substantial performance drops as the protocols become more challenging, especially under Immer-Any, Occlusion, Dilation, and Angle. In particular, traditional methods such as Gabor and OM degrade sharply, confirming their limited applicability in immersive, in-the-wild scenarios.

% In contrast, IR-BBox consistently achieves competitive or superior accuracy across all protocols. It not only matches SOTAs under controlled conditions but also maintains robustness under challenging settings such as Dilation and Angle, where most normalization-based pipelines fail. These results further demonstrate the effectiveness of waiving normalization and establish IR-BBox as a strong baseline for immersive IR identification.

\subsection{Ablation Studies}
\label{subsec:exp-abal}
% Effectiveness of bbox
% 模型大小
% 不同的预处理方法
\begin{table}[tbp]
\footnotesize
\centering
\begin{tabular}{llccc}
\toprule
\multirow{2}{*}{\textbf{Method}} & \textbf{Settings}          & \multicolumn{3}{c}{\textbf{Performance}} \\
                                 & FRR@FAR (\%) $\downarrow$ & 1e-1          & 1e-3           & 1e-5          \\
\midrule
\multirow{3}{*}{IR-Norm}         & Default                   & 5.84      & 29.72      & 55.23     \\
                                 & Alt. Model                & 6.08      & 31.51      & 55.41     \\
                                 & Alt. Normalization                 & 3.25      & 24.08      & 52.77     \\
\midrule
\multirow{2}{*}{IR-BBox}         & Default                   & 2.32      & 23.95      & 50.80     \\
                                 & Alt. Model                     & 2.11      & 24.59      & 51.99     \\
                                 % & Alt. Normalization                 & 1.90      & 21.69      & 49.89    \\
\bottomrule
\end{tabular}
\caption{Verification FRR@FAR ($\downarrow$) of IR-Norm and IR-BBox under alternative (Alt.) model architectures and iris normalization methods, averaged over left and right eyes on Immer-Any.}
\label{tab:exp-abal}
\vspace{-3mm}
\end{table}

By waiving normalization, we compared IR-BBox and IR-Norm in~\cref{tab:exp-1v1-general,tab:exp-1v1-factor} and observed consistent performance gains. In~\cref{tab:exp-abal}, we further ablate two factors:

\noindent \textbf{Model scale.} We replace the IR-50 backbone with a smaller IR-18, comparable in size to recent SOTAs~\cite{nguyen2022complex,wei2022contextual}. IR-BBox still consistently outperforms IR-Norm and surpasses SOTAs in~\cref{tab:exp-1v1-general}, confirming that the performance gain of being normalization-free is independent of model scale.

\noindent \textbf{Normalization method.} We replace the normalization step with a commercial API. Performance of IR-Norm changes only marginally, validating that the drawback of normalization is inherent rather than tied to a specific implementation.


% --- deprecated ---


% For the isolated effects of waiving normalization, we have compared IR-BBox and IR-Norm in~\cref{tab:exp-1v1-general,tab:exp-1v1-factor,tab:exp-1vn} and observed consistent performance gain. We further ablated two factors in~\cref{tab:exp-abal}: 1) To validate the performance gain does not mainly come from model complexity, we replace the IR-50 model with a smaller IR-18 model, comparable in size to latest SOTAs~\cite{nguyen2022complex,wei2022contextual}. Results show IR-BBox consistently outperforms IR-Norm on smaller model, and also outperform SOTAs in~\cref{tab:exp-1v1-general}. This shows waiving normalization can achieve high performance independent to model scale. 2) To validate the drawback of normalization is not due to a specific processing pipeline but a widely existing problem, we replace the normalization method with a commerial API. Results on IR-Norm shows the alternation of methods affect performance marginally, hence ...

% We conduct ablation experiments to further validate the design choices of our method. First, we compare IR-BBox with its normalization-based counterpart IR-Norm. Results show that IR-BBox consistently outperforms IR-Norm, especially under degradation (Occlusion, Dilation) and variation (Angle) scenarios, confirming that bounding-box cropping is more effective than normalization for immersive IR. Second, we evaluate model size by comparing IR-BBox with ResNet-18 (IR-18) and ResNet-50 (IR-50). Both models achieve strong results, and IR-18, which is on a similar scale to existing SOTAs, already outperforms them by a clear margin. This indicates that the advantage of our method does not come from larger model capacity. Third, we test with different normalization implementations and find that the performance of IR-BBox and IR-Norm remains stable, suggesting that the poor performance of existing methods cannot be attributed to a specific normalization choice.


% \input{tab/tab_onaxis}


% Five methods that extract from normalized iris using DNNs,
% 3) \textbf{Maxout}~\cite{zhang2018deep}, fusing iris and periocular features,
% 4) \textbf{Maxout-BA}~\cite{wei2022towards}, Maxout with batch normalization and ArcFace loss,
% 5) \textbf{UE-UGCL}~\cite{wei2022towards}, uncertainty-guided curricular learning,
% 6) \textbf{CM}~\cite{wei2022contextual}, dual-branch global/local context learning,
% 7) \textbf{ComplexIrisNet}~\cite{nguyen2022complex}, complex-valued iris recognition, and
% 8) our normalization-free method, referred to as \textbf{IR-BBox}.  

% \noindent \textbf{Compared SOTAs.} We compare 7 representative SOTAs with our proposed method. Specifically, two training-free methods rely on hand-crafted filters applied to normalized iris: \textbf{Gabor}~\cite{daugman2009iris}, which binarizes iris patterns with a log-Gabor filter, and \textbf{OM}~\cite{sun2008ordinal}, which encodes iris via ordinal measures. In contrast, five DNN-based methods extract features from normalized iris: \textbf{Maxout}~\cite{zhang2018deep}, which fuses iris and periocular features; \textbf{Maxout-BA}~\cite{wei2022towards}, which augments Maxout with batch normalization and ArcFace loss; \textbf{UE-UGCL}~\cite{wei2022towards}, which incorporates uncertainty-guided curricular learning; \textbf{CM}~\cite{wei2022contextual}, which employs a dual-branch design for global and local context learning; and \textbf{ComplexIrisNet}~\cite{nguyen2022complex}, which leverages complex-valued networks. Finally, we include our normalization-free method, \textbf{IR-BBox}.

% One thing we are curious in the beginning is, how significant is the domain gap between conventional and immersive IR? To demonstrate, we further test these CASIA-T-trained models on Immer-Any to show if they can be directly used for immersive IR. Note that we eliminate other interference by aligned the input form of iris (\eg, normalized iris and ocular image are samely preprocessed). Intuitively, if SOTAs achieved comparable performance on recognizing CASIA-T and Immer-Any, it suggests these two datasets bear little domain gaps, and vise versa.

% Results are shown in~\cref{tab:exp-onaxis}(b). All methods' performance are significantly deteriorated and are bearly operatable. This suggest immersive IR exhibits intrinsically different challenges compared to conventional IR.


% 为排除其它干扰,我们尽可能将输入图像矫正为具有相同的格式(如使用同样预处理得到的normalized iris或ocular image)


% We begin by evaluating SOTAs on the controlled setup, their default operating scenario by design, to establish a performance reference for later comparison. Specifically, we train learning-based SOTAs on the training set of CASIA-T, and test them together with training-free SOTAs on CASIA-T's respective test protocol. We report verification FRR@FAR in~\cref{tab:exp-onaxis}(a). Results show that Gabor and OM achieve very robust performance, whereas learning-based SOTAs achieve lower yet still operable performance. We note that this drop is not an inherent defect of the methods but is largely due to the limited training data volume. Based on prior literature~\cite{zhang2018deep,wei2022towards,wei2022contextual,nguyen2022complex}, they can surpass Gabor and OM on larger, though not publicly available, training datasets. Overall, SOTAs perform effectively under the controlled setup.

% We further test SOTA models trained on CASIA-T on the Immer-Any protocol of ImmerIris, to examine whether they can be generalized to recognize immersive iris data. Its purpose is to validate the differences between controlled and immersive data. Note that our employed normalization method can by design produce normalized irises for both on-axis and off-axis irises, so input data should 形式上 be aligned. Performance change can therefore be primarily attributed to the inherent domain gap of datasets. As shown in~\cref{tab:exp-onaxis}(b), the performance of all SOTAs deteriorates to the point of being barely usable. This reveals a significant domain gap between iris data captured from controlled and immersive setups that cannot be mitigated through normalization. These findings suggest that our proposed ImmerIris dataset represents a distinct and novel scenario compared to existing controlled datasets.
 
% SOTA methods are all primarily designed for conventional IR, where ocular images are captured under on-axis, controlled setups. Therefore, we first evaluate SOTAs under their designated settings to establish a baseline for later comparison. We train and test them on the respective training and test sets of CASIA-T, and report verification FRR@FAR. From~\cref{tab:exp-onaxis}(a), we observe that Gabor and OM achieve the best performance. As these methods are training-free and their identity templates are directly derived from normalized irises, this validates the effectiveness of normalization. DNN-based SOTAs achieve lower yet still operable performance. The drop is not a defect of the methods themselves but is largely due to the limited data volume in CASIA-T. Based on prior literature~\cite{zhang2018deep,wei2022towards,wei2022contextual,nguyen2022complex}, their performance surpasses that of Gabor and OM on larger datasets.

% \textit{How significant is the difference between conventional and immersive IR?} To examine this, we further test the CASIA-T-trained models on Immer-Any to see whether they can be transferred to recognize off-axis ocular images in the wild. Note that the normalization we employ is capable of producing normalized irises and cropped ocular images in aligned form for both CASIA-T and Immer-Any, thus minimizing format inconsistency. Intuitively, if SOTAs achieved comparable performance on these two datasets, it would suggest little domain gap, and vice versa. As shown in~\cref{tab:exp-onaxis}(b), the performance of all methods deteriorates significantly and becomes barely operable. This indicates that immersive IR poses intrinsically different challenges from conventional IR, constituting a new research scenario that our proposed ImmerIris dataset is intended to address.


% We next evaluate SOTA methods and our proposed IR-BBox on the ImmerIris dataset under four protocols of increasing difficulty. Immer-Control simulates controlled conditions and serves as the baseline. Immer-Fix introduces hard samples with various degradations while fixing gaze points. Immer-Select allows free gaze except for extreme angles, and Immer-Any represents the most unconstrained scenario with no restrictions on gaze or quality.

% 1)  Immer-Control can be regard as an extension of controlled setup (Tab. 1(a)) in difficulty except for off-axis distortion,Learning-based achieve better performance than in Tab. 1(a). This suggests这表明了learning-based方法的共同优点,即能通过在大规模数据集上的特征学习,获得更强的身份判别能力。这也说明了我们创建更大规模数据集的益处。
% 2) 与之相反,Gabor和OM的training-free方法的表现不尽如人意。由于这些方法的效果直接取决于normalization的质量,这也表明传统的normalization方法哪怕在处理最简单的open-scene immersive数据时也insuffice。
% 3) Performance of all SOTAs downgrades severely as the protocols are incorporated with challenging samples and gaze variations. 这表明这些因素确实是immersive IR中的难点。
% 4) SOTAs在Immer-Any上取得了比Tab. 1(a)更好的结果,这是因为在同分布训练集上训练消除了domain gap。Nonetheless,它们的结果仍退化至barely operable,说明SOTAs are not prepared for immersive IR in the real-world and methodological advances would be necessary.
% 5) Our proposed IR-BBox performs suprisingly well, ranking the best or second-best on almost all results. 由于它的唯一key change是waiving normalization,这显示了normalization-free的潜力。该潜力不仅在于规避了fallable,得到了准确的表示,也在于它能够提供更全局的信息。从而使模型对open-scene variation更具有鲁棒性。
% 6) The ablated IR-Norm也取得了很好的表现,这归功于FR在in-the-wild场景下的优点。它与IR-BBox的gap也表明了normalization-free's stand-alone gain.

% Beyond overall protocols, we further investigate the impact of specific challenging factors in immersive IR, including occlusion by eyelids/eyelashes (Immer-Occlusion), illumination-induced pupil dilation (Immer-Dilation), combined degradations (Immer-Dilation-Mix), and severe pose variation (Immer-Angle). These factors reflect common intra-class variations that often occur in unconstrained immersive environments.
