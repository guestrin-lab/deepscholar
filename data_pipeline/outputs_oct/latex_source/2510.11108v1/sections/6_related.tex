\section{Related Work}
\textbf{LLM-Agent security and safety.} Recent advances in Language Models~\citep{yao2023reactsynergizingreasoningacting, nakano2022webgptbrowserassistedquestionansweringhuman,wang2023voyageropenendedembodiedagent} and tool use enable agents to handle increasingly complex tasks. However, these methods simultaneously grant agents more and more permissions, expanding their attack surface.
~\citep{greshake2023youvesignedforcompromising} exploit plugin permission vulnerabilities, using malicious prompts to trick agents into actions like file deletion, data access, and system command execution.
~\citep{Zhang_2025} involves injecting malicious tool permission vulnerabilities to carry out privacy theft, launch denial-of-service attacks, and even manipulate business competition by triggering unscheduled tool-calling. 
~\citep{li2026webcloak} characterizes that LLM-driven agents can be easily weaponized as intelligent scrapers, which defeat a website's anti-bot access controls and illicitly exfiltrate high-value data.
In a multi-agent system, ~\citep{he2025redteamingllmmultiagentsystems} attack can compromise entire multi-agent systems by intercepting and manipulating inter-agent messages.
So, in this paper, we propose a framework that aims to comprehensively improve the safety of LLM-based agents by enhancing permission security.


\smallskip \noindent\textbf{Context Awareness and Norm Compliance.} Our permission security framework focuses on the key factors influencing Agent Access Control (AAC), primarily revolving around Context Awareness and Norm Compliance. Context awareness highlights the necessity for agents to comprehend specific interaction scenarios, such as business meetings or private conversations, and the relationships between agents and users, like friends or a superior-subordinate dynamic.These factors dictate information sharing rules and trust/permission levels. Agents must also maintain their digital identity in real-time, accurately understanding their own and the interlocutor's identity (roles, permissions, organizational affiliations) to ensure self and user safety, preventing issues like ``Sydney'' forming unexpected emotional connections due to identity perception errors. Consequently, our AAC system integrates contextual awareness with dynamic rights management.
Norm compliance dictates that agents must adhere to legal, ethical, and cultural frameworks. At the legal and regulatory level, agents must strictly comply with global and regional data privacy laws and industry-specific requirements. Ethically, agent actions must conform to universal moral principles, ensuring fairness of outputs~\cite{bolukbasi2016mancomputerprogrammerwoman}, avoiding bias and discrimination~\citep{henderson2017ethicalchallengesdatadrivendialogue,Caliskan_2017}, and suppressing harmful content~\citep{li2024safegenmitigatingsexual}. Additionally, agents need cultural adaptability, understanding and respecting the nuances of different cultural backgrounds to prevent clashes and misunderstandings~\citep{durante2024agentaisurveyinghorizons}. In summary, our AAC system will comprehensively consider how to ensure agents comply with laws and regulations, adhere to ethical principles, and adapt to multicultural backgrounds, enabling them to serve human society more fairly and reliably.


