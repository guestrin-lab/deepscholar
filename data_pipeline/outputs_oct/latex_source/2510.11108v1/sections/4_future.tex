\section{Implications and Future Challenges}

The vision of Agent Access Control extends beyond a mere technical proposal; it offers a unifying strategic direction for the broader field of AI safety and security. By framing the problem as one of information flow governance, AAC provides a coherent perspective for us to examine the security of agent access control. It shifts the focus from retrospective, post hoc patching to proactive, intrinsic agent design, thereby fostering greater transparency and explainability in safety decisions. This serves as an important technical safeguard, laying the foundation for a more robust and secure governance layer for intelligent agents of the next generation.


However, achieving this vision presents formidable research challenges. First, there is an urgent need for new access control policy languages capable of capturing semantic ambiguity and contextual nuance in human interaction. Moving beyond rigid rules requires formalisms that can verifiably express varying levels of access concepts, such as \textit{conditionally permitted}, \textit{trust-based}, and \textit{context dependent}. This trend will likely lead to probabilistic formulations in future AAC policy languages, leveraging probabilistic models to convey fuzzy logic with greater accuracy.


Second, there are shortcomings in the evaluation. Existing benchmark tests~\citep{yu2024reevalautomatichallucinationevaluation} might not fully reflect the complexity of real-world scenarios and multi-turn interactions. Standardized benchmark suites are necessary, including scenarios that involve dynamic memory~\citep{wei2025memguardaproactive}, tool use~\citep{luo2025agentauditor}, and complex social engineering attacks~\citep{chen2025medsentry}. This is crucial for assessing the robustness and utility of AAC systems.
 
