\section{The Agent Access Control (AAC) Framework}
In our vision, we first introduce the Agent Access Control (AAC) framework. It fundamentally rethinks access control from a static gatekeeping function into a dynamic and cognitively driven process of information governance. Unlike traditional models that treat binary allow/deny decision as the endpoint, AAC views the entire interaction as a continuous flow of information, extending beyond simple yes/no judgments to contextually grounded, end-to-end response generation process. This objective is achieved through two tightly integrated modules: multi-dimensional contextual evaluation and adaptive response formulation.

\textbf{Multi-dimensional Contextual Evaluation.} The first module moves beyond simplistic, identity-based checks to perform a holistic evaluation of the interaction context. AAC synthesizes information across multiple dimensions to build a rich understanding of appropriateness, including: \textbf{(1) Identity and Relationship:} Evaluating not just the user's role (e.g., manager, colleague) but also the potential for role shifts in dynamic interaction environments, making judgments based on the established trust and history with the agent~\citep{shuster2021iyoustateoftheartdialogue}. \textbf{(2) Interaction Scenario:} Distinguishing between different contexts, such as a formal business meeting versus a private conversation, each with distinct disclosure norms. \textbf{(3) Task Intent:} Analyzing the user's underlying goal to differentiate a legitimate request for a summary from a malicious attempt to exfiltrate raw data. \textbf{(4) Normative Adherence:} Assessing compliance with overarching legal (e.g., GDPR), ethical (e.g., fairness), and cultural norms that govern the interaction~\citep{durante2024agentaisurveyinghorizons}.

\textbf{Adaptive Response Formulation.} Based on the contextual evaluation, the second module formulates an adaptive response. This transforms AC from a mere filter into a sophisticated communication partner. Instead of simply blocking a request, AAC actively shapes the information output to maximize utility while minimizing risk. Key formulation strategies include: \textbf{(1) Granularity Control:} Deciding the appropriate level of detail, such as providing a high-level summary of a document to one user while revealing specific figures to another. \textbf{(2) Content Redaction and Anonymization:} With reference to the trust scoring mechanism, dynamically masking sensitive entities (e.g., names, ID numbers) in real-time when suspicious user intent or unsafe information flows are detected~\citep{fu2023improvinglanguagemodelnegotiation}. \textbf{(3) Semantic Paraphrasing:} Rephrasing information to align with the user's context or to mitigate potential harm, such as converting proprietary technical details into high-level insights for a collaborator.

Figure~\ref{fig:fig1} illustrates the design framework of AAC, where the two modules operate in concert to enable fine-grained control of agent permissions in complex scenarios, thereby ensuring the system's reliability and security. The user’s request is fed into an evaluation and formulation loop, replacing the previous approach of simple comparison against a static list. AAC is designed to enhance flexibility in access control under rule ambiguity and dynamically evolving interaction relationships. This characteristic enables the agent’s decisions to adapt to specific tasks, achieving the objectives of high security and strong contextual awareness~\citep{Caliskan_2017}.

\begin{figure*}[t]
  \includegraphics[width=1.0\linewidth]{sections/pipeline1.pdf}
  \caption{The overall pipeline of Agent Access control(AAC). We integrated multi-dimensional contextual information, such as user identity, task intention, and interaction history, to generate fine-grained and task-specific strategies. These strategies guide the agent in making interpretable access control decisions to respond to various potential attacks from adversaries. }
  \label{fig:fig1}
\end{figure*}
