\section{Conclusion}

The rapid rise of intelligent agents has urged researchers to scrutinize their security and robustness. One longstanding area, access control, needs to be reconceptualized within the field of agent security. We posited a necessary shift: from treating access control as a static binary allow/deny decision as the endpoint to viewing the entire interaction as a continuous flow of information. Our proposed Agent Access Control (AAC) framework embodies this shift, recasting access control not as an external constraint, but as an intrinsic and cognitive faculty for governing information flow. By equipping agents with the ability to dynamically evaluate context and adaptively shape their responses, we pave the way for systems that are not just secure in a technical sense, but are also socially and ethically aware. Although there remain theoretical gaps and technical challenges in fully realizing our envisioned AAC framework, this is still a worthwhile endeavor. The agents we aim to build are trustworthy not because they are rigidly locked, but because they can understand when, how, and why to grant or interrupt permissions, ensuring a safe system while preserving user experience.

\vspace{3em}