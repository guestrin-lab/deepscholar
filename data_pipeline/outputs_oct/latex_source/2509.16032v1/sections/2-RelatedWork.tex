\section{Related Work}
Relevant previous studies include HRI studies that explored the impact of a robot's height and compliance with robots' requests. 

\subsection{The impact of a robot's height}

Previous studies in HRI have shown that a robot’s height can influence the interaction dynamics and the way the robot is perceived \cite{rae2013influence, hiroi2008bigger, hiroi2016height, walters2009preferences, joosse2021appearance, samarakoon2022review}. While some of these studies indicated effects similar to those observed in human-human interactions \cite{rae2013telepresence, hiroi2016height}, others showed opposite patterns \cite{joosse2021appearance, walters2009preferences}. Among the studies that indicated similar effects to those observed in human-human interactions, Rae et al. (2013) studied compliance to requests presented by operators of telepresence robots. They tested whether the robot's height would impact the perceived authority and persuasiveness of the person operating it. They found that when the telepresence robot was shorter than the participants, its operators' authority was lower, and the participants reported feeling more dominant in the conversation \cite{rae2013telepresence}. Similarly, Walters et al. (2009) found that taller humanoid robots were perceived as more conscientious, while short humanoid robots were viewed as more neurotic and less conscientious \cite{walters2009preferences}, replicating the tendency to attribute positive characteristics to taller individuals. 

Other studies suggested that height may have an opposite effect to that found in human-human interactions, with people showing clear preferences for short robots. For example, Hiroi and Ito (2016) found that participants preferred to converse with a short robot, specifically with a robot that did not reach their eye level \cite{hiroi2016height}. Similarly, Joosse et al. (2021) found that shorter robots were perceived as safer and more comfortable to interact with. This suggests that humans in HRI may prioritize emotional safety and comfort over authority, which contrasts with human interactions where height positively correlates with competence and status \cite{joosse2021appearance}.

These studies indicate that a robot's height can impact the interactions with humans, the robot's perception, and people's preferences. We extend this line of studies by focusing on the impact of a non-humanoid robot's height and its influence on participants' willingness to comply with the robot's request. We tested whether, as in human-human interactions, a taller robot would lead to increased compliance—or whether the specific case of a non-humanoid robot would result in a different pattern. 

\subsection{Compliance with robots' requests} 

Several studies have explored human tendencies to comply with robots' requests \cite{lee2016role, nielsen2022prosocial, erel2024power}. The robot’s emotional behavior \cite{kuhnlenz2018effect, shiomi2017hug}, its ability to align with socially appropriate behaviors \cite{connolly2020prompting}, its social role \cite{saunderson2021persuasive}, its persuasion strategy \cite{lee2019robotic}, and its appearance \cite{kim2014effect} were all shown to encourage compliance with its requests \cite{connolly2020prompting, martin2020investigating, zaga2017gotta}. For example, Moshkina (2012) indicated that when a robot expressed affect, particularly through nonverbal expressions of negative mood such as nervousness and fear, it enhanced participants’ compliance with evacuation requests, leading to earlier and faster responses \cite{moshkina2012improving}. Similarly, Wills et al. (2016) found that participants were more compliant with a donation request when it came from a robot that followed human-like social norms, demonstrating appropriate gaze and facial expression cues, compared to a robot that did not exhibit social responsiveness to its surroundings \cite{wills2016socially}. Another aspect of HRI that impacts people's willingness to comply with a robot's request is its social role. Robots framed as peers were shown to elicit more compliance than those framed as supervisors. For example, Saunderson and Nejat (2021) found that participants were more willing to follow a peer robot's requests in comparison to a robot framed as an authority figure, suggesting that social framing may also shape compliance to robots \cite{saunderson2021persuasive}.

These studies indicate that various factors influence people's tendency to comply with a robot's request. We further explored whether robotic design characteristics can impact social dynamics and enhance participants' compliance. Specifically, we evaluated the impact of height in interactions with a highly non-humanoid robot. 














