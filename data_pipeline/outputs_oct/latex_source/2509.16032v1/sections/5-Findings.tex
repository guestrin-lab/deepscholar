\vspace{0.5cm}
\section{Findings}
A Bayesian analysis revealed no early differences between groups in the NARS questionnaire: $BF_{10}=0.06$ and the Agreeableness questionnaire: $BF_{10}=0.2$. The quantitative and qualitative analyses indicated that the robot’s height impacted participants’ compliance with the robot's request.


\subsection{Quantitative analysis}
We initially conducted 2-way ANCOVAs with gender as the additional factor. For simplicity, we report the one-way ANCOVA analyses, as there were no main effects or interactions.

\subsubsection{Compliance}
Almost all participants in both conditions picked up the tablet (17/22 in the \textit{Short robot} condition; and 15/22 in the \textit{Tall robot} condition; ${\chi}^2$$_\text{(1)}=0.46, p=0.5$). However, a one-way ANCOVA analysis (with participants’ height and age as covariates) revealed that the robot's height had a significant impact on the number of questions participants agreed to answer, F(1, 40) = 4.00, p = 0.05 (significant height covariate, F(1, 40) = 4.00, p = 0.05). The analysis indicated that in the \textit{Short robot} condition, participants agreed to answer more questions (132.9) than in the \textit{Tall robot} condition (80.6), demonstrating higher compliance tendencies (see Figure \ref{fig:Graph}A). 

\begin{figure} [t]
    \centering
    \vspace{-0.1em}
    \includegraphics[width=1\linewidth]{Figures/MORPHY-Findings-Graph.png}
    \caption{A: Averaged number of questions participants answered; B: Participants' averaged discomfort ratings.}
   %  \vspace{-1em}
    \label{fig:Graph}
\end{figure}

\subsubsection{Robotic Social Attributes Scale (RoSAS)}
 A one-way ANCOVA analysis (with participants’ height and age as covariates) revealed that the robot's height had a significant impact on participants’ discomfort, F(1, 40) = 3.9, p = 0.05 (no significant covariates). The analysis indicated that participants in the \textit{Short robot} condition experienced less discomfort (1.87) than those in the \textit{Tall robot} condition (2.74; see Figure \ref{fig:Graph}B). The robot's height did not affect the perception of its warmth (\textit{Short robot} - 3.64; \textit{Tall robot} - 3.73) and competence (\textit{Short robot} - 5.37; \textit{Tall robot} - 5.55). 




\subsection{Qualitative  analysis - Semi-structured interviews}
The interviews were transcribed and analyzed using Thematic Coding \cite{boyatzis1998transforming}. Three researchers identified initial themes and discussed inconsistencies with a fourth researcher. They then created a list of final themes. Next, they analyzed a subset of the interviews independently and calculated inter-rater reliability (Kappa=81\%). Following the inter-rater reliability, they continued to analyze all interviews. The themes were: volunteering to answer the questionnaire, functional perception of the robot, and general perception of the robot. 

\subsubsection {Volunteering to answer the questionnaire}
Participants in both conditions mentioned their willingness to comply with the robot's request. In both conditions, most participants explained that they complied with the request (\textit{Short robot} - 17/22; \textit{Tall robot} - 15/22). Participants did not attribute their compliance to the robot's characteristics. Instead, they described their behavior as a general tendency for pro-social behavior: \textit{"I could choose... and I just wanted to help."} (P.25, M, \textit{Short robot}); or curiosity \textit{"I don't mind trying new things, and I figured it's a robot with a tablet that's asking questions, so why not"} (P.32, M, \textit{Tall robot}). The few participants who did not volunteer provided practical reasons: \textit{"I did not want to spend more time"} (P.36, F, \textit{Tall robot}).

A similar pattern emerged when participants explained why they did not answer all 300 questions (\textit{Short robot} - 14/22; \textit{Tall robot} - 10/22). They provided general and practical explanations: \textit{"I had an exam and I had to go back to study"} (P.24, F, \textit{Short robot}); \textit{"It was fun at first and then like okay... enough"} (P.34, F, \textit{Tall Robot}). 


\subsubsection {Theme 2 - Functional perception of the robot}
Participants in both conditions discussed the robot's functionality. Positive aspects were mentioned more frequently at the \textit{Short robot} condition (16/22) than the \textit{Tall robot} condition (10/22): \textit{"it brings you things, it's cool"} (P.35, F, \textit{Short robot}); \textit{"It would help around with stuff you need… Instead of going and doing stuff yourself, you can use it for specific tasks."} (P.42, F, \textit{Tall robot}). The rest of the participants who discussed the robot's functionality (\textit{Short robot} - 5/22; \textit{Tall robot} - 12/22) provided either neutral or negative descriptions: \textit{"I do not need anything like this unless it could maybe do more things"} (P.43, F, \textit{Short robot}); \textit{"It's too tall and wide, I do not see what people can do with it"} (P.38, F, \textit{Tall robot}).


\subsubsection {Theme 3 - General perception of the robot}
The general perception of the robot followed a similar pattern, indicating a more positive perception of the \textit{Short robot} (14/22) than the \textit{Tall robot} (8/22): \textit{"It was really nice.... very cute... very alive and engaging"} (P.06, M, \textit{Short robot}); \textit{"It felt like a friend"} (P.15, F, \textit{Tall robot}). The rest of the participants who discussed the robot's perception (\textit{Short robot} - 7/22; \textit{Tall robot} - 14/22) provided neutral or negative descriptions: \textit{"It was a little aggressive"} (P.17, F, \textit{Short robot}); \textit{"It made me nervous"} (P.01, F, \textit{Tall robot}).