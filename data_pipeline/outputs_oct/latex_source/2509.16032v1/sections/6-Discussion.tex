\section{Discussion}

In this study, we demonstrated that the height of a non-humanoid robotic object can influence participants’ compliance with its request. Our results suggest that interactions with robotic objects evoke social responses, further supporting previous indications that people automatically interpret interactions with highly non-humanoid robots as rich social experiences \cite{erel2019robots, duffy2003anthropomorphism}. However, the direction of the effect that involved a greater tendency to comply with the request when it came from the Short robot, was opposite to the one typically observed in human-human interactions, where compliance is higher for requests coming from taller people due to their increased persuasiveness and authority \cite{young1996height, higham1992rise, stogdill1948personal}. Our findings revealed that when the request came from a Short robot, participants agreed to answer significantly more questions in an additional questionnaire than when the request came from a Tall robot. Our findings, therefore, challenge the common belief that interactions with robots are similar to interactions between humans \cite{nass1994computers, erel2024rosi}. We argue that it is essential to map the social impact of specific robotic features (such as height) when designing different robotic morphologies, as it cannot be applied or assumed based on the vast knowledge in social psychology (which has already mapped similar impacts in human-human interactions). 

Participants' descriptions and perceptions of the Short robot may provide insight into their increased compliance with its request. Although participants did not explicitly mention the robot’s height, they often used positive terms to describe it, stating that it was \textit{"Cute"}, \textit{"Nice"}, \textit{"Cool"}, and \textit{"Engaging"}. The Tall robot was described more frequently by neutral or negative terms, such as \textit{"Normal"}, \textit{"Harmless"}, \textit{"Weird"}, and \textit{"Scary"}. This explanation was supported by the RoSAS data, indicating increased discomfort when interacting with the taller robot. This suggests that the height of a robotic object may influence the valence of its perception, which in turn determines the level of its persuasiveness, with a positive perception leading to increased compliance. Another possible explanation may be attributed to the perception of the shorter robot as more vulnerable or child-like. Although there are no indications of this possibility in our results, it warrants further investigation. 

More broadly, our findings imply that people's compliance with robotic objects' requests may be shaped by social dynamics that are distinct from those in human-human interaction. While compliance with human requests is associated with the person's power, compliance with a robotic object's request may be based on simpler processes related to the robot’s perception and perceived "cuteness". Hence, mapping robotic characteristics that would enhance its perception and minimize discomfort may be sufficient for significantly increasing people's compliance with its requests. 

Taken together, our findings indicate the importance of mapping and carefully considering robotic design characteristics when designing robotic objects. We show that the robot's characteristics can significantly affect its social impact. The nature of the impact is shaped by processes unique to the interactions with robots. It is, therefore, important to further understand and map these processes and the social influence of different robotic design characteristics.
