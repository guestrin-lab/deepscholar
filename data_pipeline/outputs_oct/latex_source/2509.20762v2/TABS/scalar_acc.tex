\begingroup
\setlength{\tabcolsep}{4.2pt}
\begin{table*}[t!]
    \centering
    \caption{\uline{Observation 2 (further analysis): Real-world group anchors can be well explained by scalar anchor strengths associated with each node, together with local competition.} 
    We can achieve high accuracy (92.4\% on average) if we predict the node with the highest anchor proportion (a scalar) as the anchor in each hyperedge.}
    \label{tab:scalar_acc}
\begin{adjustbox}{max width=\textwidth}
\begin{tabular}{c|cccccccccccccccc|c}
\hline
\multirow{2}[2]{*}{\textbf{Dataset}} & \multicolumn{2}{c}{\texttt{coAA}} & \multicolumn{2}{c}{\texttt{coDB}} & \multicolumn{2}{c}{\texttt{coSM}} & \multirow{2}[2]{*}{\texttt{qaBI}} & \multirow{2}[2]{*}{\texttt{qaPH}} & \multirow{2}[2]{*}{\texttt{qaMA}} & \multirow{2}[2]{*}{\texttt{qaST}} & \multirow{2}[2]{*}{\texttt{emEN}} & \multirow{2}[2]{*}{\texttt{emEU}} & \multirow{2}[2]{*}{\texttt{emER}} & \multirow{2}[2]{*}{\texttt{soME}} & \multirow{2}[2]{*}{\texttt{soRE}} & \multirow{2}[2]{*}{\texttt{moML}} & \multirow{2}[2]{*}{Avg.} \bigstrut[t]\\
      & (first) & (last) & (first) & (last) & (first) & (last) &       &       &       &       &       &       &       &       &       &       &  \bigstrut[b]\\
\hline
\textbf{Acc. (\%)}  & 93.5 & 92.9 & 97.1 & 96.1 & 98.3 & 100.0 & 98.9 & 98.6 & 100.0 & 100.0 & 80.6* & 59.4* & 81.3* & 92.8 & 88.8 & 100.0 & 92.4 \bigstrut\\
\hline
\multicolumn{18}{l}{\small *Email datasets, especially \texttt{emEU}, contain many repeated hyperedges consisting of the same nodes but with different anchors} \\
\end{tabular}%
\end{adjustbox}
\end{table*}
\endgroup
