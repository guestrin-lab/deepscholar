\begingroup
\setlength{\tabcolsep}{4pt}
\begin{table*}[t!]
    \centering
    \caption{\uline{\ours can achieve high accuracy even when trained and tested on different datasets.} The accuracy (\%) of \ours in inductive settings is highlighted in bold if it is better than that of the strongest baseline in transductive settings. Surprisingly, the performance of \ours in inductive settings is sometimes even better than in transductive settings.}
    \label{tab:inductive}
\begin{adjustbox}{max width=\linewidth}
\begin{tabular}{l|ccc|ccc|cc|cc|cc|cc}
\hline
\textbf{Training Dataset} & \multicolumn{3}{c|}{\texttt{qaBI}} & \multicolumn{3}{c|}{\texttt{qaPH}} & \multicolumn{2}{c|}{\texttt{coDB} (first)} & \multicolumn{2}{c|}{\texttt{coDB} (last)} & \multicolumn{2}{c|}{\texttt{coAA} (first)} & \multicolumn{2}{c}{\texttt{coAA} (last)} \bigstrut[t]\\
\textbf{Test Dataset} & \texttt{qaPH} & \texttt{qaMA} & \texttt{qaST} & \texttt{qaBI} & \texttt{qaMA} & \texttt{qaST} & \texttt{coAA} & \texttt{coSM} & \texttt{coAA} & \texttt{coSM} & \texttt{coDB} & \texttt{coSM} & \texttt{coDB} & \texttt{coSM} \bigstrut[b]\\
\hline
\ours (inductive) & 86.95 & 35.07 & 31.32 & \textbf{86.94} & \textbf{39.10} & 29.59 & \textbf{45.46} & \textbf{36.56} & \textbf{46.27} & \textbf{44.98} & 41.27 & 34.50 & 42.94 & 39.03 \bigstrut\\
\hline
\ours (transductive) & 88.93 & 41.16 & 35.12 & 87.23 & 41.16 & 35.12 & 49.77 & 38.33 & 49.32 & 44.91 & 46.02 & 38.33 & 50.72 & 44.91 \bigstrut[t]\\
Strongest baseline (transductive) & 88.07 & 38.99 & 31.98 & 85.35 & 38.99 & 31.98 & 44.74 & 35.66 & 46.10 & 40.87 & 42.92 & 35.66 & 45.83 & 40.87 \bigstrut[b]\\
\hline
\end{tabular}%
\end{adjustbox}
\end{table*}
\endgroup
