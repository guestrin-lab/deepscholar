\vspace{-3mm}
\section{Preliminaries}\label{sec:prelim}


\smallsection{Basic notations.}
We use $\bbN \coloneqq \setbr{1, 2, 3, \ldots}$ to denote the set of natural numbers, and $[n] \coloneqq \setbr{1, 2, 3, \ldots, n}$ to denote the set of all natural numbers at most $n$.
We use $\mtsetbr{\cdot}$ explicitly for \textit{multisets} allowing duplicated elements and $\setbr{\cdot}$ for \textit{sets} consisting of unique elements.

\smallsection{Hypergraphs.}
We use hypergraphs to represent group interactions.
Each hyperedge corresponds to a group where the nodes represent the members.
Formally, a \textit{hypergraph} $H = (V, E)$ is defined by its \textit{topology} consisting of a \textit{node} set $V$ and a \textit{hyperedge} multiset $E$, where each hyperedge $e \in E$ is a subset of $V$, i.e., $e \subseteq V$.
We use $E^* = \setbr{e: e \in E}$ to denote the set of unique hyperedges after removing repetitions.
The \textit{degree} $d_v$ of a node $v$ is the number of hyperedges containing $v$, i.e., $d_v = |\mtsetbr{e \in E: v \in e}|$.
We assume no given node or edge attributes (i.e., features), which is the case for the real-world datasets used in our experiments, and also many other datasets~\citep{lee2024villain}.
Yet, attributes can be easily incorporated into our method when given. See Section~\ref{sec:conclusion} for more discussions.
See Appendix~\ref{appx:discussions} for discussions on general hypergraphs, e.g., directed and heterogeneous hypergraphs.



