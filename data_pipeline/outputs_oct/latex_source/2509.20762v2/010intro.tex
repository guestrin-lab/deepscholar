\section{Introduction}\label{sec:intro}

Group interactions are ubiquitous in the real world.
For example, scholars collaborate and write a paper together~\citep{lung2018hypergraph}, people participate in the same email communication~\citep{larock2023encapsulation}, users participate in the same online Q\&A session~\citep{antelmi2023age}, and actors act in the same movie~\citep{yadati2024heal}.
In many scenarios, in each group, a particular member plays a specifically important role, around whom the group is formed~\citep{hinds2000choosing}.
For example, in an academic collaboration, the first or last author of a paper is often the one who connects all the authors~\citep{osborne2019authorship}.
In an email correspondence, the sender of an email is the one who initializes the conversation to include the recipients~\citep{pallen1995guide}.
In an online Q\&A session, the questioner starts the session for others to join~\citep{stackexchangeTourStack}.
In a movie cast, the ensemble is often gathered around the main actor~\citep{colliderMoviesStarring}.

In this work, we discuss the existence of such important individuals and call them group \textit{anchors}, in the sense that each brings together the members of a group.
As widely observed across different domains (see \cref{tab:single_node_label}), there is often a single anchor in each group (see Section~\ref{sec:experiments:acc} for discussions and results with multiple anchors in each group).
Notably, the concept of group anchors is defined by the real-world meaning of groups and the roles of members, which is not necessarily aligned with (sometimes even opposite to) structural importance.
For example, the anchor (i.e., questioner) of an online Q\&A session is often a new user with low structural importance.
Finding group anchors can help us better understand and analyze real-world group interactions, and has the following important further real-world applications:
\begin{itemize}[leftmargin=*]
    \item \textbf{Group-interaction prediction.} 
    Group anchors often initiate or determine the formation of group interactions. By identifying anchors, we can better predict how groups form and evolve over time~\citep{benson2018simplicial} for, e.g., academic collaboration recommendation~\citep{liu2018context} and social group recommendation~\citep{qin2018dynamic}.
    \item \textbf{Engagement management.} Group anchors play significant roles in group interactions and need particular attention.
    In social networks, identifying anchors provides insights into the cohesion, longevity, and activity levels of social groups, helping administrators better maintain group engagement~\citep{malliaros2013stay}.
    \item \textbf{Targeted marketing.} Group anchors, being central or highly influential in their groups, can be critical targets for marketing. By focusing on group anchors, marketers can effectively influence whole groups through just a few individuals~\citep{chandra2022personalization}.
\end{itemize}
To the best of our knowledge, we are the first to discuss the existence of group anchors and study the identification problem.

\begingroup
\setlength{\tabcolsep}{4pt}
\begin{table}[t!]
    \centering
    \caption{Examples of group anchors in real-world group interactions.}
    \label{tab:single_node_label}
\begin{adjustbox}{max width=\linewidth}
\begin{tabular}{c| c c}
\hline
\textbf{Domain} & \textbf{Members in Each Group} & \textbf{Anchor in Each Group} \bigstrut\\
\hline
Co-authorship & Authors of a paper & First/last author \bigstrut[t]\\
Online Q\&A & Users in a session & Questioner \\
Email & People in a correspondence & Sender \\
Message & People involved a message & Sender \\
Retweet & People involved a retweet & Retweeted \\
Movie & Actors in a movie & Leading actor \bigstrut[b]\\
\hline
\end{tabular}%
\end{adjustbox}
\vspace{-1mm}
\end{table}
\endgroup


First, we introduce the concept of group anchors and the identification problem.
For mathematical rigor, we formulate the problem as an optimization problem on hypergraphs.
Hypergraphs allow an arbitrary number of nodes in each hyperedge and precisely model group interactions~\citep{benson2018sequences}.
Each hyperedge represents an interacting group, with its constituent nodes as the group members.
The anchor of each group is a node in the corresponding hyperedge.

Then, we discuss our observations on real-world group anchors.
We focus on the scenarios under label scarcity with limited training data (i.e., only a few groups have known anchors), which is common in the real world ~\citep{zhu2022introduction,zhu2005semi,yang2022semi}.
What can we learn with limited known group anchors?
\textbf{\textit{Our first observation}} is that topological features provide considerable information on group anchors, but they alone are insufficient.
Specifically, a simple heuristic based on node degrees 
performs comparably to sophisticated deep-learning methods sometimes, but it does not fully capture the underlying mechanisms of group anchors, leaving a clear gap compared to the strongest baseline.
Also, we would like to re-emphasize that, in some scenarios, structurally insignificant nodes are more likely to be the group anchors (e.g., questioners in online Q\&A).
\textbf{\textit{Our second observation}} is that whether an individual is the anchor or not is overall stable, yet still contextual, across different groups.
It is ``stable'' in that if an individual is (not) the anchor in one group, they are likely (not) the anchor in other groups.
It is ``contextual'' in that it is still possible for an individual to be the anchor in some groups but not in others.
We further observe that it is essential to consider \textit{local competition} in each group on top of \textit{global stability} to better capture the mechanism of group anchors.
Specifically, group anchors are well explained by
\textit{(i)} each node has its ``anchor strength'' indicating the overall likelihood that the node is the anchor across different groups, and
\textit{(ii)} the nodes compete locally in each group, and the one with the highest anchor strength is likely the anchor.

Intuitively and explicitly based on our observations, we propose \ours, a novel method for identifying group anchors.
Our method \ours has two stages.
\textbf{\textit{In the first stage}}, motivated by our first observation, we train a model to utilize the information in topological features to match the known group anchors.
After the first-stage model is trained, its predictions are used as ``references'' for the second stage.
\textbf{\textit{In the second stage}}, motivated by our second observation, we associate each node with a single scalar (i.e., ``anchor strength''), and we train the scalars to match the known group anchors and the ``references'' from the first stage.
After the second-stage strengths are trained, the node with the highest strength in each hyperedge is predicted as the anchor.

Instead of using sophisticated architectures such as deep neural networks, we aim to make \ours efficient and lightweight by using only simple architectures with few parameters.
This two-stage training balances fidelity to the given topology in the first stage and flexibility in the second stage.
Notably, \ours is semi-supervised, using information from groups both with and without known group anchors, e.g., the topological features.
In summary, the proposed method \ours is
\textbf{\textit{(i) intuitive}}, with algorithmic designs directly based on observations on real-world datasets, and
\textbf{\textit{(ii) efficient}}, using few parameters and demanding less computational resources, e.g., training time.
The empirical superiority of \ours is validated via extensive experiments on thirteen real-world hypergraph datasets.

To conclude, our contributions are four-fold:
\begin{itemize}[leftmargin=*]
    \item \textbf{New concept and new problem (\cref{sec:problem_state}).} To the best of our knowledge, we are the first to discuss the existence of group anchors, and study the problem of identifying them in real-world group interactions under realistic settings with label scarcity.
    This problem is of theoretical interest with various further practical applications.
    \item \textbf{Observations (\cref{sec:observations}).} Based on the realistic scenarios with limited known group anchors, we discuss several observations on group anchors in real-world group interactions.
    \item \textbf{Method (\cref{sec:method}).} We propose \ours, a novel method for group anchor identification, with intuitive and lightweight mechanisms motivated by our observations.
    \item \textbf{Experiments (\cref{sec:experiments}).} Via experiments on thirteen real-world hypergraphs, we validate the empirical superiority of our method \ours w.r.t. accuracy and efficiency.
\end{itemize}

\smallsection{Reproducibility.} 
The appendix, code, and datasets are available online in~\citep{onlineSuppl} (\url{https://github.com/bokveizen/anchor_radar}).
