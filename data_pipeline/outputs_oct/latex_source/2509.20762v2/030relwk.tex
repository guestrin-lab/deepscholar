\begingroup
\setlength{\tabcolsep}{5pt}
\begin{table*}[t!]
    \centering
    \caption{Basic statistics of the real-world group-interaction (hypergraph) datasets used in our experiments.}
    \label{tab:dataset_summary}
\begin{adjustbox}{max width=\textwidth}
\begin{tabular}{ccc|rrr|rrr}
\hline
\textbf{Domain} & \textbf{Dataset} & \textbf{Abbrev.} & $\mathbf{|V|}$ & $\mathbf{|E|}$ & $\mathbf{|E^*|}$ & \textbf{Min.} $\mathbf{|e|}$ & \textbf{Max.} $\mathbf{|e|}$ & \textbf{Avg.} $\mathbf{|e|}$ \bigstrut\\
\hline
\multirow{3}[2]{*}{\makecell{Co-authorship \\ ($D_{\text{co}}$)}} & AMinerAuthor~\citep{tang2008arnetminer,choe2023classification}  & \texttt{coAA} & 1,712,433 & 2,037,605 & 1,454,250 & 1     & 115   & 2.55  \bigstrut[t]\\
\multicolumn{1}{c}{} & DBLP~\citep{team2019dblp,choe2023classification} & \texttt{coDB} &  108,476 & 91,260 & 81,601 & 2     & 36    & 3.52 \\
\multicolumn{1}{c}{} & ScopusMultilayer~\citep{chodrow2020annotated,kivela2014multilayer,bianconi2018multilayer,boccaletti2014structure} & \texttt{coSM} & 1,673 & 937   & 842   & 1     & 27    & 3.09 \bigstrut[b]\\
\hline
\multirow{4}[2]{*}{\makecell{Online Q\&A \\ ($D_{\text{qa}}$)}} & StackOverflowBiology~\citep{choe2023classification} & \texttt{qaBI} &  15,418 & 26,290 & 23,242 & 1     & 12    & 2.08  \bigstrut[t]\\
\multicolumn{1}{c}{} & StackOverflowPhysics~\citep{choe2023classification} & \texttt{qaPH} & 80,434 & 194,575 & 169,274 & 1     & 40    & 2.38 \\
\multicolumn{1}{c}{} & MathOverflow~\citep{chodrow2020annotated} & \texttt{qaMA} & 410   & 154   & 154   & 2     & 57    & 4.27  \\
\multicolumn{1}{c}{} & StackOverflow~\citep{chodrow2020annotated} & \texttt{qaST} &  22,131 & 4,716 & 4,713 & 1     & 59    & 5.79 \bigstrut[b]\\
\hline
\multirow{3}[2]{*}{\makecell{Email \\ ($D_{\text{em}}$)}} & EmailEnron~\citep{choe2023classification} & \texttt{emEN} & 21,251 & 101,124 & 34,916 & 2     & 883   & 11.53 \bigstrut[t]\\
\multicolumn{1}{c}{} & EmailEu~\citep{choe2023classification,paranjape2017motifs} & \texttt{emEU} & 986   & 209,508 & 24,520 & 2     & 40    & 2.56 \\
\multicolumn{1}{c}{} & Enron~\citep{chodrow2020annotated} & \texttt{emER} & 110   & 9,603 & 1,169 & 2     & 29    & 2.47  \bigstrut[b]\\
\hline
\multirow{2}[2]{*}{Social network ($D_{\text{so}}$)} & Message~\citep{chodrow2020annotated} & \texttt{soME} & 26,059 & 34,577 & 22,700 & 2     & 14    & 2.58 \bigstrut[t]\\
\multicolumn{1}{c}{} & Retweet~\citep{chodrow2020annotated} & \texttt{soRE} & 30,073 & 88,148 & 49,828 & 2     & 2     & 2.00   \bigstrut[b]\\
\hline
Movie cast ($D_{\text{mo}}$)  & MovieLens~\citep{chodrow2020annotated,harper2015movielens} & \texttt{moML} & 73,155 & 43,058 & 42,497 & 1     & 5     & 4.70  \bigstrut\\
\hline
\multicolumn{9}{l}{\small *Data source: \url{https://github.com/young917/EdgeDependentNodeLabel}~\citep{choe2023classification} and \url{https://andrewmellor.co.uk/data/}~\citep{chodrow2020annotated}.} \\
\end{tabular}%
\end{adjustbox}        
\end{table*}
\begingroup



\section{Related Work}\label{sec:rel_wk}

\smallsection{Important nodes in hypergraphs.}
Typically, \textit{global} (i.e., for the whole hypergraph) and \textit{structural} (i.e., based on hypergraph topology) centrality measures are used to identify important nodes, where nodes with high centrality scores are considered important.
Some examples of node centrality measures include variants of core numbers~\citep{arafat2023neighborhood,ramadan2004hypergraph,limnios2021hcore,lee2023k,kim2023exploring,bu2023hypercore} and various spectral centrality measures~\citep{tudisco2021node,kovalenko2022vector,benson2019three}.
Specifically, the core-periphery structure in hypergraphs has been considered~\citep{tudisco2023core,papachristou2022core}, where the set of ``core nodes'' is typically a small set of nodes intersecting with most (or even all~\citep{amburg2021planted}) hyperedges.
In summary, most existing works identify important nodes based on their \textit{global} \textit{structural} properties, while by the idea of group anchors, we consider the \textit{conceptual} (i.e., based on the real-world meaning of the group interactions) importance of nodes \textit{locally} in each hyperedge (group).

\smallsection{Node classification in hypergraphs.}
Node classification is a common task on 
hypergraphs~\citep{feng2019hypergraph}, where one aims to assign labels to nodes.
General node classification considers node labels for a node the same across different hyperedges, i.e., the node labels are edge-independent. Therefore, methods for general node classification cannot be directly applied to identifying group anchors, which are edge-dependent.\footnote{See Appendix~\ref{appx:general_node_cls} for more discussions on general node classification.}
Indeed, as noted in \citet{chodrow2020annotated}, the same entity (i.e., node) may have different roles (i.e., labels) in different groups (i.e., hyperedges), which motivated prior works on the task of edge-dependent node classification (ENC) in hypergraphs~\citep{choe2023classification, zheng2024co}.
{WHATsNet~\citep{choe2023classification} tackles ENC by generating node and hyperedge embeddings using SetTransformer~\citep{lee2019set} with positional encoding, and aggregating them to obtain the embeddings of node-hyperedge pairs.
CoNHD~\citep{zheng2024co} uses co-representation hypergraph diffusion to directly learn node-hyperedge pair embeddings by considering interactions both between hyperedges and between nodes.}
Both WHATsNet and CoNHD are built on deep hypergraph neural network architectures.
Mathematically, identifying group anchors can be seen as a special case of ENC, where the classification is binary, and there is one positive sample (i.e., the anchor) in each hyperedge.
Indeed, we apply methods for ENC to group anchor identification with proper modifications as baseline methods.
However, the problem of group anchor identification has unique values and applications (see \cref{sec:intro}).
Also, as shown in our experiments (see \cref{sec:experiments}), even though using much more sophisticated mechanisms, existing methods for ENC are outperformed by our method w.r.t. both accuracy and efficiency.
Moreover, we consider realistic settings with label scarcity, while existing works on edge-dependent node classification~\citep{choe2023classification,zheng2024co} consider settings where the node labels in 60\% of the hyperedges (groups) are known.







