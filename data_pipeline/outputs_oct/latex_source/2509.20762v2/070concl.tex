\section{Conclusion and     Discussions}\label{sec:conclusion}

In this work, we discuss the existence of group anchors in real-world group interactions and study the problem of identifying them (\cref{sec:problem_state}).
We discuss several observations on anchors in real-world group interactions (\cref{sec:observations}).
We propose a novel method \ours for group anchor identification (\cref{sec:method}), with intuitive algorithmic designs directly motivated by our observations.
Via extensive experiments on thirteen real-world hypergraphs (\cref{sec:experiments}), we validate the empirical superiority of the proposed method \ours.

\smallsection{Limitations and discussions.}
We use the same topological features used in existing works~\citep{choe2023classification,zheng2024co} for a fair comparison (see \cref{sec:method:stage1}), while finding more informative topological features is a potential future direction.

We assume no input node/hyperedge features, which is also true for the original datasets we use. However, when additional input node/hyperedge features are given, they can be directly incorporated in Stage 1 as additional features in $X$ and further improve the performance of \ours (see Sections~\ref{sec:method:stage1} and \ref{sec:experiments:acc}).

We mainly consider the scenarios where each group contains a single anchor since is it common in the real world (see Sections~\ref{sec:intro} and \ref{sec:problem_state}).
See Section~\ref{sec:experiments:acc} for discussions and results considering multiple anchors, where \ours still outperforms the baseline methods.
See Appendix~\ref{appx:discussions} for discussions on more general hypergraphs, e.g., directed hypergraphs and heterophilic hypergraphs.
