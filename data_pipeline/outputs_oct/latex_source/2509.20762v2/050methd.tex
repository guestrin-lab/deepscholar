\section{The Proposed Method: \ours}\label{sec:method}

We shall introduce the proposed method \ours for group anchor identification, whose mechanisms are intuitively and explicitly based on our observations in \cref{sec:observations}.

\revisekddcolor
\smallsection{Overview.}
The proposed method \ours has two stages.
In Stage 1, based on \cref{intu:topological_feat} (topological features are informative about group anchors), we train a multilayer perceptron (MLP) to learn a \textit{topology-based score} for each node-hyperedge pair $(v, e)$.
In Stage 2, based on \cref{intu:global_consistency} (anchorship is overall stable across different hyperedges), we train a scalar \textit{anchor strength} for each node shared across all hyperedges.
Based on our further analysis of \cref{intu:global_consistency} (\cref{tab:scalar_acc}), we use \textit{global strength aggregation} to obtain an aggregated score for each node from different hyperedges.
Finally, in hyperedge, we predict the node with the highest aggregated score as the anchor.
Notably, \ours is semi-supervised, using information from hyperedges both with and without known anchors, e.g., the topological features.

\color{black}

\subsection{Stage 1: \revisekdd{MLP Training and Topology-based Scores of Node-Hyperedge Pairs}}\label{sec:method:stage1}

\smallsection{Overview.}
Based on \cref{intu:topological_feat} (topological features are informative about group anchors), we aim to train a model to exploit the correlations between topological features and anchorship.
The model is intended to be lightweight, using a simple architecture and only topological features as inputs
to fit known anchors.
\revisekdd{Specifically, we train a multilayer perceptron (MLP) to learn a \textit{topology-based score} for each node-hyperedge pair $(v, e)$, so that in each hyperedge $e$ with the ground-truth anchor $v^*$, the score of $(v^*, e)$ should be higher than that of $(v', e)$ for other nodes $v' \in e$.}
After training, the topology-based scores will be used as a reference to guide the training in Stage 2.
See Algorithm~\ref{algo:stage1} for an algorithmic overview.

\smallsection{Topological features.}
In principle, any topological features can be used.
Specifically, in our experiments, for a fair comparison, we use the same four topological features used in our baselines methods~\citep{choe2023classification,zheng2024co} (see \cref{sec:exp_settings}):
\textbf{(1) node degree} (see \cref{sec:prelim}),
\textbf{(2\&3) eigenvector centrality}~\citep{bonacich1987power} and \textbf{PageRank centrality}~\citep{page1999pagerank} computed on the weighted clique expansion, and \textbf{(4) coreness}~\citep{sun2020fully,seidman1983network}.
Specifically, given input topology $V$ and $E$, to each node-hyperedge pair $(v, e)$ with $v \in e$, we associate a 33-dimensional feature vector:
\begin{itemize}[leftmargin=*]
    \item \textbf{Global features (24 dimensions).} For each of the four node centrality measures, we normalize and/or aggregate it in six different ways: \textbf{(1)} min-max normalization, \textbf{(2)} rank normalization, and \textbf{(3-6)} four different aggregations: means and standard deviations of local min-max and rank normalization.
    For each node, its global features are shared across all the hyperedges that contain the node.
    \item \textbf{Local features (9 dimensions).}
    For each of the four node centrality measures, we normalize it locally in each hyperedge $e$ in two different ways: \textbf{(1)} local min-max normalization and \textbf{(2)} local rank normalization.
    We also include the size of $e$, which is a scalar, as a local feature for all the nodes in $e$.
    For each node, its local features can be distinct in different hyperedges, which allows flexibility in modeling the edge-dependent nature of group anchors.
\end{itemize}
We construct a topological feature matrix $X$ by collecting the 33-dimensional feature vectors $X_{v;e}$ for all node pairs.
See Appendix~\ref{appx:feature_process} for more details on the features. 

\smallsection{\revisekdd{MLP training and topology-based scores.}}
We use an MLP parameterized with $\theta$ to obtain the \textit{topology-based scores} $s^{(1)}_{v;e}$ of each node-hyperedge pair $(v, e)$: 
    $s^{(1)}_{v;e} = \operatorname{MLP}(X; \theta)$,
with loss function
\begin{equation}\label{eq:loss_stage_1}
    \mathcal{L}^{(1)} = -\sum\nolimits_{e \in E'} \log 
    \left(\exp\left(s^{(1)}_{A(e);e}\right) \middle/ \sum\nolimits_{u \in e} \exp\left(s^{(1)}_{u;e}\right)\right),
\end{equation}
where $E'$ is the set of hyperedges with known anchors and $A(e)$ is the anchor in each $e \in E'$ (see \cref{prob:seed_identification}).
Minimizing the loss function encourages a large (relative) score $s^{(1)}_{A(e);e}$ of the anchor compared to $s^{(1)}_{u;e}$ for other nodes $u \in e$.

\begin{algorithm}[t]
    \caption{\ours: Stage 2}\label{algo:stage2}
    \SetKwInput{KwInput}{Input}
    \SetKwInput{KwOutput}{Output}
    \KwInput{\textbf{(1)} $V$ and $E$: topology;
    \textbf{(2)} $E'$ and $A(e), \forall e \in E'$: known group anchors;
    \textbf{(3)} $s^{(1)}_{v;e}, \forall v \in e \in E$: learned strengths from Stage 1;
    \textbf{(4)} $\alpha^{(2)}$: loss term coefficient;
    \textbf{(5)} $w^{(2)}$: global aggregation weight;
    \textbf{(6)} $n^{(2)}_{ep}$: number of optimization epochs}
    \KwOutput{$\tilde{A}(e), \forall e \in E \setminus E'$: predicted group anchors}
    $s^{(2)}_{v} \gets 1, \forall v \in V$ \Comment*[f]{Initialization} \\
    \For{$i_{ep} = 1, 2, \ldots, n^{(2)}_{ep}$}{
        $\mathcal{L}^{(2)}_1 = -\sum_{e \in E'} \log \frac{\exp(s^{(2)}_{A(e)})}{\sum_{u \in e} \exp(s^{(2)}_u)}$ \Comment*[f]{Eq.~\eqref{eq:loss_stage_2_label}} \\
        $\mathcal{L}^{(2)}_2 = -\sum_{e \in E'}
    \frac{\exp(s^{(2)}_{A(e)})}{\sum_{u \in e} \exp(s^{(2)}_u)} \cdot \frac{\exp(s^{(1)}_{A(e);e})}{\sum_{u \in e} \exp(s^{(1)}_{u;e})}$ \Comment*[f]{Eq.~\eqref{eq:loss_stage_2_align}} \\
        Update each $s^{(2)}_{v}$ w.r.t. $\frac{\partial \left(\mathcal{L}^{(2)}_1 + \alpha^{(2)} \mathcal{L}^{(2)}_2\right)}{\partial s^{(2)}_{v}}$ \Comment*[f]{Gradient descent} \\
    }
    $\hat{A}(e) = \arg \max_{v^* \in e} s^{(2)}_{v^*}, \forall e \in E$ \Comment*[f]{Max anchor strength} \\         
    $\hat{p}_v = \frac{w^{(2)} 
    \sum_{e \in E'} \mathbf{1}[A(e) = v]
    + \sum_{e \in E \setminus E'} \mathbf{1}[\hat{A}(e) = v]}{d_v}, \forall v \in V$ \Comment*[f]{Eq.~\eqref{eq:pred_member_prop}} \\    
 \Return $\tilde{A}(e) = \arg\max_{v^* \in e} \hat{p}_{v^*}, \forall e \in E \setminus E'$ \Comment*[f]{Final prediction}
\end{algorithm}


\subsection{\revisekdd{Stage 2: Anchor Strength Learning, Global Aggregation, and Final Prediction}}\label{sec:method:stage2}

\smallsection{Overview.}
Based on \cref{intu:global_consistency}, we aim to find anchor strengths (scalars indicating the overall likelihood that each node is the anchor), together with local competition and global aggregation, to explain the known anchors.
\revisekdd{Specifically, for each node, we train a scalar \textit{anchor strength} shared across all hyperedges, 
so that 
(1) in each hyperedge $e$ with the ground-truth anchor $v^*$, the anchor strength of $v^*$ should be higher than that of other nodes $v' \in e$, and
(2) based on \cref{intu:topological_feat} again, the relative relations of the strengths in each hyperedge should align well with the topology-based scores in Stage 1.
Then, based on our further analysis of \cref{intu:global_consistency} (\cref{tab:scalar_acc}), we use \textit{global strength aggregation}, i.e., compute the anchor proportion (see \cref{def:seed_member_prop_purity}) assuming the predictions based on the learned anchor strengths are all correct, to further enhance the prediction robustness and accuracy.
Finally, in hyperedge, we predict the node with the highest aggregated score as the anchor.}
See Algorithm~\ref{algo:stage2} for an algorithmic overview.

\smallsection{\revisekdd{Anchor strength learning.}}
Let $s^{(2)}_v$ be the learnable anchor strength of each node $v$.
The loss function consists of two parts.
The first part, similar to that in Stage 1, aims to fit the known anchors:
\begin{equation}\label{eq:loss_stage_2_label}
    \mathcal{L}^{(2)}_1 = -\sum\nolimits_{e \in E'} \log \left({\exp\left(s^{(2)}_{A(e)}\right)} \middle/ {\sum\nolimits_{u \in e} \exp\left(s^{(2)}_u\right)}\right).
\end{equation}
The second part aligns with the learned scores from Stage 1:
\begin{equation}\label{eq:loss_stage_2_align}
    \mathcal{L}^{(2)}_2 = -\sum\nolimits_{e \in E'}
    \frac{\exp\left(s^{(2)}_{A(e)}\right)}{\sum\nolimits_{u \in e} \exp\left(s^{(2)}_u\right)} \cdot \frac{\exp\left(s^{(1)}_{A(e);e}\right)}{\sum\nolimits_{u \in e} \exp\left(s^{(1)}_{u;e}\right)},
\end{equation}
which measures the distance between $s^{(1)}_{v;e}$'s and $s^{(2)}_v$'s after within-edge normalization.
The final loss function is $\mathcal{L}^{(2)} = \mathcal{L}^{(2)}_1 + \alpha^{(2)} \mathcal{L}^{(2)}_2$ with a hyperparameter coefficient $\alpha^{(2)}$.

\smallsection{Global strength aggregation.}
Recall the observation of the high accuracy of anchor proportions (see \cref{tab:scalar_acc} in \cref{sec:intuition:global_consistency}).
The anchor proportion of a node $v$ can be seen as the global aggregation of its anchor indicators $\mathbf{1}[A(e) = v]$.\footnote{$\mathbf{1}[A(e) = v] = 1$ if $A(e) = v$, and $\mathbf{1}[A(e) = v] = 0$ otherwise.}
Inspired by this, after learning anchor strengths $s^{(2)}_v$'s, we first find the node $\hat{A}(e)$ with the highest learned strengths in each hyperedge $e$, i.e., $\hat{A}(e) = \arg \max_{v^* \in e} s^{(2)}_{v^*}, \forall e \in E$ (we randomly pick one if multiple nodes have the highest learned anchor strength), and compute the predicted anchor proportion $\hat{p}_v$ of each node $v$:
\begin{equation}\label{eq:pred_member_prop}
\small
    \hat{p}_v = \frac{1}{d_v} \left({w^{(2)} 
    \sum\nolimits_{e \in E'} \mathbf{1}[A(e) = v]
    + \sum\nolimits_{e \in E \setminus E'} \mathbf{1}[\hat{A}(e) = v]}\right),
\end{equation}
where we use a hyperparameter $w^{(2)}$ as the global aggregation weight on training (i.e., ground-truth) labels.
Such global aggregation considers the predictions across different hyperedges and increases the prediction robustness~\citep{kumar2018rev2}.

\smallsection{\revisekdd{Final prediction.}}
Finally, in each hyperedge, we predict the node with the highest aggregated score $\hat{p}_v$ as the anchor.

\begingroup
\setlength{\tabcolsep}{2.5pt}
\begin{table*}[t!]
    \centering
    \caption{\uline{Q1: \ours achieves higher accuracy in identifying group anchors than all the baselines in most cases.} For each setting, the best performance is highlighted in bold, while the second-best is underlined.
    The mean accuracy values (\%) over five random splits are reported with standard deviations. OOM represents ``out of memory''.
    \revisekdd{\ours also outperforms the baseline methods with different training ratios (see Appendix~\ref{appx:diff_training_ratios}) and w.r.t. other evaluation metrics (see Appendix~\ref{appx:diff_eval_metrics}).}}
    \label{tab:single_node_res}
\begin{adjustbox}{max width=\textwidth}
\begin{tabular}{c|cccccccccccccccc}
\hline
\multirow{2}[2]{*}{\textbf{Dataset}} 
& \multicolumn{2}{c}{\texttt{coAA}} 
& \multicolumn{2}{c}{\texttt{coDB}} 
& \multicolumn{2}{c}{\texttt{coSM}} 
& \multirow{2}[2]{*}{\texttt{qaBI}} 
& \multirow{2}[2]{*}{\texttt{qaPH}} 
& \multirow{2}[2]{*}{\texttt{qaMA}} 
& \multirow{2}[2]{*}{\texttt{qaST}} 
& \multirow{2}[2]{*}{\texttt{emEN}} 
& \multirow{2}[2]{*}{\texttt{emEU}} 
& \multirow{2}[2]{*}{\texttt{emER}} 
& \multirow{2}[2]{*}{\texttt{soME}} 
& \multirow{2}[2]{*}{\texttt{soRE}} 
& \multirow{2}[2]{*}{\texttt{moML}} \bigstrut[t]\\
      
& (first) 
      & (last) 
      & (first) 
      & (last) 
      & (first) 
      & (last) 
      &       
      &       
      &       
      &       
      &       
      &       
      &       
      &       
      &       
      &  \bigstrut[b]\\
    \hline
    WHATsNet & \uline{45.2{\scriptsize$\pm$0.2}} & \uline{45.8{\scriptsize$\pm$0.3}} & \uline{42.5{\scriptsize$\pm$0.3}} & 45.4{\scriptsize$\pm$0.2} & 34.3{\scriptsize$\pm$2.0} & \uline{39.8{\scriptsize$\pm$1.8}} & \uline{85.6{\scriptsize$\pm$0.4}} & \uline{88.1{\scriptsize$\pm$0.1}} & \uline{35.8{\scriptsize$\pm$1.1}} & \uline{31.2{\scriptsize$\pm$0.2}} & \uline{50.8{\scriptsize$\pm$3.4}} & 51.0{\scriptsize$\pm$0.3} & \uline{66.6{\scriptsize$\pm$3.1}} & \textbf{75.5{\scriptsize$\pm$0.2}} & 97.4{\scriptsize$\pm$0.3} & 41.4{\scriptsize$\pm$0.3} \bigstrut[t]\\
    CoNHD-U & 42.7{\scriptsize$\pm$1.3} & 42.2{\scriptsize$\pm$2.0} & 41.2{\scriptsize$\pm$0.1} & 42.7{\scriptsize$\pm$0.3} & 31.4{\scriptsize$\pm$1.9} & 37.9{\scriptsize$\pm$2.1} & 78.7{\scriptsize$\pm$0.5} & 76.0{\scriptsize$\pm$1.0} & 29.0{\scriptsize$\pm$4.2} & 25.4{\scriptsize$\pm$3.8} & 44.0{\scriptsize$\pm$4.3} & \textbf{52.8{\scriptsize$\pm$0.3}} & 65.3{\scriptsize$\pm$2.2} & 74.3{\scriptsize$\pm$0.3} & 96.8{\scriptsize$\pm$0.6} & 42.4{\scriptsize$\pm$0.5} \\
    CoNHD-I & 44.5{\scriptsize$\pm$0.8} & 44.7{\scriptsize$\pm$0.3} & 41.4{\scriptsize$\pm$0.5} & 43.5{\scriptsize$\pm$0.7} & 29.8{\scriptsize$\pm$2.6} & 39.4{\scriptsize$\pm$1.9} & 79.3{\scriptsize$\pm$0.4} & 77.3{\scriptsize$\pm$0.2} & 29.8{\scriptsize$\pm$6.1} & 26.6{\scriptsize$\pm$1.3} & 45.1{\scriptsize$\pm$3.8} & \uline{52.4{\scriptsize$\pm$0.4}} & 64.6{\scriptsize$\pm$1.2} & 74.6{\scriptsize$\pm$0.4} & \uline{97.5{\scriptsize$\pm$0.6}} & \uline{42.7{\scriptsize$\pm$0.3}} \\
    HNHN  & 39.7{\scriptsize$\pm$0.0} & 41.2{\scriptsize$\pm$0.0} & 35.5{\scriptsize$\pm$0.4} & 39.1{\scriptsize$\pm$0.4} & 33.2{\scriptsize$\pm$1.4} & 33.7{\scriptsize$\pm$0.6} & 63.5{\scriptsize$\pm$1.4} & 37.7{\scriptsize$\pm$0.1} & 30.5{\scriptsize$\pm$0.8} & 22.9{\scriptsize$\pm$0.1} & 35.8{\scriptsize$\pm$2.1} & 49.2{\scriptsize$\pm$1.2} & 41.8{\scriptsize$\pm$6.4} & 56.4{\scriptsize$\pm$0.4} & 53.4{\scriptsize$\pm$0.8} & 35.2{\scriptsize$\pm$0.3} \\
    HGNN  & 44.1{\scriptsize$\pm$0.0} & 45.9{\scriptsize$\pm$0.1} & 41.9{\scriptsize$\pm$0.1} & 44.6{\scriptsize$\pm$0.3} & 33.1{\scriptsize$\pm$0.3} & 38.1{\scriptsize$\pm$0.6} & 81.7{\scriptsize$\pm$0.3} & 74.9{\scriptsize$\pm$0.8} & 28.9{\scriptsize$\pm$1.0} & 30.4{\scriptsize$\pm$0.5} & 40.1{\scriptsize$\pm$0.7} & 49.3{\scriptsize$\pm$0.2} & 42.0{\scriptsize$\pm$0.9} & 62.1{\scriptsize$\pm$2.1} & 84.6{\scriptsize$\pm$3.4} & 37.9{\scriptsize$\pm$0.5} \\
    HCHA  & 38.9{\scriptsize$\pm$0.2} & 39.4{\scriptsize$\pm$0.4} & 35.3{\scriptsize$\pm$0.6} & 31.4{\scriptsize$\pm$0.7} & 33.2{\scriptsize$\pm$1.1} & 35.2{\scriptsize$\pm$3.6} & 69.8{\scriptsize$\pm$2.1} & 68.0{\scriptsize$\pm$1.5} & 31.0{\scriptsize$\pm$2.4} & 23.4{\scriptsize$\pm$2.6} & 18.8{\scriptsize$\pm$2.2} & 45.4{\scriptsize$\pm$0.5} & 46.0{\scriptsize$\pm$4.9} & 30.7{\scriptsize$\pm$2.2} & 52.7{\scriptsize$\pm$0.6} & 17.3{\scriptsize$\pm$0.4} \\
    HAT   & 43.5{\scriptsize$\pm$0.3} & 45.8{\scriptsize$\pm$0.1} & 38.1{\scriptsize$\pm$1.4} & 40.5{\scriptsize$\pm$2.0} & 30.1{\scriptsize$\pm$0.5} & 33.0{\scriptsize$\pm$1.0} & 75.8{\scriptsize$\pm$0.3} & 81.3{\scriptsize$\pm$0.2} & 29.2{\scriptsize$\pm$1.2} & 23.9{\scriptsize$\pm$0.2} & 49.8{\scriptsize$\pm$1.7} & 50.8{\scriptsize$\pm$0.4} & 42.3{\scriptsize$\pm$7.0} & 68.0{\scriptsize$\pm$0.9} & 92.4{\scriptsize$\pm$0.9} & 36.1{\scriptsize$\pm$0.8} \\
    UniGCN & 43.3{\scriptsize$\pm$0.5} & 45.8{\scriptsize$\pm$0.4} & 41.2{\scriptsize$\pm$0.7} & \uline{45.8{\scriptsize$\pm$0.6}} & \uline{34.8{\scriptsize$\pm$2.7}} & 39.2{\scriptsize$\pm$4.2} & 76.3{\scriptsize$\pm$1.2} & 78.0{\scriptsize$\pm$1.4} & 35.0{\scriptsize$\pm$3.4} & 30.7{\scriptsize$\pm$0.4} & 45.3{\scriptsize$\pm$2.7} & 49.6{\scriptsize$\pm$0.7} & 55.5{\scriptsize$\pm$2.7} & 68.9{\scriptsize$\pm$1.3} & 88.1{\scriptsize$\pm$0.6} & 40.1{\scriptsize$\pm$1.8} \\
    HNN   & 37.7{\scriptsize$\pm$0.1} & 39.3{\scriptsize$\pm$0.1} & 31.4{\scriptsize$\pm$0.8} & 36.3{\scriptsize$\pm$1.2} & 32.7{\scriptsize$\pm$0.7} & 36.8{\scriptsize$\pm$0.8} & 63.8{\scriptsize$\pm$0.9} & 62.6{\scriptsize$\pm$0.6} & 29.2{\scriptsize$\pm$3.2} & 24.8{\scriptsize$\pm$0.5} & 38.7{\scriptsize$\pm$2.7} & OOM   & 47.1{\scriptsize$\pm$4.3} & 56.0{\scriptsize$\pm$0.6} & 57.2{\scriptsize$\pm$7.6} & 33.8{\scriptsize$\pm$0.6} \bigstrut[b]\\
    \hline
    \ours & \textbf{49.7{\scriptsize$\pm$0.1}} & \textbf{50.6{\scriptsize$\pm$0.0}} & \textbf{46.5{\scriptsize$\pm$0.2}} & \textbf{49.9{\scriptsize$\pm$0.1}} & \textbf{40.9{\scriptsize$\pm$0.7}} & \textbf{48.1{\scriptsize$\pm$1.3}} & \textbf{87.4{\scriptsize$\pm$0.2}} & \textbf{88.7{\scriptsize$\pm$0.1}} & \textbf{40.6{\scriptsize$\pm$5.3}} & \textbf{36.6{\scriptsize$\pm$0.4}} & \textbf{53.6{\scriptsize$\pm$2.4}} & 50.9{\scriptsize$\pm$0.2} & \textbf{67.8{\scriptsize$\pm$2.6}} & \uline{74.9{\scriptsize$\pm$0.6}} & \textbf{97.8{\scriptsize$\pm$0.3}} & \textbf{45.0{\scriptsize$\pm$0.3}} \bigstrut\\
    \hline
\end{tabular}%
\end{adjustbox}
\end{table*}
\endgroup


\subsection{Complexity Analysis}\label{sec:method:complexity}

Due to its simplicity, \ours has low complexity, especially time complexity and the number of learnable parameters.
For the whole process of \ours,
the training time complexity is
$O(\sum_{e \in E} (n^{(1)}_{ep} |e| n_{f} D_h + n^{(2)}_{ep}))$,
the inference time complexity is
$O(\sum_{e \in E} |e| n_{f} D_h)$,
the space complexity is
$O(n_{f} (D_h + \sum_{e \in E} |e|))$, and
the number of learnable parameters is
$O(D_h (n_{f} + 2) + 1 + |V|)$.

\smallsection{Notes.}
In our experiments, we use $D_h = 64$, $n_f = 33$ (see \cref{sec:method:stage1}), and $n^{(1)}_{ep} = n^{(2)}_{ep} = 100$.
For $|V|$ and $|e|$, see \cref{tab:dataset_summary}.
See Appendix~\ref{appx:complexity} for the detailed analysis.





