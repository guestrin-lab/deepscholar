\begin{abstract}
Group interactions occur in various real-world contexts, e.g., co-authorship, email communication, and online Q\&A.
\textit{In each group}, there is often a particularly significant member, around whom the group is formed.
Examples include the first or last author of a paper, the sender of an email, and the questioner in a Q\&A session.
In this work, we discuss the existence of such individuals in real-world group interactions.
We call such individuals group \textit{anchors} and study the problem of identifying them.
First, we introduce the concept of group anchors and the identification problem.
Then, we discuss our observations on group anchors in real-world group interactions.
Based on our observations, we develop \ours, a fast and effective method for group anchor identification under realistic settings with label scarcity, i.e., when only a few groups have known anchors.
\ours is a semi-supervised method using information from groups both with and without known group anchors.
Finally, through extensive experiments on thirteen real-world datasets, we demonstrate the empirical superiority of \ours over various baselines w.r.t. accuracy and efficiency.
In most cases, \ours achieves higher accuracy in group anchor identification than all the baselines, while using 10.2$\times$ less training time than the fastest baseline and 43.6$\times$ fewer learnable parameters than the most lightweight baseline on average.
\end{abstract}

\begin{IEEEkeywords}
Group interactions, Hypergraphs, Label scarcity, Important node identification
\end{IEEEkeywords}
